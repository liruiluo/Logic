% This LaTeX document needs to be compiled with XeLaTeX.
\documentclass[a4paper,11pt,twoside]{book}
\usepackage[utf8]{inputenc}
\usepackage{amsmath}
\usepackage{amsfonts}
\usepackage{amssymb}
\usepackage[version=4]{mhchem}
\usepackage{extpfeil}
\usepackage{stmaryrd}
\usepackage{graphicx}
\usepackage[export]{adjustbox}
\graphicspath{ {./images/} }
\usepackage{multirow}
\usepackage{fvextra, csquotes}
\usepackage{xeCJK}
\usepackage{polyglossia}
\usepackage{fontspec}
\usepackage{geometry}
\usepackage{fancyhdr}
\usepackage{titlesec}
\usepackage{tocloft}
\usepackage{indentfirst}
\usepackage{setspace}
\usepackage{xcolor}

% 添加hyperref包以启用PDF中的可跳转链接
\usepackage[hidelinks]{hyperref}

% 设置页面边距
\geometry{top=2.5cm, bottom=2.5cm, left=2.5cm, right=2.5cm}

% 设置中文字体
% \IfFontExistsTF{Noto Serif CJK SC}
% {\setCJKmainfont{Noto Serif CJK SC}}
% {\IfFontExistsTF{STSong}
%   {\setCJKmainfont{STSong}[BoldFont=STHeiti]}
%   {\IfFontExistsTF{Droid Sans Fallback}
%     {\setCJKmainfont{Droid Sans Fallback}}
%     {\setCJKmainfont{SimSun}}
% }}
\setCJKmainfont{STSong}[BoldFont=STHeitiSC-Medium,ItalicFont=STKaiti]

\setCJKmonofont{STFangsong}

% 简化的标点和符号设置
\punctstyle{kaiming}
\xeCJKsetup{CJKecglue={\hskip 0.15em plus 0.05em minus 0.05em}}

% 确保中文文本和标点符号使用中文字体的命令
\newcommand{\zhtext}[1]{{\cnfont #1}}

% 设置英文字体
\setmainlanguage{english}
\IfFontExistsTF{CMU Serif}
{\setmainfont{CMU Serif}}
{\IfFontExistsTF{DejaVu Sans}
  {\setmainfont{DejaVu Sans}}
  {\setmainfont{Georgia}}
}

% 添加中英文混排支持
\XeTeXlinebreaklocale "zh"
\XeTeXlinebreakskip = 0pt plus 1pt

% 确保特定中文字符总是使用中文字体
\newCJKfontfamily\cnfont{STSong}
\newcommand{\cn}[1]{{\cnfont #1}}

% 设置页眉页脚样式
\pagestyle{fancy}
\fancyhf{}
\fancyhead[LE,RO]{\thepage}
\fancyhead[RE]{\leftmark}
\fancyhead[LO]{\rightmark}
\renewcommand{\headrulewidth}{0.4pt}
\renewcommand{\footrulewidth}{0pt}
\setlength{\headheight}{14.5pt}

% 设置章节标题格式
\titleformat{\chapter}[display]
{\normalfont\bfseries\Huge}
{\chaptertitlename\ \thechapter}{20pt}{\Huge}
\titlespacing*{\chapter}{0pt}{50pt}{40pt}

% 设置节标题格式
\titleformat{\section}
{\normalfont\Large\bfseries}
{\thesection}{1em}{}
\titlespacing*{\section}{0pt}{3.5ex plus 1ex minus .2ex}{2.3ex plus .2ex}

% 设置段落间距和首行缩进
\setlength{\parindent}{2em}
\setlength{\parskip}{1ex}
\onehalfspacing

%New command to display footnote whose markers will always be hidden
\let\svthefootnote\thefootnote
\newcommand\blfootnotetext[1]{%
  \let\thefootnote\relax\footnote{#1}%
  \addtocounter{footnote}{-1}%
  \let\thefootnote\svthefootnote%
}

%Overriding the \footnotetext command to hide the marker if its value is `0`
\let\svfootnotetext\footnotetext
\renewcommand\footnotetext[2][?]{%
  \if\relax#1\relax%
    \ifnum\value{footnote}=0\blfootnotetext{#2}\else\svfootnotetext{#2}\fi%
  \else%
    \if?#1\ifnum\value{footnote}=0\blfootnotetext{#2}\else\svfootnotetext{#2}\fi%
    \else\svfootnotetext[#1]{#2}\fi%
  \fi
}

% 设置目录格式
\renewcommand{\contentsname}{\centerline{\Large\bfseries 目录}}
\renewcommand{\cftbeforetoctitleskip}{0pt}
\renewcommand{\cftaftertoctitleskip}{2em}
\renewcommand{\cftchapdotsep}{\cftdotsep}
\renewcommand{\cftchapfont}{\bfseries}
\renewcommand{\cftsecfont}{\normalfont}
\renewcommand{\cftsubsecfont}{\normalfont\itshape}

\begin{document}

% 标题页
\begin{titlepage}
\centering
\vspace*{2cm}
{\color{black!80}\rule{\textwidth}{1pt}}
\vspace{1.5cm}

{\Huge\bfseries\color{black!90} 逻辑学导论 \par}
\vspace{0.5cm}
{\large\itshape 系统梳理逻辑学基本概念与方法 \par}

\vspace{1.5cm}
{\color{black!80}\rule{\textwidth}{1pt}}
\vspace{2cm}

\begin{minipage}{0.8\textwidth}
\centering
\begin{quotation}
\large\textit{"逻辑将使人们明辨是非,认清谬误,\\
培养批判性思维,提高分析能力。"}
\end{quotation}
\end{minipage}

\vfill

{\large \today \par}
\end{titlepage}

% 前言
\frontmatter
\chapter*{前言}
% 内容简介

\begin{center}
\rule{0.5\textwidth}{0.4pt}
\end{center}

\begin{quotation}
\large\textit{『逻辑学为人类思维提供指引,就如同灯塔为迷航的船只指明方向。』}
\end{quotation}

\begin{center}
\rule{0.5\textwidth}{0.4pt}
\end{center}

\vspace{1em}

本书旨在系统地介绍\textbf{逻辑学}的各个方面,从基础概念到高级主题,力求用简明有逻辑的语言将逻辑学的理论和应用梳理清楚。全书结构清晰,内容由浅入深,既有理论探讨,又有实践指导。

作为一门研究\textbf{推理方法}与\textbf{原则}的学科,逻辑学在哲学、数学、计算机科学等诸多领域都有着重要的应用。本书从逻辑学的基本概念出发,逐步深入到以下核心内容:

\begin{itemize}
  \item \textbf{命题逻辑}:研究命题之间的逻辑关系
  \item \textbf{谓词逻辑}:分析命题内部结构
  \item \textbf{模态逻辑}:处理必然性与可能性
  \item \textbf{归纳逻辑}:探讨归纳推理的方法
  \item \textbf{逻辑悖论}:分析逻辑中的矛盾现象
  \item \textbf{非经典逻辑}:介绍多值逻辑等新发展
\end{itemize}

本书既可作为大学逻辑学课程的参考教材,也适合对逻辑学感兴趣的读者自学使用。

\begin{center}
\fbox{\parbox{0.9\textwidth}{
  \centering
  通过学习本书,读者将能够:\\
  \begin{minipage}{0.85\textwidth}
  \begin{itemize}
    \item 识别和评估日常生活中的论证
    \item 避免常见的逻辑谬误
    \item 构建清晰、有效的论证
    \item 培养批判性思维能力
  \end{itemize}
  \end{minipage}
}}
\end{center}

\vspace{1em}

希望本书能够帮助读者培养\textbf{逻辑思维},提高分析问题和解决问题的能力,在这个信息爆炸的时代,具备清晰、理性的思考方式。

% 目录
\tableofcontents
\clearpage

% 正文
\mainmatter

% 第一部分:逻辑与语言
\chapter{逻辑与语言}
% 本章导言
\begin{quote}
\textit{逻辑学是研究\textbf{正确推理方法}和\textbf{原理}的学问。本部分介绍逻辑学的基本概念,帮助读者建立对逻辑学的初步认识,为后续内容奠定基础。通过理解逻辑的本质、论证的结构以及常见谬误,读者将能够识别和构建有效的论证。}
\end{quote}

\section{什么是逻辑学}

\begin{quotation}
\textit{逻辑学为我们提供了判断推理正确性的方法和标准,使我们能够有效地思考和表达自己的想法。}
\end{quotation}

\textbf{逻辑学}是研究用于区分\textbf{正确推理}与\textbf{不正确推理}的方法和原理的学问。正确推理的界定有着许多客观标准,而如果不了解这些标准,也就无法运用它们。逻辑学研究的宗旨,就是发现并塑述这些标准,使之能够检验论证,把好的论证与坏的论证区别开来。

\subsection{逻辑学的研究对象}

逻辑学家所关心的推理遍及所有领域:
\begin{itemize}
  \item 科学与医药
  \item 伦理与法律
  \item 政治与商务
  \item 运动与博弈
  \item 平凡的日常生活
\end{itemize}

其中所使用的多种多样的推理,都是逻辑学家感兴趣的。本书将要分析的大量论证,就涉及许多非常不同的领域。但我们所关心的不是这些论证的\textbf{题材},而始终是它们的\textbf{形式}(form)与\textbf{品质}(quality),目的在于学会如何检验与评价论证。

\subsection{论证与推理}

逻辑学家并不关心推理的思想\textbf{过程},而只关心这种过程的\textbf{结果},即\textbf{论证}。论证是推理的产品,可以被完整地写出来,并予以检验与分析。对逻辑学家来说,就每一个论证都可提出如下问题:
\begin{enumerate}
  \item 论证所得出的\textbf{结论}是从论证所使用的\textbf{前提}或假定推出的吗?
  \item 论证的前提能够为接受其结论提供良好的理由吗?
\end{enumerate}

如果论证的前提的确能够为接受结论提供充分的根据,也就是说,如果断定前提为真就能够保证可断定结论为真,那么其所使用的推理就是\textbf{正确的},否则就是\textbf{不正确的}。

\subsection{逻辑学习的意义}

不能说只有学了逻辑学才能进行良好的或正确的推理,正如不能说只有学了生理学的运动员才能跑得快一样。并不懂得发生在其身体上的实际过程的运动员经常有出色的表现,而有些学习生理学的优等生,尽管有许多关于身体机能方面的知识,但在运动场上却难有作为。同样,学了逻辑学并不能确保能够进行正确的推理。

然而,一个学了逻辑学的人,比之一个从未思考过推理原理的人,其进行正确推理的可能性要大得多。这主要有两个原因:

\begin{enumerate}
  \item 学习逻辑学可以习得许多检验推理的正确性的方法,能够更容易地识别\textbf{推理错误},从而使这些错误不容易在推理中滞留。在这些被识别出的错误中,有些普通的\textbf{推理谬误},或所谓"自然"错误,是只要把它们充分弄清就很容易避免的。
  
  \item 学习逻辑学能够提高人的推理素养,它给了人们训练(practice)\textbf{分析论证}以及\textbf{建构论证}的机会。推理是一种我们不但要学而且要做(do)的事情,因而其既属\textbf{科学}亦属\textbf{艺术},需要把握技术和开发技能。就此目标,本书提供了丰富的习题训练,以增强这种技术与技能。
\end{enumerate}

\subsection{逻辑学的局限与价值}

在人类生活中,有些事情并不能完全用逻辑方法加以分析,有些问题并不能用论证(即使是良好的论证)来解决。有时求助于\textbf{情感}比逻辑论证更有效力,在某些语境中或许也更为适当。但是,在那些必须依靠理性判断的地方,\textbf{正确推理}终究是其最坚实的基础。

\begin{center}
\fbox{\parbox{0.9\textwidth}{
  \centering
  \textbf{逻辑学的价值}\\
  运用逻辑学的方法与技术,人们可以有效地区分正确的推理与不正确的推理,\\
  这种方法与技术就是本书的主题。
}}
\end{center} 
\section*{1.2命题与语句}
任何论证都是由命题构成的,故我们从讨论命题人手。命题是一种可以被肯定或否定的东西。也就是说,命题不同于问题、命令和感叹。问题可以被提问,命令可以被下达,感叹可以被发出,但它们本身都不能被肯定或否定。唯有命题断定了事情是(或不是)如此这般,因而也唯有命题才会是真的或者是假的。真与假并不适用于问题、命令或感叹。

再者,任一命题必是或真或假的,尽管我们可能并不知道某一特定命题究竟是真的还是假的。"宇宙中其他星球上有生命存在"这个命题,就是一个我们迄今还不知道其真假的命题。但对地球外生命之存在的这种断定本身或者是真的,或者不是真的。简言之,或真或假是命题的一个基本特征。

依学界惯例,要把命题与用来断定命题的语句区别开来。两个由不同语词以不同方式组成的语句,可能在同一语境中具有同样的意义,被用来表达同一个命题。例如:

Leslie won the election.(莱斯利赢了这场选举。)\\
The election was won by Leslie.(这场选举由莱斯利赢得。)

这显然是两个不同的语句,前一个由四个词组成,后一个是六个词,以及起首词不同等等。而这两个陈述句无疑具有相同的意义。命题这个术语所指谓的就是人们通常使用陈述句所断定的东西。

再者,一个语句总是使用它的特定语言的语句,而命题并不属于任何

特定的语言,一个特定的命题可以在许多语言中被断定。例如:

It is raining.(天在下雨。下同)\\
Está lloviendo.\\
Il pleut.\\
Es regnet.

这当然是四个不同的语句,分属不同的语言:英语、西班牙语、法语和德语。但它们都具有同样的意义,从而都可以用来断定同一命题。

在不同的语境中,同一个语句也可能被用来做非常不同的陈述。例如:

美国最大的州曾经是一个独立的共和国。

这个语句在 20 世纪上半叶说出,就是做了关于得克萨斯州的一个真陈述;而在现在说出就做了关于阿拉斯加州的一个假陈述。显然,时间语境的变化,可以使完全相同的语句断定非常不同的命题或陈述。("命题"和"陈述"这两个术语并不完全同义,但在逻辑研究的文本中它们经常被用做同义词。有些逻辑学专家更喜欢使用"陈述"而不愿意使用"命题",但在逻辑学历史上后者更为常用。本书同时使用这两个术语。)

上面所举出的命题的例子都是简单命题:"莱斯利赢了这场选举", "天在下雨"等等,然而命题也经常是复合的——在一个命题中包含着别的命题。考虑如下关于1945年希特勒第三帝国末日的一段话:

美军与俄军正迅速赶往易北河会师。英军已兵临汉堡和不来梅城下,把占领丹麦的德军置于被切断后路的险境。意大利的波伦亚已经失守,而亚历山大率领的盟军部队正向波河流域挺进。俄军已于4月13日攻克维也纳,正沿着多瑙河乘胜前进。 ${}^{[1]}$

这段话就含有几个复合命题。例如,"英军已兵临汉堡和不来梅城下",就是"英军已兵临汉堡城下"与"英军已兵临不来梅城下"这两个命题的联言式。而这个联言命题本身又作为分支属于一个更大的联言命

题:"英军已兵临汉堡和不来梅城下,(英军)把占领丹麦的德军置于被切断后路的险境。"这段话中的每一个命题都是被肯定的,也就是说,都被断言为真。肯定两个命题的联言式,就等于同时肯定这两个分支命题。

但是,也有一些复合命题并不断定其所有分支命题为真。例如:

巡回法庭或者是有用的,或者是无用的。 ${}^{[2]}$

这是一个选言命题(或称析取命题),它并没有肯定任何一个分支命题,而只是肯定了整个复合的"或者一或者"析取命题。析取命题为真时,其某个分支命题可以为假。再如:

如果上帝不存在,则有必要淫造一个上帝。 ${ }^{[3]}$

这个复合命题是一个假言命题(或称条件命题),其支命题也同样没有被肯定,既没有肯定"上帝不存在",也没有肯定"有必要捏造一个上帝",而只是通过这种假言或条件陈述肯定了整个"如果一则"命题。即使分支命题均为假,条件陈述亦可为真。

本书将逐次分析多种简单命题和复合命题的内在结构。 
\section{论证、前提与结论}

\begin{quotation}
\textit{论证是逻辑思维的核心,通过合理的前提推导出有效的结论是逻辑学的基本过程。}
\end{quotation}

命题是构成\textbf{论证}的部件。\textbf{推论}这个术语则指谓以一个或更多命题作为出发点,得出另一命题的过程。逻辑学家即通过检验这种过程的出发点与结果及它们之间的关系,以判定一个推论是否正确。这种命题系列即构成一个论证。因而对于任一可能的推论,都有一个相应的论证。

\subsection{论证的本质}

论证是逻辑学所关心的主要对象。逻辑学家所使用的\textbf{论证}一词,就是指谓任一这样的命题组:一个命题从其他命题推出,后者给前者之为真提供支持或根据。当然,"论证"一词也经常在其他含义上使用,但在逻辑学中严格地限于上述含义。

显然,在这种严格含义上,一个论证不只是一组命题的汇集,一段包含一些相互关联的命题的话语可能并不包含任何论证。若要给出一个论证,则命题系列必须含有一种结构,对这种结构的描述通常要使用\textbf{前提}与\textbf{结论}这两个术语。一个论证的结论,就是以论证中的其他命题为根据所得出的那个命题,而这些其他命题,即被肯定(或假定)为接受结论的根据或理由的命题,则是该论证的前提。

\subsection{最简论证的形式}

最简单的论证是由一个前提和一个从该前提推出或被它所蕴涵的结论构成的论证。这种论证的前提与结论可以分别用两个不同的语句表述,例如出现在阿拉巴马州地理课本封签中的如下论证:

\begin{quotation}
在地球上最先出现生命时没有人存在。因此,任何关于生命起源的陈述都应视为理论的而非事实的陈述。
\end{quotation}

最简论证的前提和结论也可能被表述在同一个句子中,如下述论证:

\begin{quotation}
因为最近的进化史研究已经证明所有人都是从同一小群非洲祖先演变而来,若仍相信种族间有极大差异,则如同仍相信地球是扁平的一样荒谬可笑。${}^{[4]}$
\end{quotation}

即使在最简论证中,结论陈述也有可能出现在那个唯一的前提之前。这时候,两个命题同样既可以两个语句出现,亦可在同一个句子中出现。前者例如:

\begin{quotation}
食品与药物管理局应立即禁止烟草买卖。要知道,抽烟是导致死亡的一种最可预防的原因。${}^{[5]}$
\end{quotation}

同一陈述中所表述的论证结论在前的一个例子是:

\begin{quotation}
凡法皆恶,乃因凡法皆为自由之违背。${}^{[6]}$
\end{quotation}

大多数论证都比这些论证复杂得多。我们将会看到,有些非常复杂的论证包含由多个支命题构成的复合命题。但是不管简单还是复杂,任何论证都是由一组命题构成,其中一个命题是\textbf{结论},其他命题是用以支持结论的\textbf{前提}。

\subsection{论证与假言命题的区别}

因为一个论证由一组命题构成,故而单一命题自身不可能是论证。但有些复合命题与论证非常近似,需细心辨识以免把它们混同于论证。考虑如下假言命题:

\begin{displayquote}
如果火星在其具有与地球相似的大气层和相似气候的早期曾有生命演化,那么目前科学家确信的在我们的星系中存在的无数颗其他星球上也会有生命演化。
\end{displayquote}

在这个\textbf{假言命题}中,无论第一个支命题"火星在其具有与地球相似的大气层和相似气候的早期曾有生命演化",还是第二个支命题"目前科学家确信的在我们的星系中存在的无数颗其他星球上也会有生命演化",都没有被肯定。整个命题肯定的只是前者蕴涵后者,而两者却可以都是假的。其中没有推论得以构成,没有结论被论证为真。这是一个假言命题,而不是一个论证。现再请考虑如下段落:

\begin{quotation}
看来,目前科学家确信的在我们的星系中存在的无数颗其他星球上会有生命演化,因为火星在其具有与地球相似的大气层和相似气候的早期非常可能曾有生命演化。${}^{[7]}$
\end{quotation}

此处我们的确得到一个\textbf{论证}。命题"火星非常可能曾有生命演化"被肯定为一个前提,而命题"无数颗其他星球上会有生命演化"被从该前提推出并被论证为真。这样,假言命题可能看上去很像一个论证,但其并不是一个论证,两者不应混淆。如何识别论证,是后面1.5节讨论的主题。

\subsection{有结构的命题系列与论证}

最后应强调指出,任一论证都是有结构的命题系列,但并非任一有结构的命题系列都是论证。请考虑从近期的非洲游记上摘录的一段话:

\begin{displayquote}
骆驼并不在驼峰中储水。它们每次喝水都非常猛,在十分钟的时段中能饮入28加仑水,把这些水均匀地分布到全身。而后其耗水却非常节俭。它们的尿液黏稠、粪便干燥,并以浅呼吸而紧闭其口。如非不得已,它们一般不出汗……在失水程度达到体重的三分之一时也能存活,然后再痛饮一次并且感觉良好。${}^{[8]}$
\end{displayquote}

这段有结构的命题系列中并没有任何论证。

\begin{center}
\fbox{\parbox{0.9\textwidth}{
  \centering
  \textbf{论证的基本结构}\\
  论证是由前提和结论组成的有结构命题系列,\\
  其中结论由前提通过推理过程得出。
}}
\end{center} 
\section*{1.4 论证的分析}
许多论证是简单的,但有些论证却相当复杂。论证的前提可以用各种不同方式支持其结论。论证中前提的数量和命题顺序也有所不同。我们需要关于论证性话语的一些分析技法,借以澄清前提与结论的关联。

通常有两种分析技法用于论证分析。一种是解析(paraphrase),即 12用清楚的语言和逻辑顺序表明论证中的命题;另一种是图示(diagram),用二维空间关系图展示论证的结构。两种技法都很有用,可根据不同情况选用最方便适用的一种。

\section*{A.解析法}
考虑下面这个多于两个前提的论证的原初表述:

现代鸟类并非从直立行走的兽脚类恐龙(包括霸王龙)进化而来,有三个主要理由。首先,大多数类鸟兽脚类恐龙化石发源时间比初始鸟类遗留的化石晚七千五百万年。……其次,鸟的祖先必定已适宜飞行,而兽脚类恐龙并不适宜飞行。第三个理由在

于...…兽脚类恐龙都有锯状牙齿,而鸟类没有锯状牙齿。 ${}^{[10]}$

我们可通过解析澄清该论证,即利用清楚简明的语言列出其每一个前提及结论:

1.类鸟兽脚类恐龙化石比初始乌类遗留的化石发源时间要晚得多。\\
2.乌的祖先必定已适宜飞行,但兽脚类恐龙不适宜飞行。\\
3.兽脚类恐龙都有锯状牙齿,而鸟类没有锯状牙齿。\\
所以,现代鸟类并非从直立行走的兽脚类恐龙进化而来。

对论证的分析通常能帮助我们更好地理解论证,因为做这样的分析必须把在论证中被假定但没有充分明晰地陈述的东西揭示出来。辟如大数学家哈代曾有如下论说:

阿基米得将被永远记住而埃斯库罗斯会被遗忘,因为一种语言会消亡而数学理念不会消亡。 ${}^{[11]}$

对该论证的充分的分析,需清楚地表明其所承诺的东西:

1.一种语言会消亡。\\
2.埃斯库罗斯的伟大剧作使用一种语言。\\
3.故埃斯库罗斯的成果终究会消亡。\\
4.数学理念不会消亡。\\
5.阿基米得的伟大工作使用数学理念。\\
6.故阿基米得的成果不会消亡。\\
所以,阿基米得将被永远记住而埃斯库罗斯将被遗忘。

这种分析使我们看到,哈代这一短句之中,浓缩了几个带有可疑前提的论证。

\section*{B.图示法}
有时运用图示展示一个论证的结构是非常有益的。其步骤是给论证中

出现的每一个命题逐次赋予一个置于圆圈中的数字,然后在数字间使用箭头符号展示其中前提与结论的逻辑关联。这样可避免像解析法那样重述前提。考虑如下论证:\\
(1)与许多人的认识相反,HIV 检测呈阳性并不必定是死亡判决。一方面,(2)从(艾滋病病毒)抗体生发到出现临床症状平均将近十年时间;另一方面,(3)许多研究报告显示,相当数量的检测呈阳性者从未发展为艾滋病患者。 ${}^{[12]}$

不用重述论证中的命题,使用标示命题的圆圈数字即可把该论证图示如下:\\
\includegraphics[max width=\textwidth, center]{2025_05_15_6a28331d5e7c993ad07ag-030}

如果一个论证是简单直接的,则完全不需要借助图示去理解它。然而论证往往不是直接的,而图示法能够在平面图上直观地显示论证的结构,因而是非常有用的。 ${ }^{[13]}$ 我们在图中把结论置于前提的下方,而论证的所有前提都在图中的同一行上列出。

与解析法相比,图示法更易于展现论证的前提支持结论的方式。例如在上面这个论证中,前提(2)和(3)都分别独立地支持结论(1):HIV 检测呈阳性并不必定是死亡判决。就是说,每个前提自身都为接受结论提供了某种理由,即使没有另一个前提也不影响其为结论所提供的支持。这种分立性支持直观地展现在图示之中。

但是,在有些论证中,只有把前提结合在一起才能达到支持结论的目的。例如:

该论证的正确图示将展现出只有当前提之间相结合才能支持结论,即:\\
\includegraphics[max width=\textwidth, center]{2025_05_15_6a28331d5e7c993ad07ag-031(1)}

此处我们用托架线置于前提之下,是因为在这个例子中两个前提都不能独立地支持结论。如果第一个前提表达的原则是真的,但不存在能够最适当地维护所有当事人的利益的安乐死事例,则结论就根本没有得到支持。而如果的确存在能够最适当地维护所有当事人的利益的安乐死情形,但第一个前提所表述的原则是错误的,则结论仍然没有获得支持。

当论证有更复杂的结构时,图示法就显得特别有用。有时可以很容易地展示出原本很难说清楚的东西。考虑如下论证:

\begin{displayquote}
(1)沙漠高地是天文观测的良好场所。(2)其高度使得它们坐落于大气层之中,使得星光不用穿越整个大气层而到达望远镜。 (3)沙漠的千燥度也使之相对较少受云雾千扰。(4)云雾对天空的遮蔽会使许多天文观测归于无用。 ${}^{[15]}$
\end{displayquote}

命题(1)显然是这个论证的结论,其他三个命题提供对它的支持,但它们支持结论的方式是不一样的。命题(2)自身即可支持沙漠高地是天文观测良好场所的断言,而命题(3)和(4)必须联合起来才能支持这个断言。如下图示可清楚地表明这一点:\\
\includegraphics[max width=\textwidth, center]{2025_05_15_6a28331d5e7c993ad07ag-031}

但是某些复杂论证结构的澄清使用解析法更为奏效。例如,当一个论证含有未明确陈述出来的隐含前提时,解析法允许我们直接把隐含前提列出,而图示法则需要既列出隐含前提又要以某种直观形式(如非封闭圆圈)表明它是被附加到原来论说之上的。请考虑如下论证:

只有当我能够做出其他选择时,我对我的行为才负有道德责任。因为一个人若无力避免某行为,就不应被认为对该行为负有道德责任。 ${ }^{[16]}$

运用列出隐含前提的解析法,该论证很容易澄清如下:

1.一个人若无力避免某行为,就不应被认为对该行为负有道德责任。\\
2.只有当我能够做出其他选择时,我当下的行为才是我有能力避免的。所以,只有当我对我的行为负有道德责任时,才有我的行为是否应当的问题。

\section*{C.多重复合论证}
当一段话包含两个或更多论证和若干相互关联并不明显的命题时,图示法被证明特别有用。下面是从马克思给恩格斯的一封信中摘录的一段话:

\begin{displayquote}
(1)加速英国的社会革命就是国际工人协会的最重要的目标。(2)而加速这一革命的唯一办法就是使爱尔兰独立。因此, (3)国际的任务就是到处把英国和爱尔兰的冲突提到首要地位, (4)到处都公开站在爱尔兰方面。 ${ }^{[17]}$
\end{displayquote}

一段话中论证的数目通常取决于其中所含结论的数目。这段话中含有两个结论,因而有两个论证。但这两个结论都是从同样的两个前提推得的,如 16下图示可很好地展示这种结构:\\
\includegraphics[max width=\textwidth, center]{2025_05_15_6a28331d5e7c993ad07ag-032}

有时,含有两个结论从而有两个论证的一段话,却只含有一个前提。例如:

年纪较大的妇女更难以抵制工作中的性骚扰和离开施暴的丈夫,因为年龄的偏见使她们不容易找到其他保护自己的方式。 ${ }^{[18]}$

其中唯一的前提是年纪较大的妇女不容易找到保护自己的方式。该前提支持两个结论:年纪较大的妇女难以抵制工作中的性骚扰,以及她们(对已婚妇女而言)难以离开施暴的丈夫。我们通常用"单独论证"一词指谓只有一个结论的论证,而不管有多少用以支持它的前提。

当一段话中出现两个或更多论证,或一个论证中有两个或更多前提时,就需要弄清各个前提及结论出现的次序。结论可能在最后或最先出现,也可能出现在用以支持它的前提之间,如下例:

移斯林思想家启示的真正来源是《古兰经》及神圣先知的言论。因而很显然,移斯林哲学并不是希腊思想的复制品,其所关心的主要是那些来自移斯林和与穆斯林相关的特定问题。 ${ }^{[19]}$

这段话中的结论"穆斯林哲学并不是希腊思想的复制品",出现在论证的第一个前提之后和第二个前提之前。

同一个命题既可在一个论证中做结论,又可在另一个论记 中做前提,正如同一个人既可在一个场合做指挥者,又可在另一个场合做被指挥者。托马斯-阿奎那的著作中有一段话可以很好地说明这一点。他说:

\begin{displayquote}
人类的法律是为人类大众制定的,\\
大多数人在德行上是不完美的,\\
因此人类的法律不禁止一切罪恶。 ${ }^{[20]}$
\end{displayquote}

17 该论证的结论随即被托马斯-阿奎那用做另一个完全不同的论证的前提:

\begin{displayquote}
恶行与善行相反,\\
但人类的法律不禁止一切罪恶,\\
因此人类的法律也不规定一切善行。 ${ }^{[21]}$
\end{displayquote}

以被压缩,对于这样一串浓缩论证的分析,完全解析法会提供很大的帮助。考虑如下论证集合:

因为(1)出现在非洲人种身上的线粒体变种最多,科学家推断,(2)非洲人种的进化史最长,这表明(3)非洲人种可能是现代人类的起源。 ${ }^{[22]}$

我们可以把这段论证图示如下:\\
\includegraphics[max width=\textwidth, center]{2025_05_15_6a28331d5e7c993ad07ag-034}

而对这同一串论证的分析,解析法尽管显得不够简洁,但更完整:

1.一个人种身上的线粒体变种越多,其进化史就越长;\\
2.出现在非洲人种身上的线粒体变种最多,因此非洲人种进化史最长。

1.非洲人种进化史最长,\\
2.现代人类可能起源于进化史最长的人种,因此现代人类可能起源于非洲人种。

这样的复合论证表明,一个孤立表达的命题既非前提也非结论。在一个论证中,作为假定出现的命题就是前提,被断定为从假定命题推出的命题就是结论。也就是说,"前提"和"结论"都是相对的(relative)术语。

几个论证复合在一起,语言表达上可能不是以串联的方式出现,而是以独特的方式相互交织,这就要求我们对它们做细致的分析。图示法特别适用于这种情况。例如,在约翰•洛克的名篇《政府论》中,下面一段话

就有两个论证交织在一起:

立法机构常年运作是不必要的,也是很不方便的;但行政机关常年运作是绝对必要的,因为不是总需要制定新的法律,但总需要执行已制定的法律。

上述论证的分支命题可以用数字表示为:(1)立法机构常年运作是不必要的,也是很不方便的,(2)行政机关常年运作是绝对必要的,(3)不是总需要制定新的法律,(4)总需要执行已制定的法律。将这段论证图示如下:\\
\includegraphics[max width=\textwidth, center]{2025_05_15_6a28331d5e7c993ad07ag-035}

这个图示表明,第二个论证的结论出现在第一个论证的结论和前提之间,第一个论证的前提出现在第二个论证的结论和前提之间。这个图示还表明,两个结论都出现在它们的前提之前。

这个图示同样也展示了支持刑罚威骤理论的古罗马哲学家塞涅卡的两个相关论证的逻辑结构:\\
(1)惩罚罪行不是因为罪行已经发生,(2)而是为了不发生新的罪行。[因为](3)过去的罪行不能被取消,(4)但是可以预防将来的罪行。

在这段话中,"惩罚罪行不是因为罪行已经发生"是其中一个论证的结论,其前提是"过去的罪行不能被取消"。"惩罚罪行是为了不发生新的罪行"是这段话中第二个论证的结论,其前提是"惩罚罪行可以预防将来的罪行"。

简言之,图示法和解析法是两种有力的分析工具,运用这两种工具对论证进行分析,可以更彻底地理解论证前提与结论的关联。 
\input{chapter1/section1-5}
\section{论证和说明}

\begin{quotation}
\textit{区分论证和说明是逻辑分析的关键步骤,了解二者的差异能够帮助我们准确判断语段的真正意图。}
\end{quotation}

许多语段,无论是书面语还是口语,看起来好像是论证,实际上不是论证而是\textbf{说明}。即使有某些前提或结论指示词出现,例如"因为"、"由于"、"因此"等,也不能解决问题,因为这些语词既可用在论证中也可用在说明中。我们必须知道在这些语段中作者的意图。$^{[41]}$

\subsection{论证与说明的区别}

请比较下面两段话:

\begin{displayquote}
\textbf{例1:}为你自己积攒财宝在天上,天上没有虫子咬,不能锈坏,也没有贼挖窟窿来偷,\textit{因为}你的财宝在哪里,你的心也在哪里。
\end{displayquote}

\begin{displayquote}
\textbf{例2:}所以它(那座塔)名叫巴别,\textit{因为}耶和华在那里变乱天下人的言语。\\
——《创世记》11:19
\end{displayquote}

第一段话是一个清楚的\textbf{论证},它的结论,即一个人必须积攒财宝在天上,由前提(这里用"因为"来标明)一个人的财宝积攒在哪里,他的心也在哪里来支持。但是第二段不是论证,尽管它非常恰当地使用了"所以"一词。它\textbf{说明}了这座塔(其建造过程在《创世记》中有详细的叙述)为什么叫巴别。它告诉我们,因为之前人类在那里使用的是同一种语言,现在被许多语言变乱了,所以给塔起了这个名字。$^{[42]}$

这段话假设读者知道那座塔有这个名字,意图是说明为什么给塔起了这个名字。短语"所以它名叫巴别"不是结论而是完成了对这个名字的说明。句子"因为耶和华在那里变乱天下人的言语"不是前提,它不能作为相信巴别是那座塔名字的原因,因为巴别是那座塔的名字的事实是这段话所要为之做说明的读者所知道的。在这个语境中"因为"指示的是接下来要说明将巴别这个名字给予那座塔的原因。

\subsection{判断标准}

上面两段话说明一个事实,表面上相似的语段可能具有完全不同的功能。任何一个特定的语段究竟是\textbf{论证还是说明},这取决于那个语段所服务的\textbf{目的}。如果我们的目的是要\textbf{确立某个命题}$Q$的真,为此我们提出某个证据$P$来支持$Q$,我们可以恰当地说"$Q$因为$P$"。也就是说我们为$Q$建立一个论证,$P$是我们的前提。

但是假设$Q$是已知为真的。在这种情况下我们不必提出任何理由来支持它的真——但是我们可以希望对它为什么是真的给出一个\textbf{说明}。这样我们也可以说"$Q$因为$P$"——但在这种情况下,我们不是为$Q$建立一个论证,而是给出一个对$Q$的说明。

\subsection{实例分析}

在回答关于类星体(在我们的星系以外很远地方的一类天体)的外观颜色的问题时,一位科学家写道:

\begin{displayquote}
最远的类星体看上去像强烈的红外辐射光点。这是因为太空散布着吸收蓝光的氢微粒(大约每立方米两个微粒),如果你从可见的白光里过滤掉蓝光,那么剩下的就是红光。在其到达地球的数十亿光年的旅程中,类星体光被大气中的氢微粒吸去了全部的蓝光,留下的只有红光。$^{[43]}$
\end{displayquote}

这段话不是论证,它不是试图要让读者确信类星体具有像它们所显示的外观颜色,而是\textbf{说明}它们具有这个外观颜色的原因。

同样,在讨论不列颠对非洲早期发展的影响时,一个历史学家写道:

\begin{displayquote}
塞拉利昂在1808年成为英国直辖殖民地不是因为它的繁荣,而是因为它的萧条。由于战争和商业不景气的负担,塞拉利昂的私营公司不能支付它们的费用,而刚刚废除了贩卖奴隶制度的英国政府感到有必要接管它。$^{[44]}$
\end{displayquote}

这里没有对塞拉利昂在1808年成为英国直辖殖民地这个结论进行论证。塞拉利昂在那时确实成了英国直辖殖民地。但这是为什么呢?乃是由于在本例和前例中,"因为"很明显是\textbf{说明的标志},而不是论证的标志。

\subsection{区分方法}

我们怎么才能断定一个语段的目的是打算\textbf{说明}还是打算\textbf{说服}人呢?通常我们可以根据"$Q$因为$P$"这个形式提问,对于作者来说$Q$的身份是什么,以此来做出区分:

\begin{itemize}
  \item 若$Q$是一个其真实性\textbf{需要建立}的命题,那么"因为$P$"可能给出了支持其为真的前提,这样"$Q$因为$P$"就是一个\textbf{论证}。
  \item 若$Q$是一个已知其为真,或至少在这个语境中其真是没有疑问的命题,那么"因为$P$"就可能是对为什么$Q$成为真命题的阐释,这样"$Q$因为$P$"就是一个\textbf{说明}。
\end{itemize}

在一个说明中,人们必须把\textbf{被说明的东西}与\textbf{用来说明的东西}区别开来。在上面《创世记》所做出的说明中,被说明的内容是为何那座塔具有名字巴别,说明的内容是在那里耶和华变乱天下人的言语。在上面给出的历史学例子中,被说明的内容是塞拉利昂成为不列颠直辖殖民地,说明的内容是塞拉利昂公司的无支付能力和不列颠政府就此做出的回应。

\subsection{模糊界限}

有时被称做说明的东西实际上可能是论证,反之亦然。不久前,《纽约时报》因对待男女性别的不平等做法而受到一个读者的批评,因为它对一个著名女演员的不断增长的体重加以评论,但对在同一篇报道中提到的一个杰出商人的不断增长的体重却没有评论。后有另一个读者对此做出回应:

\begin{displayquote}
E.R.福克斯的抱怨——你特别提到凯瑟琳·丹尼芙"也许不像她以前那么苗条",但你没有提及唐纳德·杜鲁普不断增加的腰围——很容易说明。杜鲁普先生从未裸体出现在电影中以使他的体形成为人们感兴趣的事情。$^{[45]}$
\end{displayquote}

这不是一个真正的说明,而是一个\textbf{论证}。它有两个前提:
\begin{enumerate}
  \item 裸体外表出现在电影中使一个人的外表成为人们感兴趣的事情
  \item 杜鲁普先生从未以裸体外表出现在电影中,而丹尼芙女士有过
\end{enumerate}

因此,报纸对如此出现在电影中的名人的体形加以评论,而忽略未如此出现在电影中的名人的体形,这种做法就是合乎情理的(这个读者的主张),据此抱怨对待男女性别不平等就是不应当的。

为了区别说明和论证,我们必须对\textbf{语境}有一定的敏感性。总会有一些语段,其目的难以确定。一个其目的难以确定的语段可能需要给予两种同样有道理的"解读"——用一种方法去解读,被当做论证;用另一种方法解读,就是说明。

\begin{center}
\fbox{\parbox{0.9\textwidth}{
  \centering
  \textbf{论证与说明的关键差异}\\
  论证旨在证明某个命题的真实性,而说明则是解释已知为真的命题为何如此。\\
  辨别二者需要考察语段的目的、命题的地位以及语境因素。
}}
\end{center} 
\section*{1.7 演绎和有效性}
每一个论证都是断言其前提为结论的真提供理由。实际上这种断言正是论证的标志。但是论证有两大不同种类:演绎论证和归纳论证。这两类论证在其前提支持结论的方式上有着根本的不同。本节我们对演绎论证作一个简要的阐释。

任一演绎论证均断言其前提决定性地(conclusively)支持结论。相反,归纳论证均没有这种断言。在对一个语段的解释中,如果我们判定它做出了 43 这样的断言,我们就将其视为演绎论证;如果我们判定它没有做出这样的断言,我们就将其视为归纳论证。因为每个论证都会对决定性支持结论或者做出断言,或者不做出断定,所以每个论证或者是演绎的,或者是归纳的。

当一个论证断言它的前提(如果是真的)为它的结论的真提供了无可辩驳的理由时,这个断言或者是正确的或者是不正确的。如果是正确的,这个论证就是有效的。如果不是正确的(也就是说,即使前提是真的,也不能无可辩驳地确立其结论的真),那么这个论证就是无效的。

因此,对于逻辑学家而言,有效性这个术语仅仅对演绎论证才是适当的。说一个演绎论证是有效的,就是说如果其前提是真的,其结论为假就是不可能的。这样我们可把"有效性"定义如下:一个演绎论证是有效的,即如果其前提是真的,则其结论必定是真的。

每个演绎论证都要求其前提为其结论的真提供担保,但并非所有演绎论证都能做到这个要求。不能做到这个要求的演绎论证就是无效的。

因为每个演绎论证就其目标的实现而言或者是成功的或者是不成功的,所以每个演绎论证或者是有效的或者是无效的。这一点非常重要:如果一个演绎论证不是有效的,它一定是无效的;如果它不是无效的,它一定是有效的。

演绎逻辑的中心任务(将在本书第二部分详细讨论)就是对有效论证和无效论证做出区分。为此,古往今来逻辑学家们发明了许多非常有效的方法。但用来判定论证有效性的传统方法不同于大多数现代逻辑学家使用的方法。前者被叫做古典逻辑,发端于亚里士多德的分析工作,本书的第5、6、7章将对此加以阐释。现代符号逻辑的方法将在本书的第8、9、 10 章详加介绍。尽管两个流派的逻辑学家们在方法上和对某些论证的

具体阐释上不尽一致,但他们都同意演绎逻辑的主要任务是开发一种能使我们区分有效论证与无效论证的工具。 
\section*{1.8 归纳和或然性}
归纳论证不要求它们的前提必然地支持结论,纵然其前提是真的。它提出一个较弱的但仍然是很重要的要求:其前提或然性地支持结论。或然性总是必然性的缺乏,因而上述关于有效性和无效性的讨论并不适用于归纳论证:归纳论证既不是有效的也不是无效的。 ${ }^{[46]}$ 当然,我们仍然可以对它们进行评估。实际上,对归纳论证进行评估是任何领域的科学家最主要的任务之一。归纳论证的前提为它的结论提供某种支持,前提授予结论的或然性程度越高,论证的价值也就越大。一般情况下,我们可以说归纳论证"较好"或"较差","较弱"或"较强",等等。但是,甚至在所有前提都是真的并且对其结论提供了非常强的支持的情况下,归纳论证的结论也不是必然得出的。归纳理论,归纳推理的技巧,评估归纳论证的方法,以及量化和推测或然概率的方法等将在本书第三部分详加介绍。

归纳论证和演绎论证之间的区别是根本性的。因为归纳论证的前提对其结论的支持都具有某种程度的或然性,附加的信息就有可能强化或弱化这种或然性。新发现的事实可以使我们改变对或然性的估价,可能导致我们对归纳论证的判定比我们原想的更好(或更差)。在归纳论证的领域——即使当结论被认为具有很高可能性的情况下——永远不会穷尽所有的证据。正是这种发现与我们以前所相信的证据相冲突的新材料的可能性,使得我们不能断定任何归纳论证的结论具有绝对的确定性。

相反,演绎论证却不能越来越好或越来越差。它们在显示前提和结论之间的令人信服的关系上要么成功要么失败。这个对比揭示了演绎和归纳之间的根本差异。如果一个演绎论证是有效的,就没有附加的前提可以增强这个论证的有效性。例如,如果凡人皆终有一死,并且如果苏格拉底是人,我们就可以毫无保留地得出结论,苏格拉底终有一死——即苏格拉底终有一死的结论总能从那两个前提推论出来,而不管世界上别的什么可能是真的,也不管别的什么信息被发现或增加到该论证的前提当中。比如我们后来又知道苏格拉底难看,或天使永生,或奶牛产奶,但这些发现和别的发现都不能对原来的论证产生任何影响。

就每一个有效的演绎论证来说,不管附加前提的性质如何,从其前提必然推出的结论同样也能必然地从任何扩大的前提集推论出来。如果一个论证是有效的,世界上就没有什么东西能使它更有效;如果一个结论是从某个前提集有效推出的,就没有什么东西可以增加到这个前提集当中使得该结论的推出变得更严格、更合乎逻辑或更有效。

但归纳论证并不是这样,归纳论证所断言的前提和结论之间的关系远非如此严格,与演绎论证有本质上的不同。考虑下面这个归纳论证:

\begin{displayquote}
大部分公司法律顾问是保守主义者,\\
安吉拉•帕尔默瑞是一个公司法律顾问,所以安吉拉•帕尔默瑞很可能是保守主义者。
\end{displayquote}

45 这是一个非常好的归纳论证,它的第一个前提是真的,如果它的第二个前提也是真的,则其结论很可能就是真的而不是假的。但是在这种场合(与有关苏格拉底的必死性的论证形成鲜明对照),若增加某个新前提到原来的论证之中,就可能会弱化或强化(依据新前提的内容)原来的论证。假设我们还知道:

\section*{安吉拉•帕尔默瑞是美国公民自由权协会(ACLU)的一名官员。}
又假设在原论证中增加一个(真)前提:

美国公民自由权协会的大部分官员不是保守主义者。

那么那个结论(安吉拉•帕尔默瑞是一个保守主义者)不再看起来非常可能,原来的归纳论证由于这个关于安吉拉•帕尔默瑞的附加信息的出现而被大大弱化。而如果上述前提被改造成全称命题:

没有美国公民自由权协会的官员是保守主义者。

那么就会有效地从被断定的前提集演绎地推出与原来结论相反的结论。

另一方面,假设我们通过增加下面的附加前提来扩大原来的前提集:

安吉拉•帕尔默瑞长期是国家步枪协会(NRA)的一名宫员。

和\\
安吉拉•帕尔默瑞被任命为保守的《国家评论》报的特约撰稿人。

那么通过这个扩大了的前提集,原来的结论就得到了比原来的前提集更大的支持。

总之,归纳和演绎的区别依赖于两类论证对前提和结论之间的关系所作断言的性质。我们可以将两类论证的特征表示如下:

演绎论证是一种其结论被断言为从其前提绝对必然地推出的论证,这种必然性不是一个程度问题,不以任何其他事物情况为转移。反之,归纳论证是一种其结论被断言为仅仅或然性地从其前提推出的论证,这种或然性是一个程度问题,其程度受可能出现的其他事物情况的影响。

归纳论证并不总是明确表明其结论仅仅是在某种或然程度上推出来的。另一方面,在一个论证中出现"或然性"一词也并不一定表明该论证就是归纳的。这是因为有一些严格的演绎论证是关于或然性本身的。 ${ }^{[47]}$ 在这种论证中,事件之间的确定联系的或然性是从另外的事件的或然性演绎推出的,这个问题将在第 14 章讨论。
\input{chapter1/section1-9}
\input{chapter1/section1-10}
\section{推理}

\begin{quotation}
\textit{推理是逻辑学的实践应用,通过系统化的思考方法和训练,我们能够解决复杂问题并发展批判性思维能力。}
\end{quotation}

如前所述,逻辑学是研究用于区分正确推理与不正确推理的方法和原理的学问。\textbf{推理}与\textbf{论证}都是从已知的(或为了某种目的而肯定的)前提推出结论的过程。至此,我们一直在分析和评估的是别人的论证。当然,我们在决定自己应如何行动、评价他人的行动、为道德的或政治的信念进行辩护等方面,我们每天都要建构我们自己的论证。建构和运用好的论证之技能是有巨大价值的。

推理的技能可以通过训练来提高。为了促进这样的训练,许多推理游戏(例如国际象棋、围棋、Mastermind ${ }^{(1)}$ 等)都是极好的手段。而那些用以强化和测试我们的逻辑技能的推理谜题,也无疑是相当实用的。推理不仅是一项必要的活动,也是一项愉快的活动——当我们解决了为提高技能和使人娱乐双重目的而设计的一些逻辑问题时,它所产生的乐趣是不言而喻的。

人为设计的问题比现实生活中的问题更简洁,通常也更简单。但是解答它们是具有挑战性的,经常需要锲而不舍地反复推理,这在思考模式上与侦探或新闻工作者或陪审员没有很大不同。可能需要找到一个推理系列,在这个系列中,所得的次结论被用做后来推理的前提。也可能要求具有一定的洞察能力,要找到解决问题的路径需要对早先假设或发现的信息进行创造性的重组。解决人为设计的问题往往是比较困难的,有时会无功而返,但是当通过推理的成功应用解决了问题时,是非常令人满足的。逻辑游戏和谜题解答,伴随各种推理模型的运用,都是很好的娱乐。"对思虑的享受",美国哲学家约翰•杜威写道,"是受过训练的大脑的标志"。

\footnotetext{(1)种著名的网络智力游戏。
}

\subsection{推理问题的类型与解法}

推理问题的一个常见类型是智力测验,仅仅使用所提供的线索,我们被要求理清和辨识有关的几个人物的名宁,或角色,或其他事实。下面是一个比较简单的例子:

\begin{displayquote}
在某个航班的全体乘务员中,飞机驾驶员、副驾驶员和飞行工程师的职务由爱伦、布朗和卡尔三人担任,但不必是这个次序。\\
副驾驶员是个独生子,钱挣得最少。\\
卡尔与布朗的姐姐结了婚,钱挣得比驾驶员多。\\
问:三个人每人担任什么职务?
\end{displayquote}

为了解答这样的问题,我们首先要寻找一个范围,在这个范围中我们有足够的信息去得到超出前提所给信息的一些结论。我们从前提中知道许多关于卡尔的情况:他不是飞机驾驶员,因为他挣得比驾驶员多;他也不是副桇驶员,因为副驾驶员挣得最少。通过排除我们可以推出,卡尔一定是飞行工程师。

使用上述的次结论,我们可以确定布朗的职务。布朗不是副驾驶员,因为他有一个姐姐而副驾驶员是个独生子;他也不是飞行工程师,因为卡尔是飞行工程师。所以布朗一定是飞机驾驶员,而仅剩的爱伦一定是副驾驶员。

\subsection{矩阵分析法}

在解答这类问题(有时非常复杂)时,建构一个备选项的图示是非常有帮助的,这种图示叫做\textbf{矩阵},当我们积累了新的信息时就把它填人表中。要见识这种矩阵图的作用,请考虑下面的问题:

\begin{displayquote}
阿伦佐、库特、鲁道夫和威拉德是四个天资极高的创造性的艺术家。一个是舞蹈家,一个是画家,一个是歌唱家,一个是作家,但不必是这个次序。\\
(1)那天晚上歌唱家在音乐会舞台上进行他的首次演出时,阿伦佐和鲁道夫在观众席上。\\
(2)库特和作家两人有画家为他们画的生活肖像。\\
(3)作家正准备写一本阿伦佐的传记,他写的威拉德的传记是畅销书。\\
(4)阿伦佐从未听说过鲁道夫。\\
问:每个人的艺术领域是什么?
\end{displayquote}

将这些前提中断定的许多事实记在头脑中,也记住几个可以从这些前提推出的分结论,这是一项必需的工作。把我们的推论记在便条上可能是有帮助的,但是也可能导致混淆和零乱。我们需要一种有效方法,来贮备已知的信息和所引出来的中间结论,能把已知的信息和推出的信息整齐地记录下来,并随着推论的数目不断增长以及论证的链条不断拉长供我们使用。而在我们要建构的矩阵表中就有空间去表示所有相关的可能选择,并能记录下每一个引出的推论。

这个问题的矩阵表必须是显示这四个人(用四行表示)和他们从事的四种艺术职业(用四列表示)的一个列阵,如下所示:

\begin{center}
\begin{tabular}{|l|l|l|l|l|}
\hline
 & 舞蹈家 & 画家 & 歌唱家 & 作家 \\
\hline
阿伦佐 &  &  &  &  \\
\hline
库 特 &  &  &  &  \\
\hline
鲁道夫 &  &  &  &  \\
\hline
威拉徳 &  &  &  &  \\
\hline
\end{tabular}
\end{center}

当我们断定名字在某行左边的人不可能是从事某列顶端所示职业的艺术家时,我们就在那个人名字的右边以那个职业做标题的列中空格写一个 N (代表"no",或写一个"一"符)。我们立即可以从前提(1)做出推断,阿伦佐和鲁道夫都不是歌唱家,所以我们在第三列(歌唱家)他们名字的右边空格写上一个 N。同样,我们可以从前提(2)推断库特既非画家也非作家,所以我们把一个 N 记在第二列(画家)和第四列(作家)他的名字右边的空格中。从前提(3)我们看出作家既非阿伦佐也非威拉德,所以我们把 $N$ 记在第四列他们名字右边的空格中。至此我们记录的所有项目都得到了原先所给信息的证明,现在的矩阵表如下:

\begin{center}
\begin{tabular}{|l|l|l|l|l|}
\hline
 & 舞蹈家 & 画家 & 歌唱家 & 作家 \\
\hline
阿伦佐 &  &  & N & N \\
\hline
库 特 &  & N &  & N \\
\hline
鲁道夫 &  &  & N &  \\
\hline
威拉德 &  &  &  & N \\
\hline
\end{tabular}
\end{center}

从已获得的信息我们可以用排除法断定,鲁道夫一定是作家,所以我们在第四列(作家)下鲁道夫名字右边的空格中记一个 Y(代表"yes",或记一个"+"符)。现在从列阵看,很明显,画家一定或是阿伦佐或是威拉德,并且我们在这里可以排除阿伦佐:鲁道夫有画家给他画的肖像(从前提(2)可知),阿伦佐从未听说过鲁道夫(从前提(4)可知),因此阿伦佐不可能是画家。这样我们记一个 N 在第二列(画家)阿伦佐名字右边的空格中。

接着我们可断定阿伦佐一定是舞蹈家,从而在第一列(舞蹈家)阿伦佐名字右边的空格记一个 Y。现在我们可以在舞蹈家列中为库特和威拉德二人分别记人一个 N 。对库特来说剩下的唯一可能是歌唱家,所以我们记一个 Y 在那个空格中,并且记一个 N 在威拉德名字右边歌唱家列的空格中。再通过排除我们断定,威拉德一定是画家,并将一个 Y 填入矩阵表的最后一个空格。完成的图示是这样的:

\begin{center}
\begin{tabular}{|c|c|c|c|c|}
\hline
 & 舞蹈家 & 画家 & 歌唱家 & 作家 \\
\hline
阿伦佐 & Y & N & N & N \\
\hline
库 特 & N & N & Y & N \\
\hline
鲁道夫 & N & N & N & Y \\
\hline
威拉德 & N & Y & N & N \\
\hline
\end{tabular}
\end{center}

从这个填满的矩阵表中我们可以得到答案:阿伦佐是舞蹈家,库特是歌唱家,鲁道夫是作家,威拉德是画家。

当要求提供几种不同范围的答案时,这种综合性质的智力游戏就变得更复杂了。有些这样的问题非常具有挑战性,并且不使用矩阵方法几乎不可能解决。

另一些推理问题提出的是一种不同的挑战。下面是一个精致、娱人但不是很困难的问题。在阅读紧随其后的答案之前请努力解决它。

你面前有六个球:两个红球、两个绿球和两个蓝球。在每一对同色球中,你知道其中一个比另一个重。你还知道所有三个重球的重量相同,所有三个轻球也一样重。另外,这六个球(把它们分别叫做 R1、R2、G1、 G2、B1 和 B2)难以区分。你只有一架天平秤盘。

问:若在秤盘上称量不能超过两次,如何能辨认出所有三对球中的重球和轻球?

\section*{答案:第一次称量: $\mathbf{R 1 + G 1 / / R 2 + B 1}$}
如果两边平衡:R1 和 R2一对红球中,一个重一个轻。因两个红球分别在秤盘的两边,我们知道如果两边平衡,那么每一边的另一个球一定也是一重一轻——因为如果两个重球在同一边,这一边就一定沉下去。因此,我们知道两者必居其一:G1 重而 B1 轻,或者 G1 轻而 B1 重。

如果在第一次称量时两边平衡,那么第二次称量:G1//B1。无论这一次称量的结果是什么,所有球的轻重都可以辨认出来。

如果(在这次称量中)G1 沉下去,那么:\\
-G1 重(且 G2 轻),且\\
-B 1 轻(且 B 2 重),且\\
$\cdots \mathrm{R} 1$ 轻(且 R2 重)。\\
如果(在这次称量中)G1 升上去:(上述结论的)反面就是真的。\\
但是,倘使在第一次称量中 $(\mathrm{R} 1+\mathrm{G} 1 / / \mathrm{R} 2+\mathrm{B} 1)$ 两边不平衡将会怎样?假设 $\mathrm{R} 1+\mathrm{G} 1$ 沉下去。(如果 $\mathrm{R} 1+\mathrm{G} 1$ 升上去,随之答案就简单地倒转过来。)

我们知道在这种情况下 R1(在这一边的红球沉下去)一定是重的;因为如果 R1 是轻的,R2 就是重的;并且如果 R2 是重的,R1+G1 就不会沉下去。

因为 R1 是重的,下面三种联合体之一一定是如此这般:\\
(a)G1 是轻的,并且 B 1 是轻的;或者\\
(b)G1 是重的,并且 B 1 是重的;或者\\
(c)G1 是重的,并且 B1 是轻的。

\section*{如果 $\mathbf{R 1 + G 1 ~}$ 在第一次称量中沉下去,第二次称量: $\mathbf{R 1 + R 2 / /}$ $\mathrm{G} 1+\mathrm{B} 1$ 。}
我们已经知道 R1 是重的。在这第二次称量中,R1+R2(重 + 轻)一定是下列两种情况之一:沉下去或升上去,或者两边平衡。无论是哪一种结果,我们都可以辨认出所有球的轻重如下:\\
(x)如果 R1+R2 沉下去,G1 和 B1 一定都是轻的(因为一个重的加一个轻的会在重量上超过仅仅两个轻的相加)。在这种情况下联合体一定是上面的(a)型:G1 是轻的且 B1 是轻的一一所有问题都被解决。\\
(y)如果 R1+R2 升上去,G1 和 B1 一定都是重的(因为重+轻在重量上只能被两个重的相加超过)。在这种情况下联合体一定是上面的(b)型:G1 是重的且 B1 是重的——所有问题都被解决。\\
(z)如果两边平衡,G1 和 B1 也一定是重+轻。在这种情况下联合体一定是上面的(c)型:G1 是重的且 B1 是轻的一一所有问题都被解决。

将这个问题推荐给你的朋友们之前,请练习说明上述答案!\\
在现实世界中,我们经常被要求从某个当前事态推出它的起因,从事

情现在是什么推出其过去是什么。科学家——特别是考古学家、地质学家、天文学家、医学家——通常面对着探究其起源的事件或条件。企图说明事情为何从过去的状况发展到现在的状况的推理叫做回溯分析。例如,令天文学家惊奇的是,1996年在地球旁边疾驰而过的彗星海库塔克(Hy- akutake)放射出比任何一位科学家曾预言的一颗彗星能够放射出的强 100倍的可变 X 射线。德国马克斯-普朗克研究所的一位彗星专家评论说: "为了研究这些数据,我们中断了我们正在进行的工作——但这是一种你喜欢拥有的问题。"

我们确实喜欢拥有这样的问题。因此,回溯分析问题经常是为娱乐而设计的。然而这样的问题也有一种特殊的困难:由科学的或历史的知识提供的现实世界的逻辑框架一定是以某种方式由问题自身规定的。一些规则或规律一定是在逻辑分析能够进行的范围内提出的。

棋盘是一种最著名的回溯分析问题的装置,而下棋规则规定了必要的 63 理论语境。下棋无须什么技术,但对(国际)象棋规则不熟悉的读者可以跳过如下例示。

象棋中的回溯问题通常采取这样的形式:棋盘中棋子的安排是给定的,对一局棋赛进行回溯分析,以比赛中遵守了所有比赛规则为前提。例如,图 1-1 代表一局棋赛所得到的形势,比赛中所有的着数都与象棋规则一致。那么,刚刚走的一着棋或几着棋是什么?\\
\includegraphics[width=\textwidth]{images/2025_05_15_6a28331d5e7c993ad07ag-085.jpg}

图 1-1

为便于分析,所有行数从下到上加标数字 1 到 8 ,所有列数从左到右加标字母 a 到 h。那么棋盘中每一个方格都能用一个唯一的字母一数字结合体表示:黑王在 $a 8$ ,白兵在 $h 2$ ,等等。问题是:上一步由黑棋走的那着棋是什么?那么黑棋的前一步白棋的着数是什么?你能在阅读下一段之前推出答案吗?

答案:刚走的一着棋是黑王移动。因为两个王永远不能走在邻近的方格里,黑王不可能刚从 b7 或 b8 走到现在的位置上;因此我们可以确定黑王刚从 a 7 走到现在的位置,在 a 7 处黑王被将军。

这是非常容易推断的。但是前一着白棋是什么才能使黑王处于被将军的局面呢?那着棋不可能是白象(在 g1),因为白象没有一条路径能走到 g 1 格,不可能在白象走棋时黑王正处于被将军的局面!因此一定是,黑王被将军的局面是由另一个白棋子的移动造成的,这个白棋子正阻挡着象的攻击,并被走到 $a 8$ 的黑王吃掉。什么白棋子能在黑对角线上并且从那儿走到角上的白格中呢?只有在 $b 6$ 的马。所以我们可以确定,在黑棋最后一着(黑王从 a 7 到 a 8 )之前,白棋最后一着是从 $b 6$ 到 a 8 的马。

当然,现实生活中我们所面对的推理问题很少像本节所讨论的谜题这样整洁。许多现实问题的叙述不是很精确的,对它们的错误描述易于使人误解,从而不能得到答案。遇到这种情况,原问题的部分陈述就要加以拒绝或替换。而当我们试图解答本章给出的这种逻辑谜题时,我们是不能这么做的。

此外,现实世界中的一些问题,甚至当它们被准确描述时,也可能是不完善的,其中某个最初不是可供利用的条件,可能对于问题的解决是必不可少的。现实世界中一些问题的答案可能依赖于某个新的科学发现,或某个以前不可想象的发明或装置,或对某个至今未加探索的领域的研究。但是在逻辑谜题的陈述中,如同一部好的谋杀案侦探小说一样,必须给出足以得到答案的全部信息;否则我们就会认为侦探小说作家,或问题的设计者对我们是不公平的。

最后,逻辑谜题提出的问题都是清楚明确的(诸如:四个艺术家中哪一位是歌唱家?黑棋和白棋的最后一着是什么?等等),给出其答案并加以证明,就明确解决了逻辑谜题提出的问题。但那不是许多现实世界中的问题所呈现的形式。现实问题最初经常是仅仅由于某种前后矛盾的情形或一个不平常的事件的出现而被发现的,甚或只是基于人们对某种事情之不

顺畅的感觉而发现的——现实问题不是有着明确答案的精心构造的问题。\\
不管有多少区别,现实世界中的问题与精心设计的逻辑问题一样,都必须通过系统的推理才能得到解决,二者在逻辑学研究中都具有重要的作用。

\section*{第1章概要}
本章引进并举例解释逻辑学最基本的概念。

1.1节将逻辑学定义为研究用于区分正确推理与不正确推理的方法和原理的学问,并对这个定义作了阐释。\\
1.2 节阐释命题概念——命题是可以被肯定或否定,并且或真或假的东西——并且对命题与表示命题的语句作了区分。\\
1.3 节引人并阐释论证概念——串命题,其中之一是结论,另一个 (或一些)是用以支持结论的前提。\\
1.4 节说明并且例示分析论证的方法——一种是解析法,按照逻辑的顺序完整地列出论证中所有的命题;另一种是图示法,所有命题都用数字标示,这些数字以一定的方式相互连接以展现命题之间的逻辑关系。\\
1.5 节讨论辨识论证的几个方面的问题,包括结论指示词和前提指示词、语境在辨识前提和结论中的作用、有可能充当前提的非陈述形式以及包含未明确陈述出来的命题的论证等。

1. 6 节讨论论证和说明之间的区别,解释为什么做出这种区分常常是困难的,这种区分依赖于语段的语境和作者的表达意图。\\
1.7 节讨论演绎和有效性,将演绎论证定义为断言其结论从前提必然地得出的论证,一个有效的演绎论证就是一个假如其前提为真则结论必然为真的论证。\\
1.8 节讨论归纳和或然性,将归纳论证定义为其结论具有某种或然性程度,但并非(从前提)必然地得出的论证,说明归纳论证可以被判定好与坏,但不能刻画为有效与无效。\\
1.9 节讨论演绎论证的有效(或无效)与命题的真(或假)之间的某些复杂关系。

1. 10 节讨论复杂的论证语段,表明如何使用图示技术对它们进行分析。\\
1.11 节讨论推理问题,展示了一些能够训练与强化推理技术的方式,以及它们所提供的纯粹的智力快乐。

\printbibliography[heading=subbibliography,title={第1章参考文献}]

% others

% 第二部分
\chapter{命题逻辑}
% 第二部分导言
\begin{logicbox}[title=第二部分:语言与交流]
\textit{语言是人类思维和交流的重要工具,理解语言的功能和特点对于逻辑分析至关重要。本部分探讨语言的多种功能、情感色彩以及在论证中的作用,帮助读者更好地理解和运用语言进行有效交流。}
\end{logicbox}

\input{chapter2/section2-1}
\
\\section*{2.2 多功能话语}
前面所给出的信息性的、表达性的和指令性的话语的例子,打个比方说,都像纯正的化学标本。语言交流的这种三重划分是启发性的,有价值的,但是不能机械运用,因为几乎任何一种正常交流都可能会表现出语言的这三种用法。例如,一首诗可能主要是表达性话语,但也可能会有教育意义并因而也可以引导读者走向不同的生活方式。华兹华斯(Words- worth)写道:

我们身边的世界丰富精彩:迟早,无论是得到和失去,我们都要浪费掉自己的精力:

显然,诗歌也可以包含一定数量的信息。\\\\
再如,尽管布道主要是指令性的,希冀在会众中带来一定令人称赞的行动(抛弃罪恶,乐善好施),但是它也可以表达并激发情感,因而具有表达性功能,而且还可以包括一些信息,比如福音(the Gospels)的好消息。科学论文虽然本质上是信息性的,但是也可以表达作者的理性激情,而且还可以至少是含蓄地请读者去独立地证明作者的结论。语言的大多数平常的用法都是混合的。

语言的这种混合的或多功能的用法,并不是因为说话者或作者混淆了它们。相反,成功的交流都要求一定的功能结合。除清晰的语境和正式的关系——一父母与子女,雇主与雇员——之外,人们不能简单地发布命令并希望它得到执行。赤裸裸的命令会引起反感和敌对,并且经常是自生自灭。因此,必须使用一定的间接方式。通常,为了追求我们要引起的行动,我们并不直截了当地发布命令,而需要使用比较委婉的方法。

行动常常具有非常复杂的原因。与逻辑学家相比,心理学家更为适合研究动机,但是行动通常既涉及行动者的欲望又涉及他的信念,这是常识。除非饥饿的人相信他们面前的东西是食物,否则,他们就不会把它放进自己的嘴里;除非想吃,否则,即使人们毫不怀疑面前就是食物,他们也不会去碰它。

欲望是我们所谓的"态度"或"情感"的特殊类型,而信念通常会受到所接收到的信息的影响。因此,我们有时是通过激起他人的适当态度而成功地引起他们的行动,而有时则是通过提供信息以影响他们的相关信念而做到这一点。

假设你的目标是使你的听众向某个特定的慈善组织提供捐助,假定你的听众态度是助人为乐的,你就可以通过给他们提供该慈善机构的良好工作信息而促使他们行动。你的语言是指令性的,目的是引起行动,但你是通过提供信息,而不是通过发布一条毫无掩饰的命令或不客气的要求,而达到你的诉求的。再如,假定你的听众已经被深深地说服,相信我们谈论的那个慈善机构确实信誉良好,但对捐献的鲁莽要求仍然可能失败;但是,通过在一定程度上激发他们的乐善好施情感,你就可能会成功地使他们向该慈善机构提供捐助。在这种情况下,你就通过使用表达性的话语而

达到了自己的目的:你实现了一个"动人诉求"(moving appeal)。这样,你的语言自然而然地就具有了混合用法,既有表达性功能又有指令性功能。

最后,再假设你要向那些既缺乏乐善好施的态度,也缺乏对你推荐的慈善组织之信誉的认识的人进行捐献动员,那么你就必须一并使用语言的三种功能,既要有表达与信息功能,又旨在引发行动。这种一并使用并不是可偶尔为之的用法,而是必须为成功交流而精心准备的基本手段。

语言的三种基本用法是:信息性用法、表达性用法和指令性用法。但值得一提的是,语言在某些特殊语境中还具有一些特殊用法,而这些用法并不能完好地归属这三重划分。

语言的礼仪性用法是很普遍的,在有些场合下它是一种表达性的和指令性的话语的混合。社交中的问候、赴宴邀请、雇用告知等表述,都是体现礼仪功能的例子;语言还有很多其他相关的不确定用法,它们主要服务于使人们之间的互动变得融洽。礼仪功能还是宗教场合中一种庄重的语言用法。给人留下深刻印象的结婚仪式的语言,既要突出场合的庄重性(表达性的功能),又要使新郎和新娘提高对严肃的结婚誓言的正确理解而引起新的角色行为(指令性的功能)。

与礼仪性用法相近,语言还有其他一些用法,它们也不仅仅是那些主要用法的混合。假如有人邀请你在某一时间和地点去参加会议,你回答 "好,我答应你",那么你就不仅是以此表明了你的态度和预告了你的行为,你同时还用语言来许诺。相似的,在婚礼结束时,司仪或主持人说 "我宣布你们是夫妻",这也不仅是表明了说话者在干什么。在有些语境中,说出某些话实际上包含了一种重要行动。这些都是语言的践行性 (performative)用法的例子。

践行性话语实际上就是实施一种行动,即其所报告或描述的行动。践行性动词是一种特殊的种类,它们代表行动,这种行动通常是以第一人称使用动词而(在适当的情况下)完成的。这里可以再举出一些例子,如 "我祝贺你……"、"我向你道歉,我……"、"我建议……"、"我将这艘船命名为 $\\cdots \\cdots$"、"我接受你建议......"等等。

语词和语句的这些及其他的特殊用法显示了自然语言的丰富性,它们的诸多复杂功能难以归纳为任何一个单独的分类系统。 
\input{chapter2/section2-3}
\section{情感词汇}

\begin{quotation}
\textit{词汇不仅具有字面意义,还蕴含着情感色彩。了解词汇的这种双重性对于正确解读语言表达和分析语言使用十分重要。}
\end{quotation}

现在,我们从讨论语句和较复杂的语段转向探讨构成它们的词汇。正如 2.2 节所见,一个单独的语句,可以同时具有信息性的和表达性的用法。要具有前者,句子必须明确表述一个命题,而要做到这一点,它的词汇必须具有字面的或描述性的意义,以指示客体或事件以及它们的性质或关系。而当句子表达态度或感情时,其词汇就会具有情感的暗示或影响。一个语词或短语可以既具有字面意义又具有情感影响。后者通常被称为词汇的\textbf{情感意义}。

\subsection{字面意义与情感意义}

词汇的\textbf{字面意义}和\textbf{情感意义}在很大程度上是各自独立的。例如,"官僚"(bureaucrat)、"政府官员"(government official)与"公仆"(public servant)的字面意义几乎一样,但它们的情感意义却很有区别。"官僚"倾向于表达厌恶和反对,而作为敬语的"公仆"则倾向于表达尊重和赞赏。"政府官员"则更接近中性。

显然,我们用以指示事物的词汇会明显地影响我们对事物的态度。花的实际芳香不会因其名称而改变。正如莎士比亚所写的那样,一朵玫瑰不论使用其他任何名字,闻起来都是香甜的。然而,假如有人告诉我们有一种称做"臭菘"(skunkweed)的玫瑰,我们对它的反应就可能会受到影响。在华尔街(Wall Street),谨慎地选择语言可以促使人们在股票市场上采取行动。某几天是"回升",它意味着价格上涨;另几天是"取短期利息",这意味着价格在下降,因为很多人都在抛售股票,但这个词语仍然好听。如今大公司极少再进行"破产",但它们可以"重组",这听起来要好多了。

\subsection{委婉语的应用}

这种态度上的影响可以说明\textbf{委婉语}增多的现象,委婉语就是用温和的词汇表示严酷的现实。在战争中,己方军队的失败可能被称做容易接受的 "暂时撤退",而重大的撤退可能被报道为"兵力的有序集结"。在越南战争中,正在竞选总统提名的参议员尤金-麦卡锡(Eugene McCarthy)曾对美国的军事干预政策及公众不愿坦诚地面对它的状况进行了颇具讽刺性的批评:"我们不再宣称战争",他说,"我们宣称国家防御。"$^{[6]}$

我们在不断地创造新词汇以替代那些不再令人满意的旧词汇。"殡葬人员"(undertake)变成了"殡仪员"(mortician),"看门者"(janitors)变为"守护员"(maintenance men),"老人"(old people)变成了"年长公民"(senior citizens)。但是,与旧的实际情况相联系的新替换的词汇最终也会失去它们的吸引力;"守护员"结果被"保卫者"(custodian)所代替,"殡仪员"为"殡仪主管"(funeral director)所取代,等等。杰梅恩•格瑞尔(Germaine Greer)写道:

\begin{displayquote}
经过与其指代的实际相联系,委婉语便迅速地失去了它们的功能,这是它们的宿命。因此,它们必须经常被它们自己的委婉语所取代。 $^{[7]}$
\end{displayquote}

据说,杜鲁门总统的夫人贝丝(Bess)的朋友请求她阻止杜鲁门再说"大粪"(manure),她回答说,她花费了四十年时间才使他开始说"大粪"。

\subsection{情感意义的力量}

语言的确有它自己的生命,独立于它用以描绘的事实。有些包含生殖和排泄的生理活动可以用医学词汇不带情感地进行描述,而不至于引起神经质的不快;但是使用下流粗俗的词汇描述同样的活动却可以震惊除最麻木不仁者之外的所有听众。用我们的术语来说,可以说这两种词汇具有相同的字面或描述意义,但是在它们的情感意义上却有或缓和或激烈的对立。

在某个具体的人的思想中,词或短语有时可以产生情感意义,那不是产生于其字面上指代的东西,而是产生于第一次学习或遇到它的语境。一个作者曾描述说:

\begin{displayquote}
这是一个有启发意义的故事。一个小姑娘近来学会了阅读,正在拼读报纸上的一篇政治性文章。"爸爸,"她问,"什么是坦曼尼协会(Tammany Hall)?"她爸爸用通常社交禁语的口吻回答说:"亲爱的,你长大了就会明白的。"按照这种奇怪的成年人遁词,她终止了询问;但在她父亲的语气中,有某些东西使她相信坦曼尼协会必定与不正当的性事有关,以至于多年来她只要一听到这个政治机构$^{(1)}$就会体验到一种神秘的非政治性震颤。$^{[8]}$
\end{displayquote}

对于很多人来说,一定的词汇或短语,由于与我们的生活具有某些特殊联系,可以携带某种我们或许不愿公开承认的隐私情感。

\footnotetext{(1)坦曼尼协会最初是旨在通过捐赠与赞助进行控制的纽约市民主党执行委员会,成立于 1789 年。1805年转型为有明确政治宗旨的"慈着机构"。}

\subsection{情感意义的创造性应用}

字面意义和情感意义之间的对比,以及它们不同的可操作性用法,促使哲学家伯特兰•罗素设计了一种寓教于乐的游戏。他这样"调配"出一类"不规则动词"(irregular verb):

我坚定;你倔强;他头脑呆板。

伦敦的《新政治家和国家》(New Statesman and Nation)杂志随后举行了一次竞赛,以征求这样的不规则词汇表,获胜者如下:

我义愤;你生气;\\
他破口大骂。

我重新考虑过了;你改变了想法;\\
他食言了。

这种游戏证实了普通经验的教益:相同事物可以被情感色彩非常不同的词汇所指称。 

\begin{center}
\fbox{\parbox{0.9\textwidth}{
  \centering
  \textbf{情感词汇的特点}\\
  词汇同时具有字面意义和情感意义,二者在很大程度上相互独立;\\
  委婉语是通过替换词汇来改变情感反应的常见手段;\\
  同一事物可以被情感色彩截然不同的词汇表达,从而引导人们形成不同态度。
}}
\end{center} 
\section*{2.5 一致与歧见的种类}
任何事物或行动都可以选择不同的短语来描述:传达赞许或反对意见,或者中立意见;不同类型的一致(agreement)和歧见(disagree- ment)可以就任何事情展开交流。

两个人可能会在某事情是否已经实际地发生上意见相左,这种情况可称为"信念歧见"。另一方面,他们也可能都同意已经事实上发生了一件事,因而是信念一致的;但对那件事,他们仍可以具有不同的或者甚至相反的态度。你可以用语言描述那件事来表达赞许,别人却可以用语言来表达反对。这里也存在歧见,但不是信念歧见。这是对这件事的感受不同,

是态度歧见。 ${ }^{[9]}$\\
明确了这两种歧见,我们就可以区分出两个人(让我们把他们称做 A和 B)之间的四种关系,来讨论某些给定事件或者其他事实情况。

第一,他们可能达到充分一致,即他们对于事件发生的信念和对事件的态度都是一致的。

第二,他们可能对于事件的信念是一致的,但在态度上却对立,一个人认为是好事,而另一个人却认为是坏事。设想我们讨论的事件是:对于某个有争议问题,一位政治候选人的立场改变。对于的确发生了这个改变,A 和 B 可能意见一致;但是,A 认为好极了,而 B 则发现它令人担忧。如果像 A 所理解的那样,这位候选人就会被称赞为"倾听了理性的呼声";如果像 B 所理解的那样,这位候选人就将被谴责为"机会主义的反复无常"。

第三,他们可能态度一致,但对于引起态度的事实,他们却可能有信念歧见。因此,A 和 B 都可能热情地赞扬所论及的那位候选人,而他们对该候选人的实际立场却有不同理解。A 可能认为,由于"倾听了理性的呼声",那位候选人确实改变了他的立场;而 B 则可能认为,由于"坚定不移地拒斥了为奉承所左右",他根本没有改变自己的立场。乍一看,这个第三种可能性好像不合理,但经过思考之后就会被认为是平常的;我们知道,在政治选举中,同一候选人常常可能有不同的支持原因,这些原因不但不同而且有时还不相容。

第四,这两个人可能会处于一种完全对立的状态,他们不但在事实上有歧见而且对事实的态度也对立。由于相信那位候选人改变了立场,A 可能会非常热烈地称赞这种改变是"明智的重新考虑"的结果;由于认为候选人的立场保持未变, B 可能会激烈地贬斥他"顽固地拒绝承认错误"。

当我们以解决歧见为目标时,我们就必须既要关心给定情况下的事实,又要关心争论者对这些事实的不同态度。不同类型的歧见需要不同的解决方法。因此,如果我们不清楚所存在的歧见是什么类型,那么我们就不清楚去使用什么方法。信念歧见可以通过确认事实而得到最好的解决。为了明确这些事实,如果它足够重要,可以询问证人、查阅文本和检查记录等等。当事实得到了确证、解决了事实问题时,歧见就会得到解决。科学的探究方法在这里都可以用到,这将足以指导他们直面有关信念歧见的事实问题。

另一方面,如果是态度歧见而不是信念歧见,那么适于解决它的方法就有相当大的差别,难以如此径情直遂。以确证事件发生与否为目的,召唤证人、查阅文本,诸如此类,对解决这样的争论不会有什么效果,因为争论的问题并不是事实,这种歧见不是关于事实是什么而是关于怎样评价它们的对立。解决这种态度歧见的努力可能会涉及有关事实问题,但不是那种存在态度冲突的事实。或许考虑那种引出愉快或不愉快的结论的事情若不发生将会怎样,可能是有益的。动机和目的也可能具有重要性。诚然,它们都是事实问题,但如果歧见在信念上而不是在态度上,它们就都不会成为争论的主题。其他一些方法有时也可以解决态度歧见,你可以大量地使用表达性语言来尝试说服方法;在凝结团体意志和取得统一态度中,修辞艺术也可能富有成效(当然,它在解决事实问题上完全没有价值)。\\
"好"和"坏","对"和"错",诸如此类的语词,在严格的伦理用法中,往往具有非常强烈的情感色彩。无疑,当我们把某行动描述为对或把某情形描述为好时,我们就对它表达了一种赞赏态度。有些伦理学者认为,这些语词"没有"词汇意义或认知意义,而只有适合它们的情感意义;而另一些伦理学者则坚持这些语词的确具有认知意义,它们指称所讨论事物的客观特征。在这种争论中,学习逻辑的学生不必偏向某一方。但显然,不是所有的赞同或不赞同态度都蕴涵道德判断,因为还有审美价值的考虑,还有个人偏好或口味的影响。对某些事物(例如有些食品或服装款式)的否定态度,不必牵涉伦理或审美判断,但它也可以由情感色彩强烈的语言来表达。

当歧见是在态度上而不是在信念上时,最强烈的(当然也是实质的)歧见可以用朴实无华的真实陈述来表达。当双方针锋相对并以逻辑上相容的陈述来明确表述他们的分歧时,若认为双方的歧见不是"实质的"或者是"纯粹言辞的"就是错误的。他们并不是仅仅"用不同的言辞说相同的事物"。当然,他们可以用不同言辞来断言词汇意义上相同的事实,但是,他们也可以用不同言辞来表达对相同事实的矛盾态度。在这样的情况下,虽然他们的歧见不是"词汇意义上的",然而却是实质的。这不是"纯粹言辞的"歧见,因为语词既具有表达性功能也具有信息性功能。如果我们有兴趣解决歧见,我们就必须弄清楚其本性,因为适于解决一种歧见的技巧,正如我们已看到的,可能对另一种毫无用处。

有时,我们难以确定一种歧见是信念的或是态度的,或者既是信念的又是态度的;这可能取决于争论者对词汇的某种解说。冲突意见的表达方式可能会把不同态度之间的区别以及不同信念之间的区别弄得模糊,因此争论的关键核心就会难以把握。当两个人对在一个事物是否比另一个"更好"或"更重要"上意见对立时,他们都可能认为,真实情况很可能是不同信念使他们产生了分歧。但是在某些情况下,一个其表面形式是关于所谓的事实问题的差别论争,实际上是一个关于态度的实质争论,当争论的东西是事物或行为的价值时尤其如此。

在关于赢球的重要性上,一位著名体育运动作家和一位著名足球教练产生了深刻的分歧。新闻记者格兰特兰德•赖斯(Grantland Rice)写道:

\begin{displayquote}
当球星走进球场,为其声名留下光芒,这光芒不论输与赢,只记录场上飞奔的身影。
\end{displayquote}

文斯•隆巴蒂(Vince Lombardi)教练却说:

\begin{displayquote}
赢球不是别的,它就是唯一(竞赛目标)。
\end{displayquote}

很明显,这两位对赢球的态度是冲突的。你相信这种态度上的歧见的根源是信念歧见吗?

尽管有这些不可避免的困难,态度歧见与信念歧见之间的区分还是非常有用的;留心语言的不同用法有助于理解我们可能遭遇的种种歧见。当然,找出区别本身并不能解决问题或消除歧见。但是,由此可以澄清讨论的问题,揭示出它们的类型并找出冲突的所在。我们越充分地理解歧见的本性,我们就越能够更好地去解决歧见。 
\input{chapter2/section2-6}

% 第三部分
\chapter{论证分析}
\section*{3.1 论争、言辞之争与定义}
语言是一种非常复杂的设施,是人类最重要的交流工具。但是,当语词被漫不经心或错误地使用时,这种设施就会变成我们的负担。在 2.5 节中,我们阐释了冲突双方的歧见既可能是信念的也可能是态度的;而我们看到,两者之中任何一种对立都可能是实质歧见。但是,也存在这样的情况:表面上的歧见实际上却不是真正的歧见,而仅仅是误解或词汇误用的结果。在前一章中,我们考察了不同类型的实质歧见;在这里,我们转而讨论不同类型的论争,分辨其是否具有真正的分歧。

必须区分出三种不同的论争。第一种是明显的实质论争,在这种论争中,各方或者在信念上或者在态度上,明确地毫不含糊地对立。例如,如果美国佬(Yankees)赢得了世界联赛,虽然在胜利者本身的认同上没有争论,但如果 A 为此高兴而 B 为此恼怒,那么他们的态度歧见(不可能解决的问题)就是显然的,或许甚至是激烈的。在另一种语境中,如果 A坚持认为巴拿马运河的太平洋人口比其大西洋人口更靠东,而 B 则否定如此,那么他们的争论就不是在态度上而是在事实上;而一张好地图就可以平息这个论争。 ${ }^{[1]}$ 无论是态度上的还是信念上的,这种论争总是包含某种实质歧见。将论争双方区别开来的不仅仅是语言,在他们对事实的断定上或对事实的评价上亦存在实质差别。因此,这样的实质论争不能通过定义或任何简单的语言调整来解决。

当然,可以存在关于词汇本身的实质论争,例如,某个单词怎样拼写或者怎么使用;也可以存在关于态度本身的实质论争,例如,是否某第三方不友好或仅是差怯。事实可以是物理的或地理的,也可以是语言的或心理的,而且各方可以在任何种类事实上产生歧见。但是,如果论争确实是关于某个事实的,那么它就是实质的,并可以通过确认某些事实而得到解决。

然而,也存在第二种类型的论争,即纯粹的言辞之争;在这种论争中,双方之间根本没有实质歧见,然而却好像是具有歧见。语言的误解或误用可能是这里的症结所在。当论争者的信念表达中某个关键语词有歧义,而这种歧义又遮蔽了双方并没有实质对立的情况时,言辞之争就会产生。若争论某方误用一个重要词语,或者争论的某个核心语词或短语具有

不同含义,而这些含义可能同等合法但产生了不应有的混淆,或者由于各方对语词或短语的用法都对但含义不同,而这一点又没有被明确认知,就可能产生这种表面的言辞之争。

言辞之争并非总是容易发现,但一旦识别了它,通过具体化歧义语词或短语的不同含义,就可以相当容易地获得解决。在这种语境中,良好的定义对相互理解是非常关键的。

威廉•詹姆士给出了这种言辞之争的一个经典例子:

几年前,我随野营队一起在山上露营。当我独自散步返回时,发现每个人都参加了一场激烈的形而上学论战。争论主题是一只松鼠。设想一只松鼠抓附在树干一侧,而一个人站在树的另一侧;那人绕树迅速转动以试图看到松鼠,但无论他转多么快,松鼠在相对的方向都以同样快的速度转动,在它自己和那人之间总隔着那棵树,因此使他看不到松鼠。作为结果的形而上学问题是:这个人是否绕着松鼠走了一圈?确确实实,他绕树走了一圈,而且松鼠就在树上;但是,他绕松鼠走了一圈吗?原野中的讨论持续良久,直到变得乏味。每个人都赞同一种观点并固执己见,并且双方的人数势均力敌。因此,当我出现时,每方都希望我加入以便成为多数派。 ${ }^{[2]}$

显然不难看出,这场论争的双方之间不存在实质歧见,这正是詹姆士讲这个故事所要说明的。所有论争者对松鼠和树的态度都是中立的,都完全理解和赞同给定事例的所有事实。因此,在这个事例中(许多其他事例中也同样),争论不过是言辞之争。詹姆士继续写道:有绕它走一圈,因为松鼠做了相对运动,它始终保持着将其腹部

对着那个人,而将背部朝着外面。做出这种区分,就没有什么可争论的了。你们都又对又不对,就看你们对"绕走一圈' 这个动词实际上是怎么理解的。"[3]

解决这个论争不要求新的事实,并且那样做也不可能有帮助;它需要的仅仅是詹姆士所提供的东西:对争论中一个关键语词的不同意义做出区分。使用"绕走…圈"这个词语的不同定义,这个争论就消失了;故这种歧见根本不是实质的。无论何处,如果论争纯粹是言辞之争,我们就可以通过提供能够消除关键歧义的定义来解决。在这种情况下,我们可表明论争各方并不是真正的相互对立;它们可能仅仅是运用相同的词汇的不同含义或意义来维护不同主张或者运用不同语词来维护相同主张罢了。一旦确定了不同意义以及源于对不同意义的使用而涉及意义的不同主张,那么双方之间就不会再有什么论争。 ${ }^{[4]}$

第二种论争,指那些表面上是言辞的但实际上是实质的论争。当双方互相误解了对㡰词语的用法时会出现混淆,而这种混淆可以得到识别。但是,有时也会出现这样的争执远超出语词不同用法的范围。在这种情况下,仅仅解决歧义问题不会平息论争,因为争论双方之间还存在某种实质歧见:可能在信念上,更可能在态度上。

举例来说:对给定的有露骨性活动镜头的影片是否应该作为"色情作品"来处理,双方可能产生争执。一方坚持认为,它的露骨使它成了邪恶的色情作品;另一方则坚持,考虑到其细腻的情感和美学价值,它是真正的艺术,根本不是什么色情作品。显然,双方的歧见在于"色情作品"一词的意义;但是,即使言语的不同得到了充分理解并清除了所有歧义,双方很可能对影片仍然存在实质歧见;论争实际上不是真正关于"色情作㗊"这个词的适用性的,他们的歧见更深人地涉及影片的性感露骨性质是否造成影片的好与坏。

第三种论争有时被称为"标准"之争或"概念"之争。论争双方对某个关键词语的运用有着不同标准,也就是说,该词语指谓的是不同概念。因而,在不同标准的明智或正确性之下,各方就处于尖锐冲突之中。比如在上例中,即使双方都认识到他们有歧义地使用了"色情作品"这个词,甚至词语的歧义已经得到阐明和区分,但各方都仍可能声称其对手误用了他们的标准来确定什么是色情作品。一方可能坚决主张,如果影片包含露

骨的性活动场景,便可以将它划归为色情作品;而对手则可能回应,那种划归是一种概念错误。这种论争表面上仅仅是言辞之争,但在表面之下,却是非常实质的论争。

为帮助人辨识和理解论争的这些不同种类,我们可以做一个有用的 "流程"。一旦我们确定存在某种论争,我们就可以问:"出现歧义了吗?"如果对该问题回答"没有",那么我们得到的论争就是类型—1(显然是实质的)。如果回答"出现了",那么我们就问第二个问题:"清除歧义可以消除对立吗?"如果对此问题回答"可以",那么我们得到的论争就是类型——2(纯粹言辞之争)。如果对第二个问题回答"没有",那么我们得到的论争就是类型——3(表面上是言辞的但实际上是实质的)。

这三种论争可以概述如下:\\
1.在明显的实质论争中,不存在言辞歧义,争论双方的确有歧见,或是在态度上或是在信念上的歧见。

2.在纯粹言辞之争中,存在言辞歧义但根本没有实质歧见。\\
3.在表面上是言辞的但实际上是实质的论争中,既存在言辞歧义又有歧见论争,或在态度上或在信念上,或者是关于事实的或者是关于某些语词的运用标准的。 
\section{定义的类型和论争的解决}

\begin{quotation}
\textit{定义是解决语言争端的重要工具,理解不同类型的定义及其应用场景,有助于我们更准确地表达思想,避免不必要的争论。}
\end{quotation}

在对论争的各种类型和起因进行讨论之后,我们可以考虑定义的类型和它们在解决论争中的作用。定义是对词项意义的解说;因此,定义就旨在减少或消除由词项意义的不确定性或模糊性引起的困难。定义既可以阐释一个既有词项的既有意义,也可以赋予一个新词项以新意义。当目的在于前者时,定义就报告一种既定的用法;当目的在于后者时,定义就规定一种用法。当然,报告的意义可以是准确的也可以是不准确的;规定的意义可以是有用的也可以是无用的,可以是方便的也可以是不方便的。然而,在真或假与有用或无用之间存在根本区别。任何\textbf{报告性定义}(即词典定义)都可以是真的或假的,而任何\textbf{规定性定义}却都不能是真的或假的。

\subsection{报告性定义与规定性定义}

例如,"奇数"(odd number)可以定义为"任何不能被 2 整除的整数"。这是一个报告性定义,因为它报道了"奇数"这个词的既定用法,它是真的。或者,人们也可以提出"圆方形"(circle-square)这个新词,并把它定义为"既是圆又是正方形的图形",这是规定性定义,由于不存在圆方形,它在应用上就是无用的;由于它结合了两个不相容的属性,它就是逻辑上不可能的;但是,它既非真也非假。这个定义是"规定"这个新词的意义,而不是"报道"这个新词的既有意义。

\subsection{精确定义的作用}

当我们面对歧义时,词项的既定意义并不总是很清楚的。在这种情况下,仅仅报道词项的既有意义是不够的。例如,考虑短语"安乐死",它的定义可以为"有意致病人于死地,而又没有该病人同意的医疗行为"。这是对这个词的一种常见用法的报道性定义。但是,我们可能想要一个更精确的定义,一个可以澄清这个词项的模糊性的定义。为了达到这个目的,我们可以提出这样一个\textbf{精确定义}:"安乐死就是,经过病人的自由和知情的同意,由医生有意地导致的一个不能忍受持久痛苦的病人无痛苦的死亡"。在这种情况下,报告性定义的作用就是建立一种标准,精确定义则使这种标准更为精确。

对一个模糊词项给出一个精确定义,与给一个新词项赋予一个意义的规定定义,这两者之间是有区别的。精确定义并不是武断的,因为精确定义所定义的是一个已经具有固定用法的词项,而精确定义必须尽可能地保持这种固定用法。因此,在提出一个精确定义时,我们不仅要考虑使词项摆脱模糊性,也要考虑与现有用法的融贯性。一个词项的精确定义仍然可以被评价为真的或假的,但这要根据它是否符合现有用法而定。

\subsection{规定定义的应用场景}

然而,这种真或假的问题,对于一个规定定义来说是完全无关的。规定定义通常是当一个新词项引进时,或者当一个既有词项要在一个新语境中使用时所必需的。例如,物理学家引进了"焦耳"这个新词来表示一个功或能的单位,并规定性地将它定义为" 1 牛顿的力使物体在力的方向上移动 1 米所做的功"。或者,当一个计算机科学家把"硬盘"这个短语用于一种新型存储装置时,她对这个短语的使用就是规定性的。规定定义不能是真的或假的,因为它们定义的是一些此前没有定义的词项。

在规定性地定义一个词项时,我们通常会受到某些动机的引导。通常,规定定义是为了方便,或者是因为有些新事物需要命名。例如,"因特网"这个词就是一个相对较新的词,它被规定性地定义为"一个全球性的计算机网络系统,它使用 TCP/IP 协议族来连接全世界数以百万计的计算机"。在这种情况下,规定定义是为了用一个简短的词来指称一个复杂的概念。

\subsection{理论定义与说服定义}

然而,有时规定定义是出于理论的目的。在这种情况下,定义就旨在阐明一个理论概念。例如,在经济学中,"效用"这个词可以规定性地定义为"一种衡量消费者从消费一种商品或服务中获得的满足程度的指标"。这种定义是为了在一个经济理论的框架内阐明"效用"这个概念。这种\textbf{理论定义}也是规定性的,因为它们赋予一个词项一种新的、更专门的意义。

最后,有时定义是出于说服的目的。在这种情况下,定义的目的是影响人们的态度或行为。例如,政治家可能会将"爱国主义"定义为"对自己国家盲目的、不加批判的忠诚",以此来贬低那些持有不同政见的人。或者,广告商可能会将某种产品定义为"成功的象征",以此来吸引消费者购买。这种\textbf{说服定义}也是规定性的,因为它们赋予一个词项一种带有情感色彩的意义。

\subsection{定义在解决论争中的应用}

在解决论争时,识别和理解不同类型的定义是至关重要的。如果一个论争是由于对一个词项的意义存在分歧而引起的,那么通过提供一个清晰的、双方都能接受的定义,就可以解决这个论争。如果论争涉及到词项的模糊性,那么一个精确定义可以帮助澄清问题。如果论争涉及到新的概念或理论,那么规定定义或理论定义可以帮助阐明相关的思想。然而,如果论争涉及到说服定义,那么就需要警惕定义中可能包含的情感偏见。

总之,定义是解决论争和促进清晰思考的重要工具。通过仔细地考察和运用不同类型的定义,我们可以更有效地进行交流和推理。

\begin{center}
\fbox{\parbox{0.9\textwidth}{
  \centering
  \textbf{定义的类型及其应用}\\
  报告性定义:阐释既有词项的既有意义,可评价为真或假;\\
  精确定义:消除既有词项的模糊性,需兼顾与现有用法的融贯性;\\
  规定性定义:赋予新词项以意义或赋予既有词项以新意义,不能评价为真或假;\\
  理论定义:为阐明理论概念而提出的规定定义;\\
  说服定义:为影响态度或行为而提出的带有情感色彩的定义。
}}
\end{center} 
\section{外延和内涵}

\begin{quotation}
\textit{在理解词项意义时,外延与内涵是两个不可或缺的概念。正确把握二者的关系有助于我们更精确地定义概念、避免语义混淆,从而提高思维和论证的准确性。}
\end{quotation}

定义旨在表明一个词项的意义(meaning),但是意义这个词却有不同含义(sense)。我们前面已区分了词项的描述或字面意义与表达性意义,现在,我们要更仔细地考察字面意义,尤其是普遍词项的字面意义。普遍词项就是可以运用于多于一个对象的类(class)的词项。在推理中,普遍词项的定义是特别重要的。

\subsection{外延意义与内涵意义}

普遍词项"行星"对水星、金星、地球、火星和土星等都是在同等含义上适用的。在一种含义上,词项"行星"意谓所有这些不同对象,而所有行星的汇集(collection)就构成"行星"的意义。如果我说所有行星都有椭圆轨道,那么我所断定的部分东西是火星有椭圆轨道,另一部分是金星有椭圆轨道,等等。在这种重要含义下,词项"行星"的意义便是由它适用的那些对象而构成的。"意义"的这种含义被称做词项的\textbf{外延意义}。通常,人们认为,普遍词项或曰类词项指谓(denote)其可以正确适用的那些对象。一个普遍词项可以正确适用的对象的汇集构成那个词项的\textbf{外延}。

理解普遍词项的意义就是知道怎样正确使用它;但是,这样做并不是一定要知道它可以正确适用的所有对象。对一个给定词项,其外延内的所有对象具有某些共同的性质或属性,这些性质或属性可以引导我们使用同一词项来指谓它们。因此,我们可以知道一个词项的意义而无须知道其外延。在第二种含义上,"意义"设定了决定任一对象是否属于那个词项外延的某种标准。"意义"的这种含义被称做词项的\textbf{内涵意义}。普遍词项指谓的所有对象并且仅仅那些对象共同拥有的属性集,称做那个词项的\textbf{内涵}(intension)。$^{[14]}$

\subsection{外延与内涵的关系}

这样,我们看到,每个普遍或类词项都既有一个内涵意义又有一个外延意义。普遍词项"摩天大厦"的内涵包括所有超过一定高度的建筑物的共同和特有性质。"摩天大厦"的外延是一个类,这个类包括纽约的世贸中心(World Trade Center)、芝加哥(Chicago)的希尔斯塔(Sear Tow- er)、上海世界金融中心(Shanghai World Financial Center)、吉隆坡 (Kuala I-umpur)的国油双峰塔(Petronas Twin Towers)等等,也即该词项适用对象的汇集。

有时,人们断言一个词项的外延不时发生变化,尽管它的内涵没有变化。例如,有人认为,词项"人"的外延,正如人的死亡和婴儿的降生一样,持续变化。这个说法源于一种混淆。词项"人"用来指谓所有的人,包括死去的以及尚未出生的,它并没有一个不确定的外延。变化的外延是词项"活着的人"的外延。但是,"活着的人"这个词项的外延具有"现在活着的人"这种含义,其中"现在"这个词是指不断变化的现时。因此,词项"活着的人"的内涵在不同的时候也是不同的。这样就清楚了,任何具有变化外延的词项必定也有一个变化的内涵,二者是同等恒定的。

\subsection{内涵决定外延而非相反}

当一个词项的内涵固定下来时,它的外延也就固定了。注意,词项的外延由它的内涵决定,但是反过来说却不对。词项"等边三角形"的内涵是由三条等长的直线所围成的平面图形的性质。它的外延是所有那些并且仅仅那些具有这种性质的对象的类。而"等角三角形"这个词项具有的内涵却不同,它是指由三条相互相交而形成等角的直线所围成的平面图形的性质。当然,"等角三角形"这个词项的外延与"等边三角形"这个词项的外延是完全相同的。因此,确认了这些词项其中一个词项的外延,而它的内涵却处于不确定状态;外延不决定内涵,但是,内涵却必定决定外延。因此,词项可以具有不同的内涵但外延却相同;而具有不同外延的词项却不可能有同样的内涵。

\subsection{内涵与外延的反变关系}

当给一个词项的内涵添加性质时,我们就说该内涵增加了。在下面一串词项中,每个词项的内涵都包含其后相随的词项的内涵:"人"、"活着的人"、"活着的二十岁以上的人"、"活着的二十岁以上有红发的人"。在这个序列中,每个词项的内涵都比其前的那些词项的内涵多;这些词项是按照内涵增加的次序来排列的。但是,如果我们倒过来看这些词项的外延,就会发现情况相反。"人"的外延比"活着的人"的外延大,等等,并且这些词项是按照外延减少的次序排列的。

有些逻辑学家得出一条公式化的\textbf{"反变规律"},断言外延与内涵总是反向变化。这种断言具有启发性,但并不完全正确。我们可以按照增加内涵的次序构建一系列词项,但外延却不减少而保持原样。考虑这样的序列:"活着的人"、"活着的有脊骨的人"、"活着的有脊骨的不超过一千岁的人"、"活着的有脊骨的不超过一千岁的没有读完国会图书馆(Library of Congress)里所有书的人"等。显然,这些词项的次序是增加内涵,但是它们每个的外延都是相同的,完全没有减少。正确的修订"规律"是,如果词项按照内涵增加的次序排列,那么它们的外延将处于非递增的次序;也就是说,如果外延变化,那么它们将是沿着内涵的反向变化。

\subsection{外延为空的词项与意义歧义}

当然,有些词项的外延,例如"独角兽"的外延,可能是空的。认识到这一点,并运用我们对内涵与外延的区分,就可以把玩弄"意义"歧义的谬误论证揭露出来。例如,下述论证的提出旨在证明上帝的存在:

\begin{displayquote}
"上帝"这个词不是无意义的,因此它有意义。但是按照定义,"上帝"这个词的意思是全能的至善的存在(being)。因此,全能的至善的存在,即上帝,必然存在(exist)。
\end{displayquote}

这里的歧义在于"意义"和"无意义"这两个词,其中的"意义"在一种含义上指的是内涵,而在另一种含义上指的却是外延。"上帝"这个词不是无意义的,因此可以肯定,存在一个内涵是它的意义。但是,由此并不能得出:一个具有内涵的词项,其内涵一定指谓一个存在物。$^{[15]}$我们在下面这个语段中也发现了一个类似的谬误:

\begin{displayquote}
kitsch (低劣作品)$^{(1)}$ 以展示粗鄙、卑劣、下贱、脆弱和邪恶信仰来表现并败坏人类境况。这就是乌托邦之所以能被定义为kitsch 这一词项已消失的状况的原因, 因为在乌托邦中该词项已没有所指了。$^{[16]}$
\end{displayquote}

这里列举的这个歧义谬误,就是因为作者没能在意义与所指(referent)之间做出区分。许多有价值的词项(例如,那些命名希腊神话中的动物的词项)都不存在所指,但是,我们并不要求或期望这样的词项消失。实际上,具有内涵但没有外延的词项是非常有用的;如果有一天乌托邦变成了现实,那么,我们也许想要表达对减少或消除"低劣作品"或 "粗鄙"等的庆幸。而要这样做,我们就需要能够有意义地使用这些词项。

在前面的几节中,我们考察了定义的种类和它们的用途:词典定义和规定定义可消除或避免歧义,精确定义可以减少模糊性,等等。在随后的几节中,我们将考察构建定义的方法。有些定义通过外延或所指来处理普遍词项,而其他定义则通过内涵来处理之。我们将会看到,每种处理方法都既有优点又有缺点。

\begin{center}
\fbox{\parbox{0.9\textwidth}{
  \centering
  \textbf{外延与内涵的重要特征}\\
  外延:指词项可以正确适用的所有对象的集合;\\
  内涵:指词项所表示的属性或特征的集合;\\
  关系:内涵决定外延,而非相反;词项可有相同外延但内涵不同;\\
  反变规律:当词项内涵增加时,其外延不会增加,通常会减少;\\
  空外延:词项可以有内涵而无外延,这不影响词项的有意义性。
}}
\end{center} 
\section{外延定义}

\begin{quotation}
\textit{外延定义是通过列举词项所适用的对象来界定其意义的方法。尽管这种定义方式直观明了,但它存在一定的局限性,理解这些局限性有助于我们更合理地运用外延定义。}
\end{quotation}

\textbf{外延定义}就是指被定义的普遍词项所适用对象的汇集。告诉某人词项外延的最方便有效的方法就是列举出其指谓的那些对象。不过,我们须认清这种方法的局限性。

\subsection{外延定义的基本局限}

上一节我们曾指出(以"等边三角形"和"等角三角形"为例),具有不同意义即不同内涵的两个词项可以具有恰好相同的外延。因此,即使我们能够完全列举出其中一个词项指谓的对象,由此而得出的外延定义也不能把它与另一个指谓同样对象的词项区分开来。这样的两个词项不是同义词,但是,外延定义却不能在它们之间做出区分。

然而,这还不是问题的棘手之处,因为外延可以完全列举出来的词项是极少的。要列举出"数"这个词所指谓的所有数字是完全不可能的。要列举出"恒星"这个词所指谓的天文数字的对象,实际上也是绝对不可能的。就其他大多数普遍词项来说,完全列举其外延都是不可能的。

\subsection{部分列举的困难}

这样,外延定义通常就要限定于所指谓对象的\textbf{部分列举},这是个引起严重困难的局限。任何给定对象[比如,约翰•多伊(John Doe)这个人]都具有许许多多性质,因而被包括在许许多多不同的普遍词项的外延之中。当约翰-多伊作为一个词项的外延定义实例而被给出时,在许多其他词项的外延定义中,他也可以作为实例而被同样恰当地提及。约翰•多伊是"人"、"动物"、"哺乳动物"的实例,或许也是"丈夫"、"父亲"和 "学生"等等的实例。因此提到他,并不能帮助我们在这些词项的意义之间做出区分。即使我们给出两个、三个或更多实例,也会遇到同样的困难。

再譬如,在定义"摩天大厦"这个词时,我们可以使用帝国大厦(Empire State)、克莱斯勒大厦(Chrysler building)和沃尔华斯大厦(Woolworth building)等明显的范例,但是,这三座大厦作为实例同样也完全可以作为词项"二十世纪的伟大建筑"、"曼哈顿的昂贵房地产"或"纽约城的地面标志物"的外延。然而,每个这些普遍词项都指谓其他词项不指谓的对象,因此通过使用部分列举,我们甚至不能在具有不同外延的词项之间做出区分。引进\textbf{"反面事例"},例如"不是泰姬陵"(Taj Mahal)或"不是五角大楼"(the Pentagon),可以帮助说明被定义项的意义;但是,否定事例也必定是不完善的,这个基本局限仍然存在。

\subsection{通过子类列举的方法}

我们可以设法换一种列举方式,即不是每次列举一例,而是列举一整组例子。使用这种方法,也就是通过\textbf{子类来定义},有时可能做到完全的列举。例如,我们把"脊椎动物"定义为两栖动物、鸟类、鱼类、哺乳动物和爬行动物。通过列举定义,无论完全列举还是部分列举,无论是列举类的个体元素还是列举子类,虽然都具有某些心理上的用处,但是要完全确定被定义项的意义,在逻辑上都是不充分的。

\subsection{实指定义及其局限}

用列举法下定义的一个特殊类型被称做\textbf{"实指定义"}或\textbf{"示范定义"}。与一般外延定义不同,实指定义是通过用手或其他姿势指着对象来定义,不是通过命名或描述被定义词项指谓的对象来定义。例如,"'桌子' 这个词意指这",伴随着一个姿势如用手指指着桌子的方向,就是一个实指定义。

实指定义既有其自身的某些特殊局限性,也有前面所提到的各种局限性。首先,它有显而易见的地域局限:一个人只能指着看得见的东西,例如,他不能在乡村来实指地定义"摩天大厦",也不能在内陆山谷中去实指地定义"海洋"。其次,更为严重的是,姿势也有着不可避免的歧义。指着一张桌子也是指着它的一部分,以及它的颜色、大小、形状、质料等等,事实上,也就是指着位于桌子所在的方向上的所有东西,包括它后面的墙壁或者更远处的花园。

这种歧义有时可以通过给定义项增加一些描绘的短语而得到解决,其结果被称做\textbf{准实指定义}。例如,"桌子"这个词意指"这件"家具(伴随以相应的姿势)。但是,因为这种附加假设了对"家具"这个短语的事先理解,就使实指定义的宗旨难以达到。

\subsection{实指定义的非原初性}

实指定义历来被某些人视为"基本"或"原初"定义,其意思是说:我们最初都是凭借这种方式来理解词项意义的,而其他定义都依赖于这种理解。但是,这种原初性断言是错误的,因为人们必须理解姿势本身的意义。当我们用手指指向婴儿床沿时,如果那个婴儿的注意力被吸引,其注意力可能放到被指向的东西上,也可能放到我们的手指上。如果我们想用其他姿势来定义一个姿势,也会出现同样的困难。如果你要理解任何符号(sign)的定义,那么某些符号就必须已经得到理解。我们学习使用语言的根本途径是经过观察和模仿,而不是经过定义。

一如有些逻辑学家所做的那样,人们可以很宽泛地解说"实指定义"这个术语,甚至包括"当这个词指谓的对象出现时,不断地听到这个词"的过程。但是,这种过程根本不是定义,与我们这里对词项"定义"的使用不一样,它是学习使用语言的基本的和前定义的(predefinitional)方式。

\subsection{空外延词项的问题}

最后,需要指出的是,虽然有些词意义丰富,但是它们完全不指谓任何事物,因此不能从外延上定义它们。例如,当我们说不存在独角兽时,我们在断定"独角兽"这个词没有所指,具有一个\textbf{"空"外延}。这类词项不仅仅展示了外延定义的局限性,也显示出"意义"的确更适用于内涵而不是外延。因为虽然"独角兽"这个词的外延是空的,但由此肯定不是说它就是无意义的。的确,它不指谓任何事物,因为根本就没有独角兽;但是,如果"独角兽"这个词项毫无意义,那么说"不存在独角兽"也就是无意义的。然而,这个陈述并不是没有意义的,我们完全理解它的意义,而且它是真的。显然,内涵对定义来说是真正的关键所在,我们下一节即转向讨论内涵。

\begin{center}
\fbox{\parbox{0.9\textwidth}{
  \centering
  \textbf{外延定义的类型与局限}\\
  完全列举:大多数词项无法完全列举其所有适用对象;\\
  部分列举:难以准确区分不同词项,同一对象可能是多个不同词项的实例;\\
  子类列举:通过列举子类而非个体对象,但仍逻辑上不充分;\\
  实指定义:通过指示动作来定义,具有地域限制和指示歧义问题;\\
  空外延问题:某些有意义的词没有所指对象,无法通过外延定义。
}}
\end{center} 
\input{chapter3/section3-5}
\input{chapter3/section3-6}
\section*{第3章概要}
解释词项的意义就是给出它的定义。在本章中,我们讨论了几种定义及其用法,以及构建定义的方法和运用这些方法的规则。\\\\
3.1 节解释了三种论争:

1.明显的实质争论,其中没有语词歧义,而且论争双方的确在态度上或信念上对立。

2.纯粹言辞之争,其中出现语词歧义,但根本没有实质歧见。\\\\
3.表面上是言辞的但实际上是实质的论争,其中既存在语词歧义,也存在论争双方在态度上或在信念上的歧见。

3. 2 节首先解释了定义总是符号的定义,并且引进了术语被定义项 (被定义的符号)和定义项(用来解释被定义项意义的符号)。还在五种定义及其基本用法中进行了区分:

1.规定定义,把一个意义指派给某个符号。规定定义不是报道,因而既不真也不假;它是运用被定义项来意指定义项指谓事物的建议、解决、请求或工具。

2.词典定义,它报道被定义项已经具有的意义,因而它可以或对或错。

3.精确定义,它超出了平常用法,用于消除与临界状况有关的麻烦的不确定性。其被定义项有一个现存的意义,但这个意义是模糊的;增添什么可以达至精确性,部分上是个规定问题。

4.理论定义,它寻求对它的适用对象精确表述一个理论上足够或科学上有用的描述。

5.说服定义,它运用表达性语言而不是信息性语言来寻求影响态度或激发情感。

在这五种定义中,前两种(规定定义和词典定义)主要用于消除歧义;第三种(精确定义)主要用于降低模糊性;第四种(理论定义)用于促进理论理解;而第五种(说服定义)用于影响行为。\\\\
3.3 节解释了普遍词项指谓其可以正确适用的多个对象。这些对象的汇集构成该词项的外延。说明了为词项外延中的所有对象并且仅为那些对象所共有的属性集就是该词项的内涵。词项的内涵决定其外延,但外延却不能决定内涵;因此,几个词项可以具有不同内涵而外延却相同;但外延不同的词项却不可能具有相同内涵。\\\\
3.4 节解释了怎样利用普遍词项的外延来构造外延定义;外延定义有几种类型,其局限性也被揭示出来:

1.列举定义,即在定义中列出或给出词项指谓对象的范例。\\\\
2.实指定义,在定义时,我们用手指出或以姿势标明被定义项的外延。

3.准实指定义,在定义中,姿势或手指的指示伴有一些其意义被认为是已为人所知的描述短语。\\\\
3.5 节解释了怎样利用普遍词项的内涵来构建内涵定义;内涵定义也有几种类型,其局限性也被揭示出来:

1.同义定义,在定义中提供另一个其意义已为人所知的词,这个词与被定义的词具有相同意义。

2.操作定义,它表明词项正确运用于一个给定场合,当且仅当,在该场合下特有的操作行为产生特有结果。

3.属加种差定义,首先要找出一个属,被定义项所指代的种是该属的一个子类;然后找出属性(或种差),即把该种的分子与属的所有其他种的分子区分开来的那种属性。

内涵定义的方法可以用于构建 3.2 节中五种定义的任何一种:规定定义、词典定义、精确定义、理论定义和说服定义。\\\\
3.6 节明确表述和解释了传统的属加种差定义的五条规则:

1.定义应当揭示种的本质属性。\\\\
2.定义不能循环。\\\\
3.定义既不能过宽又不能过窄。\\\\
4.定义不能用歧义的、䀲涩的或比喻的语言来表述。\\\\
5.定义在可以用肯定的地方就不应当用否定定义。
\section*{【注释】}
[1]A 是正确的,巴拿马运河的太平洋入口确实是在其大西洋人口的东面。\\
[2]William James,Pragmatism(1907).\\\\
[3]lbid.\\\\
[4]为掩盖实质论争,语词有时也被故意地用做两种含义以避免争执。拉比- A•J•鲁丁(Rabbi A.J.Rudin)把"有趣的"(interesting)释义为"英语中有争议的最该县咒的词",经常被大批的宗教会众使用以掩盖说话者的真正意见。鲁丁写道: "当用于布道时,'有趣的'的通常意思是'我患有失眠症',或者'我认为你说得精彩极了'。在其最为隐䀲的意义上,'有趣的'意指职员鲁葬……表达说话者不赞同的观点。"Religious News Service,January 1992。\\\\
[5]1991年,度量衡总委员会(General Committee on Weights and Measures)对它们进行了规定定义。该委员会是一家国际机构,其管理领域是科学的单位。另外,一千亿亿也叫一"zepto",一万亿亿也叫一"yocto"。\\\\
[6]这个新术语是在纽约城由普林斯顿大学的约翰•阿奇贝尔德•威勒(John Archibald Wheeler)博士在1967年的空间研究组织的一次会议上引进的。\\\\
[7]"夸克"出现在詹姆斯-乔伊斯(James Joyce)的小说(Finnegan's Wake)的 "Three Quark for Muster Mark"一行文字中;但是,盖尔曼博士报告说,他在看到那个名字之前就已经选择了它,他仅仅是根据乔伊斯拼写了它。\\\\
[8]See The Chronicle of Higher Education, 30 May 1993.\\\\
[9]一匹 600 千克(1 323 磅)重的真马的功率要比这个数大得多,估计大约为 18000 瓦。因此,一辆 200 马力的汽车大约相当于 8 匹真马的功率。\\\\
[10]与 1 升水的质量相同的单位长期被接受为 1 "千克"的定义。但是 1 千克现在已经更精确地定义为"与巴黎附近的保险库内的金属块具有相同质量的单位"。然而,人们仍在为"千克"寻找更加精确的定义,一种以一定数量的某种原子质量为依据的精确定义。\\\\
[11]Cali fornia v.Hodari D., 499 U.S.621, 1991.\\\\
[12]American Civil Liberties Union v.Reno, 929 Fed.Supp.824, 11 June 1996.\\\\
[13]"Defining Abortion a Tricky Business,"Honolulu Advertise, 14 February 1970.\\\\
[14]逻辑学家有时用"含义"(connotation)这个词来取代内涵,并较为普遍地使用"指称"(denotation)这个词来取代外延。但是,"指称"在日常话语中有其他更加普通的用法;大多数时候,它意指一个词项的情感意义,因此,此处引进它并无帮助。正如我们这里所做,通过运用词项内涵和外延来处理这种关键的区分,什么也不会丢失,并且可以避免一些混淆。\\\\
[15]内涵与外延之间非常有用的区分是由坎特伯雷的圣安瑟伦(St.Anselm of Canterbury,1033~1109)引进并强调的,他以他的"本体论论证"而著称,上述那个谬误的论证与他的论证并不相同。请参见 Wolfgang L.Gombocz,"Logik and Existenz in Mittelater",Philosophische Rundschau(1997)。\\\\
[16]John P.Sisk,"Art,Kitsch and Politics",Commentary,May 1988.\\\\
[17]Jay Livingston,Compulsive Gamblers(New York:Harper \& Row,1974), p. 2.\\\\
[18]W.H.Voge,"Strees-The Neglected Variable in Experimental Pharmacology and Toxicology,"Trends in Pharmacological Science,January 1987.\\\\
[19]Herbert Spencer,Principles of Biology, 1864.\\\\
[20]Samuel Johnson,Dictionary of the English Language, 1755.\\\\
[21]Ambrose Bierce,The Devil's Dictionary, 1911. 

% 第四部分
\chapter{谓词逻辑}
\section*{藛紷密}
\section*{4.1 什么是谬误?}
一个论证,无论其主题或领域是什么,一般都是为证明其结论为真而构建的。但是,在两种情形下,任何论证都不能实现这一宗旨。一种情形是将一个虚假命题设定为论证的前提之一。在第1章中我们看到,每个论证都断言其结论之真是从前提到结论的真推导出来或为前提之真所蕴涵。因此,如果论证的前提不真,那么就不能确立其结论的真,即使从前提到结论的推理是正确的。然而,检验前提的真与假并不是逻辑学家的特殊职责,那是所有研究工作的共同任务,因为前提可以牵涉任何研究主题。

论证不能确立结论之真的另一种情形是,其所依赖的前提并不蕴涵结论。这才是逻辑学家的特殊领地。逻辑学家主要关心的是结论与前提之间的逻辑关系。一个论证的前提不支持它的结论,即使它的所有前提都是真的,它的结论也可能是假的。在这种情况下,其推理便是糟糕的,而这种论证就称为谬误。谬误就是推理错误。

然而,逻辑学家所用的"谬误"这个词,并不指称所有过失推理或虚假信念,而是指称一种典型错误,即经常出现在日常话语中破坏论证的错误。每个谬误都是不正确论证的一种类型。若论证中出现了一个特定类型的错误,就称为犯有那种谬误。因为每个谬误都是一种类型,故而我们可以说,两个或更多的不同论证可以包含或犯有相同谬误;也就是说,它们在推理中表现为同一种错误。包含或犯有特定类型谬误的一个论证,也可以被称为是一个谬误,也即那种类型错误的一个实例。

推理进入歧路的方式可以有很多种,也就是说,论证错误有很多种。习惯上,人们将"谬误"这个词用在那些虽然不正确但在心理上具有一定说服力的论证。有些论证错误是非常明显的,不能欺骗和说服任何人。但是,谬误却是危险的,因为我们大都会偶尔被某些谬误所愚弄。因此,我们将谬误定义为一种看似正确但经过检验可证其为错误的论证类型。研究这些错误论证是非常有益的,因为当我们明确理解它们后,就可以最有效地避开它们布下的陷阱。有备无患!

特定的论证是否事实上真是谬误,可能取决于其作者对词项的解释。看来是谬误的语段,若脱离语境,就可能难以确定作者使用的词项打算意味什么。有时,"谬误"的指责就会不公平地对准这样的语段,而其作者

想要表达的观点却被批评者漏掉了(或许,作者甚至是开玩笑的)。当我们将对谬误论证的分析运用到实际谈话中时,应当注意这种不可避免的复杂情况。我们的逻辑标准应当高,但将这些标准运用到日常生活的论证中时,也应当宽宏大量和公平。

我们可以在论证中区分出多少种不同谬误呢?亚里士多德是第一位对它有系统研究的逻辑学家,他曾列举出 13 种 ${}^{[1]}$ ;近来,超过 100 种的谬误名单被列了出米。 ${ }^{[2]}$ 然而,谬误并没有一个精确的可以确定下来的数目,因为在列举它们时,在很大程度上取决于所使用的分类体系。在此,我们挑出 17 种谬误,即推理中最普通且最有欺骗性的错误,分成三大组,分别称为:(a)相干谬误(fallacy of relevance);(b)预设谬误(fallacy of presumption);(c)含混谬误(fallacy of ambiguity)。 ${}^{[3]}$

谬误的分组总有某种程度的任意性,因为一种错误会与另一种错误具有密切的相似性,有时还是相重合的。一个给定的谬误语段应属于哪个特定组别也常常引起人们的争论,因为语段中可能会有一个以上的推理错误。如果人们谨记这种不可避免的不精确性,那么理解三种主要种类的每一种本质特征及其各种子类别的特别特征,将具有很大的实际用处。当推理中最难缠的错误出现于通常话语中时,这些理解就能够使人们发觉这些错误。辨识这些相互联系的谬误也有益于提高我们的逻辑敏感性,而这种敏感性也有益于我们识别那些在三大组中未能包含的谬误。 
\section{相干谬误}

\begin{quotation}
\textit{相干谬误是逻辑推理中最常见的一类错误,它表现为论证前提与结论之间缺乏必要的逻辑联系。识别这类谬误不仅有助于避免错误推理,也能帮助我们构建更有说服力的论证。}
\end{quotation}

\subsection{相干谬误的本质}

当一个论证所依据的前提与其结论不相干因而不可能确立结论之真时,其所犯的就是\textbf{相干谬误}。或许,称之为不相干谬误更贴切,但是,(在实际论证中)这种论证的前提常常在心理上与结论是相干的,而正是这种相干性使得它们似乎正确和有说服力。\textbf{心理的相干}怎么会与\textbf{逻辑的相干}相混淆,可以用我们在第2章讨论的语言的不同用法进行部分阐释;这些混淆的机制在随后的分析中将变得更加清晰。

很多谬误传统上都有个拉丁名称;有些拉丁名称,像\textit{ad hominem}(人身攻击),已经进入普通英语语言之中。我们在这里将既使用拉丁名称又使用英语名称。 
\input{chapter4/section4-2-R1.tex}
\input{chapter4/section4-2-R2.tex}
\subsection{R3.人身攻击论证(Argument Ad Hominem)}

短语\textbf{"ad hominem"}译做"人身攻击"。它命名的是一种谬误性反驳,即它的抨击不是指向结论,而是指向断定结论或为结论辩护的人。当一个论证攻击提出主张的人而非主张本身时,就犯了\textbf{人身攻击谬误}。

\paragraph{A.诽谤型人身攻击}

在激烈的论辩中,参与者有时贬低对手的品格,否认他们的智力或推理能力,质疑他们的正直,等等。但是,个人的品格与他主张的命题的真假或推理的正误在逻辑上并无关联。如果认为某种意见是糟糕的或断定是错误的,而其原因却只是它们是由"激进派"或"极端派"提出的,那么这就构成了人身攻击谬误的一种典型特例:\textbf{诽谤}。

诽谤的前提与结论是不相干的,然而它却可能通过转移心理进路来说服人,可以鼓动对一个人的反对态度,情感上的反对范围甚至扩展得与鼓动者做出的判断也相对立。

\paragraph{哲学辩论中的诽谤现象}
几位当代美国哲学家之间的一场尖锐论争就例示了这种谬误攻击。其中一位论辩者写道:

被体面对手以体面方式抨击的事情,在哲学中一直出现。但是,在我看来,索莫斯(Sommers)的智力方法是不诚实的。她无视哲学争论的最基本礼仪。${ }^{[9]}$

争论的对方回答道:

几个诋毁我的人所用的一个不诚实和毫无价值的策略是,认为我从没有做过的抱怨是我所做的,然后把这些"抱怨"作为"我不负责任的和轻率不公正的证据"来打发。${ }^{[10]}$

但冲突双方所居地位的应有美德,却没有在这种论证中显示出来。诽谤性人身攻击有很多种变形。对手可能被诽谤为巧舌如簧,"孤立主义者"或"干涉主义者","极右"或"极左"分子,如此等等。当诽谤性攻击论证采用攻击对立方出身(这当然与真假无关)的形式时,就称之为\textbf{"遗传谬误"}。

\paragraph{连带罪恶与证人可信度}
有时,一个结论或它的拥护者可能会因为拥护其观点者都是那些被广泛认为品质不好的人而受到指责。在其臭名昭著的审判中,苏格拉底被判决不敬之罪,部分原因就是他与那些被广泛认为对雅典不忠和品行上贪婪的人有联系。1997年,克莱德•柯林斯•斯诺(Clyde Collins Snow)因为他在科学研究中所得出的结论而被指责为种族主义者,他回答如下:

\begin{displayquote}
在过去十年中,我的工作倾注于调研许多国家的失踪、毒打和超越法律迫害的人权受害者,这使我成了公众批评和政府撒气的靶子。然而,直到今天没有一个批评我的人把我视为种族主义者。对我的诋毁,有阿根廷(Argentina)的野蛮的军事政务会辩护者、智利(Chile)的皮诺切特(Pinochet)将军的军事代表、危地马拉的(Guatemalan)国防部长以及塞尔维亚(Serbian)政府的说客。因而,古德曼(Goodman)先生(斯诺的指责者)发现他自己处于有趣的伙伴中。${ }^{[11]}$
\end{displayquote}

不公平指责是人身诽谤的极其普通的形式;\textbf{连带罪恶}是诽谤的另一种方式,它不那么广泛但却是同等谬误。

不过,在法律程序中,有时禁止不可靠者及"存疑证人"作证乃是可取的。如果不诚实在其他问题上已显示出来并因而破坏了信用,那么在法律程序中,这种存疑在这种背景下可能不是谬误。但是,由此却不能简单地断定这种证人说的是谎话。我们必须禁止各种不诚实或欺骗,也必须揭露与过去证词的矛盾。即使在这种特殊背景下,攻击品格也不能确立所给出的证词是假的;如果那样,推理便是谬误。

\paragraph{B.背景谬误}

\textbf{背景谬误}是人身攻击谬误的一种形式。引起背景谬误的是,在本不相干的信念与该信念持有者的背景之间加以牵连。人们做出或拒绝某个主张的背景并不承载该主张为真。

\paragraph{职业与身份的背景影响}
因此,如果仅仅因为对手的职业、国籍、政治联系或其他背景,就固执地迫使对手接受或拒绝某个结论,那么这样的论证就是谬误的。如果认为圣职人员必须接受某个给定观点,因为否定它就与《圣经》相矛盾,那么这是不公平的。再如,若认为政党候选人必须支持某项政策,因为它是其所属政党的纲领中公开宣示的,这也是不公正的。这样的论证与所论及的命题真假无关,它仅仅是力促某人接受背景。

\paragraph{tu quoque与偏见指控}
有人指责猎人毫无用途地屠杀没有惹人的动物,而猎人有时却通过指出其批评者食用无害牲畜来回应。这样的回应显然是人身攻击,批评者食肉的事实与证明猎人为娱乐而猎杀动物合理性根本不沾边。拉丁术语\textbf{"tu quoque"}(意思是"你是另一个"),有时被用来命名这种人身攻击论证的背景谬误。

在严肃论证中,对手的背景并不是重要问题,要求注意它们可能在取赞扬或说服他人方面起心理作用。但是,无论多么有说服力,这种论证本质上都是谬误的。

有时,背景谬误被用来表明应当拒绝对手的结论,指责导致他们做出判断的是他们的特殊处境而不是推理或证据,所以他们的判断是\textbf{有偏见的}。但是,一个有利于某团体的论证,并非就没有讨论价值;若仅仅依据其被该团体成员提出从而为该团体服务为由而非难之,就是犯了背景谬误。例如,赞成保护关税的论证可能是糟糕的,但它们之所以糟糕,却并不是因为它们是由从关税保护中获得好处的制造商提出的。

\paragraph{污泉谬误}
背景谬误论证之一,称做\textbf{"污泉"}(poisoning the well),尤为悖理。产生这个名字的事件典型地例示了这种论证。英国小说家和教士查尔斯•金斯利(Charles Kingsley)攻击著名的天主教智者约翰•亨利•卡迪拉尔•纽曼(John Henry Cardinal Newman)说,卡迪拉尔•纽曼的主张是不能信任的,因为作为一名罗马天主教的牧师,他首先要忠诚的不是真理。纽曼反驳道,这种人身攻击使他并且也使全体天主教徒的进一步论辩成为不可能,因为他们为自己辩护所说的任何东西都可以因被他人指责为根本不关心真理而遭到拒斥。卡迪拉尔•纽曼说,金斯利"污染了对话之泉"。

人身攻击论证的诽谤谬误和背景谬误之间,存在一种清晰的联系:背景谬误可以被看做诽谤谬误的一种特殊情况。譬如,当使用背景谬误明显或暗含地指责对手缺乏一贯性(在他们的信念中,或者在他们的言行之间),它很明显就是一种诽谤;而用背景谬误指责对手由于其属于某集团或具有集团信仰而缺乏信任价值,显然也是指责对手具有自利偏见的诽谤手段。无论何种形式,人身攻击论证都是对论辩对手的谬误性诋毁。 
\input{chapter4/section4-2-R4.tex}
\input{chapter4/section4-2-R5.tex}
\input{chapter4/section4-2-R6.tex}
\subsection{R7.不相干结论(Irrelevant Conclusion:Ignoratio Elenchi)}

当一个论证声称要确证一个特定的结论,但却去证明另一个与之不同的结论时,就犯有\textbf{不相干结论谬误}(Ignoratio elenchi的字面意义是"错误证明")。它的前提"不得要领";它的推理本身可能并非不合理,但它在争论树结论的辩护却没有效力。

\paragraph{政策辩论中的不相干结论}
社会法律领域中的论证经常犯有这种谬误。一个特殊方案的确是为某种被广泛支持的更大目标服务的,但为该方案进行论证的前提所提供的理由却只能支持那个大目标,而没有告诉我们关于那个特定方案的任何东西。有时这种方式是故意为之的,有时则是由于过于热情关心那种更大目标,而认识不到现有前提与特定方案的结论并不\textbf{相干}。

\paragraph{不相干结论的常见形式}
这种谬误在日常生活和政治辩论中非常普遍。例如,有人主张某政策会增加就业机会,因此应当实施,然而其论证却只证明了就业增长是件好事,而没有表明该政策实际上能够增加就业。又如,有人反对某项技术研究,理由是科技发展可能带来危害,但其论证却没有指出这项特定研究会如何造成危害。

不相干结论谬误之所以具有欺骗性,是因为它的前提往往确实支持了某个结论,只是那个结论与原本要证明的命题并不相干。仔细辨识讨论的实际焦点,是避免这种谬误的关键。 
\section*{4.3 预设谬误}
论证的精确表述而显示出来。一段话的作者、讲者,或读者、听者,都有可能会假定某些未经证明的和无根据的前提为真,无论是出于疏忽还是故意设计。而当掩藏在论证里的这种可疑假设对支持结论非常关键时,论证就是糟糕的并可使人陷人误区。这类无根据的跳跃就被称为预设谬误。

在这类论证谬误中,前提也常常与结论不相干。的确,在大多数谬误中都存在前提与结论之间不相干的缺口,但是,预设谬误展示出一种特殊的错误:那种不为人支持甚至是不可支持的暗含假定。要揭露这样的谬误,注意留心那种偷偷溜进的假设及其可疑与虚假性,通常是很奏效的。 
\input{chapter4/section4-3-P1.tex}
\input{chapter4/section4-3-P2.tex}
\section*{P3.甹题(Begging the Question:Petitio Principii)}
在所有非形式谬误中,丕题谬误可能是最多讨论、最多批评和最被滥用的一种。作为一个技术术语,它并不意味着它有时在现代用法中被赋予的意思:"引出一个问题"或"表明需要讨论一个问题";而是有它的传统含义:一个论证犯有丕题谬误,当且仅当该论证所使用的前提蕴涵、寄生于或以某种方式预设了它所要确证的结论时。正如它的拉丁名所暗示的,这种逻辑错误是"恳求"问题的原则(或开端),即恳求得到确证该结论的许可。

用同一个命题断言作为前提和结论的愚蠢尝试,是一种丕题的最明显的情形。如果有人想证明上帝存在,而断言"上帝存在,因为圣经如是说",那么,只要他同意"圣经是上帝的话语",他就做了一个圆圈般的论证,其中的结论已经被假定在前提之中。在宣称圣经是确证上帝存在的权威证据之前,他必须首先被确信有这么一个上帝。本质上,他就一直在断言,上帝的确存在,因为上帝说他存在。

这种丕题形式被称做"恶性循环"论证,即结论作为假定在前提中自现而不为人所注意,因而论证无法保护结论免于质疑。可惜地是,我们在日常谈话中经常不加心思地使用这种论证形式的一些变形,以至于甚至许多谨慎的思想家也掉入了它的陷阱。哲学家弗朗西斯•培根曾就一个几乎陷入这种谬误的自然哲学例子发出警告(他在他的《新工具》或《新方法》一书中指出了它):

人的理智(此处是指日常思维方式)不是纯粹的光明;受到意志和情感的浸染;使这些特性遂其所欲;因为一个人希望为真的,他便很容易相信是真的……他所观察的特例在他之前泛滥变多或者缺乏变少,这取决于那些特例是否会导出他先前决定的结论。[23]

但是,更细微且更容易误导的丕题谬误形式,也可以借助另一种所提出的论证来讲解,该论证旨在确证归纳原理——这种原理不是关于过去经验如何引导现在的行为,而是关于将过去经验视为一个关于未来的可靠依据的原理。任何这样的论证都企图,通过再次假定该原理为真,来寻求确证归纳程序的真实性。这种原理是,自然法则像它们操控今天一样也会操控明天,本质上自然法则在基本方面是无变化的,因而我们可以依赖过去的经验来指导我们未来的行为。"未来本质上像过去一样"的断言是问题的焦点,但是,这个断言——在平常生活中从未遭到质疑,结果非常难以证明。有些思想家断言,通过表明当我们过去依赖归纳原理时,我们总是发现这种方法能够帮助我们获取目标,这样就可以证明它。他们问:"为什么得出未来将与过去一样?"回答道:"因为它总是与过去一样。"

但是,正如大卫•休谟所指出的那样,这种常见论证是一个"peti- tio",它犯了甹题谬误。因为所讨论问题的焦点正是,自然将是否继续有规律地运行;它过去如此不能作为它未来还将如此的证据,除非一个人事先假定了正在讨论的那种原则:未来将与过去一样。因而,休谟承认过去中的未来的确都与过去一样,但他问道(这个著名的休谟问题哲学家们仍在争论):"我们怎么能够知道未来的未来将与过去一样呢?当然它们可能一样,但是,我们不能为了证明它们而假定它们。"[26] 
\input{chapter4/section4-3-P4-P5.tex}
\section*{4.4 含混谬误}
由于用心不专或故意操作,在论证过程中,词或短语的意义可能会变化。一个词项在前提中可能具有一种意义,但是在结论中却是另一种相当不同的意义。当推论依赖这样的变化时,当然就是谬误。这种错误称做 "含混谬误",有时或称为"诡论"(sophisms)。故意使用这样的方法常常是粗糙的和易于发现的,但是,有时(虽并非经常)这种含混是隐蔽的、难以把捉的。我们在下面区分出它的五种类型。 
\subsection{A1.歧义}

大多数词汇都有多于一个的字面意义,但在多数情况下,通过注意语境和利用我们良好的感觉,我们在阅读和听讲时不难将这些意义分辨开来。但是,当人们有意无意地混淆一个词或短语的几个意义时,就是在\textbf{歧义}地使用这个词或短语。如果在论证中这样做,就犯了\textbf{歧义谬误}。

\paragraph{文学中的歧义}
有时,这种歧义谬误非常明显,在某些玩笑的字里行间使用。刘易斯•卡罗尔(Lewis Carroll)在《爱丽丝镜中奇遇记》(\textit{Through the Looking Glass})中对爱丽丝的奇遇的讲述,就包含着机智和逗乐的歧义。其中一个如下:\\
"你们谁走过这条路?"国王继续走着,并向送信人伸出手要些千草。\\
"没有人(nobody)。"送信人说。\\
"很对,"国王说,"这位年轻的女子也看到过他(him)。所以,当然 Nobody 比你们走得更慢。"

在这段话中,歧义谬误其实用得是相当巧妙的。第一次使用时"nobody"这个词仅仅是指"没有人"(no person)的意思。但是,接着用代词"他(him)"来指称,就好像"nobody"这个词命名了一个人。结果,当相同的词被大写并明显地用做一个名字"Nobody"时,它就显然命名了一个人,这个人具有没有走过这条路的特性,而这个特性又是从该词的第一次运用中得来的。有时,歧义是机智的工具,刘易斯-卡罗尔就是一位非常机智的逻辑学家。${ }^{[28]}$\\

\paragraph{相对词的歧义问题}
歧义论证总是谬误的,但它们却不总是愚蠢和滑稽的,这一点将在下面节录的例子中看出来:

有一种歧义谬误特别值得一提。这是一种由错误使用\textbf{相对性}(relative)词项而来的错误;在不同语境中,相对词具有不同意义。例如,"高"就是一个相对词,高个子人与高建筑物就处于非常不同的类别。一个高的人是一个比大部分人都高的人,而一座高的建筑物是一座比大部分建筑物都高的建筑物。某些论证形式可以对没有相对性的词有效,但当用相对词来代替那些词的时候,这种论证就垮掉了。"象是动物,因此灰色的象是灰色的动物"这个论证是完全有效的。"灰色"这个词不是相对的。但是,"象是动物,因此小象是小动物"这个论证却是荒唐的。这里的关键之点是,"小"是个相对词:小象是非常大的动物。这个谬误就是一个关于相对词"小"的一种歧义谬误。然而,并非所有的有关相对词的歧义谬误都是这样显然。"好"这个词是个相对词,关于它,经常出现歧义谬误。例如,有人论证说某某是一位好将军,因此也会成为一位好总统,或者是一位好学者,从而也一定是一位好教师。 
\subsection{A2.双关}

由于前提的语法结构原因,会导致前提的表达歧义。当人们从这样的前提出发来论证时,就会出现\textbf{双关}(amphiboly)谬误。"双关"这个词来源于希腊语,它的意思实质是"一团两个",或一团的"两倍"。一个陈述是双关的,是指由于它的词汇组合松散或笨拙导致它的意义不确定。一个双关陈述可能在一种解释下可能是真的,而在另一种解释下却是假的。当以使其为真的解释来表述论证前提,而以使其为假的解释得出结论时,那么就犯了\textbf{双关谬误}。

\paragraph{政治中的双关现象}
在指导选举策略时,双关既可以迷惑人也可以误导人。20世纪90年代,当众议员托尼•科埃略(Tony Coelho)作为来自加利福尼亚州的一位民主党员而进入美国白宫代表中时,据报道,他说:"Women prefer Democrats to men."${ }^{(1)}$ 双关陈述构成危险前提,但是,在严肃的话题中人们很少遭遇它。

\paragraph{垂悬分词与短语}
文法家所谓的"垂悬"分词和短语经常有娱乐类型的双关出现。《纽约客》(The New Yorker)中的小栏新闻就曾给粗心忽视了双关的作者和编辑开了一个讽刺玩笑:\\
"Leaking badly,manned by a skeleton crew,one infirmity after another overtakes the little ship."${ }^{(2)}$(The Herald Tribune, book section)

这些游戏几乎没有缺陷![30]

\footnotetext{(1)这句话可以有两种解释:女人比男人更喜欢民主党;女人更喜欢民主党而不是男人。\\
(2)这句话可有两种解释:小船突然出现严重泄漏、配备人员最少等一个接一个的缺陷;严重泄漏、配备人员最少等一个接一个的(游戏)缺陷,突然降临小船。前者是指小船的缺陷,而后者意指游戏自身的缺陷。
} 
\section*{A3.重读}
当论证的意义变化源于对其词汇或组成部分的强调的变动时,该论证就可以证明是欺骗性的和无效的。若前提的明显意义依赖于一个可能的强调,但是,得出的结论却依赖于对相同词汇不同的重读意义,这时就犯了重读(accent)谬误。

作为示例,请考虑我们可以把不同的意义给予如下陈述:

我们不应当说朋友的坏话(We should not speak ill of our friends)。

在印刷字体及图片方面,有很多伎俩常常是通过强调某处而起误导之效。出现在新闻报道标题中的大号字敏感词汇,故意向那些勿勿浏览的人暗示错误的结论,而该标题后面却很可能用其他词汇以很小的字来加以限制。为避免在看新闻报道或在签订合同时被欺骗,我们力劝人们注意"小字印刷"。在政治宣传中,特别是在声称所谓事实报道中,选择令人误解的敏感标题或选择使用部分省略的图片,都是对重读谬误的精心使用,力图使读者得出宣传者明知为假的结论。解说可能不是彻底的谎言,但它也可以利用故意或虚假的重读方式来歪曲事实。

在广告中,这样做的也很多。非常低的价格往往以非常大的字出现,而后面却跟随着字体极小的"以及完全说明"。飞机票价打折的通告后面都跟有一个星号,以远远的一个脚注说明该价格仅仅可用于提前三个月预订星期四的飞行航班,或可能还会有其他"适用限制"。名牌昂贵商品都以非常低的价格做广告,在广告某处附有一个小注解"所列价格存货数量有限"。读者被吸引到商店,但可能以广告价格买不到商品。重读语段本身并不是严格谬误;源于重读的语段解释,当它依赖一个非常可疑的结论暗示时,即当其采用令人误解的重读来解释时,重读语段就变成了谬误 (例如,飞机票或品牌商品可以按照所列价格优先购买)。

甚至字面上为真的语段,也可以通过操纵其位置而以重读来欺骗人。

一位船长厌恶他的首席助手上班时再三喝醉,在该船的航行日记上,他几乎每天都记上:"助手今天喝醉了。"愤怒的助手进行报复。一天,船长病了,助手就自己保管日志,他在上面记着:"船长今天清醒了。" 
\section*{A4.合成}
合成谬误是不正当地从部分到整体的推理,我们可以区分两类错误。第一类合成谬误是从部分的性质错误地推出整体的性质。每一个砖块都很轻,所以由砖块砌成的墙很轻,这就是这种合成谬误的简单形式。这种谬误可能完全显而易见,因而不会欺骗任何人,但是,也有些推论,虽然犯同样的错误,却能导致正确的结论,例如:"每个砖块都是红的,所以砌成的墙是红的。"这样,属上次推理与此次推理存在一种模式相似性,却一个是非常错误的另一个却是合理的,这既有趣也有啓发性。

使这些推理看起来类似的形式是:每个部分都具有性质 P,所以整体也具有性质 P。归纳出来的规则是:判定一个特定的推理形式是否会犯合成谬误,取决于所涉及的具体性质(这里是 P)。那些只为部分所有,却不为整体所有的性质,转移到整体上就是合成谬误,而那些为部分所有,也可能为整体所有的性质,转移到整体上则可能是正当的。例如"轻"这个性质只属于部分,不属于整体,所以就出现谬误;而红色这个性质,属于部分,也可能属于整体,所以该推论不是谬误。为了辨识可能的合成谬误,我们必须研究并理解特定性质能否从部分转移到整体。

第二类合成谬误是不正当地从所有个别成员的性质(这些成员彼此分离或做单独考虑时)到该集体做一个整体时的性质的推理。例如,某位运动员观察到的事实:任何足球队员,单独看时都容易被击垮,因而断言任何足球队都可以轻易地被击垮,他就犯了这种合成谬误。团体有时候所具有的凝聚力是其个别成员所不具有的;因此单独个体的易碎性(可击垮性)不是一组受训练选手的特性。所以,从个别成员的特性到整体的特性的推理,确实容易误入歧途。那些归因于大学、公司、军队或体育团队的特性,不能轻易地由所有个别而分离的成员的特性来推出。

这两类合成谬误虽然是平行的,但却是根本上有别的,因为元素的纯粹汇集与那些元素所构成的整体是不同的。例如,机器的各部分的纯粹汇集不是机器;砖头的纯粹汇集既不是房子也不是墙壁。整体,比如机器、房子或墙壁,是将其部分以某种特定方式组织或安排起来的。正由于组织的整体与纯粹的汇集是截然不同的,所以这两种形式的合成谬误也是如此,一种是从部分到整体的无效推广,另一种是从分子或元素到汇集的无效推广。 
\section*{A5.分解}
分解谬误是合成谬误的简单颠倒;在分解谬误中,存在相同的混淆,但推论是以相反方向进行的。与合成的情形相应,我们也可以区分出两种分解谬误。第一种分解谬误断言对一个整体为真的东西一定对它的部分也真。因为某公司非常重要,并且某先生是那个公司的官员,因此某先生就是非常重要的,这个论证就犯了分解谬误。同样,从某机器沉重、复杂或者贵重这个前提而得出该机器的任何部分都一定沉重、复杂或者贵重,这个结论也属于分解谬误。一个学生一定住着一个大房间,因为该房间位于一座大楼中,这也是这种分解谬误的实例。

第二种分解谬误是从元素的汇集性质而得出元素自身的性质。因为大学生学习医学、法律、工程、牙科和建筑学,所以任何大学生都学习医学、法律、工程、牙科和建筑学,这个论证就犯了这种分解谬误。汇集地看,大学生学习所有这些科目是真的,但分布地看,大学生学习所有这些科目却是假的。这种分解谬误的例子常常看起来好像是有效论证,因为对一个类分布地为真的东西,肯定对其每一成员也是真的。例如如下论证:

狗是肉食的。\\
阿富汗猎犬都是狗。\\
因此,阿富汗猎犬都是肉食的。 
\chaptersummary{
在本章中,我们看到,\logicterm{谬误}是那种看起来\logicemph{正确}但经过考察而证明并非如此的论证。我们对常见的欺骗性推理\logicwarn{错误}类型给出了传统名称,区分出三大类非形式\logicwarn{谬误}:\logicterm{相干谬误}、\logicterm{预设谬误}和\logicterm{含混谬误}。
}

\subsection*{相干谬误}
在这类\logicwarn{谬误}中,\logicwarn{错误}论证依赖于看起来可能与结论相关但事实上无关的前提。我们分七种\logicterm{相干谬误}来解释这类推理\logicwarn{错误}。

\paragraph{R1.诉诸无知论证}
当以一命题没有被证明是\logicwarn{假的}为理由来论证该命题是\logicemph{真的},或当论证一命题是\logicwarn{假的}因为它没有被证明是\logicemph{真的}。

\paragraph{R2.诉诸不当权威}
一个论证的前提诉诸某方或多方判断,而它或它们却不能合法地声称对手头问题具有权威。

\paragraph{R3.人身攻击论证}
攻击不是针对所做的主张或针对论证的优点,而是针对对手本身。

人身攻击论证有两种形式。当攻击直接针对人,以寻求诋毁和侮辱他们时,就称做"\logicterm{诽谤性人身攻击论证}"。当攻击间接地对准人,暗示他们坚持他们的观点主要是因为他们的特殊环境或利益时,就称做"\logicterm{背景性人身攻击论证}"。

\paragraph{R4.诉诸情感}
细心推理被激起狂热或情感来支持预先结论的精心策划所取代。

\paragraph{R5.诉诸同情}
细心推理被激起听者同情来达到说者所关注目标的精心策划所取代。

\paragraph{R6.诉诸武力}
为了得到对某些结论的承诺,细心推理被直接或含沙射影的威胁所取代。

\paragraph{R7.不相干结论}
前提不得要领,声称支持一个结论而事实上却支持或证实另一个结论。

\subsection*{预设谬误}
在这类\logicwarn{谬误}中,\logicwarn{错误}论证源于依赖于某些被假定为\logicemph{真}的命题,而这些命题实际上是\logicwarn{假的}、可疑的或没有得到证明的。我们分五种\logicterm{预设谬误}来解释这类推理\logicwarn{错误}。

\paragraph{P1.复杂问语}
以问句预设了某些假设为\logicemph{真}的方式来询问问题。

\paragraph{P2.虚假原因}
把一个东西当做一个事物的原因而它实际上并不是那个事物的原因,或更一般地说,在以因果关系为基础的推理中犯\logicwarn{错}。

\paragraph{P3.乞题}
在某个论证前提中假定了结论要寻求确证的东西。

\paragraph{P4.偶然}
把某个概括运用于它不能适当管辖的个别情况。

\paragraph{P5.逆偶然}
粗心大意地从单个情况转移到一个无辩护余地的广泛概括。

\subsection*{含混谬误}
在这类\logicwarn{谬误}中,\logicwarn{错误}论证的形成方式是,它依赖于词或短语从在前提中的用法到在结论中的用法的意义变化。我们分五种\logicterm{含混谬误}来解释这类推理\logicwarn{错误}。

\paragraph{A1.歧义}
在论证的明确表述中,有意或无意地使用同一个词或短语的两个或更多意义。

\paragraph{A2.双关}
因为陈述中的词或短语结合得松散或笨拙,论证中的这个陈述具有多于一个合理意义。

\paragraph{A3.重读}
意义的变化作为对论证的词或短语的强调改变的结果而源于该论证之内。

\paragraph{A4.合成}
(a)\logicwarn{错误地}从部分性质到整体性质进行推理,(b)或者,\logicwarn{错误地}从某汇集的个别分子性质到整个汇集的性质进行推理。

\paragraph{A5.分解}
(a)\logicwarn{错误地}从整体性质到它的一个部分的性质进行推理,(b)或者,\logicwarn{错误地}从某些实体汇集的某个全体性质到该汇集的个别实体性质进行推理。

\begin{center}
\fbox{\parbox{0.9\textwidth}{
  \centering
  \textbf{谬误分类总结}\\
  \logicterm{相干谬误}:前提与结论不相干,但表面上看似有关\\
  \logicterm{预设谬误}:依赖未经证实或可疑的假设\\
  \logicterm{含混谬误}:依赖词语或短语意义的变化\\
}}
\end{center}

% 参考文献将在主文档末尾统一显示

% 第五部分
\chapter{逻辑推理}
\section*{舄 5 䓙}
\section*{5.1 演绎理论}
前面几章探讨的主要是语言及其对论证的影响,现在我们来讨论论证本身。首先来分析一种特殊的论证——演绎。演绎论证是这样一种论证,其前提被要求为结论的真提供决定性基础。如果前提之真确实能够决定其结论为真,那么,这个论证就是有效的。任何一个演绎论证都或者有效或者无效:如果不可能出现前提真而结论假的情况,那么论证就是有效的,否则就是无效的。

演绎理论旨在阐明有效论证的前提与结论之间的关系,为评估演绎论证提供方法。也就是说,演绎理论要给出区别有效演绎与无效演绎的方法。为此,历史上出现了两种杰出的理论。第一种被称为"古典的" (classical)或"亚里士多德型的"逻辑,开创这种理论的是古希腊大哲学家亚里士多德。另一种称为"现代"逻辑或"现代符号"逻辑。本章与接下来的两章(即 $5 、 6 、 7$ 三章)主要探讨古典逻辑问题,而 8、9、10 三章主要探讨现代逻辑问题。

亚里士多德是古代伟大智者之一。在柏拉图学园钻研 20 年之后,他成为亚历山大大帝的家庭教师,后来建立了自己的学园:Lyceum(吕克昂),在那里他做出了许多杰出贡献,几乎涵盖了人类知识的所有领域。亚里士多德去世以后,他关于推理的论述被收集成册,称为《工具论》 (Organon)。虽然一直到公元 2 世纪"逻辑"这个词才获得它的现代含义,但逻辑学的主题早已在《工具论》中确定了。 
\section*{5.2 直言命题及其类别}
亚里士多德对演绎的研究主要集中在由一种特殊命题组成的论证上,这种命题是关于范畴(categories)和类(classes)的,被称为"直言 (categorical)命题"。直言命题是演绎理论的基石。要了解这种关于类的演绎理论,必须首先对直言命题进行非常精细的分析。请考虑如下论证:

\begin{displayquote}
没有运动员是素食主义者,所有足球队员都是运动员,
\end{displayquote}

\begin{displayquote}
所以,没有足球队员是素食主义者。
\end{displayquote}

这个论证中的三个命题都是直言命题,包括两个前提、一个结论。这些命题肯定或否定某个类 $\boldsymbol{S}$ 全部或部分地包含于另一个类 $\boldsymbol{P}$ 之中。三个命题涉及的是运动员的类、素食者的类和足球队员的类。

有关类的知识在第 3 章讨论定义时已经简要地说明,一个类就是共有某种特定属性的所有对象(objects)的汇集。两个类之间有着多种不同的关系。

1.如果一个类的所有元素(member)都是另一个类的元素,例如狗的类与哺乳动物的类,则称第一个类包含于(be included)或包括在(be contained)第二个类之中;

2.如果一个类中有元素是另一个类的元素,但并非其所有元素都是另一个类的元素,例如女人的类和运动员的类,则称第一个类部分地包含于第二个类之中;

3.如果两个类没有共同的元素,例如三角形的类和圆形的类,则称这两个类之间是相互排斥(exclude)的。

类与类之间的这些关系被直言命题所肯定或否定,其结果是恰好能形成直言命题的四种标准形式,可分别由如下标准命题例示:

1.所有政客是说谎者。\\
2.没有政客是说谎者。\\
3.有政客是说说者。\\
4.有政客不是说谎者。

下面我们就细致地考察直言命题这四种标准形式。\\
第一个例子——所有政客是说谎者——是一个全称肯定命题。其中涉及两个类,即政客的类和说谎者的类,它说的是第一个类包含于或包括在第二个类中。全称肯定命题断言第一个类中所有元素都是第二个类的元素。在这个例子中,主项"政客"指称(designate)政客的类,谓项"说谎者"指称说谎者的类。所有全称肯定命题都可以写成如下形式:

其中字母 $S$ 和 $P$ 分别代表主项和谓项。"全称肯定命题"这一名称是恰当的,因为这个命题肯定了两个类之间的包含于关系,并且是完全或者说全部包含于关系:断言 $S$ 的所有元素同时都是 $P$ 的元素。

第二个例子——没有政客是说谎者——是一个全称否定命题。它是对全部政客而言,否定他们是说谎者。就这样两个类来说,全称否定命题断言第一个类与第二个类是完全排斥的,也就是说第一个类中没有元素是第二个类的元素。所有全称否定命题都可以写成如下形式:

没有 $S$ 是 $P$ 。\\
其中 $S$ 和 $P$ 也分别代表主项和谓项。"全称否定命题"这一名称是恰当的,因为这个命题否定了这两个类之间的包含于关系——并且是全部否定:断言在 $S$ 的所有元素中,没有一个是 $P$ 的元素。 ${ }^{(1)}$

第三个例子—有政客是说谎者——是一个特称肯定命题。显然,这个例子肯定的是政客类中有元素(也)是说谎者类的元素。但并没有对政客类作全部断言:它说的并不是所有政客,而是某个或某些政客是说谎者。此命题既没有肯定也没有否定所有政客是说谎者,对此并没有给出主张。从字面含义看,它并没有断言有政客不是说谎者,尽管在某些语境中它可能暗含这样的意思。这个命题的字面含义或者说最小的(minimal)解释,即政客的类和说谎者的类之间有某个或某些元素是共同的。为确定性起见,我们这里采取最小解释。\\
"有"(some)这个词的含义是不确定的。它指的是"至少有一个"、 "至少有两个",还是"至少有一百个"呢?到底有多少个?尽管与某些场合中的通常用法不太一致,但为了保持确定性,我们一般把"有"看做 "至少有一个"的意思。这样,特称肯定命题可以写成如下形式:

有 $S$ 是 $P$ 。\\
它断言的是,主项 $\boldsymbol{S}$ 指称的类中至少有一个元素是谓项 $\boldsymbol{P}$ 指称的类的元素。"特称肯定命题"这个名称是恰当的,因为这种命题肯定了类之间具有某种包含于关系,但不是全部而只是部分地(partially)肯定第一个类

\footnotetext{(1)我国逻辑教材中一般把全称否定命题的形式写为:"所有 $S$ 不是 $P$",其与"没有 $S$ 是 $P$"同义。但根据英语语法,"All $S$ are not $P$"并不与"No $S$ are $P$"同义,而等义于"Not all $S$ are $P$"。故英文著作一般将"No $S$ is(are)$P$"作为全称否定命题的形式。
}中的某个或某些元素包含于第二个类。\\
第四个例子——有政客不是说谎者———是——个特称否定命题。这个例子,正如上面的例子一样,谈论的并不是全部政客,而只是政客类中某个或某些元素,因而是特称的。不同于第三个例子的是,它并非肯定第一类中的某部分包含于第二个类中,相反,它是否定的。所有特称否定命题可以写成如下形式:

\section*{有 $S$ 不是 $P$ 。}
\section*{它断言的是,主项 $\boldsymbol{S}$ 指称的类中至少有一个元素被谓项 $\boldsymbol{P}$ 指称的类的全体所排斥。}
并非所有标准式直言命题都像以上四个例子那样简单明了。标准式命题的主项、谓项指称的都是类,但这些词项可能是复杂的表达式而非一个单词。举例来说,在命题"所有这个职位的候选人都是诚实而正直的人"中,主项是"这个职位的候选人",谓项是"诚实而正直的人"。

曾经有一种传统观点,认为所有演绎论证都可以用类或范畴以及它们之间的关系加以分析。这样,如上说明的直言命题的四种标准形式,就被认为是所有演绎论证的基石:

\begin{displayquote}
全称肯定命题(称为 A 命题)\\
全称否定命题(称为 E 命题)\\
特称肯定命题(称为 I 命题)\\
特称否定命题(称为()命题)
\end{displayquote}

(尽管这种传统观点是不正确的,但)${ }^{(1)}$ 的确有许多逻辑理论一一正如我们将要看到的——就是以这四种命题为基础建立起来的。
\footnotetext{(1)括㚽内的话为译者所加。
} 
\section*{5.3 质、量与周延性}
\section*{A.质}
每个标准式直言命题或是肯定的或是否定的,这叫做命题的质。如果一个命题肯定了类与类间的包含于关系,不管是全部地还是部分地肯定,那么,它的质就是肯定的。因此全称肯定命题和特称肯定命题的质都是肯定的。它们的简写名称,即 A 和 I,分别来自于拉丁文"AffIrmo",该词的意思是"我肯定"。如果一个命题否定类与类间的包含关系,不管是全部地还是部分地否定,那么,它的质就是否定的。因此全称否定命题和特称否定命题的质都是否定的。它们的简写名称,即 E 和 O ,分别来自于拉丁文" nEgO ",该词的意思是"我否定"。

\section*{B.量}
每个标准式直言命题或是全称的或是特称的,这称为直言命题的量。如果一个命题述及主项所指称的类的所有元素,那么,它的量就是全称的。因此 A 命题和 E 命题的量都是全称的。如果一个命题只述及主项所指称的类的某些元素,那么,它的量就是特称的。因此 I 命题和 O 命题的量都是特称的。

每个标准式直言命题都以"所有"、"没有"或者"有"等词开头,这些词表明了命题的量。"所有"和"没有"表示命题是全称的,"有"表示

命题是特称的。另外,"没有"还表明了 E 命题的质是否定的。\\
我们发现"全称肯定"、"全称否定"、"特称肯定"和"特称否定"这几个名称都是先描述量再描述质,从而唯一地描述了每一种标准式直言命题。

\section*{C.标准式直言命题的一般模式}
每个标准式直言命题的主项、谓项之间都有一个动词形式"是"(O命题需在"是"前面加上一个"不"字),这个动词把主项和谓项联结起来,称为联项(copula)。前一节给出的公式中的联项只有"是"和"不是"两种,但依据不同的措辞需要,有时可能用其他形式的联项更为适当。例如,下面三个命题中:

\begin{displayquote}
有罗马统治者曾经是(were)独裁者。\\
所有正方形均为(are)四边形。\\
有士兵不会成为(will not be)英雄。
\end{displayquote}

联项分别是"曾经是"、"均为"和"不会成为"。标准式直言命题的一般模式由四个部分组成:首先是量项,其次是主项,再次是联项,最后是谓项。可以记为:

量项(主项)联项(谓项)。

\section*{D.周延性}
基于类的解释,标准式直言命题指称的都是对象的类,而命题被看做是关于这些类的。当然,命题谈及类的方式不尽相同。一个命题可能谈及一个类的全部元素,也可能只谈及这个类的一些元素。这样一来,下面这个命题:

\begin{displayquote}
所有参议员是公民。
\end{displayquote}

述及或曰关乎全部参议员,但没有述及所有公民。它断定的是参议员类的任何一个元素都是公民,但并没有就所有公民做出断言。它既没有肯定,也没有否定所有公民都是参议员。这样一来,任何一个有如下形式的 A命题:

都述及了主项 $S$ 指称的类的全部元素,但并没有述及谓项 $P$ 指称的类的全部元素。

我们引入"周延"这个技术性术语,用以刻画出现于直言命题中的主谓项的性质。如果一个命题述及了某个词项所指称的类的全部元素,则称该词项在这个命题中是周延的。我们来考察一下各种标准式直言命题,看看其中哪些词项周延、哪些词项不周延。

首先来看 A 命题。仍以"所有参议员都是公民"为例。A 命题的主项(在命题中)是周延的,而谓项(在命题中)是不周延的。

接下来看 E 命题,比如:

没有运动员是素食主义者。

这样的 E 命题,断定了任何一个运动员都不是素食主义者。整个运动员的类都被排除在素食主义者的类之外。由于 E 命题述及了主项指称的类的全部元素,因此可以说 E 命题的主项是周延的。同时,由于断定了整个运动员的类被排除在素食主义者的类之外,这个命题也就断定了整个素食主义者的类也被排除在整个运动员的类之外。上述例句显然断定了任何一个不是运动员的素食者,因此,它就涉及了谓项指称的类的全部元素,所以说,E命题的谓项也是周延的。总之,E命题的主项周延,谓项也周延。

说到 I 命题,情况就有所不同了。例如:

有士兵是胆小鬼。

既没有对所有士兵进行断定,也没有对所有胆小鬼进行断定。不能说一个类完全包含于另一个类之中,也不能说完全排除在外。在任何特称肯定命题中,主项、谓项都是不周延的。

特称否定命题或者说 O 命题与特称肯定命题一样,主项不周延。例如:

并不言说所有的马,而只述及主项指称的类的一些元素。它说的是所有马中被排除在良种马之外的那一部分,亦即这部分被排除在后一个类的全体之外。假如谈的只是特定的这部分马,那么,任何一个是良种马的元素都不在这部分之中。说某事物被排除在一个类之外,也就述及了这个类的全部。正像说一个人被排除在某个国家之外,就等于说这个国家的任何地方都不接纳此人一样。特称否定命题的谓项是周延的,但主项不周延。

周延性问题可以总结如下:全称命题,包括肯定的和否定的,其主项是周延的,而特称命题,不管是肯定的还是否定的,其主项都是不周延的。也就是说,标准式直言命题的量决定了主项的周延情况。肯定命题,无论全称的还是特称的,其谓项都是不周延的,而否定命题,包括全称的和特称的,其谓项都是周延的。也就是说,标准式直言命题的质决定了谓项的周延情况。

下图作为这些知识的总括,有助于我们掌握命题当中各词项的周延 188情况。\\
\includegraphics[max width=\textwidth, center]{2025_05_15_6a28331d5e7c993ad07ag-234} 
\section{传统对当方阵}

\begin{quotation}
对当方阵是传统逻辑中分析直言命题之间关系的重要工具,包含矛盾关系、反对关系、下反对关系和差等关系四种基本关系类型。这些关系为直接推论提供了逻辑基础,帮助我们从一个命题的真假推断出相关命题的真假情况。
\end{quotation}

具有相同主项和相同谓项的标准式直言命题,可能在量上有所不同,或者在质上有所不同,也可能在质与量上都不同。先前的逻辑学家称之为\textbf{对当关系}(opposition),各种对当关系之间有很重要的真值联系。

\subsection{矛盾关系(Contradictories)}
两个命题之间具有\textbf{矛盾关系},如果一个是另一个的拒斥(denial)或否定(negation),也就是说,它们既不能同真也不能同假。显而易见,如果两个标准的直言命题的主项相同、谓项也相同,而质、量都不同,那么,它们就是矛盾的。例如 A 和 O 就是这样的,比如:

\begin{quote}
所有法官都是律师。

与

有法官不是律师。
\end{quote}

这两个命题的质与量都是对立的,显然它们是矛盾的。其中之一为真时,另一个恰恰为假。

同理,E 和 I 也是这样:

\begin{quote}
没有政客是理想主义者。

与

有些政客是理想主义者。
\end{quote}

这两个命题的质与量都是对立的,因而它们是矛盾的。用公式表示就是:"所有 $S$ 是 $P$"的矛盾命题是"有 $S$ 不是 $P$",而"没有 $S$ 是 $P$"的矛盾命题是"有 $S$ 是 $P$ ";A 和 O 互为矛盾,E 和 I 互为矛盾。

\subsection{反对关系(Contraries)}
两个命题之间具有\textbf{反对关系},如果它们不能同时为真,也就是说,可以由一个的真推出另一个的假。例如,"得克萨斯队将在比赛中战胜俄克拉何马队"与"俄克拉何马队将在比赛中战胜得克萨斯队"就是反对的;如果两个命题中的一个(当然指的是在同一场比赛中)是真的,那么另一个必定为假。但它们不是矛盾的,因为如果他们打成平手,两个命题就同时为假。具有反对关系的两个命题,不能同时为真,但可以同时为假。

传统逻辑或曰亚里士多德逻辑认为,如果两个直言命题都是全称的,其主、谓项分别相同而质不同,那么它们就是互相反对的。\cite{kneale1962} A 命题和相应的 E命题不能同时为真,却可以同时为假,所以它们之间是反对关系,例如"所有诗人都是懒汉"与"没有诗人是濑汉"。

如果 A 命题或者 E 命题是必然真的一一即在逻辑上或数学上为真一一那么,说它们是互相反对的就是不正确的。例如,"所有三角形都是四边形"与"没有三角形是四边形"。如果一个命题是必然真的一一不可能为假的一一那么,它就没有反对命题,因为互相反对的命题可以同假。我们把既非必然真也非必然假的命题称为\textbf{偶真的}(contingent)。如果一个 A 命题和一个 E 命题都是偶真的,并且它们有相同的主项和相同的谓项,那么,说它们是反对的就是正确的。本章其他部分的讨论都假定 A和 E 是偶真的。

\subsection{下反对关系(Subcontraries)}
两个命题之间具有\textbf{下反对关系},如果它们不能同假,但可以同真。传统上认为,如果两个直言命题都是特称的,其主、谓项分别相同而质不同,那么它们之间是下反对关系。也就是肯定了 I 和 O 命题,可以同真但不可同假,如:

\begin{quote}
有钻石是珍贵的石头。

有钻石不是珍贵的石头。
\end{quote}

必定是下反对的。

如果 I 和 O 必然为假,那么,说它们是下反对的就不正确。例如"有正方形是圆"与"有正方形不是圆"。如果一个命题必然为假——也就是说,不可能为真一一那么它就不会有下反对的对立面。因为,两个互为下反对的命题是可以同真的。当然,如果 I 和 O 都是偶真的,就可以同真。与反对关系一样,本章其他部分的讨论亦假定 I 和 O 都是偶真的。

\subsection{差等关系(Subalternation)}
如果两个命题有相同的主项和相同的谓项,并且它们的质相同(即都是肯定的或者都是否定的),但量不同(即一个为全称,另一个为特称),那么,它们之间的关系就是\textbf{差等关系}。例如,A 命题:

\begin{quote}
所有蜘蛛都是八脚动物。
\end{quote}

有一个相应的 I 命题:

\begin{quote}
有蜘蛛是八脚动物。
\end{quote}

而 E 命题:

\begin{quote}
没有鲸是鱼。
\end{quote}

也有一个相应的 O 命题:

\begin{quote}
有鲸不是鱼。
\end{quote}

此前用于说明命题间对当关系的例子都有"对立"(disagreement)的含义。但这里使用的"对当关系"是个专业术语,也适用于并不存在"对立"含义的情况。在上面的例子中,A 命题和相应的 I 命题之间不"对立",E命题和 O 命题之间也不"对立",但它们却都具有一种特殊的"对当关系"。全称命题与相应的特称命题之间的对当关系叫做差等关系。在一对相应的命题中,比如刚才给出的两对,全称命题叫做\textbf{上位式},特称命题叫做\textbf{下位式}。

传统上认为,在差等关系中,上位的真蕴涵下位的真。举例来说,从全称肯定命题"所有鸟是有羽毛的"可以得出特称肯定命题"有鸟是有羽毛的";而从全称否定命题"没有鲸是鱼"可以得出特称否定命题"有鲸不是鱼"。但下位并不蕴涵上位。从特称肯定命题"有动物是猫"不能得出全称肯定命题"所有动物是猫"。同样,从特称否定命题"有动物不是猫"当然也不能推出"没有动物是猫"的结论。

\subsection{对当方阵}
命题之间这四种对当关系——矛盾关系、反对关系、下反对关系以及上位与下位之间的差等关系——可以用一个重要且广为应用的图示来表示,称为\textbf{对当方阵},见图 5-1。

\begin{center}
\includegraphics[width=\textwidth]{images/2025_05_15_6a28331d5e7c993ad07ag-238.jpg}

图 5-1 对当方阵
\end{center}

展示在对当方阵中的关系,为把握论证的一些基本形式的有效性提供了一个逻辑基础。如果我们按照惯例将论证分为\textbf{直接推论}和\textbf{间接推论},那就更容易理解了。

任何论证都是从一个或多个前提得出一个结论。包括一个以上前提的推论叫做间接推论,例如三段论,其结论就是从第一个前提经由第二个前提为中介得出的。而如果从唯一的前提出发,不经过任何中介推得结论,这样的推论叫做直接推论。容易见得,许多非常有用的直接推论,可从传统对当方阵所包含的知识中获得。

下面看一些例子。如果以 A 命题为前提,根据对当方阵,可以有效地推出相应的(即主、谓项分别相同的)O命题为假。从同样的前提也可以有效地推出相应的 I 命题为真。当然,从 I 命题的真不能推出相应的 A命题为真,但可以推出其矛盾命题 E 为假。以对当方阵为基础,还可以得到许多这样的直接推论。给定任一标准式直言命题的真假情况,就可以直接得到其他某个或者所有其他相应命题的真假情况。基于对当方阵的直接推论可以列表如下:

\begin{enumerate}
\item 如果 A 真,那么, E 假、 I 真、 O 假;
\item 如果 E 真,那么, A 假、 I 假、O 真;
\item 如果 I 真,那么,E 假,A、O 真假不定;
\item 如果 O 真,那么,A 假,E、I 真假不定;
\item 如果 A 假,那么,O 真,E、I 真假不定;
\item 如果 E 假,那么, I 真, A 、O 真假不定;
\item 如果 I 假,那么, A 假、 E 真、O 真;
\item 如果 O 假,那么, A 真、 E 假、 I 真。
\end{enumerate}



\begin{center}
\fbox{\parbox{0.95\textwidth}{
\textbf{本节要点}
\begin{itemize}
\item 对当方阵展示了四种基本的命题关系:矛盾关系、反对关系、下反对关系和差等关系
\item 矛盾关系:两命题不能同真也不能同假(A与O, E与I)
\item 反对关系:两命题不能同真但可以同假(A与E)
\item 下反对关系:两命题不能同假但可以同真(I与O)
\item 差等关系:上位命题的真蕴涵下位命题的真,但下位的真不蕴涵上位的真(A与I, E与O)
\item 对当方阵为直接推论提供了重要的逻辑基础
\end{itemize}
}}
\end{center} 
\section{其他直接推论}

\begin{quotation}
除了基于对当方阵的直接推论外,逻辑学还发展了其他几种重要的直接推论形式。本节将介绍三种主要的直接推论方法:换位法、换质法和换质位法,这些方法允许我们从一个标准式直言命题有效地推出另一个相关命题。
\end{quotation}

除了基于对当关系的直接推论,还有另外一些直接推论,本节我们来讨论其中的三种。

\subsection{换位法(Conversion)}
第一种叫做\textbf{换位法},它是一种仅仅通过交换命题中主、谓项的位置而进行的推论。对于 E 命题和 I 命题来说,换位法肯定是有效的。很显然,若断言"没有人是天使",也就可以断言"没有天使是人"。这两个命题可以通过换位法进行有效的互推。同样显然的是,"有作家是妇女"与"有妇女是作家"在逻辑上也是等价的,可以通过换位从其中一个有效地推出另一个。一个标准式直言命题叫做另一个的\textbf{换位命题},如果它是通过交换另一个命题的主、谓项的位置而得到的。例如,"没有理想主义者是政治家"是"没有政治家是理想主义者"的换位命题,它们可以通过换位有效地互推。换位法直接推论中的前提叫做\textbf{被换位命题}(convertend),结论叫做\textbf{换位命题}(converse)。

请注意,从被换位 A 命题不能普通有效地推出换位 A 命题。比如,已知"所有狗是动物",当然不能有效地推出它的换位命题"所有动物是狗",因为被换位命题为真,而换位命题为假。传统逻辑自然也认识到这一点,但它认为对于 A 命题进行某种类似换位的推论可以是有效的。如 5.4 节表明,依据对当方阵,从 A 命题(所有 $S$ 是 $P$ )可以有效地推出其相应的下位 I命题(有 $S$ 是 $P$ )。A 命题说的是 $S$ 类中全部元素的情况,而 I 命题则限制为只述及 $S$ 类中的部分元素的情况。我们已经知道 I 命题是可以有效换位的。

这样,给定一个 A 命题(所有 $S$ 是 $P$ ),就可以根据差等关系,有效地得到相应的下位命题(有 $S$ 是 $P$ ),而下位命题(有 $S$ 是 $P$ )又可以进行有效换位,因此,通过差等关系和换位法的结合,从所有 $\boldsymbol{S}$ 都是 $\boldsymbol{P}$ 就可有效地推出有 $\boldsymbol{P}$ 是 $\boldsymbol{S}$ 。这种推论称为\textbf{限制换位}(或"偶然换位"[conversion per accidens]),即交换主谓项的位置,同时将命题的量由全称改为特称。因此,按照传统逻辑的认识,"所有狗都是动物"可以有效地推出"有动物是狗",这个推论就是"限制换位"。下一节将进一步探讨这个问题。

请注意,作为限制换位结论的换位命题与原来的 A 命题并不等价,原因在于限制换位需要改变命题的量,把全称改为特称。因此,限制换位的结果不是一个 A 命题而是 I 命题,它不可能与被换位命题有同样的意义,从而不可能在逻辑上等价。但 E 命题的换位命题仍是一个 E 命题, I命题的换位命题仍是一个 I 命题,在这样的情况下,被换位命题与换位命题有同样的量,并且在逻辑上是等价的。

最后需要注意的是, O 命题的换位一般是无效的。 O 命题"有动物不是狗"很明显是真的,但它的换位命题"有狗不是动物"显然是假的。O命题与其换位命题并不等价。

一命题的换位命题总与原命题词项相同(只是位置互换),并且质也相同。下表是对传统换位推论的完整描述:

\begin{center}
\begin{tabular}{|l|l|}
\hline
\multicolumn{2}{|c|}{换位法} \\
\hline
被换位命题 & 换位命题 \\
\hline
A:所有 $S$ 是 $P$ & I:有 $P$ 是 $S$ \\
\hline
E :没有 S 是 $P$ & E :没有 $P$ 是 $S$ \\
\hline
I:有 $S$ 是 $P$ & I:有 $P$ 是 $S$ \\
\hline
O :有 $S$ 不是 $P$ & (换位无效) \\
\hline
\end{tabular}
\end{center}

\subsection{换质法(Obversion)}
接下来讨论的直接推论类型叫做\textbf{换质法}。在解说之前,我们先简要回顾一下"类"这个概念,并由此引人一些新的概念,以便更容易讨论换质法。一个类就是具有某种共同属性的所有对象的汇集。这种共同属性叫做 \textbf{类的定义特征}(class-defining characteristic)。举例来说,所有人的类就是所有具有"是人"这个特征的事物的集合,属性"是人"就是这个类的定义特征。类的定义特征不一定是"简单"的属性,任何一个属性都可以确定一个类。比如"左撒子、有红头发并且是学生"这个复杂属性就确定了一个类——所有是左撇子、有红头发的学生的类。

所有的类都有一个相应的\textbf{补类}(complementary class),或简称\textbf{补}(complement),即不属于原来的类的所有东西的汇集。比如,所有人的类的补就是所有不是人的东西组成的类。该类的定义特征是不是人这样一个(否定的)属性。所有人的类的补包括除人之外的所有东西:鞋子、轮船、封蜡和大白菜等等——但不包括国王,因为国王是人。把所有人的类的补称为"非人的类"更简洁一些,词项 $S$ 所指称的类的补则由词项非 $S$ 指称,因而可以说词项非 $S$ 就是词项 $S$ 的补。\cite{kneale1962}

我们在两种意义上使用"补"这个词:一是类意义上的补,二是词项的补。尽管两者有所不同,但却是密切联系着的。一个词项是另一词项的词项补,仅当第一个词项指称第二个词项所指称的类的补。应当说明的是,正如一个类是其(类)补的补一样,一个词项也是其(词项)补的补。其中用到了"双重否定"法则,这样就可以省去许多用做前缀的 "非"字。例如,如果把词项"选举人"的补写做"非选举人",而"非选举人"的补就简记为"选举人",而不是"非非选举人"。

必须注意不要把\textbf{反对词项}当做\textbf{互补词项},比如将"懦夫"等同于"非英雄"。没有既是懦夫又是英雄的人,但并非每个人--当然更不是任何东西--都必须或者是懦夫或者是英雄,所以词项"懦夫"与"英雄"之间是反对关系。再比如 "胜者"的补不是"败者"而是"非胜者",因为并非所有东西——或者说所有人——必须或是胜者或是败者。但每个东西必定或是胜者或是非胜者。

了解了词项补的含义,换质法就比较容易描述了。在换质法中,主项保持不变,被换质命题的量也不需改变。对一个命题进行换质,就是改变其质,并用谓项的补替换原来的谓项。例如下面这个 A 命题:

\begin{quote}
所有居民都是选举人。
\end{quote}

换质后成为一个 E 命题:

\begin{quote}
没有居民是非选举人。
\end{quote}

显然,这样两个命题在逻辑上是等价的,因此从一个可以有效地推出另一个。换质法应用到任何标准式直言命题,都是有效的直接推论。例如,下面的 E 命题:

\begin{quote}
没有仲裁人是偏心的。
\end{quote}

换质后得到一个等值的 A 命题:

\begin{quote}
所有仲裁人都是不偏心的。
\end{quote}

同样的,I 命题:

\begin{quote}
有金属是导体。
\end{quote}

换质后得到一个 O 命题:

\begin{quote}
有金属不是非导体。
\end{quote}

最后,O 命题:

\begin{quote}
有国家不是好战的。
\end{quote}

换质后得到一个 I 命题:

\begin{quote}
有国家是不好战的。
\end{quote}

换质法直接推论中的前提叫做\textbf{被换质命题}(obvertend),结论叫做\textbf{换质命题}(obverse)。所有标准式直言命题与其换质命题在逻辑上都是等价的,所以,对任何一个标准式直言命题而言,换质法都是有效的。要得到一个命题的换质命题,不需改变原命题的量和主项,而是要改变它的质,并用谓项的补替换原来的谓项。下表对传统上的换质推理进行了全面的刻画:

\begin{center}
\begin{tabular}{|l|l|}
\hline
\multicolumn{2}{|c|}{换质表} \\
\hline
被换质命题 & 换质命题 \\
\hline
A:所有 $S$ 是 $P$ & E :没有 $S$ 是非 $P$ \\
\hline
E :没有 S 是 $P$ & A:所有 $S$ 是非 $P$ \\
\hline
I:有 $S$ 是 $P$ & O :有 $S$ 不是非 $P$ \\
\hline
O :有 $S$ 不是 $P$ & I :有 $S$ 是非 $P$ \\
\hline
\end{tabular}
\end{center}

\subsection{换质位法(Contraposition)}
讨论第三种直接推论并不需要引人新的原理,从一定意义上讲,这种方法可以还原为前面两种推论。对给定的命题进行\textbf{换质位},就是将主项换为原命题谓项的补,并将其谓项换为原命题主项的补。例如,A命题:

\begin{quote}
所有会员都是选举人。
\end{quote}

换质位后是 A 命题:

\begin{quote}
所有非选举人都是非会员。
\end{quote}

容易见得,以上两个命题在逻辑上是等价的。对 A 命题进行换质位是有效的直接推论形式。对 A 命题首先换质,再换位,然后再换质,于是就从最初的"所有 $S$ 是 $P$"转化为"所有非 $P$ 是非 $S$"。因此,对任何一个 A 命题进行换质位,都是将原命题先换质,再换位,然后再换质。

换质位法用于 A 命题是最有用的,用于 O 命题也是有效的直接推论形式。例如对于 $O$ 命题:

\begin{quote}
有学生不是理想主义者。
\end{quote}

换质位后得到一个有点绕口的 O 命题:

\begin{quote}
有非理想主义者不是非学生。
\end{quote}

它与第一个命题在逻辑上是等价的。如果每次只转化一步,即先换质,再换位,再换质,那么就可以显示出其逻辑等价性。可把其中的推论用公式表示为:从"有 $S$ 不是 $P$"换质得"有 $S$ 是非 $P$",再换位得"有非 $P$ 是 $S$",继续换质得"有非 $P$ 不是非 $S$"(换质位命题)。

一般说来,换质位法对于 I 命题无效。用下面这个真的 I 命题可以证明这一点:

\begin{quote}
有公民是非议员。
\end{quote}

换质位后得到一个假命题:

\begin{quote}
有议员是非公民。
\end{quote}

其无效的原因在于对 I 命题进行换质位,就要对 I 命题先换质,再换位,然后再换质。I命题"有 $S$ 是 $P$"换质后得 O 命题"有 $S$ 不是非 $P$",而后者一般不能有效换位。

E 命题"没有 $S$ 是 $P$"的换质位命题是"没有非 $P$ 是非 $S$",这也不是从原命题有效地得出的,下面的例子可以说明这一点, E 命题:

\begin{quote}
没有摔跤运动员是体弱的人。
\end{quote}

其为真,但完全换质位命题却是假的:

\begin{quote}
没有非体弱的人是非挥跤运动员。
\end{quote}

为了得到其换质位命题,我们对 E 命题先进行换质,再换位,然后再换质,就可以找到无效的原因。 E 命题"没有 $S$ 是 $P$"换质后得 A 命题 "所有 $S$ 是非 $P$"。一般说来,A命题不能有效地换位,除非进行限制换位。于是,通过限制换位得"有非 $P$ 是 $S$",再换质得"有非 $P$ 不是非 $S$",我们称之为\textbf{限制换质位}。下一节我们将进一步讨论这个问题。

请注意,通过限制性换质位法,我们可从一个 E 命题推得一个 O 命题——即从"没有 $S$ 是 $P$"推出"有非 $P$ 不是非 $S$"——与限制换位有同样的特点。由于从全称命题只能推特称命题,结果得到的换质位命题与原命题意义不同,与作为原命题的 E 命题逻辑上不等价。而 A 命题的换质位命题仍是 A 命题, O 命题的换质位命题仍是 O 命题,在这两种情况下,换质位命题与其前提是等值的。

因此,换质位法只对 A 命题和 O 命题是有效的,对 I 命题是无效的,对 E 命题进行限制换质位才是有效的。也可以用一个图表来完整刻画这种推理:

\begin{center}
\begin{tabular}{|l|l|}
\hline
\multicolumn{2}{|c|}{换质位表} \\
\hline
前提 & 完全换质位命题 \\
\hline
A:所有 $S$ 是 $P$ & A:所有非 $P$ 是非 $S$ \\
\hline
E :没有 $S$ 是 $P$ & O :有非 $P$ 不是非 $S$(限制) \\
\hline
I:有 $S$ 是 $P$ & (换质位无效) \\
\hline
O :有 $S$ 不是 $P$ & O :有非 $P$ 不是非 $S$ \\
\hline
\end{tabular}
\end{center}

还有一些其他类型的直接推论,也都有各自的分类与特定名称,但并不需要引入新原理,我们在此就不再讨论了。

若要解决关于命题之间关系的某些问题,最好的方法就是研究从其中一个可以推得另一个的各种直接推论。例如,假定命题"所有外科医生都是内科医生"为真,是否可以推知"没有非外科医生是非内科医生"的真假情况?在此可以给出一个有用的方法,就是尽可能从给定命题推出多个有效结论,来看要考察的命题——或其矛盾命题和反对命题——是否能从为真的原命题有效地推出。上面的例子中,已知"所有 $S$ 是 $P$",我们可以有效地推出其换质位命题"所有非 $P$ 是非 $S$",再限制换位得"有非 $S$是非 $P$"——按照传统逻辑,它是已知命题的有效结论,因此是真的。根据逻辑方阵,它与被考察的命题"没有非 $S$ 是非 $P$"为矛盾关系,因此被考察的命题就是假的。

如 1.9 节所指出的,一个有效推理,如果前提为真,其结论必然为真。但如果前提为假,结论却可能为真。我们立即会联想到限制换位、限制换质位以及差等关系推理,它们正是后面一种情况的例子。例如,从假前提"所有动物是猫",根据差等关系推理,可以推出"有动物是猫"这样一个真结论。而假前提"所有父母都是学者"限制换位后也可以得到一个真结论"有学者是父母"。因此,如果已知一个命题为假,那么另一个(与之多少有些关系的)命题的真假情况就成了问题。比较好的方法是,从已知命题的矛盾命题或被考察命题本身着手。因为一个假命题的矛盾命题必然为真,所有从后者开始的有效推理也必然是真命题。而如果从被考察命题能够推出已知为假的命题,那么它本身也必然是假的。

\begin{center}
\begin{tabular}{|l|l|}
\hline
\multicolumn{2}{|c|}{换位法、换质法、完全换质位法} \\
\hline
\multicolumn{2}{|c|}{换位法} \\
\hline
被换位命题 & 换位命题 \\
\hline
A:所有 $S$ 是 $P$ & I:有 $P$ 是 $S$ \\
\hline
E :没有 S 是 $P$ & E :没有 $P$ 是 $S$ \\
\hline
I:有 $S$ 是 $P$ & I:有 $P$ 是 $S$ \\
\hline
O :有 $S$ 不是 $P$ & (换位无效) \\
\hline
\multicolumn{2}{|c|}{换质法} \\
\hline
被换质命题 & 换质命题 \\
\hline
A:所有 $S$ 是 $P$ & E :没有 $S$ 是非 $P$ \\
\hline
E :没有 $S$ 是 $P$ & A:所有 $S$ 是非 $P$ \\
\hline
I:有 $S$ 是 $P$ & O:有 $S$ 不是非 $P$ \\
\hline
O :有 $S$ 不是 $P$ & I:有 $S$ 是非 $P$ \\
\hline
\multicolumn{2}{|c|}{换质位法} \\
\hline
前提 & 完全换质位命题 \\
\hline
A:所有 $S$ 是 $P$ & A:所有非 $P$ 是非 $S$ \\
\hline
E :没有 $S$ 是 $P$ & O :有非 $P$ 不是非 $S$(限制) \\
\hline
I:有 $S$ 是 $P$ & (换质位无效) \\
\hline
O :有 $S$ 不是 $P$ & O :有非 $P$ 不是非 $S$ \\
\hline
\end{tabular}
\end{center}

\footnotetext{(3)我们有时用类的"相对补"(relative complement)来进行推论,即它的补包含在另外一个类中。比如:"我的孩子"这个类有一个子类"我的女儿",后者的补是另一个子类"我的不是女儿的孩子",即"我的儿子"的类。但换质法以及其他直接推论通常建基于类的绝对补之上,正如上面所定义的那样。}

\begin{center}
\fbox{\parbox{0.95\textwidth}{
\textbf{本节要点}
\begin{itemize}
\item \textbf{换位法}:交换命题的主谓项位置
  \begin{itemize}
  \item 对E和I命题总是有效的
  \item 对A命题仅限制换位有效
  \item 对O命题无效
  \end{itemize}
\item \textbf{换质法}:改变命题的质并用谓项的补替换原谓项
  \begin{itemize}
  \item 对所有标准式直言命题都有效
  \item 所有命题与其换质命题在逻辑上等价
  \end{itemize}
\item \textbf{换质位法}:用谓项的补替换主项,用主项的补替换谓项
  \begin{itemize}
  \item 对A和O命题有效
  \item 对I命题无效
  \item 对E命题只有限制换质位有效
  \end{itemize}
\item 通过这些直接推论方法,可以解决许多关于命题间关系的问题
\end{itemize}
}}
\end{center} 
\input{chapter5/section5-6.tex}
\input{chapter5/section5-7.tex}
\section{第5章概要}
本章介绍并讨论的是\textbf{古典逻辑}即\textbf{亚里士多德型逻辑}的基本构件,也是\textbf{演绎逻辑}的基本构件。

\subsection{类与标准式直言命题}
5.2节介绍了\textbf{类}的概念。传统逻辑正是以类为基础建立起来的。我们阐明了四种基本的标准式直言命题:

\begin{itemize}
  \item \textbf{A命题}:全称肯定命题
  \item \textbf{E命题}:全称否定命题
  \item \textbf{I命题}:特称肯定命题
  \item \textbf{O命题}:特称否定命题
\end{itemize}

\subsection{命题的质与量}
5.3节更加详细地考察这四种命题。探讨了命题的\textbf{质},即肯定和否定,以及命题的\textbf{量},即全称和特称。说明了\textbf{周延的项}与\textbf{不周延的项}。

\subsection{对当关系}
5.4节探讨这几种直言命题之间的\textbf{对当关系}的种类:命题之间的\textbf{矛盾关系}、\textbf{反对关系}、\textbf{下反对关系}以及上位式与下位式之间的\textbf{差等关系}。并用一个\textbf{对当方阵}图示了这几种关系,进而说明了一些基于传统方阵的直接推理。

\subsection{直接推论方法}
5.5节阐明其他三种直接推论:\textbf{换位法}、\textbf{换质法}和\textbf{换质位法}。

\subsection{存在含义问题}
5.6节探讨\textbf{存在含义}问题。要保留传统对当方阵,只有做出一种假定,即全盘假定命题主项所指称的类总是有元素的——这是现代逻辑极不赞同的。然后,我们又对本书通篇采用的\textbf{布尔解释}作了说明。布尔解释能保留传统逻辑对当方阵中的大部分内容,同时又避免了非空类的假定。在布尔解释中,特称命题,即称为$\mathbf{I}$和$\mathbf{O}$的命题之中有存在含义,但全称命题,即$\mathbf{A}$和$\mathbf{E}$则没有存在含义。我们也很细致地说明了采用这种解释的结果。

\subsection{命题的符号化与图示化}
5.7节介绍将直言命题符号化与图示化的方法,包括使用\textbf{文恩图},用交叉的圆加以恰当的标记或阴影来刻画类与类之间的关系。

\begin{center}
\fbox{\parbox{0.9\textwidth}{
  \centering
  \textbf{第5章要点总结}\\
  \textbf{演绎逻辑基础}:古典逻辑的基本构件与概念\\
  \textbf{命题分类}:A全称肯定、E全称否定、I特称肯定、O特称否定\\
  \textbf{推理方法}:直接推论、对当关系、布尔解释、文恩图\\
}}
\end{center}

有了这些必要的工具,我们就可以考察——在接下来的两章中——建基于标准式直言命题之上的\textbf{三段论},以及传统演绎逻辑在日常语言中的其他主要用途。 

% 第六部分
\chapter{归纳逻辑}
\input{chapter6/section6-1.tex}
\section{三段论论证的形式性质}

\begin{quotation}
三段论的有效性完全取决于其逻辑形式,而非内容。本节将阐明三段论的形式性质,解释为什么相同形式的三段论具有相同的有效性或无效性,并介绍如何使用逻辑类推法来检验和论证三段论的有效性。
\end{quotation}

三段论的形式由格与式唯一确定——从逻辑的观点讲,这种形式是三段论最重要的方面。三段论的有效性与无效性(其构成命题都是偶真的)仅仅依赖于形式,而完全独立于具体内容和题材。例如,任何形式为 AAA-1 的三段论:

\begin{quote}
所有 $M$ 是 $P$ ,\\
所有 $S$ 是 $M$ ,\\
所以,所有 $S$ 是 $P$ 。
\end{quote}

无论其题材是什么,它都是有效的论证。这就是说,无论用什么词项代替这种形式或结构中的字母 $S$、$P$ 和 $M$,得到的论证总是有效的。例如用这几个字母分别代表"雅典人"、"人"和"希腊人",代入后就得到这样一个有效论证:

\begin{quote}
所有希腊人是人,\\
所有雅典人是希腊人,\\
所以,所有雅典人是人。
\end{quote}

\begin{quote}
所有钠盐是水溶性物质,\\
所有肥皂是钠盐,\\
所以,所有肥皂是水溶性物质。
\end{quote}

这样一个论证也是有效的。

说一个有效的三段论是有效的论证,是仅就其形式而言的。这说明如果某个三段论是有效的,那么,任何与它形式相同的三段论也是有效的。如果一个三段论是无效的,那么,任何与它形式相同的三段论也是无效的。$^{[2]}$ 这是人们在实际论辩中经常使用逻辑类推法而获得的共识。假如有人提出下面这个论证:

\begin{quote}
所有自由主义者都是国家健康保险的支持者,\\
有行政人员是国家健康保险的支持者,\\
所以,有行政人员是自由主义者。
\end{quote}

我们会感觉到,无论其构成命题的真假,这个论证是无效的。揭示这种三段论荒谬性的最好方式,是构造一个形式相同但其无效性可直接显示出来的论证。比如,我们可以这样去问,你是否也可以说:

\begin{quote}
所有兔子都是跑得很快的,\\
有马是跑得很快的,\\
所以,有马是兔子。
\end{quote}

我们可以补充说明:你不可能为后面这个论证作辩护,因为毫无疑问,其前提明显为真但结论明显为假。你刚才的论证与这个马兔论证的形式完全相同。马兔论证是无效的,所以你刚才的论证也是无效的。逻辑类推是一种很好的论辩方法,是用于争辩的有力武器之一。

这种逻辑类推法的根据是:直言三段论的有效性或无效性是纯形式问题。要证明任何荒谬论证无效,都可以找另一个论证,使之与一个明显无效的即其前提明显为真而结论明显为假的论证有着相同形式。(不过应当牢记,无效论证也可能得到为真的结论——说推理是无效的,只是意味着结论与前提之间不构成逻辑蕴涵关系,或者说它们之间的关系不是必然联系。)

但是,这种检验论证有效性的方法有很大的局限性。有时很难一下子"想出"恰当的逻辑类推。并且,三段论论证有太多无效的形式(200多个)。此外,尽管我们只要想到一个前提为真而结论为假的逻辑类推,就可以证明原论证的形式无效,但是,若我们不能想到这样的逻辑类推,并不就能证明该形式有效,因为这可能只是由于我们的思维局限性所使然。很可能实际上存在着无效性类推,只是我们没有想到而已。这就需要一种更有效力的方法,来判定形式有效或无效的三段论。本章以下各节就是要介绍检验三段论的一些最有力的方法。

\footnotetext{(2)严格来说,这一论断只适用于被解释为不含存在预设的标准式三段论。对于某些其他三段论形式来说,尽管逻辑类比未必总是有效的,但如果用所依靠的前提所含的具体概念来代替这个形式中出现的词项,那么所得到的论证仍然是有效的——即如果前提为真,结论也必为真。}

\begin{center}
\fbox{\parbox{0.95\textwidth}{
\textbf{本节要点}
\begin{itemize}
\item 三段论的有效性或无效性仅取决于其逻辑形式(格与式的组合)
\item 相同形式的三段论具有相同的有效性或无效性特征
\item \textbf{逻辑类推法}:通过构造相同形式但前提明显真而结论明显假的论证,证明某个形式的三段论无效
\item 逻辑类推法虽然有用,但有其局限性:
  \begin{itemize}
  \item 有时难以找到恰当的类推例子
  \item 无法有效证明三段论形式的有效性
  \item 三段论无效形式太多(200多个)
  \end{itemize}
\item 需要更系统的方法来判定三段论的有效性
\end{itemize}
}}
\end{center} 
\section*{6.3 检验三段论:文恩图解法}
前一章已经介绍了两个圆的文恩图如何用于描述标准式三段论的命题。要运用文恩图解法检验直言三段论,就必须把两个前提在同一个图示中描述出来。这就要求画相互重叠的三个圆,因为标准式三段论有两个前提,共包含着三个不同的项——大项、小项和中项——分别记为 $S 、 P$ 和 $M$ 。我们首先并列画两个交叉圆,与图示单个命题一样,然后再在其下方画出第三个圆,与前两个圆都有重叠部分,依次给三个圆标记 $S 、 P$ 和 $M$ 。既然一个圆既能表示类 $S$ 也能表示类 $\bar{S}$ ,标有 $S$ 和 $P$ 的两个交叉的圆能表示四个类,即 $S P 、 S \bar{P} 、 \bar{S} P$ 和 $\overline{S P}$ 。这样,标有 $S 、 M$ 和 $P$ 的三个交叉的圆可以表示八个类:$S \overline{P M} 、 S P \bar{M} 、 \bar{S} P \bar{M} 、 S \bar{P} M 、 S P M 、 \bar{S} P M 、 \overline{S P} M$ 和 $\overline{S P M}$ 。三个圆可以将其所在平面分为八个部分,它们分别表示了上列八个类,见图 6-1。\\
\includegraphics[max width=\textwidth, center]{2025_05_15_6a28331d5e7c993ad07ag-274}

图6-1\\
以瑞典人的类( $S$ )、农民的类( $P$ )和音乐家的类 $(M)$ 为例,可以作如下解释。SPM 是这三个类的积,由所有瑞典农民音乐家组成。SPM是前两个类与第三个类之补的积,由不是音乐家的瑞典农民组成。S $\bar{P} M$是第一、第三个类与第二个类之补的积:所有不是农民的瑞典音乐家组成的类。 $S \overline{P M}$ 是第一个类与其他两个类之补的积:所有不是农民也不是音乐家的瑞典人组成的类。接下来, $\bar{S} P M$ 是第二、第三个类与第一个类之

补的积:所有不是瑞典人的农民音乐家组成的类。 $\bar{S} P \bar{M}$ 是第二个类与其他两个类之补的积:所有不是瑞典人也不是音乐家的农民组成的类。 $\overline{S P M}$ 是第三个类与前两个类之补的积:所有不是瑞典人也不是农民的音乐家组成的类。最后,$\overline{S P M}$ 是三个类的补的积:所有不是瑞典人、不是农民,也不是音乐家的事物组成的类。

我们把注意力集中到标有 $P$ 和 $M$ 的两个圆上,显然,加上阴影或写人 $x$ 就能表示出 $P 、 M$ 构成的任何标准式直言命题,无论哪个是主项哪个是谓项。例如,要用图表示命题"所有 $M$ 是 $P$"$(M \bar{P}=0)$ ,我们就把所有不包含在 $P$ 中的 $M$ 的部分加上阴影。这个区域,包括了标有 $S \bar{P} M$ 和 $\overline{S P M}$ 的部分,这样就形成了图6-2。

我们再把注意力集中到 $S$ 和 $M$ ,加上阴影或写人 $x$ 就能表示由 $S$ 、 $M$ 构成的任何标准式直言命题,无论它们出现的顺序。要用图表示命题 "所有 $S$ 是 $M$"$(S \bar{M}=0)$ ,我们就把所有不包含在 $M$ 中的 $S$ 的部分加上阴影。这个区域,包括了标有 $S \overline{P M}$ 和 $S P \bar{M}$ 的部分。这样就形成了图 6-3。

现在,利用三个交叉的圆,就可以在一个图中同时表示两个命题——当然,条件是其中只出现三个不同的项。这样,"所有 $M$ 是 $P$"和"所有 $S$ 是 $M$"可以同时表示在图6-4中。\\
\includegraphics[max width=\textwidth, center]{2025_05_15_6a28331d5e7c993ad07ag-275(2)}

图6-2\\
\includegraphics[max width=\textwidth, center]{2025_05_15_6a28331d5e7c993ad07ag-275(1)}

图6-3\\
\includegraphics[max width=\textwidth, center]{2025_05_15_6a28331d5e7c993ad07ag-275}

图6—4

这正是三段论 AAA-1 的两个前提:

此三段论是有效的,当且仅当两个前提蕴涵或曰能推出结论,即两个前提已断言了结论所断言的东西。因此,在文恩图中画出有效论证的前提,也就已经把结论画出来了,而不需要画更多的圆。结论"所有 $S$ 是 $P$"的文恩图,应为在标有 $S \overline{P \bar{M}}$ 和 $S \bar{P} M$ 的部分加上阴影。我们看到表示两个前提的文恩图,也确实已经把结论表示了出来。这种情况说明 AAA- 1 一定是有效式。 ${ }^{[3]}$

我们再用文恩图检验一个明显无效的三段论:

所有狗是动物,\\
所有猫是动物,\\
所以,所有猫是狗。

用文恩图表示两个前提就是图6-5。\\
\includegraphics[max width=\textwidth, center]{2025_05_15_6a28331d5e7c993ad07ag-276}

动物

图6-5\\
在这个图中,$S$ 指称所有猫组成的类,$P$ 指称所有狗组成的类,而 $M$指称所有动物组成的类,$S \overline{P M} 、 S P \bar{M}$ 和 $\bar{S} P \bar{M}$ 部分已经加上了阴影,但结论却没有被表示出来,因为 $S \bar{P} M$ 部分没有阴影,要图示结论就必须把 $S \overline{P M}$ 和 $S \bar{P} M$ 两部分都加上阴影。这样,就能看出 AAA-2 的两个前提的图示并没能表示结论,这证明结论的断定超出了前提,前提并不蕴涵结论。而一个前提不蕴涵结论的论证是无效的,所以,我们所画出的图示证明了这个三段论是无效的。(实际上,它证明任何形如 AAA-2 的三段论都是无效的。)

如果我们用文恩图检验由一个全称前提和一个特称前提构成的三段论,那么,很重要的一点是:要首先在图中表明全称前提。举例来说,要

所有艺术家都是自我主义者,\\
有艺术家是乞亞,\\
所以,有乞丐是自我主义者。

我们应当先画出全称前提"所有艺术家都是自我主义者",再写人 $x$ 表示特称前提"有艺术家是乞丐"。正确的图示见图 6-6。\\
\includegraphics[max width=\textwidth, center]{2025_05_15_6a28331d5e7c993ad07ag-277}

图6-6\\
$S \bar{P} M$ 连同 $\overline{S P} M$ 两个部分表明的是全称前提,如果在给这两个部分加上阴影之前,试图先表明特称前提,我们就不能确定到底该把 $x$ 加在 $S P M$ 部分还是 $S \bar{P} M$ 部分,或者两个部分都加上。如果加在 $S \bar{P} M$ 当中,或者加在这部分与 $S P M$ 的交界处,那么加在 $S \bar{P} M$ 中的阴影就让图的原意变得含混不清。既然前提中包含的信息已经表示在图中了,检验时就看它是否也表明了结论。如果结论"有乞丐是自我主义者"能被表示出来,就应该把 $x$ 放在"乞甹"和"自我主义者"两个类的交叉部分。这个部分既包含了 SPM 也包含 SPM 部分,它们共同表明 SP。此三段论的结论所断定的东西,已经在表示前提的图示中表明了,因此,这个三段论是有效的。

再来看另一个例子,对这个例子的讨论将说明文恩图一个更重要的作用。考虑论证:

\begin{displayquote}
所有大科学家都是大学毕业生,有职业运动员是大学毕业生,
\end{displayquote}

首先在 $S P \bar{M}$ 和 $\bar{S} P \bar{M}$ 部分加上阴影,这样就表明了全称前提(见图6— 7),但我们仍然不明白应该把 $x$ 加到哪一部分才能表明特称前提。特称前提是"有职业运动员是大学毕业生",所以 $x$ 必须加在标有"职业运动员"和"大学毕业生"的两个圆交叉的部分。但是,交叉的区域又包括两个部分:$S P M$ 和 $S \bar{P} M$ 。 $x$ 应该放到其中哪个部分呢?前提并没有告诉我们答案,如果任意选一部分把 $x$ 加上去,就可能在图中加上了一些前提中本来没有的信息——也就破坏了文恩图检验的有效性。如果在两个部分都加一个 $x$ ,也会超出前提的断言。但是,把 $x$ 放在两部分的交界线上,我们就正好表明了第二个前提,而没有提供任何更多的东西。 $x$ 放在交界线上,指的是有东西属于它们两个类中的一个,但确实没有说明到底属于哪一类。两个前提的完整图示见图 6--8。\\
\includegraphics[max width=\textwidth, center]{2025_05_15_6a28331d5e7c993ad07ag-278}

图 6-7\\
\includegraphics[max width=\textwidth, center]{2025_05_15_6a28331d5e7c993ad07ag-278(1)}

图6-8\\
通过考察,我们会发现前提的图示中并没有把结论表达出来。要表明结论"有职业运动员是大科学家",就必须要加一个 $x$ 在上面两个圆的交叉部分,或者加在 $S P \bar{M}$ 中,或者加在 $S P M$ 中。前面一个部分已经被加上阴影排除出去了,当然就不会包含 $x$ 。但图示也并没有显示 $S P M$ 中有 $x$ 。确实,\\
$S P M$ 或者 $S \bar{P} M$ 之中必有一个元素,但这个图示并没有说明究竟是在前者当中,还是在后者当中,因为前提就没有说明这一点。所以,结论就可能为假。当然,我们由此并不能确定结论就是假的,而只能知道结论没有被前提断定或蕴涵。但这已足以告诉我们论证是无效的。这个图也充分说明并非只有给定的这个三段论是无效的,而是所有形如 AII-2 的三段论都是无效的。

使用文恩图检验标准式三段论的一般做法可以总结如下:首先,在三圆的文恩图上标记三段论的三个项。接下来,把两个前提在图中都表示出来,如果一个前提是全称、另一个是特称的话,要首先标明全称前提。特别注意,如果特称前提并没有明确表明应该把 $x$ 加在哪一部分时,就把 $x$放在两个部分的交叉线上。最后,检查图示中是否已经包含了结论:如果包含了,那么三段论就是有效的,否则就是无效的。

使用文恩图区分三段论有效性与无效性的理论依据是什么?这个问题的答案可分为两个部分。首先,必须结合 6.2 节讲到的三段论论证的形式性质来讨论。那一节讲过,对给定三段论有效还是无效的一种合理检验是去判定另外一个三段论的有效性如何,而这个三段论与要考察的三段论恰有相同的形式。这种方法正是文恩图解法的基本依据。而说明文恩图解法如何达到这种检验目标,就构成问题解答的第二部分。

三段论一般都是就对象类而言的,其中的对象并不都呈现在我们面前,比如音乐家的类、大科学家的类、钠盐的类等等。这些类之间的包含或排斥关系可能是由论证得出的,也可能是在科学研究过程中经验地发现的。但它们绝不会自己呈现出来,因为涉及的类的元素不可能全部展现出来接受观察。但我们可创设一种情形,在这种经特殊界定的情形中,所涉及各个类的元素都可以呈现在人们面前以供直接观察,从而可以构造关于这种自设情形的三段论论证。文恩图本来是为表示标准直言命题而发明的,但其亦属于一种创设情形,一种可以使用各种工具在纸上、在黑板上绘制出来的情形。其所表示的命题亦可以解释为指涉图形本身。兹举一例即可说明。假如我们有这样一个关于分布在世界各地的各色人等的特殊三段论,其词项分别指称成功人士、热爱工作者和心猿意马者(工作中注意力分散者):

所有成功人士都是热爱工作者,没有热爱工作者是心猿意马者,

所以,没有心猿意马者是成功人士。

它的形式是 AEE-4,即:

所有 $P$ 是 $M$ ,\\
没有 $M$ 是 $S$ ,\\
$\therefore$ 没有 $S$ 是 $P$ 。 
\section{三段论规则和三段论谬误}

\begin{quotation}
为了系统判断三段论的有效性,逻辑学发展出一系列规则。本节介绍六条重要的三段论规则,并详细说明违反这些规则会导致的各种形式谬误。掌握这些规则与谬误,不仅有助于检验三段论的有效性,还能帮助我们在实际推理中避免常见的逻辑错误。
\end{quotation}

在许多情况下,一个三段论并不能真正推得其结论。为帮助人们避免常见的错误,人们制定了一系列规则(本书列出六条)用来范导论证:对于任何给定的标准式三段论,通过考察其中是否有违反规则的情况,就能对它进行评判。

违反任何一条规则都会导致错误。这是一种特殊种类的论证错误,所以我们称之为\textbf{三段论谬误};又因为这种错误是论证形式方面的,所以称之为\textbf{形式谬误}(与第4章所讲非形式谬误相对照)。在三段论论证中,必须谨防违反规则,避免产生谬误。每一种形式谬误都有一个传统名称,以下详加介绍。

\subsection{规则1 避免四项}
一个有效的标准式直言三段论必须仅仅包含三个项,在整个论证中,每一个项都须在相同的意义上使用。

在直言三段论中,结论断定了两个项即主项(小项)与谓项(大项)之间的关系。因此,只有前提断定的是这两个项分别与同一个第三项(中项)的联系时,结论才能是合理的。如果前提不能做到这一点,就不能在结论的两个项之间建立联系,论证就不能进行。所以,每个有效的直言三段论必须只有三个项——不能多也不能少。如果包含了多于三个的项,三段论就是无效的。这种谬误叫做\textbf{四项谬误}。

这种谬误通常源于语词歧义,即用同一个词或短语表达两种不同的含义。最常见的是中项的含义发生转换,同一个词以某种用法与小项发生联系,而以另一种用法与大项发生联系。这样一来,与结论中的两个项发生联系的是两个不同的项(而不是同一个中项),所以结论断定的关系也就不能成立。$^{[4]}$

本章开始定义"直言三段论"时,就指出每一个三段论一定有且只有三个项。$^{[5]}$ 所以,可以把这个规则("避免四项")看做是一个论证成为一个真正的三段论的保证。

\subsection{规则2 中项至少在一个前提中周延}
如果(如5.3节所说明)命题述及一个项所指称的全部对象,该项在命题中就是"周延"的。如果中项在两个前提中都不周延,推出结论所需要的词项关联就不能建立。

历史学家芭芭拉•塔克曼(Barbara Tuchman)认为,许多无政府主义的早期批判家是以下面这样一个"无意识的三段论"为依据进行论证的。

\begin{quote}
所有俄国人是革命者,\\
所有无政府主义者是革命者,\\
所以,所有无政府主义者是俄国人。$^{[6]}$
\end{quote}

这个三段论显然是无效的。错误在于它根据无政府主义者、俄国人两个类分别与革命者的类之间的联系,断定了前两个类的关系——但革命者这个项在两个前提中都是不周延的。第一个前提没有述及全部革命者,第二个前提同样没有。"革命者"在论证中做中项,如果它在三段论两个前提中都不周延,那么三段论就不可能是有效的。这样的谬误叫做\textbf{中项不周延谬误}。

这个规则的依据是小项和大项之间的联系需要中项做中介。而要建立这种联系,结论的主项或者谓项就必须与中项所指称类的全部对象相关联。否则,结论中的两个项就有可能分别与中项的不同部分发生联系,因而不必然与另一个项相关联。

这恰好是上面给出的三段论所存在的问题。俄国人只包含在革命者类的一部分当中(据第一个前提),无政府主义者也只是包含在革命者类的一部分之中(据第二个前提)——这两部分却是与另一个类(三段论的中项)的不同部分发生联系的,所以,中项就不能成功地联结小项和大项。一个有效的三段论,其中项必定至少在一个前提中周延。

\subsection{规则3 在结论中周延的项在前提中也必须周延}
述及一个类的全部对象,比述及其中某些对象要断定更多。所以,如果三段论前提中不周延的项在结论中周延,也就是结论断定了比前提更多东西。但是,有效的论证要求其前提必须能逻辑地推出结论,结论绝不能比前提断定得更多。可以说,在结论中周延而在前提中不周延的项确实是个信号,说明结论超出了前提,跑得太远了。这种谬误叫做\textbf{不当周延}。

结论可能是小项(主项)超出了前提,或者大项(谓项)超出了前提。所以,不当周延有两种不同形式,我们分别给它们一个名字:

\textbf{大项不当周延}("非法大项"),\\
\textbf{小项不当周延}("非法小项")。

举个例子来说明第一种,看下面这个三段论:

\begin{quote}
所有的狗是动物,\\
没有猫是狗,\\
所以,没有猫是动物。
\end{quote}

很明显,这个论证是不对的,但错在哪里呢?就错在结论是对所有动物的断言,即结论断定的是所有动物都在猫的类之外,而前提并没有对所有动物做出断言——故结论不当地超出了前提的断定。由于"动物"在三段论中做大项,所以此处的谬误就是非法大项。

再举个例子来说明第二种,看下面这个三段论:

\begin{quote}
所有传统教徒都是原教旨主义者(fundamentalist),\\
所有传统教徒都是宽容堕胎行为的,\\
所以,所有宽容堕胎行为的都是原教旨主义者。
\end{quote}

我们立刻会感觉到这个论证也有问题,其错误就在于:结论断定了所有堕胎行为的宽容者,而在前提中并没有这样的断言,没有述及所有宽容堕胎行为者的情况。这样,结论就不能为前提所担保。这个例子中"宽容堕胎行为的"是小项,所以此处的谬误就是非法小项。

\subsection{规则4 避免出现两个否定前提}
任何否定命题($\mathbf{E}$ 或 $\mathbf{O}$)都否认类的包含关系,断定一个类的部分或者全部被排除在另一类的全体之外。但是,由两个断定这种排斥性的前提不能得出结论中的联系,因此,不可能是有效的论证。这种错误叫做\textbf{排斥前提谬误}。

理解这个谬误需要进一步思考。考虑三段论的小项 $S$、大项 $P$ 和中项 $M$,对于这三个项之间的联系,两个否定前提能告诉我们什么呢?它们说明 $S$(结论的主项)完全或部分地排斥 $M$(中项)的一部分或者全部,并且 $P$(结论的谓项)完全或部分地排斥 $M$ 的一部分或者全部。但是,不管 $S$ 和 $P$ 的关系如何,这些关系中的任何一个都可能成立。这样的否定前提不能告诉我们 $S$ 和 $P$ 之间究竟是包含还是排斥,究竟是全部地包含或排斥,还是部分地包含或排斥。因此,如果三段论的两个前提都是否定的,论证肯定是无效的。

\subsection{规则5 如果有一个前提是否定的,那么结论必须是否定的}
如果结论是肯定的,也就是说,如果它断言两个类中的一个($S$ 或 $P$)完全或部分地包含在另一个之中,那么,前提必须断定这样的第三个类存在才能推出结论,即第三个类必须包含第一个并且被第二个包含,而类之间的这种包含关系只能由肯定命题表示。所以,肯定的结论只能由两个肯定的前提得到。违反这条规则的错误叫做\textbf{从否定推肯定谬误}。

要想得出肯定结论必须要有两个肯定前提,如上所述,我们可以确定地说,只要两个前提中有一个是否定的,结论就必须也是否定的,否则论证无效。

与其他谬误不同,这个谬误并不常见,因为对于任何从否定前提得肯定结论的论证,很容易就可以看出是极不合理的。举一个例子就能说明:

\begin{quote}
没有诗人是会计,\\
有艺术家是诗人,\\
所以,有艺术家是会计。
\end{quote}

立即可以看到,由第一个前提对诗人和会计的排斥关系的断言,已使得该论证不可能为艺术家和会计之间的包含关系提供任何有效辩护。

\subsection{规则6 两个全称前提得不出特称结论}
在直言三段论的布尔解释中(见5.6节),全称命题(A和E)没有存在含义,但特称命题($\mathbf{I}$ 和 $\mathbf{O}$)却有存在含义。只要像本书这样设定了布尔解释,就要避免从没有存在含义的前提得出有存在含义的结论。

最后这个规则在传统逻辑或者亚里士多德逻辑对直言三段论的解释中并不需要,因为它们并不关心存在含义问题。但是,仔细考虑一下预设问题就会很清楚,如果一个论证的前提根本没有断定什么东西存在,但是从这些前提却推出了有些东西的存在,那么结论就是不合理的。这种错误叫做\textbf{存在谬误}。

下面这个例子就犯有这种谬误:

\begin{quote}
所有宠物都是家养动物,\\
没有独角兽是家养动物,\\
所以,有独角兽不是宠物。
\end{quote}

假如这个论证的结论是全称的"没有独角兽是宠物",它是完全有效的。在传统解释下,由于全称命题与特称命题一样都有存在含义,例子中的结论只是上述有效论证结论的"下位"。

但从布尔解释的角度说,上例的结论("有独角兽不是宠物")不仅仅是个"下位",因为特称命题与全称命题有很大不同。结论是特称的 $\mathbf{O}$ 命题,有存在含义,而 $\mathbf{E}$ 命题("所有独角兽不是宠物")是没有存在含义的。传统观点下接受的推论在布尔解释下不再被接受,因为在后者看来这样的论证犯了存在谬误——种在传统解释下不会出现的错误。$^{[7]}$

以上给出的六条规则只适用于标准式直言三段论。它们提供了足够的工具,用以检验这一领域内任何论证的有效性。对于任一标准式直言三段论,如果违反了任一规则就是无效的,如果遵循了所有的规则就一定是有效的。

\footnotetext{(4)关于词项歧义的一个经典例子是论证"有机物是化学物质,化学物质是无机物,所以,有机物是无机物"。从表面看,它的式是AAA-1,应该是个有效论证。但是,中项"化学物质"的含义在两个前提中有所不同,所以,这个三段论实际上包含了四个项,犯有四项谬误,是无效的。}
\footnotetext{(5)请注意,三段论可能包含多于三个词项。它只需包含三种不同的项即可,但每种项可能重复出现。实际上,一个标准式直言三段论包含四个词项,但其中有一个重复出现两次。}
\footnotetext{(6)见Barbara Tuchman,《历史的棱镜》(The Proud Tower)(New York:MacMillan,1966),130页。}
\footnotetext{(7)关于存在含义问题,可以参考5.6节。}

\begin{center}
\fbox{\parbox{0.95\textwidth}{
\textbf{本节要点}
\begin{itemize}
\item \textbf{三段论六大规则}:
  \begin{itemize}
  \item \textbf{规则1 避免四项}:有效三段论必须只有三个项,且每项意义不变
  \item \textbf{规则2 中项至少在一个前提中周延}:否则无法建立结论项之间的联系
  \item \textbf{规则3 结论中周延的项在前提中也必须周延}:结论不能断定比前提更多内容
  \item \textbf{规则4 避免出现两个否定前提}:仅依靠排斥关系无法建立结论项之间的联系
  \item \textbf{规则5 若有一个前提是否定的,结论必须是否定的}:肯定结论需要两个肯定前提
  \item \textbf{规则6 两个全称前提得不出特称结论}:无存在含义的前提不能推出有存在含义的结论
  \end{itemize}
\item \textbf{三段论谬误}:
  \begin{itemize}
  \item \textbf{四项谬误}:通常因词项歧义而包含超过三个项
  \item \textbf{中项不周延谬误}:中项在两个前提中都不周延
  \item \textbf{不当周延}:分为大项不当周延和小项不当周延
  \item \textbf{排斥前提谬误}:两个否定前提不能确立结论项之间的联系
  \item \textbf{从否定推肯定谬误}:从包含否定前提的论证得出肯定结论
  \item \textbf{存在谬误}:从不含存在含义的前提得出有存在含义的结论
  \end{itemize}
\end{itemize}
}}
\end{center} 
\section*{6.5 直言三段论的 15 个有效形式}
三段论的式取决于其中所含三个命题的类型(A、E、I、O)。直言三段论有 64 个不同的式,即这三个命题的 64 种可能组合:AAA、AAI、 AAE 等等,一直到 $\cdots \cdots$ EOO $、 \mathrm{OOO}$ 。

三段论的格是其逻辑形状,由中项在前提中的不同位置决定。所以一共有四种不同的格,如果头脑中有一个图表或者用图标说明,就可以很清晰地记住这几个格,见图6—11:\\
\includegraphics[max width=\textwidth, center]{2025_05_15_6a28331d5e7c993ad07ag-288}

图6-11\\
\includegraphics[max width=\textwidth, center]{2025_05_15_6a28331d5e7c993ad07ag-288(1)}

从图中可见:

\begin{itemize}
  \item 第一格的中项是大前提的主项、小前提的谓项;
  \item 第二格的中项在两个前提中都做谓项;
  \item 第三格的中项在两个前提中都做主项;
  \item 第四格的中项是大前提的谓项、小前提的主项。
\end{itemize}

64 个式都可以有四个格。把两者结合起来,给定了三段论的式与格,也就唯一地确定了三段论的形式。因此,标准式直言三段论恰有 $256(64 \times 4)$个可能的形式。

这些形式中绝大部分是无效的。根据前一节阐明的三段论规则,我们可以排除那些违反了一条或几条规则的形式,剩下的就是直言三段论的有效式。 256 个形式中,只有 15 个形式不能被排除,因而它们是有效的。 ${ }^{[8]}$

为更好地掌握三段论,古典逻辑学家给每一个有效式都起了独特的名称,每一个都完全刻画了其格与式。了解有效式的这个小集合,记住每一个有效形式的名称,对于我们实际运用三段论论证是很有帮助的。这些名称都是精心设计的,每个名称都包含了三个元音,代表着被命名三段论的式(依据标准的顺序:大前提、小前提、结论)。对于同式不同格的有效三段论形式,都分别给它们指派唯一的名称。例如,对于式为 EAE 的三段论,如果是第一格的就叫做 Celarent,而如果是第二格的就叫做 Cesare。 ${ }^{[9]}$

这些名称曾经有(现在仍然有)很实用的功能:如果懂得只有式与格的某些特定组合是有效的,并且通过名字就能识别那些有效论证,那么,无论给出任何式或格的三段论,就都能立即判定其正误。例如,AOO 式只在第二格才是有效的。这个唯一的形式(AOO-2)就叫做 Baroko。 ${ }^{[10]}$熟悉 Baroko 并能很容易指认它的人就会确信这个式的其他格都是无效的,是必须拒斥的。

古典逻辑学家很细致地研究了这些形式,谙熟它们的结构和逻辑"感

应"。这种精心设计好的逻辑系统,会使得一个人在言语或文本中碰到三段论论证时,能立即确切指认哪些是有效的,哪些是无效的。许多世纪以来,逻辑训练的一种常用方式,就是通过给出三段论有效形式的名称,来为三段论论证的可靠性进行辩护。而在激烈的日常论辩中具备这种迅速识别有效论证与无效论证的能力,一直被视为富有学养、思维敏锐的标志。而依赖演绎论证所建立起来的论证链条之坚固也得到了充分显示。一旦完全掌握三段论理论,这种实际论辩能力就会得到富有成效的令人愉悦的提升。

三段论论证曾经有如此广泛的应用,并被普遍视为学术论证最不可缺少的工具,因此,最先系统论述三段论理论的学术大师亚里士多德,得到了人们上千年的尊崇。他关于三段论的分析的论集迄今仍沿用着一个简单但令人肃然起敬的名字:Organon,即《工具论》。

作为这个著名逻辑体系的初学者,我们对三段论的掌握可能难以非常精通。但列出所有有效三段论形式并加以熟练掌握,无疑是最为有用的。 15 个有效三段论形式(布尔解释下的)可以根据格的不同分为四组:在前三个格中都有四个有效形式,第四格有三个有效形式。 ${ }^{[11]}$

下图即列出了 15 个有效形式,以及它们相应的传统名称:

\section*{标准式直言三段论的 15 个有效形式}
第一格(中项在大前提中做主项、在小前提中做谓项):\\
1.AAA-1 Barbara\\
2.EAE-1 Celarent\\
3.AlI-1 Darii\\
4.EIO-1 Ferio\\
第二格(中项在两个前提中都做谓项):\\
5.AEE-2 Camestres\\
6.EAE-2 Cesare\\
7.AOO-2 Baroko\\
8.EIO-2 Festino\\
第三格(中项在两个前提中都做主项):\\
9.All-3 Datisi\\
10.IAI-3 Disamis\\
11.EIO-3 Ferison\\
12.OAO-3 Bokardo

第四格(中项在大前提中做谓项、在小前提中做主项):\\
13.AEE-4 Camenes\\
14.IAI-4 Dimaris\\
15.EIO-4 Fresison 
\section*{6.6 直言三段论 15 个有效形式的演绎推导}
直言三段论的 15 个有效形式是从 256 个可能形式中排除无效式以后得以确立的。我们可以通过确定哪些形式违反了三段论的基本规则来实行这种排除——演绎推导出三段论的 15 个有效形式。

对逻辑初学者来说,不必一定弄清如何排除无效式的细节。但对于那些从三段论分析的复杂性中获取乐趣的人而言,这应是一种虽有难度但令人愉悦的挑战。如果只想认识和把握三段论的有效式,即 6.5 节讲到的那些内容,就可以绕过本节不看。

这种演绎推导并不那么容易理解。从事这项工作必须清晰地记住以下两点:(1)6.4 节设定的六条三段论基本规则;(2)三段论四个格的模式,即图6-11。

根据结论的不同形式,我们首先把三段论的所有可能形式分为四组。每个结论都是 A、E、I、O 四种直言命题之一,没有其他可能,据此可以分四种情形考察一个有效的三段论需要具备什么特性,即可以这样提问:如果结论是 A 命题,通过某一条或几条规则能够排除什么形式;如果结论是 E 命题可以排除什么形式,以此类推。下面我们就逐个进行考察。

\section*{情形1:如果三段论的结论是 $\mathbf{A}$ 命题}
在这种情形下,前提不可能是 E 命题,也不可能是 O 命题,因为如果前提为否定命题的话,结论就应该是否定的(规则5)。所以,两个前提必定是 A 命题或 I 命题。小前提不能是 I 命题,因为小项(结论的主项,也就是一个 A 命题的主项)在结论中是周延的,如果小前提是 I 命题,那么在前提中不周延的项在结论中周延,违反了规则 3 。两个前提,即大前提和小前提,不能是 I 和 A,因为如果是的话,有两种可能,或者是在结论中周延的项在前提中不周延,违反规则 3 ,或者是中项两次不周延,违反规则2。所以两个前提(结论是 A 命题时)必须都是 A 命题,这意味着唯一有效的形式是 AAA 式。而第二格的 AAA 式会使中项两次不周延,第三格和第四格的 AAA 式都会造成前提中不周延的项在结论中周

延的错误。所以,如果三段论的结论是 A 命题,唯一的有效形式就是第一格的 AAA 式,即 AAA-1,传统上称这个有效形式为 Barbara。

情形 1 的总结:如果三段论的结论是 $\mathbf{A}$ 命题,只能有一种有效形式:\\
AAA-1-Barbara。

\section*{情形2:如果三段论的结论是 $\mathbf{E}$ 命题}
E 命题的主项和谓项都是周延的,因此,如果结论为 E 命题,三段论前提中的三个项也都必须至少周延一次 ${ }^{(1)}$ ,这只有当前提之一也是 E 命题时才有可能。但不能两个前提都是 E 命题,因为不能允许两个否定前提 (规则 4),同理可知另一个前提也不能是 O 命题。另一个前提也不能是 I命题,否则在结论中周延的项在前提中不周延,违反规则 3 。这样,另一个前提必须是 A 命题,两个前提的组合可能是 AE 或 EA。因此,在结论是 E 命题的情况下,可能的正确形式为 AEE 和 EAE。

如果是 AEE 式,它不能是第一格,也不能是第三格。因为如果是这两个格的话,结论中周延的项在前提中不周延。所以,有效的AEE式只能是第二格的,即AEE-2(传统上称为 Camestres),或者是第四格的,即 AEE-4(传统上称为 Camenes)。如果是 EAE 式,它不能是第三格,也不能是第四格,因为那也都导致结论中周延的项在前提中不周延。所以,有效的 EAE 式只能或者是第一格的,即 EAE-1(传统上称为 Celarent),或者是第二格的,即 EAE-2(传统上称为 Cesare)。

情形 2 的总结:如果三段论的结论是 $\mathbf{E}$ 命题,只能有四种有效形式: AEE-2、AEE-4、EAE-1 和 EAE-2——分别是 Camestres、Camenes、Celar- ent 和 Cesare。

\section*{情形3:如果三段论的结论是 I命题}
在这种情形下,前提不能是 E 或 O 命题,因为如果有一个否定前提的话,结论也应该是否定的。两个前提也不能都是 A 命题,因为结论为特称的三段论其前提不能都是全称的(规则 6)。同样,两个前提也不能都是 I 命题,因为中项必须至少在一个前提中周延(规则 2 )。这样,前提的组合必须是 AI 或者 IA,因而结论为 I 命题的三段论可能的有效形式为 AII 和 IAI。

AII 在第二格和第四格中不可能有效,因为中项至少要周延一次。因

\footnotetext{(1)据规则 $2 、 3$ 。
}此保留下来的 AII 式就是 AII-1(传统上称为 Darii)和 AII-3(传统上称为 Datisi)。如果是 IAI 式,它不能是 IAI-1 和 IAI-2,因为这两个形式都违反中项至少在一个前提中周延的规则。剩下的有效形式就是 IAI-3(传统上称为 Disamis)和 IAI-4(传统上称为 Dimaris)。

情形 3 的总结:如果三段论的结论是 I 命题,只能有四种有效形式: AII-1、AII-3、IAI-3 和 IAI-4——分别是 Darii、Datisi、Disamis 和 Dima- ris。

\section*{情形4:如果结论是 $\mathbf{O}$ 命题}
在这种情形下,大前提不能是 I 命题,因为结论中周延的项在前提中也必须周延。所以大前提可能是 A 命题、 E 命题或者 O 命题。

假设大前提是 A 命题。这样,小前提就不能是 A 命题和 E 命题,因为结论为特称( O 命题)时,前提不能都是全称的。小前提也不能是 I 命题,否则,或者中项一次也不周延(违反规则 2 ),或者结论中周延的项在前提中不周延。因此,如果大前提是 A 命题,小前提必须是 O 命题,结果就是 AOO 式。但在第四格, AOO 式不可能有效,因为中项两次不周延。在第一格和第三格也不可能有效,因为结论中周延的项在前提中不周延。因此当大前提是 A 命题时, AOO 式保留下来的有效形式只有第二格 AOO-2(传统上称为 Baroko)。

再假设(如果结论是 O 命题)大前提是 E 命题。在这种情况下,小前提将不能是 E 命题或 O 命题,因为不允许两个否定前提。小前提也不能是 A 命题,因为结论如果为特称的,前提就不能是两个全称命题(规则6)。因而只剩下了 EIO 式——它在四种格中都是有效的,传统上分别叫做 Ferio(EIO-1)、Festino(EIO-2)、Ferison(EIO-3)和 Fresison ( $\mathrm{EIO}-4$ )。

最后,假设大前提是 O 命题。同样小前提也不能是 E 命题或 O 命题,因为不能允许两个否定前提。小前提也不能是 I 命题,因为那样的话,或者中项一次都不周延,或者结论中周延的项在前提中不周延。因此,如果大前提是 O 命题,小前提必须是 A 命题,即必为 OAO 式。但要排除 $\mathrm{OAO}-1$ ,因为中项两次都不周延。也要排除 OAO-2 和 OAO-4,因为这两种情况都会使结论中周延的项在前提中不周延。于是就只剩下一个有效形式 OAO-3(传统上称为 Bokardo)。

情形4的总结:如果结论是 O 命题,则有六个有效形式:AOO-2、

\section*{EIO-1、EIO-2、EIO-3、EIO-4 和 OAO-3,分别叫做 Baroko、Ferio、Festi- no、Ferison、Fresison 和 Bokardo。}
以上的分析通过排除法证明了直言三段论恰有 15 个有效形式:结论是 A 命题时有 1 个,结论是 E 命题时有 4 个,结论是 I 命题时有 4 个,而结论为 O 命题时有 6 个。这 15 个有效形式中,四个是第一格的,四个是第二格的,四个是第三格的,三个是第四格的。这样,就完成了标准式直言三段论的 15 个有效形式的演绎推导。 
\section*{第6章概要}
第6章考察标准式直言三段论:组成成分、形式、有效性和制约其正确使用的规则。

6. 1 节给出了三段论大项、小项和中项的定义:

\begin{itemize}
  \item 大项:结论的谓项
  \item 小项:结论的主项
  \item 中项:两个前提中都出现,但结论中不出现的第三个项
\end{itemize}

继而又分别定义了大前提和小前提,包含大项的前提叫做大前提,包含小项的前提叫做小前提。如果几个命题出现的次序正好是:大前提在第一位、小前提在第二位、结论在最后,我们就把这样的三段论指定为标准式的。\\
6.1 节也说明了三段论的式与格是如何确定的。

三段论的式由识别三个命题类型的字母来确定,即A、E、I、O中的三个。总共有 64 个不同式。

三段论的格由中项在前提中的不同位置来确定。对四个可能的格描述并定义如下:

第一格:中项在大前提中做主项、在小前提中做谓项。\\
模式为:$M-P, S-M$ ,所以 $S-P$ 。\\
第二格:中项在两个前提中都做谓项。\\
模式为:$P-M, S-M$ ,所以 $S-P$ 。\\
第三格:中项在两个前提中都做主项。

模式为:$M-P, M-S$ ,所以 $S-P$ 。\\
第四格:中项在大前提中做谓项、在小前提中做主项。\\
模式为:$P-M, M-S$ ,所以 $S-P$ 。\\
6.2 节说明了标准式三段论的式与格如何共同地确定其逻辑形式。由于 64 个式每一个都有四个格,所以共有 256 个标准式的直言三段论,但其中只有一小部分是有效式。

6. 3 节介绍检验三段论有效性的文恩图方法,即在几个交叉的圆中,作上恰当的标记或涂上阴影以表示前提的含义。\\
6.4 节阐明标准式三段论的六条基本规则,同时定义了违反各条规则所造成的谬误。\\
-规则 1 一个有效的标准式直言三段论必须仅仅包含三个项,在整个论证中,每一个项都须在相同的意义上使用。

违反本规则所犯的错误:四项谬误。\\
-规则 2 在一个有效的标准式直言三段论中,中项必须至少在一个前提中周延。

违反本规则所犯的错误:中项不周延谬误。\\
-规则 3 在一个有效的标准式直言三段论中,在结论中周延的项在前提中也必须周延。

违反本规则所犯的错误:大项不当周延谬误,或者小项不当周延谬误。\\
-规则 4 任何有两个否定前提的标准式三段论都不是有效的。\\
违反本规则所犯的错误:排斥前提谬误。\\
-规则 5 如果一个标准式三段论有一个前提是否定的,那么结论必须是否定的。

违反本规则所犯的错误:从否定推肯定谬误。\\
-规则 6 一个有效的标准式直言三段论,如果结论为特称命题,那么其前提不能都是全称的。

违反本规则所犯的错误:存在谬误。\\
6. 5 节给出了标准式直言三段论的 15 个有效形式的说明,识别它们的格与式,并说明了它们传统的拉丁名称:

AAA-1(Barbara)、EAE-1(Celarent)、AII-1(Darii)、EIO-1(Fe- rio)、AEE-2(Camestres)、EAE-2(Cesare)、AOO-2(Baroko)、EIO-2\\
(Festino)、AII-3(Datisi)、IAI-3(Disamis)、EIO-3(Ferison)、OAO-3 (Bokardo)、AEE-4(Camenes)、IAI-4(Dimaris)、EIO-4(Fresison)。

6. 6 节展示了 15 个有效形式的演绎推导,通过排除法程序,证明了只有 15 个形式是完全遵守三段论的六条基本规则的。 

% 第七部分
\chapter{形式系统}
\section*{舄 7 䓙}
\section*{7.1 日常语言中的三段论论证}
前几章考察的标准式直言三段论往往显得生硬、不自然。它们就像 "化学纯净物"一样,不含任何杂质和不相关的东西。但是,日常语言论证并不总是这么整齐划一地出现的。在此,我们更广义的使用三段论这一术语,用来指谓符合如下条件的任一论证:或者本来就是标准式直言三段论,或者是可以变形为标准式直言三段论而没有失掉或改变原意的论证。

三段论论证相当常见,所以我们要设法检验其有效性。但由于日常论证通常比标准形式松散,前面提到的检验方法——文恩图和直言三段论的规则一一不能直接适用于它们。日常的三段论论证形式变化多样,不可能为每一种形式都发明一个特殊的检验方法,除非有一种极度复杂的逻辑工具。要检验众多三段论论证的有效性,最明智的方法通常是:在不改变原意的前提下,把它们变形(reformulate)为标准式三段论。这个方法就是向标准形式的化归(reduction)或翻译(translation),最后得到的三段论叫做原给定三段论的标准式翻版。

评估日常语言三段论要满足两个条件。首先,要有一种便于应用的检验方法,将标准式三段论的有效式和无效式区分开来,这种方法我们已经有了(前面章节中讲到的图示和规则)。其次,要有一种翻译方法,将任何形式的三段论推理转变为标准形式,一旦掌握了这种方法,再用先前介绍的判定有效三段论的规则或文恩图解方法进行检验,我们就能评估任何三段论。

要说明将日常语言中的非标准三段论论证翻译为标准形式的方法,首先要区分非标准形式偏离标准形式的不同情形。下面是三种基本的偏离情形:

1.前提和结论的顺序不标准。这是小问题,因为如果仅仅是叙述的顺序不标准,很容易调整过来。

2.日常语言论证的构成命题中表面上包含不止三个项,但可以证明事实上并非如此。

3.日常语言论证的构成命题不都是标准式直言命题。\\
第二、三种偏离情形同样有可能翻译为标准形式,下面即讨论翻译方法。 
\input{chapter7/section7-2.tex}
\section*{7.3 直言命题的标准化}
如 7.1 节所述,日常语言中三段论论证的形式可能偏离标准形式,不仅可能出现含有三个以上词项的情况(如 7.2 节讨论的那样),还可能有

这样的情况,即构成命题不都是标准的直言命题。显然,A、E、I、O命题有些生硬,而日常生活中许多三段论都是由非标准的命题组成的。要把这些论证化归为标准形式,就要把构成命题都翻译为标准形式。但日常语言内容丰富、形式多样,根本无法找出一套完善的翻译规则。在各种情形中,最关键的是理解已知的非标准命题的含义,这样才能在翻译时不丢失,也不改变原意。

尽管没有完善的规则,我们仍然可以介绍一些方便的方法,它们在处理某些特殊命题时常常十分有用。这些方法——本节介绍九种方法——只能被看做一种指针而不是规则,也就是说,它们是处理某些特定种类的非标准命题的技巧。

1.单称命题。有些命题肯定或否定的是一个特定的个体或对象属于某个类,例如"苏格拉底是哲学家"、"这张桌子不是古董"等。这样的命题叫做单称命题。它们肯定或否定的不是一个类与另一个类的包含关系 (像标准式直言命题那样),但我们可以把单称命题解释为处理类与类间关系的命题。可以按如下方式做到这一点:

每一个个体对象都对应着一个单元类(由一个元素组成的类),其中只有一个对象。这样,断定一个对象 $s$ 属于类 $P$ ,在逻辑上等价于断定了只含有一个元素的单元集 $S$ 完全包含于类 $P$ 之中。而断定一个对象 $s$ 不属于类 $P$ ,在逻辑上等价于断定只含有一个元素的单元类 $S$ 完全排斥在类 $P$之外。通常将这种解释看做自然而然的,无须调整记法。据此,我们就可以将任何一个单称肯定命题"$s$ 是 $P$"看做逻辑上等价的 A 命题"所有 $S$是 $P$"。同样,可以简单地将单称否定命题"$s$ 不是 $P$"看做逻辑上等价的 E 命题"没有 $S$ 是 $P$"——S 指称的都是只有一个对象 $s$ 的单元类。因此,不需要对单称命题进行明确的翻译,一般把它们分别归到 $\mathrm{A} 、 \mathrm{E}$ 命题当中。康德说过"在三段论中判断之使用,逻辑学者把单称判断类如全称判断处理,是很恰当的"${ }^{[1]}$ 。

然而,情况并不那么简单。特称命题有存在含义,而全称命题没有。在布尔解释下(如5.6节说明),如果机械地把单称命题当做三段论推理的 A、E 命题;再用文恩图或三段论规则来检验其有效性,就会出现严重的困难。

很明显,在某些情况中,可以把含单称命题的有效的两前提论证转化为有效的三段论。例如:

$$
\begin{array}{ll}
\text { 所有 } H \text { 是 } M, & \text { 可以变为三段论的 Barba- } \\
\frac{s \text { 是 } H,}{\therefore s \text { 是 } M_{0}} & \text { ra, 即 AAA-1 式, 很明显 } \\
\text { 是有效的 } &
\end{array}
$$

$$
\begin{aligned}
& \text { 所有 } H \text { 是 } M, \\
& \text { 所有 } S \text { 是 } H, \\
& \therefore \text { 所有 } S \text { 是 } M \text { 。 }
\end{aligned}
$$

但在另外的某些情形下,把含单称命题的有效的两前提论证转化为三段论,却是明显无效的。例如:

$$
\begin{array}{ll}
s \text { 是 } M, & \text { 得到的直言三段论是无效 } \\
\frac{s \text { 是 } H,}{\therefore \text { 有 } H \text { 是 } M_{0}} & \text { 的 AAI-3 式 }
\end{array}
$$

后者违反了规则 6 ,犯了存在谬误。\\
再者,如果把单称命题转化为特称命题,也会有同样的困难。有些情况下转化是有效的,例如:

\begin{center}
\begin{tabular}{lll}
所有 $H$ 是 $M$, & 可以变为三段论的 Darii, & 所有 $H$ 是 $M$, \\
$\frac{s \text { 是 } H,}{\therefore s \text { 是 } M \text { 。 }}$ & 即 AII-1 式,很明显是有效的 & 有 $S$ 是 $H$, \\
$\therefore$ 有 $S$ 是 $M$ 。 &  &  \\
\end{tabular}
\end{center}

但在另一些情况中,这种翻译却会得出明显无效的直言三段论。例如:

\begin{center}
\begin{tabular}{lll}
$s$ 是 $M$, & 得到的直言三段论是无效的 & 有 $S$ 是 $M$, \\
$\frac{s \text { 是 } H,}{\therefore}$ 有 $H$ 是 $M$ 。 & 有 $S$ 是 $H$, &  \\
$\therefore$ & 有 $H$ 是 $M$ 。 &  \\
\end{tabular}
\end{center}

后者违反了规则 2 ,犯了中项不周延谬误。\\
问题来自如下事实:单称命题要比任何一个标准式命题负载更多信息。如果把"$s$ 是 $P$"当做"所有 $S$ 是 $P$",那么,就丢掉了单称命题的存在含义,实际上这里 $S$ 非空。而如果把"$s$ 是 $P$"当做"有 $S$ 是 $P$",又漏掉了单称命题的全称性,即主项周延,它说的是全部 $S$ 是 $P$ 。

解决此问题的办法,就是把单称命题分析为两个直言命题的合取,即

一个单称肯定命题等价于相互关联着的 A、I 命题的合取。这样"$s$ 是 $P$"就等价于"所有 $S$ 是 $P$"合取"有 $S$ 是 $P$",单称否定命题则等价于"没有 $S$ 是 $P$"合取"有 $S$ 不是 $P$"。图7—1就是单称命题的肯定式和否定式的文恩图。在用三段论规则评估这种推理时,必须考虑它提供的所有信息,既考虑周延性也考虑存在含义。\\
\includegraphics[max width=\textwidth, center]{2025_05_15_6a28331d5e7c993ad07ag-310}

图7—1\\
对于含有单称命题的三段论,引用文恩图或规则检验其有效性时,只要我们记住其中有存在含义,就可以直接把它们看做全称(A 或 E )命题。

2.谓项为形容词或形容词短语,而非名词或类词项的直言命题。例如"有花是美的"、"没有战船是可调用的"都是直言命题,但"美的"和 "可调用的"表示的只是属性而不是类,所以它们的形式不标准,必须转化为标准形式。不过,每个属性都可以确定一个类,即具有这种属性的事物组成的类,所以对于每个这样的命题,都有一个相应的标准式直言命题。两个例句分别对应的是:I命题"有花是美的事物"和 E 命题"没有战船是可调用的事物"。如果一个直言命题的形式是标准的,只有谓项为形容词或形容词短语时,就把形容词或短语替换为这样一个词项,它指称由所有具有形容词表示之属性的事物所组成的类。

3.主要动词不是标准的联项"是"或"不是"的直言命题。常见的例子有"所有人都寻求赞誉"、"有人饮用希腊酒"。通常,转化的方法是把主项和量项之外的所有成分看做类的定义特征。先把能被替换的成分换成这样的词项,它们指称由类定义特征所确定的类,再改用标准的联项把它们同主项联结起来。这样上面两个例子就成了:"所有人是赞誉的寻求者"、"有人是希腊酒的饮用者"。

4.标准形式的各成分都出现,却没有按标准顺序排列的陈述句。"赛马全是良种马"和"结果好的事总是好事"就是这样的例子。在这种情形

下,首先要找出哪个是主项,然后再重新把各个成分排列一下,使之成为标准式直言命题。这种翻译通常都很直接。十分清楚,上述两个例句可翻译为"所有赛马是良种马"和"所有结果好的事是好事"。

5.量词不是"所有"、"没有"和"有"这些标准语词的直言命题。以 "每一"、"任何"等开头的陈述句很好转化。"每一只狗都有其得意之时"、 "任何贡献都会得到赞赏"可分别转化为"所有狗是有其得意之时的动物"和"所有贡献是会得到赞赏的事情"。"每一事物"、"任何东西"类似于 "每一"、"任一"。与此同一系列但限于人类的是"每人"、"任何人"、"无论谁"、"不管是谁"、"那些……的人"以及"每个......的人"等等。以上各表达式都不会带来什么麻烦。

语法冠词"a"和"an"("一个"等)也可用于指代量词,必须依据当时的语境,确定它们的意思是"所有"还是"有"。例如,"A bat is a mammal"(一只蝙蝠是一个哺乳动物)与"An elephant is a pachyderm" (一头大象是一个厚皮动物)可以合理地解释为"所有蝙蝠都是哺乳动物"与"所有大象都是厚皮动物"。但"A bat flew in the window"(一只蝙蝠飞进窗户)和"An elephant escaped"(一头大象逃跑了)显然指的是 "有蝙蝠是飞进窗户的动物"和"有大象是逃跑的动物"。

冠词"the"("这"、"这些"等)既可以用于指称一个特定的个体,也可以指称一个类的全部元素,有可能引起混淆。例如"The whale is a mammal"(鲸是哺乳动物)这句话,在一般情况下都会被理解为"所有鲸都是哺乳动物",而单称命题"The first president was a military hero" (第一任总统是军旅英雄)可以说是标准形式的 A 命题(一个有存在含义的单称命题),其道理本节前面已经讨论过了。 ${ }^{[2]}$

尽管以"每一"和"任一"开头的肯定句都可以译为"所有 $S$ 是 $P$",但对于以"not every"(并非每一个)和"not any"(并非任一)开头的否定句,却有很大区别。它们的译法不那么明确,需要更加小心。比如, "Not every $S$ is $P$"意思是有 $S$ 不是 $P$ ,而"Not any $S$ is $P$"意思是没有 $S$ 是 $P$ 。

6.排斥命题(exclusive propositions)。含有"只"(only)、"只有" (none but)的直言命题通常叫做排斥命题,因为一般说来,它们断言的是谓项排他性地适用于主项。例如"只有公民能成为选民"、"只有勇敢者是值得公平对待的",第一句转化为标准形式是"所有能成为选民的是公

民",第二句转化为"所有值得公平对待的人是勇敢者"。以"只"、"只有"开头的命题一般可以按以下途径转化为 A 命题:将主、谓项互换位置,把"只有"换为"所有"。因此"只有 $S$ 是 $P$"和"只有 $S$'s 是 $P$'$s$"通常被理解为"所有 $P$ 是 $S$"。

但是,在某些语境中,"只"、"只有"被用于表达某种更多的含义。 "只有 $S$ 是 $P$"和"只有 $S$'s 是 $P$'$s$"表明的可能是"所有 $S$ 是 $P$"或者 "有 $S$ 是 $P$"。但这种情况并不常见。这个时候就需要语境的辅助了。如果没有附加信息,前面的翻译就是适当的。

7.不含量词的直言命题。例如"狗是肉食动物"、"孩子在场"。欠缺量词,句子的含义就不十分明确。只有考察它们所处的语境才能确定其含义,一般来说,考察之后就能把疑义清理掉。第一个例句"狗是肉食动物"很可能述及了所有的狗,可以转化为"所有狗都是肉食动物"。而第二个例句一般只述及某些孩子,转化为标准形式为"有孩子是在场的人"。

8.完全不像标准式直言命题但也可以有标准式翻版的命题。例如 "不是所有孩子都相信圣诞老人"、"有白色的大象"、"没有粉色的大象"以及"没有既圆又方的东西"。反思这些命题就会发现,它们在逻辑上等价于(因而可翻译为)下面的标准式命题:"有孩子不是相信圣诞老人的人"、"有大象是白色的事物"、"没有大象是粉色的事物"和"没有圆的东西是方的"。

9.除外命题(exceptive propositions)。还有一些这样的例子:"除了雇员(all except)都是合格的"、"雇员之外的人(all but)都是合格的"与"只有(alone)雇员不是合格的"。要把这样的除外命题翻译为标准形式,情况就会复杂一些,因为这种命题(与单称命题很类似)做出了两个而不是一个方面的断定。所给例子断言的不仅是所有非雇员是合格的,还断定了(在通常的语境中)没有雇员是合格的。如果把"雇员"记为 $S$ 、 "合格的人"记为 $P$ ,那么,这两个命题可以写成"所有非 $S$ 是 $P$"和 "没有 $S$ 是 $P$"。这两个命题是独立的,但联合起来就断定了 $S$ 和 $P$ 互为补类。

每个除外命题都是复合句,因此,不能转化为单一的标准式直言命题。确切地说,每一个除外命题应当翻译为一个合取式,即两个标准式直言命题的合取式。所以,上面关于合格性的三个例句都可以翻译为"所有非雇员是合格者,并且没有雇员是合格者"。

应该注意到,有些论证的有效性离不开数字或类数字(quasi-numeri- cal),但数字无法译为标准形式。这些推理本身就是非三段论的(asyllo- gistic)。因此,对它们进行分析就需要一种比直言三段论复杂一些的理论。当然,有些含有类数字量词的推理也可以用三段论分析。"几乎所有"、"并非全部"、"除少数几个之外都"、"几乎每个人"等就是这样的词。如果一个命题含有看起来像量词的词项,那么就可以处理为刚刚讲过的除外命题。下面几个除外命题都含有类数字:"几乎所有学生都参加了舞会"、"并非所有学生都参加了舞会"、"除少数几个之外,学生们都参加了舞会"和"只有一些学生参加了舞会",它们都肯定了有些学生参加了舞会,同时又否定了所有学生都参加了舞会。从三段论推论的观点看,它们给出的类数字信息并不相干,转化之后都是"有学生是参加了舞会的人,并且有学生不是参加了舞会的人"。

由于除外命题不是直言命题,而是合取式,含有这些命题的论证并不是我们所说的三段论论证。但是,对它们进行三段论分析和评估也未尝不可。含有除外命题的论证,要依据该命题所处的位置来进行检验。如果它是前提,那么就要分两次进行检验。举例来说,看下面这个论证:

\begin{displayquote}
每个看过比赛的人都参加了舞会,\\
不是全体学生都参加了舞会,\\
所以,有学生没有看过比赛。
\end{displayquote}

其中,第一个前提以及结论都是直言命题,很容易译为标准形式。但第二个前提是一个除外命题,不是简单句而是复合句。要检查前提是否蕴涵结论,首先要检验由论证的第一个前提、第二个前提的前一半以及结论组成的三段论。我们有:

\begin{displayquote}
所有看过比赛的人都是参加了舞会的人,\\
有学生是参加了舞会的人,\\
所以,有学生不是看过比赛的人。
\end{displayquote}

这个标准式的直言三段论是 AIO-2,违反了规则 2,犯了中项不周延的谬

误。但不能由此就得出结论说原来的论证是无效的,因为受检验的三段论只包含它的一部分前提。现在再来检验由第一个前提、第二个前提的后一半以及结论组成的三段论。译为标准形式后,得到一个非常不同的三段论:

所有看过比赛的人都是参加了舞会的人,\\
有学生不是参加了舞会的人,\\
所以,有学生不是看过比赛的人。

这是一个标准的 Baroko,即三段论的 AOO-2。很容易看出它是有效的。原来的三段论与这个有效式的结论相同,并且前者的前提包含着后者的前提,所以原来的论证也是有效的。因此,如果一个论证中有一个前提是除外命题,那么,对其有效性的检验要分为两次,即分别对两个不同的标准式直言三段论进行检验。

如果前提都是直言命题,但结论是除外命题,那么我们就可断言它是无效的。尽管两个直言命题可以蕴涵其中一个,即蕴涵结论复合句的一半,但不可能同时蕴涵两个。最后,如果两个前提和结论都是除外命题的话,那么,由原来论证所能建构的任何一个可能的三段论都要接受检验,才能确定其有效性。以上解释已足够处理这种情况了。

学会将多种非标准命题翻译为标准形式的技巧是很重要的,因为我们已经掌握的检验方法——文恩图解和三段论规则——只能直接用于标准式直言三段论。 
\input{chapter7/section7-4.tex}
\input{chapter7/section7-5.tex}
\section*{7.6 连锁三段论(Sorites)}
有时会出现一个三段论论证,其前提多于两个。如果其结论是由前提依次推得的,那么它就是有效的,否则就是无效的。例如下面这个论证,它的前提有四个:

所有外交官都是机敏的人,\\
有外交官是欠思考的人,\\
所有欠思考的人都是轻率的,\\
没有轻率的人是谨慎的,\\
所以,有谨慎的人不是机敏的。

这个论证可以通过一系列环环相扣的直言三段论来进行检验。如果能把这个链条上的所有三段论都写出来,那么,任何一个违反了三段论六条规则的三段论都会使整个推理无效。

上述论证中的结论("有谨慎的人不是机敏的")可以由前提"没有轻率的人是谨慎的"和一个未出现的命题共同推出,这个未出现的命题就是"有轻率的人是机敏的"。这个未出现的命题,正是前面三个前提的结论。这样,我们就可以从一个论证推出另一个来。第一个论证是:

所有外交官都是机敏的人,\\
有外交官是欠思考的人,\\
所有欠思考的人都是轻率的,\\
所以,有轻率的人是机敏的。

而第二个论证的两个前提是:第一个论证的结论,以及原论证的第四个前提("没有轻率的人是谨慎的")。第二个论证是:

有轻率的人是机敏的,\\
没有轻率的人是谨慎的,\\
所以,有谨慎的人不是机敏的。

这样,就可以分别检验这两个三段论了。如果两个都有效,原论证就有效。由于第二个论证(结论是"有谨慎的人不是机敏的")的前提中,"轻率的"是中项,但两个前提中都没有出现"机敏的人"(大项)和"谨慎的人"(小项),所以,它并不符合标准形式。第二个论证的大前提(其中有大项)是"有轻率的人是机敏的",小前提是"没有轻率的人是谨慎的"。

这样,第二个论证的形式就是 IEO-3。这个形式违反了规则 3,因为大项在结论中周延而在前提中不周延,因此犯了大项不当周延谬误。这样,原论证的第二个环节无效,就使得整个论证无效。

这种包含几个前提和若干结论的三段论,如果每一个结论都成为下一个三段论的前提,就称为连锁三段论(sorites)。如果这些前提都是以标准形式排列,也就是说,每个词项(除了第一个前提的主项和最后一个前提的谓项)都分别作为前提的主项和谓项出现,这样的连锁三段论就可以看做是标准式的。如下例所示:

所有 $A$ 是 $B$ ,\\
所有 $B$ 是 $C$ ,\\
所有 $C$ 是 $D$ ,\\
没有 $D$ 是 $E$ ,\\
所以,没有 $A$ 是 $E$ 。

任何一个标准形式的连锁三段论都可以通过依次进行的三段论推论而得到检验。例如上面的连锁三段论,就可以通过如下三个三段论进行检验:

(1)所有 $B$ 是 $C$ ,\\
所有 $A$ 是 $B$ ,\\
所以,所有 $A$ 是 $C$ 。

(2)所有 $C$ 是 $D$ ,\\
所有 $A$ 是 $C$ ,(前一个三段论的结论)\\
所以,所有 $A$ 是 $D$ 。

(3)没有 $D$ 是 $E$ ,\\
所有 $A$ 是 $D$ ,(前一个三段论的结论)\\
所以,没有 $A$ 是 $E$ 。

这里所有的三段论都是第一格的。第一个和第二个是 Barbara 式,第三个是 Celarent 式,它们都是有效的。因此,原连锁三段论是有效的。

一个连锁三段论的前提可以写成任何顺序,为了检验其有效性,需要先把它们整理为标准顺序。一个标准式连锁三段论的有效性(或无效性)取决于构成它的所有三段论的有效性(或无效性)。 
\input{chapter7/section7-7.tex}
\input{chapter7/section7-8.tex}
\section{第7章概要}
本章考察日常语言中的三段论论证。我们看到标准式三段论的理论如何应用于这些论证。

7.1 节指出,日常语言论证很少以标准形式出现。要把它们翻译为标准形式,需要理解它们的含义。

7.2 节解释如何把一个表面上有三至六个词项的论证归约为只有三个词项的标准式三段论。这需要(1)去除同义词,以及(2)对某些词项换质以处理其补类。

7.3 节提出九种有用的方法,用以处理那些构成命题不是标准式的三段论论证。

1.单称命题,如"苏格拉底是哲学家",可当做全称命题(A 或 E)对待。

2.如果命题的谓项是形容词或形容词短语,可把它们替换为指称相应类的词项。

3.如果命题的主要动词不是标准联项"是"或"不是",可把动词及其他词语(主项与量项之外)看做类的定义特征,从而把原命题改写为标准式。

4.如果命题的各成分虽已出现但顺序不标准,找出主项,重新调整各成分的顺序。

5.处理非标准量词时,通常要把它们替换为"所有"、"没有"或"有"。要把"并非每个……"翻译为"有……不是……"。

6.排斥命题,如"只有公民是选民",一般要通过颠倒主、谓项位置翻译为 A 命题。结论通常是"所有选民是公民"。

7.不含量词的命题,要依据语境把量词"所有"或"有"补充完整。

8.有些命题的表达形式完全不像标准式直言命题,但其逻辑上等价的直言命题可以明确地表述出来。

9.除外命题,如"除了雇员之外所有人都合格",不是简单的直言命题,而要翻译为两个标准式直言命题的合取。

7.4 节说明并举例解释了协同翻译的方法,即有时需要把同一个辅助词项(参项)引入三段论的三个构成命题中,以便把整个论证翻译为标准形式。

7.5 节考察省略三段论,即前提或结论未明确表述出来的三段论。我们看到,在对省略三段论进行检验之前,如何发现并明确表述出未出现的命题。

7.6 节解释并举例说明连锁三段论。其中有些包含三个以上前提,有些则包含一系列相互关联的三段论。

7.7 节解释了析取三段论和假言三段论,指出了它们的有效形式和可能产生的谬误。

7.8 节讨论二难推论的各种形式,并说明了反驳二难推论的三种方法:抓住联言前提的虚假性、抓住析取前提的虚假性以及构造反二难推论。 

% 第八章
\chapter{模态逻辑}
\section*{藛8密}
\section*{符号逻辑}
\section*{8.1 现代逻辑的符号语言}
我们一直在寻求对演绎论证进行分析和评估的技术。演绎理论旨在提供这样的技术,它已经发展出两个不同的分支来做这件工作:此前三章所考察的是经典逻辑或亚里士多德型逻辑,本章和下两章的主题则是现代符号逻辑。

然而,论证的分析和评估经常因其表述语言的特性(如英语或任何其他自然语言的特性)而非常困难。自然语言使用的语词可能是模糊的或歧义的,论证的结构可能是含混的,比喻和习语可能会引起混淆或误导,诉诸情感可能会引起混乱等,这些问题在第一部分已经探讨过了。要避免这些困难就要直接进人论证的逻辑核心,为此逻辑学家们构造了一种能避免自然语言缺陷的人工符号语言。使用这种符号语言能精确地表述论证。

符号也能便利我们对论证的思考。"由于符号系统之助,"一位杰出的现代逻辑学家写道,"我们几乎用眼睛就可以机械地进行推理转换,否则,这种转换本来要求大脑有很高的智能。"${ }^{[1]}$ 这似乎有点悖谬,但符号语言确实可以帮助我们不需大伤脑筋就能完成某些智力活动。

古代的和古典的逻辑学家们也承认某种特殊逻辑记号的价值。亚里士多德在自己的分析中就使用了变项,而如前面几章所表明,改进了的亚里士多德型逻辑也以很复杂的方式使用了符号。 ${ }^{[2]} 20$ 世纪又有很大的改进。

在现代逻辑中,处于核心地位的不是三段论(如亚里士多德传统上的),而是逻辑联结词,它们是每个演绎论证,不管是不是三段论,在其构成要素之间的关系中所必须运用的。命题和论证的内在结构是现代逻辑关注的焦点。要理解这种结构,我们必须首先掌握现代逻辑分析中所使用的一些特殊符号。

现代符号逻辑不受演绎论证要转换成三段论形式的制约(亚里士多德型逻辑受这种制约)。正如我们在第 7 章所见,那种工作是很费力的。不必进行这种转换使得我们可以更直接地追求演绎分析的目标。下面给出的现代逻辑的符号记法是分析论证的特别有力的工具。使用这种记法我们可以更全面地达到演绎逻辑的核心目标:区分有效论证和无效论证。  
\section{合取、否定和析取符号}

\begin{quotation}
本节介绍符号逻辑中最基本的三种逻辑联结词:合取、否定和析取,以及它们的符号表示和真值定义。通过掌握这些基本符号和标点符号的使用,我们能够将复杂的自然语言论证转换为明确的符号形式,消除歧义,并为后续的形式化分析做准备。
\end{quotation}

在本章,我们将关注一些如下述例子般简单的论证:

那个盲囚戴红帽子或者那个盲囚戴白帽子。\\
那个盲囚没戴红帽子。\\
因此,那个盲囚戴白帽子。

以及

如果鲁宾逊先生是那个司闸员的邻居,那么鲁宾逊先生住在底特律和芝加哥之间。

鲁宾逊先生不住在底特律和芝加哥之间。\\
因此,鲁宾逊先生不是那个司闸员的邻居。

这种类型的论证都至少包含一个复合陈述。研究这样的论证时,我们把所有陈述分为两个大类,即\textbf{简单的}和\textbf{复合的}。一个简单陈述就是一个不包含任何其他陈述作为其分支的陈述。臂如,"查理是整洁的"就是一个简单陈述。一个复合陈述就是包含另外一个陈述作为其分支的陈述。譬如,"查理是整洁的并且查理是可爱的"就是一个复合陈述,因为它包含两个简单陈述作为其分支。当然,一个复合陈述的分支陈述自身也可以是复合的。 ${ }^{[3]}$

\subsection{合取}
复合陈述有几种不同类型,每种都需要有其逻辑记法。第一种复合陈述是\textbf{合取}。通过在两个陈述之间使用语词 and("和"、"并且"),可以形成它们的合取;被如此联结的两个陈述叫\textbf{合取支}。因此,复合陈述"查理是整洁的并且查理是可爱的"就是一个合取,它的第一个合取支是"查理是整洁的",第二个合取支是"查理是可爱的"。

语词"和"是个简短且便利的词,但除了联结陈述外,它还有其他一些用法。臂如,陈述"林肯和格兰特是同时代人"不是一个合取,而是一

个表达关系的简单陈述。为了有一个其唯一功能是合取地联结陈述的独特符号,我们引人圆点"•"作为合取符号。于是,前述合取可以写成"查理是整洁的-查理是可爱的"。更一般的,如果 $p$ 和 $q$ 代表任意两个陈述,它们的合取就写为 $p \cdot q$ 。

我们知道每个陈述是或真或假的。因此我们说,每个陈述都有一个\textbf{真值},一个真陈述的真值是真,一个假陈述的真值是假。用这种"真值"概念,按照一个复合陈述的真值是完全由它的分支陈述的真值确定,还是由它的分支陈述的真值以外的任何其他东西确定,可以把复合陈述分成两个不同的种类。

我们把这种区分运用到合取上。两个陈述的合取的真值完全地由它的两个合取支的真值确定。如果它的两个合取支都是真的,该合取就是真的;否则它就是假的。基于这个原因,我们说合取是\textbf{真值函项复合陈述},其合取支是它的\textbf{真值函项分支}。

然而,并非所有复合陈述都是真值函项的。例如,复合陈述"奥赛罗相信苔丝德蒙娜爱卡西奥"的真值,无论如何都不是由作为它的分支的简单陈述"苔丝德蒙娜爱卡西奥"的真值确定的,因为不管苔丝德蒙娜是否爱卡西奥,奥赛罗相信苔丝德蒙娜爱卡西奥仍然可以是真的。因此,"苔丝德蒙娜爱卡西奧"不是陈述"奥赛罗相信苔丝德蒙娜爱卡西奧"的真值函项分支,该陈述自身也不是一个真值函项复合陈述。

为当前目的起见,如果一个复合陈述中的某个分支被任何有相同真值但互相区别的陈述替换,其所得不同复合陈述相互之间有相同的真值,那么,我们就把这个复合陈述的分支定义为它的一个真值函项分支。这样,如果一个复合陈述的所有分支都是它的真值函项分支,我们就可以把该复合陈述定义为一个真值函项复合陈述。 ${ }^{[4]}$

我们将只关注真值函项复合陈述。因此,在本书的余下部分,我们将用术语简单陈述指称不是真值函项复合陈述的任何陈述。

一个合取就是一个真值函项复合陈述,因此,我们的圆点符号就是一个真值联结词。已知任何两个陈述 $p$ 和 $q$ ,它们只有四种可能的真值组合。这四种可能情形及每种情形下该合取的真值可以排列如下:

如果 $p$ 为真且 $q$ 为真,那么 $p \cdot q$ 为真。如果 $p$ 为真且 $q$ 为假,那么 $p \cdot q$ 为假。

如果 $p$ 为假且 $q$ 为真,那么 $p \cdot q$ 为假。\\
如果 $p$ 为假且 $q$ 为假,那么 $p \cdot q$ 为假。

如果我们分别用大写字母 $\mathbf{T}$ 和 $\mathbf{F}$ 代表真值"真"和"假",那么,一个合取的真值由其合取支的真值确定的情形,可以用"真值表"的方式更简明地刻画如下:

\begin{center}
\begin{tabular}{|ccc|}
\hline
$p$ & $q$ & $p \cdot q$ \\
\hline
T & T & T \\
T & F & F \\
F & T & F \\
F & F & F \\
\hline
\end{tabular}
\end{center}

该真值表可看做是圆点符号的定义,因为它表明了在每种可能情形下, 303 $p \cdot q$ 所拥有的真值。

我们将发现用大写字母缩写简单陈述很方便。为此,我们一般用一个有助于我们记住它所缩写的那个陈述的字母。于是,我们把"Charlie's neat and Charlie's sweet"(查理是整洁的并且查理是可爱的)缩写为 $N \cdot$ $S$ 。(1)在自然语言中,通过在两个谓项之间加"和"而不重复主项,可以使得合取支有相同主项的那些合取更简明甚或更自然。譬如,"拜伦是一个伟大的诗人并且拜伦是一个伟大的冒险家"就可以写成"拜伦是一个伟大的诗人和伟大的冒险家"。我们把后者看做和前者一样表示了同样的陈述,并且把它们无差别地符号化为 $P \cdot A$ 。同样,在自然语言中,如果一个合取的所有合取支都有相同的谓项,该合取通常被写成在两个主项之间加"和"而不重复谓项。例如,"刘易斯是一个著名的探险家并且克拉克是一个著名的探险家"可以写成"刘易斯和克拉克是著名的探险家"。这两种表述中的任何一个都可以符号化为 $L \cdot C$ 。

正如圆点号的真值表定义所表明的,一个合取是真的,当且仅当,它的合取支都是真的。但语词 and("和"、"并且")还有另外一种用法,其指谓的不只是(真值函项)陈述,还有"随之而来"的意味,即时序关联。例如,陈述"琼斯从纽约进入该国并且直接赶往芝加哥"是有意义的且可能是真的,而陈述"琼斯直接赶往芝加哥且从纽约进入该国"则几乎

\footnotetext{(1)在汉语中可采用汉语拼音首位字母的方式。
}不可理解。"他脱了鞋并且上了床"和"他上了床并且脱了鞋"之间也有很大的区别。 ${ }^{[5]}$ 对这样例子的更深人的把握,就需要一个不同于真值函项联结词用法的特殊符号。

请注意,自然语言语词"但是"、"还"、"也"、"仍然"、"尽管"、"然而"、"此外"、"虽然如此"等,甚至逗号和分号都可以用来把两个陈述联结成一个复合陈述,在合取的意义上来说,它们都可以用圆点符号表示。

\subsection{否定}
在自然语言中,一个陈述的否定(或拒斥、否认)的形成通常是在原陈述前加一个"并非"。或者可以通过给一个陈述加一个前(后)缀"这是假的"或"事情并非如此",来表达该陈述的否定。通常用符号"~" (叫做"波浪号"或"波形号")来表示一个陈述的否定。例如,若用符号 M 表示陈述"所有人都是有死的",则陈述"并非所有人都是有死的"、 "有的人不是有死的"、"所有人都是有死的是假的",以及"情况并非是所 804 有人都是有死的"等都可以无差别地符号化为 $\sim M$ 。更一般的,如果 $p$ 是一任意陈述,则它的否定可写为 $\sim p$ 。显然,波浪号是一个真值函项算子。任何真陈述的否定都是假的,任何假陈述的否定都是真的。这一事实可以用真值表简明地刻画如下:

\begin{center}
\begin{tabular}{|cc|}
\hline
$P$ & $\sim P$ \\
\hline
T & F \\
F & T \\
\hline
\end{tabular}
\end{center}

这个真值表可以看做是否定符号"~"的定义。

\subsection{析取}
在自然语言中,两个陈述的析取(或选言)是通过在它们中间插入语词"或"形成的。如此结合的两个分支陈述叫"析取支"(或"选声支")。

自然语言语词"或"很模糊,它有两个相关但可区分的含义。其中一个含义可以用陈述"保险金会因生病或失业而被取消"为例来说明。这里的含义显然是,不仅生病的人和失业的人没有保险金,而且那些既生病又失业的人也没有保险金。"或"的这种含义叫做\textbf{弱的}或\textbf{相容的}含义。当某一个析取支为其或者两个析取支都为真时,该相容析取式是真的;仅当两个析取支均为假时,这两个析取支构成的相容析取式是假的。相容意义上的"或"有"两者之一,可能两者都"之意。保险单里的这种精确含义与

合同和其他法律文本中的一样,可以用词组"和/或"给予明晰表达。\\
语词"或"也可以用做强的或不相容的含义,此时其含义不是"至少一个",而是"至少一个且至多一个"。如果餐馆的菜单上列有"沙拉或甜点",很清楚,它的意思是说,根据所标的就餐价格,就餐者可以点一种或另外一种,但不能两者都点。在保险单里要表达"或"的不相容的精确含义,通常要加上词组"二者不可得兼"。

我们把两个陈述的相容析取解释为断言至少其中有一个是真的,把它们的不相容析取解释为,断言至少其中有一个为真,但并非两者都为真。注意,这两种析取的含义有一部分是共同的。这部分共同含义——至少有一个析取支为真——是相容的"或"的全部含义,是不相容的"或"的含义的一部分。

尽管在现代自然语言中析取的表述很模糊,但在拉丁文中并不模糊。对应于上述"或"的两种不同含义,拉丁文有两个不同的语词。拉丁语词 vel 指谓弱的或相容的析取,aut 对应强的或不相容意义上的语词"或"。习惯上用 vel 的第一个字母来代表弱的、相容意义上的"或"。如果 $p$ 和 $q$是任意两个陈述,它们的弱的或相容的析取写为 $p \vee q$ 。相容析取符号 (叫"楔劈号",有时也叫做"$\vee$ 形号")也是一个真值函项联结词。一个弱析取为假,仅当它的两个析取支均为假。我们可以用真值表把楔劈号定义如下:

\begin{center}
\begin{tabular}{|ccc|}
\hline
$p$ & $q$ & $p \vee q$ \\
\hline
T & T & T \\
T & F & T \\
F & T & T \\
F & F & F \\
\hline
\end{tabular}
\end{center}

本节所举的第一个样本论证就是一个析取三段论 ${ }^{[6]}$ :

那个盲囚戴红帽子或者那个盲囚戴白帽子。\\
那个盲囚没戴红帽子。\\
因此,那个杗囚戴白帽子。

其形式特征可以描述为:第一个前提是一个析取;第二个前提是第一个前提的第一个析取支的否定;结论与第一个前提的第二个析取支一样。很显

然,无论对语词"或"作何种解释,即不管是相容析取还是不相容析取,如此定义的析取三段论都是有效的。 ${ }^{[7]}$ 既然像析取三段论这样的以析取为前提的典型有效论证,无论对语词"或"作何种解释都是有效的,那么,我们可以简单地把语词"或"翻译为逻辑符号"$V$",而不管语词"或"采取何种含义。一般的,只有通过对上下文进行严格考察,或明确追问说话者或写作者,才能发现其采取的是何种含义。如果我们约定把语词 "或"的任意一次出现都当做相容的,那么,这个通常难以解决的问题就可以避免。另一方面,如果通过附加词组"二者不可得兼"的方式,明确地表达了是不相容析取,那么,正如即将见到的,我们有符号方法来描述这种附加意义。

在自然语言中,当两个析取支有同样的主项或谓项时,用"或"来压缩它们的析取表述,而不必重复这两个析取支的公共部分,这是很自然的。例如,"或者史密斯是所有者或者史密斯是管理者"可以同等好地表述为"史密斯或是所有者或是管理者",并且两者中的任何一个都可以合适地符号化为 $O \vee M$ 。"或者瑞德有罪或者巴奇有罪"通常被陈述为"瑞德或者巴奇有罪",它们都可以符号化为 $R \vee B$ 。

语词"除非"(unless)通常用来形成两个陈述的析取。例如,"除非你努力学习,否则你考不好"可正确地符号化为 $P \vee S$ 。原因在于我们用 "除非"意指,如果一个命题不是真的,则另一个会是真的。上面的例子可以理解为"如果你不努力学习,你就会考不好"——这正是析取的要义,因为它断言其中一个析取支是真的,由此,如果其中一个是假的,则另外一个必定是真的。当然,你也可能努力学习了但考得不好。

然而语词"除非"有时也被用来传达比这更多的信息。它的意思可以是:一个或另一个命题是真的但并非两者都是真的。也就是说,"除非"意指不相容析取。例如,杰里米•边沁(Jeremy Bentham)写道:"政治上好的东西不可能在道德上是坏的,除非对大数目来说是好的算术规则,对小数目来说不好。"${ }^{[8]}$ 在此,作者的意思确实是说,两个析取支中至少有一个是真的,但他显然也暗示它们不能两者都真。"除非"的其他用法有点含混。当我们说,"野餐将举行,除非下雨"(或者,"除非下雨,野餐将举行"),我们的意思当然是,如果不下雨,将举行野餐。但我们是否有如果下雨就不举行野餐这样的意思呢?这是不清楚的。在这里和其他地方一样,把每个析取当成弱的或相容的是明智的做法;"除非"最好简单

地用楔劈号(V)来符号化。

\subsection{标点符号}
在自然语言中,要使复杂陈述意义明确,标点符号是必需的。若没有大量不同的标点符号的使用,许多句子就会非常含混。譬如,给"The teacher says John is a fool"加不同的标点符号,它就会有很不相同的含义。 ${ }^{(1)}$ 有些语句加上标点才可以理解,如"Jill where Jack had had had had had had had had had had had the teacher's approval"。在数学中,标点符号也同样必要。在没有特别约定的情况下, $2 \times 3+5$ 不能确定指称某个特定的数,而在使用标点清楚地表明其成分如何组合的情形下,$(2 \times 3)+5$ 指称 $11,2 \times(3+5)$ 指称 16 。为了避免歧义和使意义明确,数学中的分组符号以圆括号()、方括号[]和大括号 \textbackslash {\} 等形式出现。 () 用来组合基本符号,[]用来组合包含圆括号的表达式,\textbackslash {\} 用来组合包含方括号的表达式。

在符号逻辑语言中,分组标点符号——圆括号、方括号、大括号——也是同样基本的。因为在逻辑中,复合陈述自身通常复合成一些更复杂的陈述。例如,$p \cdot q \vee r$ 是含混的:它可能意指 $p$ 与 $q$ 和 $r$ 的析取的合取,或者意指这样一个析取,其第一个析取支是 $p$ 和 $q$ 的合取,第二个析取支是 $r$ 。通过把公式加标点为 $p \cdot(q \vee r)$ 或 $(p \cdot q) \vee r$ ,我们可以区分这两种不同含义。不同标点方式所产生的差别,可以通过考察 $p$ 为假,$q$ 和 $r$都为真的情形看出。在这种情形中,第二个加标点的公式是真的(因为它的第二个析取支是真的),而第一个公式是假的(因为它的第一个合取支是假的)。在此,标点的不同导致了真和假的区别,因为不同的加标点方式会对含混的 $p \cdot q \vee r$ 赋不同的真值。

语词"either"(或者)在英语中有很多不同的意义和用法。在语句 "There is danger on either side"(两边都有危险)中,它有合取的力量。但它更常用来引入析取式的第一个析取支,如"Either the blind prisoner has a red hat or the blind prisoner has a white hat"(或者那个盲囚戴红帽子或者那个盲囚戴白帽子)。在此,它有助于语句修辞上的平衡,但并不影响语句的意义。"either"最重要的用法其实是给复合陈述加标点。例如,语句"The organization will meet on Thursday and Anand will be

\footnotetext{(1)此句可分别标点为:"The teacher says,John is a fool"(那个教师说约翰是優瓜)和 "The teacher,says John,is a fool"(约輸说那个教师是便瓜)。
}
elected or the election will be postponed"(那个组织星期四将开会并且安纳德会当选或者选举被推迟)是有歧义的,可以通过把"either"放在该语句的开头,或者把它插在名字"Anand"之前以消除歧义。在符号语言中,这种加标点的作用是通过加括号的方式实现的。前一段所讨论的那个含混公式 $p \cdot q \vee r$ 恰与刚才所考察的这个含混语句相对应。该公式的两种不同的加标点方式可与这个语句的两种不同的加标点方式相对应,而该语句的两种加标点方式是通过"either"的两种不同插人实现的。

析取的否定通常是用词组"不一也不"形成的。因此,陈述"或者费尔莫尔或者哈定是最伟大的美国总统"与陈述"费尔莫尔不是最伟大的美国总统,哈定也不是"矛盾。这个析取陈述可以符号化为 $F \vee H$ ,其否定或者是 $\sim(F \vee H)$ ,或者是 $(\sim F) \cdot(\sim H)$ 。(这两个符号公式的逻辑等价将在 8.5 节讨论。)应该清楚的是,否定断言两个陈述至少一真的析取式,要求把两个析取支都断言为假。

语词"两者都"在逻辑标点上扮演着重要角色,值得给予仔细的关注。正如上面所提到的,当我们说"杰玛和德勒克两者都不……"时,我们是说"杰玛不……德勒克也不……";我们是对他们每一个都进行否定。但当我们说"杰玛和德勒克并非两者都……"时,说的却是某件非常不同的事,我们是在对他们共同组成的对子进行否定,说的是"他们两者都……情况并非如此"。这种差别是非常根本的。在日常句子中,当"两者都"放在不同的地方时,会产生完全不同的意义。考虑下面语句意义的重要差别:

杰玛和德勒克不会两者都当选。\\
杰玛和德勒克两者都不会当选。

第一个语句否定的是合取 $J \cdot D$ ,可以符号化为 $\sim(J \cdot D)$ 。第二个语句是说他们中的每一个都不会当选,可以符号化为 $\sim(J) \cdot \sim(D)$ 。只需改变两个语词"两者都"和"不"的位置就改变了所断言的东西的逻辑力量。

当然,"两者都"并不总是扮演这种角色;有时只用它来增强语气。我们说"刘易斯和克拉克两者都是伟大的探险家",只是以之更强调地陈述"刘易斯和克拉克是伟大的探险家"所言说的东西。但在进行逻辑分析时,必须非常小心地确定"两者都"的标点符号作用。

为简化起见,即为了减少所需的括号数量,作如下约定是很便利的:在任意公式中,否定符号将被理解为施加于标点符号所管辖的最小陈述。没有这种约定,公式 $\sim p \vee q$ 是含混的,它意谓 $(\sim p) \vee q$ ,或者 $\sim(p \vee$ $q)$ 。但采用上述约定,其意指的就是备选者中的第一个,波浪号只能(根据约定)施加于第一个分支 $p$ ,而不是更大的公式 $p \vee q$ 。

为符号语言建立一套标点符号,不仅可以用来表述合取、否定和弱析取,而且也能够表述不相容析取。 $p$ 和 $q$ 的不相容析取式,断言它们当中至少有一个是真的,但并非两者都为真,可以简单地刻画为 $(p \vee q) \cdot \sim(p \cdot q)$ 。

任何仅用真值函项联结词——如圆点、波浪号和楔劈号——从简单陈述构造而成的复合陈述的真值,都完全由组成它的简单陈述的真或假确定。只要知道简单陈述的真值,它们的任何真值函项复合体的真值就很容易计算。在处理这样的复合陈述时,我们总是从它们最内部的组成分支开始,然后逐步外推。例如,设 $A$ 和 $B$ 都是真陈述且 $X$ 和 $Y$ 都是假陈述,即可计算复合陈述 $\sim[\sim(A \cdot X) \cdot(Y \vee \sim B)]$ 的真值如下:因为 $X$ 为假,故 $A \cdot X$ 为假,从而否定式 $\sim(A \cdot X)$ 为真;因为 $B$ 为真,故它的否定~ $B$ 为假,又因为 $Y$ 也为假,故 $Y$ 和 $\sim B$ 的析取 $Y \vee \sim B$ 亦为假;加方括号的公式 $[\sim(A \cdot X) \cdot(Y \vee \sim B)]$ 是一个真陈述和一个假陈述的合取,因此是假的;由此,它的否定即原整个陈述是真的。这样一种逐步程序,使得我们总能根据一个复合陈述的分支的真值来确定它的真值。

在某些情形下,即使我们不能确定一个或多个简单分支陈述的真或假,我们也能确定一个真值函项复合陈述的真值。首先通过假定某简单分支陈述为真,计算出该复合陈述的真值,然后假定该同一简单分支陈述为假,计算出该复合陈述的真值,对其真值未知的每个分支施行同样的步骤,我们就可以做到这一点。如果这些计算对被考察的复合陈述产生同样的真值,我们不必先确定它的分支的真值,就可以确定该复合陈述的真值,因为我们知道真值不是真就是假。

\begin{center}
\fbox{\parbox{0.95\textwidth}{
\textbf{本节要点}
\begin{itemize}
\item \textbf{简单陈述}与\textbf{复合陈述}的区别:
  \begin{itemize}
  \item 简单陈述不包含其他陈述作为分支
  \item 复合陈述包含其他陈述作为分支
  \end{itemize}
\item \textbf{合取}($\cdot$)的特点:
  \begin{itemize}
  \item 用圆点"$\cdot$"表示
  \item 当且仅当两个合取支都为真时,合取才为真
  \item 是真值函项复合陈述
  \end{itemize}
\item \textbf{否定}($\sim$)的特点:
  \begin{itemize}
  \item 用波浪号"$\sim$"表示
  \item 真陈述的否定为假,假陈述的否定为真
  \end{itemize}
\item \textbf{析取}($\vee$)的特点:
  \begin{itemize}
  \item 用楔劈号"$\vee$"表示
  \item 相容析取:当至少一个析取支为真时,析取为真
  \item 不相容析取:恰好一个析取支为真时,析取为真
  \end{itemize}
\item 标点符号在逻辑中的重要性:
  \begin{itemize}
  \item 消除歧义和不明确性
  \item 通过括号明确指明复合陈述的结构
  \item 允许表示复杂的逻辑关系
  \end{itemize}
\end{itemize}
}}
\end{center}  
\section*{8.3 条件陈述与实质蕴涵}
当把语词"如果"放在第一个陈述之前,把语词"那么"放在第一个和第二个陈述之间来结合两个陈述时,如此构成的复合陈述就是一个条件陈述(也叫"假言陈述"、"蕴涵"或"蕴涵陈述")。在一个条件陈述中,跟在"如果"后面的分支陈述叫前件(或"蕴涵者",偶尔也叫"前式"),跟在"那么"后面的分支陈述叫后件(或"被蕴涵者",偶尔也叫"后式")。例如,"如果琼斯先生是那个司闸员的邻居,那么琼斯先生挣的钱是那个司闸员的三倍"是一个条件陈述,其中,"琼斯先生是那个司闸员的邻居"是前件,"琼斯先生挣的钱是那个司闸员的三倍"是后件。

一个条件陈述断言在其前件为真的任何情形下,它的后件也是真的。它并不断言其前件为真,而只是断言如果其前件为真,其后件也为真。它也并不断言其后件为真,而仅仅断言它的后件会为真,如果前件为真的

话。一个条件陈述的基本含义,是断言其前后件之间的某种关系以特定次序成立。要理解一个条件陈述的含义,我们必须理解何为蕴涵关系。\\
"蕴涵"一词不止一个含义。我们已经看到,在引进一个特殊的逻辑符号来表示日常语词"或者"的某个单一含义之前,区分它的不同含义是有用的。要是我们不这样做,日常语言的含混性就会影响我们的逻辑符号系统,妨碍我们达到所欲获得的明晰性和精确性。在我们把一个特殊的逻辑符号引入这种联系中之前,区分"蕴涵"或"如果一那么"的不同含义亦同样有用。

考查下面的四个条件陈述,它们每个都断言一种不同类型的蕴涵,都对应于一种不同含义的"如果…那么":

A.如果所有人都有死且苏格拉底是人,那么苏格拉底有死。

B.如果莱士里是单身汉,那么莱士里是未婚的。\\
C.如果把这张蓝色的石蕊纸放在酸液中,那么这张蓝色的石蕊纸会变红。

D.如果斯塔德輸掉了这场比赛,那么我就吞下我的帽子。

即使随意地观察一下这四个条件陈述也会发现,它们具有非常不同的类型。 A 的后件乃由它的前件逻辑地推出,而 B 的后件是根据其前件中的术语"单身汉"的定义而得来,而"单身汉"的定义就是未婚男人。C 的后件不是仅根据逻辑或其词项的定义从其前件推出,这种联系必须经验地发现,因为这里所陈述的蕴涵是因果关系。最后,D的后件既不是根据逻辑或定义从前件推得,也没有涉及因果性定律——就这个词的通常意义来说。大多数因果性定律,臂如物理学和化学中发现的那些定律,描述的是世界发生了什么,而不管人的希望或欲求如何。当然,没有这样一种定律和陈述 D 相联系。这个陈述表述的是说话者在某种特定的情形下以特定的方式行事的决策。

可见,这四个条件陈述的不同之处,就在于每个断言了其前件和后件之间的一种不同类型的蕴涵关系。但它们并非完全不同,它们所断言的都是蕴涵的类型。那么,它们是否存在任何可识别的共同含义,即是否存在尽管可能不是其中任何一个的完整含义,但是这些公认的不同种类蕴涵所

共有的部分含义呢?\\
关于探求共同的部分含义的重要性,我们可以回想一下对日常语词 "或"进行符号刻画的过程。那时我们是如下进行的。首先,在对比相容和不相容析取时,我们强调"或"的两种含义之间的区别。我们注意到,两个陈述的相容析取的意思是说,它们当中至少一个为真。不相容析取的意思是说,它们当中至少一个为真,但不是两者都为真。其次,我们注意到这两种类型的析取有一个共同的部分含义。这个部分的共同含义,即至少有一个析取支为真,被看做是弱的、相容的"或"的整个含义,是强的、相容的"或"的含义的一部分。然后,我们引入特殊符号"V"来表达这个共同的部分含义(它是"或"的弱意义上的整个含义)。最后,我们注意到,表达共同的部分含义的符号刻画也是对语词"或"在下述意义上的合适翻译,即可以把析取三段论作为一个有效的论证形式保留下来。我们承认把不相容的"或"翻译成符号"V",忽略和丢掉了它的部分含义。但由这种翻译所保留的那个部分含义,是析取三段论继续成为一个有效论证必需的全部东西。既然析取三段论是我们这里所关注的涉及析取的典型论证,那么,语词"或"的这种部分翻译——在某些情形,可以从它的"完全的"或"全部的"含义中抽取出来——对我们目前的目的是完全合适的。

现在,我们希望以同样的方式抽取日常语言辞组"如果一那么"的含义。第一步已经完成:我们已经强调了短语"如果一那么"对应于四种不同蕴涵的四种意义之间的区别。现在准备做第二步,即发现一个至少是所有这四种不同类型的蕴涵的含义的一部分的那种意义。

要解决这个问题,可先看什么情形足以确立一个给定条件陈述的假。在什么情形下,我们会同意下面的条件陈述为假呢?

\begin{displayquote}
如果把这张蓝色的石蘂纸放进那种溶液中,那么这张蓝色的石蕊纸会变红。
\end{displayquote}

这个条件陈述并未断言任何一张蓝色的石蕊纸实际上被放进了这种溶液中,或任何一张蓝色的石蕊纸实际上变红了,认识到这一点是很重要的。它仅仅断言如果把这张蓝色的石総纸放进那种溶液中,那么这张蓝色的石蕊纸会变红。如果这张蓝色的石䓗纸实际上被放进这种溶液中,并且它没

变红,就证明该陈述是假的。可以说,当一个条件陈述的前件为真时,就获得一个关于该条件陈述的虚假性的严峻检验,因为如果它的后件为假且前件为真,该条件陈述本身就被证明为假。

对任一条件陈述"如果 $p$ 那么 $q$"来说,如果已知合取 $p \cdot \sim q$ 为真,也就是说,如果它的前件为真且后件为假,则可知该条件陈述为假。而若一个条件陈述为真,则上面所示合取式必定为假,也就是说,它的否定 $315 \sim(p \cdot \sim q)$ 必定为真。换句话说,对任何为真的条件陈述"如果 $p$ 那么 $q$"而言,它的前件和后件的否定的合取的否定,即 $\sim(p \cdot \sim q)$ ,必定也为真。据此,我们可把~$(p \cdot \sim q)$ 当做"如果 $p$ 那么 $q$"的含义的一部分。

每个条件陈述都意谓否定其前件为真且后件为假,但这不必是其整个含义。前面的 A 那样的条件陈述还断言了其前件和后件之间的一种逻辑联系,B 那样的条件陈述还断言了一种定义性联系,C 那样的条件陈述还断言了一种因果性联系,而 D 那样的条件陈述则还断言了一种决策性联系。但不管一个条件陈述断言的是何种蕴涵,它的一部分含义是对其前件和后件的否定的合取的否定。

现在,我们引进一个特殊的符号来表达短语"如果一那么"的这种共同的部分含义。通过以 $p \supset q$ 缩写 $\sim(p \cdot \sim q)$ ,我们来定义新符号"つ" (叫"马蹄号")。符号"つ"的确切含义可以用真值表方法揭示如下:

\begin{center}
\begin{tabular}{|l|l|l|l|l|l|}
\hline
$p$ & $q$ & $\sim q$ & $p^{\cdot \sim q}$ & $\sim(p \cdot \sim q)$ & $p$ つ $q$ \\
\hline
T & T & F & F & T & T \\
\hline
T & F & T & T & F & F \\
\hline
F & T & F & F & T & T \\
\hline
F & F & T & F & T & T \\
\hline
\end{tabular}
\end{center}

其中,前两列是导引列,它们只是列出 $p$ 和 $q$ 真值组合的所有可能情形。第三列据第二列得来,第四列据第一和第三列得来,第五列据第四列得来,根据定义,第六列与第五列真值相同。

符号"コ"不应被看成是指谓"如果一那么"的某种含义,或代表 (上列蕴涵类型中的)某种蕴涵关系。那是不可能的,因为没有单一的 "如果一那么"的含义,而是有几个含义。不存在该符号所刻画的单一蕴涵关系,而是有几种不同的蕴涵关系。故符号"つ"不应被看成是代表 "如果一那么"的所有含义。这些含义各不相同,用单个逻辑符号来缩写

所有这些含义的任何企图都会使符号变得含混,正如日常语言辞组"如果一那么"或"蕴涵"一样含混。符号"つ"是完全不含混的。 $p \supset q$ 缩写的就是 $\sim(p \cdot \sim q)$ ,它的含义包含在被探讨的各种蕴涵的含义之中,但它并不构成它们中任何一个的完整含义。

既然读 $p \supset q$ 的一种方便方式是"如果 $p$ 那么 $q$",我们也可以把符号 "D"看成表示了另一种蕴涵,而且这样做是很有好处的。但它不是与前面提到过的任何一种蕴涵相同的蕴涵,它被逻辑学家叫做实质蕴涵。给出这个特殊的名称,就是承认它是一个特殊概念,不应该把它和其他更常见类型的蕴涵相混淆。

日常语言中的所有条件陈述并非都必须断言前面所讨论的四种蕴涵之一。实质蕴涵实际上也是日常话语中所断言的第五种蕴涵。考虑这样一个评论:"如果希特勒是军事天才,那么我是猴子的叔叔"。很显然,它不是断言逻辑的、定义性的或因果性的蕴涵。它也不表达决策性蕴涵,因为说话者并没有能力使后件为真。这里的前后件之间没有"真正的联系",不管是逻辑的、定义性的还是因果性的。这种条件陈述经常被当做一种强调或幽默的方法来使用,它否定的是其前件,其后件通常是一个滑稽的、显然为假的陈述。既然没有任何为真的条件陈述有这样的真前件和假后件,那么,肯定这样一个条件陈述就意味着否定它的前件为真。上述条件陈述的完整含义就是,只要"我是猴子的叔叔"为假,即可否定"希特勒是军事天才"为真。既然前者明显为假,该条件陈述必被理解为否定后者。

这里的关键在于,实质蕴涵没有表明前后件之间的"实在关联",实际上,它所断言的仅仅是并非后件为假时前件为真。请注意:实质蕴涵符号像合取和析取符号一样,是真值函项联结词。它可用真值表定义如下:

\begin{center}
\begin{tabular}{|ccc|}
\hline
$p$ & $q$ & $p \supset q$ \\
\hline
T & T & T \\
T & F & F \\
F & T & T \\
F & F & T \\
\hline
\end{tabular}
\end{center}

正如这个真值表定义所表明,马蹄符"つ"有几个乍看起来很奇怪的特征:假前件实质蕴涵真后件的断言是真的;假前件实质蕴涵假后件的断言也是真的。这种表面的怪异可以由下面的探讨得到部分驱散。因为数 2 比数 4 小 (用符号表示为 $2<4$ ),可以推出任何小于 2 的数都小于 4 。条件公式:

如果 $x<2$ 那么 $x<4$\\
对任一 $x$ 都是真的。我们来看数 $1 、 3$ 和 4 ,依次以它们中的每一个代人前述条件公式的数字变项 $x$ ,可以观察到如下结果:

如果 $1<2$ 那么 $1<4$\\
在这种情形下,前后件都是真的,该条件陈述当然也是真的。\\
如果 $3<2$ 那么 $3<4$\\
在这种情形下,前件为假且后件为真,该条件陈述当然也是真的。\\
如果 $4<2$ 那么 $4<4$\\
在这种情形下,前件和后件都是假的,但该条件陈述仍然是真的。这三种情形分别对应于马蹄符"コ"的真值表定义中的第一、第三和第四行。可见,在一个条件陈述的前后件皆为真、前件为假且后件为真或前后件皆为假时,该条件陈述应该为真,这一点并不特别令人奇怪或惊讶。当然,没有小于 2 且不小于 4 的数,也就是说,没有其前件为真且后件为假的真条件陈述。这恰好是"つ"的真值表定义所表明的。

现在,我们打算把词组"如果——那么"的任何一次出现翻译成逻辑符号 "つ"。这种处理方式的意思是说,在把条件陈述翻译成符号时,我们把它们都只看做是实质蕴涵。当然,大多数条件陈述断言,在前后件之间不只实质蕴涵成立。因此,这种处理方式即意味着在把一个条件陈述翻译成符号语言时,应该忽略、撇开或"抽掉"它的部分含义。怎样辩护这种处理方式呢?

前面对用符号"$V$"来翻译相容和不相容析取这个处理方式的辩护是基于这样的理由:即使忽略附着在不相容析取"或"之上的附加含义,析取三段论的有效性也得到了保留。我们现在提议用符号"$\supset$"把所有的条件陈述仅翻译成实质蕴涵,可用完全同样的方式得到辩护。许多论证包含各种不同类型的条件陈述,但是,即便忽略这些论证的条件陈述的附加含义,我们所关注的一般类型的有效论证的有效性也都得到了保留。当然,这一点还需要证明,这是本章下一节的主题。

条件陈述可用多种不同方式表述。如下陈述:

如果他有一个好律师,那么他会被宣判无罪。

可以不用"那么"而被同样适当地表述为:

如果他有一个好律师,他会被宣判无罪。

前件和后件的表述次序可以颠倒,此时"如果"仍应在前件之前:

他会被宣判无罪,如果他有一个好律师的话。

显然,在上面所给的任何一个例子中,语词"如果"可被诸如"一旦"、318 "假如"、"倘若"或"在……条件下"等短语代替,而含义没有任何改变。经措辞调整还可把上述条件陈述表述为:

他有一个好律师蕴涵他会被宣判无罪。

或

他有一个好律师涵衍(entail)他会被宣判无罪。

从主动语态到被动语态的转换伴随着前后件次序的颠倒,可得其逻辑等价表述:

他会被宣判无罪被他有一个好律师所蕴涵(或涵衍)。

上列表述均可符号化为 $L \supset A$ 。\\
必要条件和充分条件的观念提供了条件陈述的其他一些表述形式。对任何一个特定事件来说,它的出现需要有许多必要情境。例如,一辆正常的轿车要能行使,油箱里有油,火花塞被校准,油泵能运转等都是必要条件。因此,如果该事件出现,它的出现所必需的每个条件必定都已经得到满足。据此,下述陈述:

油箱里有油是轿车行驶的一个必要条件。

可以同样适当地表述为:

轿车行使仅当它的油箱里有油。

它是如下说法的另一方式:

如果轿车行使,那么它的油箱里有油。

这些表述形式中的任何一个都可以符号化为 $R \supset F$ 。一般地说,"$q$ 是 $p$ 的必要条件"和"$p$ 仅当 $q$"可以符号化为 $p \supset q$ 。

对某特定情形而言,会有许多备选条件,它们中的任何一个都足以产生该情形。例如,就一个钱包里不止一美元来说,它里面有 101 便士、21个五分镍币、 11 个一角的硬币、 5 个两角五分钱等都是充分条件。如果获得其中的任何一个条件,那个特定的情形就会实现。因此,说"那个钱包里有 5 个两角五分钱是它里面超过一美元的充分条件",与说"如果那个钱包里有 5 个两角五分钱,那么它里面超过一美元"是一样的。一般的, "$p$ 是 $q$ 的充分条件"被符号化为 $p \supset q$ 。

如果 $p$ 是 $q$ 的一个充分条件,我们就有 $p \supset q$ ,并且 $q$ 必定是 $p$ 的一个必要条件。如果 $p$ 是 $q$ 的一个必要条件,我们就有 $q \supset p$ ,并且 $q$ 必定是 $p$ 的一个充分条件。因此,如果 $p$ 是 $q$ 的必要且充分条件,那么,$q$ 是 $p$的充分且必要条件。

并非每个含有"如果"(或类似语词)的陈述都是条件陈述。下列陈述中没有一个是条件陈述:"冰箱里有食品,如果你想吃","您的桌子准备好了,如果您乐意的话","假如感兴趣,有个消息给你","即便没得到允许,会议也会举行"。特定语词的出现与否决不是决定性的。在每种情形下,必须先理解给定语句的含义,然后用符号公式重新表述这种含义。

语词"如果"和"不确定的"之间没有必然的或逻辑的联系,尽管经常有这样一种说法:跟在语词"如果"后面的东西有点不确定。这一点可由下面的逸事所例示:

有一次,乔治•伯纳德•肖给温斯顿•丘吉尔送了两张他的新剧的首演式的票,附言"带一个朋友——如果你有的话";对

对此,丘吉尔回复说他忙于出席别的首演式,但他会很感激第二场演出的票,"如果有这样一张票的话"${ }^{[9]}$ 。 
\section{论证形式与论证}

\begin{quotation}
本节讨论如何评估论证的有效性,介绍论证形式的概念及真值表检验方法。通过学习逻辑类推反驳法和几种常见的有效论证形式,我们能够系统地判断一个论证是否在逻辑上成立,从而更准确地分析推理的正确性。
\end{quotation}

\subsection{运用逻辑类推进行反驳}
本节我们将更精确地阐明术语"有效"的含义。通过探讨运用逻辑类推进行反驳的方法,我们把形式定义与某些更熟悉的、直觉的观念联系起来。\cite{copi1980} 兹以如下论证为例:

培根是一位伟大的作家。\\
因此,培根写了那些通常归功于莎士比亚的剧本。

我们可能同意其前提但不同意其结论,从而断定该论证无效。证明无效性的方式之一就是运用逻辑类推的方法。我们可以反驳说,"你是否也可以这样来论证",

如果华盛顿是被暗杀的,那么华盛顿死了。\\
华盛顿死了。\\
因此,华盛顿是被暗杀的。\\
"但你不可能对该论证进行严格的辩护",我们可以继续说,"因为在这里,已知前提为真且结论为假,这个论证显然是无效的;而你前面的论证有同样的形式,因此,你的论证也是无效的。"这种类型的反驳是非常有效力的。

这种用逻辑类推进行反驳的方法,为获得一种检查论证的极好的一般方法指示了方向。要证明一个论证的无效性,构造另外一个这样的论证就足够了:(1)它与第一个论证有完全一样的形式;(2)它有真的前提和假的结论。这种方法建立在这样一个事实之上,即\textbf{有效性}和\textbf{无效性}是论证的纯粹形式的特征,这就是说,不管它们所探讨的题材有何差别,任何两个有完全相同形式的论证或者都是有效的或者都是无效的。\cite{jevons1886}

当用大写字母缩写给定论证中的简单陈述时,该论证就很清楚地展示了它的形式。例如,分别用 $B 、 G 、 A 、 D$ 来缩写"培根写了那些通常归功于莎士比亚的剧本"、"培根是一位伟大的作家"、"华盛顿是被暗杀的"、 "华盛顿死了",用熟悉的三点符"$\therefore$"代替"因此",我们可以把前面的两个论证分别符号化为

\begin{center}
\begin{tabular}{ll}
$B \supset G$ & $A \supset D$ \\
$G$ & $D$ \\
$\therefore B$ & $\therefore A$ \\
\end{tabular}
\end{center}

经过如此改写,它们的共同形式就很容易看清楚。\\
要讨论论证的形式而不是具有这些形式的特定论证,我们需要某种把这些论证形式本身符号化的方法。为了获得这种方法,我们引人变元的概念。在前面几节中,我们是用大写字母来符号化特定的简单陈述的。为避免混淆,我们从字母表的中间部分选取小写字母 $p, q, r, s \cdots \cdots$ 作为\textbf{陈述变元}。我们将如此使用这个术语:一个陈述变元就是这样一个字母,一个陈述可以被代入它或它所在的位置。复合陈述和简单陈述一样,也可以被代人到陈述变元中。

我们把一个\textbf{论证形式}定义为,任何这样一列包含陈述变元而不包含陈述的符号序列,当用陈述代入陈述变元时——同一陈述始终代入同一陈述变元——其结果就是一个论证。为确定性起见,我们作这样一个约定:在任何论证形式中,$p$ 是在其中出现的第一个陈述变元,$q$ 是第二个,$r$ 是第三个,等等。例如,表达式

$$
\begin{aligned}
& p \supset q \\
& q \\
& \therefore p
\end{aligned}
$$

是一个论证形式。因为当分别用陈述 $B 、 G$ 代人陈述变元 $p 、 q$ 时,其结果就是本节中的第一个论证。如果用陈述 $A$ 和 $D$ 代入陈述变元 $p 、 q$ ,其结果就是第二个论证。以陈述代入一个论证形式中的陈述变元而产生的任何论证,就叫该论证形式的一个\textbf{代入例}。显然,一个论证形式的任何代入例都可以说成具有该形式的论证,具有某种形式的任何论证都是该形式的一个代入例。

对任何论证来说,通常都有多个论证形式,它们以该给定论证作为代人例之一。例如,本节的第一个论证:

$$
\begin{aligned}
& B \supset G \\
& G \\
& \therefore B
\end{aligned}
$$

就是下列四个论证形式巾每一个的代入例:

\begin{center}
\begin{tabular}{llll}
$p \supset q$ & $p \supset q$ & $p \supset q$ & $p$ \\
$q$ & $r$ & $r$ & $q$ \\
$\therefore p$ & $\therefore p$ & $\therefore s$ & $\therefore r$ \\
\end{tabular}
\end{center}

如此,在第一个论证形式中以 $B$ 代人 $p$ ,以 $G$ 代人 $q$ ,在第二个形式中以 $B$ 代人 $p$ ,以 $G$ 代人 $q$ 和 $r$ ,在第三个论证中以 $B$ 代人 $p$ 和 $s$ ,以 $G$ 代人 $q$和 $r$ ,在第四个论证中以 $B \supset G$ 代人 $p$ ,以 $G$ 代人 $q$ ,以 $B$ 代人 $r$ ,我们都得到了上述论证。在这四个论证形式中,第一个比其他几个更紧密地对应

于给定论证的结构。这是因为,该论证是通过以不同的简单陈述,代人其中的每个不同陈述变元而获得的。我们把第一种论证形式称为该给定论证的特征形式。我们将一个论证的特征形式定义为:只要一个论证是通过一致地以不同的简单陈述代入一个论证形式中每个不同的陈述变元而产生的,该论证形式就是这个论证的特征形式。对任何给定论证来说,都有一个独特的论证形式作为该论证的特征形式。

现在,我们可以对用逻辑类推进行反驳的方法做更精确的描述。如果一个给定论证的特征形式有任意一个其前提为真且结论为假的代人例,那么,该论证就是无效的。我们可以把运用在论证形式上的术语"无效的"定义如下:一个论证形式是无效的,当且仅当,它至少有一个前提为真且结论为假的代入例。运用逻辑类推进行反驳建立在这样一个事实之上:其特征形式是一个无效的论证形式的任何论证都是一个无效论证。任何一个不是无效的论证形式必定是有效的。因此,一个论证形式是有效的,当且仅当,它没有前提为真且结论为假的代入例。由于有效性是一个形式概念,所以,一个论证有效,当且仅当,该论证的特征形式是一个有效论证形式。

如果能找到与所论证的一个反驳性类推,那么,该论证就被证明为无效,但"想出"这样的反驳性类推并非总是很容易。幸而这并不是必需的,因为对这种类型的论证来说,有一种建立在同样原则基础上的、更简单的、纯机械性的检验方法。给定任何论证,我们只需检验它的特征形式,因为特征形式的有效和无效决定了该论证的有效和无效。

\subsection{根据真值表检验论证}
要检验一个论证形式,我们可以考察它的所有可能的代人例,看它们当中是否有一个前提为真而结论为假。当然,任何一个论证形式都有无穷多个代人例,但不必担心,我们用不着逐一去考察它们。因为我们感兴趣的只是它们的前提和结论的真或假,在此只需考虑真值问题。我们这里所探讨的论证只含有简单陈述和用真值联结词联结简单陈述而构成的复合陈述,这些真值联结词可用上述圆点、波浪号、楔劈号和马蹄号符号化。因此,通过考察某些陈述真值的所有可能的不同排列组合(这些陈述被用来代人到被检验论证形式的不同陈述变元中),我们就获得了其前提和结论有不同真值的所有可能代人例。

如果一个论证形式只包含两个不同的陈述变元 $p$ 和 $q$ ,它们的所有代入例就是:或者 $p$ 和 $q$ 都代人真陈述,或者 $p$ 代人真陈述而 $q$ 代人假陈

述,或者 $p$ 代人假陈述而 $q$ 代人真陈述,或者 $p$ 和 $q$ 都代人假陈述。用真值表形式可以最方便地把这些不同的情形集结在一起。为判定下列论证形式的有效性:

$$
\begin{aligned}
& p \supset q \\
& q \\
& \therefore p
\end{aligned}
$$

可构造下列真值表:

\begin{center}
\begin{tabular}{|ccc|}
\hline
$p$ & $q$ & $p \supset q$ \\
\hline
T & T & T \\
T & F & F \\
F & T & T \\
F & F & T \\
\hline
\end{tabular}
\end{center}

这个表的每一行代表一整类代人例。两个初始栏或导引栏中的 T 和 F ,表示该论证形式中的变元 $p$ 和 $q$ 的代人陈述的真值。回过来依据初始栏或导引栏及马蹄号的定义,即可填写第三栏。第三栏的题头是该论证形式的第一个"前提",第二栏的题头是第二个"前提",第一栏的题头是"结论"。考察这个真值表,我们发现在第三行中,两个前提下都是 T ,但结论下面是 F 。这表明,上列论证形式至少有一个前提为真结论为假的代人例,这一行足以表明该论证形式是无效的。具有这种特征形式的论证(也就是说,任何以上列论证形式为特征形式的论证),被看做犯了肯定后件的谬误,因为其第二个前提肯定的是条件前提的后件。

尽管概念上很简单,真值表却是非常有力的工具。以其来判定一个论证形式的有效性和无效性时,首先且至关重要的是正确地构造真值表。要正确地构造真值表,就要为论证形式中的每个陈述变元( $p 、 q$ 和 $r$ 等)都列出一个导引栏,其排列必须展示所有这些变元的真假值的全部组合。因此,真值表的横行必须满足:如果有两个变元,就要有四行,如果有三个变元,就要有八行,如此等等。每个前提和结论都必须附有一个竖栏,而构成前提和结论的每个符号表达式也都有一竖栏。用这种方式来构造真值表,本质上是一件机械性工作;它只要求仔细地计数,小心地把 T 和 F填人合适的栏中。所有这些都受我们对几个真值函项联结词——圆点、楔劈号和马蹄号——的理解的支配,还受真值函项复合陈述为真和为假的条

件的支配。\\
一旦真值表构造完毕,完整地排列呈现在我们面前,正确地解读它,即正确地用它来评价被检验的论证形式也是很重要的。我们必须仔细观察哪些栏表达被检验论证的前提,哪一栏表达该论证的结论。例如,在检验前述无效论证时,我们观察到真值表的第二栏和第三栏是表示前提的,而结论则由第一(最左边的)栏表示。但是,根据我们检验的论证形式的不同,以及构造真值表时我们放置栏的次序,前提和结论有可能以任何次序出现在真值表顶端。它们的位置在左或在右并不重要,重要的是使用真值表时必须理解各栏表示的是什么,以及我们追寻的是什么。我们要问的是,是否存在这样一种情形,即某行中的所有前提为真而结论为假?如果有这样一行,该论证形式就是无效的;如果没有这样一行,则该论证形式必定是有效的。在完整的排列被整齐而精确地陈列出来以后,准确而细心地解读真值表是极端重要的。

\subsection{一些常见的有效论证形式}
\subsubsection{析取三段论}
析取三段论是最简单的有效论证形式之一,其依赖这样一个事实:在每个为真的析取式中,至少有一个析取支必定是真的。因此,如果其中一个析取支为假,则另一个必定为真。析取三段论可用符号表示如下:

$$
\begin{aligned}
& p \vee q \\
& \sim p \\
& \therefore q
\end{aligned}
$$

为表明它的有效性,可构造如下真值表:

\begin{center}
\begin{tabular}{|cccc|}
\hline
$p$ & $q$ & $p \vee q$ & $\sim p$ \\
\hline
T & T & T & F \\
T & F & T & F \\
F & T & T & T \\
F & F & F & T \\
\hline
\end{tabular}
\end{center}

这里,初始栏或导引栏亦展示了用来代人变元 $p$ 和 $q$ 的那些陈述的所有可能的不同真值。依据前两栏可以填上第三栏,依据第一栏可以填上第四栏。现在第三行是 T 出现在两个前提栏(第三和第四栏)的唯一一行,而在此行上 T 也出现在结论栏(第二栏)。于是该真值表表明,这个论证形

式没有前提为真而结论为假的代人例,从而证明了该被检验论证形式的有效性。\cite{jevons1879}

此处真值表的准确解读同样关键:应该细致地识别出表示结论的栏 (左起第二栏)和表示前提的栏(左起第三和第四栏)。只有正确地使用这三栏,我们才能可靠地确立被检验论证形式的有效性或无效性。请注意,相同的真值表可以用来检验一个非常不同的论证形式的有效性,该论证形式的前提由第二和第三栏表示,而结论则由第四栏表示。从该真值表的第一行可看到,这样的论证形式是无效的。

真值表技术为检验这里所讨论的任何一个一般类型的论证的有效性,提供了一种完全机械的方法。我们现在即可以之为把短语"如果一那么"的任何一次出现翻译成实质蕴涵符"コ"进行辩护。在前一节中,我们作了这样一个断言:当我们这里所涉及的"如果一那么"陈述都被解释为只断定实质蕴涵时,这种一般类型的所有有效论证仍然是有效的。可以用真值表来证实这个断言,并为我们把"如果一那么"翻译成马蹄号提供辩护。

\subsubsection{肯定前件式}
最简单的一种涉及条件陈述的、直觉上有效的论证可以用下列论证来例示:

如果第二个土著人说真话,那么只有一个土著人是政客。\\
第二个土著人说真话。\\
因此,只有一个土著人是政客。

这个论证的特征形式被称为\textbf{肯定前件式}:\\
$p \supset q$\\
$p$\\
$\therefore q$

如下真值表可以证明它是有效的:

\begin{center}
\begin{tabular}{|ccc|}
\hline
$p$ & $q$ & $p \supset q$ \\
\hline
T & T & T \\
T & F & F \\
F & T & T \\
F & F & T \\
\hline
\end{tabular}
\end{center}

在此,两个前提由第三栏和第一栏表示,结论由第二栏表示。只有第一行表示两个前提都真的代人例,而第二栏该行上的 T 表明,在这样的论证中结论也为真。这个真值表确立了具有肯定前件式的任何论证的有效性。

\subsubsection{否定后件式}
如前所见,如果一个条件陈述是真的,那么,如果其后件为假,其前件必假。这种论证形式很普遍地用来确定被探究命题的假。举例来说:在最近一次世界拼字游戏冠军赛中,马特•格雷厄姆和约耳•谢尔曼迎战叫马文的计算机程序。在比赛的某一局,他们发现他们的牌可以组成语词 "triduum";但对宾果游戏来说——用所有八张牌组成一个词——他们还要使用" s "。约耳对他的合作者说,"triduum"确实是一个词,但 "triduums"是否是正确的复数形式呢?马特用一种非常普遍的有效论证形式进行回答:"它必定是。如果它的复数是'tridua',我们就应该知道那个词,但我们不知道。"\cite{carroll1896}

该论证可以符号化为:

$$
\begin{aligned}
& p \supset q \\
& \sim q \\
& \therefore \sim p
\end{aligned}
$$

这个叫\textbf{否定后件式}的论证形式的有效性可用如下真值表表明:

\begin{center}
\begin{tabular}{|ccccc|}
\hline
$p$ & $q$ & $p \supset q$ & $\sim q$ & $\sim p$ \\
\hline
T & T & T & F & F \\
T & F & F & T & F \\
F & T & T & F & T \\
F & F & T & T & T \\
\hline
\end{tabular}
\end{center}

这里同样没有这样的代人例:在其中有这样一行,其前提 $p \supset q$ 和 $\sim q$ 都为真,而结论 $\sim p$ 为假。

\section*{D.一些常见的无效论证形式}
有两个无效的论证形式值得特别注意,因为它们与有效形式具有表面的相似性,因而经常迷惑粗心的作者或读者。在7.7节中讨论过的肯定后件谬误可以符号化为:

$$
\begin{aligned}
& p \supset q \\
& q \\
& \therefore p
\end{aligned}
$$

尽管这个论证形式在形式上有点类似肯定前件式,但这两个论证形式实际上很不相同。当然,该形式是无效的。例如,假如我们论证说,既然只要是美国公民自由联盟的成员就强烈支持言论自由,可知一个保护言论自由的人必定是美国公民自由联盟的支持者,这就犯了肯定后件谬误。

另一个无效形式叫否定前件谬误,它和否定后件式在形式上有点相像,可以符号化为:

$$
\begin{aligned}
& p \supset q \\
& \sim p \\
& \therefore \sim q
\end{aligned}
$$

这种谬误的一个例子是几年前一个纽约市市长候选人所用的一条竞选标

语:"如果不懂得赚钱,就不懂得这项工作——但亚伯懂得赚钱。"投票者被有意引导而未被陈述出来的结论是"亚伯懂得这项工作"一一个不能从所陈述的前提推出来的命题。

这两种常见的谬误都可以用真值表方法表明是无效的。在每种情形下,真值表中都有这样一行:这些谬误论证的前提都是真的,但结论是假的。

\section*{E.代入例与特征形式}
如我们早先在定义"论证形式"时所注意到的那样,一个给定论证可以是几个不同论证形式的代人例。本章开头所考察的那个有效析取三段论可以符号化为:

$$
\begin{aligned}
& R \vee W \\
& \sim R \\
& \therefore W
\end{aligned}
$$

它是下列有效论证形式的一个代人例:

$$
\begin{aligned}
& p \vee q \\
& \sim p \\
& \therefore q
\end{aligned}
$$

而且,它也是下列无效论证形式的一个代人例:

$$
\begin{aligned}
& p \\
& q \\
& \therefore r
\end{aligned}
$$

显然在最后一个形式中,从两个前提 $p$ 和 $q$ ,我们不能有效地推出 $r$ 。因此很清楚,一个无效的论证形式能够以一个有效的或一个无效的论证作为其代人例。所以,在确定某给定论证是否有效时,我们必须注意被探究论证的特征形式。只有论证的特征形式才准确地揭示了它的完整逻辑结构,正因为如此,我们才能够知道,如果一个论证的特征形式有效,那么该论证本身必定有效。

在上面所举例子中,我们看到了一个论证( $R \vee W, \sim R$ ,因此,$W$ )和以该论证为代人例的两个论证形式。这两个论证形式中的第一个( $p \mathrm{~V}$ $q, \sim p$ ,因此,$q$ )是有效的,因为该形式是给定论证的特征形式,它的

有效性确立了给定论证的有效性。第二个论证形式无效,但因为它不是给定论证的特征形式,所以,它不能被用来表明该给定论证无效。

应该强调指出:一个有效的论证形式只能以有效论证作为代人例。这就是说,一个有效形式的所有代人例必定有效。有效论证形式的有效性的真值表证明可以确证这一点,这表明,一个有效形式有前提为真而结论为假的代人例是不可能的。

\begin{center}
\fbox{\parbox{0.95\textwidth}{
\textbf{本节要点}
\begin{itemize}
\item \textbf{论证形式与有效性}:
  \begin{itemize}
  \item 有效性和无效性是论证的纯粹形式特征
  \item 同一形式的论证要么都有效,要么都无效
  \item 论证形式由陈述变元组成的符号序列表示
  \end{itemize}
\item \textbf{逻辑类推反驳法}:
  \begin{itemize}
  \item 通过构造相同形式但前提真结论假的例子
  \item 利用形式相同论证具有相同有效性的原理
  \end{itemize}
\item \textbf{真值表检验方法}:
  \begin{itemize}
  \item 检验论证形式的所有可能代入例
  \item 如果存在前提真而结论假的情况,论证无效
  \item 所有可能情况下前提真时结论也真,论证有效
  \end{itemize}
\item \textbf{常见有效论证形式}:
  \begin{itemize}
  \item 析取三段论:$p \vee q, \sim p, \therefore q$
  \item 肯定前件式:$p \supset q, p, \therefore q$
  \item 否定后件式:$p \supset q, \sim q, \therefore \sim p$
  \end{itemize}
\end{itemize}
}}
\end{center} 
\section*{8.5 陈述形式与实质等值}
\section*{A.陈述形式与陈述}
现在,我们来明确一下前一节所假定的一个概念,即"陈述形式"。以论证和论证形式之间的关系为一方,陈述和陈述形式之间的关系为另一方,这两者是完全平行的。"陈述形式"的如下定义可使这一点很明显:一个陈述形式是任何一个含有陈述变元但不含陈述的符号序列,若用陈述代入这些陈述变元——用同一个陈述始终一致地代入同一个陈述变元——其结果是一个陈述。例如,$p \vee q$ 是陈述形式,因为若用陈述代人变元 $p$和 $q$ ,就会产生一个陈述。由于所产生的陈述是一个析取句,$p \vee q$ 就叫做 "析取陈述形式"。同样,$p \cdot q$ 和 $p \supset q$ 分别叫做"合取陈述形式"和"条件陈述形式",$\sim p$ 叫做"否定形式"或者"否认形式"。正像某种形式的论证称为该论证形式的代人例一样,具有某种形式的任一陈述称为该陈述形式的代入例。正像我们判别一个给定论证的特征形式一样,我们把一个给定陈述的特征形式判别为这样一种陈述形式:通过一致地用不同的简单陈述代入每个不同的陈述变元,就可以从其产生该给定陈述。例如,$p \vee q$就是陈述"那个盲囚戴红帽子或者那个盲囚戴白帽子"的特征形式。

\section*{B.重言的、矛盾的和偶真的陈述形式}
尽管陈述"林肯是被暗杀的"(记为 $L$ )和"林肯或者是被暗杀的,或者不是"(记为 $L \vee \sim L$ )都是真的,但我们会非常自然地感觉到,它们是"在不同方面"为真,或有"不同种类"的真。同样,尽管陈述"华盛顿是被暗杀的"(记为 $W$ )和"华盛顿既是被暗杀的又不是被暗杀的"(记为 $W \cdot \sim W)$ 这两者都为假,但我们也会非常自然地感觉到,它们也是 "在不同方面"为假,或有"不同种类"的假。尽管我们不能给这些"感觉"以心理学解释,但我们仍然可以指出它们相应的逻辑区别。

陈述 $L$ 为真和陈述 $W$ 为假乃属于历史事实,它们没有逻辑必然性。所有事件都有以不同方式出现的可能,因而像 $L$ 和 $W$ 这样的陈述的真值,必须通过对历史的经验研究才能被发现。而陈述 $L \vee \sim L$ 尽管是真的,但它不是历史地真,而具有逻辑的必然性:事件不可能如此这般以致使它为假,它的真可以独立于任何经验研究而被知晓。陈述 $L \vee \sim L$ 是一个逻辑真理,或曰形式真理,其真仅因其形式,它是一个其所有代人例都是真陈

述的陈述形式的代人例。\\
一个只有真代入例的陈述形式叫重言的陈述形式,或重言式。要表明陈述形式 $p \vee \sim p$ 是一个重言式,可构造如下真值表:

\begin{center}
\begin{tabular}{|ccc|}
\hline
$p$ & $\sim p$ & $p \vee \sim p$ \\
\hline
T & F & T \\
F & T & T \\
\hline
\end{tabular}
\end{center}

这个真值表只有一个初始栏或导引栏,因为被探究的形式只含有一个陈述变元。它只有两行,代表了所有可能的代人例。被检验陈述形式下面的那一栏里只有 T ,这表明,它的所有代人例都是真的。任何一个作为重言的陈述形式的代人例的陈述,依据其形式就是真的,其本身被称为重言陈述,亦称为一个重言式。

一个只有假代入例的陈述形式称为自相矛盾的陈述形式,或矛盾式,它是逻辑地为假的。陈述形式 $p \cdot \sim p$ 是自相矛盾的,因为在它的真值表中只有 F 在它下面出现,这表明它的所有代人例都是假的。任何一个作为自相矛盾的陈述形式的代人例的陈述,如 $W \cdot \sim W$ ,依据其形式就是假的,其本身称为自相矛盾的陈述,亦称为一个矛盾式。

其代入例既有真陈述又有假陈述的陈述形式,叫做偶真陈述形式。其特征形式是偶真的陈述称为偶真陈述。 ${ }^{[15]}$ 例如,$p, \sim p, p \cdot q, p \vee q$ 和 $p \supset q$ 都是偶真陈述形式,$L, \sim L, L \cdot W, L \vee W, L \supset W$ 这样的陈述都是偶真陈述,因为它们的真值取决于它们的内容,而不只是它们的形式。

并非所有陈述形式都如上面所引的简单例子那样,明显是重言的、自相矛盾的或者偶真的。例如,陈述形式 $[(p \supset q) \supset p] \supset p$ 就一点也不明显,虽然真值表将表明它是一个重言式。它甚至还有一个特殊的名称—— "皮尔士法则"。

\section*{C.实质等值}
正如析取和实质蕴涵一样,实质等值也是一个真值函项联结词。如前所释,任何真值函项的真值,都取决于其所联结的陈述的真或假(是它们的一个函项)。例如,如果 A 或 B 是真的,或者 A 和 B 都是真的,那么, $A$ 和 $B$ 的析取就是真的。实质等值则是这样一种真值函项联结词:它断言它所联结的陈述有同样的真值。因此,两个在真值上相同的陈述,就是实质上等值的。可将之径直定义为:当两个陈述都为真或都为假时,它们就

是"实质等值的"。\\
正像析取的符号是楔劈号、实质蕴涵的符号是马蹄号一样,实质等值也有一个特殊的符号,即三杜号"三"。三杜号同样也可以用真值表定义如下:

\begin{center}
\begin{tabular}{|ccc|}
\hline
$p$ & $q$ & $p \equiv q$ \\
\hline
T & T & T \\
T & F & F \\
F & T & F \\
F & F & T \\
\hline
\end{tabular}
\end{center}

任何两个真陈述彼此实质地蕴涵,这是实质蕴涵含义的一个推论;同样,任何两个假陈述也彼此实质地蕴涵。因此,任何两个实质等值的陈述必定彼此蕴涵,因为它们或者都是真的,或者都是假的。

由于任何两个实质等值的陈述 A 和 B 彼此蕴涵,故而从它们的实质等值,我们可以推断出 B 是真的,当 A 是真的;也可以推断出 B 是真的,仅当 A 是真的。由于这两种关系都被实质等值所蕴涵,我们可以把三杠号"三"读做"当且仅当"。

在日常话语中,我们只偶尔使用这种逻辑关系词。有人会说,我去看冠军赛,当且仅当,我获得人场券。当我确实获得了人场券,我会去;但仅当我获得人场券,我才能去。这就是说,我去看比赛和我获得人场券,是实质上等值的。

如前所见,每个蕴涵式都是一个条件陈述。若已知 A 和 B 两个陈述实质上等值,既可推出条件陈述 $\mathrm{A} \supset \mathrm{B}$ 的真,也可推出条件陈述 $\mathrm{B} \supset \mathrm{A}$ 的真。由于在实质等值成立时,蕴涵是双向的,故而一个形如 $\mathrm{A} \equiv \mathrm{B}$ 的陈述通常称为双条件陈述。

合取、析取、实质蕴涵和实质等值,就是演绎论证通常所依赖的四个真值函项联结词。我们现在已经完成了对它们的讨论。

\section*{真值函项联结词}
真值函项联结词,就是真值函项复合命题中的逻辑联结词。真值函项复合命题,就是其真(或假)完全取决于其组成分支的真或假的复合命题。具有核心重要意义的真值函项联结词有四个:\\
-圆点号.表示合取。读做:"P且Q"。\\
$P \cdot Q$ 为真,当且仅当,$P$ 为真且 $Q$ 为真。\\
V 楔劈号.表示析取。读做:" P 或 Q "。\\
$P \vee Q$ 为真,当且仅当,$P$ 为真,或 $Q$ 为真,或 $P$ 和 $Q$ 两者都为真。\\
$\supset$ 马蹄号.表示实质蕴涵。读做:" P 蕴涵 Q "。\\
$P \supset Q$ 为真,当且仅当,并非 $P$ 为真且 $Q$ 为假,也就是,当且仅当, P 为假或 Q 为真。

三三杠号.表示实质等值。读做:" P 当且仅当 Q "。\\
$\mathrm{P} \equiv \mathrm{Q}$ 为真,当且仅当, P 和 Q 有同样的真值,也就是,当且仅当, P 为真且 Q 为真,或 P 为假且 Q 为假。

\section*{D.论证、条件陈述与重言式}
每个论证都对应着这样一个条件陈述:它的前件是该论证的前提的合取,它的后件是该论证的结论。例如,一个论证若具有肯定前件式的形式:

$$
\begin{aligned}
& p \supset q \\
& p \\
& \therefore q
\end{aligned}
$$

则可以被表达成一个具有形式 $[(p \supset q) \cdot p] \supset q$ 的条件陈述。如果原论证具有有效的论证形式,即在每种情形下其结论必定可以从其前提推出,那么,可以在真值表中表明,转化后的条件陈述是一个重言式。这就是说,一个论证的前提的合取蕴涵它的结论这样一个陈述有且只有真代人例(如果该论证有效的话)。

真值表是评价论证的有力工具。一个论证形式有效,当且仅当,在真值表中,其所有前提下面都是 $\mathbf{T}$ 的每一行上,其结论栏的下面也是 $\mathbf{T}$ 。这一点可以从"有效性"的精确含义中得出。如果表达该论证形式的条件陈述成了真值表中某一栏的题头,那么,$F$ 只能出现在该栏中其所有前提下面都是 $T$ ,且结论下面是 $F$ 的那一行。但如果该论证是有效的,就不会有这样一行。因此,只有 T 会出现在与一个有效论证相对应的条件陈述的下面,从而该条件陈述必定是一个重言式。所以,我们可以断言:一个论证

形式有效,当且仅当,其条件陈述表达形式(其前件是该论证形式的前提 339 的合取,其后件是该论证形式的结论)是一个重言式。

显然,对关于真值函项的任一无效论证来说,相应的条件陈述必定不是重言式。由一个无效论证的前提的合取蕴涵其结论构成的条件陈述,或者是偶真陈述,或者是矛盾陈述。 
\section{逻辑等价}

\begin{quotation}
本节引入逻辑等价的概念,探讨其与实质等值的区别以及在逻辑推理中的重要应用。通过德摩根定理和实质蕴涵的重要等价关系,我们能够在复杂的逻辑分析中进行命题转换,简化论证过程,并更深入地理解不同逻辑联结词之间的内在联系。
\end{quotation}

本节不是要引进一种新的联结词,而是要引进一种非常重要且十分有用的新关系,这种关系比刚才讨论过的任何一个真值函项联结词都要复杂些。

当陈述有相同的真值时,它们是实质等值的。因为两个实质等值的陈述或者都是真的,或者都是假的,既然假前件(实质)蕴涵任何陈述,真后件被任何陈述所(实质)蕴涵,我们就能看出它们必定彼此(实质)蕴涵。因此,我们可以把三杜号"三"读做"当且仅当"。

但确定为实质等值的陈述并不能互相替换。知道它们实质等值,我们只是知道它们的真值相同。陈述"木星比地球大"和陈述"东京是日本的首都"是实质等值的,因为它们都是真的,但我们显然不能用一个替换另一个。同样,陈述"所有蜘蛛都有毒"和陈述"所有蜘蛛都无毒"都是假的,所以它们实质等值,它们当然也不能彼此替换!

但在很多情况下,我们必须表示那种允许相互替换的关系。两个陈述可以在比实质等值强得多的意义上等值;它们可以在真值相同的同时,意义(meaning)也相等。如果它们有同样的意义,那么,与它们中的某个相结合的任何命题,也可以和另一个结合。没有一一也不可能有一一这样一种情形,即这些陈述中的一个是真的,而另一个是假的。这种非常强的意义上等值的陈述,我们称之为\textbf{逻辑等价}。

当然,任何两个逻辑等价的陈述也是实质等值的,因为它们显然必须有相同的真值。无疑,如果两个陈述逻辑等价,那么,它们在所有情形下都实质等值一一由此可获得逻辑等价的简短而有力的定义:若两个陈述的实质等值陈述是一个重言式,则两个陈述逻辑等价。这就是说,它们有同样的真值这样一个陈述自身必然是真的。这就是我们为什么要用三杜号的上方加一个 T,即"坖",来表示这种很强的逻辑关系。"$\xlongequal{T}$"表示的是这种逻辑关系的本质:两个逻辑等价陈述的实质等值式是一个重言式。因为实质等值式是一个"双条件陈述"(两个陈述互相蕴涵),所以,我们可以把这个逻辑等价符号"乍"视为表示一个重言的双条件陈述。

某些运用得非常普遍的简单逻辑等价式可以使这种关系及其威力得以彰显。 $p$ 和 $\sim \sim p$ 意谓同样的东西,这是一个常识;"他意识到那个困难"和"他不是没有意识到那个困难"是两个有同样意义的陈述。要言之,这两个表述中的任何一个都可以由另一个替换,因为它们说的是同一件事。这个双重否定原则的真对所有人来说都是显然的,这一点可以用真值表来展示。在此,两个陈述形式的实质等值式被表明是一个重言式:

\begin{center}
\begin{tabular}{|cccc|}
\hline
$p$ & $\sim p$ & $\sim \sim p$ & $p \stackrel{T}{\equiv} \sim \sim p$ \\
\hline
T & F & T & T \\
F & T & F & T \\
\hline
\end{tabular}
\end{center}

这个真值表证明 $p$ 和 $\sim \sim p$ 是逻辑等价的。因此,这个非常有用的逻辑等价式,即\textbf{双重否定式},可以符号化为:

$$
p \stackrel{\mathrm{~T}}{=} \sim \sim p
$$

实质等值和逻辑等价这两者之间的差别很大,并且也很重要。前者是一个真值函项联结词,即"三",它为真或为假仅取决于它所联结的分支的真或假;但后者,即逻辑等价"坖",不只是一个联结词,它还表达两个陈述之间某种非真值函项的关系。两个陈述逻辑等价,仅当它们绝对不可能有不同的真值。但如果它们总是有相同的真值,那么,逻辑等价陈述必定有同样的意义。在此种情形下,它们可以在任何真值函项语境中互相替换而不改变在该语境中的真值。反之,如果两个陈述仅仅碰巧有相同的真值,甚至它们之间没有实际的联系,那么,它们就只是实质等值的。只是实质等值的陈述当然不能互相替换!

有两个著名的逻辑等价式(即逻辑地真的双条件陈述)非常重要,因为它们表示了合取、析取及它们的否定之间的相互关系。下面即严格地考察这两个逻辑等价式。

首先,我们用什么来否定一个析取为真呢?任何析取式 $p \vee q$ 只是断言它的两个析取支中至少有一个是真的。若断言其析取支至少有一个为假,并不能与之相矛盾;(要否定它)我们必须断言两个析取支都为假。因此,断言析取 $\boldsymbol{p} \vee \boldsymbol{q}$ 的否定,逻辑地等价于断言 $\boldsymbol{p}$ 的否定和 $\boldsymbol{q}$ 的否定的合取。要在真值表中表明这一点,可以构造双条件陈述 $\sim(p \vee q) \equiv$ $(\sim p \cdot \sim q)$ ,把它放在它自己那一栏的顶端,然后检査它在所有情形下即每一行中的真值。

\begin{center}
\begin{tabular}{|l|l|l|l|l|l|l|l|}
\hline
$p$ & $q$ & $p \vee q$ & $\sim(p \vee q)$ & $\sim p$ & $\sim q$ & $\sim p \cdot \sim q$ & \begin{tabular}{l}
$\sim(p \vee q) \equiv$ \\
$(\sim p \cdot \sim q)$ \\
\end{tabular} \\
\hline
T & T & T & F & F & F & F & T \\
\hline
T & F & T & F & F & T & F & T \\
\hline
F & T & T & F & T & F & F & T \\
\hline
F & F & F & T & T & T & T & T \\
\hline
\end{tabular}
\end{center}

我们看到,无论 $p$ 和 $q$ 的真值如何,这个双条件陈述必定总是真的,因而是一个重言式。因为这个实质等值的陈述是一个重言式,所以可得出这两个陈述逻辑等价的结论。故我们已经证明:

$$
\sim(p \vee q) \stackrel{\mathrm{T}}{=} \sim p \cdot \sim q
$$

同样,由于断言 $p$ 和 $q$ 的合取,就是断言这两者都为真,要与该断言相矛盾,我们只需断定其中至少有一个为假。因此,断定合取 $(p \cdot q)$ 的否定,逻辑地等价于断定 $p$ 的否定和 $q$ 的否定的析取。在真值表中可以用符号表明,双条件陈述 $\sim(p \cdot q) \equiv(\sim p \vee \sim q)$ 是一个重言式。这样一个真值表就证明了:

$$
\sim(p \cdot q) \stackrel{\mathrm{T}}{=} \sim p \vee \sim q
$$

这两个重言的双条件陈述或逻辑等价式,被叫做\textbf{德摩根定理},因为它们是由数学家兼逻辑学家奥古斯塔•德摩根(1806-1871)正式表述出来的。德摩根定理用自然语言可以表述为:

(a) 两个陈述的析取的否定逻辑等价于这两个陈述的否定的合取;\\
(b) 两个陈述的合取的否定逻辑等价于这两个陈述的否定的析取。

这两个徳摩根定理被证明是特别有用的。\\
当我们试图系统处理真值函项联结词时,另一个重要的逻辑等价式非常有帮助。在本章的早些地方(8.3节),我们把实质蕴涵"コ"定义为说 $\sim(p \cdot \sim q)$ 的一种简略方式。也就是说,根据定义,"$p$ 实质蕴涵 $q$"的意思就是,并非 $p$ 为真而 $q$ 为假。在这个定义中,我们可以看到,定义项 $\sim(p \cdot \sim q)$ ,是一个合取的否定。根据德摩根定理,我们知道,任何这种否定逻辑等价于这些合取支的否定的析取;也就是说,我们知道, $\sim(p \cdot \sim q)$ 逻辑等价于 $(\sim p \vee \sim \sim q)$ ;再运用双重否定原则,这个表达式又逻辑等价于~pVq。逻辑等价的表达式意谓同样的事情,因此,马蹄号原来的定义项 $\sim(p \cdot \sim q)$ ,可以用一个更简单的表达式 $\sim p \vee q$ 来替换而不改变其含义。这就给了我们一个非常有用的实质蕴涵定义:$p \supset q$ 逻辑地等价于 $\sim p \vee q$ 。 可用符号写为:

$$
p \supset q \xlongequal{\mathrm{~T}} \sim p \vee q
$$

在表述逻辑陈述和分析论证时,需广泛地依赖实质蕴涵的这一定义。当我们所处理的陈述有相同的核心联结词时,操作起来通常会很简便也更有效力。运用我们刚才所建构的马蹄号的简单定义,即 $p \supset q \xlongequal{\mathrm{~T}} \sim p \vee q$ ,那些以马蹄号为联结词的陈述,可以方便地用那些以楔劈号为联结词的陈述来替换;同样,析取形式的陈述可以用蕴涵形式的陈述替换。在给出演绎论证有效性的形式证明时,这种替换被证明确实非常有用。

\begin{center}
\fbox{\parbox{0.95\textwidth}{
\textbf{本节要点}
\begin{itemize}
\item \textbf{逻辑等价}的特点:
  \begin{itemize}
  \item 表示两个陈述具有相同的意义
  \item A和B逻辑等价当且仅当"A当且仅当B"是重言式
  \item 逻辑等价的陈述可在任何真值函项语境中互相替换
  \end{itemize}
\item \textbf{逻辑等价}与\textbf{实质等值}的区别:
  \begin{itemize}
  \item 实质等值仅表示真值相同(偶然相同也可)
  \item 逻辑等价表示必然具有相同真值(基于意义相同)
  \item 仅实质等值的陈述不能互相替换
  \end{itemize}
\item \textbf{常见逻辑等价式}:
  \begin{itemize}
  \item 双重否定式:$p \xlongequal{\mathrm{~T}} \sim \sim p$
  \item 德摩根第一定理:$\sim(p \vee q) \xlongequal{\mathrm{~T}} \sim p \cdot \sim q$
  \item 德摩根第二定理:$\sim(p \cdot q) \xlongequal{\mathrm{~T}} \sim p \vee \sim q$
  \item 实质蕴涵的等价式:$p \supset q \xlongequal{\mathrm{~T}} \sim p \vee q$
  \end{itemize}
\item 逻辑等价在演绎推理中的应用:
  \begin{itemize}
  \item 允许在论证分析中进行有效的命题转换
  \item 简化复杂论证形式的分析过程
  \item 将一种逻辑联结词转换为另一种联结词
  \end{itemize}
\end{itemize}
}}
\end{center} 
\section*{8.7 实质蕴涵怪论}
有两个很容易证明是重言式的陈述形式,$p \supset(q \supset p)$ 和 $\sim p \supset(p \supset$ $q)$ 。在它们的符号表述中,这些陈述形式可能是无关紧要的,但若用日常语言表述出来,它们看起来令人惊奇,甚至怪异。第一个可以表述为: "如果一个陈述是真的,那么它被任何一个陈述所蕴涵。"由于"地球是圆的"是真的,可以推出"月亮是新鲜奶酪做的蕴涵地球是圆的",这确实

十分怪异,特别是因为它也可以得出:"月亮不是新鲜奶酪做的蕴涵地球是圆的。"第二个重言式可以表述为:"如果一个陈述是假的,那么它蕴涵任何陈述。"由于"月亮是新鲜奶酪做的"是假的,可以推出"月亮是新鲜奶酪做的蕴涵地球是圆的";当我们意识到由之也可以得出"月亮是新鲜奶酪做的蕴涵地球不是圆的"时,这就更怪异了。

这些陈述之所以看起来怪异,是因为我们相信,地球的形状和月亮的质料彼此之间是完全不相干的;我们还相信,没有任何真的或假的陈述能真正地蕴涵任何一个与之完全不相干的假的或真的陈述。可是真值表表明:一个假陈述蕴涵任何一个陈述,一个真陈述被任何陈述所蕴涵。然而,若我们认识到语词"蕴涵"的歧义性,该怪论很容易解决。根据语词 "蕴涵"的某几种含义,没有一个偶真陈述能蕴涵与其主题毫不相干的任何其他偶真陈述,这一点是非常正确的。诸如在逻辑蕴涵、定义性蕴涵和因果性蕴涵场合,这都是正确的。甚至在决策性蕴涵场合,这也是正确的,尽管相干概念在此必须作更宽泛的解释。

但严格说来,主题或意义与实质蕴涵不相干,实质蕴涵是一个真值函项。这里只有真和假是相干的。说任何一个至少含有一个真析取支的析取是真的,并没有任何怪异之处。这一事实是具有 $p \supset(\sim q \vee p)$ 和 $\sim p \supset$ $(\sim p \vee q)$ 形式的陈述所断言的所有东西,这两种形式的陈述逻辑等价于那两个"怪论"陈述。我们已经为把实质蕴涵当做"如果一那么"的一种含义提供了辩护,并且为把"如果一那么"的每次出现都翻译成符号 "つ"这种逻辑的权宜之计提供了辩护。这种辩护基于这样一个事实:把 "如果一那么"翻译成"コ",保留了在我们的逻辑研究所关注的那种论证中的所有有效论证的有效性。有人还提出了另一些符号体系,它们适合于其他类型的蕴涵,但它们超出了本书的范围,属于逻辑的更高级部分。 
\section*{8.8 三大"思想法则"}
一些早期思想家把逻辑定义为"关于思想法则的科学",并进一步断言:刚好有三个基本思想法则,它们如此基本以至遵从它们既是正确思维的必要条件又是其充分条件。传统上,这三大法则叫做:

这个原理断言:如果一个陈述是真的,那么它就是真的。我们可以用符号这样重述它:同一原理断言的是每个具有 $p \supset p$ 形式的陈述必定是真的,每个这样的陈述都是重言式。\\
-不矛盾原理。\\
这个原理断言:没有陈述是既真又假的。我们可以用符号这样重述它:不矛盾原理断言的是每个具有 $p \cdot \sim p$ 形式的陈述必定是假的,每个这样的陈述是自相矛盾的。\\
-排中原理。\\
这个原理断言:每个陈述或者是真的或者是假的。我们可以用符号这样重述它:排中原理断言的是每个具有 $p \vee \sim p$ 形式的陈述必定是真的,每个这样的陈述都是重言式。

显然,这三大原理确实是真的,是逻辑地为真的一一但说它们具有最基本的思想法则这一特权地位,是值得怀疑的。第一个(同一原理)和第三个(排中原理)是重言式,但还有许多其他的重言形式,它们的真是同等确定的。第二个(不矛盾原理)(所排除的 $p \cdot \sim p$ )也绝不是唯一的自相矛盾的陈述形式。

在构造真值表时,我们确实使用了这几个原理。受排中原理指导,我们在真值表每一行的初始栏下填人一个 T 或 F。受不矛盾原理指导,我们不在任何地方同时既填 T 又填 F 。一旦在某个指定行中把 T 填在某个符号下面,那么(受同一原理指导),当我们在那一行的其他栏下遇到该符号时,我们把它看做仍然被赋予T。因此,我们可以把这三大思想法则看做是支配真值表构造的原理。

不过,在考虑整个演绎逻辑体系时,这三大原理并不比其他许多原理更重要或更富有成效。确实,为演绎起见,有一些比它们更有成效的重言式。在这个意义上说,它们比这三大原理更重要。更深人地讨论这一点超出了本书的范围。 ${ }^{[16]}$

由于相信自己设计出了某种新的不同逻辑,一些思想家声称这三大原理实际上是不正确的,遵循它们是不必要的限制。但这些批评都是建立在误解的基础上的。

基于事物都是变化的而且一直在变化这一理由,同一原理遭到了攻击。例如,对原来的由 13 个州所组成的美国来说为真的某些陈述,对今

天有 50 个州的美国来说就不再是真的。然而,这并不能伤害同一原理。语句"美国只有 13 个州"是不完整的表述,它是陈述"1790 年的美国只有 13 个州"的一种省略表述,和它在1790年时一样,这个陈述在今天也是真的。若我们把注意力限制到命题的完整的、非省略的表述,我们就会看到,它们的真(或假)并不随时间而改变。同一原理之为真,并不妨碍我们对连续性变化的认识。

不矛盾原理受到了黑格尔主义者和马克思主义者的非难,其理由是:实际矛盾是普遍存在的,世界充满着不可避免的矛盾力量的冲突。说实在世界中存在着相冲突的力量,这当然是对的,但把这些冲突力量称为"矛盾",则是对该术语的一种不精确且令人误解的使用。劳工联盟和工厂私有者发现他们确实处于冲突之中——但私有者和劳工联盟都不是对方的 "否定"、"否认"或"矛盾"。若径直按照逻辑学家所意谓的那种意义理解,不矛盾原理是不可反驳、完全准确的。

基于其导致"二值化"这一理由,排中原理成了许多批评的靶子。 "二值化"意味着断言世界上的事物必定是"或白或黑"的,由此,它妨碍了妥协的实现,导致绝对化分层 ${ }^{(1)}$ 。这种反对意见也来自误解。陈述 "这是黑的"当然不能与陈述"这是白的"同时为真——假如"这"指的恰是同一事物的话。尽管这两个陈述不能同时为真,但它们却能同时为假。"这"可以既不是白的又不是黑的;这两个陈述是反对关系,而不是矛盾关系。与陈述"这是白的"有矛盾关系的陈述是"并非这是白的",并且(如果在这两个陈述中,"白的"都是在完全同样的意义上使用的话),它们当中必定有一个为真而另一个为假。排中原理是不可摆脱的。

总之,所有这三大"思想法则"都是不可驳倒的——只要它们被运用于那些使用非歧义、非省略且精确的词项的陈述。它们可能不具有某些哲学家所赋予它们的那种尊贵地位 ${ }^{[17]}$ ,但它们无疑都是正确的。 

% 第九章
\chapter{逻辑悖论}
\input{chapter9/section9-1.tex}
\section{替换规则}

\begin{quotation}
本节讨论如何使用替换规则来增强形式证明的能力。我们将学习十种重要的逻辑等价关系,它们可以作为替换规则使用,从而使我们能够构造更复杂论证的有效性证明。本节还会探讨形式证明的能行性及其与真值表方法的区别。
\end{quotation}

效性得不到证明。例如,要为下述明显有效的论证构造一个有效性的形式证明:

$$
\begin{aligned}
& A \supset B \\
& C \supset \sim B \\
& \therefore A \supset \sim C
\end{aligned}
$$

就要求增加新的规则。\\
在任何真值函项复合陈述中,如果它的一个分支陈述被另外一个有相同真值的陈述替换,该复合陈述的真值保持不变,而我们这里所关注的只有真值函项复合陈述,因此,我们可以把\textbf{替换规则}作为新的推论规则接受下来,该规则允许我们对任何陈述都可以做如下替换:该陈述的任一分支陈述都可被替换为与其逻辑等价的陈述。例如,根据断言 $p$ 逻辑地等价于 $\sim \sim p$ 的双重否定原则,通过替换,我们可以从 $A \supset \sim \sim B$ 推出下面的任何一个陈述:

$$
A \supset B, \sim \sim A \supset \sim \sim B, \sim \sim(A \supset \sim \sim B) \text {, 或 } A \supset \sim \sim \sim \sim B
$$

为把这项新规则加以确定,我们列出可以使用的十个重言的或逻辑地为真的双条件式。这些双条件式提供了在证明复杂论证的有效性时可使用的一些新增推论规则。我们接着 9.1 节所列的九条规则,给它们连续编号。

\subsection{替换规则列表}

下面任一逻辑等价的形式,在它们出现的任何地方,都可以相互替换:

10.\textbf{德摩根律}(De M.):$\sim(p \cdot q) \xlongequal{\mathrm{T}}(\sim p \vee \sim q)$

$$
\sim(p \vee q) \xlongequal{\mathrm{T}}(\sim p \cdot \sim q)
$$

11.\textbf{交换律}(Com.):$(p \vee q) \xlongequal{\text { T }}(q \vee p)$

$$
(p \cdot q) \stackrel{\mathrm{T}}{=}(q \cdot p)
$$

12.\textbf{结合律}(Assoc.):$[p \vee(q \vee r)] \stackrel{\mathrm{T}}{=}[(p \vee q) \vee r]$

$$
[p \cdot(q \cdot r)] \stackrel{\mathrm{T}}{=}[(p \cdot q) \cdot r]
$$

13.\textbf{分配律}(Dist.):$[p \cdot(q \vee r)] \stackrel{\mathrm{T}}{=}[(p \cdot q) \vee(p \cdot r)]$

$$
[p \vee(q \cdot r)] \stackrel{\mathrm{T}}{=}[(p \vee q) \cdot(p \vee r)]
$$

14.\textbf{双重否定律}(D.N.):$p \stackrel{\mathrm{~T}}{=} \sim p$

15.\textbf{易位律}(Trans.):$(p \supset q) \xlongequal{\text { T }}(\sim q \supset \sim p)$\\
16.\textbf{实质蕴涵律}(Impl,):( $p \supset q$ )$\xlongequal{T}(\sim p \vee q)$\\
17.\textbf{实质等值律}(Equiv.):$(p \equiv q) \stackrel{\mathrm{T}}{=}[(p \supset q) \cdot(q \supset p)]$

$$
(p \equiv q) \stackrel{T}{=}[(p \cdot q) \vee(\sim p \cdot \sim q)]
$$

18.\textbf{输出律}(Exp.):$[(p \cdot q) \supset r] \stackrel{\mathrm{T}}{=}[p \supset(q \supset r)]$\\
19.\textbf{重言律}(Taut.)${ }^{[1]}: p \stackrel{\mathrm{~T}}{=}(p \vee p)$

$$
p \stackrel{T}{\equiv}(p \cdot p)
$$

\subsection{替换与代入的区别}

替换的过程与代人非常不同:代人是以陈述代人陈述变元,而替换是以其他陈述替换陈述。从一个陈述形式到它的代人例的过程中,或者在从一个论证形式到其代人例的过程中,只要一个陈述被代人一个陈述变元的某次出现,它也必须被代人到该陈述变元的所有其他出现;在遵守此规定的条件下,我们就能用任何陈述代人任何陈述变元。但在从一个陈述到另一个陈述的替换过程中,运用10-19 中的某个逻辑等值式,我们只用一个与之逻辑等价的陈述,就能够替换第一个陈述中的某个分支陈述,我们可以只替换该分支陈述的某次出现,而不需要替换它的任何其他出现。

这 19 个推论规则并不构成这样一个极小集,即用它足以形式地证明复杂论证的有效性,在这个意义上说,它们有点多余。例如,否定后件式可以从表中去掉而并不真正削弱我们的证明手段,因为依据否定后件式的任何一行,实际上都能由表中的其他规则给予辩护。本章第 350 页(边码)所给的第一个形式证明例子中的第 8 行,$\sim A$ ,就是根据否定后件式,从第四和第七行,即 $\sim D$ 和 $A \supset D$ ,演绎出来的。但如果不把否定后件式作为推理规则,我们仍然能从 $A \supset D$ 和 $\sim D$ 演绎出 $\sim A$ 。譬如,在它们中间插人~Dつ~A这样一行,就可以做到这一点。 $\sim D \supset \sim A$ 可以根据易位原则(Trans.)从 $A \supset D$ 推出,然后根据肯定前件式(M.P.),可以从 $\sim D$ $\supset \sim A$ 和 $\sim D$ 得到 $\sim A$ 。但否定后件式作为一个运用如此频繁且直觉上如此显明的推论规则,应该被包括在推论规则之内。这 19 个中的其他一些规则,在这个意义上也是多余的。

\subsection{推论规则表的完备性}

这个推论规则表不仅有冗余的特点,它还有某种不足。例如,尽管论证:

$$
\begin{aligned}
& \sim B \\
& \therefore A
\end{aligned}
$$

直觉上有效,但它的形式:

$$
\begin{aligned}
& p \vee q \\
& \sim q \\
& \therefore p
\end{aligned}
$$

却没有包括在推论规则之内。尽管结论 A 可以根据两个推论规则,从前提 $A \vee B$ 和 $\sim B$ 演绎出来,但它并不是根据任何单一的推论规则,从这两个前提推出来的。该论证有效性的形式证明可以写为:

1.$A \vee B$\\
2.$\sim B$\\
$\therefore A$\\
3.$B \vee A \quad 1$ ,Com.\\
4.A $3,2, \mathrm{D} . \mathrm{S}$ .\\
若在推论规则表中添加另外一个规则,我们可以消除这种不足。但是,如果我们对每个这样的情形都添加一个规则,我们最终会有一个长得多且更不易处理的规则表。

对任何有效的真值函项论证来说,目前这个有 19 个推理规则的列表使得我们都可以为之构造一个有效性的形式证明。在这个意义上说,该表构成了真值函项逻辑的一个\textbf{完全的系统}。 ${ }^{[2]}$

\subsection{形式证明的能行性}

形式证明是一个\textbf{能行的}(effective)概念。所谓"能行的"的意思是说,根据给定的推论规则表,可以在有限步骤内机械地判定一个给定陈述序列是否构成一个形式证明。这里不需要任何思维。所谓不需要思维,就是既不需要思考序列中的陈述的"意义",也不需要用逻辑直觉来检查任何步骤的有效性。这里只需要做两件事。第一件事是,能够看出在一个地方出现的某个陈述与在另一个地方出现的一个陈述是完全相同的,因为我们必须能够核对出,证明中的某些陈述是所欲证明其有效性的那个论证的前提,以及证明中的最后一个陈述是该论证的结论。所要求的第二件事是,能够看出一个给定陈述是否有某种模式,即能看出它是否是某个陈述形式的代人例。

这样,关于上列陈述序列是否是一个有效性的形式证明的问题,就很容易用一种完全机械的方式来解答。一眼就可以看出,第1行和第2行是该论证的前提,第 4 行是结论。第 3 行是根据某个给定推论规则从前面几行推出的,这一点可以在有限几步内确定——即使符号"1,Com."不写在旁边。第二栏中的解释性符号起帮助作用,它应该包括在证明内。但严格说来,它本身并不是证明的一个必要部分。每一行的前面只有有限几行,并且只有有限多的推论规则或凭据形式可查。尽管费时,但通过对形式进行观察和比较,可以确定第 3 行不是根据肯定前件式、否定后件式或假言三段论等从第 1 行和第 2 行推出的。一直依照这种程序进行,直到我们碰到这样一个问题:第 3 行是否是根据交换律从第 1 行和第 2 行推出来的?此时,仅通过观察形式,我们就可以知道的确如此。任何形式证明中的任何陈述的合法性,都可用同样的方式在有限步骤内得到检验。没有哪一步涉及形式或形态比较之外的任何其他东西。为了保持这种能行性,我们要求一次只采取一个步骤。可能有人想合并几步以缩短证明,但所节约的时间和空间是微不足道的,通过每步只用一个推论规则而获得能行性才是更重要的。

尽管在有效性的形式证明能够机械地确定一个给定序列是否一个证明的意义上,形式证明是能行的,但建构一个形式证明并没有一个能行的程序。在这方面,形式证明不同于真值表方法。真值表的构造是完全机械的:给定任何一个我们现在所关注的那类论证,依照上一章规定的简单程序规则,我们总能构造一个真值表来检验其有效性。但我们没有能行的或机械的规则来构造形式证明。我们必须思考或"想出"从哪儿着手,以及怎样前进。不过,通过构造一个有效性的形式证明来证明一个论证的有效性,比纯机械构造的真值表方法要简单得多。这样的真值表可能有几百甚至几千行。

\subsection{前九条与后十条规则的区别}

前九条和后十条推论规则之间有很重要的区别。前九条规则只能运用到证明中的完整行上。例如,在有效性的形式证明中,只有当 $A \cdot B$ 构成一完整行时,陈述 $A$ 才能根据简化律从陈述 $A \cdot B$ 推出。显然,$A$ 不能有效地从 $(A \cdot B) \supset C$ 或 $C \supset(A \cdot B)$ 推出,因为在 $A$ 为假时,后两个陈述可以为真。陈述 $A \supset C$ 也不能根据简化律或任何其他推论规则,从 $(A \cdot B) \supset C$ 推出。它根本推不出,因为如果 $A$ 为真,并且 $B$ 和 $C$ 都为假,那么 $(A \cdot B) \supset C$为真,而 $A \supset C$ 为假。再如,尽管根据附加律,$A \vee B$ 可以从 $A$ 推得,但

我们不能根据附加律或其他任何推论规则,从 $A \supset C$ 推出 $(A \vee B) \supset C$ 。因为如果 $A$ 和 $C$ 都为假而 $B$ 为真,则 $A \supset C$ 为真而 $(A \vee B) \supset C$ 为假。另一方面,后十条推论规则中的任何一个都既可运用到整行,也可运用到行中的某些部分。根据输出律,不仅可以从整行 $(A \cdot B) \supset C$ 推出陈述 $A \supset$ $(B \supset C)$ ,我们还可以从行 $[(A \cdot B) \supset C] \vee D$ 推出 $[A \supset(B \supset C)] \vee D$ 。根据替换规则,逻辑等价式可以相互替换它们的每次出现,即使它们并不构成证明中的一个完整行。但前九条推论规则只能运用到证明中的完整行中,而且这些完整行是作为前提来使用的。

\subsection{构造形式证明的技巧}

尽管没有构造形式证明的纯机械性规则,但可以给出一些大略的规则,或一些关于证明进程的提示。第一个提示是,要根据给定推论规则从给定前提着手演绎结论。随着越来越多的子结论成为进一步演绎的前提,会越来越清楚该如何演绎出所欲证明为有效的那个论证的结论。另一个提示是,要努力消除那些在前提中出现而结论中不出现的陈述。当然,这种消除只能依据推论规则进行。这些推论规则中含有许多消除陈述的技巧。简化律就是这样一个规则,借此可以去掉整行中合取式右边的合取支。交换律允许我们把合取式左边的合取支换到右边,然后根据简化律就可以去掉那个合取支了。给定两个具有模式 $p \supset q$ 和 $q \supset r$ 的陈述,根据假言三段论,可以消除"中项"$q$ 。分配律是一个把形如 $p \vee(q \cdot r)$ 的析取式变换为合取式 $(p \vee q) \cdot(p \vee r)$ 的有用规则。根据简化律,就可以消除这个合取式右边的合取支。另一个值得提出的规则是,可根据附加律,引人一个结论中出现但前提中未出现的陈述。再一个常用的方法是,从结论倒澌寻找结论从中演绎出来的那个或那些陈述,然后试着从前提演绎出那些中间陈述。然而,要想熟练地掌握构造形式证明的方法,习题训练是无可替代的途径。

\begin{center}
\fbox{\parbox{0.95\textwidth}{
\textbf{本节要点}
\begin{itemize}
\item \textbf{替换规则}允许我们用逻辑等价的陈述替换复合陈述中的部分内容
\item 十种重要的逻辑等价关系(替换规则):
  \begin{itemize}
  \item \textbf{德摩根律}、\textbf{交换律}、\textbf{结合律}、\textbf{分配律}
  \item \textbf{双重否定律}、\textbf{易位律}、\textbf{实质蕴涵律}
  \item \textbf{实质等值律}、\textbf{输出律}、\textbf{重言律}
  \end{itemize}
\item 替换与代入的区别:
  \begin{itemize}
  \item 代入是以陈述代入陈述变元,且必须在所有出现处一致替换
  \item 替换是以逻辑等价的陈述替换另一陈述的部分,可仅替换某次出现
  \end{itemize}
\item 形式证明的能行性:
  \begin{itemize}
  \item 可以机械地验证一个给定序列是否构成形式证明
  \item 但构造形式证明没有机械的程序,需要思考和技巧
  \end{itemize}
\item 前九条与后十条规则的区别:
  \begin{itemize}
  \item 前九条规则只能应用于证明中的完整行
  \item 后十条规则可应用于整行或行中的部分内容
  \end{itemize}
\end{itemize}
}}
\end{center} 

\begin{center}
\begin{tabular}{|l|l|}
\hline
\multicolumn{2}{|c|}{推论规则} \\
\hline
\multicolumn{2}{|r|}{我们阐述了构造有效性证明要使用的19条规则。它们是:} \\
\hline
基本有效论证形式: & 逻辑等价表达式: \\
\hline
1.肯定前件式(M.P.): & 10.德摩根律(De M.): \\
\hline
$p \supset q, \quad p, \quad \therefore q$ &  \\
\hline
 & \( \begin{aligned} & \sim(p \cdot q) \stackrel{\mathrm{T}}{=}(\sim p \vee \sim q) \\ & \sim(p \vee q) \stackrel{\mathrm{T}}{=}(\sim p \cdot \sim q) \end{aligned} \) \\
\hline
\end{tabular}
\end{center}

2.否定后件式(M.T,):\\
$p \supset q, \sim q, \therefore \sim p$

3.假言三段论(H.S.):\\
$p \supset q, q \supset r, \therefore p \supset r$

4.析取三段论(D.S.):\\
$p \vee q, \sim p, \therefore q$

5.构造式二难(C.D.):\\
$(p \supset q) \cdot(r \supset s), p \vee r, \therefore q \vee s$\\
6.吸收律(Abs.):\\
$p \supset q, \therefore p \supset(p \cdot q)$\\
7.简化律(Simp.):\\
$p \cdot q, \therefore p$

8.合取律(Conj.)\\
$p, q, \therefore p \cdot q$

9.附加律(Add.):\\
$p, \therefore p \vee q$

11.交换律(Com.):\\
$(p \vee q) \stackrel{\mathrm{T}}{=}(q \vee p)$\\
$(p \cdot q) \stackrel{\mathrm{T}}{=}(q \cdot p)$\\
12.结合律(Assoc.):\\
$[p \vee(q \vee r)] \stackrel{\mathrm{T}}{\equiv}[(p \vee q) \vee r]$\\
$[p \cdot(q \cdot r)] \stackrel{\mathrm{T}}{=}[(p \cdot q) \cdot r]$\\
13.分配律(Dist.):\\
$[p \cdot(q \vee r)] \stackrel{\mathrm{T}}{=}[(p \cdot q) \vee(p \cdot r)]$\\
$[p \vee(q \cdot r)] \stackrel{\mathrm{T}}{\equiv}[(p \vee q) \cdot(p \vee r)]$\\
14.双重否定律(D.N.):\\
$p \stackrel{\mathrm{~T}}{=} \sim p$\\
15.易位律(Trans.):\\
$(p \supset q) \stackrel{T}{\equiv}(\sim q \supset \sim p)$\\
16.实质蕴涵律(Impl.):\\
$(p \supset q) \stackrel{\mathrm{T}}{=}(\sim p \vee q)$\\
17.实质等值律(Equiv.):\\
$(p \equiv q) \stackrel{\mathrm{T}}{\equiv}[(p \supset q) \cdot(q \supset p)]$\\
$(p \equiv q) \stackrel{\mathrm{T}}{=}[(p \cdot q) \vee(\sim p \cdot \sim q)]$\\
18.输出律(Exp.):\\
$[(p \cdot q) \supset r] \stackrel{\mathrm{T}}{=}[p \supset(q \supset r)]$\\
19.重言律(Taut.):\\
$p \stackrel{\mathrm{~T}}{=}(p \vee q)$\\
$p \stackrel{\mathrm{~T}}{=}(p \cdot p)$ 
\section*{9.3 无效性的证明}
对一个无效论证来说,当然没有其有效性的形式证明。但如果我们找不到一个给定论证的有效性的形式证明,这种失败并不就能证明该论证无效,也并不证明不可能构造出形式证明。这可能只意味我们的努力还不够。我们未能发现一个论证有效性的形式证明,可能是由该论证无效这一事实造成的,但也可能是由于我们缺乏聪明才智造成的,这是证明的构造过程的非能行性的后果。未能构造某个论证的有效性的形式证明并不能证明该论证无效。那么,怎样构成一个给定论证无效性的一个证明呢?

如下描述的方法与真值表方法密切相关,尽管这个方法比后者简略得多。回想我们如何用真值表证明一个无效的论证形式无效,有助于如下讨论。如果能在真值表中发现这样一个情形(或一行),即对一个论证形式

中的陈述变元进行这样一种真值指派,使得其前提为真而结论为假,那么该论证形式就是无效的。如果我们能对一个论证的简单分支陈述进行这样的真值指派,即使得它的前提为真且结论为假,那么,这种指派就足以证明该论证无效。实际上,进行这种指派正是真值表所做的。但如果我们不实际构造完整的真值表就能进行这种真值指派,那么可以省去很多工作。考查这样一个论证:

如果地方官赞同政府为低收入者修建住房,那么他会赞成限制私有企业的规模。

如果地方官是一个社会主义者,那么他会赞成限制私有企业的规模。

因此,如果地方官赞同政府为低收入者修建住房,那么他是一个社会主义者。

它可以符号化为:

$$
\begin{aligned}
& F \supset R \\
& S \supset R \\
& \therefore F \supset S
\end{aligned}
$$

我们不必构造完整的真值表就可以证明它无效。首先可以提问:"使结论为假要求何种真值指派?"显然,一个条件陈述为假,仅当它的前件为真而后件为假。因此,给 $F$ 指派真值"真",且给 $S$ 指派真值"假",会使得结论 $F \supset S$ 为假。现在,如果把真值"真"指派给 $R$ ,那么两个前提都是真的,因为只要它的后件为真,该条件陈述就为真。于是我们可以说,如果把真值"真"指派给 $F$ 和 $R$ ,把真值"假"指派给 $S$ ,该论证就有真前提和假结论,据此,它就被证明为无效。

这种证明无效性的方法是真值表证明方法的一个变种,因而应注意这二者之间的本质联系。实际上,在我们进行如上所示的那种真值指派时,我们所做的就是构造给定论证的真值表中的一行。若我们把这种真值指派水平地写成下述形式:

\begin{center}
\begin{tabular}{|cccccc|}
\hline
$F$ & $R$ & $S$ & $F \supset R$ & $S \supset R$ & $F \supset S$ \\
\hline
真 & 真 & 假 & 真 & 真 & 假 \\
\hline
\end{tabular}
\end{center}

这种关系会看得更清楚。其中,上述真值指派构成了论证真值表中的一行 (第二行)。通过显示其真值表中至少有这样一行,即其前提都为真而结论为假,一个论证就被证明为无效。因此,要发现一个论证的无效性,我们不必检验它的真值表的每一行:只要发现有一行,它的前提都为真而结论为假,这就足够了。当前证明无效性的方法,就是一种构造这样一行而不必构造整个真值表的方法。 ${ }^{[3]}$

目前这种方法比写出整个真值表要简略,一个论证所涉及的简单分支陈述越多,这种方法相应地节省的时间和空间也越多。对一个前提相当多,或相当复杂的论证来说,进行所需的真值指派可能不太容易。虽然没有机械的处理办法,但亦可证明某些提示是有帮助的。

如果要证明无效性,给那些立即就能看出是基本的陈述指派真值是最有效的做法。例如,任何仅断言某陈述 $S$ 为真的前提,立刻提示我们对 $S$指派 $T$(或 $F$ ,如果作为前提的 $S$ 已被断言为假),因为我们知道所有前提必须被处理为真。同一原则适用于结论中的陈述,只是那里的真值指派必须使结论为假。因此,一个形如 $A \supset B$ 的结论,会立时提示对 $A$ 指派 $T$ ,对 $B$ 指派 $F$ ;一个形如 $A \vee B$ 的结论,会立时提示对 $A$ 指派 $F$ ,对 $B$也指派 $F$ ,因为只有这种指派才能产生无效性的证明。

我们应该从寻求使前提为真出发,还是从寻求使结论为假出发,取决于这些命题的结构。一般来说,我们最好从最有把握的指派开始。当然,会有许多这样的情形,其第一次指派不得不是任意的和试探性的。一定数量的试错是必要的。但即使这样,这种证明无效性的方法,也几乎总比写出完整的真值表简略和容易。 
\section{不相容性}

\begin{quotation}
本节探讨前提互不相容的情况及其对演绎论证有效性的影响。我们将分析为什么不相容的前提集可以推出任何结论,即使这些结论看似无关或荒谬。通过理解这一"严格蕴涵怪论",我们能更全面地把握逻辑有效性的本质以及相容性在理性思维中的重要地位。
\end{quotation}

如果一种真值指派能使得一个论证的所有前提为真而结论为假,那么,这表明该论证是无效的。而如果一个演绎论证不是无效的,那么,它必定是有效的。因此,如果不可能对一个论证的简单分支陈述进行这种真值指派,即不可能使得它的前提为真而结论为假,那么该论证必定有效。尽管从"有效性"的定义可以推出这一点,但它也有一个怪异的推论。考虑下面的论证,它的前提看上去与结论完全不相干:

\begin{quote}
如果飞机的引擎出了故障,它就降落在本德了。\\
如果飞机的引擎没有出故障,它就降落在克利夫兰了。\\
飞机没有降落在本德或克利夫兰。\\
因此,飞机必定降落在丹佛了。
\end{quote}

把它翻译成符号就是:

$$
\begin{aligned}
& A \supset B \\
& \sim A \supset C \\
& \sim(B \vee C) \\
& \therefore D
\end{aligned}
$$

对其简单分支陈述进行使其所有前提为真而结论为假的真值指派的努力,都注定会失败。如果我们忽略结论,把注意力放在对其简单分支陈述进行使其所有前提都为真的真值指派上,我们也一定会失败——尽管这个计划初看上去并不难以实现。

\subsection{前提不相容与有效性}

这里之所以不能获得前提都真而结论为假的真值指派,乃因为在任何情形下使用任何真值指派都不能使前提都真。由于前提是\textbf{互不相容}的,故没有真值指派能使它们都真。前提的合取作为一个矛盾的陈述形式的代人例,乃是自相矛盾的。如果我们构造该论证的真值表,就会发现在每一行中至少有一个前提是假的。因为没有所有前提都为真这样一行,也就没有所有前提为真而结论为假这样一行。因此,该论证的真值表确立了它的有效性。下面的形式证明也可以确立它的有效性:

1.$A \supset B$\\
2.$\sim A \supset C$\\
3.$\sim(B \vee C)$\\
$\therefore D$\\
4.$\sim B \cdot \sim C$\\
5.$\sim B$\\
6.$\sim A$\\
7.$C$\\
8.$\sim C \cdot \sim B$\\
9.$\sim C$

3,De M.\\
4,Simp.\\
$1,5, \mathrm{M} . \mathrm{T}$ .\\
2,6,M.P.\\
4,Com.\\
8,Simp.

\begin{center}
\begin{tabular}{ll}
10.$C \vee D$ & 7, Add. \\
11.$D$ & 10,9, D.S. \\
\end{tabular}
\end{center}

在这个证明中, 1 至 9 行表明了前提中隐含的不相容性。这种不相容性呈现在第 7 行和第 9 行,它们分别断言了 $C$ 和 $\sim C$ 。一旦这种明显的矛盾被表示出来,根据附加律和析取三段论原理,很快就可以推出结论。

由此可见,如果一组前提不相容,这些前提就会有效地产生任何结论,而不论它们如何不相干。下面的论证更简单地表明了这一问题的精髓,其公然不相容的前提使得我们可以有效地推出一个不相干且荒谬的结论:

今天是星期天。\\
今天不是星期天。\\
因此,月亮是鲜奶酪做的。

用符号表示就是:\\
1.$S$\\
2.$\sim S$\\
$\therefore M$\\
它的有效性的形式证明十分显然:

\begin{center}
\begin{tabular}{ll}
3.$S \vee W$ & 1, Add. \\
4.$M$ & 3,2, D.S. \\
\end{tabular}
\end{center}

\subsection{不相容前提的问题}

问题出在哪里呢?如此贫乏甚至不相容的前提怎能使得它们在其中出现的论证有效?首先要注意到,如果一个论证因其前提的不相容性而有效,那么它不可能是一个合理的论证。如果前提互不相容,它们不可能都是真的。一个前提不相容的论证不能确立任何结论的真,因为它的前提本身不可能都是真的。

目前情形与所谓\textbf{实质蕴涵怪论}密切相关。在讨论后者时,我们注意到 (在 8.7 节),陈述形式 $\sim p \supset ~(p \supset q) ~$ 是一个重言式,其所有代人例都为真。它的自然语言表述断言的是:"如果一个陈述为假,那么它实质蕴涵任何陈述。"用真值表很容易证明这一点。当下讨论所确立的是下述论证形式有效:\\
$\sim p$\\
$\therefore q$\\
我们已经证明:不管其结论是什么,任何前提不相容的论证都是有效的。它的有效性可以用真值表,或者用形式证明判定。

一个有效论证的前提蕴涵它的结论,不仅仅是"实质"蕴涵意义上的,还有逻辑的或"严格"意义上的蕴涵。在一个有效论证中,当结论为假时,其前提为真是逻辑不可能的。只要前提为真是逻辑不可能的,即使忽略结论的真假问题,这种情形也照样成立。它和实质蕴涵相应性质的相似性,使某些逻辑学者称之为"\textbf{严格蕴涵怪论}"。然而,根据逻辑学家对 "有效性"的技术性定义.它似乎并不是特别怪异的。所宣称的这个怪论之所以产生,主要是由于把一个技术性术语当成日常语言中的普通术语。

\subsection{相容性的重要性}

前面的讨论有助于解释为什么对相容性评价如此之高。其基本原因当然是,两个不相容的陈述不能都是真的。这一事实乃是交互询问策略的基石。在交互询问中,律师会设法使对方证人陷人自相矛盾。如果证词肯定了不能自圆其说或不相容的断言,那么证词不能都是真的,证人的可信性就被破坏…或至少被动摇。 ${ }^{[4]}$ 不相容性令人如此反感的另一个原因是,任何结论都可从一些被当做前提的不相容陈述逻辑地推出。不相容陈述并不是"没有意义的",它们的麻烦正好相反:其意谓太多。在蕴涵任何东西这个意义卜说,它们意谓着所有东西。如果所有东西都被断言,那么被断言的有一半肯定是假的,因为每个陈述都有一个否定。

上面的讨论附带地为我们解答了一个古老难题:一个不可抗拒的力量遇到一个不可移动的物体,会发生什么事?这个描述含有一个矛盾。要一个不可抗拒的力量遇到一个不可移动的物体,这两者都必须存在。必定存在一个不可抗拒的力量,并且也必定存在一个不可移动的物体。但如果存在不可抗拒的力量,就不会存在不可移动的物体。在此,矛盾被表述得很清楚:存在一个不可移动的物体,并且不存在一个不可移动的物体。给定这种不相容的前提,任何结论都可有效地推出。因此,对"一个不可抗拒的力量遇到一个不可移动的物体,会发生什么事?"这一问题的正确回答是"任何事"!

\subsection{不相容性与幽默}

尽管在一个论证中发现不相容性是灾难性的,但正如伟大的棒球运动员扬基队的贝拉经常被引用的评论那样,不相容性是非常有趣的。据说,

贝拉曾宣称"那个餐馆如此拥挤以致不再有人去那儿了"。在谈到他的那段长而幸福的婚姻中的伴侣时,他说:"我们长时间待在一起,即使我们不在一起时也是如此。"

这些话语很有趣,因为它们所包含的矛盾(若照字面意义理解,这些评论都是胡说),似乎没被它们的作者意识到。因此,当我们听到学生说,澳大利亚内地的气候如此不好,以致居民不再住在那儿了,我们会暗自发笑。这种漫不经心且未意识到的不相容话语,有时被称为"Irish Bull" (爱尔兰牛皮)。

从逻辑上看,不相容的命题集不可能同时为真。但人们并非总是合乎逻辑的,有时确实会说出甚至会相信两个互相矛盾的命题。这一点似乎难以置信,但逻辑领域一个非常值得信赖的权威刘易斯•卡罗尔告诉我们,《爱丽丝漫游奇境记》中的白衣女王形成了这样一个习惯,即在早餐之前相信六件不可能的事。 

\begin{center}
\fbox{\parbox{0.95\textwidth}{
\textbf{本节要点}
\begin{itemize}
\item \textbf{前提不相容与论证有效性}:
  \begin{itemize}
  \item 互不相容的前提集意味着不可能所有前提同时为真
  \item 因此无法找到"前提全真而结论为假"的真值指派
  \item 这导致不相容前提的论证形式必定是有效的
  \end{itemize}
\item \textbf{严格蕴涵怪论}:
  \begin{itemize}
  \item 不相容的前提可以有效推出任何结论
  \item 形式上类似于实质蕴涵中"假陈述蕴涵任何陈述"
  \item 这种怪论源于逻辑术语的技术性定义与日常用法的差异
  \end{itemize}
\item \textbf{相容性的价值}:
  \begin{itemize}
  \item 相容性是理性思维的基础
  \item 不相容陈述的问题不是"无意义"而是"意义过多"
  \item 不相容陈述集逻辑上蕴涵一切可能的结论
  \end{itemize}
\item \textbf{不相容性在日常语言中}:
  \begin{itemize}
  \item 有时产生幽默效果(如"爱尔兰牛皮")
  \item 在法律中作为质疑证词可信度的基础
  \item 说明人类思维并非总是严格遵循逻辑规则
  \end{itemize}
\end{itemize}
}}
\end{center} 

% 第十章
\chapter{非经典逻辑}
\section*{10.1 单称命题}
前两章的逻辑技术使得我们可以对某种类型的有效论证和无效论证进行区分。该类型的论证可以粗略地刻画为:其有效性仅取决于简单陈述通过真值函项结合成复合陈述的方式。然而,还有另外一些类型的论证,前两章的有效性标准对它们不够用。一个明显有效的不同类型论证的例子是:

所有人都是有死的。\\
苏格拉底是人。\\
因此,苏格拉底是有死的。

如果把前面介绍的评估方法运用到这个论证上,我们可以将之符号化为:\\
A\\
H\\
$\therefore M$\\
在这种符号式中,该论证显然无效。因此,到目前为止所介绍的符号逻辑技术不能直接运用到这种新型论证上。该论证的有效性并不取决于简单陈述的复合方式,因为在该论证中没有出现任何复合陈述。毋宁说它的有效性取决于所涉及非复合陈述的内在逻辑结构。要阐明检验这种新型论证有效性的方法,就必须依据它们的内在逻辑结构,设计出一些描述和符号化非复合陈述的技术。

一种最简单的非复合陈述的例示是前述论证中的第二个前提:"苏格拉底是人。"这种类型的陈述传统上叫做单称命题。一个(肯定的)单称命题断言的是,一个特定个体具有某种特定属性。在上述例子中,日常语法和传统逻辑都一致地把"苏格拉底"划为主项,把"人"划为谓项。主项指称某特定个体,谓项指谓该个体据称所具有的某种属性。

显然,同一主项可以在不同的单称命题中出现。因此,在下述每个命题中,我们都以词项"苏格拉底"做主项:"苏格拉底是有死的","苏格拉底是胖的","苏格拉底是聪明的"和"苏格拉底是漂亮的"。当然,有些是

真的(第一和第三个),有些是假的(第二和第四个)。 ${ }^{[2]}$ 同一谓项显然也可以出现在不同的单称命题中。因此在下述每个命题中,我们都以词项"人"做谓项:"亚里士多德是人","巴西是人","芝加哥是人"和"奥基夫是人"。当然,有些是真的(第一和第四个),有些是假的(第二和第三个)。

从前所述,我们应该清楚语词"个体"不仅用来指人,还可以指事物,譬如,国家、城市,实际上可以指谓像是人或有死的这样能被有意义地断言为其属性的任何事物。前面所举的例子中,有些谓项是形容词。从日常语法的观点看,形容词与名词的区分是相当重要的。但在本章中这种区别并不重要,我们并不区分"苏格拉底是有死的"和"苏格拉底是有死者",或"苏格拉底是聪明的"和"苏格拉底是一个聪明的人"。一个谓项可以是一个形容词或者是一个名词,甚或是一个动词。如在"亚里士多德写作"中,它有时可以被表述为"亚里士多德是一个写作者"。

假定我们能区分开具有某属性的个体和它们所具有的属性,我们引进并使用两种不同的符号来指称它们。在随后的讨论中,我们将用从 $a$ 到 $w$的小写字母来指谓个体。这些符号是个体常元。在它们出现的任何特定上下文,每个在该整个上下文中都指称一个特定的个体。用它(他,或她)的名称的第一个字母指称一个个体,通常是很方便的。因此在当前的上下文中,我们应分别用字母 $s 、 a 、 b 、 c 、 o$ 指称苏格拉底、亚里士多德、巴西、芝加哥和奥基夫。我们将用大写字母来符号化属性,在此使用同样的指导原则是很便利的。因此,我们用字母 $H 、 M 、 F 、 W 、 B$ 分别符号化属性是人、有死的、胖的、聪明的、漂亮的。

有两组符号,一组是个体的符号,另一组是个体属性的符号。我们采取这样一个约定:把属性符号直接写在个体符号的左边,表征被命名的个体具有规定的属性这样一个单称命题。于是,单称命题"苏格拉底是人"可以符号化为 $H s$ 。上面提到的涉及谓项"人"的其他一些单称命题,分别可以符号化为 $H a 、 H b 、 H c$ 和 $H o$ 。我们注意到它们都有某种共同的模式,即它们不是被符号化为 $H$ 自身,而是 $H$ —。在此,"一"表示在谓项符号的右边有另一个符号即个体符号出现。习惯上用字母 $x$(这是可以的,因为我们只用从 $a$ 到 $w$ 的字母做个体变元)而不是用破折号 ("——")作替代标示。我们用 $H x$[有时写成 $H(x)$ ]来符号化所有以 "是人"作为个体属性的单称命题的共同模式。被称做个体变元的字母 $\boldsymbol{x}$只是一个位置标示,用来指示从 $a$ 到 $w$ 的各个字母——个体常元——可以

\section*{填入以便产生单称命题的位置。}
各种单称命题 $H a 、 H b 、 H c$ 和 $H d$ 是或真或假的;但由于 $H x$ 根本不是陈述或命题,它既不真也不假。表达式 $H x$ 是一个命题函项,它可以被定义成这样一个表达式:(1)含有个体变元;(2)当一个个体常元代入个体变元时,它就变成一个陈述。 ${ }^{[3]}$ 个体常元被认为是个体的专名。任何单称命题都是一个命题函项的代入例,是用个体常元代人该命题函项中的个体变元所产生的结果。一般说来,一个命题函项有真代人例和假代入例。到目前为止所讨论的命题函项——Hx、Mx、Fx、Bx 和 $W x$ ——都是这种类型。为了把它们与后面几节将介绍的更复杂的命题函项区分开,我们把这些命题函项叫"简单谓述"。因此,一个简单谓述是一个有一些真代人例和假代人例的命题函项,并且每个代人例都是一个单称肯定命题。 
\section{量化}

\begin{quotation}
本节讨论如何通过量化将命题函项转变为命题。我们将介绍全称量词和存在量词的概念及符号表示,分析它们之间的逻辑关系,并探讨如何使用量化符号来表达不同类型的普遍命题,从而为理解更复杂的逻辑结构奠定基础。
\end{quotation}

\subsection{从命题函项到命题}

用个体常元代人个体变元,并不是从命题函项获得命题的唯一方式。通过概括或量化程序也可以得到命题。谓项通常不仅仅出现在单称命题中。例如,命题"每个事物是有死的"和"有些事物是漂亮的"含有谓项,但不是单称命题,因为它们不含有任何特定个体的名称。相反,作为普遍命题,它们不特别指称任何特定个体。

第一个命题可以用各种不同的逻辑等价的方式表示:或者表示为"所有事物都是有死的",或者表示为:

给定不管任何个体事物,它都是有死的。

在后一种表述中,语词"它"是一个关系代词,回指该陈述中前面的语词 "事物"。用字母 $x$ ,即个体变元,代替代词"它"及其先行词,我们可以把第一个普遍命题重写为:

给定任何 $x, x$ 是有死的。

或者,用前一节所介绍的符号,我们可以写成:

给定任何 $x, M x$ 。

尽管命题函项 $M x$ 不是一个命题,但我们这里有了一个含有它的表述式,而这个表述式是命题。短语"给定任何 $x$"习惯上用符号"$(x)$"表示,称为全称量词。上述第一个普遍命题可以完全符号化为:\\
(x)$M x$

\subsection{存在量化与全称量化}

第二个普遍命题,即"有些事物是漂亮的",也可以表达成:

至少存在这样一个事物,它是漂亮的。

在后一个表述中,语词"它"也是一个关系代词,回指语词"事物"。用个体变元 $x$ 代替代词"它"及其先行词,我们可以把第二个普遍命题重写为:

至少存在这样一个 $x, x$ 是漂亮的。

或者,我们可以用给定符号把它写成:

至少存在这样一个 $x, B x$ 。

同样,尽管 $B x$ 是一个命题函项而不是命题,但我们这里又有一个含有它的表述式,这个表述式是命题。短语"至少存在这样一个 $x$"习惯上用符号"$(\exists x)$"表示,称为存在量词。第二个普遍命题可以完全符号化为:\\
$(\exists x) B x$

于是我们看到,命题可以用列举方法从命题函项生成,即通过用个体常元代入个体变元,或者可以用概括方法生成,即在它的前面放一个全称词或存在量词。

现在请考虑:一个命题函项的全称量化式 $(x) M x$ 为真,当且仅当,它的所有代人例都为真;这正是普遍性的意义之所在。很显然,一个命题函项的存在量化式 $(\exists x) M x$ 为真,当且仅当,它至少有一个真代人例。我们假定(没人会否认这一点)至少存在一个个体。在这种非常弱的假定

下,每个命题函项必定至少有一个代人例,这个实例或真或不真。但可以确定的是,在这种假定下,如果一个命题函项的全称量化式为真,那么它的存在量化式也必定为真。也就是说,如果每个 $x$ 都是 $M$ ,那么,如果至少存在一个事物,则这个事物是 $M$ 。

\subsection{否定与量化}

到此时为止,只举了单称肯定命题作为命题函项的代人例。 $M x(x$ 是有死的)是一个命题函项。 $M s$ 是它的一个实例,是一个单称肯定命题,即"苏格拉底是有死的"。但并非所有命题都是肯定的。一个人可以否认苏格拉底是有死的,即 $\sim M s$ ,"苏格拉底不是有死的"。如果 $M s$ 是 $M x$ 的一个代入例,那么,~Ms可以看成是命题函项 $\sim M x$ 的一个代人例。因此,我们可以超出前一节所介绍的简单谓述,把我们的命题函项概念扩大到能包括否定符"~"。

如下所示,使用否定符可以丰富我们对量化的理解。从下述普遍命题出发:

没有任何事物是完美的。

我们可以把它解释为:

每个事物都是不完美的。

它又可以写成:

给定不管任何个体事物,它不是完美的。

它可以改写成:

给定任何 $x, x$ 不是完美的。

如果用 $P$ 符号化属性"是完美的",用刚才给出的符号(量词和否定符),我们可以把这个命题("没有任何事物是完美的")表示为 $(x) \sim P x$ 。

\subsection{量词间的逻辑关系}

现在我们可以列出并举例说明全称量化和存在量化之间的一系列重要关系。

第一,(全称)普遍命题"每个事物都是有死的",被(存在)普遍命题"有些事物不是有死的"否定。我们可以用符号说成,( $x$ )$M x$ 被 ( $\exists x$ )~$M x$ 否定。因为它们每个都是另一个的否定,我们当然可以说 (从有否定符的那个开始),下述双条件陈述是必然真的、逻辑真的:

$$
\sim(x) M x \stackrel{\stackrel{\mathrm{~T}}{=}(\exists x) \sim M x}{ }
$$

第二,"每个事物都是有死的"正好表示了"不存在任何不是有死的事物"所表示的东西——这可以表述成另一个逻辑真的双条件陈述:

$$
(x) M x \stackrel{\mathrm{T}}{=} \sim(\exists x) \sim M x_{0}
$$

第三,很清楚,(全称)普遍命题"没有任何事物是有死的",被(存在)普遍命题"有些事物是有死的"否定。用符号我们可以说 $(x) \sim M x$被( $x$ )$M x$ 否定。既然它们每个都是另一个的否定,我们当然可以说(还从有否定符的那个开始),下列双条件陈述是必然的、逻辑真的:

$$
\sim(x) \sim M x \stackrel{\stackrel{\mathrm{~T}}{=}}{(\exists x) M x}
$$

第四,"每个事物都不是有死的"正好表示了"不存在任何有死的事物"所表示的东西——这可以表述成另一个逻辑真的双条件陈述:

$$
(x) \sim M x \stackrel{\mathrm{T}}{=} \sim(\exists x) M x
$$

这四个逻辑真的双条件陈述阐明了全称量词和存在量词的相互关系。任何一个否定符在量词之前的命题,(利用这些逻辑真的双条件陈述)我们都可以用另一个与其逻辑等价但量词前面没有否定符的命题替换之。现在以符号 $\phi$(希腊字母 phi)替换例示谓词 $M$(有死的),$\phi$ 代表任何一个简单谓词,我们立即可列出下面这四个双条件陈述式:

$$
\begin{aligned}
& {\left[(x) \phi_{x}\right] \stackrel{\mathrm{T}}{=}\left[\sim(\exists x) \sim \phi_{x}\right]} \\
& {\left[(\exists x) \phi_{x}\right] \stackrel{\mathrm{T}}{=}\left[\sim(x) \sim \phi_{x}\right]} \\
& {\left[(x) \sim \phi_{x}\right] \stackrel{\mathrm{T}}{=}\left[\sim(\exists x) \phi_{x}\right]} \\
& {\left[\exists(x) \sim \phi_{x}\right] \stackrel{\mathrm{T}}{=}\left[\sim(x) \phi_{x}\right]}
\end{aligned}
$$

\subsection{量词逻辑方阵}

全称量化和存在量化之间的一般关系,可以用图 10-1 中的方阵进行

更图示化的描述。\\
\includegraphics[width=\textwidth]{images/2025_05_15_6a28331d5e7c993ad07ag-463.jpg}

图10—1\\
继续假定至少存在一个个体,就该方阵我们可以说:\\
1.顶端的两个命题是反对关系;就是说,它们可以同时为假,但不能同时为真。

2.底端的两个命题是下反对关系;就是说,它们可以同时为真,但不能同时为假。

3.对角线相反两端的命题是矛盾关系;它们中一个为真,则另一个必定为假。

4.在方阵的每侧,下面命题的真被它正上方命题的真所蕴涵。 

\begin{center}
\fbox{\parbox{0.95\textwidth}{
\textbf{本节要点}
\begin{itemize}
\item \textbf{量化的基本概念}:
  \begin{itemize}
  \item 量化是将命题函项转变为命题的方法
  \item 区别于列举法(通过代入个体常元)
  \item 通过量词可表达普遍命题
  \end{itemize}
\item \textbf{两种基本量词}:
  \begin{itemize}
  \item 全称量词(x):表示"对所有x都成立"
  \item 存在量词(∃x):表示"至少存在一个x使...成立"
  \item 全称量化为真当且仅当所有代入例为真
  \item 存在量化为真当且仅当至少有一个代入例为真
  \end{itemize}
\item \textbf{量词与否定的关系}:
  \begin{itemize}
  \item ~(x)φx 等价于 (∃x)~φx
  \item (x)φx 等价于 ~(∃x)~φx
  \item ~(x)~φx 等价于 (∃x)φx
  \item (x)~φx 等价于 ~(∃x)φx
  \end{itemize}
\item \textbf{量词逻辑方阵}:
  \begin{itemize}
  \item 描述全称肯定、全称否定、特称肯定、特称否定命题间的关系
  \item 包含反对关系、下反对关系和矛盾关系
  \item 展示量化命题间的逻辑蕴涵关系
  \end{itemize}
\end{itemize}
}}
\end{center} 
\section*{10.3 传统主——谓命题}
运用存在和全称量词,以及根据对图 10-1 中对当方阵的理解,我们现在开始分析(并且在推理中准确地使用)以下四种为传统逻辑研究所注重的普遍命题。这四种命题的标准例子如下:

\begin{center}
\begin{tabular}{ll}
所有人是有死的。 & [全称肯定:A] \\
所有人都不是有死的。 & [全称否定:E] \\
有些人是有死的。 & [特称肯定:I] \\
有些人不是有死的。 & [特称否定:O] \\
\end{tabular}
\end{center}

每种命题通常由其字母来指称:两种肯定命题用 A 和 I(来自拉丁文\\
affirmo,我肯定);两种否定命题用 E 和 O (来自拉丁文 nego,我否认)。 ${ }^{[4]}$

用量词符号化这些命题,使得我们进一步扩大了命题函项概念。首先来看 A 命题"所有人是有死的",我们从下述命题开始逐次解释:

给定不管任何事物,如果它是人,它是有死的。

其中关系代词"它"的两次出现显然是回指它们共同的先行词"事物"。与上节的前部分一样,因为它们有同样的(不确定的)指称,从而都能用字母"$x$"替换。于是该命题可改写成:

给定任何 $x$ ,如果 $x$ 是人,那么 $x$ 是有死的。

现在,用先前引入的"如果一那么"的符号,可以把前一个命题改写成:

给定任何 $x, x$ 是人 $\supset x$ 是有死的。

最后,用我们已掌握的命题函项符号和量词,原来的 A 命题可表示为:

$$
(x)(H x \supset M x)
$$

在我们的符号翻译中, A 命题是以一种新的命题函项的全称量化形式出现的。表述式 $H x \supset M x$ 是一个命题函项,它既没有单称肯定命题又没有单称否定命题作为其代人例,而是以条件陈述作为代人例,这些条件陈述的前件和后件是具有同样主项的单称命题。命题函项 $H x \supset M x$ 的代人例有条件陈述 $H a \supset M a 、 H b \supset M b 、 H c \supset M c 、 H d \supset$ $M d$ 等等。

另一些命题函项则以有同样主项的单称命题的合取为代入例。例如, $H a \cdot M a 、 H b \cdot M b 、 H c \cdot M c 、 H d \cdot M d$ 等等都是命题函项 $H x \cdot M x$ 的代人例。还有一些形如 $W x \vee B x$ 的命题函项,它们的代入例是诸如 $W a \vee$ $B a$ 和 $W b \vee B b$ 这样的析取式。实际上,任何以具有相同主项的单称命题为分支陈述的真值函项复合命题,都可以看做由某些或所有真值函项联结词(圆点号、楔劈号、马蹄号、三杠等值号和波浪号)加之简单谓词\\
( $A x 、 B x 、 C x 、 D x \cdots \cdots$ )所构成的命题函项的代人例。在把 A 命题翻译成( x )( $H \mathrm{x} \supset M x$ )时,圆括号充当标点符号,用以表明全称量词( $x$ ) "作用于"整个(复合)命题函项 $H x \supset M x$ ,或命题函项 $H x \supset M x$"作为其辖域"。

在继续讨论直言命题的其他传统形式之前,应该注意符号公式( $x$ ) ( $H x \supset M x$ )不仅是对标准形式的命题"所有 $H$'s 都是 $M$'s"的翻译,而且是对任何一个有同样含义的自然语言句子的翻译。 ${ }^{[5]}$ 在自然语言中,述说这同一件事有许多不同的方式。它们的部分清单如下:"$H$ 是 $M$","一个 $H$ 就是一个 $M$","每个 $H$ 是 $M$","每一个 $H$ 是 $M$","任何 $H$ 是 $M$", "没有 $H$ 不是 $M$","是 $H$ 的每个事物都是 $M$","是 $H$ 的任何事物都是 $M "$ ,"如果任何事物是 $H$ ,那么它是 $M$","如果某事物是 $H$ ,那么它是 $M$","是 $H$ 的无论什么东西都是 $M$","$H$'$s$ 全都是 $M$'$s$","只有 $M$'$s$ 是 $H^{\prime} s "$ ,"除了 $M^{\prime} s$ 以外,没什么是 $H^{\prime} s "$ ,"没什么是 $H$ ,除非它是 $M^{\prime}$",以及"没有什么是 $H$ 但不是 $M$"。再者,同一含义的命题可以用抽象名词表达:"人蕴涵(或涵衍)有死"可以正确地符号化为一个 A 命题。符号逻辑语言对相当数量的自然语言句子的共同含义有一个单一的表达式,这一点被认为是符号逻辑在认知或信息方面比自然语言优越之处,一一尽管从修辞力或诗意表现的观点看,我们承认这是一种劣势。 E 命题"所有人都不是有死的"可以被依次释为:

\begin{displayquote}
给定不管任何个体事物,如果它是人,那么它不是有死的。\\
给定任何 $x$ ,如果 $x$ 是人,那么 $x$ 不是有死的。\\
给定任何 $x, x$ 是人 $\supset x$ 不是有死的。
\end{displayquote}

最后可以释为:

$$
(x)(H x \supset \sim M x)
$$

这种符号翻译不仅表示了自然语言中传统的 E 形式,同样也表示了一些说同一件事的不同方式,如"没有是 $M$ 的 $H$","没有什么既是 $H$ 又是 $M$",以及"$H$ 从不是 $M$"。

同样,I命题"有些人是有死的"可以依次释为:

至少有一个是人且有死的事物。\\
至少有这样一个 $x, x$ 是人并且 $x$ 是有死的。\\
至少有这样一个 $x, x$ 是人• $x$ 是有死的。

进而可以释为:

$$
[(\exists x)(H x \cdot M x)]
$$

最后, O 命题"有些人不是有死的"可以依次释为:

至少存在一个是人但不是有死的事物。\\
至少存在这样一个 $x, x$ 是人并且 $x$ 不是有死的。\\
至少存在这样一个 $x, x$ 是人• $\sim x$ 是有死的。

它可以完全符号化为:

$$
[(\exists x)(H x \cdot \sim M x)]
$$

若用希腊字母 phi( $\phi$ )和 psi( $\Psi$ )表示任何一个谓词,传统逻辑的四个主—谓型普遍命题可以在图 10—2 所示的方阵中得到表达。\\
\includegraphics[max width=\textwidth, center]{2025_05_15_6a28331d5e7c993ad07ag-466}

图10-2\\
显然, A 命题和 O 命题是矛盾关系,一个是另一个的否定; E 命题和 I 命题也是矛盾关系。

有人可能会认为,一个 I 命题可以从与之相对的 A 命题推出,一个 O命题可以从与之相对的 E 命题推出,但情况并非如此。一个 A 命题为真时,与之对应的 I 命题却可能是假的。如果 $\phi_{x}$ 是一个没有真代人例的命

题函项,那么,不管命题函项 $\Psi x$ 有何种代人例,(复合)命题函项 $\phi_{x} \supset$ $\Psi x$ 的全称量化式都是真的。例如,考虑命题函项"$x$ 是一个人首马身的怪物",我们把它简写为 $C x$ 。因为不存在人首马身的怪物,$C x$ 的每个代入例都是假的,即 $C a 、 C b 、 C c \cdots \cdots$ 都为假。因此,复合命题函项 $C x \supset B x$的每个代人例都是一个前件为假的条件陈述。这样,其代人例 $\mathrm{Ca} \supset \mathrm{Ba}$ 、 $C b \supset B b 、 C c \supset B c$ 都是真的,因为任何一个断言实质蕴涵的条件陈述,如果其前件为假,那么它必定为真。由于其所有代人例都是真的,所以,命题函项 $C x \supset B x$ 的全称量化式为真,即 A 命题 $(x)(C x \supset B x)$ 为真。但与之相对的 I 命题 $(\exists x)(C x \cdot B x)$ 却是假的,因为命题函项 $C x \cdot B x$没有真代人例。 $C x \cdot B x$ 没有真代人例可以从 $C x$ 没有真代人例推出。 $C x \cdot B x$ 的各个代人例,如 $C a \cdot B a 、 C b \cdot B b 、 C c \cdot B c \cdots \cdots$ 都是第一个合取支为假的合取式,因为 $C a 、 C b 、 C c \cdots \cdots$ 都为假。由于其所有代人例都为假,所以命题函项 $C x \cdot B x$ 的存在量化式为假,即 I 命题( $\exists x$ )( $C x \cdot B x$ )为假。因此,有可能一个 A 命题是真的,而与之对应的 I 命题却是假的。

这种分析还可表明,为什么有可能一个 E 命题是真的,而与之对应的 $O$ 命题却是假的。如果我们以命题函项 $\sim B x$ 替换前面讨论中的命题函项 $B x$ ,那么,$(x)(C x \supset \sim B x)$ 可以是真的,而 $(\exists x)(C x \cdot \sim B x)$ 却是假的。当然,这也是因为并没有人首马身的怪物。

问题的关键在于:A 命题和 E 命题并不断言或假定任何事物存在,它们仅断言情况是这样的:如果有某事,则有另外一件事。但 I 命题和 O 命题却假定某物存在,它们断言情形是这样的:有这件事并且有另一件事。 1 命题和 $O$ 命题中的存在量词是区别的关键所在。从一个并不断言或假定任何事物存在的命题推出某物的存在,这显然是错误的。

如果我们假定至少有一个个体存在,那么( $x$ )( $C x \supset B x$ )确实蕴涵 ( $\exists x) ~(C x \supset B x)$ 。但后者不是一个 I 命题。 I 命题"有些人首马身的怪物是漂亮的"应符号化为 $(\exists x)(C x \cdot B x)$ ,它说的是,至少存在一个漂亮的人首马身的怪物。但在自然语言中,被符号化为( $\exists x$ )( $C x \supset B x$ )的东西,可以被理解为"至少存在一个如此这般的事物,如果它是人首马身的怪物,那么它是漂亮的"。它并没说存在一个人首马身的怪物,而只是说存在一个个体,它或者不是人首马身的怪物,或者是漂亮的。而这个命题只在两种情况下是假的:第一,如果根本不存在个体;第二,如果所有个体都是人首马身的怪物,并且它们当中没有一个是漂亮的。通过作这样

一个明确的(并且显然是真的)假定,即假定宇宙中至少存在一个个体,我们可以排除第一种情形。第二种情形是如此极端地不合理,以致与 I 形命题( $\exists x$ )( $\phi_{x} \cdot \Psi x$ )的重要性相反,任何形如( $\exists x$ )( $\phi_{x} \supset \Psi x$ )的命题都必定是非常平庸的。显而易见,尽管在自然语言中 A 命题"所有人是有死的"和 I 命题"有些人是有死的"的区别,仅在于初始词"所有"和"有些"的不同,但它们意义上的区别并不限于全称量化和存在量化,而是比这深刻得多。经量化而产生 A 命题和 I 命题的命题函项不仅在量化上有区别,而且它们还是不同的命题函项,一个含有"$\supset$",另一个含有"•"。换言之,A 命题和 I 命题并不像它们在自然语言中看起来那么相似。它们之间的区别可通过使用命题函项符号和量词符号得以彰显。

就逻辑操作来说,处理那些否定号的出现——如果有否定号出现的话——只作用于简单谓述的公式最为方便。因此,我们将在必要时通过替换来得到这种公式。要做到这一点很简单。从第9章所确立的推论规则可知,我们可以用另一个与之逻辑等价的表述式来替换一个表述式。而我们有四个这样的逻辑等价式(10.2节),它们当中否定号在量词之前的命题,都与另一个否定号直接作用于简单谓述的命题等价。用我们熟悉已久的推论规则,可以移动否定号,使它们最终不再作用于复合表达式,而只作用于简单谓述。譬如说,公式:

$$
\sim(\exists x)(F x \cdot \sim G x)
$$

可以依次改写。首先,如果我们用 10.2 节所给的第三个逻辑等价式,它可以变形为:

$$
(x) \sim(F x \cdot \sim G x)
$$

然后,可运用德摩根律使之变成:

$$
(x)(\sim F x \vee \sim \sim G x)
$$

再用双重否定律可得公式:

$$
(x)(\sim F x \vee G x)
$$

最后,若援引实质蕴涵定义,原公式也可以改写成下述 A 命题:

$$
(x)(F x \supset G x)
$$

在转到关于非复合陈述推论的话题之前,读者应该进行一些把非复合陈述从自然语言翻译成逻辑符号的训练。自然语言有如此之多不规则的和惯用的构造,以致不可能有把自然语言语句翻译成逻辑符号的简单规则。在任何情形下都要先理解语句的含义,然后用命题函项和量词术语予以重述。

\section{有效性证明}

\begin{quotation}
本节讨论如何证明那些有效性取决于非复合陈述内在结构的论证。我们将介绍四个新的推论规则:全称列举、全称概括、存在列举和存在概括,通过这些规则,我们能够构造出涉及量化命题的形式证明,从而扩展我们的逻辑分析工具,使之能够处理更复杂的论证形式。
\end{quotation}

\subsection{扩充推论规则}

某些论证的有效性取决于在其中出现的非复合陈述的内在结构,为了构造它们有效性的形式证明,我们必须进一步扩充推论规则表。只需增加四个规则,我们将在涉及必须使用它们的那些论证时逐次引人。

考虑本章所引的第一个论证:"所有人都是有死的。苏格拉底是人。因此,苏格拉底是有死的。"可以符号化为:

$$
(x)(H x \supset M x)
$$

Hs\\
$\therefore M s$\\
第一个前提断定了命题函项 $H x \supset M x$ 的全称量化式。由于一个命题函项的全称量化式为真,当且仅当它的所有代人例都为真,从第一个前提可以推出命题函项 $H x \supset M x$ 的任何一个我们需要的代人例。此处即可以推出代人例 $H s \supset M s$ 。从它和第二个前提 $H s$ ,根据肯定前件式,可以直接得出结论 $M s$ 。

如果给我们的推论规则表加上这样一个原则,即一个命题函项的任一代入例都可以有效地从其全称量化式推得,那么,依照扩充了的基本有效论证形式表,我们可以给出该论证之有效性的形式证明。这种新的推论规则就是\textbf{全称列举原则},简写为"UI"。用希腊字母 $n u(v)$ 表示任一个体符号,我们可以把该新规则表述为:

$$
\begin{aligned}
& \mathrm{UI}:(x)(\phi x) \\
& \quad \therefore \phi v \quad(v \text { 是任一个体符号 })
\end{aligned}
$$

其有效性的形式证明现在可以写成:

\begin{center}
\begin{tabular}{ll}
1.$(x)(H x \supset M x)$ &  \\
2.$H s$ &  \\
 & $\therefore M s$ \\
3.$H s \supset M s$ & $1, \mathrm{UI}$ \\
4.$M s$ & $3,2, \mathrm{M} . \mathrm{P}$. \\
\end{tabular}
\end{center}

\subsection{全称概括原则}

增加 UI 大大地强化了我们的证明工具,但我们还需要更多的规则。需要另一些支配量化的规则,这种需要是和这样的论证相联系的,如"所有人都是有死的。所有希腊人都是人。因此所有希腊人都是有死的。"这个论证的符号翻译是:

$$
\begin{aligned}
& (x)(H x \supset M x) \\
& (x)(G x \supset H x) \\
& \therefore(x)(G x \supset M x)
\end{aligned}
$$

其中,前提和结论都是普遍命题而不是单称命题,是命题函项的全称量化式而不是其代人例。根据 UI,我们可以有效地从这两个前提推出下述条件陈述对子:

$$
\begin{array}{lllll}
G a \supset H a & G b \supset H b & G c \supset H c & G d \supset H d & \\
H a \supset M a & H b \supset M b & H c \supset M c & H d \supset M d & \ldots \ldots
\end{array}
$$

通过连续使用假言三段论规则,我们可以有效地推出结论:

$$
G a \supset M a, \quad G b \supset M b, \quad G c \supset M c, \quad G d \supset M d \quad \ldots \ldots
$$

如果 $a, b, c, d \cdots \cdots$ 是所有存在的个体,那么,我们从前提的真就可以有效地推出命题函项 $G x \supset M x$ 的所有代人例的真。由于一个命题函项的全称量化式为真,当且仅当,它的所有代人例都为真,我们可以继续推出 ( $x$ )( $G x \supset M x$ )为真,而它就是该论证的结论。

前面一段可以看做构成了上述论证有效性的一个非形式的证明,在证明中运用了假言三段论规则和支配量化的两个规则。它描述了一个长度不确定的陈述序列:前提中两个被全称量化的命题函项的所有代人例的序列,以及其全称量化式是结论的那个命题函项的所有代人例的序列。一个形式证明不能包含这样的不确定的甚或无限长的陈述序列。因此,必须寻求某种方法,它能以某种有限的、确定的方式来表达这些长度不确定的序列。

基础数学的一个一般技巧为做到这一点提供了提示。一个试图证明所有三角形都具有某种属性的几何学者,可以从"令 $A B C$ 是一个任意选取的三角形"出发。然后对三角形 $A B C$ 进行推理,确立它具有被探究的那种属性,由此可得出结论,所有三角形具有该属性。是什么东西能为他的最后结论进行辩护呢?承认这个特定的三角形 $A B C$ 有该属性,为什么可以得出所有的三角形都有这种属性?答案很容易见得:如果除了假定它是三角形外,我们对三角形 $A B C$ 没作任何其他假定,那么,符号"$A B C$"可以被看做是指称你所挑选的任何三角形。几何学者的论证确立了任一三角形具有所探究的属性,而如果任一三角形都具有某属性,那么所有的三角形都具有该属性。我们现在也可引进一个符号,它类似于几何学者所谈论的"一任意选取的三角形 $A B C$"。这使我们可以谈论某命题函项的任一代人例,而不用去罗列其不确定的或无限数量的代人例。

我们将用(迄今还没用过)小写字母 $y$ 来指称一任意选取的个体,以一种类似于几何学者使用字母 $A B C$ 的方式来使用它。由于从一个命题函项的全称量化式可以推出它的任一代人例,故亦可推出以 $y$ 替换 $x$ 所得到的那个代人例。在此,$y$ 指称"一任意选取的个体"。这样,我们可以着手进行上述论证有效性的形式证明:

1.$(x)(H x \supset M x)$\\
2.$(x)(G x \supset H x)$

$$
\therefore(x)(G x \supset M x)
$$

\begin{center}
\begin{tabular}{ll}
3.$H y \supset M y$ & 1, UI \\
4.$G y \supset H y$ & 2, UI \\
5.$G y \supset M y$ & 4,3, H.S. \\
\end{tabular}
\end{center}

我们从前提演绎出了陈述 $G y \supset M y$ ,由于 $y$ 指称"一任意选取的个体",所以,该陈述实际上是断言命题函项 $G x \supset M x$ 的任一代人例为真。既然任一代人例为真,所有的代人例必定为真,因此,该命题函项的全称量化也必是真的。我们可以把这个原则加到推论规则表中,表述如下:从一个

命题函项关于任意选取的个体名称的代入例,我们可以有效地推出该命题函项的全称量化式。这个规则允许我们进行概括,也就是从一个特定的代人例进到一个概括的或全称量化的表述式,故称为\textbf{全称概括原则},并缩写为"UG"。它被表述成:

$$
\begin{aligned}
\mathrm{UG}: & \phi_{y} \\
\therefore(x)\left(\phi_{x}\right) & (y \text { 指称"一任意选取的个体") }
\end{aligned}
$$

前面的形式证明的第六行即最后一行,现在就可以写(并被证明)为:\\
6.$(x)(G x \supset M x)$\\
5,UG

我们来回顾一下前面的讨论。在几何学者的证明中,对 ABC 所作的唯一假定就是它是一个三角形,因此,被证明为对 ABC 为真的东西也就被证明为对任一三角形为真。在我们的证明中,对 $y$ 所作的唯一假定是它是一个个体词,因此,被证明为对 $y$ 为真的东西也就被证明为对任一个体为真。符号 $y$ 是一个个体符号,但它是一个很特殊的个体符号。特别是通过使用 UI,它被引人到证明中,并且只有当出现了 $y$ 时才允许使用UG。

以下是另一个有效论证,它的有效性的证明要求使用 UG 和 UI:"没有人是完美的。所有希腊人都是人。因此没有希腊人是完美的。"${ }^{[7]}$ 它的有效性的形式证明是:

1.$(x)(H x \supset \sim P x)$\\
2.$(x)(G x \supset H x)$

$$
\therefore(x)(G x \supset \sim P x)
$$

3. $\mathrm{H} y \supset \sim \mathrm{P} y \quad 1$ ,UI\\
4.GyつHy 2,UI\\
5.Gyコ~Py 4,3,H.S.\\

6.$(x)(G x \supset \sim P x) \quad$ 5,UG\\
上面的证明看起来多少有点不自然,需要我们对 $(x)\left(\phi_{x}\right)$ 和 $\phi_{y}$ 做出仔细的区分。说它们尽管不同但根据 UG 和 UI 又必定可以相互推出,似乎二者之间没有实质差别。但它们之间确实有一种形式的差别。陈述 ( $x$ )( $H x \supset M x$ )是一个非复合陈述,而 $H y \supset M y$ 作为一个条件陈述,是一个复合陈述。依照原先含有 19 个规则的推论规则表,从两个非复合陈述 $(x)(G x \supset H x)$ 和 $(x)(H x \supset M x)$ 出发,我们不能作相关的推理。

但从复合陈述 $G y \supset H y$ 和 $H y \supset M y$ 出发,根据假言三段论,就可以得出所要的结论 $G y \supset M y$ 。规则 $U I$ 用来从非复合陈述得出复合陈述,我们先前的推论规则无法施于非复合陈述,但可以施于复合陈述以得出想要的结论。因此,量化规则增加了我们的逻辑工具,使得我们能够证明本质地涉及非复合(概括的)命题的论证的有效性,以及前一些章节所讨论的另一类(更简单的)论证的有效性。另一方面,尽管有这种形式的差别,( $x$ ) ( $\phi_{x}$ )和 $\phi_{y}$ 必定是逻辑等价的,否则,规则 UG 和 UI 就不是有效的。对依据推论规则表来证明论证的有效性来说,这种差别和逻辑等价都很重要。把 UG 和 UI 加到推论规则表中使之得到了很大强化。

\subsection{存在量化规则}

当我们转向涉及存在命题的论证时,推论规则表必须进一步扩充。我们可从这样一个很便捷的例子着手:"所有罪犯都是邪恶的。有些人是罪犯。因此有些人是邪恶的。"它可以符号化为:

$$
\begin{aligned}
& (x)(C x \supset V x) \\
& (\exists x)(H x \cdot C x) \\
& \therefore(\exists x)(H x \cdot V x)
\end{aligned}
$$

一个命题函项的存在量化式为真,当且仅当,它至少有一个真代人例。因此,无论 $\phi$ 指谓何种属性,( $\exists x$ )( $\phi_{x}$ )所说的就是,至少存在一个具有属性 $\phi$ 的个体。如果一个个体常元(除了特定的符号 $y$ )在早先的上下文中没有使用过,我们可以用它来指称具有属性 $\phi$ 的那个个体,或者,如果有几个具有属性 $\phi$ 的个体,用它指称其中的某一个。若知道存在这样一个个体,臂如 $a$ ,我们就知道 $\phi_{a}$ 是命题函项 $\phi_{x}$ 的一个真代人例。故我们给推论规则表加上这样一个规则:从一个命题函项的存在量化式,可以推得关于在其语境中早先没有出现过的任一个体常元(除 $y$ 之外)的代入例。这个新推论规则叫\textbf{存在列举原则},可缩写为"EI"。它可以表述成:

$$
\mathrm{EI}:(\exists x)\left(\phi_{x}\right)
$$

$\therefore \phi U \quad[U$ 是任一在语境中先前没有出现过的个体常元(除 $y$之外)〕

如果确认所添加的推理规则 EI,我们即可着手证明上述论证的有效性:

1.$(x)(C x \supset V x)$

2.$(\exists x)(H x \cdot C x)$

$$
\therefore(\exists x)(H x \cdot V x)
$$

3. $\mathrm{Ha} \cdot \mathrm{Ca}$ 2,EI\\
4. $\mathrm{Ca} \supset \mathrm{Va} \quad 1$ ,UI\\
5. $\mathrm{Ca} \cdot \mathrm{Ha}$ 3,Com.\\
6. Ca 5,Simp.\\
7.Va 4,6,M.P.\\
8. Ha 3 ,Simp.\\
9. $\mathrm{Ha \cdot Va} \quad 8,7$, Conj.\\
到目前为止,我们演绎出了 $\mathrm{Ha} \cdot \mathrm{Va}$ ,它是其存在量化式被结论所断定的那个命题函项的代人例。由于一个命题函项的存在量化式为真,当且仅当,它至少有一个为真的代人例,我们为推论规则表再增加这样一个规则:从一个命题函项的任一为真的代入例,我们可以有效地推出该命题函项的存在量化式。这第四个也是最后一个推论规则叫\textbf{存在概括原则},缩写为"EG",它可以表述为:

$$
\begin{gathered}
\text { EG: } \phi_{v} \quad(v \text { 是任一个体符号) } \\
\therefore(\exists x)\left(\phi_{x}\right)
\end{gathered}
$$

前面开始的那个证明的第十行也即最后一行,现在可以写(并且被证明)为:

$$
\text { 10. }(\exists x)(H x \cdot V x) \quad 9, \mathrm{EG}
$$

\subsection{使用限制}

对 EI 的使用必须施加必要的限制,这一点可以通过考察如下明显无效的论证看出来:"有些短伆鳄被关在笼子里。有些鸟被关在笼子里。因此有些短吻鳄是鸟。"如果我们不对 EI 施加这样一种限制——根据 EI 从一个命题函项的存在量化式推出的代人例,只能含有一个在语境中早先没出现过的个体符号(除 $y$ 之外),那么,我们就可以构造出这个无效论证的有效性"证明"。这样一个错误的"证明"可以如下进行:

1.$(\exists x)(A x \cdot C x)$\\
2.$(\exists x)(B x \cdot C x)$

$$
\therefore(\exists x)(A x \cdot B x)
$$

4. $\mathrm{Ba} \cdot \mathrm{Ca}$\\
5.$A a$\\
6.$B a$\\
7.$A a \cdot B a$\\
8.$(\exists x)(A x \cdot B x)$

2,EI(错!)\\
3,Simp.\\
4,Simp.\\
5,6,Conj.\\
7,EG

这个"证明"的错误出现在第 4 行。我们从第二个前提 $(\exists x)(B x \cdot C x)$可知,至少存在这样一个事物,它既是鸟又被关在笼子里。如果我们在第 4 行给它自由地指派一个名称 $a$ ,我们当然就可以断言 $B a \cdot C a$ 。但我们绝不能自由地指派这样一个"$a$",因为它作为一只关在笼子里的短吻鳄的名字,已经先在第 3 行中出现了。为避免这种错误,我们使用 EI 时必须服从这种必要的限制。由前面的讨论可明显见得:在任何要使用 EI 和 UI 的证明中,应该总是先使用 EI 。

对更复杂的论证模式来说,特别是那些涉及关系的论证,我们还必须对四个量化规则施加某些附加限制。但就目前这种类型的论证即传统上叫做直言三段论的论证来说,目前的限制已足以避免出错。

\subsection{四个附加推论规则总结}

下述四个规则使得我们可以把非复合的、概括的命题转化为与其等值的复合命题,第 9 章所列的那 19 个推论规则适用于这些复合命题。它们还使我们可以把复合命题转化为等值的非复合命题。因此,这四个附加规则使得构造某些论证的有效性的形式证明成为可能,这些论证的有效性取决于它们所包含的一些非复合陈述的内在结构。这四个附加规则如下:

\paragraph{1.全称列举}
UI:$(x)(\phi x), \therefore \phi \cup$(在此,$v$ 是任一个体符号)\\
这个规则大体上说的是:一个命题函项的任何代入例都可以从它的全称量化式推出。

\paragraph{2.全称概括}
UG:$\phi_{y}, \therefore(x)\left(\phi_{x}\right)$(在此,$y$ 指称"一任意选取的个体")\\
这个规则大体上说的是:从一个命题函项关于一任意选取的个体名称的代入例,我们可以有效地推出该命题函项的全称量化式。

\paragraph{3.存在列举}
EI:$(\exists \mathrm{x})\left(\phi_{\mathrm{x}}\right), \therefore \phi_{\mathrm{u}}$[在此,$v$ 是任一在上下文中先前没有出现

过的个体常元(除了 y)〕\\
这个规则大体上说的是:从一个命题函项的存在量化式,我们可以推出,它关于早先上下文的任何地方都没出现的任一个体常元(除了 $y) ~$ 的代入例为真。

\paragraph{4.存在概括}
EG:$\phi_{v}, \therefore(\exists x)\left(\phi_{x}\right)$(在此,$v$ 是任一个体符号)\\
这个规则大体上说的是:从一个命题函项的任一为真的代入例,我们可以有效地推出该命题函项的存在量化式。 

\begin{center}
\fbox{\parbox{0.95\textwidth}{
\textbf{本节要点}
\begin{itemize}
\item \textbf{四个量化推论规则}:
  \begin{itemize}
  \item 全称列举(UI):从(x)(φx)推出φv
  \item 全称概括(UG):从φy推出(x)(φx)
  \item 存在列举(EI):从(∃x)(φx)推出φv
  \item 存在概括(EG):从φv推出(∃x)(φx)
  \end{itemize}
\item \textbf{规则使用的重要限制}:
  \begin{itemize}
  \item UG规则中,y必须是"任意选取的个体"
  \item EI规则中,v必须是先前未使用过的个体符号
  \item 在证明中应先使用EI再使用UI
  \item 这些限制避免了错误的推论
  \end{itemize}
\item \textbf{规则的功能和意义}:
  \begin{itemize}
  \item 使我们能够在非复合陈述和复合陈述间转换
  \item 使已有的19个推论规则能够应用于量化表达式
  \item 扩展了我们处理更复杂论证形式的能力
  \item 使我们能够形式化地证明直言三段论的有效性
  \end{itemize}
\item \textbf{规则的应用}:
  \begin{itemize}
  \item 可以处理类似"所有S是M,所有P是S,因此所有P是M"的传统三段论
  \item 可以证明包含存在命题的论证的有效性
  \item 为分析非复合陈述的内在结构提供了必要工具
  \end{itemize}
\end{itemize}
}}
\end{center} 
\section{无效性证明}

\begin{quotation}
本节讨论如何证明涉及量词的论证的无效性。通过引入特定的模型和真值指派方法,我们可以系统性地处理含有量化命题的论证,并确定它们是否无效。相比于简单的逻辑类比,这种方法提供了更加严格和普遍的无效性证明技术。
\end{quotation}

\subsection{逻辑类比的局限性}

要证明一个涉及量词的论证无效,我们可以用逻辑类推进行反驳的方法。例如:"所有保守派都是行政机关的反对者;有些代表是行政机关的反对者;因此,有些代表是保守派。"这个论证可以通过这样一个逻辑类推被证明为无效,即"所有猫都是动物;有些狗是动物;因此,有些狗是猫"。这个论证显然无效,因为已知它的前提为真而结论为假。但这种类比并非总是很容易构造。因此,需要某种更有力的证明无效性的方法。

\subsection{真值指派方法的扩展}

在前一章中,我们详述了一种证明涉及真值函项复合陈述的论证之无效性的方法。这种方法是通过对论证中的简单分支陈述进行真值指派,使得论证的前提为真而结论为假。我们可以设法使这种方法适用于使用量词的论证。这涉及这样一个一般假定,即至少存在一个个体。若一个涉及量词的论证有效,那么,只要至少有一个个体存在,这个论证的前提为真而结论为假就必定是不可能的。

\subsection{有限域模型中的量化}

如果恰好存在一个个体,两个个体,三个个体……那么,至少存在一个个体这个一般假定就得到了满足。如果作了任何这样一个关于个体的确切数量的假定,就有一个关于普遍命题与单称命题的真值函项复合式的等价式。如果刚好存在一个个体,譬如说 $a$ ,那么:

$$
(x)\left(\phi_{\mathrm{x}}\right) \stackrel{\mathrm{T}}{=} \phi_{a} \stackrel{\mathrm{T}}{=}(\exists x)\left(\phi_{x}\right)
$$

如果刚好存在两个个体,臂如说 $a$ 和 $b$ ,那么:

$$
(x)\left(\phi_{x}\right) \stackrel{\mathrm{T}}{\equiv}\left[\phi_{a} \cdot \phi_{b}\right] \text {, 而 }(\exists x)\left(\phi_{x}\right) \stackrel{\mathrm{T}}{\equiv}\left[\phi_{a} \vee \phi_{b}\right]
$$

如果刚好存在三个个体,譬如说 $a 、 b$ 和 $c$ ,那么:

$$
(x)\left(\phi_{x}\right) \stackrel{\mathrm{T}}{=}\left[\phi_{a} \cdot \phi_{b} \cdot \phi_{c}\right] \text {, 而 }(\exists x)\left(\phi_{x}\right) \stackrel{\mathrm{T}}{=}\left[\phi_{a} \vee \phi_{b} \vee \phi_{c}\right]
$$

一般的,如果刚好存在 $n$ 个个体,譬如说 $a 、 b 、 c \cdots \cdots n$ ,那么:

$$
\begin{aligned}
& (x)\left(\phi_{x}\right) \stackrel{\mathrm{T}}{=}\left[\phi_{a} \cdot \phi_{b} \cdot \phi_{c} \cdots \cdot \phi_{n}\right] \\
& \text { 而 }(\exists x)\left(\phi_{x}\right) \cong\left[\phi_{a} \vee \phi_{b} \vee \phi_{c} \vee \cdots \vee \phi_{n}\right]
\end{aligned}
$$

由于它们是我们关于全称和存在量词定义的推论,所以这些双条件陈述为真。这里并没有用到前一节所阐释的四个量化规则。

\subsection{使用模型证明无效性}

一个涉及量词的论证有效,当且仅当,不管存在多少个体它都是有效的,一一假定至少存在一个个体的话。因此,如果存在一个至少含有一个个体的可能域或\textbf{模型},它使得某论证相对该模型来说,其前提为真而结论为假,那么,这样一个涉及量词的论证就被证明为无效。考察论证:"所有雇佣兵都是不可靠的。没有游击队员是雇佣兵。因此没有游击队员是不可靠的。"它可以符号化为:

$$
\begin{aligned}
& (x)(M x \supset U x) \\
& (x)(G x \supset \sim M x) \\
& \therefore(x)(G x \supset \sim U x)
\end{aligned}
$$

如果刚好存在一个个体,譬如说 $a$ ,这个论证逻辑地等价于:

$$
\begin{aligned}
& M a \supset U a \\
& G a \supset \sim M a \\
& \therefore G a \supset \sim U a
\end{aligned}
$$

给 $G a$ 和 $U a$ 指派真值真,给 Ma 指派真值假,即可以证明上式是无效的。 (这种真值指派是一种简略的描述方式,它把所讨论的模型描述成只含有一个个体 $a$ ,这个个体是游击队员且不可靠,但不是雇佣兵。)于是,原来的论证对于一个只含有一个个体的模型来说不是有效的,因此它是无效

的。类似的,通过描述只含有一个个体 $a$ 的模型,使得 $A a$ 和 $D a$ 被赋值为真,且 $C a$ 被赋值为假,我们就可以证明本节提到的第一个论证的无效性。 ${ }^{[8]}$

\subsection{逐渐扩大模型的方法}

有些论证对于刚好只有一个个体的模型来说是有效的,但对于有两个或更多个体的模型来说则不然。譬如:

$$
\begin{aligned}
& (\exists x) F x \\
& \therefore(x) F x
\end{aligned}
$$

这样的论证必须被当做是无效的,因为只要至少存在一个个体,那么,一个有效的论证就必定有效而不管存在多少个体。这种论证的另一个例子是:"所有牧羊犬都是可爱的。有些牧羊犬是看门狗。因此,所有看门狗都是可爱的。"它的符号翻译是:

$$
\begin{aligned}
& (x)(C x \supset A x) \\
& (\exists x)(C x \cdot W x) \\
& \therefore(x)(W x \supset A x)
\end{aligned}
$$

对一个刚好只有一个个体 $a$ 的模型来说,该论证逻辑地等价于:

$$
\begin{aligned}
& C a \supset A a \\
& C a \cdot W a \\
& \therefore W a \supset A a
\end{aligned}
$$

这个论证是有效的。但对一个有两个个体譬如 $a$ 和 $b$ 的模型来说,它逻辑地等价于:

$$
\begin{aligned}
& (C a \supset A a) \cdot(C b \supset A b) \\
& (C a \cdot W a) \vee(C b \cdot W b) \\
& \therefore(W a \supset A a) \cdot(W b \supset A b)
\end{aligned}
$$

通过对 $C a 、 A a 、 W a 、 W b$ 指派真,对 $C b 、 A b$ 指派假,可以证明该论证无效。于是,原论证对一个刚好有两个个体的模型来说不是有效的,因此它是无效的。对任何这种一般类型的无效论证来说,有可能描述一个含有有限数量个体的模型,用真值指派的方法可以证明,与这个论证逻辑等价的真值函项论证相对于该模型是无效的。

需要再次强调:在从一个涉及普遍命题的论证转化为一个真值函项论

证(相对于某特定模型,它逻辑等价于给定论证)的过程中,并没有用到我们的那四个量化规则。相反,真值函项论证的每个陈述,逻辑地等价于给定论证中与之对应的普遍命题。这种逻辑等价可以由本节中早些时候所阐述的那些双条件陈述来解释。相对于所讨论的那个模型,它们的逻辑真可以从全称量词和存在量词的定义推出。

\subsection{无效性证明的一般步骤}

证明一个含有普遍命题的论证无效的程序如下。首先,考察一个只含有一个个体 $a$ 的\textbf{一元模型}。然后,写出该论证相对于此模型的逻辑等价真值函项论证。通过把原论证的每个普遍命题(量化的命题函项)转化为该命题函项关于 $a$ 的代人例,就可以做到这一点。如果对它的简单分支陈述进行真值指派可以证明该真值函项论证无效,那么这就足以证明原论证无效。如果不能做到这一点,就接着考察一个含有两个体 $a$ 和 $b$ 的\textbf{二元模型}。为了得到相对于这个更大模型来说逻辑等价的真值函项论证,我们可以简单地把原来关于 $a$ 的每个代人例和一个关于 $b$ 的新代人例结合起来。这种"结合"必须依照前面所陈述的那些逻辑等价式。也就是说,在原论证含有一个全称量化的命题函项( $x$ )( $\phi_{x}$ )时,就用合取("•")把新的代人例 $\phi_{b}$ 和第一个代人例 $\phi_{a}$ 结合起来;在原论证含有一个存在量化的命题函项( $\exists x$ )( $\phi_{x}$ )时,就用析取(" V ")把新的代人例 $\phi_{b}$ 和第一个代入例 $\phi_{a}$ 结合起来。前述例子说明了这种程序。如果对它的简单分支陈述进行真值指派可以证明该真值函项论证无效,那么这就足以证明原论证无效。如果做不到这一点,就接着考察一个含有个体 $a 、 b$ 和 $c$ 的三元模型等等。本书中没有哪个习题要求一个含有超过三个元素的模型。 

\begin{center}
\fbox{\parbox{0.95\textwidth}{
\textbf{本节要点}
\begin{itemize}
\item \textbf{无效性证明方法}:
  \begin{itemize}
  \item 逻辑类比方法:构造结构相似但明显无效的类比论证
  \item 真值指派方法:在特定模型中找到使前提为真而结论为假的情况
  \item 模型扩展法:从一元模型开始,逐渐扩大模型规模直到找到反例
  \end{itemize}
\item \textbf{有限域中的量化等价式}:
  \begin{itemize}
  \item 一元模型:(x)(φx) ≡ φa ≡ (∃x)(φx)
  \item 二元模型:(x)(φx) ≡ [φa·φb],(∃x)(φx) ≡ [φa∨φb]
  \item 三元模型:(x)(φx) ≡ [φa·φb·φc],(∃x)(φx) ≡ [φa∨φb∨φc]
  \item 这些等价式基于量词的定义,而非推论规则
  \end{itemize}
\item \textbf{无效性的定义}:
  \begin{itemize}
  \item 论证有效当且仅当在任何模型中前提为真时结论也为真
  \item 只要存在一个模型使前提为真而结论为假,论证就是无效的
  \item 有些论证在一元模型中有效,在更大模型中无效
  \end{itemize}
\item \textbf{实际证明步骤}:
  \begin{itemize}
  \item 从一元模型开始,将量化命题转换为真值函项命题
  \item 若找不到反例,扩展到二元模型,按相应规则合并新代入例
  \item 全称量化用合取连接各代入例,存在量化用析取连接各代入例
  \item 在扩展的模型中寻找使前提为真而结论为假的真值指派
  \end{itemize}
\end{itemize}
}}
\end{center} 
\input{chapter10/section10-6.tex}

% 第十一章
\chapter{逻辑与哲学}
\section*{11.1 类比论证}
前几章讨论的是演绎论证。演绎论证是否有效,取决于其前提是否能够证明地(demonstratively)得到结论。然而,还有许多良好的和重要的论证,这些论证的结论不能得到确定性的证明。我们充分相信许多因果连接(causal connections),只是基于盖然性(probability)一一尽管盖然性程度可能非常高。我们能够不加迟疑地说,吸烟是癌症的一个原因,但我们不能够赋予我们的这种知识与从前提中推得一个演绎有效的论证结论这种知识以相同的确定性。一个著名的医科专家根据演绎标准声称:"没有人将能够证明(prove)吸烟导致癌症,或者说任何事情导致任何事情。从理论上讲,你不能够证明任何事情。"${ }^{[1]}$ 的确,当我们评价我们关于世界的事实的知识时,演绎确定性的标准太高了。

本章及以后的各章将转向分析这样的论证:人们在这些论证中并不声称结论的真理性是从前提必然地得到,而仅仅表明,前提对结论的支持是或然的(probable),或者说结论盖然为真。这种论证被称为归纳论证,其与演绎论证具有根本性差异。我们已经在第1章中讨论了演绎和归纳之间的基本区别。本书第二部分已经对演绎进行了讨论,第三部分则用来讨论归纳。

在归纳论证中有一种被普遍使用的论证类型:类比(analogy)论证。下面是两个类比论证的例子:

一些人认为教师资格测验是不公正的双重测试。"教师已经是大学毕业生,"他们说,"他们为什么还要被测试?"其实这很简单。律师是大学毕业生,而且还是职业学院的毕业生,但他们不得不参加律师资格考试。还有其他大量的行业,如会计、精算师、医生、建筑师等,这些行业对想成为其成员的人都要求参加并通过资格考试,以证明他们的专业素质。没有理由说明教师不应当被要求做同样的事情。 ${ }^{[2]}$

在我们居住的地球和其他行星(土星、木星、火星、金星和水星)之间,我们可以观察到许多类似之处。它们均如地球一样

围绕太阳运行,尽管它们绕太阳的半径不同、周期也不同。它们均从太阳那里获得光,地球也是如此。我们已经知道,其中一些行星,如地球一样,围绕它们的辑自转,因而它们必定有类似白天和黑夜的更替。一些行星有卫星,当太阳不再照射时,这些卫星给行星以光亮,就如我们的月亮给我们以光一样。这些行星的运动均与地球一样受制于万有引力定律。根据所有这些类似,认为这些行星可能与我们地球一样,有不同等级的生命存在,这不是不合理的。通过类比得到的这个结论具有一定程度的可能性。 ${ }^{[3]}$

我们的许多日常推论是通过类比进行的。我推论我将从一台新的计算机那里得到好的服务,根据是,我从同样的生产厂家购买的一台计算机曾给了我很好的服务。我看到某个作者的新作,根据我读过该作者的其他著作并且喜欢这些著作而推断,我将喜欢读这本新作。我们过去的经验在未来同样成立的大多数日常推论,其基础就是类比。当然,我们无法给出一个清楚的公式化的论证,我们只能说,曾被烧伤的孩童躲避火的行为即涉及类比推论。

这些论证中没有一个是确定的或者说是证明性地有效的。这些论证中的结论,没有一个能够从前提中获得逻辑必然性。这是逻辑可能的:用来判断律师和医生资格的方法,并不适合于判断教师的资格;这也是逻辑可能的:地球可能是唯一可以居住的行星,新的计算机可能运转不灵,我喜欢的作者的新书可能无趣而难以卒读;甚至这也是逻辑可能的:一团火能够烧伤人,另外一团火则不会。没有一个类比论证可以指望具有数学的那种确定性。类比论证不是按有效和无效来区分的,我们只能用概率来刻画它们。

除了在论证中频繁使用类比外,人们为了描述生动,经常将类比用于非论证的活动中。明喻和暗喻为类比在文学中的用法,它们为作家给读者的心中创造鲜活的画面提供了莫大的帮助。例如:砧骨上做马蹄铁而产生的副产品火花一样。火花比马蹄铁更为灿

烂,但它们在本质上是无意义的。 ${ }^{[4]}$

类比也用于说明,将读者不熟悉的某种东西,与读者比较熟悉的另一种东西进行对照,比较它们的类似之处,而使读者得以理解。麻省理工学院基因组研究中心主任埃瑞克-兰德试图说明人类基因组计划的巨大影晌。为了加强那些对基因研究不熟悉的人的理解,类比是他所用的一个工具:

基因组计划完全类似于化学中创立周期表。正如门捷列夫在周期表中安排化学元素,使得以前不相关的大量数据变得连贯,同样,当前有机体中上万的基因,将能够从较少数量的简单基因模块或单元即所谓原始基因的组合中得到。 ${ }^{[5]}$

类比在描述和说明中的使用不同于在论证中的使用,尽管在某些实例下,不容易区分是哪一种用法。但是,无论是论证地使用类比还是其他的使用法,类比都是不难定义的。在两个或更多的实体之间进行一个类比,就是表明它们在一个或多个方面(respect)是类似的(similar)。

这说明了什么是类比,但是仍然没有刻画什么是类比论证。让我们考察一个类比论证事例并分析它的结构。我们选择上面引用的例子中最简单的例子:我新买的计算机将给我好的服务,因为我的一台旧计算机是从同样厂家购买的,它给了我好的服务。具有类似方面的两个事物是两台计算机。这里存在三点类比,两个事物被认为在三个方面相似:第一,均为计算机;第二,均在同样厂家购买;第三,给我好的服务。

然而,类比的这三点在论证中并不起相同的作用。前两点出现在前提中,而第三点既出现在前提中又出现在结论中。容易见得,该论证具有这样的前提:首先断定两个事物在两点类似,其次断定了其中的一个事物还具有另外一个特点,从而推论得出另一个事物也具有这个特点的结论。

类比论证是法庭最基本的工具之一。法官不是事先摆出严格的法规或原理,他们往往这样推理,因为两个案件——早先的已经被判决的案件和手头上待判决的案件——有相同的特点,它们应当具有相同的判决结果。例如,一旦做出了不能禁止 3 K 党发表言论的判决,那么法庭可能通过类比论证而得出不能禁止纳粹党游行的结论。 ${ }^{[6]}$ 通过判例的论证一旦做出,

人们将确定和强调以前的案子和手头案子之间类似的那些特点。\\
当然,不是每个类比论证都必须精确地涉及两个事物或者精确涉及三个不同的特点。托马斯-雷德(在上面我们已经提到)认为其他行星可能有人居住,他的论证是对六个事物(当时知道的行星)的八个方面进行类比的。然而,除了这些数量存在差别外,所有的类比论证均具有相同的一般结构或模式。每个类比推理都是这样进行的:从在一个或多个方面两个或更多的事物之间的类似性,到这些事物在某个其他方面具有类似性。我们可以将之公式化:$a 、 b 、 c 、 d$ 是实体,$P 、 Q 、 R$ 是属性或"相似方面",一个类比论证可以表示成下列形式:

$$
\begin{aligned}
& a 、 b 、 c 、 d \text { 均具有属性 } P \text { 和 } Q, \\
& a 、 b 、 c \text { 均具有属性 } R,
\end{aligned}
$$

$$
\text { 因而 } d \text { 可能具有属性 } R \text { 。 }
$$

在识别并且特别是评价类比论证时,将之表示成这种形式是有帮助的。
\section*{11.2 类比论证的评价}
没有一个类比论证是演绎有效的。这里演绎有效的含义是指,从前提得出结论具有逻辑必然性。但是,有些类比论证比其他类比论证的论证力要强。类比论证被评价为比较好的还是比较差的,依赖于结论得以断定的或然性程度。

两个日常例子可以帮助我们弄清楚,什么因素影响类比论证的有力性。你决定去购买特定的一双鞋,因为以前与之类似的其他鞋子使你感觉很舒服;你挑选了一只某一品种的狗,因为该品种的其他狗所表现出来的特征是你所喜欢的。在这两个例子中使用了类比推理。为了评价这两个例子的论证强度——"实际上也就是所有类比论证的强度,我们可以确定 6 个显著的标准:

1.实体数量。如果我过去对特定种类的鞋子的经历仅限于我穿过的并喜欢的一双,对一双明显类似的鞋,我穿后发现具有意想不到的缺陷,这将使我很失望。但是如果我多次购买了那类鞋子,我可以有理由地认为,下一次购买的鞋子会与我以前穿的一样好。在同样对象上的多次的同种经验将支撑结论(购买的鞋子将是合脚的),比单个经验支撑结论有力得多。每一个经验可看成是一个附加实体,在评价类比论证中实体数量是第一个标准。

一般地讲,实体数量——过去所经历的场合数量——越大,论证越强。但是实体数量和结论成真的概率之间没有简单的比例关系。与机敏、温顺的金色猎犬愉快相处的 6 次经历,使人们相信下一个金色猎犬同样是机敏和温顺的。但是,前提中具有 6 个经历的类比论证其结论在可靠性上并不是前提中有 2 个经历的一个类似论证的 3 倍。增加实体数量是重要的,但其他因素也要增加。

2.前提中实例的多样性。如果我先前购买的那些合脚的鞋子,既有购买于大商店的,又有购买于专卖店的,既有在纽约制造的又有在加利福尼亚制造的,既有通过邮寄销售的,又有通过商店直接销售的,那么,我可以有信心地认为,鞋子合脚的原因在于鞋子本身,而不是售货员的服务。如果我先前的金色猎犬,既有公的也有母的,既有从小就领养的小犬,也有从仁慈的社会中得来的成年犬,我可以相信,正是犬的品种,而不是它们的性别、年龄或其来源,是它们先前与我愉快相处的原因。

我们可直觉地理解这个标准:类比论证的前提中所涉及的实例越不相似,论证越强。

3.相似方面的数量。在作为前提的实例中可能出现了大量的相似性:也许鞋子属于同一类型,具有同样的价格,由同样种类的皮革制成;也许猎犬是同样品种,在同样的年龄由同一个饲养人饲养,等等。前提中的这些实例在所有这些方面类似,并与结论中的实例在这些方面类似,增强了结论中的实例具有的新属性的可能性——新鞋子将合脚,一只新的狗会具有温顺的品行。这是该论证所要达到的目的。

我们同样可以直观地理解这个标准:结论中的实体与前提中的实体之间类似的方面越多,结论越可靠。但是,在结论和所识别出的类似方面的数量之间,也不存在简单的数字比例关系。

4.相关性。与共有的相似方面的数量同样重要的是,前提中的实例与结论中的实例在共有的相似方面的种类的类似。如果新鞋子与以前的鞋子一样,是在星期二购买的,这是一个与合脚没有关系的类似;但是,如果新的鞋子与先前购买的鞋子一样,由同样的厂商生产,这自然相当重要。当相似方面是相关的时候(如鞋子的样式、价格以及材料),相似方面便增加论证的力度,并且,单个的具有高相关因素比一批不相关的类似对论证的贡献更大。

至于哪些属性确实与论证结论的可靠性相关,人们有时意见不一致。但相关性本身的意义则不存在争论。当一个属性连接另外一个的时候,即当它们之间存在某种因果联系的时候,它们之间存在相关,那就是为什么确定因果联系在类比论证中是关键的原因,以及为什么在法庭上在确定证据是否有效(即相关还是不相关)的过程中,建立这样的连接往往是至关重要的原因所在。

类比论证之进行,无论是从原因到结果,还是从结果到原因,都是可能的。甚至,前提中的属性既不是结论巾的属性的原因,也不是其结果,而两者是同一原因的结果的话,此时,类比论证也是可能的。医生注意到她的病人出现了某种症状,她能够精确地预测另外的症状。这不是因为其中一个症状是另外一个的原因,而是因为身体的某种紊乱造成了它们的相继出现。一个产品的颜色往往与功能无关。但是,当那种颜色与众不同并且共同出现在前提和结论之中的时候,它可以作为论证的相关相似方面来使用。颜色本身可能与产品的功能无关,但是,如果我们知道该颜色是制

造商生产过程的一个相似方面,它可以用来进行一个论证。\\
因果连接是评价类比论证的关键,我们只能够通过观察和实验经验地发现它们。关于经验研究的一般理论是归纳逻辑关心的中心问题,在下面的几章中我们将以一定的篇幅来对之进行讨论。

5.差异性。一个差异(disanalogy)就是一个不同点,它是这样一个方面,在这个方面我们进行推理得出结论的实例有别于论证所基于的实例。回到鞋子的例子上来,如果我们想购买的这双鞋子看上去好像我们以前所穿的鞋子,但事实上这双鞋子更便宜,并且由不同的厂家生产,那么,这些差异使我们有理由对它能否使我们穿起来舒服产生怀疑。

上面论述的相关性在这里同样是重要的。当被确定了的差别具有相关性,与我们正在寻找的东西有因果连接的时候,差异使类比论证失效。投资者购买共有基金的股票,他们往往根据股票成功的"走势记录"。他们这样推理:先前的购买使资本得益,下一回的购买将同样使资本增益。当我们获悉在基金赢利期间操纵该基金的人刚刚被替换,我们面临着一个实质上的差异,它降低了类比论证的强度。

差异使类比论证减弱。因而,它们往往被用来攻击一个类比论证。正如批评者所认为的,结论中的情形在关键方面不同于早先发生的情形,因而在先前情形中正确的东西不可能在后面的情形中也正确。在司法中这是普遍使用类比的地方,某个(或某些)早先的案子通常作为对手头案件的判例被提供给法庭。论证是类比的。对方辩护律师努力将本案与以前的案子区别开来;即辩护律师努力表明,由于在本案中的事实与以前案子中的事实之间存在某个关键差别,以前的案件不是本案的恰当判例。如果差异较大,并且差异的确是关键性的,它能够成功地推翻所提出的类比论证。

因为差异性是反对类比论证的主要武器,因而,能够使潜在的差异得以消解的做法将加强该论证。这说明为什么在前提中实例的多样性会增强一个论证的力度,正如我们已经在前面第二个标准中所述。前提中的实例之间变化越大,批评者越不可能在前提中的实例与结论之间找到使论证减弱的差异。举例来说:吉姆-库玛尔进入一所大学,成为大一学生;来自于吉姆所在高中的另外十个学生已经在该大学里成功完成了学业。我们可以类比地说,她成功地完成学业是很可能的。在与大学的学习有关的某个方面,如果所有这些学生之间都类似,但他们在该方面与吉姆不同,该差异削弱吉姆将成功的论证。但是,如果我们了解到十个成功的师兄师姐在

许多方面——"在经济背景、家庭关系、宗教背景等方面——相互不同,他们间的这些不同使潜在的差异得以消解。如果以来自于同一高中的其他学生作为论证的前提,这些学生并不紧密地相似,而呈现出变化,那么,正如我们前面看到的,吉姆将成功的论证得到加强。

必须避免的一个混淆是:差异是类比论证得以弱化的原理,与前提中的差别(dissimilarity)使同样的论证加强的原理不同。对于前者,差异发生在前提中的实例与结论中的实例之间;对于后者,差别仅仅发生在前提的实例之间。一个差异(不相似)指的是,我们已经经历的实例和要得出的结论的实例之间的区别。由于结论中的实例与早先的实例不一样,结论得不到保证。(我们可以通过提出差异来进行反驳。)该类比被认为是 "牵强的"(strained)或者"行不通的"。但是当我们指出前提间的差别的时候,我们则强化了论证:我们说,该类比有广泛的效力,它在这些实例和那些实例中都行得通,因而前提中实例多种多样的相似方面与结论涉及的相似方面不相关。

总之,差异削弱一个类比论证;而前提中的差别使类比论证加强。这两方面都与恰当性问题相关联:差异表明了前提中的实例和结论中的实例在某些相关方面存在不同;而前提中的差别所表明的是,我们原以为与某个属性存在因果相关的事情事实上毫无关联。

需要注意的是,被认为是第一重要的标准,即被认为具有相似性的实体的数量,也与相关性有关。实例数越多,它们之间的差别也就可能越多。因而增加实体数量是人们所希望的——但是随着实体数量的增加,增加的实例其作用在降低。因为它所提供的差别更可能由先前的实例所提供———这样的话,增加的实例对于保护结论免遭差异性产生的破坏,起不到或几乎起不到作用。

6.结论所做的断言。每个论证均断言其前提给出了接受结论的理由。容易看到,论证断言得越多,保持该断言的负担也就越大。这对每个类比论证均是正确的。结论相对于前提而言是否适度在推理的评价中起关键作用。

如果我的朋友的新车每加仑汽油能行驶 30 英里,我会得出如果我购买同样厂家和同样型号的车,我至少能够使该车每加仑汽油行驶 20 英里。该结论是适度的,因而可靠性十分大。如果我的结论十分大胆,如,我将至少使每加仑汽油行驶 29 英里,该结论受我拥有的证据的支持程度就很

低。一般而言,断言越适度,加于前提的负担越轻,论证越强;断言越大胆,前提的负担越大,论证也就越弱。

通过减少在确定的前提下的断言的内容,或者使断言维持不变但用附加的或更强大的前提给予它以支持,一个类比论证就得以加强。类似的,如果一个类比论证的结论变得更大胆,而前提保持不变,或者断言维持不变,但我们发现支持它的证据存在较大的缺陷,一个类比论证就被减弱。 
\input{chapter11/section11-3.tex}

% 第十二章
\chapter{逻辑与数学}
\input{chapter12/section12-1.tex}
\input{chapter12/section12-2.tex}
\input{chapter12/section12-3.tex}

% 第十三章
\chapter{逻辑与计算机科学}
\section*{篘 13 第}
\section*{科学和假说}

\section*{13.1 科学的价值}
现代科学几乎改变了我们生活的每个方面。它的实践价值在于,它使得更方便、更健康和更丰富的生活成为可能。尽管人们对于它的一些成果颇感忧虑,然而,大多数人同意,科学进步及其在通信、运输、制造、种植、娱乐和公共卫生等等方面的技术应用,总的来说大大地有益于人类。

科学在实现认识愿望方面也实现了内部价值。很久以前,亚里士多德写道:"认识某个事情(不仅哲学方面的,而且人类的其余方面的)是最大的乐趣。"${ }^{[1]}$ 爱因斯坦则代表所有时代的科学家写道:

\begin{displayquote}
什么因素推动我们发明一个又一个理论?我们为什么要发明理论?答案很简单:因为我们乐于"理解"(comprehending),即,通过逻辑过程将现象归约为已经知道或有(明显)证据的事物。 ${ }^{[2]}$
\end{displayquote}

科学的目标就是发现普遍真理。当然单个事实是关键的;用事实建造科学大廈,如同用石头建造房屋。采集了石头不等于建成房屋,仅仅事实的收集更不能成为科学。科学家寻求理解现象,为此,他们努力揭示现象发生的方式以及它们之间的系统的关系。

仅仅知道事实是不够的,对它们进行说明是科学的任务。因而,这需要理论(如爱因斯坦所说),以及与理论相连的支配事实的自然定律和基础性原理。 
\section{说明:科学的和非科学的}

\begin{quotation}
本节探讨科学说明与非科学说明的本质区别。我们将分析什么构成一个有效的说明,辨别科学态度与非科学态度之间的关键差异,并理解科学说明的可检验性特征。通过理解科学说明必须是可证实的、探索性的,而非教条主义的,我们将能够区分真正的科学解释与仅凭权威或习俗的非科学说法。
\end{quotation}

当要对某个事情进行说明(explanation)时,我们需要什么?一个被寻求的解释(account)就是对世界的某个陈述集合,或某个叙说(sto- ry),从该解释中能够選辑地推导出需要解释的事情。该解释能够对需要解释的有疑问的问题进行消解或者简约。说明和推论可以被看成是同一个过程,只是方向相反。一个逻辑推论之进行是从前提到结论,而对任何给定事实的说明是确定从中事实能够被逻辑地推论出来的前提。在本书第1

章(1.6 节)中我们阐述了,当我们要推得 $Q$ 时,"由 $P$ 得 $Q$"如何表达一个论证;如果我们所进行的是从一个已经建立的 $Q$ 到能够对之说明的前提的推理,它也可以表达一个说明。

自然,每一个好的说明必须是相关的(relevant)。如果我解释说,我上班迟到是因为在中部非洲发生持续的政治混乱,那么这会被认为什么都没有说明;它是不相关的一一因为需要说明的我迟到的事实,不能从中被推论出来。当然每个真正的说明不仅是相关的而且是真实的。

无论我迟到的正确的说明是什么,之所以需要这个说明,是因为在我迟到的这个事件上存在疑问。然而,科学的说明除了相关和真实外,必须超越特定事件,而能够对给定种类的所有事件提供解释。牛顿力学的伟大在于万有引力定律。牛顿写道:

\begin{displayquote}
宇宙中每个质点以一个力吸引另外一个质点。该力正比于质\\
点质量的乘积,反比于它们间距离的平方。
\end{displayquote}

非科学的说明也可以是相关的和普遍的。可以用神秘的小鬼动了手脚,来解释引擎不能启动,这是非科学的;疾病可以解释成邪恶的精灵侵人人体所引起的。在长达数个世纪的时间里,人们一直用在行星上生活并控制它们运动的"智慧生物"来解释行星的规则运动。

但是我们对真正科学的说明感兴趣。科学的说明与非科学的说明在两个相互关联的方面相区别:

第一个区别是态度上的区别。接受非科学说明的人是教条的,解释被认为是绝对真的,是不能改进的。亚里士多德的观点在几个世纪里被非科学地接受成对事实的最终权威。尽管亚里士多德本人是谦虚的,但是一些中世纪的学者却以僵化的、非科学的态度对待他的观点。 ${ }^{13}$ 相反.真正科学的态度则与之十分不同。每个提出的说明都是探索性的或暂时的。科学说明被认为是假说一一它们在现有证据下具有不同的可靠程度。

科学说明和非科学说明之间的第二个同时也是最根本的区别是,接受或拒绝某个观点所基于的基础。一个非科学的说明被简单地认为是真的,因为"每个人知道"它如此。一个非科学信念之被坚持,不依赖于有利于它的证据之上。但是在科学中,一个假说仅仅在存在支持它的证据的条件下才值得接受,人们总是对它的真或假保持怀疑,寻找证据的过程永不停

止。科学是经验的一一真理的检验在于经验之中,因而科学说明的本质是,它是可检验的。

真理的检验可以是直接的也可以是间接的。为了弄清外面是否下雨,我只要看一下外面。但是用做说明的假说是普遍性命题,它们不能是直接可检验的。如果我对我上班迟到的解释是交通事故,我的老板如果对之怀疑,他能够借助于警察的事故报告而间接地检验我的解释。一个间接的检验从待检验的命题(如我遭遇到一次交通事故)演绎出其他某个能够被直接检验的命题(如我提交了一个事故报告)。如果那个演绎出来的命题是错的,包含这个命题的说明必定是错的。如果演绎出来的命题是真的,它提供了某个证据证明这个说明是真的、已经被间接证实。但是这个证据不是结论性的。

间接检验永远不会是确定的。它总是依赖于某些辅助的前提,比如这样的前提:我对我的老板描述的该起事故与警察记载的一样。但是警察部门应当对我所涉案的事故的记录进行备案,但可能还没有备案;缺乏该记录并不证明我的说明是假的。并且,某个附加前提即使是真的,它并不给说明赋予确定性——尽管演绎出的结论(本例中事故报告的真实性)得到成功检验确实加固了它的前提。

即使一个非科学的说明也有某个它喜爱的证据,即用它来解释的那个事实。行星上居住着"智慧生物",他们使行星沿着我们观察到的轨道运动,这个非科学理论能够称这个事实一一行星确实在它们的轨道上运动——为证据。但是,在该假说和关于行星运动的可靠的天文学说明之间存在巨大的差别:对于非科学假说,不能够从中演绎出其他的可直接检验的命题。另外一方面,一个给定现象的任何一个科学说明能够演绎出可直接检验的命题,而不是陈述待解释事实的命题。这就是当我们说一个说明是经验可证实的时所要表达的意思。这样的可证实性是科学说明最本质的特征。 ${ }^{[5]}$ 

\begin{center}
\fbox{\parbox{0.95\textwidth}{
\textbf{本节要点}
\begin{itemize}
\item \textbf{说明的基本性质}:
  \begin{itemize}
  \item 说明是从中能推导出待解释现象的陈述集合
  \item 说明与推论是相反方向的逻辑过程
  \item 有效的说明必须是相关的和真实的
  \end{itemize}
\item \textbf{科学说明的特点}:
  \begin{itemize}
  \item 超越特定事件,能对同类事件提供普遍解释
  \item 具有可检验性,能从中演绎出可直接验证的命题
  \item 是探索性的、暂时的,而非绝对或最终的
  \end{itemize}
\item \textbf{科学与非科学说明的态度区别}:
  \begin{itemize}
  \item 科学态度:假说被视为暂时的,总是存在修正可能
  \item 非科学态度:教条式接受,解释被视为绝对真理
  \item 科学家对所有理论保持适当怀疑,不断寻求新证据
  \end{itemize}
\item \textbf{科学与非科学说明的本质区别}:
  \begin{itemize}
  \item 科学说明基于证据,依赖经验检验
  \item 非科学说明基于权威、习俗或"常识"
  \item 科学说明能产生可检验的预测,而非仅解释已知事实
  \item 经验可证实性是科学说明最关键的特征
  \end{itemize}
\end{itemize}
}}
\end{center} 
\section{对科学说明的评价}

\begin{quotation}
本节探讨如何评价相互竞争的科学说明。我们将分析科学家用于判断假说优劣的三个关键标准:与已有理论的协调性、预测力或说明力,以及简单性。通过理解这些评价标准,我们将能够认识科学知识是如何在理论竞争中发展的,以及如何在相互矛盾的科学解释之间做出合理选择。
\end{quotation}

对同样的现象,人们往往会提出不同的、相互不协调的科学说明以对之进行解释。我的同事动作生硬,可以解释成她生气了,也可以解释成她害羞。在刑事调查中,对犯罪的认定有两个相互不协调的假说,都对犯罪事实有很好的解释。但是当两个假说不能都真的时候,我们将如何在其中

做出选择?\\
这里,我们在做的是评价相互竞争的科学说明。我们假定两个(或所有的)假说都是相关的并且是可检验的。我们应当采用什么标准以便从手边的理论中选择出最好的理论?我们不能指望存在这样的规则,它们引导我们发现假说;发现假说是科学事业的创造方面,它体现了天才和想象力,在某些方面类似于艺术工作。尽管不存在发现新假说的公式,但存在比相关性和可检验性更进一步的标准,我们可以用这些标准对可接受的假说进行确证 (conform)。

人们在评判竞争的科学假说的优缺点时普遍地使用三个标准:\\
1.与原有已确立假说的协调性\\
科学的目标是获得一个说明性的假说系统。当然,这样的系统必须是自我相容的,因为没有一个自相矛盾的命题集合能够是真的。进步之得出是通过渐渐发展假说以理解越来越多的事实,这样的进步要求每个新假说应与已经得到证实的那些假说相一致。例如,在天王星轨道外面存在另外一个未知的行星的假说,与天文学理论的主要部分吻合完美,它导致海王星的发现(1846 年)。 ${ }^{[6]}$ 科学中的进步是有序的,这要求任何新的理论与以前的理论相一致。

科学的理想是通过一个又一个新理论的增加而使知识发生渐渐地增长,但是科学进步的实际历史不总是遵循这种有序的方式。有时重要的新假说与已有理论不相容,它直接替代了已有理论,而不是努力与旧理论相一致。爱因斯坦的相对论就是这种假说,它突破了旧的牛顿理论中的许多原有概念。19世纪后期放射性物质的发现推翻了物质守恒原则。物质守恒原则断定,物质既不能被创造也不能被消灭。镭原子发生自发衰变的假说直接与这旧的、已被接受的原则不相容,最终这个旧的原则不得不被抛弃。

科学中旧理论被抛弃,较新的和较好的理论被接受,这个过程不是很快或者无抵抗的。事实上,旧理论不是被认为一无是处地被抛弃。爱因斯坦自己总是坚持,他自己的工作是对牛顿工作的一个修正,而非抛弃。物质守恒原则被修正成更为广泛的质能守恒原则。一个理论之被建立,因它显示出能够解释大量的数据或已知事实的能力。它不能被某些新假说所废弃,除非新假说对同样的事实能够进行解释甚至更好。

因此,科学通过采用更为广泛因而更为相关的说明而得以进步。通过

说明,世界将它展现在我们的经验面前。这种进步不会是反复无常的。当不相容产生的时候,一个假说的年岁较长不能自动证明它是正确的,但是这个假定(年岁较长)有利于旧假说——如果旧假说已经得到广泛的确证。如果与它发生冲突的新假说同样获得广泛的确证,考虑假说的年岁或提出的先后是不合适的。当两个假说发生冲突的时候,为了在它们间做出选择,我们必须求助于可观察的事实。上诉的最终法庭是经验。

这个标准一一与先前良好建立的假说相协调——所要达到的最终结果是,任何时候所接受的假说全体必须是相互融贯的。 ${ }^{[7]}$ 在其他情况一样的条件下,与已接受的科学理论吻合得较好的假说应当被偏爱。与"其他情况一样"有关联的问题将我们带到第二个标准。

\subsection{预测力或说明力}
正如我们已经看到的,每个科学假说必须是可检验的;如果某个可观察的事实能够从中演绎出来,它就是可检验的。当我面临两个可检验的假说,其中一个比另外一个演绎出更大范围的事实,我们说该假说具有较大的预测力或解释力。

举例来说明。伽利略(1564-1642)建立了落体定律公式,该定律对靠近地球表面的物体的行为给出了一个十分普遍的解释。差不多同时,德国天文学家乔哈恩斯•开普勒,用丹麦的第谷•布拉赫收集的天文数据建立了行星运动定律,该定律描述了行星绕日运行的椭圆轨道。这些科学家将各自研究领域里(伽利略——陆地上的力学,开普勒——天体力学)的不同现象统一起来。这些发现自然是辉煌的成就,但是它们是各自分离的。艾萨克-牛顿提出了三大运动定律和万有引力理论,将这些分离理论统一了起来并给予了解释。牛顿万有引力解释了所有伽利略和开普勒解释的结果,以及除此之外更多的事实。从一个给定假说中演绎出一个可观察的事实,我们说该事实被该假说所说明,并且我们也能够说该事实被该假说所预测。牛顿理论具有巨大的预测力。一个假说预测力越大,它解释得越多,并且它对我们理解它所涉及的现象的贡献越大。 ${ }^{[8]}$

这第二个标准具有否证作用。如果一个假说与某个得到证实的观察不一致,该假说便是错的,必须被拒绝。当两个不同假说都能完全解释某个事实集合,都是可预测的,并且都与已经构建的整个科学理论相协调,此时,在它们之间做出选择是可能的:从它们推出可检验的但相互不协调的命题。为了在冲突的理论中做出选择,可以建立一个判决性实验。根据第

一个假说,在确定条件下一个给定结果将发生;而根据第二个假说,在那些同样的条件下给定结果将不发生,我们可以通过观察该结果发生还是不发生,而在两个假说中做出选择:它的发生否证了第二个假说,它的不发生则否证了第一个假说。

对两种相竞争的假说进行判决的这种判决性实验也许不容易实现。原因在于,制造那些关键性事件是困难的或者不可能的。牛顿理论和爱因斯坦广义相对论之间的决策不得不等到日全食的发生——这是一个明显超出我们自己能够创立的事件。 ${ }^{[9]}$ 在其他情况,判决性实验可能要等到新工具的发明:这些新工具或者是为了创造所需的条件,或者是为了对已经做出预测的现象进行观察或测量。因此,竞争的天文假说的支持者有时必须等待时机,等待建造出新的、威力更强大的望远镜。对于判决性实验,在 13.6 节中我们将进一步讨论。

\subsection{简单性}
两个竞争性假说可能是相关的和可检验的,可能与已有理论吻合得同样好,甚至可能具有大致相当的预测力。在这样的条件下,我们可能支持两个中比较简单的那个。关于天体运动的托勒密理论(地球中心)和哥白尼理论(太阳中心)之间的冲突就是如此。两者都与早先的理论吻合良好,它们都同等程度地预测天体运动。两个假说都依赖于一个笨拙的(自然是错误的)工具——假想的本轮(较小的圆在较大圆上运动),以解释已做出的天文观察。但是哥白尼系统依赖这样的本轮更少,因而它更简单,这个较大的简单性是后来的天文学家接受该理论的主要原因。 ${ }^{[10]}$

简单性似乎是一个可以求助的"自然"标准。在日常生活中我们同样趋向于接受符合所有事实的最简单的理论。在刑事法庭上对一犯罪行为会提出两种观点,最终在该案子上更简单、更自然的观点可能被支持(或应当被支持)。

但是"简单性"是一个不好捉摸的观念;只有在非常少量的情况下,如在托勒密和哥白尼的冲突中,我们根据较少的实体数的要求选择比较简单的理论。在两个竞争的理论中可能是,在不同的方面一个比另外一个简单。一些人可能依赖于比较少的实体数量,而其他人可能基于较简单的数学方程。甚至"自然"(naturalness)可能是欺骗人的。许多人会更"自然"地相信,明显不运动的地球事实上是不动的,而明显运动着的太阳确实环绕我们在运行。简单性是一个重要的标准,有时甚至是决定性的。但

是它是难以公式化,并且不总是易于应用的。 

\begin{center}
\fbox{\parbox{0.95\textwidth}{
\textbf{本节要点}
\begin{itemize}
\item \textbf{科学说明评价的必要性}:
  \begin{itemize}
  \item 同一现象常有多种相互竞争的科学解释
  \item 需要可靠标准来判断哪些假说更优越
  \item 假说的创造依赖创造力和想象力,但评价需要严格标准
  \end{itemize}
\item \textbf{与已有理论的协调性}:
  \begin{itemize}
  \item 科学追求自洽的假说系统,新理论应与已建立理论相容
  \item 科学进步通常是渐进的,新假说要兼容已证实的理论
  \item 重大科学革命可能打破旧理论框架,但必须解释旧理论能解释的现象
  \item 任何时候所接受的假说体系必须是相互融贯的
  \end{itemize}
\item \textbf{预测力或说明力}:
  \begin{itemize}
  \item 能从假说演绎出更广泛事实的理论具有更大预测力
  \item 如牛顿理论统一并解释了伽利略和开普勒的分离理论
  \item 具有否证作用:若理论与观察不符,必须被拒绝
  \item 判决性实验可以在竞争理论间做出选择
  \end{itemize}
\item \textbf{简单性原则}:
  \begin{itemize}
  \item 在其他条件相同时,应选择更简单的理论
  \item 如哥白尼日心说比托勒密地心说更简单
  \item 简单性标准难以精确定义,不同理论可能在不同方面简单
  \item 简单性是重要标准,但难以公式化且应用存在挑战
  \end{itemize}
\end{itemize}
}}
\end{center} 
\section*{13.4 科学研究的七个阶段}
我们现在所要做的是,描述科学研究的一般模式。这个模式可以分解成七个步骤或阶段。在抽象分析中它们易于区别,但在实践中它们绝不总是界限分明的。在许多情况下它们是相互渗透并混合在一起的。我们首先将简单地提出这七个阶段。在我们心里有了这些研究模式后,我们用例子来阐述这七个阶段。

\section*{1.确定问题}
科学研究开始于某个问题。一个问题可以表示成一个或一组没有可接受的说明的事实。例如,侦探面临一个案子,他的问题是如何将之侦破,即确定犯罪人并给予证明。在某些情况下,如在柯南•道尔关于伟大的夏洛克-福尔摩斯的故事中,问题产生于尚未发生犯罪的特定事件或环境之中。科学家的研究可能开始于十分明确的问题;然而,更为普遍的是,他们是渐渐地发现了不相容或奇怪之处,这些不相容或奇怪之处演化成一个特定问题。

如果不存在可思考的事情,甚至夏洛克•福尔摩斯或者爱因斯坦也不能从事深刻的思考。一个天才必须面对一个问题。正如约翰•杜威和许多其他现代哲学家正确认为的那样,反思性的思考——从犯罪侦查到物理学、数学的抽象思考这个范围广泛的活动——是解决问题的(problem- solving)活动。科学家开始工作之前,问题必须被确定,或者至少以模糊的形式被确定。

2.构建初始假说\\
对手边的问题的哪怕最初始的思考都要求某个初步理论。最初的尝试不可能产生最后的答案,但是需要某个理论以便能够知道,需要收集何种类的证据,到哪里寻找它们,以及如何寻找它们。侦探考察犯罪现场,询问嫌疑人,并寻找线索,但是赤裸裸的事实不是线索。只有当线索能够被安排进某个融贯的模式,哪怕是粗糙的和临时的模式之中时,它们才有意义。

科学家与此相同,他们用某个初始假说开始收集证据。这个假说是关于待寻求的说明的本质的。科学家必须依赖于某些以前的知识,科学不会

从绝对无知中开始。事实上,如果被说明的事实出现真正的问题,必定存在某些先验的信念。

对于任何一个严肃的问题,世界上存在太多的相关的事实、太多的数据、以至于人们不能将它们全部收集起来。一些事实将被注意并被观察,另外的事实则没有。最耐心和最全面的研究者必须选择:被发现的事实中哪些要研究,哪些要放弃。这需要某个假说来工作:为了这个假说或者根据这个假说而收集相关的证据。该假说不必是完善的理论一一但是在它那里至少显示出理论的轮廊。否则的话,研究者不能确定从整个事实全体中挑选出何种事实来。一个临时的初始假说不管是如何的不完善,任何严格的探究开始的时候它都是必需的。

\section*{3.收集额外事实}
一般来说,最初令人迷惑不解的事实似乎太多了,以至于不能提出一个对它们非常满意的说明;如果情况不是这样的话,这些事实不可能表现出问题来。但是,特别的,对那些熟悉普遍种类(如天体现象、社会现象或历史现象)的事实或事件的科学家而言,原初的问题会激发出一个初始假说,该假说引导他们寻找额外的相关事实。这个额外证据可以起到引导作用.引导我们得出较完全和较接近的合适答案。收集证据的任务,既艰辛又耗时,经常是失望和沮丧。好的科学意味着艰巨的工作。这个费力的收集过程是许多科学工作的主要内容。

在实际的科学活动中,步骤 2 和步骤 3 自然不是完全分离的。它们紧密相连、相互依赖。在开始收集证据时,某个初始假说是必需的;使用该假说来收集证据的过程,也是调整和精练假说本身的过程,这又引导我们进一步地寻找……也许导致新的发现……它又使我们更加精练假说,等等。

4.形成说明性假说\\
在任何成功的研究中迟早会达到这样一点,研究者(科学家、侦探,甚至普通人)将最终相信,解决原初问题所需要的所有事实都已经获得。一个难题,更可能的是一组难题,摆在他或她的面前,其任务是将它们组装成一个可以理解的整体。这样思考的终结产品——如果成功的话,是这样的假说:它将解释所有数据、产生问题的原有事实集,以及初始假说所涉及的额外事实。

不存在实现某个完善理论的机械方法。对真正说明性的假说的实际发

现或发明是一个创造性的过程,在这个过程中需要想象,也需要知识。某些研究者,如夏洛克•福尔摩斯和阿尔伯特•爱因斯坦,在对存在的现象进行说明的"逆向推理"的过程中展示了其才能。但是每一个成功的科学家必须完成智力整合这个挑战性的任务:对激发研究兴趣的成问题事实进行解释,以构造和形成最终假说。 ${ }^{[11]}$

5.推导出进一步结果\\
我们已经看到预测力是评价说明的标准之一。一个真正富于成果的假说将不仅说明激发假说形成的原初事实,而且解释许多其他的事实。好的假说超越初始的事实,它涉及新的和不同的事实——这些事实较早没有被怀疑。假说所引出的这些事实被证实,使得假说得以确证——当然不能给予确定性的证明。

被称做"大爆炸的"宇宙学理论可以看做对这样的预测进行阐述的例子。这个理论认为,如果目前的宇宙开始于一个大爆炸事件,最初的火球应是平稳的和均匀的,而没有任何结构。与此对照的是,目前的宇宙具有大量的结构,是多块状的;可见物质组成星系、星系群,等等。这样的结构对生命的起源和演化是基本的。但是该结构是何时产生并如何产生的?通过观察膨胀的宇宙中那些遥远的天体,天文学家能够"回顾过去"。通过这样的观察,他们最终必定能够找到目前结构的原初证据。如果如此早的证据不能通过最敏感的仪器来探测到,大爆炸理论将是不可辩护的。如果这样的结果被探测到,大爆炸理论得到确证,尽管不是被证明。

\section*{6.对结果进行检验}
在生物学领域里我们可以提出这个假说:在哺乳动物中蛋白质之产生是为了对抗特定的酶,而这种酶是在一个特定基因引导下产生的。从该假说中我们可以推论出进一步的结论:缺少该基因的地方,该蛋白质将不出现或者蛋白质数量不足。

为了检验该生物学假说是否正确,我们构造某个特定基因的作用能够被测定的实验。经常的做法是,将去除特定基因的老鼠进行繁殖——被称为"基因剔除老鼠"。如果在这样的老鼠中被研究的酶以及与之有关的蛋白质发生缺失,我们的假说将得到证实。 ${ }^{[12]}$ 在医学中许多有价值的信息正是以这种方法获得的。这种实验是广泛的生物学研究中典型的实验。我们设计实验,以弄清我们认为是对的东西,在如此这般的条件得到满足的情况下,是否确实是真的。为此,我们必须构造这样或那样的特定的条件。\\
"一个实验",正如伟大的物理学家马克斯•普朗克说,"是科学给自然提出的一个问题,而测量是对自然回答的记录。"

对某些预测的结果,如同夏洛克-福尔摩斯的许多预测的结果,其检验可能是直接的。银行窃贼将打破拱顶而人?福尔摩斯和华生等待他们,并且他们确实来了。 ${ }^{[13]}$ 医生将避开从假的通风口进来的毒蛇吗?福尔摩斯和华生从躲藏的地方观看,发现医生避开了。 ${ }^{[14]}$ 那些说明性的理论直接地被检验,并牢固地得到了证实。

当然,许多科学理论不能被简单的观察所检验。早期宇宙的结构不可能被直接观察到。但是,如果存在某个早期的结构,如大爆炸理论预测的那样,那么,由于目前的背景辐射根源于早期,在它之中将必定存在不规则、不均匀。在原则上测量背景中的微波辐射是可能的,因而我们能够以这种方式间接地确定在大爆炸之后十分短暂的时刻是否存在这样的不规则性。几年后,一个卫星被设计来探测这些不规则性——如果它们存在的话。该卫星(宇宙背景探测者 COBE)的观察对大爆炸理论是至关重要的。如果最终没有探测到长期寻找的宇宙中早期结构的证据,膨胀宇宙的大爆炸解释将遭受严重的质疑。然而,1992年春天,预测到的不规则性被 COBE 探测到并被测量出来。这些不规则性来自于最遥远的过去,一直存在到宇宙学家回顾它们的今天。这个成功检验尽管不能证明该理论是正确的,但的确给人印象深刻地确证了大爆炸理论。

\section*{7.应用该理论}
通过科学,我们的目的是说明我们观察到的现象,但是我们另一个目的是控制这些现象,为我们所用。牛顿和爱因斯坦的抽象理论在太阳系的现代探索中发挥中心作用。举一个不同种类的例子,假定我们面临的问题是某个疾病,发明的说明性假说是某个特定细菌引起该疾病。假定通过给老鼠或啮齿动物注射该病菌,对该理论进行检验,并假定在进行实验的动物中产生了该种同样的疾病,这些检验给说明性假说以强的支持。我们当然试图在临床医学中使用该理论。做法是,通过消灭患有该病的病人身上的细菌而将病治愈——先在实验人群中进行,然后再按照常规来进行。正是按照这种方法,我们学会了如何与许多可怕的人类疾病进行战斗,在一些情况下甚至完全消灭这些疾病。通过科学,我们试图理解世界;同样通过科学,我们使用一些手段,对世界给予我们的危险进行控制。

\section*{科学研究的七个阶段}
1.确定问题\\
2.选择初始假说\\
3.收集额外事实\\
4.形成精练的说明性假说\\
5.从精练的假说中演绎出结果\\
6.检验演绎出的结果\\
7.应用该理论\\
这七个基本阶段往往发生重叠和相互渗透,但是我们可以通过反思,从实际的科学研究中识别出来。 
\input{chapter13/section13-5.tex}
\section{判决性实验和特设性假说}

\begin{quotation}
本节探讨科学方法中的两个重要概念:判决性实验和特设性假说。我们将分析判决性实验如何帮助科学家在竞争性理论之间做出选择,以及特设性假说如何用于保护理论不被反驳证据所否定。通过理解这两个概念,我们将能够更好地评估科学理论的可靠性和科学知识发展过程中的方法论挑战。
\end{quotation}

\subsection{判决性实验}
不同的理论有时会预测相同现象的相同结果;在这种情况下,为了在竞争的理论之间做出选择时,只能使用评判理论的普遍规则。但是当不同的理论对某个现象预测出不同的结果时,科学家便设计出被称做判决性的实验。两个理论中一个所预测的结果是正确的,那么它就通过,如果另外一个所预测的结果是错误的,那么该理论就被拒绝。

科学中的进步很少是直接的和容易的。认为通过对某个问题简单地使用几步假说一演绎法就能够达到答案,这种看法是愚奪的。答案一一正确的说明性假说一一往往是模糊的,需要非常精心制作的理论武器。建立最后的正确假说会极其困难。这个过程完全不是机械的,除了需要艰辛的观察和实验外,还需要深刻的洞察力和很大的创造性。

新的假说得以形成之后,如果它与某个先前已经接受的理论相矛盾,很难确定哪个正确。在某些场合下,两个竞争性假说用被称为一个"判决性实验"的东西进行检验。判决性实验是指这样一个实验,它被精心构造出来以表明所提出的说明中的一个而非另一个实际上是正确的。这样的判决性实验一旦被建立起来,可能是激动人心的和极其有成果的。

例如,美国物理学家阿尔伯特•迈克尔逊和化学家爱德华•莫雷在 1887年精心构造了一个测量光速的实验。通过这个实验使一个被广泛接受的理论(他们原来相信它是正确的)置于一个判决性实验之中。人们长期相信空间中充满着一个被称做"以太"(ether)的假设物质,(人们假定)该物质使光波运行,如同空气使声波运行一样。或者以太存在,或者它不存在。如果它存在,那么测量出的沿着地球运动方向的光速,应当与地球运动成一个直角方向上的光速不同。该实验产生了一个"否定的"结果。因该实验是一个对当时被广泛接受的理论的判决性检验,它成为物理

学史上最著名的实验之一。没有发现这两个不同方向上运动的光速存在差别。这个结果有力地破除了人们长期相信的以太概念。 ${ }^{[23]}$

但是遗憾的是,威力如此强大的判决实验不总是可行的。不同的可观察结果可能不会从不同假说中推演出来;或者,它们能够被推演出来,但是我们没有能力创造条件,以检验哪一个假说的结果将出现。

物理学在 21 世纪初面临的一个主要问题也正属此类。在两个最强有力的理论之间,存在一个明显的目前不能解决的冲突。广义相对论已经得到很好的证实,其定律(描述引力以及引力如何形成空间和时间)的一个必然推论是:某些塌陷的大质量的恒星将形成"黑洞",从该黑洞中逃脱是不可能的,因为它要求比光要快的速度。量子力学定律同样得到很好证实,但它们明确推论得,信息不能永久消失,即使掉到黑洞里也是如此。要么存在某个目前还不知道的时空性质,它能够用来对该信息的保持进行说明,要么在物理学中存在错误,指出它可以对该信息的永久消失给出解释。最终两个理论中的一个必定得到修正,但我们现在仍然不知道哪个要修正,我们也无法构造所需要的判决性实验。 ${ }^{[24]}$

判决性实验是科学探究一个重要方面,然而与构造判决性实验相关的另外一个困难是,人们提出某个说明性假说,我们希望通过进行某个判决性实验来检验它的推论,但它的推论不可能仅从该假说自身演绎出来。我们是使用该假说与其他理论一起而推论得到要检验的结论的。为此,我们假定那些其他理论完全可靠。它们确实可能是完全可靠的,当然它们也可能不完全可靠。如果它们不可靠,即使判决性实验似乎否证了待考察的假说,也有可能待考察的假说恰恰是正确的。科学中的进展依赖于假说集合,其中的任何一个都可能是有缺陷的。

当涉及相当高抽象程度的假说时,仅仅单个假说不可能直接演绎出可直接检验的预测。用做演绎前提的必定是一个统一的假说群体,如果观察到的事实不是预测的事实,那么我们可以得出结论,该假说群中至少一个是错的。但是,这个结论不能表明哪一个假说是错的。例如,在前面的对 DNA 结构发现的解释中,华生和克瑞克在检验核酸丝的形式是双螺旋、它的基指向内部的假说时,他们发现这样的安排不能与所有已知的事实和已接受的理论相一致。"已知的事实和已接受的理论"——水含量、双螺旋斜度、基(腺嘌呤、鸟嘌呤、氧氨嘧啶和胸腺嘧啶)的连接方式——在假说的检验中被假定是正确的。如果所有这些假定的确正确,长丝的结构

不可能是双螺旋。然而,在实际中,华生和克瑞克对他们的假说有足够的自信,他们开始怀疑描述基(A、G、C 和 T )相互结合的理论不完全正确。该理论被他们放弃,他们提出一个不同的理论,即假定结合物是氢的理论,此时,双螺旋的新假说(以及与之相连的理论)得到证实。

因此,在揭示一群假说有缺陷的过程中,一个实验能够是"判决的"。这样一群假说通常包含许多独立的假说。其中任何一个假说,无论实验结果对它多么不利,我们可以拒绝该群体中的某个其他假说,而坚持它的真理性。这就使得某些人得出结论说,从来不会有单个假说遭受判决性的实验。

\subsection{特设性假说}
针对上面的批评,有人认为,一个实验在否证单个新假说中的确能够是判决性的,因为通过拒绝假说群体中某个其他假说(上面已经表明这是可能的)而"拯救"该假说的努力是完全特设的(Ad Hoc)。Ad Hoc 为拉丁术语,字面意义是"为此[特定目的]"。Ad Hoc 包含这样一个意思,所有的假说都是特设的。因为,一个假说之发明如果不是为了解释某个先前得到确立的事实或者其他事实,那是没有意义的。但是当我们以这样的意义滥用它的时候,特设性意味着,对假说集合进行调整仅仅是为了拯救被检验的假说这样一个目的,它没有其他的说明力或者可检验的结果。

没有科学假说是这第二个意义上特设的。如果"鬼是机器故障的原因"被用来解释一个复杂的机器发生故障,那么它明显不是科学的解释;我们嘲笑这样的假说,它是否定意义上特设性的。但是,在任何实际科学研究中,当一个新的假说之提出以调整一个旧的理论的时候,该调整是否是该否定意义上特设性的,这需要进一步确定。

科学史中的另外一个例子可以帮助我们弄清这个问题。在19世纪,天体力学理论被人们很好地理解。对于天文学家来说,天王星和水星这两个行星的轨道与当时所接受的理论对它们所预测的轨道不一致。行星运动的理论在当时应当被修改,但是事实上它被保留了下来。为了解决该理论的协调问题,有人提出,存在某个未发现的行星,其引力造成观察到的反常现象。引起天王星轨道偏差的新行星的轨道,由勒维烈在 1845 年预测出来,预测结果不久被海王星的发现而证实,其位置精确地解释了那些偏 513 差。 ${ }^{[25]}$ 这个假说——存在这样一颗行星——当然不是否定意义上特设性的假说。原因是,从该假说中能够演绎出许多结论,该假说是独立可检

验的。\\
但是在水星的案例中,存在另外一颗行星[该行星过早地被命名为 "火神星"(Vulcan)]干扰水星轨道的假说,不能得到证实。如果一个理论假想有"水星力",用它们来解释水星轨道的异常,而这些力不能解释其他任何事情,并且绝不能被找到,那么这样的一个理论发明自然是特设性的。实际情况是,该疑难长期得不到解决;直到1915年广义相对论提出后,观察到的水星轨道的不规则,才完全与不同的但完美的天文学理论相吻合。水星轨道异常,能够使用广义相对论来预测,这个事实构成该理论最引人注目的证实之一。爱因斯坦称它为"我生命中最辉煌的工作"${ }^{[26]}$ 。只有在那个时候,我们才给出了关于该现象的合适的(即真正理论化的)说明。

天文学史中的这个疑难给出了人们在使用特设性这个术语的第三个意义,它也是否定性的:表示一个单纯的描述性概括。一个描述,它是第三个意义上特设性的,它仅断定一个特定种类的所有事实只在某些特定种类的条件下发生;但是该假说和前面的那些特设假说一样没有任何解释力或解释范围。这样的假说的一个古典例子是,"菲兹吉拉德收缩效应"被提出来对迈克尔逊-莫雷在光速实验结果的解释。菲兹吉拉德断定,物体以极其高的速度运动会发生收缩,他确实对给定现象给出解释,并且他的假说能够为重复进行的同样实验所检验。但是他的"收缩效应"不能解释其他的任何东西。在当时它被普遍认为是特设性的而不是说明性的。(正如与水星行为中出现明显差别的情况一样)直到相对论的提出(爱因斯坦的狭义相对论),人们才得到迈克尔逊一莫雷实验结果的一个合适的理论说明。

我们可这样总结,实验对单个假说绝不能是判决性的,这不仅因为假说经常是在否定意义上特设性的;进一步地说,在本节前面已经表明,因为假说只是在群体中才是可检验的,实验绝不能成为判决单个假说的东西。 ${ }^{[27]}$ 这个限度阐明了科学的系统化特点。科学进步就是建立永远更加恰当(ever-more-adequate)的理论,以便解释不断增加的观察结果和实验事实。某些分离事实能够具有较大的价值,因为科学的最终基础是事实。但是科学结构主要不是通过点滴累积而得以发展的,其发展是在一个已得到普遍认同的理论体系的框架内整体地进行的。认为科学假说或者定律是分散的和独立的观点是朴素的,也是过时的。

在这样一个理论框架中工作,此时我们不对该理论框架进行质疑;进行一个"判决性实验"以证实或否证某个假说的观点仍然能够有意义。如果得到一个否定性结果,即,根据某个有疑问的假说与已经接受的科学理论一道进行预测的某个现象,它没有发生,那么该实验是判决性的,这个有疑问的假说可以被拒绝。但是正如我们已经看到的,在这个过程中不存在任何绝对的事情,因为,那些即使被人们普遍接纳的科学理论面临新的和矛盾的证据时,也要发生调整。科学无论在实践中还是在目标上,都不是一成不变的。

从前面的讨论中得到的启示是,将"隐藏的假设"揭示出来,以便能够对那些默认的假定进行重新审视,这在科学进步中是重要的。当一个关键假设是潜藏的时候,没有明显的必要,因而没有好的机会对之进行考察并确定它到底是真还是假;通过将以前潜藏的假设揭示出来,对之进行分析并(也许)否定它,科学往往获得进步。

例如,在日常生活中谈论两个事件"在同一个时刻"发生,这似乎完全没有问题。我们普遍假定事件常常同时发生。但是科学中一个重要的和巨大的进步开始于爱肉斯坦将这个假定揭露出来。他问,一个观察者如何能够确定两个距离遥远的事件是否真的在同一时刻发生。最终他得出这个结论:两个事件对于某些观察者来说能够是同时的,但对其他的观察者来说则不是;这依赖于观察者相对于待研究事件的相对位置和速度。正是对同时性假定的拒绝,使爱因斯坦发明了狭义相对论,从而为解释迈克尔逊——莫雷实验所揭示的现象跨出了重大的一步。当然,一个假定在它被挑战之前必须被人们所认识,因而,在科学中具有重大意义的是,将理论中起作用的所有有关的假定揭示出来,而不让任何一个隐藏起来。

通过描述和讨论科学史中最辉煌的篇章之一一伽利略对太阳系的哥白尼理论进行的观察证实,是对科学方法的进行总结并阐释科学整体的进步的意义的极好的途径。 

\begin{center}
\fbox{\parbox{0.95\textwidth}{
\textbf{本节要点}
\begin{itemize}
\item \textbf{判决性实验的作用}:
  \begin{itemize}
  \item 在竞争性理论预测相同结果时,依靠普遍评判规则
  \item 当理论预测不同结果时,判决性实验可帮助确定哪个理论更可靠
  \item 真正的判决性实验能够排除某些假说,而保留其他假说
  \end{itemize}
\item \textbf{判决性实验的局限}:
  \begin{itemize}
  \item 科学进步过程复杂,很少有简单直接的决定性实验
  \item 判决性实验不仅要考虑主要假说,还要考虑相关的辅助假说
  \item 批评者认为没有真正单一判决性实验,因为假说总是集群出现
  \end{itemize}
\item \textbf{特设性假说的特点}:
  \begin{itemize}
  \item 特设性假说是为了拯救理论而临时构建的解释
  \item 区别于常规假说的是缺乏独立可检验性和其他说明力
  \item 不能预测新现象,仅用于解释已知的反例
  \end{itemize}
\item \textbf{对特设性假说的评价}:
  \begin{itemize}
  \item 使用特设性假说保护理论违背科学精神
  \item 良好的科学理论应预测尚未观察到的现象
  \item 过度依赖特设性假说的理论往往不是真正科学的
  \end{itemize}
\end{itemize}
}}
\end{center}
\section{作为假说的分类}

\begin{quotation}
本节探讨分类作为科学假说的特殊性质。我们将分析科学分类如何不仅仅是事物的简单归类,而是反映自然界内在联系的理论假设。通过理解分类系统如何帮助科学家发现新知识、预测未知性质,我们将认识到一个好的分类系统既有理论基础,也有实际应用价值。
\end{quotation}

分类与划分紧密相关。但是分类,即在哲学中所称的"等级"(genus,种的上义词)中的"种"(species)的秩序,可以看做是假说,而严格地说,划分并非如此。分类不仅是要求完备和互斥,而且要求物体、生物或者思想在"本质特征"的基础上得以分组,其目的是对所考察的现象进行解释,或者至少为之提供一个解释的框架。这些"本质特性"因而必须是解释性的。

存在这样一个观点,假说仅在比较发达的科学中而不是在相对不发达的科学中才发挥重要作用。这个观点应当被拒绝。有人主张,尽管说明性假说在物理学、化学这样的科学中起重要作用,它们在生物学和社会科学中则没有这样的作用,或者至少现在没有这样的作用;后者仍然处于描述性阶段,而假说方法对所谓描述性的科学如植物学、历史学是不合适的。我们很容易对这个观点给出反驳。对描述本性的考察将显示,描述本身是建立在假说之上的,或者说描述本身包含假说。假说在生物学的不同体系的分类法或分类学中是基本的,在历史学或其他社会科学中假说也是基本的。

在历史科学中,假说的重要性容易得到阐明,我们首先讨论它。一些历史学家相信,历史研究能够揭示存在着的单个宇宙目的或模式,该目的或模式或者是宗教的或者是自然的,它对有记载的历史的整个进程进行解释或说明。其他的历史学家则否认有任何这样的宇宙设计的存在,但他们坚持认为,历史研究将揭示某个历史规律,该规律解释过去事件的实际次序,并能够用来预测未来。无论哪种观点的历史学家,他们寻求的说明必须解释过去记载的事件,并被它们所证实。因而,无论是哪种观点,历史学是一个理论性的科学而不仅仅是描述性的科学,必须承认假说在历史学家事业中的中心作用。

然而,有第三种历史学家,他们更为谦虚地设定他们的目标。根据他

们的观点,历史学家的任务只是简单地将过去进行编年史记录,即以他们的编年史顺序将过去的事件进行简单的记录。似乎是,根据这种观点, "科学的"历史学家没有进行假说的必要,因为他们所关心的只是事实本身,而非与事实有关的理论。

但是过去的事件没有像该观点试图使我们相信的那样容易编年。过去本身不能提供这种记录。现在能够得到的是现在的记录和过去的痕迹。它们的范围是:从关于过去的官方档案,到对半传说式的英雄的征服行为的赞美的史诗;从以前的历史学家的作品,到考古学家挖掘出土的过去年代的物品。这些只是历史学家能够获得的事实,而从这些事实中他们必须推论得出过去事件的本质——这是他们描述的目的。不是所有的假说是全称的,有些是特称的。历史学家关于过去的描述是特称假说,使用该假说的意图是解释现有数据,而现有数据构成了它的证据。

从大范围来看,历史学家犹如侦探。 ${ }^{[30]}$ 他们的方法是共同的,遇到的困难也类似。其困难主要是证据不充足,并且,许多证据如果不是被笨手笨脚的地方警察所损坏,就是被相关的战争和自然灾害所破坏。正如罪犯可能留下了假的或误导的线索以甩掉追踪者,太多的现存"记录",据说是对过去的描述,而实际上是对过去的歪曲:或者是有意的,如"康斯坦丁的赠款"这样伪造的历史文档的案子,或者是无意的,如早期没有批判性的历史学家的著作。正如侦探建立和检验假说必须使用科学方法一样,历史学家也必须如此。即使将自己限于对过去的纯粹描述的那些历史学家,也必须使用假说来工作:他们是理论家,不管他们自己是如何认为的。

生物学家所处的位置稍微有利。他们处理的事实是现在的,易于检查。为了描述一个地区的动植物群落,他们不必精心构造历史学家那种遭到诟病的复杂推理。数据可以被直接地认识到。对这些项目的描述不是因果的,而只是系统的。他们被认为是对动物和植物进行分类,而不仅仅描述它们。但是分类和描述实际上是同一个过程。将一给定动物描述成食肉类,即是将它分类为一个食肉动物;将它归类为爬行类,即是将它描述成爬行动物。某个物体被描述具有一个给定属性,即是将之归类于具有该属性的对象类中的一个成员。

分类,正如通常理解的那样,不仅仅要将对象划分成不同的群体,而且要将每个群体进一步划分成次一级的群体或次一级的类,等等。这个模

式是我们大多数人所熟悉的:如果不是从学校学习中知道,那么可能是从 "动物,植物,还是矿物?"这个古老的游戏(或者更普遍地被叫做" 20个问题"${ }^{(1)}$ )中获得。分类是一个普遍的需要。原始人不得不将草根、浆果划分为可食用的还是有毒的,将动物分为危险的还是安全的,以及将部落分为友善的还是敌对的。所有人都根据自己的实际需要而进行区分,并忽视那些在他们的事务中不怎么重要的区别。农民会对谷物和蔬莱进行小心和仔细的分类,而将各种花只统称为"花";而卖花人却会细致地将他们的商品进行分类,而将农民的所有收成一起称为"农产品"。

我们对事物进行分类有两个基本的动机。一个是实践的,一个是理论的。某人仅有 3 或 4 本书,他对它们了如指掌,他一瞥就能够分辨它们,就此没有对它们进行分类的必要。但是在一个包含上万册书的图书馆里,情况便不同。如果不对图书进行分类,图书管理员就不能找到所需要的书,该图书馆的收藏将无实际用处。物体数量越大,越有必要对它们进行分类。分类的实际目的是使大量的采集成为可能。在图书馆、展览馆和各类公共记录展厅的情形中这特别明显。

当我们考虑分类的理论用途时,我们必须认识到,使用一种分类法或另一种分类法与真理和错误无关。可以用不同方式、以不同的观点来描述物体。使用的分类方案依赖于分类者的目的和兴趣。例如,图书管理员、图书装订商和藏书家对书的分类便有所不同。图书管理员根据书的内容或主题对书进行分类,图书装订商根据的是装订方式,图书收藏者根据的是印刷日期和相对稀有程度。当然我们不能穷尽各种可能性:图书包装者会根据书的形状和大小对书进行分类,而对书有其他兴趣的人根据他们不同的兴趣进行不同的分类。

那么,科学家的什么样的特别兴趣或目的,使他们偏爱一个分类方案而不是另外一个?科学家的目的是获得知识:不仅仅是关于这个或那个特别的事实的知识,尤其是关于用事实来确认的普遍定律的知识,以及事实之间因果连接的知识。从科学家的观点来看,一个分类方案比另外一个要好,一定程度上在于,在得出科学定律的过程中它是更富于成果的,并且在形成说明性假说过程中它是更有帮助的。

\footnotetext{(1)这是一个游戏。多人确定出一个东西,让另外的人通过询问问题来猜他们所确定的是什么东西。
}对物体进行分类,其理论的或科学的动机是增进这些物体知识的愿望。事物的知识的增加可以增进我们对事物的属性,它们的相似性及差别,以及它们的相互关系的进一步理解。一个分类方案如果只为狭窄的实际目的而制定,就可能抹杀了重要的相似性和差异性。把动物划分成"危险的"和"没有危险的",如把野猪和响尾蛇归为一类,把家猪和草蛇归为一类,这种划分为了强调表面的相似性,而忽视了更本质的相似性。对物体的任何科学的、富有成果的分类需要具有关于它们的大量的知识。对比较明显的特征的粗浅了解,会使人们将蝙蝠作为会飞的生物归为鸟类,把鲸作为生活在大海中的生物归为鱼类。如果我们具有更广阔的知识,我们便将蝙蝠和鲸两者均归为哺乳动物,因为它们都属于温血、胎生并哺育幼崽的动物——这些是分类所根据的更为重要的特征。

如果一个特征能够作为线索,以发现其他特征,它便是重要的特征。从科学的视角来看,一个重要的特征是指这样一个特征:它与许多其他的特征有因果连接关系,因而它能够作为最大数量的因果律的框架,并且有助于形成最普遍的说明性假说。因此,这样的分类方案是最好的,如果它建立在所要分类的物体的最重要的特征之上的话。但是正如我们已经强调的那样,我们事先并不知道会得到哪些因果律,而且因果律本身也带有假设的性质,因而,在哪个分类方案是最好的问题上的决策本身就构成一个假说,一个后续研究可能将之否定的假说。如果后来的研究揭示了,其他特征更为重要(即它与大量的因果律和说明性假说相关),我们能够合理地预期,原来的分类方案应当被否决,我们会选取基于更重要的特征之上的新的分类方案。

分类方式是假说的观点被它在科学中实际所起的作用所证实。分类学是生物学中一个正统的、重要的并且欣欣向荣的分支学科。在生物学中,某些分类方式.如林奈的分类法,被采纳、使用,后来因有了更好的方案而被弃用;更好的方案本身在新的数据下也经受着修改。一般来说,在科学的早期阶段或不发达阶段,分类最为重要。然而,随着科学的发展其重要性不一定总是降低。例如,由门捷列夫表所表明的元素的标准分类法、仍然是化学家的一个重要工具。

前面对自然科学中使用分类的解释,可启发我们进一步认识在历史研究中使用分类的重要性。我们已经说明,历史学家对过去事件的描述本身即是基于冒前资料之上的假说。然而,假说在描述的历史学家的事业中发

挥着另外一个同等重要的作用。任何数量级的历史事件都不能被完完全全地描述。即使人们能够知道它的所有细节,历史学家也不可能将之全部写进著作之中。生命过于短暂,它不允许人们对事物进行详尽无遗的描述。因此,历史学家必须对过去进行有选择地记载,记录下的仅仅是过去的一些特征。历史学家进行选择的基础是什么?显然,历史学家要叙述的是有意义的或重要的,而忽略无意义的或琐碎的。这个或那个历史学家的主观偏见会使他或她过分强调历史进程中宗教、经济、人物或其他某个方面的作用。但如果历史学家们考虑到,要做出客观的或科学的评价,他们便会重视那些能够形成因果律和普遍的说明性假说的因素。自然,这样的评价会随着进一步研究而经受着改变。

西方第一个历史学家希罗多德(Herodotus)细致描写了他编入编年史的事件,有人物的、文化的,以及政治的、军事的。所谓第一位科学的历史学家修昔底德,将自己的写作更多地限于政治和军事方面。在很长一段时间里,大多数历史学家跟随修昔底德,但是现在钟摆正摆向另外一个方向:历史学家十分重视过去的经济和文化方面。正如生物学家的分类方案包含了他们的假说——通过这个假说生物的特征与最大数量的因果律相关联,历史学家选择用一个典型事件集合而非另外一个集合来描述过去事件,这种选择包含了他们的假说:什么样的典型事件与最大数量的其他典型事件因果地连接在一起。这样的假说是必需的,哪怕历史学家对过去进行系统的描述的工作只是刚刚开始。正是分类和描述——无论是生物学的还是历史学的一一所具有的假说性的特点,使我们将假说看成是科学探究的通用方法(the all-method)。 

\begin{center}
\fbox{\parbox{0.95\textwidth}{
\textbf{本节要点}
\begin{itemize}
\item \textbf{分类作为科学假说的本质}:
  \begin{itemize}
  \item 分类不仅是将事物归类,而是基于"本质特征"的理论假设
  \item 科学分类系统是对自然界内在联系的假设性解释
  \item 分类方案本身即是假说,可被后续研究证实或否定
  \end{itemize}
\item \textbf{分类在科学中的广泛应用}:
  \begin{itemize}
  \item 不仅在自然科学中,在历史等描述性学科中同样重要
  \item 分类在科学发展早期尤为重要,但在成熟科学中仍有价值
  \item 历史学家对事件的分类和描述同样包含假说性质
  \end{itemize}
\item \textbf{科学分类的评价标准}:
  \begin{itemize}
  \item 好的分类基于与多种特征有因果联系的重要特征
  \item 科学价值取决于能否揭示事物间的内在联系和规律
  \item 能够促进形成普遍定律和说明性假说的分类更有价值
  \end{itemize}
\item \textbf{分类的实践与理论意义}:
  \begin{itemize}
  \item 实践意义:使对大量对象的管理和利用成为可能
  \item 理论意义:增进对事物属性、相似性和差异的理解
  \item 反映科学家对事物本质特征的理解和认识目标
  \end{itemize}
\end{itemize}
}}
\end{center}

% 第十四章
\chapter{逻辑与认知}
\section*{14. 1 关于概率的几种观点}
在归纳逻辑中概率(probability)是一个核心概念,在前面关于科学方法的讨论中已多次阐述。一个假说即使符合所有接触到的事实,它也不是决定性地得以确立;它只具有或然性。我们看到,即使我们慎之又慎地使用实验探究的密尔方法,也不能决定性地证明我们所得到的因果律的真理性,而只是以高的或然性(概率)确证它们。即使最好的归纳论证也不具有有效演绎论证所拥有的那种确定性。

因而,恰当地说,我们对归纳逻辑的考察离不开对概率这个关键概念的分析。我们必须区分"盖然的"(probable)\footnote{(1)} 和"概率"的不同用法。下面三个命题显示了概率的最典型的用法:

1.一个投出去的硬币出现正面的概率是 $1 / 2$ 。

2.一个 25 岁的妇女过 26 岁生日的概率为 0.971 。

3.基于现有的证据,爱因斯坦相对论的正确性是高度盖然的。

存在使用"盖然的"和"概率"的其他情境,如我们说测量中的"可能错误"(probable errors),但这三种是最重要的。在前两个命题中,一个数字——被称为概率数值——被赋予一个特定事件;第三个命题则不同,它没有被赋予这样的数字。当我们谈论一个可疑的科学假说时,我们通常赋予一个盖然程度。比如,人们说达尔文理论比《创世记》中对生命起源的解释更可靠(probable),再比如,原子理论具有比其他的关于原子核内部结构的假说有较高的盖然度。

前两个例子中所给予的数字是十分有用的,并且似乎十分合理。但它们从何处得来?

硬币有两面:正面和反面。当硬币落地时,必定有一面朝上。两个机会中的一个机会是正面向上,因此以概率 $1 / 2$ 赋予正面。为了得到第二个例子中的概率值,我们必须进行死亡率统计并进行比较。在 1000 个庆祝

正如前两个例子所表明的,概率研究与赌博和死亡统计有关;事实上,现代概率研究开始于这两个领域。熟知的是,概率论起步于帕斯卡 (Blaise Pascal,1623-1662)和费玛(Pierre Fermat,1608-1665)关于机会赌博中合理赌注的通信,另外一种说法是,概率起源于帕斯卡给切瓦里•德•梅尔(Chevalier De Mere)一一个著名的赌徒——如何在掷骰子时下赌注的建议。与死亡率相关的是,自从1592年伦敦开始保存死亡记录;1662年,约翰•格朗特上尉发表了对这些记录的一个研究,探讨了从这些记录中用概率能够推得什么。可能是因为这复合的血缘,概率有如下两种解释。

\section*{A.概率的先验解释}
经过拉普拉斯、德摩根、凯恩斯等人的发展,关于概率本质的古典理论认为,概率是合理信念(rational belief)度的测定。当我们完全相信某个事情,我们信念度的测定被赋予数字1。当我们绝对相信一个特定事件不可能发生,该事件将发生的信念度被赋予数字 0 。因而,一个理性人在一个掷出去的硬市或者出现正面或者不出现正面上的信念度是 1 ;既出现正面又不出现正面的信念度是 0 。当他不能肯定的时候,他的合理信念度将为 0 和 1 之间的某个数。概率是关于事件的一个属性,它是人们合理地相信一个事件将发生的程度。或者说,概率是一个陈述或命题的谓词,一个完全理性的人总是依据这个值相信该陈述或命题。

在古典理论看来,概率总是部分有知和部分无知的结果。如果我们能够知道掷硬市的手指的精确运动,加上硬币的初始位置、大小、重量分布,人们确信能够预测硬市的轨道以及最后的不动的位置。但是,这些完全的信息不可能得到。我们只能知道某些信息:硬币有两面;它将下落;等等。因此,硬币正面向上的信念度由几种可能性所决定——这里可能性为 2 个、出现正面的可能性为 1 个。因而, $1 / 2$ 的概率值被赋予硬市出现正面的事件。类似的,人们要分发一副纸牌时,纸牌以一确定的顺序被分发。如果发牌诚实,牌中的黑桃、红桃、方块、梅花,以及 A、K、Q、 $J$ ,均以洗牌时确定的次序而得以分发。但我们不知道这个次序。我们只

知道总共的 52 张牌中有 13 张黑桃,因此,所发的第一张牌为黑桃的概率精确地为 $13 / 52$ ,或者 $1 / 4$ 。

这个观点为概率论的先验论观点。之所以如此称呼,是因为无须做实验,也无须选择样本来考察,就可以得到概率。只需要知道先行条件:纸牌中只有 13 张黑桃;总共有 52 张牌;发牌是诚实的,任何一张牌与其他牌有同样的机会被第一次分发。以先验的观点,为了计算在某些特定情形下一个事件发生的概率,我们把该事件能够发生的途径数,除以该情形下可能的结果总数——如果我们没有任何理由相信任何一个可能的结果比其他的更有可能的话。于是,一个事件的概率以一个分数来表示,其中,除数是等可能的结果总数,被除数是使待考察事件发生的结果数。一种诚实地出售 1000 张彩票的彩票发行,有 1000 个等可能的结果。因而,其中任何一张彩票能够中彩的概率是 1 除以 1000 ,即 $1 / 1000$ 。

\section*{B.概率的相对频率解释}
与先验论不同的一个理论认为概率是相对频率的一个度量。相对频率理论特别适合解释统计研究的概率判断。例如,保险公司精算师希望确定 25 岁妇女的死亡率。这里,我们有一个对象总体和一个属性:这个总体是 25 岁的妇女;属性是活到 26 岁生日。该理论中,赋予的概率是这样的相对频率测度:该人群以这个频率体现了这个被研究的属性。这里同样的是,概率也表示成分数。不过,在这里,分母是对象总体数量,分子是具有该属性的对象的数量。如果考察了 1000 个 25 岁的妇女的记录,发现其中有 971 个活到 26 岁生日,那么 0.971 就是该对象总体出现该属性的概率系数。这里没有出现合理信念。在概率的相对频率理论中,概率被定义成总体成员体现某一特定属性的相对频率。

必须说明的是,在这两个理论中,被赋予的概率是相对于采集的证据而言的。在相对频率理论中这是明显的。因为一个给定属性的概率,必定随着选择用来计算的特定对象总体的变化而变化。在上面用到的例子中,构成研究总体的 1000 名妇女是随机地从埃及人中选取,人们会发现,活到 26 岁的这个频率,将与随机地从法国人中选取的 1000 名妇女活到 26岁的概率不同。 25 岁的妇女再活 1 年的概率在埃及和在法国是不同的。类似的,在斯堪的纳维亚地区的人口总体中金发的概率高于在世界总人口中金发的概率。因此,在使用概率的相对频率理论时,一个关键的步骤是选择最合适的研究总体。

但是,在先验理论那里概率也是相对的。根据该理论的古典解释,任何事件均不具有内在的概率。一个事件的概率值之获得只能建立在做出其概率值指派的人所获得证据之上。这样的概率被解释成这样的一个观点,概率为合理信念的测度,因为一个理性人的信念随着他的知识的变化而变化。

譬如,假设两个人观看洗牌。当洗牌完成时,洗牌者因某个偶然的因素意外地使最上面的一张牌"露"了一下。第一个观察者看到了那张牌是黑的,但他没有看到是黑桃还是梅花。第二个观察者没有任何察觉。如果让这两个观察者估计第一张牌是黑桃的概率,第一个观察者将指派概率值 $1 / 2$ ,因为他知道有 26 张牌(黑色的牌),其中一半是黑桃。但第二个观察者将指派概率值 $1 / 4$ ,因为他知道的仅仅是 52 张牌中黑桃为 13 张。两个观察者对同一个事件指派了不同的概率。其中一个观察者犯了错误?当然没有:每个人相对于可用证据赋予了正确的概率。即使这张牌被翻开后为梅花,两个人的估计均是正确的。任何事件自身不具有内在的或关于它的概率。这里的意思是,任何预测所具有的不同概率是相对于不同背景而言的,即相对于不同证据集而言的。然而,人们在做出概率断定之前,应当尽可能地寻求收集最大量的证据集。

概率的这两种解释——相对频率解释和先验解释——在认为概率是相对于证据的这一点上是一致的。因而,这两个理论的信奉者在接受和使用概率计算上也是一致的。下一节将介绍概率计算的初步知识。 
\
\\section*{14.2 概率计算}
我们来确定一个复合事件的概率。复合事件可以被看做由多个事件构成的整体。例如,我们问:从一副牌中连续抽出两张黑桃的概率是多少?连续抽两张牌这样的复合事件是一个由两个部分组成的整体。这两个部分是,第一次抽出黑桃的事件,和第二次抽出黑桃的事件。再举一个例子,新娘和新郎活到庆祝金婚纪念日的复合事件,是由新娘再活 50 年的事件和新郎再活 50 年的事件,以及不发生离婚的事件组成的。当人们知道各个组成事件是如何相互关联的时候,人们能够根据单个事件的概率而求得该复合事件的概率。因而,我们把"概率计算"——用单元事件的概率计算出复合事件的概率——规定为纯数学的一个分支。

概率计算在日常生活中是极其有用的。知道某个结果的可能性可以帮助我们进行决策,而使我们做事谨慎。因而,其基本定理的掌握和运用是逻辑研究最有用的结果之一。

概率计算最容易用机会游戏(games of chance)——掷骰子、玩扑克等等——的术语来解释。原因是,这些游戏所限定的人工世界使概率定理的直接使用成为可能。因此,尽管概率计算有广泛的应用范围,在这一章中,我们通过赌博中引申出来的问题,初步地阐明概率计算。在阐释过程中我们使用了概率的先验理论,当然,所有结果经过少量的重新解释后也能够用相对频率理论来表述和分析。 
\input{chapter14/section14-3.tex}
\section{替代性发生的概率}

\begin{quotation}
\textit{计算多个事件中至少一个发生的概率是概率论中的重要问题,它不仅帮助我们评估赌博风险,还在医学诊断、工程安全性和科学推理中发挥着关键作用。}
\end{quotation}

我们有时对一系列事件中的一个或多个发生的概率感兴趣。例如,当我们掷两枚硬币时,我们想知道一枚或另外一枚着地时正面向上的可能性是多少。在抽两张牌的扑克牌游戏中,我们想知道抽到或者一张黑桃或者一张梅花的概率为多少。替代性发生的概率 ${ }^{(1)}$ 总是大于每个事件发生的概率。如同在共同发生的情况下,两个事件共同发生的概率将小于其中一个单独事件发生的概率。

\footnote{(1)这里将 alternative occurrence 译成"替代性发生",意为两个或两个以上的事件至少一个发生,亦可译为"择代性发生"。}

\subsection{计算替代性发生概率的基本方法}

人们如何计算替代性发生的概率?在共同发生的情况下我们将两个分数相乘,得到了一个低的概率值。不同的是,当我们求替代性发生的概率时,我们将分数相加,概率值增加。然而,我们同样碰到了复杂的情况,需要我们将之分为两类进行考虑。

替代性发生的事件可能是相互排斥的,也可能不是相互排斥的。两个事件如果不能同时发生,它们便是相互排斥的。如果我掷两枚硬币并得到两个正面,我不能在这两次投掷中得到两次反面。两个正面和两个反面明显是相互排斥的。但是如果我从一副牌中抽取两张牌,两张牌中一张是黑桃或一张是梅花,是可以出现的不同情形。在一副牌中抽取两张牌,"抽到一张黑桃"和"抽到一张梅花"不是相互排斥的事件。计算替代性发生概率的方法将因事件是否为相互排斥而大大不同。我们依次来分析。

\subsection{相互排斥事件的替代性发生}

如果事件是相互排斥的,计算直接而且容易:将两个事件的概率进行简单相加即可。将一枚硬币掷两次,出现两次正面或者两次反面的概率是多少?自然的是,一个概率与另外一个概率相加。两次正面的概率为 $1 /$ 4 ,两次反面的概率为 $1 / 4$ ,或者两次正面或者两次反面的概率为 $1 / 4+1 /$ $4=1 / 2$ 。

计算两个相互排斥事件构成的复合事件的概率公式为:

$$
P(a \text { 或 } b)=P(a)+P(b)
$$

这是加法定理,它可以推广到适合任意多的事件( $a, b, c \cdots \cdots$ )。如果所有的事件是相互排斥的,它们中至少一个发生的概率为它们的概率和。

通过对扑克牌游戏中能够被分发到同花色牌(5 张牌为同一种花色)的概率的计算,我们来说明上面的公式。这里,有四个相互排斥的可能性:拿到 5 张黑桃的事件,拿到 5 张红桃的事件,拿到 5 张梅花的事件,拿到 5 张方块的事件。让我们先来计算拿到 5 张黑桃的概率。这是一个由 5 个明显非独立的子事件构成,因为分发到黑桃将减低下面得到黑桃的概率。利用非独立的乘法定理,我们有 $13 / 52 \times 12 / 51 \times 11 / 50 \times 10 / 49 \times$ $9 / 48=33 / 66640$ 。其他每一个可能性(5 张红桃、5 张梅花、5 张方块)均有与此相同的概率。这 4 种同花色是相互排斥的事件,因此,利用加法定理,得到任何一种同花色的概率为 $33 / 66640+33 / 66640+33 / 66640+$ $33 / 66640=33 / 16660$ 。

\subsection{多球问题示例}

再举一个例子。从两个袋子中各摸一个球,一个袋子中有两个白球和 4 个黑球,另一个袋子中有 3 个白球和 9 个黑球,摸到两个同颜色的球的概率是多少?我们感兴趣的概率的事件是两个互斥事件的替代性发生:一个是摸到两个白球的事件,另外一个是摸到两个黑球的事件。分别计算这两个事件的概率,然后相加。摸到两个白球的概率为 $2 / 6 \times 3 / 12=1 / 12$ ;摸到两个黑球的概率为 $4 / 6 \times 9 / 12=1 / 2$ 。因此摸到两个同样颜色的概率为 $1 / 12+1 / 2=7 / 12$ 。

\subsection{非互斥事件的替代性发生}

到目前为止,我们对替代性发生的讨论都是针对互斥事件。但是我们必须计算由非互斥的两个或更多的事件中至少一个发生的复合事件的概率。例如,将一枚硬币掷两次,至少得到一次正面的概率是多少?事件不是互斥的,因为可以肯定的是,能够两次投掷都得到正面。我们清楚,第一次投掷得到正面的概率为 $1 / 2$ ,第二次投掷得到正面的概率也是 $1 / 2$ ,但这两个概率之和为 1 ,即事件为确定的,然而至少一次投掷为正面是不确定的!这个例子说明,当我们计算非互斥事件替代性发生的概率时,加法定理不能直接应用。我们可以用两个间接的方法来计算这种类型的概率。

\subsection{计算非互斥事件替代性发生的方法}

计算两个非互斥事件中至少一个发生的概率的第一个方法,要求我们将事件分解成互斥事件。在求解将一枚硬币投掷两次得到至少一面为正面的概率的问题中,等可能的状态是 $\mathrm{H}-\mathrm{H}, \mathrm{H}-\mathrm{T}, \mathrm{T}-\mathrm{H}, \mathrm{T}-\mathrm{T}$ 。它们是相互排斥的,每一个状态的概率为 $1 / 4$ 。前三个状态为我们要求的;即在前三个状态中的任何一个状态发生的条件下,两次投掷中至少一次为正面就是真的事实。于是,投掷出至少一面为正面的概率,等于所有符合要求的互斥状态的单独概率之和,即为 $1 / 4+1 / 4+1 / 4=3 / 4$ 。

计算两个非互斥事件中至少一个发生的概率的另外一种方法,建立在这样的事实上,没有状态既是满足条件的又是不满足条件的。我们用 $a$ 表示将一枚硬币投掷两次得到至少一次正面的事件,那么,我们用符号 $\bar{a}$ 表示与 $a$ 不同的事件,即两次投掷没有一次正面的事件。因为没有状态既是我们需要的又是我们不需要的,$a$ 与 $\bar{a}$ 是相互排斥的,$a$ 与 $\bar{a}$ 不能都发生。由于每个状态必定是,或者是这个事件或者不是这个事件,可以肯定的是,或者 $a$ 或者 $\bar{a}$ 必定发生。我们给不能发生的事件指派概率值 0 ,给必定发生的事件指派概率值 1。下面两个等式是成立的:

$$
P(a \text { 且 } \bar{a})=0
$$

$$
P(a \text { 或 } \bar{a})=1
$$

这里,$P(a$ 且 $\bar{a})$ 为 $a$ 和 $\bar{a}$ 均发生的概率,$P(a$ 或 $\bar{a})$ 为 $a$ 或者 $\bar{a}$ 发生的概率。由于 $a$ 和 $\bar{a}$ 是互斥的,可以应用加法定理。我们得到:

$$
P(a \text { 或 } \bar{a})=P(a)+P(\bar{a})
$$

$$
P(a)+P(\bar{a})=1
$$

由上式得到非常有用的等式:

$$
P(a)=1-P(\bar{a})
$$

于是,我们可以通过计算一个事件不发生的概率,再用 1 减去这个数,就得到一个事件发生的概率。我们应用这个方法来求投掷两次硬币得到至少一次正面事件的概率。我们容易看到,该事件不发生的唯一情况为,两次投掷均为反面——这是不满足条件的状态,由乘法定理,概率值为 $1 / 2 \times$ $1 / 2=1 / 4$ ,据此,投掷两次硬币中得到至少一次正面事件确实发生的概率为 $1-1 / 4=3 / 4$ 。

\subsection{实际应用:概率的违反直觉性质}

有时,应用概率计算得到一个尽管正确的结果,但是与我们对已知事实进行因果分析后所期望的结论不同。这样的结果被认为是违反直觉的。当一个问题的解违反直觉的时候,人们可能在概率判断上发生错误。这样 \zhtext{"自然"}的错误驱使人们在狂欢场所及其他地方进行如下的赌博\zhtext{。}摇三个骰子,赌场庄家与你打一赔一的赌(如果打一元的赌,如果你赢了,你取回你押的一元,庄家再给你一元),庄家赌三个骰子中均不出现么点(一点)\zhtext{。}骰子有六面,每个面上有不同的数字\zhtext{。}你有三个机会得到么点,表面上看,这似乎是一个公平的赌博\zhtext{。}

事实上,这不是一个公平的赌博\zhtext{。}利用这个与直觉相反的事实的骗子能够获得丰厚的利润\zhtext{。}这个赌博仅当在这样的条件下才是公平的\zhtext{:}三个骰子中的一个骰子出现某一特定点数后,而使另外两个骰子中的任一个骰子不出现该点数\zhtext{。}这显然不正确\zhtext{。}粗心的下注者错误地\zhtext{(}和下意识地\zhtext{)}认为它们具有互斥性\zhtext{。}然而它们不是相互排斥的,一些投掷中两个或者三个骰子会出现相同点数\zhtext{。}试图通过确定并计算所有可能结果,以计算至少一个幺点出现的结果数,很快就会发现,这样的努力是难以进行的\zhtext{。}但是,因为任何给定点数的出现并不能排除其他骰子也出现同样点数,这样的赌博确实是一个欺骗\zhtext{。}我们先确定输的概率然后从 1 中减去这个概率值,从而计算出胜出的概率,此时,这个欺骗便显出来\zhtext{。}单个骰子非-幺点\zhtext{(}出现 2 点,或 3 点,或 4 点,或 5 点,或 6 点\zhtext{)}向上的概率为 $5 / 6$ \zhtext{。}输的概率为 3 个非 - 幺点出现向上的概率,其概率\zhtext{(}由于骰子之间是不相互影响的\zhtext{)}为 $5 / 6 \times 5 / 6 \times 5 / 6=125 / 216$ ,即 0.579 \zhtext{。}下注者摇到至少一个幺点的概率为 $1-125 / 216=91 / 216$ ,即为 0.421 \zhtext{。}这就是赌博的原理\zhtext{!}

\subsection{双骰赌博问题分析}

让我们用概率求解一个中等难度的问题。双骰赌博(craps)是用两个骰子进行。下注者如果在第一次投掷中得到(总和为) 7 点或者 11 点,那么他赢了;如果在第一次投掷中得到 2 点或 3 点或 12 点,那么他就输了。如果第一次摇出的骰子出现其他的点数( $4 \zhtext{、} 5 \zhtext{、} 6 \zhtext{、} 8 \zhtext{、} 9 \zhtext{、} 10$ ),摇骰子者继续摇盅。在以后的摇骰子中,如果出现与上次同样的点数,那么下注者赢了;如果出现 7 点,那么下注者输了。双骰赌博被普遍认为是公平的赌博——下注者有一半的获胜机会。真是这样的吗?让我们计算在双骰赌博中下注者获胜的概率。

为此,我们首先得有不同点数出现的概率。两个骰子落下后,有 36 个等可能的情况。2 点只有一种出现方式,即 1-1。它出现的概率为 $1 / 36$ 。只有一种状态出现 12 点,其概率为 $1 / 36$ 。有两种状态得到 3 点: $1-2,2-1$ ,点数 3 的概率为 $2 / 36$ 。类似的,得到 11 点数的概率为 $2 / 36$ 。 3 种状态可以得到 4 点: $1-3,2-2$ 及 $3-1$ ,因此点数为 4 的概率为 $3 / 36$ 。类似的,点数为 10 的概率值为 $3 / 36$ 。由于有 4 种状态得到 5 点 $(1-4,2-3,3-2,4-1)$ ,其概率为 $4 / 36$ ,这同样是点数 9 的概率。得到点数 6 的状态有 5 种 $(1-5,2-4,3-3,4-2,5-1)$ ,点数 6的概率为 $5 / 36$ ,点数 8 的概率值与此相同。有 6 种可能状态产生点数 $7(1-6,2-5,3-4,4-3,5-2$ 及 6-1),摇出点数 7 的概率值为 $6 / 36$。

下注者在第一次摇骰子中获胜的概率为出现点数 7 的概率和出现点数 11 的概率之和,其值为 $6 / 36+2 / 36=8 / 36$ 。第一次摇骰子中他输的概率为出现点数 $2 \zhtext{、} 3 \zhtext{、} 12$ 的概率和,值为 $1 / 36+2 / 36+1 / 36=4 / 36$ ,即 $1 / 9$ 。在第一次摇骰子中下注者赢的可能性为输的可能性的两倍。然而在第一次摇骰子中下注者很有可能既不赢又不输,即摇到点数 $4 \zhtext{、} 5 \zhtext{、} 6 \zhtext{、} 8 \zhtext{、} 9$ 或 10 。如果掷出这 6 个数中的一个,下注者得再次摇盅,直到该点数重新出现——下注者赢了,或者点数 7 出现——下注者输了。第一次摇骰子中没有出现的点数和点数 7 的状态可以忽略,因为它们不起决定作用。假定下注者在第一次摇骰中得到点数 4 ,下一次摇骰子中起决定作用的是出现点数 4 或者 7 。在决定作用的摇骰子中,等可能的状态是使点数出现 4 的 3种组合 $(1-3 \zhtext{、} 2-2 \zhtext{、} 3-1)$ ,和使点数 7 出现 6 种组合;因而第二次投掷得到点数 4 的概率为 $3 / 9$ 。第一次摇骰子中得到 4 点的概率为 $3 / 36$ ,因此,第一次摇得点数 4 、第二次又摇得点数 4 而未出现点数 7 的概率为 $3 /$ $36 \times 3 / 9=1 / 36$ 。类似的,下注者第一次摇得点数 10 、第二次又摇得点数 10 而未出现点数 7 的概率也是 $3 / 36 \times 3 / 9=1 / 36$ 。

下注者赢的方式有 8 个不同种类:第一次出现点数 7 或点数 11 ;或者第一次得到 $4 \zhtext{、} 5 \zhtext{、} 6 \zhtext{、} 8 \zhtext{、} 9 \zhtext{、} 10$ 中的一个点数,并且第二次得到同样的点数。这些方式都是相互排斥的,所以下注者总的赢的概率为能够获胜的各个可能性的概率之和。这个概率为 $6 / 36+2 / 36+1 / 36+2 / 45+25 / 396+$ $25 / 396+2 / 45+1 / 36=244 / 495$ 。如果表示成分数,概率值为 0.493 。这表明在双骰赌博中,下注者赢的机会小于输的机会——尽管略小,但仍小于 0.5 。

\begin{center}
\begin{tabular}{|p{0.95\textwidth}|}
\hline
\textbf{加法定理} \\
\hline
\textbf{计算两个或更多的替代性的事件发生的概率的方法:} \\[6pt]
\textbf{A.如果事件(如 $a \zhtext{、} b$ )是相互排斥的:} \\
至少一个事件发生的概率为它们概率的简单相加:
$P(a \text { 或 } b) = P(a)+P(b)$ \\[6pt]
\textbf{B.如果事件(如 $a \zhtext{、} b \zhtext{、} c$ )不是相互排斥的:} \\
它们中至少一个发生的概率由下面的方法确定: \\
(1)将满足条件的状态区分为互相排斥的事件,然后将这些事件的概率相加; \\
(2)计算这些可能事件不发生的概率,然后用1减去这个概率。 \\
\hline
\end{tabular}
\end{center}

\begin{center}
\fbox{\parbox{0.95\textwidth}{
\textbf{本节要点}
\begin{itemize}
\item \textbf{替代性发生的基本概念}:
  \begin{itemize}
  \item 替代性发生指多个事件中至少一个发生的情况
  \item 替代性发生的概率总大于单个事件的概率
  \item 计算方法取决于事件是否互斥
  \end{itemize}
\item \textbf{互斥事件的替代性发生}:
  \begin{itemize}
  \item 互斥事件指不能同时发生的事件
  \item 概率通过简单加法计算:$P(a \text{ 或 } b) = P(a) + P(b)$
  \item 适用于掷骰子点数\zhtext{、}抽牌花色等互相排斥的情形
  \end{itemize}
\item \textbf{非互斥事件的替代性发生}:
  \begin{itemize}
  \item 事件可以同时发生,需要特殊处理
  \item 方法一:将事件分解为互斥状态后相加
  \item 方法二:用1减去所有事件都不发生的概率
  \end{itemize}
\item \textbf{与直觉相悖的概率}:
  \begin{itemize}
  \item 直觉判断在概率计算中常常出错
  \item 非互斥事件的替代性发生尤其容易误判
  \item 精确计算能避免在赌博和决策中的错误判断
  \end{itemize}
\end{itemize}
}}
\end{center}
% The rest of the file, including exercises, is removed. 
\input{chapter14/section14-5.tex}
\section*{第14章概要}
在所有归纳论证中,前提只是以某个概率度对结论进行支持,在科学假说中我们只是简单地把这个度描述成"更"可能或"不太"可能。本章

说明了如何能够将一个定量的概率(表示为 0 与 1 之间的小数)分派给归纳结论。

14. 1 节给出两种概率概念,它们都可以给予定量配置:(1)相对频率理论,根据这个理论,概率被定义成一个类的成员出现一个特定属性的相对频率。(2)先验理论,根据这个理论,一个事件发生的概率,由事件能够发生的途径数除以等可能的后果数来确定。

这两个理论均与 14.2 节介绍的概率计算相协调。如果复杂事件的各单元事件的概率能够确定,复杂事件的概率就能够计算出。在概率计算中使用两个基本的定理:乘法定理和加法定理。

如果复杂事件是一个共同发生的事件,两个或更多的单元事件均发生的概率可用乘法定理得到, 14.3 节给出说明。乘法定理断定,如果单元事件是独立的,它们共同发生的概率等于它们各自的概率的积。但如果单元事件是不独立的,可以运用通用乘法定理:( $a$ 且 $b$ )的概率等于 $a$ 的概率乘以在 $a$ 发生的条件下 $b$ 的概率。

如果复杂事件是替代性发生的(两个或更多事件中至少一个发生的概率),可应用加法定理,在 14.4 节得到说明。加法定理断定,如果单元事件是相互排斥的,它们的概率之和给出了替代性发生的概率。但如果单元事件不是相互排斥的,它们替代性发生的概率可以这样计算:(1)通过将所需要的场合分解成相互独立的事件,然后将他们的概率相加;或者 (2)确定至少替代性发生事件将不发生的概率,然后用 1 减去这个数。

为了计算一项投资或赌博的预期值(14.5节的内容),我们既要考虑可能后果的概率,又要考虑每个可能事件下获得的收益。先将每个后果预期回报与该回报发生的概率相乘,然后将这些乘积相加便得到投资的预期值。 
\section*{【注释】}
[1]病人爱丽莎•爱亚拉(Anissa Ayala)在手术成功的一年后结了婚,救了她的命的妹妹玛丽莎-爱亚拉(Marissa Ayala)在婚礼上为她撒花。该例子的具体细节见 1993 年第 12 月 Life 的报道。\\
[2]关于该问题的讨论参见:L.E.Rose,"Countering a Counter-Intuitive Proba- bility",Philosophy of Science 39 (1972):523-524;A.I.Dale,"On a Problem in Con- ditional Probability",Philosophy of Science 41 (1974):202-206;R.Faber,"Re-En-\\
countering a Counter-Intuitive Probability",Philosophy of Science 43 (1976):283- 285;S.Goldberg,"Copi's Conditional Probability Problem",Philosophy of Science 43 (1976):286-289。\\
[3]尽管下注于"每天 3 个数字"是不明智的,但它十分受欢迎,以至于现在一天开奖两次:中午和晚上。人们可能会认为,不是购买该奖券的那些人没有计算他们下注的期望值,就是这样的赌博给了他们满足,这个满足与他们下注的金钱期望值无关。\\
[4]事实上,持续出现一个结果(正面或反面)的情况包含在一个长的正面和反面(或者转轮中黑色和红色,等等)的随机序列之中,其频率比我们普遍认为的要高得多。出现一打正面不是十分稀奇的。如果赌博者从 $\$ 1$ 开始下注在反面,如果出现正面就持续加倍下注,在第 12 局它要求赌博者下 $\$ 2048$ 。第 12 局之后,第 13 局为反面的机会还是 $1 / 2$ ! 

% 后记部分
\backmatter
\chapter*{参考文献}
% 可以添加参考文献

\end{document}