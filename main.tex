% This LaTeX document needs to be compiled with XeLaTeX.
% ===============================================
% 逻辑学导论 - 专业排版配置文件
% 优化版本:增强视觉效果和专业性
% ===============================================

% 文档类设置 - 使用专业书籍格式
\documentclass[a4paper,12pt,twoside,openright]{book}

% ===============================================
% 核心宏包加载
% ===============================================
\usepackage[utf8]{inputenc}
\usepackage{amsmath,amsfonts,amssymb}
\usepackage[version=4]{mhchem}
\usepackage{extpfeil,stmaryrd}
\usepackage{graphicx}
\usepackage[export]{adjustbox}
\graphicspath{ {./images/} }
\usepackage{multirow,array,booktabs}
\usepackage{fvextra,csquotes}

% 中文支持
\usepackage{xeCJK}
\usepackage{polyglossia}
\usepackage{fontspec}
\usepackage{unicode-math}

% 页面布局和设计
\usepackage{geometry}
\usepackage{fancyhdr}
\usepackage{titlesec}
\usepackage{indentfirst}
\usepackage{setspace}

% 颜色和视觉效果
\usepackage{xcolor}
\usepackage{tikz}
\usepackage{tcolorbox}
\usepackage{mdframed}
\usepackage{framed}
\usepackage{enumitem}
\usepackage{caption}
\usepackage{subcaption}

% 参考文献支持
\usepackage[style=gb7714-2015,backend=biber,language=chinese]{biblatex}
\DeclareLanguageMapping{chinese}{chinese}
\addbibresource{references.bib}

% PDF链接和书签
\usepackage[hidelinks,bookmarks=true,bookmarksnumbered=true,colorlinks=true,linkcolor=LogicBlue,citecolor=LogicGreen,urlcolor=LogicRed]{hyperref}

% ===============================================
% 颜色定义 - 专业配色方案
% ===============================================
\definecolor{LogicBlue}{RGB}{25,118,210}      % 主色调:深蓝色
\definecolor{LogicGreen}{RGB}{46,125,50}      % 辅助色:深绿色
\definecolor{LogicRed}{RGB}{198,40,40}        % 强调色:深红色
\definecolor{LogicGray}{RGB}{97,97,97}        % 中性色:深灰色
\definecolor{LogicLightBlue}{RGB}{227,242,253} % 浅蓝色背景
\definecolor{LogicLightGreen}{RGB}{232,245,233} % 浅绿色背景
\definecolor{LogicLightGray}{RGB}{250,250,250} % 浅灰色背景
\definecolor{LogicDarkBlue}{RGB}{13,71,161}   % 深蓝色
\definecolor{LOGICDARKBLUE}{RGB}{13,71,161}   % 深蓝色(兼容性)
\definecolor{LogicAccent}{RGB}{255,193,7}     % 金黄色强调

% ===============================================
% 页面布局设置
% ===============================================
\geometry{
  top=3cm,
  bottom=3cm,
  left=3cm,
  right=2.5cm,
  headheight=25pt,
  headsep=20pt,
  footskip=30pt,
  marginparwidth=2cm,
  marginparsep=0.5cm
}

% ===============================================
% 字体设置 - 专业字体配置
% ===============================================

% 中文字体设置 - 使用思源黑体
\setCJKmainfont{Source Han Sans SC}[
  BoldFont = Source Han Sans SC Bold,
  ItalicFont = Source Han Sans SC Light,
  BoldItalicFont = Source Han Sans SC Bold,
  Scale = 1.0
]
\setCJKsansfont{Source Han Sans SC}[
  BoldFont = Source Han Sans SC Bold,
  Scale = 1.0
]
\setCJKmonofont{Source Han Sans SC}[Scale = 0.9]

% 英文字体设置 - 使用专业字体组合
\setmainfont{Times New Roman}[
  BoldFont = Times New Roman Bold,
  ItalicFont = Times New Roman Italic,
  BoldItalicFont = Times New Roman Bold Italic,
  Scale = 1.0
]
\setsansfont{Arial}[
  BoldFont = Arial Bold,
  ItalicFont = Arial Italic,
  BoldItalicFont = Arial Bold Italic,
  Scale = 0.95
]
\setmonofont{Courier New}[Scale = 0.85]

% 数学字体设置
\setmathfont{TeX Gyre Termes Math}

% 中文字体命令
\newCJKfontfamily\cnfont{Source Han Sans SC}
\newCJKfontfamily\cntitle{Source Han Sans SC Bold}
\DeclareTextFontCommand{\textcn}{\cnfont}
\DeclareTextFontCommand{\texten}{\normalfont}

% 中文排版设置
\punctstyle{kaiming}
\xeCJKsetup{CJKecglue={\hskip 0.15em plus 0.05em minus 0.05em}}
\XeTeXlinebreaklocale "zh"
\XeTeXlinebreakskip = 0pt plus 1pt

% 中文字体命令
\newcommand{\zhtext}[1]{{\cnfont #1}}
\newcommand{\cn}[1]{{\cnfont #1}}

% ===============================================
% 页眉页脚设计 - 专业样式
% ===============================================
\pagestyle{fancy}
\fancyhf{}

% 页眉设置
\fancyhead[LE]{\color{LogicGray}\small\thepage\quad\textbar\quad\leftmark}
\fancyhead[RO]{\color{LogicGray}\small\rightmark\quad\textbar\quad\thepage}

% 页眉装饰线
\renewcommand{\headrulewidth}{0.5pt}
\renewcommand{\headrule}{\hbox to\headwidth{%
  \color{LogicBlue}\leaders\hrule height \headrulewidth\hfill}}

% 页脚设置
\fancyfoot[C]{}
\renewcommand{\footrulewidth}{0pt}
\setlength{\headheight}{25pt}

% 章节首页样式
\fancypagestyle{plain}{%
  \fancyhf{}
  \fancyfoot[C]{\color{LogicGray}\small\thepage}
  \renewcommand{\headrulewidth}{0pt}
  \renewcommand{\footrulewidth}{0pt}
}

% ===============================================
% 章节标题设计 - 现代专业样式
% ===============================================

% 章标题格式
\titleformat{\chapter}[display]
{\normalfont\cntitle\color{LogicDarkBlue}}
{\tikz[remember picture,overlay]{
  \fill[LogicLightBlue] (current page.north west) rectangle ([yshift=-3cm]current page.north east);
  \node[anchor=south east,font=\fontsize{60}{60}\selectfont\color{LogicBlue!30}]
    at ([xshift=-2cm,yshift=-1cm]current page.north east) {\thechapter};
}
\vspace{-2cm}\chaptertitlename\ \thechapter}
{20pt}
{\Huge\color{LogicDarkBlue}}
\titlespacing*{\chapter}{0pt}{0pt}{40pt}

% 节标题格式
\titleformat{\section}
{\normalfont\Large\bfseries\color{LogicBlue}}
{\colorbox{LogicLightBlue}{\makebox[2em]{\color{LogicDarkBlue}\thesection}}\quad}
{0pt}
{}
\titlespacing*{\section}{0pt}{3.5ex plus 1ex minus .2ex}{2.3ex plus .2ex}

% 子节标题格式
\titleformat{\subsection}
{\normalfont\large\bfseries\color{LogicGreen}}
{\thesubsection\quad}
{0pt}
{}
\titlespacing*{\subsection}{0pt}{3ex plus 1ex minus .2ex}{2ex plus .2ex}

% 子子节标题格式
\titleformat{\subsubsection}
{\normalfont\normalsize\bfseries\color{LogicGray}}
{\thesubsubsection\quad}
{0pt}
{}
\titlespacing*{\subsubsection}{0pt}{2.5ex plus 1ex minus .2ex}{1.5ex plus .2ex}

% ===============================================
% 段落和间距设置
% ===============================================
\setlength{\parindent}{2em}
\setlength{\parskip}{0.5ex plus 0.2ex minus 0.1ex}
\onehalfspacing
\setlength{\baselineskip}{1.2\baselineskip}

%New command to display footnote whose markers will always be hidden
\let\svthefootnote\thefootnote
\newcommand\blfootnotetext[1]{%
  \let\thefootnote\relax\footnote{#1}%
  \addtocounter{footnote}{-1}%
  \let\thefootnote\svthefootnote%
}

%Overriding the \footnotetext command to hide the marker if its value is `0`
\let\svfootnotetext\footnotetext
\renewcommand\footnotetext[2][?]{%
  \if\relax#1\relax%
    \ifnum\value{footnote}=0\blfootnotetext{#2}\else\svfootnotetext{#2}\fi%
  \else%
    \if?#1\ifnum\value{footnote}=0\blfootnotetext{#2}\else\svfootnotetext{#2}\fi%
    \else\svfootnotetext[#1]{#2}\fi%
  \fi
}

% ===============================================
% 专业文本框和环境设置
% ===============================================

% 定义专业的文本框样式
\tcbuselibrary{skins,breakable,theorems}

% 定义重点提示框
\newtcolorbox{logicbox}[1][]{
  enhanced,
  colback=LogicLightBlue,
  colframe=LogicBlue,
  boxrule=1pt,
  arc=3pt,
  left=10pt,
  right=10pt,
  top=8pt,
  bottom=8pt,
  breakable,
  #1
}

% 定义定理框
\newtcolorbox{theorembox}[1][]{
  enhanced,
  colback=LogicLightGreen,
  colframe=LogicGreen,
  boxrule=1pt,
  arc=3pt,
  left=10pt,
  right=10pt,
  top=8pt,
  bottom=8pt,
  breakable,
  fonttitle=\bfseries,
  title=定理,
  #1
}

% 定义例题框
\newtcolorbox{examplebox}[1][]{
  enhanced,
  colback=LogicLightGray,
  colframe=LogicGray,
  boxrule=1pt,
  arc=3pt,
  left=10pt,
  right=10pt,
  top=8pt,
  bottom=8pt,
  breakable,
  fonttitle=\bfseries,
  title=例题,
  #1
}

% 定义注意事项框
\newtcolorbox{notebox}[1][]{
  enhanced,
  colback=yellow!10,
  colframe=LogicAccent,
  boxrule=1pt,
  arc=3pt,
  left=10pt,
  right=10pt,
  top=8pt,
  bottom=8pt,
  breakable,
  fonttitle=\bfseries,
  title=注意,
  #1
}

% ===============================================
% 列表和枚举优化
% ===============================================
\setlist[itemize]{
  leftmargin=2em,
  itemsep=0.3ex,
  parsep=0.2ex,
  topsep=0.5ex
}

\setlist[enumerate]{
  leftmargin=2em,
  itemsep=0.3ex,
  parsep=0.2ex,
  topsep=0.5ex
}

% ===============================================
% 目录格式优化
% ===============================================
\renewcommand{\contentsname}{\centerline{\Large\bfseries\color{LogicDarkBlue} 目录}}

% ===============================================
% 图表标题优化
% ===============================================
\captionsetup{
  font={small,bf},
  labelfont={color=LogicBlue},
  textfont={color=LogicGray},
  margin=20pt,
  skip=10pt
}

% ===============================================
% 自定义命令
% ===============================================

% 强调文本命令
\newcommand{\logicemph}[1]{\textcolor{LogicBlue}{\textbf{#1}}}
\newcommand{\logicterm}[1]{\textcolor{LogicGreen}{\textbf{#1}}}
\newcommand{\logicwarn}[1]{\textcolor{LogicRed}{\textbf{#1}}}

% 章节总结命令
\newcommand{\chaptersummary}[1]{
  \begin{logicbox}[title=本章要点]
    #1
  \end{logicbox}
}

\begin{document}

% 标题页
\begin{titlepage}
\centering
\vspace*{2cm}
{\color{black!80}\rule{\textwidth}{1pt}}
\vspace{1.5cm}

{\Huge\bfseries\color{black!90} 逻辑学导论 \par}
\vspace{0.5cm}
{\large\itshape 系统梳理逻辑学基本概念与方法 \par}

\vspace{1.5cm}
{\color{black!80}\rule{\textwidth}{1pt}}
\vspace{2cm}

\begin{minipage}{0.8\textwidth}
\centering
\begin{quotation}
\large\textit{"逻辑将使人们明辨是非,认清谬误,\\
培养批判性思维,提高分析能力。"}
\end{quotation}
\end{minipage}

\vfill

{\large \today \par}
\end{titlepage}

% 前言
\frontmatter
\chapter*{前言}
% 内容简介

\begin{center}
\rule{0.5\textwidth}{0.4pt}
\end{center}

\begin{quotation}
\large\textit{『逻辑学为人类思维提供指引,就如同灯塔为迷航的船只指明方向。』}
\end{quotation}

\begin{center}
\rule{0.5\textwidth}{0.4pt}
\end{center}

\vspace{1em}

本书旨在系统地介绍\textbf{逻辑学}的各个方面,从基础概念到高级主题,力求用简明有逻辑的语言将逻辑学的理论和应用梳理清楚。全书结构清晰,内容由浅入深,既有理论探讨,又有实践指导。

作为一门研究\textbf{推理方法}与\textbf{原则}的学科,逻辑学在哲学、数学、计算机科学等诸多领域都有着重要的应用。本书从逻辑学的基本概念出发,逐步深入到以下核心内容:

\begin{itemize}
  \item \textbf{命题逻辑}:研究命题之间的逻辑关系
  \item \textbf{谓词逻辑}:分析命题内部结构
  \item \textbf{模态逻辑}:处理必然性与可能性
  \item \textbf{归纳逻辑}:探讨归纳推理的方法
  \item \textbf{逻辑悖论}:分析逻辑中的矛盾现象
  \item \textbf{非经典逻辑}:介绍多值逻辑等新发展
\end{itemize}

本书既可作为大学逻辑学课程的参考教材,也适合对逻辑学感兴趣的读者自学使用。

\begin{center}
\fbox{\parbox{0.9\textwidth}{
  \centering
  通过学习本书,读者将能够:\\
  \begin{minipage}{0.85\textwidth}
  \begin{itemize}
    \item 识别和评估日常生活中的论证
    \item 避免常见的逻辑谬误
    \item 构建清晰、有效的论证
    \item 培养批判性思维能力
  \end{itemize}
  \end{minipage}
}}
\end{center}

\vspace{1em}

希望本书能够帮助读者培养\textbf{逻辑思维},提高分析问题和解决问题的能力,在这个信息爆炸的时代,具备清晰、理性的思考方式。

% 目录
\tableofcontents
\clearpage

% 正文
\mainmatter

% 第一部分:逻辑与语言
\chapter{逻辑与语言}
% 本章导言
\begin{quote}
\textit{逻辑学是研究\textbf{正确推理方法}和\textbf{原理}的学问。本部分介绍逻辑学的基本概念,帮助读者建立对逻辑学的初步认识,为后续内容奠定基础。通过理解逻辑的本质、论证的结构以及常见谬误,读者将能够识别和构建有效的论证。}
\end{quote}

\section{什么是逻辑学}

\begin{logicbox}[title=引言]
\textit{逻辑学为我们提供了判断推理正确性的方法和标准,使我们能够有效地思考和表达自己的想法。}
\end{logicbox}

\logicterm{逻辑学}是研究用于区分\logicemph{正确推理}与\logicemph{不正确推理}的方法和原理的学问。在日常生活和学术研究中,我们经常需要做出各种推理和判断,但并非所有的推理都是可靠的。正确推理有着明确的客观标准,只有掌握了这些标准,我们才能有效地运用它们。

\logicemph{逻辑学研究的核心目标}就是发现、阐述并系统化这些推理标准,使我们能够准确地检验论证的有效性,从而将\logicterm{有效论证}与\logicwarn{无效论证}明确区分开来。

\subsection{逻辑学的研究对象}

逻辑学具有\logicemph{普遍适用性},其研究的推理形式遍及人类活动的所有领域:

\begin{itemize}
  \item \logicterm{科学与医药}——实验设计、假说检验、诊断推理
  \item \logicterm{伦理与法律}——道德论证、法律推理、判决依据
  \item \logicterm{政治与商务}——政策分析、商业决策、风险评估
  \item \logicterm{运动与博弈}——策略分析、竞技判断、规则应用
  \item \logicterm{日常生活}——购物决策、人际交往、问题解决
\end{itemize}

虽然这些领域中使用的推理在内容上千差万别,但它们都遵循相同的逻辑原理。\logicwarn{重要的是},逻辑学关注的不是论证的具体\logicterm{题材内容},而是论证的\logicterm{逻辑形式}和\logicterm{推理品质}。我们的目标是掌握检验和评价各种论证的通用方法。

\subsection{论证与推理}

在逻辑学研究中,我们需要明确区分\logicterm{推理过程}和\logicterm{推理结果}。逻辑学家主要关注的不是推理的心理\logicterm{思维过程},而是这种过程的客观\logicterm{产物}——\logicemph{论证}。

\begin{theorembox}[title=论证的特征]
\logicterm{论证}是推理活动的具体表现形式,它具有以下重要特征:
\begin{itemize}
  \item 可以用语言完整地表达出来
  \item 具有明确的逻辑结构
  \item 能够被客观地检验和分析
  \item 独立于个人的思维过程
\end{itemize}
\end{theorembox}

对于任何一个论证,逻辑学家都会提出以下核心问题:

\begin{examplebox}[title=逻辑学的核心问题]
\begin{enumerate}
  \item 论证所得出的\logicemph{结论}是从论证所使用的\logicemph{前提}或假定推出的吗?
  \item 论证的前提能够为接受其结论提供良好的理由吗?
\end{enumerate}
\end{examplebox}

这两个问题的答案决定了论证的逻辑价值。如果论证的前提确实能够为接受结论提供充分的逻辑支持——即当前提为真时,结论必然为真或很可能为真——那么这个论证就是\logicemph{逻辑上正确的}。反之,如果前提无法为结论提供足够的支持,那么这个论证就是\logicwarn{逻辑上有缺陷的}。

\subsection{逻辑学习的意义}

学习逻辑学并不是进行正确推理的\logicwarn{必要条件}。正如优秀的运动员不一定需要深入了解运动生理学就能取得卓越成绩一样,许多人在没有系统学习逻辑学的情况下也能进行有效的推理。相反,仅仅掌握逻辑学理论知识也不能\logicwarn{保证}在实际中总是进行正确的推理。

\logicemph{但是},系统学习逻辑学确实能够显著提高我们进行正确推理的能力和概率。这种提升主要体现在以下两个方面:

\begin{enumerate}
  \item \logicemph{错误识别与预防能力}:逻辑学为我们提供了系统的方法来检验推理的正确性,使我们能够更敏锐地识别各种\logicwarn{推理错误}和\logicwarn{逻辑谬误}。一旦我们熟悉了这些常见的"自然"错误模式,就能在自己的推理中有效地避免它们,从而提高推理的可靠性。

  \item \logicemph{推理技能的系统训练}:逻辑学不仅是一门\logicterm{理论科学},更是一门\logicterm{实践艺术}。它为我们提供了系统训练\logicemph{分析论证}和\logicemph{构建论证}能力的机会。通过大量的练习和应用,我们可以逐步掌握推理的技术要领,培养敏锐的逻辑直觉。本书正是基于这一理念,提供了丰富的实例分析和练习题目。
\end{enumerate}

\subsection{逻辑学的局限与价值}

我们必须认识到逻辑学的\logicwarn{适用范围和局限性}。在人类生活的某些领域,纯粹的逻辑分析可能并不总是最合适的方法。例如,在处理情感问题、艺术创作或人际关系时,\logicterm{直觉}、\logicterm{情感}和\logicterm{经验}往往比严格的逻辑论证更为重要和有效。

\logicemph{然而},在需要进行理性分析和客观判断的场合——如科学研究、法律推理、政策制定、学术讨论等——\logicemph{正确的逻辑推理}始终是我们最可靠的工具和最坚实的基础。

\chaptersummary{
\logicterm{逻辑学}是研究正确推理方法和原理的学科,具有普遍适用性。它的核心目标是帮助我们区分有效论证与无效论证,提高推理的准确性和可靠性。虽然逻辑学有其适用范围的局限,但在需要理性分析的领域中,掌握逻辑学的方法与技术是进行正确推理的重要保障。
}
\section{命题与语句}

\begin{logicbox}[title=引言]
\textit{命题是逻辑推理的基本单位,理解命题的本质及其表达方式是进行逻辑分析的前提。}
\end{logicbox}

\logicterm{命题}是一种可以被肯定或否定的东西。也就是说,命题不同于问题、命令和感叹。问题可以被提问,命令可以被下达,感叹可以被发出,但它们本身都不能被肯定或否定。唯有命题断定了事情是(或不是)如此这般,因而也唯有命题才会是\logicemph{真的}或者是\logicemph{假的}。真与假并不适用于问题、命令或感叹。

再者,任一命题必是\logicemph{或真或假}的,尽管我们可能并不知道某一特定命题究竟是真的还是假的。"宇宙中其他星球上有生命存在"这个命题,就是一个我们迄今还不知道其真假的命题。但对地球外生命之存在的这种断定本身或者是真的,或者不是真的。简言之,或真或假是命题的一个基本特征。

\subsection{命题与语句的区别}

\begin{theorembox}[title=重要区分]
依学界惯例,要把\logicterm{命题}与用来断定命题的\logicterm{语句}区别开来。两个由不同语词以不同方式组成的语句,可能在同一语境中具有同样的意义,被用来表达同一个命题。
\end{theorembox}

例如:

\begin{itemize}
  \item Leslie won the election.(莱斯利赢了这场选举。)
  \item The election was won by Leslie.(这场选举由莱斯利赢得。)
\end{itemize}

这显然是两个不同的语句,前一个由四个词组成,后一个是六个词,以及起首词不同等等。而这两个陈述句无疑具有相同的意义。\logicterm{命题}这个术语所指谓的就是人们通常使用陈述句所断定的东西。

再者,一个语句总是使用它的特定语言的语句,而命题并不属于任何特定的语言,一个特定的命题可以在许多语言中被断定。例如:

\begin{itemize}
  \item It is raining.(天在下雨。下同)
  \item Está lloviendo.
  \item Il pleut.
  \item Es regnet.
\end{itemize}

这当然是四个不同的语句,分属不同的语言:英语、西班牙语、法语和德语。但它们都具有同样的意义,从而都可以用来断定同一命题。

\subsection{语境与命题}

在不同的\logicterm{语境}中,同一个语句也可能被用来做非常不同的陈述。例如:

\begin{center}
"美国最大的州曾经是一个独立的共和国。"
\end{center}

这个语句在20世纪上半叶说出,就是做了关于得克萨斯州的一个真陈述;而在现在说出就做了关于阿拉斯加州的一个假陈述。显然,时间语境的变化,可以使完全相同的语句断定非常不同的命题或陈述。("命题"和"陈述"这两个术语并不完全同义,但在逻辑研究的文本中它们经常被用做同义词。有些逻辑学专家更喜欢使用"陈述"而不愿意使用"命题",但在逻辑学历史上后者更为常用。本书同时使用这两个术语。)

\subsection{简单命题与复合命题}

上面所举出的命题的例子都是\textbf{简单命题}:"莱斯利赢了这场选举","天在下雨"等等,然而命题也经常是\textbf{复合的}——在一个命题中包含着别的命题。考虑如下关于1945年希特勒第三帝国末日的一段话:

\begin{quotation}
美军与俄军正迅速赶往易北河会师。英军已兵临汉堡和不来梅城下,把占领丹麦的德军置于被切断后路的险境。意大利的波伦亚已经失守,而亚历山大率领的盟军部队正向波河流域挺进。俄军已于4月13日攻克维也纳,正沿着多瑙河乘胜前进。\cite{shirer1960}
\end{quotation}

这段话就含有几个复合命题。例如,"英军已兵临汉堡和不来梅城下",就是"英军已兵临汉堡城下"与"英军已兵临不来梅城下"这两个命题的\textbf{联言式}。而这个联言命题本身又作为分支属于一个更大的联言命题:"英军已兵临汉堡和不来梅城下,(英军)把占领丹麦的德军置于被切断后路的险境。"这段话中的每一个命题都是被肯定的,也就是说,都被断言为真。肯定两个命题的联言式,就等于同时肯定这两个分支命题。

但是,也有一些复合命题并不断定其所有分支命题为真。例如:

\begin{center}
"巡回法庭或者是有用的,或者是无用的。"\cite{lincoln1861}
\end{center}

这是一个\textbf{选言命题}(或称析取命题),它并没有肯定任何一个分支命题,而只是肯定了整个复合的"或者一或者"析取命题。析取命题为真时,其某个分支命题可以为假。再如:

\begin{center}
"如果上帝不存在,则有必要捏造一个上帝。"\cite{voltaire1770}
\end{center}

这个复合命题是一个\textbf{假言命题}(或称条件命题),其支命题也同样没有被肯定,既没有肯定"上帝不存在",也没有肯定"有必要捏造一个上帝",而只是通过这种假言或条件陈述肯定了整个"如果一则"命题。即使分支命题均为假,条件陈述亦可为真。

\chaptersummary{

  命题是断定事实的陈述,必定或真或假,可以通过不同语句表达,\\
  且可以组合形成各种复合结构。

}

本书将逐次分析多种简单命题和复合命题的内在结构。
\section{论证、前提与结论}

\begin{logicbox}[title=引言]
\textit{论证是逻辑思维的核心,通过合理的前提推导出有效的结论是逻辑学的基本过程。}
\end{logicbox}

在逻辑学中,\logicterm{命题}是构成\logicterm{论证}的基本要素。我们需要区分两个相关但不同的概念:

\begin{theorembox}[title=推论与论证的区别]
\begin{itemize}
  \item \logicterm{推论}:从一个或多个命题出发,得出另一个命题的\logicemph{思维过程}
  \item \logicterm{论证}:推论过程的\logicemph{客观表现},即具有特定结构的命题系列
\end{itemize}
\end{theorembox}

逻辑学家主要关注的是论证而非推论过程本身。通过分析论证中命题之间的逻辑关系,我们可以判定相应的推论是否正确。\logicemph{每一个推论过程都对应着一个可以被分析的论证结构}。

\subsection{论证的本质}

\logicterm{论证}是逻辑学研究的核心对象。在逻辑学的严格意义上,论证是指这样一种命题组合:其中一个命题(结论)从其他命题(前提)中推导出来,后者为前者的真实性提供逻辑支持或根据。

\logicwarn{注意}:"论证"一词在日常语言中有多种用法(如争论、辩论等),但在逻辑学中具有特定的技术含义。

\begin{theorembox}[title=论证的逻辑结构]
一个真正的论证必须具备以下结构特征:
\begin{itemize}
  \item \logicterm{前提}:作为推理起点的命题,为结论提供支持或根据
  \item \logicterm{结论}:从前提中推导出的命题,是论证要证明的目标
  \item \logicterm{推理关系}:前提与结论之间的逻辑联系
\end{itemize}

\logicemph{重要区别}:论证不是命题的简单堆积,而是具有特定逻辑结构的命题系列。一段包含多个相关命题的文字未必构成论证。
\end{theorembox}

\subsection{最简论证的形式}

最简单的论证形式是\logicterm{单前提论证},由一个前提和一个从该前提推导出的结论组成。这类论证可以有多种表达方式:

\begin{examplebox}[title=分句表达的论证]
\begin{quotation}
在地球上最先出现生命时没有人存在。因此,任何关于生命起源的陈述都应视为理论的而非事实的陈述。
\end{quotation}

\logicemph{分析}:
\begin{itemize}
  \item \logicterm{前提}:在地球上最先出现生命时没有人存在
  \item \logicterm{结论}:任何关于生命起源的陈述都应视为理论的而非事实的陈述
  \item \logicterm{推理指示词}:"因此"
\end{itemize}
\end{examplebox}

\begin{examplebox}[title=单句表达的论证]
\begin{quotation}
因为最近的进化史研究已经证明所有人都是从同一小群非洲祖先演变而来,若仍相信种族间有极大差异,则如同仍相信地球是扁平的一样荒谬可笑。\cite{hayden2000}
\end{quotation}

\logicemph{分析}:
\begin{itemize}
  \item \logicterm{前提}:最近的进化史研究已经证明所有人都是从同一小群非洲祖先演变而来
  \item \logicterm{结论}:相信种族间有极大差异如同相信地球是扁平的一样荒谬可笑
  \item \logicterm{推理指示词}:"因为"
\end{itemize}
\end{examplebox}

在论证中,\logicemph{结论和前提的位置顺序}是灵活的。结论可以出现在前提之前,这种情况同样常见:

\begin{examplebox}[title=结论在前的论证(分句形式)]
\begin{quotation}
食品与药物管理局应立即禁止烟草买卖。要知道,抽烟是导致死亡的一种最可预防的原因。\cite{ban1992}
\end{quotation}

\logicemph{分析}:
\begin{itemize}
  \item \logicterm{结论}:食品与药物管理局应立即禁止烟草买卖
  \item \logicterm{前提}:抽烟是导致死亡的一种最可预防的原因
  \item \logicterm{推理指示词}:"要知道"
\end{itemize}
\end{examplebox}

\begin{examplebox}[title=结论在前的论证(单句形式)]
\begin{quotation}
凡法皆恶,乃因凡法皆为自由之违背。\cite{bentham1802}
\end{quotation}

\logicemph{分析}:
\begin{itemize}
  \item \logicterm{结论}:凡法皆恶
  \item \logicterm{前提}:凡法皆为自由之违背
  \item \logicterm{推理指示词}:"乃因"
\end{itemize}
\end{examplebox}

\logicwarn{重要原则}:无论论证多么复杂,都必须具备基本的逻辑结构——一个或多个\logicterm{前提}为一个\logicterm{结论}提供支持。复杂论证可能包含复合命题或多层推理,但这一基本结构始终不变。

\subsection{论证与假言命题的区别}

由于论证由多个命题组成,\logicwarn{单一命题本身不能构成论证}。但某些复合命题在表面上与论证相似,容易造成混淆。我们必须学会准确区分它们:

\begin{examplebox}[title=假言命题(非论证)]
\begin{displayquote}
如果火星在其具有与地球相似的大气层和相似气候的早期曾有生命演化,那么目前科学家确信的在我们的星系中存在的无数颗其他星球上也会有生命演化。
\end{displayquote}

\logicemph{分析}:这是一个\logicterm{假言命题},具有以下特征:
\begin{itemize}
  \item 前件和后件都\logicwarn{没有被断定为真}
  \item 只断定了前件与后件之间的\logicterm{条件关系}
  \item 即使前件和后件都为假,整个假言命题仍可能为真
  \item \logicwarn{不构成论证}:没有前提支持结论的推理过程
\end{itemize}
\end{examplebox}

\begin{examplebox}[title=真正的论证]
\begin{quotation}
看来,目前科学家确信的在我们的星系中存在的无数颗其他星球上会有生命演化,因为火星在其具有与地球相似的大气层和相似气候的早期非常可能曾有生命演化。\cite{zare1996}
\end{quotation}

\logicemph{分析}:这是一个真正的\logicterm{论证},具有以下特征:
\begin{itemize}
  \item \logicterm{前提}被明确断定:火星非常可能曾有生命演化
  \item \logicterm{结论}从前提推导:无数颗其他星球上会有生命演化
  \item 存在明确的\logicterm{推理关系}:从前提到结论的逻辑推导
\end{itemize}
\end{examplebox}

\logicwarn{关键区别}:假言命题只是表达条件关系,而论证则是通过前提为结论提供支持。准确识别论证是逻辑分析的基础技能。

\subsection{有结构的命题系列与论证}

我们必须明确一个重要区别:\logicemph{所有论证都是有结构的命题系列,但并非所有有结构的命题系列都是论证}。

\begin{examplebox}[title=非论证的命题系列]
\begin{displayquote}
骆驼并不在驼峰中储水。它们每次喝水都非常猛,在十分钟的时段中能饮入28加仑水,把这些水均匀地分布到全身。而后其耗水却非常节俭。它们的尿液黏稠、粪便干燥,并以浅呼吸而紧闭其口。如非不得已,它们一般不出汗……在失水程度达到体重的三分之一时也能存活,然后再痛饮一次并且感觉良好。\cite{langewiesche1996}
\end{displayquote}

\logicemph{分析}:这段文字虽然包含多个相关的命题,但它是\logicterm{描述性叙述}而非论证:
\begin{itemize}
  \item 各命题都是关于骆驼的\logicterm{事实陈述}
  \item 没有前提与结论的\logicwarn{推理关系}
  \item 目的是\logicterm{信息传达}而非\logicterm{逻辑证明}
  \item 不存在"因此"、"所以"等推理指示词
\end{itemize}
\end{examplebox}

\logicwarn{判断标准}:区分论证与非论证的关键在于是否存在前提为结论提供逻辑支持的推理结构。

\chaptersummary{
\logicterm{论证}是由\logicemph{前提}和\logicemph{结论}组成的特殊命题系列,其中前提为结论提供逻辑支持。论证具有明确的推理结构,这使它区别于假言命题和一般的描述性命题系列。理解论证的本质和结构是进行逻辑分析的基础。
}
\section{论证的分析}

\begin{logicbox}[title=引言]
\textit{论证分析是逻辑学中至关重要的技能,通过系统的分析方法可以清晰地展现复杂论证的结构和逻辑关系。}
\end{logicbox}

在现实中,论证的复杂程度差异很大。\logicemph{简单论证}可能只包含一个前提和一个结论,而\logicemph{复杂论证}则可能包含多个前提,这些前提以不同方式为结论提供支持。此外,论证中命题的数量、排列顺序以及相互关系也会呈现出多样化的特征。

为了准确理解和评估各种论证,我们需要掌握系统的\logicterm{论证分析技法},这些方法能够帮助我们清晰地揭示前提与结论之间的逻辑关联,从而更好地把握论证的内在结构。

\begin{theorembox}[title=两种主要分析方法]
论证分析主要有两种系统性方法:
\begin{itemize}
  \item \logicterm{解析法}(paraphrase):用清晰的语言和逻辑顺序重新表述论证中的各个命题,明确标识前提和结论
  \item \logicterm{图示法}(diagram):运用二维空间关系图直观地展示论证的逻辑结构和命题间的支持关系
\end{itemize}

这两种方法各有优势,可以根据论证的具体特点和分析需要选择最适合的方法,有时也可以结合使用。
\end{theorembox}

\subsection{解析法}

\logicterm{解析法}的核心是将原始论证重新组织,用清晰、简洁的语言明确列出各个前提和结论。让我们通过一个具体例子来说明:

\begin{examplebox}[title=原始论证]
现代鸟类并非从直立行走的兽脚类恐龙(包括霸王龙)进化而来,有三个主要理由。首先,大多数类鸟兽脚类恐龙化石发源时间比初始鸟类遗留的化石晚七千五百万年。……其次,鸟的祖先必定已适宜飞行,而兽脚类恐龙并不适宜飞行。第三个理由在于……兽脚类恐龙都有锯状牙齿,而鸟类没有锯状牙齿。${}^{[10]}$
\end{examplebox}

通过解析法,我们可以将这个论证的逻辑结构清晰地展现出来:

\begin{examplebox}[title=解析后的论证结构]
\logicemph{前提}:
\begin{enumerate}
  \item 类鸟兽脚类恐龙化石比初始鸟类遗留的化石发源时间要晚得多
  \item 鸟的祖先必定已适宜飞行,但兽脚类恐龙不适宜飞行
  \item 兽脚类恐龙都有锯状牙齿,而鸟类没有锯状牙齿
\end{enumerate}

\logicemph{结论}:现代鸟类并非从直立行走的兽脚类恐龙进化而来
\end{examplebox}

\logicemph{解析法的重要价值}在于它能够揭示论证中的\logicterm{隐含假设}和\logicterm{未明述前提}。通过系统的分析,我们可以发现原始表述中被默认但未明确说出的逻辑环节。

\begin{examplebox}[title=揭示隐含前提的例子]
大数学家哈代曾说:
\begin{quotation}
阿基米得将被永远记住而埃斯库罗斯会被遗忘,因为一种语言会消亡而数学理念不会消亡。${}^{[11]}$
\end{quotation}

通过解析法,我们可以揭示这个论证的完整逻辑结构:

\logicemph{第一个子论证}:
\begin{enumerate}
  \item 一种语言会消亡
  \item 埃斯库罗斯的伟大剧作使用一种语言
  \item 因此,埃斯库罗斯的成果终究会消亡
\end{enumerate}

\logicemph{第二个子论证}:
\begin{enumerate}
  \item 数学理念不会消亡
  \item 阿基米得的伟大工作使用数学理念
  \item 因此,阿基米得的成果不会消亡
\end{enumerate}

\logicemph{总结论}:阿基米得将被永远记住而埃斯库罗斯将被遗忘
\end{examplebox}

这种深入分析揭示了哈代这句话实际上包含了一个复杂的论证链,其中包含了若干可能存在争议的隐含前提。

\subsection{图示法}

\logicterm{图示法}是另一种强有力的论证分析工具,它通过视觉化的方式直观地展现论证的逻辑结构。图示法的基本步骤包括:

\begin{theorembox}[title=图示法的操作步骤]
\begin{enumerate}
  \item 为论证中的每个命题分配一个编号,并用圆圈标示
  \item 用箭头表示前提与结论之间的支持关系
  \item 通过图形布局直观展现论证的逻辑结构
\end{enumerate}
\end{theorembox}

图示法的优势在于它能够简洁地展现复杂的逻辑关系,避免重复叙述命题内容。让我们通过一个例子来说明:

\begin{examplebox}[title=图示法应用实例]
\begin{quotation}
(1)与许多人的认识相反,HIV检测呈阳性并不必定是死亡判决。一方面,(2)从(艾滋病病毒)抗体生发到出现临床症状平均将近十年时间;另一方面,(3)许多研究报告显示,相当数量的检测呈阳性者从未发展为艾滋病患者。${}^{[12]}$
\end{quotation}

使用图示法,我们可以清晰地展现这个论证的结构:
\end{examplebox}

\begin{center}
\includegraphics[width=\textwidth]{images/2025_05_15_6a28331d5e7c993ad07ag-030.jpg}
\end{center}

\logicwarn{图示法的适用性}:对于简单直接的论证,图示法可能显得过于复杂;但对于结构复杂的论证,图示法能够在二维平面上直观地展现逻辑关系,具有独特的优势。${}^{[13]}$

\begin{theorembox}[title=图示法的布局原则]
\begin{itemize}
  \item \logicterm{结论}通常置于图示的下方
  \item \logicterm{前提}置于结论的上方
  \item 同层级的前提在图中水平排列
  \item 箭头方向从前提指向结论
\end{itemize}
\end{theorembox}

图示法的一个重要优势是能够清晰地展现\logicemph{前提支持结论的不同方式}。在上述HIV检测的例子中,前提(2)和(3)都\logicemph{独立地}支持结论(1)。这意味着:
\begin{itemize}
  \item 每个前提单独就能为结论提供一定的支持
  \item 即使缺少其中一个前提,另一个前提仍能支持结论
  \item 这种\logicterm{分立性支持}在图示中通过分别的箭头清晰地表现出来
\end{itemize}

与分立性支持不同,有些论证中的前提必须\logicemph{联合起来}才能有效支持结论。这种情况下,单个前提无法独立提供充分的支持。

\begin{examplebox}[title=联合支持的例子]
\begin{quotation}
(1) 我们应该允许安乐死,(2) 如果这样做是最适当地维护所有当事人的利益的话。(3) 而有时候安乐死确实是最适当地维护所有当事人的利益。(4) 因此我们有时候应该允许安乐死。${}^{[14]}$
\end{quotation}

在这个论证中,前提(2)和(3)必须联合起来才能支持结论(4):
\end{examplebox}

\begin{center}
\includegraphics[width=\textwidth]{images/2025_05_15_6a28331d5e7c993ad07ag-031(1).jpg}
\end{center}

\begin{theorembox}[title=联合支持的特征]
在图示中,我们用\logicemph{托架线}连接需要联合支持的前提,这表明:
\begin{itemize}
  \item 单个前提无法独立支持结论
  \item 只有当所有相关前提都为真时,结论才能得到支持
  \item 缺少任何一个前提,整个论证就失去了说服力
\end{itemize}

在上述例子中:
\begin{itemize}
  \item 仅有原则(2)而无具体情况(3),结论无法成立
  \item 仅有具体情况(3)而无指导原则(2),结论同样无法成立
  \item 只有两个前提同时为真,结论才能得到有效支持
\end{itemize}
\end{theorembox}

\subsection{复杂论证的分析}

当论证结构变得复杂时,图示法的优势就更加明显。它能够清晰地展现那些用文字描述可能显得混乱的逻辑关系。

\begin{examplebox}[title=混合支持模式的复杂论证]
\begin{displayquote}
(1)沙漠高地是天文观测的良好场所。(2)其高度使得它们坐落于大气层之中,使得星光不用穿越整个大气层而到达望远镜。(3)沙漠的干燥度也使之相对较少受云雾干扰。(4)云雾对天空的遮蔽会使许多天文观测归于无用。${}^{[15]}$
\end{displayquote}

\logicemph{结构分析}:
\begin{itemize}
  \item 命题(1)是论证的\logicterm{结论}
  \item 命题(2)\logicemph{独立支持}结论——高度优势本身就足以说明沙漠高地适合天文观测
  \item 命题(3)和(4)\logicemph{联合支持}结论——必须结合起来才能构成完整的论证链
\end{itemize}

这个论证展现了\logicwarn{混合支持模式}:既有独立支持,又有联合支持。
\end{examplebox}

\begin{center}
\includegraphics[width=\textwidth]{images/2025_05_15_6a28331d5e7c993ad07ag-031.jpg}
\end{center}

\logicwarn{方法选择的考虑}:虽然图示法在展现复杂结构方面很有优势,但在某些情况下,解析法可能更为有效。特别是当论证包含\logicemph{隐含前提}时,解析法能够更直接地处理这些未明述的假设。

\begin{examplebox}[title=处理隐含前提的例子]
\begin{quotation}
只有当我能够做出其他选择时,我对我的行为才负有道德责任。因为一个人若无力避免某行为,就不应被认为对该行为负有道德责任。${}^{[16]}$
\end{quotation}

这个论证在表面上看起来是完整的,但实际上缺少一个关键的连接环节。
\end{examplebox}

\begin{theorembox}[title=解析法揭示隐含前提]
通过解析法,我们可以清晰地展现完整的论证结构:

\logicemph{明述前提}:
\begin{enumerate}
  \item 一个人若无力避免某行为,就不应被认为对该行为负有道德责任
\end{enumerate}

\logicemph{隐含前提}:
\begin{enumerate}
  \setcounter{enumi}{1}
  \item 只有当我能够做出其他选择时,我当下的行为才是我有能力避免的
\end{enumerate}

\logicemph{结论}:
只有当我能够做出其他选择时,我对我的行为才负有道德责任

\logicwarn{优势}:解析法能够直接列出隐含前提,而图示法则需要用特殊符号(如虚线圆圈)来标示补充的前提。
\end{theorembox}

\subsection{多重复合论证}

当文本包含多个相互关联的论证时,图示法能够清晰地展现这些论证之间的复杂关系。这种情况在政治文献、学术论文等复杂文本中经常出现。

\begin{examplebox}[title=多重论证的实例]
下面是马克思给恩格斯信件中的一段话:
\begin{displayquote}
(1)加速英国的社会革命就是国际工人协会的最重要的目标。(2)而加速这一革命的唯一办法就是使爱尔兰独立。因此,(3)国际的任务就是到处把英国和爱尔兰的冲突提到首要地位,(4)到处都公开站在爱尔兰方面。${}^{[17]}$
\end{displayquote}
\end{examplebox}

\begin{theorembox}[title=多重论证的识别原则]
\logicemph{论证数量的判断标准}:
\begin{itemize}
  \item 一段文字中论证的数目通常等于其中结论的数目
  \item 每个结论对应一个独立的论证
  \item 不同论证可能共享相同的前提
\end{itemize}

在上述例子中:
\begin{itemize}
  \item 有两个结论:(3)和(4)
  \item 因此包含两个论证
  \item 两个论证共享相同的前提:(1)和(2)
\end{itemize}
\end{theorembox}

\begin{center}
\includegraphics[width=\textwidth]{images/2025_05_15_6a28331d5e7c993ad07ag-032.jpg}
\end{center}

有时,含有两个结论从而有两个论证的一段话,却只含有一个前提。例如:

\begin{quotation}
年纪较大的妇女更难以抵制工作中的性骚扰和离开施暴的丈夫,因为年龄的偏见使她们不容易找到其他保护自己的方式。${}^{[18]}$
\end{quotation}

其中唯一的前提是年纪较大的妇女不容易找到保护自己的方式。该前提支持两个结论:年纪较大的妇女难以抵制工作中的性骚扰,以及她们(对已婚妇女而言)难以离开施暴的丈夫。我们通常用"\logicemph{单独论证}"一词指谓只有一个结论的论证,而不管有多少用以支持它的前提。

\subsection{论证中的命题次序}

当一段话中出现两个或更多论证,或一个论证中有两个或更多前提时,就需要弄清各个前提及结论出现的次序。结论可能在最后或最先出现,也可能出现在用以支持它的前提之间,如下例:

\begin{quotation}
穆斯林思想家启示的真正来源是《古兰经》及神圣先知的言论。因而很显然,穆斯林哲学并不是希腊思想的复制品,其所关心的主要是那些来自穆斯林和与穆斯林相关的特定问题。${}^{[19]}$
\end{quotation}

这段话中的结论"穆斯林哲学并不是希腊思想的复制品",出现在论证的第一个前提之后和第二个前提之前。

\subsection{串联式论证}

同一个命题既可在一个论证中做结论,又可在另一个论证中做前提,正如同一个人既可在一个场合做指挥者,又可在另一个场合做被指挥者。托马斯·阿奎那的著作中有一段话可以很好地说明这一点。他说:

\begin{displayquote}
人类的法律是为人类大众制定的,\\
大多数人在德行上是不完美的,\\
因此人类的法律不禁止一切罪恶。${}^{[20]}$
\end{displayquote}

该论证的结论随即被托马斯·阿奎那用做另一个完全不同的论证的前提:

\begin{displayquote}
恶行与善行相反,\\
但人类的法律不禁止一切罪恶,\\
因此人类的法律也不规定一切善行。${}^{[21]}$
\end{displayquote}

\subsection{浓缩论证的分析}

当一系列复杂的论证关系被压缩,对于这样一串浓缩论证的分析,完全解析法会提供很大的帮助。考虑如下论证集合:

\begin{quotation}
因为(1)出现在非洲人种身上的线粒体变种最多,科学家推断,(2)非洲人种的进化史最长,这表明(3)非洲人种可能是现代人类的起源。${}^{[22]}$
\end{quotation}

我们可以把这段论证图示如下:

\begin{center}
\includegraphics[width=\textwidth]{images/2025_05_15_6a28331d5e7c993ad07ag-034.jpg}
\end{center}

而对这同一串论证的分析,解析法尽管显得不够简洁,但更完整:

\begin{enumerate}
  \item 一个人种身上的线粒体变种越多,其进化史就越长;
  \item 出现在非洲人种身上的线粒体变种最多,\\
  因此非洲人种进化史最长。
\end{enumerate}

\begin{enumerate}
  \item 非洲人种进化史最长,
  \item 现代人类可能起源于进化史最长的人种,\\
  因此现代人类可能起源于非洲人种。
\end{enumerate}

这样的复合论证表明,一个孤立表达的命题既非前提也非结论。在一个论证中,作为假定出现的命题就是前提,被断定为从假定命题推出的命题就是结论。也就是说,"\logicterm{前提}"和"\logicterm{结论}"都是相对的(relative)术语。

\subsection{交织式论证}

几个论证复合在一起,语言表达上可能不是以串联的方式出现,而是以独特的方式相互交织,这就要求我们对它们做细致的分析。图示法特别适用于这种情况。例如,在约翰·洛克的名篇《政府论》中,下面一段话就有两个论证交织在一起:

\begin{quotation}
立法机构常年运作是不必要的,也是很不方便的;但行政机关常年运作是绝对必要的,因为不是总需要制定新的法律,但总需要执行已制定的法律。
\end{quotation}

上述论证的分支命题可以用数字表示为:(1)立法机构常年运作是不必要的,也是很不方便的,(2)行政机关常年运作是绝对必要的,(3)不是总需要制定新的法律,(4)总需要执行已制定的法律。将这段论证图示如下:

\begin{center}
\includegraphics[width=\textwidth]{images/2025_05_15_6a28331d5e7c993ad07ag-035.jpg}
\end{center}

这个图示表明,第二个论证的结论出现在第一个论证的结论和前提之间,第一个论证的前提出现在第二个论证的结论和前提之间。这个图示还表明,两个结论都出现在它们的前提之前。

这个图示同样也展示了支持刑罚威慑理论的古罗马哲学家塞涅卡的两个相关论证的逻辑结构:
\begin{quotation}
(1)惩罚罪行不是因为罪行已经发生,(2)而是为了不发生新的罪行。[因为](3)过去的罪行不能被取消,(4)但是可以预防将来的罪行。
\end{quotation}

在这段话中,"惩罚罪行不是因为罪行已经发生"是其中一个论证的结论,其前提是"过去的罪行不能被取消"。"惩罚罪行是为了不发生新的罪行"是这段话中第二个论证的结论,其前提是"惩罚罪行可以预防将来的罪行"。

\chaptersummary{
论证分析是逻辑学的核心技能,主要包括\logicemph{解析法}和\logicemph{图示法}两种方法。\logicterm{解析法}通过重新组织和明确表述来揭示论证的逻辑结构,特别适合处理隐含前提和复杂的推理链。\logicterm{图示法}通过视觉化的方式直观展现前提与结论的支持关系,能够清晰地区分独立支持、联合支持等不同模式。

掌握这两种分析方法,能够帮助我们更准确地理解和评估各种论证,识别其逻辑结构的优势和缺陷,从而提高逻辑思维能力。在实际应用中,应根据论证的具体特点选择最适合的分析方法,或者将两种方法结合使用。
}
\section{论证的辨识}

\begin{logicbox}[title=引言]
\textit{辨识论证中的前提和结论是逻辑分析的关键步骤,通过掌握辨识方法能够准确把握论证的真正含义。}
\end{logicbox}

\subsection{结论和前提指示词}

如前所见,出现在论证性话语中的命题的次序不能作为辨识其结论或前提的依据。那么用什么来辨识呢?有一些被叫做\logicterm{结论指示词}的词或短语有助于这样的辨识,因为它们典型地适合引导出一个论证的结论。下面所列的就是部分结论指示词${}^{(1)}$:

\begin{center}
\begin{tabular}{|l|l|}
\hline
therefore(所以) & for these reasons(基于这些理由) \\
\hline
hence(因此) & it follows that(可推得) \\
\hline
thus(因而) & we may infer(我们可推出) \\
\hline
so(故而) & I conclude that(我推断) \\
\hline
accordingly(由此可见) & which shows that(这表明) \\
\hline
in consequence(于是) & which means that(这意味着) \\
\hline
consequently(可得) & which entails that(据此可得) \\
\hline
proves that(据此证明) & which implies that(这蕴涵) \\
\hline
as a result(之所以) & which allows us to infer that(据此我们可以推出) \\
\hline
for this reason(为此缘故) & which points to the conclusion that(据此可得结论) \\
\hline
\end{tabular}
\end{center}

另一些词或短语典型地适合作为论证\logicterm{前提}的标志,因而被叫做\logicterm{前提指示词}。通常,但非总是,跟在任一前提指示词之后的命题就是某个论证的前提。下面所列的是部分前提指示词:

\begin{center}
\begin{tabular}{|l|l|}
\hline
since(因为) & may be inferred from(可从......推出) \\
\hline
because(由于) & may be derived from(可从......引申) \\
\hline
for(因) & may be deduced from(可从......得出) \\
\hline
as(根据) & in view of the fact that(有鉴于) \\
\hline
follows from(从......推出) & the reason is that(理由是) \\
\hline
as shown by(正如......所表明) & for the reason is that(理由在于) \\
\hline
inasmuch as(缘于) & as indicated by(正如......所示) \\
\hline
\end{tabular}
\end{center}

\footnotetext{(1)下列英汉指示词并非一一对应,在自然语言中识别论证,主要应诉诸语境分析。——译者注(以下凡脚注均为译者注,不一一标明)}

\subsection{语境中的论证}

上面所列的词和短语可以帮助我们认识话语中所含的论证,辨识其前提或结论,但它们在实际论证中并不一定出现。论证的出现可以由话语的背景或意义来表明。例如,一个女作家用如下陈述对吸烟提出严厉的批评:

\begin{quotation}
是否吸烟是在拥有关于烟草对健康的致命影响的充足信息的情况下做出的有意识的决定。无疑,那些对此做出不明智选择的人,应为其导致健康恶化的后果负责。${}^{[23]}$
\end{quotation}

这段话中既没有前提指示词,也没有结论指示词,但其中所含的论证是很清楚的。同样,下面一段话中所包含的论证可以从其所含命题本身的意义辨识出来:

\begin{displayquote}
近年来,有关死刑处罚的威慑作用的论证受到人们的反驳。谋杀率最高的二十个州中的十八个州有死刑处罚。谋杀率最高的十七个大城市拥有死刑处罚的司法权。过去十年中,得克萨斯州处死的罪犯比其他任何一个州都多,但得州仍有三个城市的谋杀率列于谋杀率最高的二十五个城市之中。近二十年来,有两个接壤的州的谋杀率基本相当,一个是没有死刑处罚的密歇根州,另一个是有死刑处罚的印第安纳州。${}^{[24]}$
\end{displayquote}

这些语段的论证性功能由它们的\logicterm{语境}和它们的\textbf{意义}展现出来。这就好比当我说晚饭时带只龙虾回家,你不会怀疑我是打算吃掉而不是饲养它。

另一个没有结论或前提指示词的论证出现在最近一篇为比例代表制进行辩护的文章中:

\begin{quotation}
单一成员选区(the single-member-district)的选举制度看来有许多严重的弊端。这种制度通常不能代表为数众多的选民的意志,它产生的立法机关不能准确反映公众的看法,它歧视第三党,挫伤选民投票的积极性。${}^{[25]}$
\end{quotation}

虽然可将这段话看做是首先陈述一个广泛的实际情况,然后用单一成员选区的选举制度的各种后果去阐明它,但也可以将这段话同样很好地理解为一个首先陈述其结论,然后从支持这个结论的前提推出这个结论的论证。

下面一段最高法院关于公立学校反种族隔离问题的评判中,有一个既无结论指示词又无前提指示词的更复杂一些的论证:

\begin{displayquote}
在学生人数上存在种族不平衡的现象,这不等于表明乡村学校不履行其法律义务。种族平衡本身不是目的。若种族不平衡是由于违背宪法使然,必须予以追究。只要杜绝违法的种族不平衡,乡村学校并没有义务去纠正因人口因素而造成的种族不平衡。${}^{[26]}$
\end{displayquote}

这段话的第一个句子是其所含论证的结论,这个结论可以被解释为"种族不平衡的存在并不表明乡村学校违背了法律"。我们怎么知道它是结论呢?这里\logicterm{语境}是决定性的:接在第一个句子后面的几个句子提供了之所以如此的理由。我们看到,在第一个句子中所指谓的"乡村学校"的行为处于争论之中;后面的几个句子表达了几个与乡村学校的行为有关的更一般的命题。话语中所选用的语词也给出了线索,虽然短语"不等于表明"不是结论指示词,但它传达了这样一个暗示,即第一个句子是这段话的逻辑终点。

包含论证的语段经常含有一些既不能作为前提也不能作为结论的附加材料。有时那些提供\textbf{背景信息}的材料能使读者(或听者)理解论证是关于什么内容的。在下面这段话中,论证出现在最后一句中,但如果不抓住它前面句子的内容,这个论证就不可理解:

\begin{quotation}
由于政府削减了对学生的财政援助,许多一流学院和大学都将较大比例的学费收入用做贫困学生的奖学金。正如慈善捐款可免征所得税一样,这部分学费也应享受税务免征。${}^{[27]}$
\end{quotation}

严格地讲,这段话中的第一个句子不是论证的一部分,但没有这句话我们就不能理解"这部分"学费是指用做奖学金的那部分学费。据此理解,我们就可以对该论证做如下解释:

\begin{enumerate}
  \item 对贫困者的慈善捐款是免征所得税的。
  \item 很大一部分学费收入被学校用来作为给贫困学生提供奖学金的慈善捐款。\\
  所以,作为给贫困学生提供奖学金的那部分学费应免征所得税。
\end{enumerate}

可见,上下文中命题之间相互参照,对于理解论证本身是必不可少的。哲学家阿瑟·叔本华在为自杀行为进行(无罪)辩护时所做的一个论证就例示了这种对相互参照的依赖:

\begin{displayquote}
如果刑法禁止自杀,那么在基督教中这并不是一个有根据的论证;而且这个禁令是荒唐的,因为有什么惩罚能让一个连死都不怕的人害怕呢?${}^{[28]}$
\end{displayquote}

这段话分号前面的句子既非前提也非结论,但是若没有它,我们就不知道在随后出现的论证结论("这个禁令是荒唐的")中的"禁令"乃指谓刑法的自杀禁令。

\subsection{非陈述形式的前提}

在上一例子中,论证前提以疑问句的形式出现:"有什么惩罚能让一个连死都不怕的人害怕呢?"而正如1.2节所述,问题无所断定,不表达命题,那么一个疑问句何以能起到前提的作用呢?这取决于该问句是\logicterm{反诘问句}。就是说,当提问者相信问题的答案显然或确定无疑时,可用问句暗示或设定一个前提。在上例中,叔本华认为他的问题的答案明显是"没有",因而,尽管以问句的形式出现,其论证的前提乃是这样一个不言而喻的命题:"没有任何惩罚能够让一个连死都不怕的人害怕。"

前提之一是反诘问句,而问题的答案被设定为明显的,这样的论证是非常普遍的,它们也很有修辞效果。可是这样使用问句是有风险的。例如苏格拉底的如下论证:

\begin{quotation}
美诺啊,如果没有人欲望痛苦,那么就没有人欲望罪恶;因为除了欲望和拥有灾难,还有什么是痛苦的呢?${}^{[29]}$
\end{quotation}

严格地讲,其中的问句既不真也不假。如果设定为显然或确定无疑的答案事实上并非如此,那么这个论证就是有缺陷的,而其缺陷正可能被问句掩盖。苏格拉底所假定的痛苦就是欲望和拥有罪恶这个答案是正确的吗?回答并不是显而易见的。

依赖反诘问句的论证,其结论有时是可疑的。人们使用设定有明显答案的问句来做论证的前提,有时就是为了回避直截了当地肯定其前提的责任,而实际上其设定的答案是含糊的甚或是假的。

不过,把真正的反诘问句作为前提使用确是一种很机敏的方法。通过暗示被期望的答案并且引导读者自己引出那个答案,可以增强论证的说服力。考虑下面两个使用反诘问句的例子。《新约全书》中有如下一段话:

\begin{displayquote}
人若说:"我爱神",却恨他的兄弟,就是说谎者:因为不爱他所看见的兄弟,如何能爱他看不见的神呢?${}^{[30]}$
\end{displayquote}

在最近的一篇对安乐死主张进行评论的文章中,有下面一段论证:

\begin{displayquote}
如果安乐死的权利基于自己的决定,那么将其限制到垂死病人就是不合情理的。如果人们有死亡权,那么为什么必须要等到已濒临死亡的时候才能行使这个权利呢?${}^{[31]}$
\end{displayquote}

在上面的两例论证中,两个设问的答案(一个是"不能爱他的兄弟的人也不能爱神",另一个是"人们不必要等到已濒临死亡的时候才能行使死亡权")都被假定是非常明显的。这些答案就是支持预期结论的前提。两个预期的结论分别是:"爱神的人不能恨他的兄弟"和"如果人们有基于自己决定的安乐死的权利,那么就不能将死亡权限制到垂死病人"。

\subsection{命令式的结论}

有时论证的结论可以采用祈使句或命令句的形式。在给出劝说我们去采取一个特定的行动的理由后,我们被指导要如此这般地去行动。例如《箴言》\footnote{Proverbs,见《旧约全书》。}中有这样一句话:

\begin{displayquote}
智慧为首,所以要求得智慧。
\end{displayquote}

在《哈姆雷特》中,波洛涅斯对他的儿子雷欧提斯提出如下忠告:

\begin{displayquote}
不要向人告贷,也不要借钱给人;\\
因为债款放了出去,往往不但丢了本钱,而且还失去了朋友;\\
向人告贷的结果,容易养成因循懒惰的习惯。${}^{[32]}$
\end{displayquote}

因为命令句像一般疑问句一样不能表达命题,所以(严格来讲)命令句不能作为论证的结论。但是为简明计,我们可以把在这些语境中的命令句与命题同样对待,这是很有益处的。在这些语境中听者(或读者)被告知他们\textbf{应当}(should)或\textbf{应该}(ought to)以在命令句中已经说明的方式去行事。那么上面两个论证中的结论可以被解释为:"求得智慧是你应当去做的事情"和"你应该既不借钱给别人也不向别人告贷"。

几乎所有人都会同意,这种断言可以是或真或假的。如果在一个应当去做某事的命令和一个应当做某事的陈述之间有什么区别的话,这种区别恰恰是一个困难的问题,这个问题在这里不必探究。通过忽略这种区别(如果确有区别的话),我们可以对用命令形式和用陈述形式表达结论的论证做相同的处理。

我们的目的是要更彻底地理解论证。这需要借助于澄清论证的构成命题的作用,尽可能减少依赖语境的因素,从而使论证得以完整地重塑。我们要聚焦于命题本身,探索它们是真的还是假的,它们蕴涵着什么,它们是否被别的命题所蕴涵,在某个论证中它们是否被作为前提或结论。我们要抓住命题的实质,而无论它们的语法形式是什么。

\subsection{短语形式的论证}

有些论证的完整重塑仅限于语法方面。论证由命题组成,但是表示命题(因此也能表示前提)的话语有时可能采用短语的形式,而不是陈述句形式。下面讨论地外生命可能性的一段话,可以很好地例示这一点:

\begin{displayquote}
地外有生命吗?至今仍无定论。但是,有大量的行星;有能够无须近地恒星的能量而生存的生物;有丰富的能产生水的广袤无垠的氢和氧的宇宙资源;有行星产生内部热量的各种自然方式;有生命能在海底火山产生并且能足够耐寒地繁殖变体,从而把它们的后代传播到别的世界的可能性;有能够作为星际交流运载工具的坚固的陨石,凡此种种,生命在宇宙的其他地方演化的思想似乎不再像几年前那样让人感到异想天开。${}^{[33]}$
\end{displayquote}

这段话的结论——地球以外有生命的观念至少比以前更能让人接受——得到六个独立前提的支持,每个前提都让人注意到近来发现的事实或可能性,每个前提都表达支持存在地外生命的理由。当这些前提被重新解释为陈述句时,如:(1)宇宙间有大量的其他行星存在;(2)有许多生命能够不依靠近地恒星的能量而生存;等等,这段话中所表达的论证也就变得明显起来。

\subsection{未明确陈述的命题}

当论证中有一个或更多构成命题未明确陈述出来但又假设能为人理解时,论证的分析可能变得更复杂。在2000年4月美国最高法院对著名的米兰达规则进行辩论的会议上,就有这样的例子。(米兰达规则规定:除非被监禁的嫌疑人在审问开始前被告知有权保持沉默并有权请律师,否则法庭不得采信嫌疑人在接受警察审问时所做的认罪供述。)米兰达规则的辩护人论证如下:

\begin{quotation}
如果米兰达规则被推翻,将不再强制性地要求警察预先给予(有权保持沉默等的)告知;如果不强制性地要求警察预先给予告知,他们将不会预先告知。但是因为警察的审问是在公共视域以外进行,仅当总是给予米兰达告知,这些审问的完善性才能得到维护。${}^{[34]}$
\end{quotation}

此处辩护人论证的结论——必须始终给予预先告知,最高法院不应当推翻米兰达规则——不必陈述出来。

在另一个完全不同的语境中,著名小说家安奈斯·林这样描写她的一个小说人物:"梦想家拒绝平凡,杰伊向往平凡。"${}^{[35]}$我们可以推断出作者试图传达的内容——"杰伊不是梦想家"——即使没有陈述出来。

由于论证者假设的论证前提之一是人所共知的或他认为很容易就被人承认的,就可能不陈述出来。在莎士比亚的《裘利斯·恺撒》中,当马克·安东尼正在做关于恺撒的野心的著名演说时,一个市民听众评论恺撒说:

\begin{quotation}
如果他真的有野心,那就证明他的确够不上称王的位置。${}^{[36]}$
\end{quotation}

这是一个论证,但省略了一部分前提,它明显依赖一个合理的但未陈述出来的前提:"不愿接受王冠的人一定没有野心"。日常语言中的三段论论证经常依赖某个未陈述出来的命题。这样的论证叫做\logicterm{省略三段论}。${}^{[37]}$

究竟如何揭示说话人所依赖的(省略)命题,有时并不是很明显的,尽管一旦将其表示出来就很容易被接受。在最近的一部有关美国奴隶制的历史争论以及在那个争论中道德论证所起的作用的著作中,作者写道:

\begin{displayquote}
如果不相信道德论证能产生任何影响,那就是不相信共和政体的政府。${}^{[38]}$
\end{displayquote}

在这个省略式中,未陈述出来的前提是这样一个断定:"相信共和政体的政府要求人们相信道德论证能产生一定的影响"——一个我们多数人都会认可的断言。

此外,省略三段论所依赖的未陈述出来的命题有可能并不显然,而是可质疑的;不把它明确陈述出来,可能正是为了使之避受责难。例如,使用胚胎干细胞进行医学研究广受质疑,一位美国参议员用下面的省略三段论抨击允许政府筹措资金进行这项研究的法案:

\begin{displayquote}
这项研究(包含对胚胎干细胞的使用)是非法的,这是因为:故意杀死人类胚胎是这项研究的基本组成部分。${}^{[39]}$
\end{displayquote}

该论证陈述出来的前提是真的:如果胚胎不被杀死,这项研究是不可能进行的。但是,这项研究是非法的这个结论却依赖其未表述出来的命题:杀死人类胚胎是非法的,而这正是一个处于激烈争论中的论断。

省略三段论极其依赖语境,也经常依赖于听话者关于某个表述出来的命题为假的知识。当论证的目的是强调某个命题的虚假性时,说话人常常构造这样一个假言命题:以该命题作前件("如果"部分),以一个普遍认为为假的命题作后件("那么"部分)。例如,18世纪著名的巴伐利亚风琴制造商之一约瑟夫·瑞普就他的管风琴说过一句广为人知的豪言:"如果在欧洲能发现更好的管风琴,那么我的名字就叫杰克。"因为所有人都会明白,在一个真的假言陈述中,如果后件是假的,前件就不能是真的。对这个假言命题的肯定,实际上就是一个省略式论证,即旨在嘲讽其前件命题:在欧洲能发现更好的管风琴。论证的结论(在欧洲不能发现更好的管风琴)和另一前提(我的名字不叫杰克)都没有表述出来。${}^{[40]}$

\chaptersummary{
辨识论证时需要注意\logicemph{结论和前提指示词},分析\logicemph{语境意义},理解\logicemph{非陈述形式}和\logicemph{未明确陈述的命题},从而准确把握论证的真正结构。掌握这些技能对于有效的逻辑分析至关重要。
}
\section{论证和说明}

\begin{logicbox}[title=引言]
\textit{区分论证和说明是逻辑分析的关键步骤,了解二者的差异能够帮助我们准确判断语段的真正意图。}
\end{logicbox}

在日常交流和学术写作中,许多语段表面上看起来像论证,但实际上是\logicterm{说明}。这种混淆的产生有一个重要原因:\logicwarn{相同的语言指示词}(如"因为"、"由于"、"因此"等)既可以用在论证中,也可以用在说明中。

因此,仅仅依靠语言形式无法准确区分论证和说明。关键在于理解\logicemph{作者的真正意图}:是要证明某个命题的真实性,还是要解释某个已知事实的原因。$^{[41]}$

\subsection{论证与说明的区别}

请比较下面两段话:

\begin{examplebox}[title=论证与说明的对比]
\textbf{例1(论证):}为你自己积攒财宝在天上,天上没有虫子咬,不能锈坏,也没有贼挖窟窿来偷,\textit{因为}你的财宝在哪里,你的心也在哪里。

\textbf{例2(说明):}所以它(那座塔)名叫巴别,\textit{因为}耶和华在那里变乱天下人的言语。\\
——《创世记》11:19
\end{examplebox}

\begin{theorembox}[title=论证与说明的核心区别]
\logicemph{例1分析(论证)}:
\begin{itemize}
  \item \logicterm{目的}:说服读者接受"应该积攒财宝在天上"这一观点
  \item \logicterm{结论}:一个人必须积攒财宝在天上
  \item \logicterm{前提}:一个人的财宝积攒在哪里,他的心也在哪里
  \item \logicterm{逻辑关系}:前提为结论提供支持理由
\end{itemize}

\logicemph{例2分析(说明)}:
\begin{itemize}
  \item \logicterm{目的}:解释一个已知事实的原因
  \item \logicterm{已知事实}:那座塔名叫巴别(读者已知)
  \item \logicterm{解释内容}:耶和华在那里变乱天下人的言语
  \item \logicterm{逻辑关系}:解释为什么会有这个名字
\end{itemize}
\end{theorembox}

\logicwarn{关键区别}:在说明中,"所以它名叫巴别"不是需要证明的结论,而是需要解释的已知事实。"因为耶和华在那里变乱天下人的言语"不是支持命题真实性的前提,而是解释命名原因的说明内容。$^{[42]}$

\subsection{判断标准}

上述对比清楚地表明:\logicemph{表面上相似的语段可能具有完全不同的逻辑功能}。区分论证和说明的关键在于理解语段的\logicterm{根本目的}。

\begin{theorembox}[title=判断标准的形式化表述]
对于形式为"$Q$因为$P$"的语段:

\logicemph{论证情况}:
\begin{itemize}
  \item \logicterm{目的}:确立命题$Q$的真实性
  \item \logicterm{$Q$的地位}:需要证明的结论
  \item \logicterm{$P$的作用}:提供支持$Q$的证据或理由
  \item \logicterm{逻辑关系}:$P$为$Q$提供逻辑支持
\end{itemize}

\logicemph{说明情况}:
\begin{itemize}
  \item \logicterm{目的}:解释命题$Q$为什么为真
  \item \logicterm{$Q$的地位}:已知为真的事实
  \item \logicterm{$P$的作用}:解释$Q$为真的原因
  \item \logicterm{逻辑关系}:$P$解释$Q$的成因
\end{itemize}
\end{theorembox}

\subsection{实例分析}

在回答关于类星体(在我们的星系以外很远地方的一类天体)的外观颜色的问题时,一位科学家写道:

\begin{displayquote}
最远的类星体看上去像强烈的红外辐射光点。这是因为太空散布着吸收蓝光的氢微粒(大约每立方米两个微粒),如果你从可见的白光里过滤掉蓝光,那么剩下的就是红光。在其到达地球的数十亿光年的旅程中,类星体光被大气中的氢微粒吸去了全部的蓝光,留下的只有红光。$^{[43]}$
\end{displayquote}

\begin{examplebox}[title=科学说明实例分析]
\logicemph{分析}:这段关于类星体的文字是典型的\logicterm{说明}:
\begin{itemize}
  \item \logicterm{已知事实}:最远的类星体看上去像强烈的红外辐射光点
  \item \logicterm{说明目的}:解释为什么类星体呈现这种颜色
  \item \logicterm{解释机制}:氢微粒吸收蓝光,留下红光
  \item \logicwarn{不是论证}:不是要证明类星体具有这种外观,而是解释其原因
\end{itemize}
\end{examplebox}

类似地,在历史研究中也经常出现说明:

\begin{displayquote}
塞拉利昂在1808年成为英国直辖殖民地不是因为它的繁荣,而是因为它的萧条。由于战争和商业不景气的负担,塞拉利昂的私营公司不能支付它们的费用,而刚刚废除了贩卖奴隶制度的英国政府感到有必要接管它。$^{[44]}$
\end{displayquote}

\begin{examplebox}[title=历史说明实例分析]
\logicemph{分析}:这段历史叙述也是\logicterm{说明}:
\begin{itemize}
  \item \logicterm{已知事实}:塞拉利昂在1808年成为英国直辖殖民地
  \item \logicterm{说明目的}:解释这一历史事件的原因
  \item \logicterm{解释内容}:经济萧条、公司破产、政府政策变化
  \item \logicwarn{识别标志}:"因为"在此处是说明的标志,不是论证的标志
\end{itemize}
\end{examplebox}

\subsection{区分方法}

要准确区分语段的目的是\logicemph{说明}还是\logicemph{说服},我们需要运用系统的分析方法。

\begin{theorembox}[title=实用区分方法]
对于"$Q$因为$P$"形式的语段,关键问题是:\logicemph{$Q$在语境中的地位是什么?}

\logicterm{判断步骤}:
\begin{enumerate}
  \item 确定$Q$的认知地位:是否已被接受为真?
  \item 分析作者意图:是要证明还是要解释?
  \item 考察语境背景:读者是否已知$Q$为真?
\end{enumerate}

\logicterm{判断标准}:
\begin{itemize}
  \item 若$Q$的真实性\logicwarn{需要建立},则"$Q$因为$P$"是\logicterm{论证}
  \item 若$Q$的真实性\logicwarn{已被接受},则"$Q$因为$P$"是\logicterm{说明}
\end{itemize}
\end{theorembox}

\logicemph{说明的结构分析}:在任何说明中,我们都必须区分两个要素:
\begin{itemize}
  \item \logicterm{被说明的现象}(explanandum):需要解释的已知事实
  \item \logicterm{说明的内容}(explanans):提供解释的原因或机制
\end{itemize}

\begin{examplebox}[title=说明结构的实例]
\begin{itemize}
  \item \logicemph{《创世记》例子}:被说明现象=塔名为巴别;说明内容=语言变乱事件
  \item \logicemph{历史学例子}:被说明现象=塞拉利昂成为殖民地;说明内容=经济和政治因素
\end{itemize}
\end{examplebox}

\subsection{模糊界限}

在实际分析中,论证和说明的界限有时并不清晰。\logicwarn{表面上的说明可能实际上是论证,反之亦然}。让我们通过一个具体例子来说明这种复杂性:

\begin{examplebox}[title=界限模糊的实例]
《纽约时报》因性别不平等报道受到批评:对女演员体重变化加以评论,但对男商人体重变化不予关注。一位读者回应:

\begin{displayquote}
E.R.福克斯的抱怨——你特别提到凯瑟琳·丹尼芙"也许不像她以前那么苗条",但你没有提及唐纳德·杜鲁普不断增加的腰围——很容易说明。杜鲁普先生从未裸体出现在电影中以使他的体形成为人们感兴趣的事情。$^{[45]}$
\end{displayquote}
\end{examplebox}

\begin{theorembox}[title=深层分析:说明还是论证?]
\logicemph{表面判断}:这段话声称要"说明"报道差异的原因

\logicemph{深层分析}:实际上这是一个\logicterm{论证},其结构为:
\begin{itemize}
  \item \logicterm{前提1}:裸体出现在电影中使外表成为公众关注点
  \item \logicterm{前提2}:丹尼芙有过裸体出镜,杜鲁普没有
  \item \logicterm{结论}:报纸的差别对待是合理的,性别歧视指控不成立
\end{itemize}

\logicwarn{关键识别}:作者的真正目的不是解释已知事实,而是为报纸的做法进行辩护,试图说服读者接受其观点。
\end{theorembox}

\logicemph{语境敏感性的重要性}:准确区分说明和论证需要对\logicterm{语境}保持高度敏感。在某些情况下,同一语段可能允许多种合理的解读:
\begin{itemize}
  \item 从一个角度看,可能是论证(试图说服)
  \item 从另一个角度看,可能是说明(试图解释)
\end{itemize}

\logicwarn{实践建议}:当遇到目的不明确的语段时,应该:
\begin{enumerate}
  \item 仔细分析语境背景
  \item 考虑作者的可能意图
  \item 评估命题在语境中的地位
  \item 必要时承认存在多种合理解读
\end{enumerate}

\chaptersummary{
\logicterm{论证}和\logicterm{说明}是两种不同的语言功能:论证旨在\logicemph{证明}某个命题的真实性,而说明旨在\logicemph{解释}已知为真的命题为何如此。

区分二者的关键在于:分析语段的\logicterm{根本目的},确定相关命题的\logicterm{认知地位},以及考虑\logicterm{语境因素}。虽然相同的语言指示词可能出现在两种情况中,但通过系统的分析方法,我们通常能够准确识别语段的真正性质。在某些模糊情况下,保持开放态度并承认多种解读的可能性是明智的做法。
}
\section{演绎和有效性}

\begin{logicbox}[title=引言]
\textit{演绎论证是逻辑学的核心研究对象,了解演绎论证的本质及其有效性标准是进行逻辑分析的基础。}
\end{logicbox}

所有论证的共同特征是断言其前提为结论的真实性提供支持理由。\logicemph{这种断言正是论证区别于其他语言形式的根本标志}。然而,论证并非铁板一块,而是可以分为两大基本类型:\logicterm{演绎论证}和\logicterm{归纳论证}。

这两类论证的根本区别在于\logicemph{前提支持结论的方式和程度}:
\begin{itemize}
  \item \logicterm{演绎论证}:声称前提为结论提供\logicwarn{决定性的、无可辩驳的}支持
  \item \logicterm{归纳论证}:声称前提为结论提供\logicwarn{或然性的、可能的}支持
\end{itemize}

本节我们将深入探讨演绎论证的本质特征及其有效性标准。

\begin{theorembox}[title=演绎论证的识别标准]
\logicterm{演绎论证}的核心特征是断言其前提\logicemph{决定性地}(conclusively)支持结论。这意味着:
\begin{itemize}
  \item 前提与结论之间存在\logicterm{必然的逻辑联系}
  \item 如果前提为真,结论\logicwarn{必须}为真
  \item 不存在前提为真而结论为假的可能性
\end{itemize}

\logicemph{分类原则}:
\begin{itemize}
  \item 如果论证声称前提决定性地支持结论 → \logicterm{演绎论证}
  \item 如果论证不声称前提决定性地支持结论 → \logicterm{归纳论证}
\end{itemize}

由于每个论证要么声称决定性支持,要么不声称,所以\logicwarn{每个论证都可以明确归类为演绎论证或归纳论证}。
\end{theorembox}

\subsection{有效性的概念}

当演绎论证声称其前提为结论提供无可辩驳的支持时,这种声称要么成立,要么不成立:
\begin{itemize}
  \item 如果声称\logicemph{成立}:论证是\logicterm{有效的}(valid)
  \item 如果声称\logicwarn{不成立}:论证是\logicterm{无效的}(invalid)
\end{itemize}

\begin{theorembox}[title=有效性的精确定义]
\logicwarn{重要限制}:\logicterm{有效性}概念仅适用于演绎论证,不适用于归纳论证。

\logicemph{有效性的核心含义}:
一个演绎论证是\logicterm{有效的},当且仅当:
\begin{center}
\textbf{如果其前提是真的,则其结论必定是真的}
\end{center}

\logicemph{等价表述}:
\begin{itemize}
  \item 前提为真而结论为假是\logicwarn{逻辑上不可能的}
  \item 前提与结论之间存在\logicterm{必然的逻辑蕴涵关系}
  \item 论证的\logicterm{逻辑形式}保证了从前提到结论的有效推导
\end{itemize}
\end{theorembox}

\subsection{有效性的二值性}

\logicemph{演绎论证的目标}:每个演绎论证都声称其前提为结论的真实性提供完全的逻辑担保。然而,\logicwarn{并非所有演绎论证都能实现这一目标}。

\begin{theorembox}[title=有效性的二值原理]
对于任何演绎论证,有效性具有\logicemph{二值性}:
\begin{itemize}
  \item 要么是\logicterm{有效的}(实现了其逻辑目标)
  \item 要么是\logicterm{无效的}(未能实现其逻辑目标)
\end{itemize}

\logicwarn{不存在中间状态}:
\begin{itemize}
  \item 没有"部分有效"或"基本有效"的演绎论证
  \item 如果不是有效的,就必然是无效的
  \item 如果不是无效的,就必然是有效的
\end{itemize}
\end{theorembox}

\subsection{演绎逻辑的发展}

\logicemph{演绎逻辑的核心使命}是开发系统的方法来区分有效论证与无效论证。为了实现这一目标,逻辑学家们在历史上发展出了多种分析方法。

\begin{theorembox}[title=逻辑学的两大传统]
\logicterm{古典逻辑}:
\begin{itemize}
  \item \logicemph{起源}:亚里士多德的分析工作
  \item \logicemph{特点}:以自然语言为基础的逻辑分析
  \item \logicemph{内容}:本书第5、6、7章详细阐述
  \item \logicemph{核心}:三段论理论和直言命题逻辑
\end{itemize}

\logicterm{现代符号逻辑}:
\begin{itemize}
  \item \logicemph{起源}:19-20世纪的逻辑革命
  \item \logicemph{特点}:使用人工符号系统进行精确分析
  \item \logicemph{内容}:本书第8、9、10章详细介绍
  \item \logicemph{核心}:命题逻辑和谓词逻辑
\end{itemize}
\end{theorembox}

\logicwarn{共同目标}:尽管古典逻辑和现代符号逻辑在方法和某些具体问题的处理上存在差异,但它们都致力于同一个根本目标:\logicemph{开发能够准确区分有效论证与无效论证的分析工具}。

\chaptersummary{
\logicterm{演绎论证}声称其前提为结论提供\logicemph{决定性支持},这种声称要么成立(论证有效),要么不成立(论证无效)。\logicterm{有效性}是演绎论证的核心概念:有效的演绎论证保证当前提为真时结论必然为真。

有效性具有\logicwarn{二值性}——每个演绎论证要么有效,要么无效,不存在中间状态。\logicemph{演绎逻辑的根本任务}是开发系统的方法来区分有效论证与无效论证,这一任务在古典逻辑和现代符号逻辑两大传统中都得到了深入发展。
}
\section{归纳和或然性}

\begin{logicbox}[title=引言]
\textit{归纳论证是科学研究和日常推理的基础,它与演绎论证有着根本的区别,理解其依赖的或然性原则是进行有效分析的关键。}
\end{logicbox}

\logicterm{归纳论证}与演绎论证的根本区别在于其对前提与结论关系的不同要求。归纳论证\logicwarn{不要求}前提必然地支持结论,而是提出一个相对较弱但极其重要的要求:前提\logicterm{或然性地}支持结论。

\begin{theorembox}[title=归纳论证的基本特征]
\logicemph{核心特征}:
\begin{itemize}
  \item 前提为结论提供\logicterm{或然性支持}而非必然性支持
  \item \logicterm{或然性}本质上是必然性的缺乏
  \item \logicwarn{不适用有效性概念}:归纳论证既不是有效的也不是无效的
\end{itemize}

\logicemph{评估标准}:
\begin{itemize}
  \item 可以评估为"较强"或"较弱"
  \item 可以评估为"较好"或"较差"
  \item 前提授予结论的或然性程度越高,论证价值越大
\end{itemize}

\logicwarn{重要限制}:即使所有前提都为真且提供很强支持,归纳论证的结论也\logicemph{不是必然得出的}。
\end{theorembox}

\logicemph{科学意义}:对归纳论证进行评估是科学研究的核心任务之一。归纳理论、推理技巧、评估方法以及概率量化等内容将在本书第三部分详细介绍。$^{[46]}$

\subsection{归纳与演绎的根本区别}

归纳论证和演绎论证之间存在\logicemph{根本性的区别},这种区别体现在它们对新信息的不同反应上。

\begin{theorembox}[title=归纳论证的开放性特征]
\logicemph{或然性的可变性}:
\begin{itemize}
  \item 归纳论证的前提对结论的支持具有\logicterm{程度性}
  \item \logicterm{附加信息}可能强化或弱化这种或然性支持
  \item 新发现的事实可能改变我们对论证强度的评估
\end{itemize}

\logicemph{证据的开放性}:
\begin{itemize}
  \item 在归纳推理中,\logicwarn{永远不会穷尽所有相关证据}
  \item 总是存在发现新证据的可能性
  \item 新证据可能与已有证据相冲突
\end{itemize}

\logicwarn{确定性的限制}:正是由于这种证据的开放性,我们\logicemph{不能断定任何归纳论证的结论具有绝对的确定性},即使该结论被认为具有很高的可能性。
\end{theorembox}

\begin{theorembox}[title=演绎论证的封闭性特征]
与归纳论证形成鲜明对比,\logicterm{演绎论证}具有\logicemph{封闭性}特征:

\logicemph{二值性质}:
\begin{itemize}
  \item 演绎论证\logicwarn{不能}越来越好或越来越差
  \item 在显示前提与结论关系上要么成功(有效)要么失败(无效)
  \item 不存在程度上的变化
\end{itemize}

\logicemph{有效性的稳定性}:
\begin{itemize}
  \item 如果论证有效,\logicwarn{没有附加前提}可以增强其有效性
  \item 有效性不受外部信息影响
  \item 逻辑关系一旦确立就不会改变
\end{itemize}
\end{theorembox}

\begin{examplebox}[title=演绎论证的稳定性示例]
考虑经典的三段论:
\begin{itemize}
  \item \logicterm{前提1}:凡人皆终有一死
  \item \logicterm{前提2}:苏格拉底是人
  \item \logicterm{结论}:苏格拉底终有一死
\end{itemize}

这个结论\logicemph{必然地}从前提推出,无论我们后来发现什么其他信息(如苏格拉底的外貌、天使的性质、奶牛的习性等),都\logicwarn{不能影响}原论证的有效性。
\end{examplebox}

\logicemph{单调性原理}:对于每个有效的演绎论证,无论添加什么性质的前提,原结论都能从扩大的前提集中必然地推出。\logicwarn{有效性具有绝对性}:没有任何东西能使有效论证"更有效",也没有任何东西能使有效推导"更严格"或"更合乎逻辑"。

\subsection{归纳论证示例及其特征}

归纳论证的情况截然不同。归纳论证所声称的前提与结论之间的关系\logicwarn{远非如此严格},与演绎论证有本质区别。

\begin{examplebox}[title=归纳论证的可变性示例]
考虑以下归纳论证:
\begin{displayquote}
大部分公司法律顾问是保守主义者,\\
安吉拉•帕尔默瑞是一个公司法律顾问,\\
所以安吉拉•帕尔默瑞很可能是保守主义者。
\end{displayquote}

\logicemph{初始评估}:这是一个相当强的归纳论证。如果两个前提都为真,结论很可能为真而非假。
\end{examplebox}

然而,与苏格拉底必死性的演绎论证形成鲜明对比,\logicemph{新信息可能显著改变这个归纳论证的强度}:

\begin{examplebox}[title=弱化论证的新信息]
假设我们发现:
\begin{itemize}
  \item \logicterm{新信息1}:安吉拉•帕尔默瑞是美国公民自由权协会(ACLU)的一名官员
  \item \logicterm{新前提}:美国公民自由权协会的大部分官员不是保守主义者
\end{itemize}

\logicemph{结果}:原结论(安吉拉是保守主义者)不再显得很可能,原论证被\logicwarn{大大弱化}。

\logicemph{极端情况}:如果新前提改为全称命题"没有美国公民自由权协会的官员是保守主义者",那么就能从扩大的前提集\logicterm{演绎地推出}与原结论相反的结论。
\end{examplebox}

\begin{examplebox}[title=强化论证的新信息]
相反,假设我们发现以下信息:
\begin{itemize}
  \item \logicterm{新信息1}:安吉拉•帕尔默瑞长期是国家步枪协会(NRA)的一名官员
  \item \logicterm{新信息2}:安吉拉•帕尔默瑞被任命为保守的《国家评论》报的特约撰稿人
\end{itemize}

\logicemph{结果}:通过这个扩大的前提集,原来的结论得到了\logicemph{比原来更强的支持}。这些新信息都指向同一个方向,增强了安吉拉是保守主义者的可能性。
\end{examplebox}

这个例子清楚地展示了归纳论证的\logicterm{非单调性}特征:新信息可能增强或削弱论证的强度,这与演绎论证的单调性形成鲜明对比。

\subsection{两类论证的本质特征}

\begin{theorembox}[title=两类论证的本质特征对比]
归纳和演绎的根本区别在于它们对前提与结论关系的不同断言:

\logicemph{演绎论证}:
\begin{itemize}
  \item 断言结论从前提\logicterm{绝对必然地}推出
  \item 必然性\logicwarn{不是程度问题}
  \item 不受任何其他情况影响
  \item 具有\logicterm{单调性}:新信息不改变有效性
\end{itemize}

\logicemph{归纳论证}:
\begin{itemize}
  \item 断言结论仅仅\logicterm{或然性地}从前提推出
  \item 或然性\logicwarn{是程度问题}
  \item 受其他情况影响
  \item 具有\logicterm{非单调性}:新信息可能改变强度
\end{itemize}
\end{theorembox}

\logicwarn{识别注意事项}:
\begin{itemize}
  \item 归纳论证\logicwarn{不总是}明确表明其结论仅在某种或然程度上推出
  \item 论证中出现"或然性"一词\logicwarn{不一定}表明该论证是归纳的
  \item 存在关于或然性本身的\logicterm{严格演绎论证}$^{[47]}$(将在第14章讨论)
\end{itemize}

\chaptersummary{
\logicterm{演绎论证}和\logicterm{归纳论证}代表了两种根本不同的推理模式。演绎论证的结论\logicemph{必然地}从前提推出,具有封闭性和单调性,新信息不能影响其有效性。归纳论证的结论仅\logicemph{或然地}从前提推出,具有开放性和非单调性,其强度可能随新证据的增加而增强或减弱。

这种根本区别决定了两类论证采用不同的评估标准:演绎论证评估有效性,归纳论证评估强度。它们在不同领域发挥着不同的作用:演绎论证在数学和逻辑中占主导地位,归纳论证在科学研究和日常推理中不可或缺。
}
\section{有效性和真实性}

\begin{logicbox}[title=引言]
\textit{理解有效性与真实性的区别是逻辑分析的基础,这两个概念分别适用于论证和命题,正确把握它们的关系对评估推理至关重要。}
\end{logicbox}

如前所述,成功的演绎论证具有\logicterm{有效性}。\logicemph{有效性}是一种特殊的逻辑关系——它描述的是演绎论证中前提集与结论之间的必然联系。当结论从前提中\logicwarn{逻辑必然地}推出时,我们说该论证是有效的。

\begin{theorembox}[title=有效性的适用范围]
\logicterm{有效性概念的限制}:
\begin{itemize}
  \item \logicwarn{不适用于归纳论证}:因为归纳论证永远达不到逻辑必然性
  \item \logicwarn{不适用于单个命题}:任何独立的命题内部都不存在前提与结论的关联
  \item \logicemph{仅适用于演绎论证}:描述前提与结论之间的逻辑关系
\end{itemize}
\end{theorembox}

与有效性形成对比,\logicterm{真和假}是单个命题的属性:
\begin{itemize}
  \item 论证中的每个前提都可能是真的或假的
  \item 论证的结论可能是真的或假的
  \item 结论可以被\logicemph{有效地推论出来}
  \item 但说任何单一命题"有效"或"无效"都是\logicwarn{无意义的}
\end{itemize}

\subsection{真实性与命题}

\logicterm{真实性}是命题与现实世界相符合的属性。当命题的断言与实际情况一致时,该命题就是\logicemph{真的};当命题的断言与实际情况不符时,该命题就是\logicwarn{假的}。

\begin{examplebox}[title=真实性的例子]
\begin{itemize}
  \item \logicemph{真命题}:"苏必利尔湖是北美洲五大湖中最大的湖"——与实际情况一致
  \item \logicwarn{假命题}:"密歇根湖是北美洲五大湖中最大的湖"——与实际情况不符
\end{itemize}
\end{examplebox}

\begin{theorembox}[title=真实性与有效性的根本区别]
\logicemph{适用对象不同}:
\begin{itemize}
  \item \logicterm{真实性和虚假性}:单一命题或陈述的属性
  \item \logicterm{有效性和无效性}:论证的属性
\end{itemize}

\logicemph{概念的不可互换性}:
\begin{itemize}
  \item 有效性概念\logicwarn{不适用于}单一命题
  \item 真实性概念\logicwarn{不适用于}论证整体
\end{itemize}
\end{theorembox}

论证中的各个命题可以分别是真的或假的,但\logicwarn{论证作为整体既不"真"也不"假"}。论证只能是有效的或无效的。

\logicemph{核心关系}:真(假)命题与有效(无效)论证之间的关系构成了演绎逻辑的核心内容。本书第二部分将详细分析这些复杂关系,这里我们先进行初步讨论。

\subsection{有效性与真假前提}

\logicwarn{重要原理}:即使论证的一个或多个前提为假,该论证仍可能是有效的。这是理解有效性概念的关键点。

\logicemph{有效性的独立性}:每个论证都声称其前提与结论之间存在特定的逻辑关联。这种逻辑关联的成立与否\logicwarn{独立于前提的实际真假值}。即使前提被证明为假或其真实性受到质疑,逻辑关联本身仍然可以是有效的。

历史上有一个著名的例子完美地说明了这一点。1858年,亚伯拉罕·林肯在与斯蒂芬·道格拉斯的辩论中,巧妙地运用了这一逻辑原理来批评德雷德-司各特决议:

\begin{quotation}
考虑到人们的论辩能力,我把德雷德•司各特决议用三段论形式表述如下,可以看出这个论证中是否有错误:

任何一个州的任何法规和法律都不能破坏美国宪法中所清楚、明确地规定的权利。

美国宪法中清楚、明确地规定了对奴隶的财产权。

所以任何一个州的任何法规和法律都不能破坏对奴隶的财产权。
\end{quotation}

我相信这个论证挑不出什么毛病。假设其前提都是真的,从这些前提必然会推出上述结论,这个结论我完全有能力理解。但我认为其中确有一个毛病,但这毛病不在于推理,事实上这个毛病是有一个前提是错误的。我相信对奴隶的财产权并不是宪法中清楚明确地规定的,而道格拉斯法官认为是的。我相信最高法院和那个决议(德雷德•司各特决议)的拥护者要想在宪法中查找到对奴隶的财产权的清楚明确的规定将会是徒劳的。所以我说,我认为事实上上述推理前提之一不是真的。$^{[48]}$

\begin{examplebox}[title=林肯的逻辑分析]
林肯对这个论证的分析展现了深刻的逻辑洞察:

\logicemph{林肯的发现}:
\begin{itemize}
  \item 第二个前提("美国宪法中清楚、明确地规定了对奴隶的财产权")是\logicwarn{假的}
  \item 论证的\logicterm{推理形式}本身是\logicemph{正确的}
  \item 但由于前提为假,结论\logicwarn{没有得到证明}
\end{itemize}

\logicemph{逻辑原理}:林肯正确地认识到,即使论证的前提和结论都可能为假,论证本身仍可能是有效的。
\end{examplebox}

这个历史例子完美地说明了一个\logicemph{核心逻辑原理}:\logicwarn{论证的有效性仅仅依赖于前提与结论之间的逻辑关联,而不依赖于前提或结论的实际真假值}。

\subsection{真实性和有效性的组合关系}

为了全面理解真实性与有效性的关系,我们需要系统地考察它们的各种可能组合。通过以下七个典型例子,我们可以完整地展现这两个概念之间的复杂关系。

\begin{theorembox}[title=真实性与有效性的组合类型]
在演绎论证中,前提和结论的真假值与论证的有效性可以形成多种组合。以下例子将系统地展示所有可能的情况。
\end{theorembox}

\begin{examplebox}[title=类型I:有效论证 + 真前提 + 真结论]
\begin{quotation}
所有哺乳动物都有肺,所有鲸鱼都是哺乳动物,所以所有鲸鱼都有肺。
\end{quotation}

\logicemph{分析}:这是理想的论证形式——有效的逻辑结构配合真实的前提,必然得出真实的结论。
\end{examplebox}

\begin{examplebox}[title=类型II:有效论证 + 假前提 + 假结论]
\begin{quotation}
所有四条腿的生物都有翅膀,所有蜘蛛都是四条腿的,所以所有蜘蛛都有翅膀。
\end{quotation}

\logicemph{分析}:这个论证是\logicterm{有效的},因为\logicwarn{如果}前提为真,结论也必然为真——尽管我们知道前提和结论实际上都是假的。这说明有效性与命题的实际真假值无关。
\end{examplebox}

\begin{examplebox}[title=类型III:无效论证 + 真前提 + 真结论]
\begin{quotation}
如果我拥有福特•诺克斯的所有财富,那么我将是富有的,我不拥有福特•诺克斯的所有财富,所以我不是富有的。
\end{quotation}

\logicemph{分析}:这个论证是\logicwarn{无效的},尽管前提和结论都是真的。这说明结论的真实性不能保证论证的有效性。
\end{examplebox}

\begin{examplebox}[title=类型IV:无效论证 + 真前提 + 假结论]
\begin{quotation}
如果比尔•盖茨拥有福特•诺克斯的所有财富,那么比尔•盖茨将是富有的,

比尔•盖茨不拥有福特•诺克斯的所有财富,

所以比尔•盖茨不是富有的。
\end{quotation}

\logicemph{分析}:这个论证的前提是真的,但结论是假的。\logicwarn{这样的论证必然是无效的},因为有效论证不可能出现前提真而结论假的情况。
\end{examplebox}

\begin{examplebox}[title=类型V:有效论证 + 假前提 + 真结论]
\begin{quotation}
所有鱼是哺乳动物,

所有鲸是鱼,

所以所有鲸是哺乳动物。
\end{quotation}

\logicemph{分析}:这个论证的结论是真的,而且可以从两个假前提中\logicterm{有效地}推出。这说明有效论证可以从假前提得出真结论。
\end{examplebox}

\begin{examplebox}[title=类型VI:无效论证 + 假前提 + 真结论]
\begin{quotation}
所有哺乳动物都有翅膀,

所有鲸都有翅膀,

所以所有鲸都是哺乳动物。
\end{quotation}

\logicemph{分析}:这个论证也有假前提和真结论,但它是\logicwarn{无效的}。
\end{examplebox}

\logicwarn{重要观察}:从类型V和类型VI的对比可以看出,我们\logicemph{不能}仅从论证有假前提和真结论就推断该论证是有效还是无效的。

\begin{examplebox}[title=类型VII:无效论证 + 假前提 + 假结论]
\begin{quotation}
所有哺乳动物都有翅膀,

所有鲸都有翅膀,

所以所有哺乳动物都是鲸。
\end{quotation}

\logicemph{分析}:这个论证包含的都是假命题,且论证形式无效。
\end{examplebox}

\begin{theorembox}[title=七个例子的重要启示]
通过以上七个例子,我们可以得出几个\logicemph{关键结论}:

\begin{itemize}
  \item 存在结论为假的\logicterm{有效论证}(类型II)
  \item 存在结论为真的\logicwarn{无效论证}(类型III和VI)
  \item \logicwarn{结论的真假不能决定论证的有效性}
  \item \logicwarn{论证有效不能保证结论的真实性}
\end{itemize}
\end{theorembox}

为了更清楚地展示这些关系,我们用两个表格来总结所有可能的组合:

\begin{center}
\textbf{无效论证的所有可能组合}
\begin{tabular}{|c|c|c|}
\hline
\multicolumn{3}{|c|}{\logicwarn{无效的论证}} \\
\hline
 & \logicemph{真结论} & \logicwarn{假结论} \\
\hline
\logicemph{真前提} & 类型III & 类型IV \\
\hline
\logicwarn{假前提} & 类型VI & 类型VII \\
\hline
\end{tabular}
\end{center}

\logicemph{观察}:无效论证可以具有前提与结论的\logicwarn{任何真假组合}。

\begin{center}
\textbf{有效论证的可能组合}
\begin{tabular}{|c|c|c|}
\hline
\multicolumn{3}{|c|}{\logicterm{有效的论证}} \\
\hline
 & \logicemph{真结论} & \logicwarn{假结论} \\
\hline
\logicemph{真前提} & 类型I & \logicwarn{不可能} \\
\hline
\logicwarn{假前提} & 类型V & 类型II \\
\hline
\end{tabular}
\end{center}

\logicemph{关键观察}:有效论证只能有\logicterm{三种}真假组合,\logicwarn{不可能}出现真前提配假结论的情况。

\subsection{有效论证与可靠性}

表格中的空白位置揭示了一个\logicemph{极其重要的逻辑原理}:

\begin{theorembox}[title=有效性的根本保证]
\logicemph{正向原理}:如果论证是有效的且前提都为真,则结论必然为真。

\logicemph{逆向原理}:如果论证是有效的且结论为假,则至少有一个前提必然为假。

这两个原理是\logicterm{逻辑等价的},共同构成了有效性概念的核心。
\end{theorembox}

\subsection{可靠性概念}

基于以上分析,我们可以引入一个重要概念:

\begin{theorembox}[title=可靠性的定义]
\logicterm{可靠论证} = \logicemph{有效性} + \logicemph{真前提}

\logicemph{可靠论证的特征}:
\begin{itemize}
  \item 必定有\logicterm{真结论}
  \item 是\logicwarn{唯一能够确立结论真实性}的论证类型
  \item 提供了从前提到结论的\logicterm{可靠推理路径}
\end{itemize}
\end{theorembox}

\logicwarn{重要区别}:如果演绎论证不是可靠的(即不是有效的,或前提不全为真),那么即使其结论事实上为真,该论证也\logicemph{无法确立}结论的真实性。

\subsection{逻辑学的研究范围}

\begin{theorembox}[title=逻辑学与其他学科的分工]
\logicemph{科学的任务}:检验前提的真实性或虚假性(涉及具体的经验内容)

\logicemph{逻辑学的任务}:分析命题之间的\logicterm{逻辑关系}(涉及推理的形式结构)

\logicterm{逻辑关系}:决定论证形式正确性或不正确性的命题间关系,独立于命题的具体内容。
\end{theorembox}

\subsection{为什么研究假前提论证?}

一个重要问题是:为什么不把研究限制在真前提论证的范围内?答案在于\logicemph{前提真实性未知的论证}具有重要价值:

\begin{examplebox}[title=研究假前提论证的重要性]
\logicemph{科学研究中}:
\begin{itemize}
  \item 我们通过推断可检验结果来检验理论
  \item 但我们\logicwarn{无法预先知道}哪个理论是真的
  \item 必须先分析推理的有效性,再检验前提的真实性
\end{itemize}

\logicemph{日常决策中}:
\begin{itemize}
  \item 我们需要在不同行动方案间做选择
  \item 必须推断每个选择的后果
  \item 如果只关注已知为真的前提,就失去了推理的意义
\end{itemize}
\end{examplebox}

\logicwarn{逻辑悖论}:如果我们只研究前提已知为真的论证,那么当我们知道前提为真时,就不需要推理了——因为推理的目的正是帮助我们确定哪些前提应该被接受为真。

\logicemph{展望}:确定演绎论证有效性的系统方法将在本书第二部分详细介绍。

\chaptersummary{
\logicterm{有效性}和\logicterm{真实性}是两个根本不同的概念:有效性适用于论证,描述前提与结论之间的逻辑关系;真实性适用于命题,描述命题与现实世界的符合程度。

通过七个典型例子的分析,我们发现:结论的真假不能决定论证的有效性,论证的有效性也不能保证结论的真实性。只有\logicemph{可靠论证}(既有效又具有真前提的论证)才能确立结论的真实性。

逻辑学的核心任务是分析推理的形式结构,而不是判断命题的具体真假。这种分工使得逻辑学能够为科学研究和日常推理提供普遍适用的分析工具。
}
\section{复杂的论证性语段}

\begin{logicbox}[title=引言]
\textit{复杂的论证性语段是我们在实践中常见的推理形式,它们涉及多个前提和结论的交织,通过掌握分析方法,我们能更清晰地理解和评估这些论证。}
\end{logicbox}

在现实世界中,论证往往具有相当的复杂性,远超简单的前提-结论模式。这些复杂论证具有以下特征:

\begin{theorembox}[title=复杂论证的结构特征]
\logicemph{多层次结构}:
\begin{itemize}
  \item 由多个子论证组合而成
  \item 形成\logicterm{论证链}或\logicterm{论证网络}
  \item 存在多条推理路径通向最终结论
\end{itemize}

\logicemph{命题的多重角色}:
\begin{itemize}
  \item 某些命题仅作为\logicterm{前提}
  \item 某些命题仅作为\logicterm{结论}
  \item 某些命题既作为前提又作为\logicterm{分结论}(中间结论)
\end{itemize}
\end{theorembox}

\begin{theorembox}[title=图示法的作用与限制]
\logicemph{图示法的优势}:
\begin{itemize}
  \item 直观展现论证的逻辑结构
  \item 清晰显示前提与结论的关系
  \item 便于识别推理的薄弱环节
\end{itemize}

\logicwarn{方法的限制}:
\begin{itemize}
  \item \logicwarn{不存在机械化的图示构建方法}
  \item 同一语段可能有多种合理的解释
  \item 需要分析者的判断和理解能力
\end{itemize}
\end{theorembox}

\subsection{复杂语段分析方法}

分析复杂论证需要系统的方法和清晰的思路。我们的\logicemph{分析策略}包括:

\begin{theorembox}[title=复杂论证分析步骤]
\begin{enumerate}
  \item \logicterm{理解推理流程}:把握作者的整体论证思路
  \item \logicterm{识别命题角色}:确定每个命题在论证中的功能
  \item \logicterm{标记命题编号}:为便于分析,给关键命题编号
  \item \logicterm{构建逻辑图示}:用图形展现前提与结论的关系
  \item \logicterm{评估推理有效性}:判断结论是否真正从前提推出
\end{enumerate}
\end{theorembox}

下面我们通过一个具体例子来演示这种分析方法。这个论证具有典型的复杂结构:\logicemph{最终结论出现在语段开头},有四个前提直接支持这个结论,其中两个前提本身又是分结论,分别得到其他前提的支持:

\begin{displayquote}
(1)看来, 用动物实验进行科学研究的做法并不是不必要的或靠不住的; (2)在使用脊椎动物进行实验之前, 实验的草案必须经过包括一名兽医和一名公众代表在内的公共机构委员会进行的再审查, 并且(3)在研究期间, 动物的医疗和卫生情况得到定期监测。(4)研究者需要健康的动物进行科学研究和医学研究, 因为(5)不健康的动物可能导致错误的研究结果。这激励(6)科学家确保他们使用的任何动物健康并且营养良好。此外, (7)用动物进行研究是昂贵的, 因为(8)料学研究的资金受到限制, (9)只有高质量的研究才能通过有力的竞争获得对研究的支持。$^{[49]}$
\end{displayquote}

\begin{examplebox}[title=动物实验论证的结构分析]
\logicemph{论证结构解析}:
\begin{itemize}
  \item \logicterm{最终结论}:(1) 动物实验是必要且可靠的
  \item \logicterm{直接支持前提}:(2) 审查制度、(3) 监测制度、(6) 科学家的激励、(7) 研究成本高
  \item \logicterm{分结论及其支持}:
    \begin{itemize}
      \item (6) 由 (4) 和 (5) 支持
      \item (7) 由 (8) 和 (9) 支持
    \end{itemize}
\end{itemize}
\end{examplebox}

下面的图示展示了这段话的逻辑结构。\logicemph{图示解读方法}:从图中最高处(逻辑起点)开始,沿着箭头方向追踪推理路径,可以看到多条推理线路如何汇聚到最终结论。

\begin{center}
\includegraphics[width=\textwidth]{images/2025_05_15_6a28331d5e7c993ad07ag-072.jpg}
\end{center}

\subsection{处理重复和强调的命题}

在复杂论证中,我们经常遇到\logicterm{命题重复}和\logicterm{强调}现象,这给分析带来了额外的挑战。

\begin{theorembox}[title=处理重复命题的策略]
\logicemph{重复出现的原因}:
\begin{itemize}
  \item \logicterm{强调重要观点}:作者希望突出关键命题
  \item \logicterm{修辞效果}:增强说服力和表达力
  \item \logicterm{逻辑需要}:同一命题在不同推理环节中发挥作用
\end{itemize}

\logicemph{分析方法}:
\begin{itemize}
  \item 用\logicwarn{相同编号}标记相同命题的不同表述
  \item 识别命题的\logicterm{核心含义},忽略表达方式的差异
  \item 在图示中\logicemph{合并}重复出现的命题
\end{itemize}
\end{theorembox}

下面的例子展示了一个包含大量重复的复杂论证,它由三个清晰的子论证构成:

\begin{displayquote}
(1)宇宙大爆炸理论正在瓦解……(2)根据正统知识, 宇宙起源于大爆炸——200 亿年前的一次巨大的、非常匀称的爆炸。问题是(3)天文学家通过进一步观测证实: 现存的巨大星系团因为体积太大, 完全不可能在仅仅 200 亿年时间中形成……通过人造卫星所收集的新材料的研究, 以及较早前的地面测量表明(4)星系聚集成绵延数十亿光年的巨大带状, 并且(5)星系之间有亿方光年的距离。因为(6)据观测, 星系移动的速度远不及光速, 数学家证明(7)聚集成这么大的物质团必须要经过至少 1000 亿年时间——是假设的大爆炸时间的五倍……(3)像那么大的一种结构现在看来不可能在 200亿年时间中形成……(2)大爆炸理论认为, 物质均匀地散布在宇宙中。而与这种理想理论相反, (3)这么巨大的星丛无法这么快地形成。$^{[50]}$
\end{displayquote}

\begin{examplebox}[title=大爆炸理论论证的结构分析]
\logicemph{重复命题的识别}:
\begin{itemize}
  \item 命题(2):大爆炸理论的基本观点(出现3次)
  \item 命题(3):巨大结构无法快速形成(出现3次)
  \item 其他命题各出现1-2次
\end{itemize}

\logicemph{推理链条分析}:
\begin{enumerate}
  \item \logicterm{观察证据}:(4)(5)(6) → (7) 形成时间计算
  \item \logicterm{科学推论}:(7) → (3) 结构形成困难
  \item \logicterm{理论评判}:(2)(3) → (1) 理论瓦解
\end{enumerate}
\end{examplebox}

下面的图示展示了这段话的逻辑关系:

\begin{center}
\includegraphics[width=\textwidth]{images/2025_05_15_6a28331d5e7c993ad07ag-073.jpg}
\end{center}

\subsection{处理浓缩的前提}

在复杂论证分析中,我们还需要处理\logicterm{浓缩前提}的问题。

\begin{theorembox}[title=浓缩前提的特征与处理]
\logicemph{浓缩形式的表现}:
\begin{itemize}
  \item 用\logicterm{名词短语}代替完整命题
  \item 省略主语、谓语或其他成分
  \item 依赖语境理解完整含义
\end{itemize}

\logicemph{处理策略}:
\begin{itemize}
  \item \logicterm{重构完整命题}:将短语扩展为完整的陈述句
  \item \logicterm{明确逻辑关系}:确定浓缩前提在推理中的作用
  \item \logicterm{保持原意}:扩展时不改变原始含义
\end{itemize}
\end{theorembox}

下面的例子展示了浓缩前提与重复命题同时出现的复杂情况。例如,短语"在大气中的散射"作为前提(4),需要重构为"太阳的能量在大气中散射":

\begin{displayquote}
(1)太阳能汽车只是一种试验性的装置, 其他什么都不是。(2)太阳的能量太弱以至于不能发动甚至是日常使用的迷你汽车。(3)进入大气层的太阳能量大约为每平方码 1 千瓦。因为(4)在大气中的散射, 又因为(5)地球上的任何地方一天中平均只有半天时间受到阳光的照射, (6)每天接收的太阳功率平均为 $1 / 6$ 千瓦时到 4千瓦时 $\cdots \cdots$ 对通常规格的汽车的检测表明, (7)若使一辆电车勉强能够工作, 其电池组需要 300 千瓦时的能量。因此, (8)充满汽车电池必须有 40 平方码的原电池, 大约是一辆拖拉机的拖车顶部的尺寸。(1)除了用于昂贵的试验汽车外, 太阳能没有指望成为任何汽车的动力, 太阳能汽车不是一项待开发的技术。这就是结论。$^{[51]}$
\end{displayquote}

\begin{examplebox}[title=太阳能汽车论证的结构分析]
\logicemph{论证特点}:
\begin{itemize}
  \item \logicterm{结论重复}:命题(1)在开头和结尾都出现,但表述略有不同
  \item \logicterm{浓缩前提}:命题(4)"在大气中的散射"需要扩展理解
  \item \logicterm{数据支撑}:大量具体数据支持能量计算
\end{itemize}

\logicemph{推理结构}:
\begin{itemize}
  \item 从太阳能量密度(3)出发
  \item 考虑损失因素(4)(5)得出实际功率(6)
  \item 结合汽车需求(7)计算所需面积(8)
  \item 最终得出实用性结论(1)
\end{itemize}
\end{examplebox}

这段话的图示为:

\begin{center}
\includegraphics[width=\textwidth]{images/2025_05_15_6a28331d5e7c993ad07ag-074.jpg}
\end{center}

\subsection{融贯的复杂论证分析}

在分析复杂论证时,我们会发现一些语段尽管包含众多前提和分结论,但展现出\logicemph{高度的逻辑融贯性}。这种融贯性体现在:
\begin{itemize}
  \item 每个命题都有明确的逻辑作用
  \item 推理路径清晰且相互支撑
  \item 整体结构服务于统一的论证目标
\end{itemize}

下面是一个典型的融贯复杂论证,来自一位女编辑为其争议性编辑方针所做的辩护:

\begin{displayquote}
本刊(《新英格兰医学杂志》)的主张是(1)不发表不道德的研究报告,忽略它们的料学价值……

我们的主张有三个理由。首先,(2)如果普遍坚持这个主张,只发表合乎道德的研究文章,将会阻止不合乎道德的研究工作的开展。(3)文章的发表是医学研究奖赏制度的一个重要部分。(4)如果研究者知道他们不合乎道德的研究成果不能发表,他们就不会去做不道德的研究。(5)而相反的做法将有助于导致更多的不道德研究工作的开展,因为,如我已表明的,(6)这样的研究可能比较容易开展,因而(7)可能使从事不道德研究工作的人处于有利的竞争地位。其次,(8)即使发表不道德的研究成果不妨碍发表合乎道德的研究成果,为了坚持把合乎道德放在研究第一位的原则,也应该拒绝不道德的研究。(9)如果允许有所松动,我们将逐渐变得习惯于发表不道德的研究成果,并且(10)这将导致对发表合乎道德的研究成果的极大妨碍。最后,(11)对不道德研究成果的拒绝,有利于使社会普遍注意到,甚至某些科学家也不懂得科学研究应是文明的基本尺度。(12)知识尽管很重要,但对一个公平公正的社会来说知识或许远不及得到知识的方法重要。$^{[52]}$
\end{displayquote}

\begin{examplebox}[title=医学期刊编辑方针论证分析]
\logicemph{论证结构特点}:
\begin{itemize}
  \item \logicterm{三重支撑}:(2)(8)(11)三个主要命题直接支持结论(1)
  \item \logicterm{层次清晰}:每个主要前提都有相应的支持前提
  \item \logicterm{逻辑完整}:从预防效果、原则坚持、社会影响三个角度全面论证
\end{itemize}

\logicemph{论证策略}:
\begin{enumerate}
  \item \logicterm{预防论证}:发表政策影响研究行为
  \item \logicterm{原则论证}:道德标准的重要性
  \item \logicterm{社会论证}:科学研究的文明责任
\end{enumerate}
\end{examplebox}

这个复杂但推理缜密的语段展现了高质量论证的特征:每个命题都有明确的逻辑作用,共同服务于统一的论证目标。下面的图示展示了其逻辑结构:

\begin{center}
\includegraphics[width=\textwidth]{images/2025_05_15_6a28331d5e7c993ad07ag-076.jpg}
\end{center}

\subsection{逻辑分析的价值}

然而,\logicwarn{日常生活中的论证往往达不到如此高的水准}。它们可能存在以下问题:
\begin{itemize}
  \item 包含作用不明确的陈述
  \item 陈述之间的逻辑连接混乱或错误
  \item 论证者本身思路不清晰
  \item 缺乏系统的推理结构
\end{itemize}

\begin{theorembox}[title=图示法逻辑分析的功能]
\logicemph{诊断功能}:
\begin{itemize}
  \item \logicterm{暴露结构缺陷}:识别推理中的薄弱环节
  \item \logicterm{澄清逻辑关系}:明确前提与结论的真实联系
  \item \logicterm{发现遗漏}:找出论证中缺失的关键环节
\end{itemize}

\logicemph{评估功能}:
\begin{itemize}
  \item \logicterm{判断有效性}:评估推理的逻辑正确性
  \item \logicterm{识别优缺点}:全面分析论证的强弱之处
  \item \logicterm{指导改进}:为论证优化提供方向
\end{itemize}
\end{theorembox}

\logicemph{逻辑学的实践价值}:对实际论证的评估是逻辑学的重要应用领域。成功的评估需要对所分析论证的结构有清楚的把握,而图示法正是实现这一目标的有效工具。

\chaptersummary{
复杂论证性语段在现实中广泛存在,它们具有多层次结构和命题的多重角色特征。\logicterm{图示法}是分析这类语段的重要工具,能够直观展现论证的逻辑结构。

在分析过程中,我们需要处理\logicterm{重复命题}、\logicterm{浓缩前提}等复杂情况,通过系统的分析步骤来理解推理流程。高质量的复杂论证展现出逻辑融贯性,每个命题都有明确作用,共同服务于统一目标。

图示法支持的逻辑分析不仅能够揭示优秀论证的结构特征,更能暴露日常论证中的缺陷,为论证评估和改进提供科学依据。这种分析能力是逻辑学在实际应用中的重要价值体现。
}
\section{推理}

\begin{logicbox}[title=引言]
\textit{推理是逻辑学的实践应用,通过系统化的思考方法和训练,我们能够解决复杂问题并发展批判性思维能力。}
\end{logicbox}

如前所述,逻辑学是研究用于区分正确推理与不正确推理的方法和原理的学问。\logicterm{推理}与\logicterm{论证}都是从已知前提(或为特定目的而假定的前提)推出结论的思维过程。

到目前为止,我们主要专注于分析和评估他人的论证。然而,在日常生活中,我们每天都需要\logicemph{建构自己的论证}:
\begin{itemize}
  \item 决定自己的行动方案
  \item 评价他人的行为
  \item 为道德或政治信念进行辩护
  \item 解决复杂的实际问题
\end{itemize}

\logicwarn{建构和运用优质论证的技能具有巨大价值},这种技能直接影响我们的决策质量和说服能力。

\begin{theorembox}[title=推理技能的培养]
\logicemph{训练方法}:
\begin{itemize}
  \item \logicterm{推理游戏}:国际象棋、围棋、Mastermind等策略游戏
  \item \logicterm{逻辑谜题}:专门设计的推理问题
  \item \logicterm{实际应用}:日常问题的系统分析
\end{itemize}

\logicemph{双重价值}:
\begin{itemize}
  \item \logicterm{实用价值}:提高问题解决能力
  \item \logicterm{娱乐价值}:享受思维挑战的乐趣
\end{itemize}
\end{theorembox}

正如美国哲学家约翰·杜威所说:"\logicemph{对思虑的享受是受过训练的大脑的标志}"。推理既是必要的生存技能,也是令人愉悦的智力活动。

\begin{theorembox}[title=人工设计问题与现实问题的对比]
\logicemph{人工设计问题的特点}:
\begin{itemize}
  \item \logicterm{结构简洁}:信息明确,条件清晰
  \item \logicterm{逻辑纯粹}:专注于推理技巧本身
  \item \logicterm{答案确定}:存在明确的解决方案
\end{itemize}

\logicemph{解决策略}:
\begin{itemize}
  \item \logicterm{系统推理}:构建推理链条,将中间结论作为后续前提
  \item \logicterm{创造性重组}:对已知信息进行新的组合和理解
  \item \logicterm{持续探索}:面对困难时保持耐心和毅力
\end{itemize}

\logicwarn{思维模式的相似性}:解决逻辑谜题的思考方式与侦探、新闻工作者、陪审员的推理过程本质相同。
\end{theorembox}

虽然人工设计的问题有时会让人无功而返,但\logicemph{通过成功的推理应用解决问题时带来的满足感是无与伦比的}。这种智力挑战不仅锻炼了我们的逻辑思维能力,也为我们提供了纯粹的智力娱乐。

\footnotetext{(1)种著名的网络智力游戏。
}

\subsection{推理问题的类型与解法}

推理问题中最常见的类型是\logicterm{身份识别问题},这类问题要求我们仅凭提供的线索来确定相关人物的身份、角色或其他属性。

\begin{examplebox}[title=航班乘务员职务分配问题]
\begin{displayquote}
在某个航班的全体乘务员中,飞机驾驶员、副驾驶员和飞行工程师的职务由爱伦、布朗和卡尔三人担任,但不必是这个次序。\\
副驾驶员是个独生子,钱挣得最少。\\
卡尔与布朗的姐姐结了婚,钱挣得比驾驶员多。\\
问:三个人每人担任什么职务?
\end{displayquote}
\end{examplebox}

\begin{theorembox}[title=解题策略:排除法推理]
\logicemph{第一步:寻找突破口}
寻找信息最丰富的对象。从前提可知关于卡尔的多个条件:
\begin{itemize}
  \item 卡尔挣得比驾驶员多 → \logicwarn{卡尔不是驾驶员}
  \item 副驾驶员挣得最少,而卡尔挣得比驾驶员多 → \logicwarn{卡尔不是副驾驶员}
  \item 通过排除法 → \logicterm{卡尔是飞行工程师}
\end{itemize}

\logicemph{第二步:利用中间结论}
已知卡尔是飞行工程师,继续分析布朗:
\begin{itemize}
  \item 布朗有姐姐,副驾驶员是独生子 → \logicwarn{布朗不是副驾驶员}
  \item 卡尔已是飞行工程师 → \logicwarn{布朗不是飞行工程师}
  \item 通过排除法 → \logicterm{布朗是驾驶员},\logicterm{爱伦是副驾驶员}
\end{itemize}
\end{theorembox}

\subsection{矩阵分析法}

对于复杂的推理问题,我们需要更系统的分析工具。\logicterm{矩阵分析法}是一种强大的图示技术,它能够:
\begin{itemize}
  \item 系统地记录所有可能的选择
  \item 清晰地显示已知信息和推导结论
  \item 避免信息混乱和遗漏
  \item 支持复杂的多步推理
\end{itemize}

\begin{examplebox}[title=四位艺术家身份识别问题]
\begin{displayquote}
阿伦佐、库特、鲁道夫和威拉德是四个天资极高的创造性的艺术家。一个是舞蹈家,一个是画家,一个是歌唱家,一个是作家,但不必是这个次序。\\
(1)那天晚上歌唱家在音乐会舞台上进行他的首次演出时,阿伦佐和鲁道夫在观众席上。\\
(2)库特和作家两人有画家为他们画的生活肖像。\\
(3)作家正准备写一本阿伦佐的传记,他写的威拉德的传记是畅销书。\\
(4)阿伦佐从未听说过鲁道夫。\\
问:每个人的艺术领域是什么?
\end{displayquote}
\end{examplebox}

\begin{theorembox}[title=矩阵法的优势]
\logicemph{信息管理挑战}:
\begin{itemize}
  \item 需要记住多个前提中的众多事实
  \item 需要跟踪从前提推出的中间结论
  \item 便条记录可能导致混乱和遗漏
\end{itemize}

\logicemph{矩阵法的解决方案}:
\begin{itemize}
  \item \logicterm{系统化存储}:为所有相关信息提供结构化空间
  \item \logicterm{动态更新}:随着推理进展不断填入新信息
  \item \logicterm{全局视野}:同时显示所有可能性和已确定结论
\end{itemize}
\end{theorembox}

这个问题的矩阵表必须是显示这四个人(用四行表示)和他们从事的四种艺术职业(用四列表示)的一个列阵,如下所示:

\begin{center}
\begin{tabular}{|l|l|l|l|l|}
\hline
 & 舞蹈家 & 画家 & 歌唱家 & 作家 \\
\hline
阿伦佐 &  &  &  &  \\
\hline
库 特 &  &  &  &  \\
\hline
鲁道夫 &  &  &  &  \\
\hline
威拉徳 &  &  &  &  \\
\hline
\end{tabular}
\end{center}

\begin{theorembox}[title=矩阵填写规则]
\logicemph{符号约定}:
\begin{itemize}
  \item \logicwarn{N}(或"-"):表示"不可能",某人不可能从事某职业
  \item \logicterm{Y}(或"+"):表示"确定",某人确定从事某职业
\end{itemize}

\logicemph{填写策略}:
\begin{enumerate}
  \item 从前提直接推出的否定信息先填入N
  \item 通过排除法确定的肯定信息填入Y
  \item 一旦确定某人的职业,在该行其他位置填入N,在该列其他位置填入N
\end{enumerate}
\end{theorembox}

\logicemph{第一轮分析}:根据前提直接推导
\begin{itemize}
  \item 前提(1):阿伦佐和鲁道夫在观众席 → \logicwarn{都不是歌唱家}
  \item 前提(2):库特和作家有画家画的肖像 → \logicwarn{库特既非画家也非作家}
  \item 前提(3):作家写阿伦佐和威拉德的传记 → \logicwarn{作家既非阿伦佐也非威拉德}
\end{itemize}

第一轮填写后的矩阵表:

\begin{center}
\begin{tabular}{|l|l|l|l|l|}
\hline
 & 舞蹈家 & 画家 & 歌唱家 & 作家 \\
\hline
阿伦佐 &  &  & N & N \\
\hline
库 特 &  & N &  & N \\
\hline
鲁道夫 &  &  & N &  \\
\hline
威拉德 &  &  &  & N \\
\hline
\end{tabular}
\end{center}

\logicemph{第二轮分析}:运用排除法进行深度推理

\begin{theorembox}[title=关键推理步骤]
\logicemph{步骤1:确定鲁道夫的职业}
\begin{itemize}
  \item 观察作家列:阿伦佐(N)、库特(N)、鲁道夫(?)、威拉德(N)
  \item 通过排除法 → \logicterm{鲁道夫必须是作家}
  \item 填入Y,并在鲁道夫行的其他位置填入N
\end{itemize}

\logicemph{步骤2:排除阿伦佐为画家}
\begin{itemize}
  \item 前提(2):鲁道夫有画家画的肖像
  \item 前提(4):阿伦佐从未听说过鲁道夫
  \item 逻辑推理:如果阿伦佐是画家,他应该认识鲁道夫
  \item 结论 → \logicwarn{阿伦佐不可能是画家}
\end{itemize}

\logicemph{步骤3:连锁推理}
\begin{itemize}
  \item 阿伦佐只剩舞蹈家 → \logicterm{阿伦佐是舞蹈家}
  \item 库特只剩歌唱家 → \logicterm{库特是歌唱家}
  \item 威拉德只剩画家 → \logicterm{威拉德是画家}
\end{itemize}
\end{theorembox}

完成的矩阵表:

\begin{center}
\begin{tabular}{|c|c|c|c|c|}
\hline
 & 舞蹈家 & 画家 & 歌唱家 & 作家 \\
\hline
阿伦佐 & Y & N & N & N \\
\hline
库 特 & N & N & Y & N \\
\hline
鲁道夫 & N & N & N & Y \\
\hline
威拉德 & N & Y & N & N \\
\hline
\end{tabular}
\end{center}

\logicemph{最终答案}:从完成的矩阵表可以直接读出:
\begin{itemize}
  \item \logicterm{阿伦佐}是舞蹈家
  \item \logicterm{库特}是歌唱家
  \item \logicterm{鲁道夫}是作家
  \item \logicterm{威拉德}是画家
\end{itemize}

\begin{theorembox}[title=矩阵法的适用性]
\logicemph{复杂问题的处理}:
\begin{itemize}
  \item 当问题涉及多个维度的分类时,矩阵法变得更加复杂但仍然有效
  \item 某些高难度的逻辑问题\logicwarn{几乎不可能}在不使用矩阵方法的情况下解决
  \item 矩阵法提供了系统化的思维框架,避免遗漏和错误
\end{itemize}
\end{theorembox}

\subsection{称重问题:不同类型的推理挑战}

除了身份识别问题,还有其他类型的推理挑战。下面是一个经典的\logicterm{称重问题},它需要不同的解题策略。

\begin{examplebox}[title=六球称重问题]
\logicemph{问题设定}:
\begin{itemize}
  \item 六个球:两个红球(R1, R2)、两个绿球(G1, G2)、两个蓝球(B1, B2)
  \item 每对同色球中,一个重一个轻
  \item 所有重球重量相同,所有轻球重量相同
  \item 球的外观难以区分
  \item 只有一架天平,最多称重两次
\end{itemize}

\logicemph{挑战}:如何识别出每对球中的重球和轻球?
\end{examplebox}

这类问题需要\logicterm{策略性思维}:我们必须设计称重方案,使得每种可能的结果都能提供足够的信息来解决问题。

\begin{theorembox}[title=称重问题解答策略]
\logicemph{第一次称量方案}:$\mathbf{R1 + G1 \text{ vs } R2 + B1}$

\logicemph{情况一:两边平衡}
\begin{itemize}
  \item \logicterm{逻辑分析}:由于R1和R2分别在两边,且两边平衡,说明每边都是"重球+轻球"
  \item \logicterm{推论}:G1和B1必须一重一轻(否则无法平衡)
  \item \logicterm{第二次称量}:$\mathbf{G1 \text{ vs } B1}$
    \begin{itemize}
      \item 如果G1沉下去:G1重(G2轻),B1轻(B2重),R1轻(R2重)
      \item 如果G1升上去:G1轻(G2重),B1重(B2轻),R1重(R2轻)
    \end{itemize}
\end{itemize}

\logicemph{情况二:$R1+G1$沉下去}
\begin{itemize}
  \item \logicterm{关键推理}:如果R1是轻球,那么R2是重球,$R1+G1$不可能沉下去
  \item \logicterm{结论}:R1必须是重球,R2是轻球
\end{itemize}
\end{theorembox}

已知R1是重球后,G1和B1的组合只有三种可能:

\begin{theorembox}[title=可能的组合分析]
\logicemph{三种可能组合}:
\begin{itemize}
  \item (a) G1轻,B1轻
  \item (b) G1重,B1重
  \item (c) G1重,B1轻
\end{itemize}

\logicemph{第二次称量方案}:$\mathbf{R1 + R2 \text{ vs } G1 + B1}$

\logicemph{结果分析}:
\begin{itemize}
  \item \logicterm{如果$R1+R2$沉下去}:
    \begin{itemize}
      \item 重+轻 > G1+B1,说明G1+B1都是轻球
      \item 对应组合(a):G1轻(G2重),B1轻(B2重)
    \end{itemize}
  \item \logicterm{如果$R1+R2$升上去}:
    \begin{itemize}
      \item 重+轻 < G1+B1,说明G1+B1都是重球
      \item 对应组合(b):G1重(G2轻),B1重(B2轻)
    \end{itemize}
  \item \logicterm{如果两边平衡}:
    \begin{itemize}
      \item 重+轻 = 重+轻,说明G1+B1也是一重一轻
      \item 对应组合(c):G1重(G2轻),B1轻(B2重)
    \end{itemize}
\end{itemize}
\end{theorembox}

\logicwarn{解题关键}:通过巧妙设计的两次称重,我们能够在所有可能情况下都获得完整的信息,这体现了\logicemph{策略性推理}的重要性。

\subsection{回溯分析问题}

在现实世界中,我们经常需要从当前状态推断其起因,从现在的情况推出过去的状况。

\logicterm{回溯分析}是从当前状态推断其历史成因的推理过程。这种推理在科学研究中极为重要:

\begin{theorembox}[title=回溯分析的应用领域]
\logicemph{科学应用}:
\begin{itemize}
  \item \logicterm{考古学家}:从文物推断古代文明
  \item \logicterm{地质学家}:从地层推断地球历史
  \item \logicterm{天文学家}:从观测数据推断宇宙演化
  \item \logicterm{医学家}:从症状推断疾病成因
\end{itemize}

\logicemph{经典案例}:1996年海库塔克彗星(Hyakutake)放射出比预期强100倍的X射线,这一意外发现促使科学家重新思考彗星的物理机制。
\end{theorembox}

\begin{theorembox}[title=回溯分析问题的特殊挑战]
\logicemph{逻辑框架的建立}:
\begin{itemize}
  \item 现实世界的复杂性需要简化为可分析的逻辑结构
  \item 必须明确相关的规则和规律
  \item 需要在问题中规定分析的边界条件
\end{itemize}

\logicemph{娱乐性设计}:
\begin{itemize}
  \item 回溯分析问题常被设计为智力游戏
  \item 提供了"你喜欢拥有的问题"——既有挑战性又有趣味性
  \item 锻炼逆向思维和推理能力
\end{itemize}
\end{theorembox}

\subsection{象棋回溯分析实例}

\logicterm{象棋}为回溯分析提供了理想的逻辑框架,因为棋规明确且严格。

\begin{examplebox}[title=象棋回溯分析问题]
\logicemph{问题设定}:
\begin{itemize}
  \item 给定当前棋盘局面(图1-1)
  \item 假设所有走法都符合象棋规则
  \item 要求推断最近的几步棋
\end{itemize}

\logicemph{坐标系统}:
\begin{itemize}
  \item 行:从下到上标记为1-8
  \item 列:从左到右标记为a-h
  \item 每个方格用字母-数字组合表示(如a8, h2等)
\end{itemize}
\end{examplebox}

\includegraphics[width=\textwidth]{images/2025_05_15_6a28331d5e7c993ad07ag-085.jpg}

图 1-1

\begin{theorembox}[title=回溯推理过程]
\logicemph{第一步:确定黑王的前一步}
\begin{itemize}
  \item \logicterm{约束条件}:两王不能相邻
  \item \logicterm{排除法}:黑王不可能从b7或b8移动到a8
  \item \logicterm{结论}:黑王必须从a7移动到a8,且在a7时被将军
\end{itemize}

\logicemph{第二步:确定白棋的前一步}
\begin{itemize}
  \item \logicterm{问题}:什么白棋走法导致a7的黑王被将军?
  \item \logicterm{排除}:不可能是g1的白象直接将军
  \item \logicterm{推理}:必须是某个白棋子移动后,暴露了象的攻击线
  \item \logicterm{结论}:b6的马移动到a8,被黑王吃掉,同时暴露象的将军
\end{itemize}
\end{theorembox}

\logicemph{最终答案}:
\begin{itemize}
  \item 白棋最后一步:马从b6到a8
  \item 黑棋最后一步:王从a7到a8(吃掉白马)
\end{itemize}

\subsection{逻辑谜题与现实问题的对比}

虽然逻辑谜题为我们提供了优秀的推理训练,但我们必须认识到它们与现实问题之间的重要差异。

\begin{theorembox}[title=逻辑谜题与现实问题的对比]
\logicemph{逻辑谜题的特征}:
\begin{itemize}
  \item \logicterm{信息完整}:提供解决问题所需的全部信息
  \item \logicterm{描述精确}:问题陈述清晰无歧义
  \item \logicterm{答案明确}:存在确定的、可证明的解答
  \item \logicterm{封闭系统}:不需要外部信息或新发现
\end{itemize}

\logicemph{现实问题的特征}:
\begin{itemize}
  \item \logicwarn{信息不完整}:可能缺少关键信息
  \item \logicwarn{描述模糊}:问题陈述可能不精确或有歧义
  \item \logicwarn{答案开放}:可能没有唯一解或需要新发现
  \item \logicwarn{开放系统}:可能需要新的科学发现或技术突破
\end{itemize}
\end{theorembox}

\begin{theorembox}[title=现实问题的复杂性]
\logicemph{问题发现的方式}:
\begin{itemize}
  \item 往往源于矛盾现象或异常事件
  \item 基于对某种"不顺畅"的直觉感受
  \item 不是预先设计好的精确问题
\end{itemize}

\logicemph{解决的挑战}:
\begin{itemize}
  \item 可能需要重新定义问题本身
  \item 可能需要等待新的科学发现
  \item 可能需要创新的方法或工具
\end{itemize}
\end{theorembox}

\logicwarn{重要认识}:尽管存在这些差异,现实问题和逻辑谜题都需要\logicemph{系统的推理}来解决。逻辑谜题为我们提供了推理技能的训练场,而这些技能在面对现实问题时同样不可或缺。

\chaptersummary{
本章系统地介绍了逻辑学的基本概念和分析方法,为后续深入学习奠定了坚实基础。

\logicemph{核心概念}:
\begin{itemize}
  \item \logicterm{逻辑学}:研究区分正确推理与不正确推理的方法和原理
  \item \logicterm{命题}:可以肯定或否定、具有真假值的陈述
  \item \logicterm{论证}:由前提支持结论的命题集合
  \item \logicterm{演绎论证}:声称前提必然支持结论的论证
  \item \logicterm{归纳论证}:声称前提或然性支持结论的论证
\end{itemize}

\logicemph{分析方法}:
\begin{itemize}
  \item \logicterm{解析法}:按逻辑顺序列出论证中的所有命题
  \item \logicterm{图示法}:用数字和连接线展现命题间的逻辑关系
  \item \logicterm{矩阵法}:系统化处理复杂推理问题的工具
\end{itemize}

\logicemph{重要区分}:
\begin{itemize}
  \item \logicterm{论证vs说明}:证明vs解释的不同目的
  \item \logicterm{有效性vs真实性}:逻辑关系vs事实符合
  \item \logicterm{演绎vs归纳}:必然性vs或然性的支持程度
\end{itemize}

\logicemph{实践应用}:通过各种推理问题和逻辑谜题,展示了逻辑学在问题解决中的实际价值,既提供了思维训练,也带来了智力乐趣。这些技能对于日常决策、学术研究和专业工作都具有重要意义。
}

% 参考文献将在主文档末尾统一显示

% others

% 第二部分
\chapter{语言与交流}
% 第二部分导言
\begin{logicbox}[title=第二部分:语言与交流]
\textit{语言是人类思维和交流的重要工具,理解语言的功能和特点对于逻辑分析至关重要。本部分探讨语言的多种功能、情感色彩以及在论证中的作用,帮助读者更好地理解和运用语言进行有效交流。}
\end{logicbox}

\section{语言的三种基本功能}

\begin{logicbox}[title=引言]
\textit{语言作为人类交流的工具有着多种用法,理解语言的三种基本功能对于正确分析和解读语言表达至关重要。}
\end{logicbox}

语言是一种极其精细和复杂的工具,其\logicterm{用法}(uses)的多样性常常被我们忽视。在日常交流中,我们倾向于简化理解,\logicwarn{不注意语言使用的具体语境及其不同目的},从而可能被表面的语词或话语形式所误导。

\begin{examplebox}[title=语言功能的复杂性实例]
\logicemph{表面与实际的差异}:
\begin{itemize}
  \item \logicterm{表面形式}:"你好吗?"(疑问句形式)
  \item \logicterm{实际功能}:友好问候,而非健康状况询问
  \item \logicwarn{误解后果}:详细描述健康状况的人可能被认为不懂社交礼仪
\end{itemize}
\end{examplebox}

这个例子说明,\logicemph{语词并不总是服务于它们的表面诉求}。请求、报道、问候等只是语言众多功能中比较明显的几种。

\subsection{语言功能的多样性}

\begin{theorembox}[title=贝克莱的语言功能观]
哲学家\logicterm{乔治·贝克莱}(George Berkeley)在《人类知识原理》(1710)中提出了重要观点:

\begin{displayquote}
……思想交流……并非像通常想象的那样是语言的首要的和唯一的目的。使用语言还有许多其他宗旨,诸如引起某些情感、鼓动或者抑制行动、使人专心于某些特定的安排等等。前者(交流思想)在很多情况下都基本上是从属性的;当没有前者也能实现那些宗旨时,前者甚或完全被忽略,我认为,这种情况在人们熟悉的语言用法中经常发生。
\end{displayquote}

\logicemph{核心洞察}:语言的功能远超信息传递,包括情感激发、行为引导等多重目的。
\end{theorembox}

\begin{theorembox}[title=维特根斯坦的语言游戏理论]
20世纪哲学家\logicterm{路德维希·维特根斯坦}在《哲学研究》(1953)中进一步发展了这一思想:

\logicemph{核心观点}:"我们称之为'符号'、'语词'和'语句'的东西有无数不同种类的用法。"

\logicemph{语言用法的丰富性}:
\begin{itemize}
  \item \logicterm{认知功能}:描述、报道、推测、假说检验、实验结果展示
  \item \logicterm{行为功能}:命令、询问、翻译、解题
  \item \logicterm{社交功能}:问候、开玩笑、诅咒
  \item \logicterm{文化功能}:编故事、演戏、唱歌、祈祷
  \item \logicterm{娱乐功能}:猜谜、游戏
\end{itemize}
\end{theorembox}

\begin{theorembox}[title=语言功能的三重划分]
面对语言用法的惊人多样性,学者们提出了一个实用的分类框架:

\logicemph{三种基本功能}:
\begin{itemize}
  \item \logicterm{信息性用法}(the informative):传递信息,描述世界
  \item \logicterm{表达性用法}(the expressive):表达情感,激发共鸣
  \item \logicterm{指令性用法}(the directive):引导行为,产生行动
\end{itemize}

\logicwarn{重要说明}:这种三重划分是一种简化,甚至可能过于简化,但许多逻辑学家和语言学家发现它是非常有用的分析工具。
\end{theorembox}

\subsection{信息性用法}

\begin{theorembox}[title=信息性功能的定义与特征]
\logicemph{核心定义}:语言的\logicterm{信息性功能}是指通过明确表述并肯定(或否定)命题来进行信息交流的用法。

\logicemph{主要特征}:
\begin{itemize}
  \item \logicterm{命题性质}:能够被肯定或否定
  \item \logicterm{论证功能}:可以为命题提供论证支持
  \item \logicterm{描述功能}:用来描述世界和进行推理
  \item \logicwarn{包容性}:既包括真信息也包括假信息,既包括正确论证也包括错误论证
\end{itemize}

\logicemph{适用范围}:无论所报道的事实是重要还是琐碎、是普遍还是特殊,用来描述和报道的语言都属于信息性用法。
\end{theorembox}

下面是语言信息性用法的典型例子,来自佛罗里达高等法院的报道:

\begin{displayquote}
2000年11月7日,星期二,佛罗里达州,和美国其他州一道,进行美国总统普选。11月8日,星期三,该选举分区(佛罗里达州)报道说,共和党候选人乔治•布什获得 2909135 张选票,民主党候选人小艾伯特•戈尔获得 2907351 张选票。因为投给他们的全部票数的总差(1784张),低于该选区全部投票票数的百分之一的一半,所以根据佛罗里达州法律规定进行了自动重新计票。\cite{palmbeach2000}
\end{displayquote}

\subsection{表达性用法}

\begin{theorembox}[title=表达性功能的核心特征]
正如信息性话语的典型例子来自法院或实验室报道,\logicterm{表达性用法}的最佳例子来自抒情诗歌。

\logicemph{核心目的}:表达和激发情感、感受或态度,而非传递事实信息。
\end{theorembox}

\begin{examplebox}[title=诗歌中的表达性用法]
约翰·W·伯根(John W. Burgon)面对古城帕特拉(Petra)遗迹时写下的诗句:

\begin{displayquote}
如此奇迹今我惊叹,它保留在东部的风情中——玫瑰一样红的城市——"几乎和时间一样永恒"!
\end{displayquote}

\logicemph{分析}:
\begin{itemize}
  \item \logicwarn{非信息性}:并非意图告诉我们关于世界的事实和理论
  \item \logicterm{情感表达}:表达诗人的赞赏和敬畏之情
  \item \logicterm{情感激发}:目的在于在读者心中激起类似情感
\end{itemize}
\end{examplebox}

\begin{theorembox}[title="表达"概念的界定]
\logicwarn{术语澄清}:为避免混淆,本书对"表达"一词的使用范围比日常用法更为狭窄:

\logicemph{本书用法}:
\begin{itemize}
  \item \logicterm{表达}:专指揭示或交流情感、感受和态度
  \item \logicterm{陈述/表明}:用于见解、信念或信仰的表述
\end{itemize}

\logicemph{区分意义}:这种区分有助于清晰地分离语言的信息性功能和表达性功能。
\end{theorembox}

\begin{examplebox}[title=表达性语言的多样形式]
表达性语言远不限于诗歌,在日常生活中随处可见:

\logicemph{情感表达的常见形式}:
\begin{itemize}
  \item \logicterm{悲伤表达}:"真糟糕"、"真遗憾"
  \item \logicterm{兴奋表达}:"好极了"、"太妙了"
  \item \logicterm{爱情表达}:恋人间的喃喃私语
  \item \logicterm{宗教表达}:主祷文、大卫王赞美诗第23篇
\end{itemize}

\logicemph{共同特征}:这些用法都不在于交流信息,而在于表达情感、感受或态度。
\end{examplebox}

\begin{theorembox}[title=表达性话语的真假问题]
\logicwarn{重要原理}:表达性话语既不真也不假。

\logicemph{错误评价的危害}:
\begin{itemize}
  \item 用真假标准衡量表达性话语是\logicwarn{文不对题}的
  \item 这种做法会使表达性话语的价值\logicwarn{丧失殆尽}
\end{itemize}

\logicemph{经典例子}:因为知道是巴尔波(Balboa)而非柯尔特斯(Cortés)发现太平洋,就降低对济慈《初读查普曼译荷马》的欣赏,这是\logicwarn{蹩脚读者}的表现。该诗的宗旨不是教授历史知识。
\end{theorembox}

\begin{theorembox}[title=混合用途的概念]
\logicemph{复杂情况}:某些语言表达具有多重功能:
\begin{itemize}
  \item 有些诗歌含有重要的信息性内容
  \item 伟大诗人的作品常常是优秀的"生活评判"
  \item 这类作品具有\logicterm{混合用途}(mixed usage)
\end{itemize}

\logicwarn{概念预告}:混合用途的详细讨论将在后续章节展开。
\end{theorembox}

\begin{theorembox}[title=表达的两种基本情形]
表达性语言可以根据其交流对象分为两种类型:

\logicemph{自我表达型}:
\begin{itemize}
  \item \logicterm{特征}:独自发泄,不以他人为对象
  \item \logicterm{例子}:私人写诗、孤独祷告
  \item \logicterm{功能}:表达说话者或写作者的情感
  \item \logicwarn{限制}:不是为了在别人心中引起共鸣
\end{itemize}

\logicemph{感染他人型}:
\begin{itemize}
  \item \logicterm{特征}:寻求感染他人,以他人为对象
  \item \logicterm{例子}:演讲、求爱诗歌、运动队欢呼
  \item \logicterm{功能}:既表达说话者情感,又意图在听者心中引发共鸣
\end{itemize}

\logicemph{综合性质}:表达性话语可以同时具有这两方面的功用。
\end{theorembox}

\subsection{指令性用法}

\begin{theorembox}[title=指令性功能的定义与特征]
\logicemph{核心定义}:当语言意图引起或阻止明显的行动时,它就具有\logicterm{指令性功能}。

\logicemph{主要特征}:
\begin{itemize}
  \item \logicterm{行为导向}:目标是产生或阻止具体行动
  \item \logicwarn{非信息性}:不是为了交流信息
  \item \logicwarn{非表达性}:不是为了表达或激发情感
  \item \logicterm{结果导向}:为了获得指令结果
\end{itemize}
\end{theorembox}

\begin{examplebox}[title=指令性用法的典型例子]
\logicemph{命令与请求}:
\begin{itemize}
  \item \logicterm{家庭场景}:父母告诉孩子"洗手吃饭"
  \item \logicterm{商业场景}:对售票员说"请给两张票"
  \item \logicterm{微妙差别}:命令和请求可通过语调或"请"字转换
\end{itemize}

\logicemph{问题的指令性}:当提出问题以寻求回答时,该问题通常也属于指令性话语。
\end{examplebox}

\begin{theorembox}[title=指令性话语的真假问题]
\logicwarn{重要原理}:在单纯的祈使形式中,指令性话语既不真也不假。

\logicemph{真假不适用性}:
\begin{itemize}
  \item \logicterm{例子}:"关上窗户"这样的命令既不能是真的也不能是假的
  \item \logicterm{争议焦点}:人们可能对是否应当遵从命令有不同意见
  \item \logicwarn{共识}:但对命令是否有真假不会有分歧
\end{itemize}

\logicemph{替代评价标准}:
\begin{itemize}
  \item \logicterm{合理性}:合理或不合理
  \item \logicterm{适当性}:适当或不适当
  \item \logicemph{类比关系}:这些属性与信息性话语的真假有相似之处
\end{itemize}
\end{theorembox}

\begin{examplebox}[title=指令性话语中的论证]
当指令伴随理由时,整个过程可以视为论证:

\begin{displayquote}
小心驾驶!谨记墓地中满是守法的公民,他们有走路的权利。\cite{lander1988}
\end{displayquote}

\logicemph{论证分析}:
\begin{itemize}
  \item \logicterm{命令}:小心驾驶
  \item \logicterm{理由}:墓地中满是守法的公民,他们有走路的权利
  \item \logicterm{逻辑处理}:可以将命令视为命题"你应当小心驾驶"
\end{itemize}
\end{examplebox}

\begin{theorembox}[title=祈使逻辑]
\logicemph{学术发展}:一些学者专门研究指令性语言的逻辑问题,发展出了\logicterm{祈使逻辑}(logic of imperatives)。

\logicwarn{范围限制}:祈使逻辑的详细讨论超出了本书的范围。\cite{rescher1996}
\end{theorembox}

\chaptersummary{
语言作为人类交流的复杂工具,具有远超表面形式的多样功能。通过三重划分,我们可以系统地理解语言的基本用法:

\logicemph{三种基本功能}:
\begin{itemize}
  \item \logicterm{信息性用法}:通过肯定或否定命题来交流信息,描述和解释世界,包括真假信息和正误论证
  \item \logicterm{表达性用法}:表达和激发情感、感受或态度,既不真也不假,评价标准不是真假而是情感共鸣
  \item \logicterm{指令性用法}:引起或阻止行动,如命令、请求和问题,评价标准是合理性和适当性
\end{itemize}

\logicwarn{重要认识}:
\begin{itemize}
  \item 语言的表面形式可能掩盖其真实功能
  \item 不同功能需要不同的评价标准
  \item 某些语言表达具有混合用途,同时承担多种功能
  \item 理解语言功能对于准确分析和解读语言表达至关重要
\end{itemize}
}
\section{多功能话语}

\begin{logicbox}[title=引言]
\textit{在实际应用中,语言常常不仅仅具有单一功能,而是多种功能的混合体,理解这种多功能性对于准确分析语言的意图和效果至关重要。}
\end{logicbox}

\begin{theorembox}[title=理论与现实的差距]
前面所给出的\logicterm{信息性}、\logicterm{表达性}和\logicterm{指令性}话语的例子,如同\logicemph{纯正的化学标本}——在实验室中是理想的,但在现实中很少以纯粹形式存在。

\logicwarn{重要认识}:
\begin{itemize}
  \item 三重划分具有启发性和价值,但\logicwarn{不能机械运用}
  \item 几乎任何正常交流都可能表现出语言的多种用法
  \item 现实中的语言使用通常是\logicterm{混合性的}
\end{itemize}
\end{theorembox}

\begin{examplebox}[title=诗歌的多功能性]
华兹华斯(Wordsworth)的诗句展示了表达性话语的复杂性:

\begin{displayquote}
我们身边的世界丰富精彩:迟早,无论是得到和失去,我们都要浪费掉自己的精力:
\end{displayquote}

\logicemph{功能分析}:
\begin{itemize}
  \item \logicterm{主要功能}:表达性(抒发情感和感受)
  \item \logicterm{次要功能}:信息性(包含关于世界的观察)
  \item \logicterm{潜在功能}:指令性(可能引导读者反思生活方式)
\end{itemize}
\end{examplebox}

\subsection{语言功能的融合}

\begin{examplebox}[title=不同文体的功能融合]
\logicemph{宗教布道}:
\begin{itemize}
  \item \logicterm{主要功能}:指令性(希冀带来行动:抛弃罪恶,乐善好施)
  \item \logicterm{表达性功能}:表达并激发宗教情感
  \item \logicterm{信息性功能}:传递福音的好消息
\end{itemize}

\logicemph{科学论文}:
\begin{itemize}
  \item \logicterm{主要功能}:信息性(传递科学知识和发现)
  \item \logicterm{表达性功能}:表达作者的理性激情
  \item \logicterm{指令性功能}:含蓄地请读者独立验证结论
\end{itemize}

\logicwarn{普遍规律}:语言的大多数日常用法都是\logicterm{混合的}。
\end{examplebox}

\begin{theorembox}[title=多功能用法的必要性]
语言的\logicterm{多功能用法}并非说话者的混淆,而是\logicemph{成功交流的必要条件}。

\logicemph{权威关系的局限性}:
\begin{itemize}
  \item 只有在明确的权威关系中(父母-子女、雇主-雇员),才能直接发布命令
  \item \logicwarn{赤裸裸的命令}往往引起反感和敌对
  \item 单纯的命令经常\logicwarn{自生自灭}
\end{itemize}

\logicemph{间接策略的重要性}:
\begin{itemize}
  \item 必须使用间接方式来实现目标
  \item 需要采用委婉的方法而非直截了当的命令
  \item 功能的巧妙结合是有效交流的关键
\end{itemize}
\end{theorembox}

\subsection{行动的复杂动机}

\begin{theorembox}[title=行动的心理学基础]
\logicemph{研究领域的分工}:虽然心理学家比逻辑学家更适合研究动机,但理解行动的基本机制对语言分析至关重要。

\logicemph{行动的双重条件}:
\begin{itemize}
  \item \logicterm{欲望}(desire):行动者的意愿和动机
  \item \logicterm{信念}(belief):行动者对现实的认知
\end{itemize}

\logicwarn{缺一不可}:两个条件必须同时满足才能产生行动。
\end{theorembox}

\begin{examplebox}[title=欲望与信念的相互作用]
\logicemph{饮食行为的分析}:
\begin{itemize}
  \item \logicterm{情况1}:饥饿的人(有欲望)+ 不相信面前是食物(缺乏信念)= 不会进食
  \item \logicterm{情况2}:不想吃的人(缺乏欲望)+ 确信面前是食物(有信念)= 不会进食
  \item \logicterm{情况3}:饥饿的人(有欲望)+ 相信面前是食物(有信念)= 会进食
\end{itemize}
\end{examplebox}

\begin{theorembox}[title=语言功能与行动动机的对应]
\logicemph{影响行动的两种策略}:
\begin{itemize}
  \item \logicterm{态度策略}:通过激起适当态度来引起行动(对应表达性功能)
  \item \logicterm{信息策略}:通过提供信息来影响信念(对应信息性功能)
\end{itemize}

\logicemph{概念联系}:
\begin{itemize}
  \item \logicterm{欲望}是"态度"或"情感"的特殊类型
  \item \logicterm{信念}通常受所接收信息的影响
\end{itemize}
\end{theorembox}

\begin{examplebox}[title=慈善捐助的说服策略]
假设目标是促使听众向特定慈善组织捐助,根据听众的不同状况需要采用不同策略:

\logicemph{策略一:信息导向}
\begin{itemize}
  \item \logicterm{前提条件}:听众已有助人为乐的态度
  \item \logicterm{策略}:提供慈善机构良好工作的信息
  \item \logicterm{功能分析}:表面上是信息性的,实际目的是指令性的
  \item \logicterm{优势}:避免了毫无掩饰的命令或不客气的要求
\end{itemize}

\logicemph{策略二:情感导向}
\begin{itemize}
  \item \logicterm{前提条件}:听众已相信慈善机构信誉良好
  \item \logicterm{策略}:激发乐善好施的情感
  \item \logicterm{功能分析}:实现"动人诉求"(moving appeal)
  \item \logicterm{结果}:语言具有表达性和指令性的混合用法
\end{itemize}

\logicemph{策略三:综合导向}
\begin{itemize}
  \item \logicterm{前提条件}:听众既缺乏慈善态度,也不了解机构信誉
  \item \logicterm{策略}:同时使用三种语言功能
  \item \logicterm{必要性}:这不是偶然为之,而是成功交流的\logicwarn{基本手段}
\end{itemize}
\end{examplebox}

\subsection{语言的特殊用法}

\begin{theorembox}[title=三重划分的局限性]
虽然\logicterm{信息性}、\logicterm{表达性}和\logicterm{指令性}用法构成了语言的三种基本功能,但语言在某些特殊语境中还具有一些\logicwarn{不能完全归属于三重划分}的特殊用法。

\logicemph{理论的开放性}:语言的复杂性超越了任何单一的分类系统。
\end{theorembox}

\begin{theorembox}[title=礼仪性用法]
\logicterm{礼仪性用法}是一种普遍存在的特殊语言功能:

\logicemph{基本特征}:
\begin{itemize}
  \item 通常是表达性和指令性话语的混合
  \item 主要服务于使人际互动变得融洽
  \item 具有社会润滑剂的作用
\end{itemize}

\logicemph{常见例子}:
\begin{itemize}
  \item \logicterm{社交场合}:问候、赴宴邀请、雇用告知
  \item \logicterm{宗教场合}:庄重的仪式语言
\end{itemize}
\end{theorembox}

\begin{examplebox}[title=结婚仪式的语言分析]
结婚仪式的语言展现了礼仪性用法的复杂性:

\logicemph{多重功能}:
\begin{itemize}
  \item \logicterm{表达性功能}:突出场合的庄重性,营造神圣氛围
  \item \logicterm{指令性功能}:使新郎新娘理解严肃的结婚誓言,引起新的角色行为
  \item \logicterm{礼仪性功能}:标志人生重要转折,确认社会关系变化
\end{itemize}
\end{examplebox}

\begin{theorembox}[title=践行性用法]
与礼仪性用法相近,语言还有另一种特殊用法:\logicterm{践行性}(performative)用法。

\logicemph{核心特征}:在某些语境中,\logicwarn{说出某些话实际上就是实施一种行动}。

\logicemph{与其他功能的区别}:
\begin{itemize}
  \item 不仅仅是表明态度或预告行为
  \item 不仅仅是描述说话者在做什么
  \item 语言本身就构成了行动
\end{itemize}
\end{theorembox}

\begin{examplebox}[title=践行性用法的典型例子]
\logicemph{承诺行为}:
\begin{itemize}
  \item \logicterm{情境}:有人邀请你参加会议
  \item \logicterm{回应}:"好,我答应你"
  \item \logicterm{分析}:这句话本身就是许诺行为,而不仅仅是描述许诺
\end{itemize}

\logicemph{宣告行为}:
\begin{itemize}
  \item \logicterm{情境}:婚礼结束时
  \item \logicterm{宣告}:"我宣布你们是夫妻"
  \item \logicterm{分析}:这句话本身就是使两人成为夫妻的行为
\end{itemize}
\end{examplebox}

\begin{theorembox}[title=践行性动词的特点]
\logicemph{定义}:践行性动词是一种特殊类别,它们代表的行动通常通过第一人称使用该动词而完成。

\logicemph{更多例子}:
\begin{itemize}
  \item "我祝贺你……"(祝贺行为)
  \item "我向你道歉……"(道歉行为)
  \item "我建议……"(建议行为)
  \item "我将这艘船命名为……"(命名行为)
  \item "我接受你的建议"(接受行为)
\end{itemize}

\logicwarn{条件限制}:这些行为的成功实施需要适当的语境和条件。
\end{theorembox}

\begin{theorembox}[title=自然语言的复杂性]
语词和语句的这些特殊用法展示了\logicemph{自然语言的丰富性}:

\logicwarn{分类系统的局限}:语言的诸多复杂功能\logicwarn{难以归纳为任何单一的分类系统}。

\logicemph{理论意义}:这提醒我们在分析语言时需要保持开放和灵活的态度。
\end{theorembox}

\chaptersummary{
现实中的语言使用远比理论分类复杂,呈现出丰富的多功能特征。

\logicemph{多功能性的普遍性}:
\begin{itemize}
  \item 语言在实际使用中常常兼具信息性、表达性和指令性的多种功能
  \item 纯粹的单一功能语言如同"化学标本",在现实中很少存在
  \item 成功的交流通常要求这些功能的巧妙结合
\end{itemize}

\logicemph{行动的心理学基础}:
\begin{itemize}
  \item 行动需要欲望和信念的双重条件
  \item 可以通过激发态度或提供信息来影响他人行动
  \item 不同的说服策略对应不同的语言功能组合
\end{itemize}

\logicemph{特殊用法的存在}:
\begin{itemize}
  \item 礼仪性用法:服务于人际关系的润滑和社会秩序的维护
  \item 践行性用法:语言本身就构成行动,而非仅仅描述行动
  \item 这些特殊用法超越了三重划分的范围
\end{itemize}

\logicwarn{理论启示}:自然语言的复杂功能难以归纳为任何单一的分类系统,这要求我们在语言分析中保持开放和灵活的态度。
}
\section{话语形式}

\begin{logicbox}[title=引言]
\textit{语言的形式与其功能之间不存在简单对应关系,理解这种复杂性有助于我们正确解读和分析各种语言表达的真实意图。}
\end{logicbox}

语句常常被定义为表达一个完整思想的语言单位。在语法教科书中,语句一般被分为四种类型,即\textbf{陈述句}、\textbf{疑问句}、\textbf{祈使句}和\textbf{感叹句}。但是,这四种语法分类与陈述、询问、命令和惊叹并不完全对应。我们可能会尝试把形式等同于功能,即认为陈述句和信息性话语是对应的,感叹句仅仅适用于表达性话语,或者我们会认为指令性话语完全包括祈使句或疑问语气的句子(把询问总是当做寻求回答的请求)。假如这种整齐的等同是可能的,那么交流问题就会大大地简单化了,因为这样我们就可以仅仅通过一段话的形式而说出它的原有功能,而其形式是很容易直接检查的。但是,那些把形式与功能等同的人会错误理解别人的话,而且或许会漏掉别人要传达的很多要点。

\subsection{陈述句与功能的关系}

设想任何具有陈述句形式的事物都是信息性话语,真的就予以好评,假的就予以拒斥,这显然是不正确的。"我在你的聚会上度过了一段美好的时光"是个陈述句,但它的功能完全不必是信息性的;倒不如说,它是礼仪性的或者表达性的,表达了友好和欣慰的感情。尽管很多诗歌和祷文的功能都不是信息性的,但它们却具有陈述句的形式。简单地认为它们是信息性的并简单地将它们评价为真的或假的,将会把自己排除在富有价值的审美和宗教体验之外。同样,许多请求和命令都是间接地——或许更加委婉地———用陈述句来表述的。"我喜欢咖啡"这个判断句不应当被侍者仅仅当做是顾客的表白,而应当被视为针对行动的指令或请求。对于陈述句,诸如"对这些帮助,我将非常感谢"或"我希望你课后能够在图书馆见到我",假如我们僵硬地判别它们的真与假,只是将它们视为信息报道,那么我们很快就会没朋友了。这些例子向我们表明,陈述的形式并不一定就标明信息性功能。陈述句在每种话语类型的表达方式中都能见到。

\subsection{其他语句形式的功能多样性}

其他语句形式也是如此。"你意识到我们几乎要迟到了吗?"这个疑问句,不必是在询问你的大脑状态的有关信息,而可能是要求抓紧时间。疑问句"1939年,俄国和德国签署了一个条约,它导致了第二次世界大战,不是吗?"可能完全不是询问,而可能是交流信息的间接方式或者是企图表达并激发一种对俄国的敌对情感;在第一种情况下起的是信息性功能,而在第二种情况下起的是表达性功能。甚至语法上的祈使句,如在公文中以"兹请周知 $\cdots \cdots$"开头,可能并不是命令,在其所断言的东西中是信息性话语;在其激发神圣和庄重的适当情感的语言用法中是表达性话语。就感叹句来说,尽管它与表达性话语的功能关系紧密,但也可以有相当不同的功能。感叹句"天啊,要迟到了!"在语境中可以表示抓紧时间的请求。而房地产经纪人对潜在的顾客说出"多么美好的景色啊!"这个感叹句,其祈使性功能比表达性功能更浓重。

\subsection{多功能话语的评价标准}

许多话语都企图同时达到语言的两种或者可能是其全部的三种功能。在这种情况下,给定语段的每一方面或功能都从属于它自己的适当标准。一个具有信息性功能的语段可以具有能够评价为真或假的方面。同样的语段,也可以具有指令性功能,能够以恰当或不恰当、对或错等进行评价。而假如这个语段还具有表达性功能,那么其组成成分还可以评价为真诚或虚假、是否宝贵等。正确地评价某一给定语段,需要把握语言的不同功能以及该语段本身的宗旨。

对逻辑学家来说,真与假,以及与之相关的论证正确与错误的概念,是最重要的。因此,作为学习逻辑的学生,我们必须能够将信息性功能与非信息性功能的话语区分开来。进而,我们还必须能够理顺给定语段所具有的信息性功能与其可能具有的所有其他功能之间的关系。为了做到"理顺",我们必须知道语言可以具有哪些功能,以及必须能够将它们区分开来。语段的语法结构常常能够标示其功能,但是,功能与语法形式之间并没有必然联系。功能与语段表面上断言的内容之间也不存在严格的关系。关于这一点,一位大语言学家在他对"意义"(meaning)的讨论中给出了例证说明:

\begin{displayquote}
一个顽皮的孩子,该上床睡觉时却说"我饿了",他母亲知道他的把戏,就以打发他去睡觉来回答他。这是移位语(dis- placed speech)(1)的一个例子。 ${ }^{[4]}$
\end{displayquote}

在这里,这个孩子的话是指令性的一一虽然没有成功地实现其所希望的改变。关于语段的功能,我们一般是指它想要达到的那种功能。但不幸的是,那并不是总能够轻而易举地判定的。

\footnotetext{(1)日常会话中的"托词"是"移位语"的一种典型,本处即为托词之例。
}

\subsection{语境在功能判定中的重要性}

当孤立地引用一个语段时,要判定该语段最初欲要达到的功能就特别困难。其原因在于,\logicterm{语境}在判定功能的过程中极其重要。某些本身是祈使性或者普通信息性的语句,如果在实际语境中将它们安排到一个具有诗化效果的整体之中,就可以变为表达性语句。例如,孤立的:

到窗口来吧,

是一个起指令性功能的祈使句。而:

今夜海上风平浪静

是陈述句,起着信息性功能。它们好像都没有较大的表达性威力,但是在马修•阿诺德(Matthew Arnod)的诗歌《多弗海滨》(Dover Beach)的语境中,二者都主要用为诗的表达性功能,而且效果显著。很多诗歌完全依赖于语段的表达性用法,而在其他语境中,这些语段却具有根本不同的功能。

下面的区分也是重要的,即\textbf{句子表示的命题}与\textbf{关于说话者的事实}(句子的说出就是证据)之间的区分。当有人说"天在下雨"时,这个被断言的命题是关于天气的,而不是关于说话者的。但做出断言却构成了说话者相信天在下雨的证据,这就是关于说话者的事实。也可能发生下述情况:人们做出的陈述只是表面上关乎他们的信念,实际上并不是为了给出关于他们自己的信息,而完全是一种述说其他事情的方式。如说"我认为黄金不应该用做货币的标准",通常并不能理解为关于说话者信念的心理自我表白,而只能理解为断言不应该使用黄金作为货币标准的一种方式。而当说话者发出命令时,由此推断说话者希望有人完成某事却是合理的;的确,在有些环境中,只要断言某人有某种特殊渴望,实际上就是给出了一个命令或者做出了一个请求。快乐地欢呼证明说话者非常愉快,即使他在整个过程中没有做出任何断言。但是,作为心理报道,断言说话者快乐就是肯定一个命题,这与只是快乐地欢呼相当不同。

\subsection{论证与说明的进一步探讨}

在1.6节中我们注意到,论证与说明之间的差别常常取决于语段的说话者或作者的目的。现在,对语言的不同功能的探讨允许我们可以对该问题进行更深人的考察。

当说话者论及某个有争议问题时,如果他说"我强烈反对什么什么",那么我们就理解,这样一句话的目的通常不是为了报道说话者的观点(除非这样的话是一位公职候选人或者其观点代表了公众利益的知名人士所讲的)。实际上,这种自我报道的表达形式是述说什么什么是个坏主意并且我们都应当反对它的常用方式。当说话者不断地证明我们所持的观点是正确的时候,他并不是在说明他的判断,而是有意的论证,以说服别人相信他的判断是正确的。对于有些争论性问题,通过陈述自己的观点而展开论证,并不是有意欺骗;在这种情况下,即使把判断和自我报道混合在一块儿,也不是有意欺骗。

在一个单独语段中,一种以上重要功能的组合可能会成为问题。思想表达,受我们宪法第一修正案的保护,可能会包含极具冒犯性的语言;在这种情况下,认识到冒犯言语中信息性与情感性功能的融合(integration),对保护言论自由可能是极其关键的。在洛杉矶地区法院,一个年轻人身穿故意装饰有亵渎之物的夹克来抗议越战时期的军事法案;根据加利福尼亚刑法典,他以冒犯行为而被判处有罪。高等法院推翻了对他的指控,并雄辩地精确表述了这个问题:

\begin{displayquote}
我们不能忽视这个事实,这里牵涉的事件极好地表明,许多语言表达都具有双重交流功能:加上附带说明,它们不仅可以传达相当精确的思想,还可以传达其他不可表达的情感。实际上,与其情感力量一样,语词也常常因其认知力量而被选用。我们不能赞同这样的观点:宪法关注个人言辞的认知内容,而没有或不关心其情感功能。的确,情感功能可能常常是在寻求交流的信息的更重要因素……同样,我们也不能纵容这种轻率假定:可以禁止特殊言辞,但在这个过程中又不遭遇压制思想的真正危险。 ${ }^{[5]}$
\end{displayquote}

区别语言的信息性以及论证性功能与语言的其他功能,并没有一个机械的方法。在后面几章中,我们发展的逻辑技术可以相当机械地运用于检验论证的有效性,但是没有一个机械技术可以识别论证的出现。识别在一给定语境中的话语的不同功能,要对语言的灵活性和其用法的复杂性熟虑而敏感。

\begin{center}
\begin{tabular}{|l|c|}
\hline
\textbf{语言的主要用法} & \textbf{语言的语法形式} \\
\hline
信息性用法 & 陈述句 \\
\hline
表达性用法 & 疑问句 \\
\hline
指令性用法 & 祈使句 \\
\hline
 & 感叹句 \\
\hline
\multicolumn{2}{|p{0.8\textwidth}|}{语法形式常常是功能的一个标志,但语法形式与其使用目的之间并没有必定的联系。用做三种主要功能(左侧纵列)的任何一个功能的语言,可能会采用四种语法形式(右侧纵列)中的任何一种。} \\
\hline
\end{tabular}
\end{center}

\chaptersummary{
语言的形式与功能之间的关系远比表面看起来复杂,这种复杂性是语言灵活性的体现。

\logicemph{形式与功能的非对应性}:
\begin{itemize}
  \item 语法形式(陈述句、疑问句、祈使句、感叹句)与语言功能(信息性、表达性、指令性)之间\logicwarn{不存在一一对应关系}
  \item 任何语法形式都可能承担任何语言功能
  \item 简单的形式-功能等同会导致误解和交流失败
\end{itemize}

\logicemph{语境的决定性作用}:
\begin{itemize}
  \item 语境在判定语言功能过程中\logicterm{极其重要}
  \item 同样的语句在不同语境中可能具有完全不同的功能
  \item 孤立分析语段往往无法准确判定其真实功能
\end{itemize}

\logicemph{实践意义}:
\begin{itemize}
  \item 识别语言的真正功能需要综合考虑形式、内容、语境和意图等多种因素
  \item 对于逻辑学习者,必须能够区分信息性功能与非信息性功能
  \item 没有机械的方法可以识别论证的出现,需要对语言的灵活性保持敏感
\end{itemize}
}
\input{chapter2/2-4 情感词汇.tex}
\section{一致与歧见的种类}

\begin{logicbox}[title=引言]
\textit{语言交流中的分歧不仅可能来自事实认知的差异,也可能源于对同一事实的态度不同,理解这些不同类型的分歧有助于我们采取正确的方法解决分歧。}
\end{logicbox}

任何事物或行动都可以选择不同的短语来描述,传达赞许、反对或中立的意见。这种描述方式的多样性导致了不同类型的\logicterm{一致}(agreement)和\logicterm{歧见}(disagreement)。

\begin{theorembox}[title=歧见的基本分类]
人际交流中的分歧可以分为两种根本不同的类型:

\logicemph{信念歧见}(Disagreement in Belief):
\begin{itemize}
  \item \logicterm{核心}:对事实本身的认知不同
  \item \logicterm{争议焦点}:某事是否实际发生
  \item \logicterm{性质}:关于客观现实的不同判断
\end{itemize}

\logicemph{态度歧见}(Disagreement in Attitude):
\begin{itemize}
  \item \logicterm{核心}:对同一事实的评价不同
  \item \logicterm{争议焦点}:如何看待已知事实
  \item \logicterm{性质}:关于价值判断的不同立场
\end{itemize}

\logicwarn{重要区别}:态度歧见中,双方对事实本身可能完全一致,分歧在于对事实的感受和评价。$^{[9]}$
\end{theorembox}

\subsection{一致与歧见的四种关系}

基于信念和态度这两个维度,我们可以构建一个分析框架,区分出两个人(设为A和B)之间可能存在的四种关系类型:

\begin{theorembox}[title=四种关系类型的系统分析]
\logicemph{类型一:充分一致}
\begin{itemize}
  \item \logicterm{信念维度}:对事件发生的认知一致
  \item \logicterm{态度维度}:对事件的评价一致
  \item \logicterm{特征}:完全和谐,无任何分歧
\end{itemize}

\logicemph{类型二:信念一致,态度对立}
\begin{itemize}
  \item \logicterm{信念维度}:对事实的认知相同
  \item \logicterm{态度维度}:对同一事实的评价截然不同
  \item \logicterm{特征}:事实无争议,价值观有冲突
\end{itemize}

\logicemph{类型三:态度一致,信念对立}
\begin{itemize}
  \item \logicterm{信念维度}:对事实的认知不同
  \item \logicterm{态度维度}:对各自认知的事实评价相同
  \item \logicterm{特征}:价值观相同,但基于不同的事实认知
\end{itemize}

\logicemph{类型四:完全对立}
\begin{itemize}
  \item \logicterm{信念维度}:对事实的认知不同
  \item \logicterm{态度维度}:对各自认知的事实评价也不同
  \item \logicterm{特征}:全面冲突,既有事实争议又有价值冲突
\end{itemize}
\end{theorembox}

\begin{examplebox}[title=政治候选人立场变化的案例分析]
设想讨论的事件是:某位政治候选人在争议问题上的立场改变。

\logicemph{类型二示例}(信念一致,态度对立):
\begin{itemize}
  \item \logicterm{共同认知}:A和B都认为候选人确实改变了立场
  \item \logicterm{A的态度}:认为这是好事,称赞为"倾听了理性的呼声"
  \item \logicterm{B的态度}:认为这是坏事,谴责为"机会主义的反复无常"
\end{itemize}

\logicemph{类型三示例}(态度一致,信念对立):
\begin{itemize}
  \item \logicterm{共同态度}:A和B都热情赞扬该候选人
  \item \logicterm{A的认知}:认为候选人"倾听了理性的呼声"而改变立场
  \item \logicterm{B的认知}:认为候选人"坚定不移地拒斥奉承"而未改变立场
\end{itemize}

\logicwarn{现实性说明}:类型三在政治选举中很常见,同一候选人常因不同甚至不相容的原因获得支持。

\logicemph{类型四示例}(完全对立):
\begin{itemize}
  \item \logicterm{A的立场}:认为候选人改变了立场,并称赞这是"明智的重新考虑"
  \item \logicterm{B的立场}:认为候选人未改变立场,并贬斥其"顽固地拒绝承认错误"
\end{itemize}
\end{examplebox}

\subsection{解决不同类型歧见的方法}

\begin{theorembox}[title=歧见解决的基本原则]
解决歧见的关键在于\logicwarn{准确识别歧见类型},因为不同类型的歧见需要完全不同的解决策略。

\logicemph{诊断的重要性}:
\begin{itemize}
  \item 必须既关心事实层面,又关心态度层面
  \item 如果不清楚歧见类型,就无法选择正确的解决方法
  \item 错误的方法不仅无效,还可能加剧冲突
\end{itemize}
\end{theorembox}

\begin{theorembox}[title=信念歧见的解决策略]
\logicemph{核心方法}:通过确认事实来解决分歧

\logicemph{具体步骤}:
\begin{itemize}
  \item \logicterm{证据收集}:询问证人、查阅文献、检查记录
  \item \logicterm{事实核实}:运用科学探究方法
  \item \logicterm{结果}:当事实得到确证,歧见自然解决
\end{itemize}

\logicemph{适用条件}:争议的核心是"事实是什么"
\end{theorembox}

\begin{theorembox}[title=态度歧见的解决策略]
\logicemph{基本特征}:解决方法与信念歧见\logicwarn{有相当大的差别},难以直接了当

\logicemph{无效方法}:
\begin{itemize}
  \item 召唤证人、查阅文本等事实确认方法
  \item 争议的问题不是事实本身,而是如何评价事实
\end{itemize}

\logicemph{有效策略}:
\begin{itemize}
  \item \logicterm{后果分析}:考虑相关事情不发生会怎样
  \item \logicterm{动机探讨}:分析行为的动机和目的
  \item \logicterm{表达性说服}:大量使用表达性语言
  \item \logicterm{修辞艺术}:在凝聚团体意志和统一态度方面富有成效
\end{itemize}

\logicwarn{重要说明}:这些方法涉及的虽然也是事实问题,但不是存在态度冲突的那些事实。修辞艺术在解决事实问题上完全没有价值。
\end{theorembox}

\subsection{情感语言与伦理判断}

\begin{theorembox}[title=伦理语词的双重性质]
\logicemph{伦理语词的特征}:
\begin{itemize}
  \item \logicterm{典型词汇}:"好"与"坏"、"对"与"错"等
  \item \logicterm{情感色彩}:在严格伦理用法中具有非常强烈的情感色彩
  \item \logicterm{态度表达}:描述某行动为"对"或某情形为"好"时,表达赞赏态度
\end{itemize}

\logicemph{哲学争议}:
\begin{itemize}
  \item \logicterm{情感主义观点}:这些词汇"没有"认知意义,只有情感意义
  \item \logicterm{认知主义观点}:这些词汇具有认知意义,指称客观特征
  \item \logicwarn{逻辑学立场}:学习逻辑的学生不必在此争论中偏向某一方
\end{itemize}
\end{theorembox}

\begin{theorembox}[title=态度表达的多样性]
\logicwarn{重要区别}:并非所有的赞同或不赞同态度都蕴含道德判断。

\logicemph{态度的不同来源}:
\begin{itemize}
  \item \logicterm{道德判断}:基于伦理标准的评价
  \item \logicterm{审美价值}:基于美学标准的评价
  \item \logicterm{个人偏好}:基于个人口味的评价
\end{itemize}

\logicemph{例证}:对某些食品或服装款式的否定态度,不必涉及伦理或审美判断,但同样可以用情感色彩强烈的语言表达。
\end{theorembox}

\subsection{实质歧见与言辞歧见}

\begin{theorembox}[title=态度歧见的实质性]
\logicwarn{常见误解}:认为态度歧见不是"实质的"或者是"纯粹言辞的"

\logicemph{态度歧见的特征}:
\begin{itemize}
  \item 最强烈的态度歧见可以用\logicterm{朴实无华的真实陈述}来表达
  \item 双方可能以\logicterm{逻辑上相容的陈述}来表述分歧
  \item 他们并非仅仅"用不同言辞说相同的事物"
\end{itemize}

\logicemph{实质性的体现}:
\begin{itemize}
  \item 可以用不同言辞断言词汇意义上相同的事实
  \item 也可以用不同言辞表达对相同事实的\logicterm{矛盾态度}
  \item 虽然歧见不是"词汇意义上的",但却是\logicemph{实质的}
\end{itemize}

\logicwarn{理论意义}:这不是"纯粹言辞的"歧见,因为语词既具有表达性功能也具有信息性功能。
\end{theorembox}

\begin{theorembox}[title=歧见类型识别的困难]
\logicemph{实践中的复杂性}:
\begin{itemize}
  \item 有时难以确定歧见是信念的、态度的,还是两者兼有
  \item 这可能取决于争论者对词汇的特定解释
  \item 冲突意见的表达方式可能模糊不同类型的区别
\end{itemize}

\logicemph{识别的挑战}:
\begin{itemize}
  \item 争论的关键核心变得难以把握
  \item 表面的事实争议可能实际上是态度争论
  \item 当争论涉及事物或行为的\logicterm{价值}时尤其如此
\end{itemize}

\logicwarn{实践启示}:准确识别歧见类型对于选择正确的解决策略至关重要。
\end{theorembox}

\begin{examplebox}[title=体育哲学中的态度歧见]
关于赢球重要性的经典争论展示了态度歧见的复杂性:

\logicemph{格兰特兰德·赖斯}(体育作家)的观点:
\begin{displayquote}
当球星走进球场,为其声名留下光芒,这光芒不论输与赢,只记录场上飞奔的身影。
\end{displayquote}

\logicemph{文斯·隆巴蒂}(足球教练)的观点:
\begin{displayquote}
赢球不是别的,它就是唯一(竞赛目标)。
\end{displayquote}

\logicemph{分析思考}:
\begin{itemize}
  \item 两位对赢球的态度明显冲突
  \item 这种态度歧见的根源是信念歧见吗?
  \item 还是纯粹的价值观差异?
\end{itemize}

\logicwarn{启发性问题}:这个例子帮助我们思考如何区分不同类型的歧见。
\end{examplebox}

\begin{theorembox}[title=歧见分析的价值与局限]
\logicemph{理论价值}:
\begin{itemize}
  \item 态度歧见与信念歧见的区分\logicterm{非常有用}
  \item 留心语言的不同用法有助于理解各种歧见
  \item 可以澄清讨论的问题,揭示冲突类型和所在
\end{itemize}

\logicwarn{实践局限}:
\begin{itemize}
  \item 找出区别本身\logicwarn{并不能解决问题或消除歧见}
  \item 识别过程中存在不可避免的困难
  \item 需要结合具体情况进行综合判断
\end{itemize}

\logicemph{核心原理}:我们越充分地理解歧见的本性,就越能够更好地解决歧见。
\end{theorembox}

\chaptersummary{
人际交流中的分歧具有复杂的结构,理解这种复杂性是有效解决冲突的前提。

\logicemph{歧见的基本分类}:
\begin{itemize}
  \item \logicterm{信念歧见}:对事实本身的认知不同,争议焦点是"事实是什么"
  \item \logicterm{态度歧见}:对同一事实的评价不同,争议焦点是"如何看待事实"
  \item 两种歧见在性质和解决方法上根本不同
\end{itemize}

\logicemph{四种关系类型}:
\begin{itemize}
  \item \logicterm{充分一致}:信念和态度都一致
  \item \logicterm{信念一致,态度对立}:事实无争议,价值观有冲突
  \item \logicterm{态度一致,信念对立}:价值观相同,但基于不同事实认知
  \item \logicterm{完全对立}:既有事实争议又有价值冲突
\end{itemize}

\logicemph{解决策略的差异}:
\begin{itemize}
  \item \logicterm{信念歧见}:通过确认事实、科学探究等方法解决
  \item \logicterm{态度歧见}:需要后果分析、动机探讨、表达性说服、修辞艺术等策略
  \item 准确识别歧见类型是选择正确解决方法的关键
\end{itemize}

\logicwarn{实践意义}:态度歧见同样是实质性的,不是"纯粹言辞的";理解歧见的本性越充分,就越能够更好地解决分歧。
}
\section{情感中性语言}

\begin{logicbox}[title=引言]
\textit{在追求事实真相和理性分析时,情感中性的语言扮演着至关重要的角色。理解何时该使用情感语言、何时该避免情感色彩,有助于我们更有效地进行交流和推理。}
\end{logicbox}

语言的表达性用法与信息性用法一样是正当的。情感语言本身没有什么不当,非情感语言或曰中性语言也没有什么不当。这就如同我们可以说枕头没什么错,锤子也没有什么错一样毋庸置疑。但是我们不能用枕头钉钉子,也不能枕在锤子上舒服地休息。同样,当我们用实话实说的话语来替换诗人的情感语言时,尽管可以保留罗曼蒂克诗句的字面意义,但它会失去很多兴味。在某些类型的诗歌中,情感色彩浓郁的语言比中性语言更受喜爱;而在另一些领域中,中性语言则比情感色彩浓郁的语言更为可取。

\subsection{中性语言在追求真理中的价值}

那是些什么领域呢?如果把探求现实真理作为我们的目标,那么\textbf{中性语言}就应更受重视。当我们试图了解事实的真相所在,或者试图加以论证时,心猿意马就会招致失败,而情感因素正是一种分散注意力的力量。激情倾向于掩盖理性,"冷静"(dispassionate)与"客观"(objective)两词接近同义的用法就反映了这个道理。因此,当我们试图以冷静和客观的方式推论事实时,使用强烈的情感语言便是有害而无益的。

\subsection{面对价值冲突的语言选择}

当我们处理某些冲突的话题时,使用完全中性的、不受感情影响的语言或许并不可行。例如,在流产是否正当的论辩中,论辩对手(无论哪一方)所使用的关键词汇就非常可能为情感所扭曲,因为此时并不存在完全不带情感色彩的所有各方都接受的价值中性词汇。在这种情况下,如果真正的目标仍然是求真,那么就应当尽可能地减少所用词汇的情感负荷。情感中性的目标不可能完全达到,但是我们至少可以尽力使用这种语言:它仅预设论辩者都赞同的信念(无论何种)。情感色彩的语言肯定是分散注意力的,情感意义负荷过重的语言是不可能促进求真的。

在论文《决定论的两难》(The Dilemma of Determinism)中,威廉•詹姆士(William James)提出"希望消灭"自由'这个词",其理由是"它的歌功颂德的联想……使它的所有其他意义黯然失色"。他更喜欢恰当地使用词汇"决定论"和"非决定论"来讨论"意志自由",他说,这是因为"它们冰冷的和数学的面孔没有情感牵连,而情感牵连可以预先以各种方式来贿赂我们"。詹姆士的做法为我们树立了榜样。

\subsection{民意调查中的情感陷阱}

从事专业民意调查的采访者必须非常谨慎,以免对调查询问中使用情感措辞而收到的反馈产生偏见。如果忽视这种慎重,调查结果就可能是没有价值的。1993年,《时代》(Time)和有线电视新闻网(Cable News Network)在一项大型民意调查中询问道:"应当通过立法来禁止利益集团赞助竞选吗?或者,利益集团的确应当拥有赞助他们支持的候选人的权利吗?"在所有回复者中, $40 \%$ 的人说他们赞同禁止利益集团的赞助, $55 \%$ 的人回答利益集团有赞助的权利。同年,罗斯-佩罗(Ross Perot),一位非常富有的总统候选人,组织了其自己的民意测验,其中如此询问: "应该通过立法来消除所有特殊利益者给候选人大笔金钱的可能性吗?"冊庸惊奇,对这个问题 $80 \%$ 的回答是"是的",这样的赞助应当禁止。包含像"特殊利益"和"大笔金钱"这样的短语无疑妨害了了解人们对这种事情的真实看法。\cite{perot1993}\cite{timecnn1993} 可以说,这两个民意测验问的不是同样的问题。但即使如此,逻辑要点仍然是:如果我们的目标是交流信息,如果我们希望避免误会,那么我们就应该尽可能少地使用情感色彩浓厚的语言。

\subsection{警惕情感语言的操纵}

玩弄情感,而不是诉诸理性,是那些从歪曲真相中获得好处的人的常用手段。这种操纵的努力最公开的展示是广告世界,那里压倒一切的目的总是说服、销售并且常常是获利。我们必须经常警惕这些情感负荷的语言用法,也要警惕它们在政治活动中的使用,几乎每一种修辞手段都在政治活动中一再使用。本杰明•迪斯雷利说:"我们能利用语词来支配人。"我们最好的防御就是对语言及其不同的用法要多思而敏感,并具备识别那些不讲道德原则的人强词夺理的伎俩的能力。

\chaptersummary{

  在追求事实真相和客观论证时,应优先使用情感中性的语言;\\
  面对价值冲突话题,应尽量减少词汇的情感负荷;\\
  警惕广告和政治中情感语言的操纵,培养对修辞手段的辨别能力。

} 

% 第三部分
\chapter{论争与定义}
\section{论争、言辞之争与定义}

\begin{logicbox}[title=引言]
\textit{在交流过程中,人们的分歧有时是实质性的,有时却只是语言上的误解。理解不同类型的论争有助于我们更有效地解决问题,避免在无实质内容的争论上浪费时间和精力。}
\end{logicbox}

\begin{theorembox}[title=语言的双重性质]
语言是一种极其复杂的设施,是人类最重要的交流工具。然而,这种强大的工具也可能成为我们的负担:

\logicemph{积极作用}:
\begin{itemize}
  \item 促进思想交流和理解
  \item 传递信息和知识
  \item 表达情感和态度
\end{itemize}

\logicwarn{消极影响}:当语词被漫不经心或错误地使用时,语言会阻碍而非促进交流。
\end{theorembox}

\begin{theorembox}[title=从歧见到论争的理论发展]
\logicemph{理论回顾}:在2.5节中,我们阐释了冲突双方的歧见既可能是\logicterm{信念的}也可能是\logicterm{态度的},两者都可能构成实质歧见。

\logicemph{新的发现}:除了实质歧见外,还存在另一种情况:
\begin{itemize}
  \item \logicwarn{表面歧见}:看似存在分歧,实际上并无真正对立
  \item \logicterm{根本原因}:误解或词汇误用的结果
  \item \logicterm{识别意义}:避免在虚假争议上浪费时间和精力
\end{itemize}

\logicemph{研究转向}:从考察实质歧见转向讨论不同类型的\logicterm{论争},重点在于分辨其是否具有真正的分歧。
\end{theorembox}

\subsection{三种不同类型的论争}

\begin{theorembox}[title=三种论争类型]
必须区分出三种不同的论争:

\textbf{第一种:}\logicterm{明显的实质论争},在这种论争中,各方或者在信念上或者在态度上,明确地毫不含糊地对立。

\textbf{第二种:}\logicterm{纯粹的言辞之争},表面上的分歧实际上只是语言使用的差异。

\textbf{第三种:}\logicterm{表面上的言辞之争但实际上的实质论争},既有语言问题又有实质分歧。
\end{theorembox}

\begin{examplebox}[title=实质论争的典型例子]
\logicemph{态度歧见的例子}:
\begin{itemize}
  \item \logicterm{情境}:美国佬(Yankees)赢得了世界联赛
  \item \logicterm{共同认知}:双方对胜利者本身的认同没有争论
  \item \logicterm{态度分歧}:A为此高兴,B为此恼怒
  \item \logicterm{特征}:态度歧见显然且可能激烈,属于难以解决的问题
\end{itemize}

\logicemph{信念歧见的例子}:
\begin{itemize}
  \item \logicterm{争议内容}:巴拿马运河的太平洋入口是否比大西洋入口更靠东
  \item \logicterm{分歧性质}:不是态度上的,而是事实认知上的
  \item \logicterm{解决方法}:一张好地图就可以平息这个论争
\end{itemize}
\end{examplebox}

\begin{theorembox}[title=实质论争的基本特征]
\logicemph{核心特征}:无论是态度上的还是信念上的,实质论争总是包含某种\logicterm{实质歧见}。

\logicemph{区别标准}:
\begin{itemize}
  \item 将论争双方区别开来的\logicwarn{不仅仅是语言}
  \item 在对事实的断定上或对事实的评价上存在\logicterm{实质差别}
  \item 不能通过定义或简单的语言调整来解决
\end{itemize}
\end{theorembox}

\begin{theorembox}[title=事实的多样性与论争的实质性]
\logicemph{事实的广泛性}:
\begin{itemize}
  \item \logicterm{物理事实}:如地理位置、物理现象
  \item \logicterm{语言事实}:如单词的拼写或使用方法
  \item \logicterm{心理事实}:如某人是不友好还是仅仅害羞
\end{itemize}

\logicemph{论争的实质性判断}:
\begin{itemize}
  \item 各方可以在任何种类的事实上产生歧见
  \item \logicwarn{关键原则}:如果论争确实是关于某个事实的,那么它就是实质的
  \item \logicterm{解决途径}:可以通过确认相关事实而得到解决
\end{itemize}
\end{theorembox}

\subsection{纯粹的言辞之争}

\begin{theorembox}[title=纯粹言辞之争的定义与特征]
\logicemph{基本定义}:\logicterm{纯粹的言辞之争}是指双方之间根本没有实质歧见,但却表面上好像具有歧见的论争。

\logicemph{根本原因}:语言的误解或误用是症结所在。

\logicemph{产生机制}:
\begin{itemize}
  \item 论争者信念表达中的关键语词有歧义
  \item 这种歧义遮蔽了双方实际上并没有实质对立的事实
\end{itemize}
\end{theorembox}

\begin{theorembox}[title=言辞之争的具体成因]
\logicemph{三种典型情况}:
\begin{itemize}
  \item \logicterm{误用情况}:争论某方误用了一个重要词语
  \item \logicterm{歧义情况}:核心语词或短语具有不同含义,这些含义可能同等合法但产生了不应有的混淆
  \item \logicterm{认知盲区}:各方对语词或短语的用法都正确但含义不同,而这一点没有被明确认知
\end{itemize}

\logicwarn{共同特点}:所有这些情况都可能产生表面的言辞之争。
\end{theorembox}

\begin{theorembox}[title=言辞之争的识别与解决]
\logicemph{识别难度}:言辞之争并非总是容易发现,需要仔细分析。

\logicemph{解决方法}:
\begin{itemize}
  \item 一旦识别了言辞之争,可以相当容易地获得解决
  \item \logicterm{核心策略}:具体化歧义语词或短语的不同含义
  \item \logicterm{关键工具}:良好的定义对相互理解非常关键
\end{itemize}
\end{theorembox}

威廉•詹姆士给出了这种言辞之争的一个经典例子:

\begin{displayquote}
几年前,我随野营队一起在山上露营。当我独自散步返回时,发现每个人都参加了一场激烈的形而上学论战。争论主题是一只松鼠。设想一只松鼠抓附在树干一侧,而一个人站在树的另一侧;那人绕树迅速转动以试图看到松鼠,但无论他转多么快,松鼠在相对的方向都以同样快的速度转动,在它自己和那人之间总隔着那棵树,因此使他看不到松鼠。作为结果的形而上学问题是:这个人是否绕着松鼠走了一圈?确确实实,他绕树走了一圈,而且松鼠就在树上;但是,他绕松鼠走了一圈吗?原野中的讨论持续良久,直到变得乏味。每个人都赞同一种观点并固执己见,并且双方的人数势均力敌。因此,当我出现时,每方都希望我加入以便成为多数派。\cite{james1907}
\end{displayquote}

\begin{theorembox}[title=松鼠例子的深层分析]
\logicemph{实质歧见的缺失}:
\begin{itemize}
  \item 论争双方之间不存在实质歧见
  \item 所有论争者对松鼠和树的态度都是中立的
  \item 都完全理解和赞同给定事例的所有事实
\end{itemize}

\logicwarn{争论的本质}:这个事例(以及许多其他类似事例)中,争论不过是言辞之争。
\end{theorembox}

\begin{examplebox}[title=詹姆士的解决方案]
詹姆士继续写道:

\begin{displayquote}
有绕它走一圈,因为松鼠做了相对运动,它始终保持着将其腹部对着那个人,而将背部朝着外面。做出这种区分,就没有什么可争论的了。你们都又对又不对,就看你们对"绕走一圈"这个动词实际上是怎么理解的。\cite{james1907}
\end{displayquote}

\logicemph{解决策略}:
\begin{itemize}
  \item \logicterm{概念区分}:明确"绕走一圈"的不同含义
  \item \logicterm{相对性认识}:双方都既对又不对,取决于对概念的理解
  \item \logicterm{争论消解}:一旦做出区分,就没有什么可争论的了
\end{itemize}
\end{examplebox}

\begin{theorembox}[title=纯粹言辞之争的解决原理]
\logicemph{解决要求}:
\begin{itemize}
  \item \logicwarn{不需要新事实}:解决这个论争不要求新的事实,那样做也不可能有帮助
  \item \logicterm{核心需求}:仅需要对争论中关键语词的不同意义做出区分
  \item \logicterm{效果}:使用不同定义,争论就消失了
\end{itemize}

\logicemph{一般原理}:
\begin{itemize}
  \item 如果论争纯粹是言辞之争,可以通过提供能够消除关键歧义的定义来解决
  \item 可以表明论争各方并不是真正的相互对立
  \item 他们可能仅仅是运用相同词汇的不同含义来维护不同主张,或运用不同语词来维护相同主张
\end{itemize}

\logicwarn{最终结果}:一旦确定了不同意义以及源于对不同意义使用的不同主张,双方之间就不会再有什么论争。\cite{rudin1992}
\end{theorembox}

\subsection{表面上的言辞之争与实质论争}

\begin{theorembox}[title=第三种论争的定义与特征]
\logicemph{基本定义}:\logicterm{表面上是言辞的但实际上是实质的论争}

\logicemph{表面特征}:
\begin{itemize}
  \item 当双方互相误解了对方词语的用法时会出现混淆
  \item 这种混淆可以得到识别
  \item 看似只是语言问题
\end{itemize}

\logicemph{深层实质}:
\begin{itemize}
  \item 争执远超出语词不同用法的范围
  \item 仅仅解决歧义问题\logicwarn{不会平息论争}
  \item 争论双方之间还存在某种实质歧见:可能在信念上,更可能在态度上
\end{itemize}
\end{theorembox}

\begin{examplebox}[title=色情作品争议的案例分析]
\logicemph{争议情境}:对有露骨性活动镜头的影片是否应该作为"色情作品"来处理

\logicemph{双方立场}:
\begin{itemize}
  \item \logicterm{甲方观点}:露骨性使它成了邪恶的色情作品
  \item \logicterm{乙方观点}:考虑到细腻的情感和美学价值,它是真正的艺术,根本不是色情作品
\end{itemize}

\logicemph{分析层次}:
\begin{itemize}
  \item \logicterm{表面层次}:双方的歧见在于"色情作品"一词的意义
  \item \logicwarn{深层实质}:即使言语的不同得到充分理解并清除所有歧义,双方很可能对影片仍然存在实质歧见
  \item \logicterm{真正争议}:不是关于"色情作品"这个词的适用性,而是影片的性感露骨性质是否造成影片的好与坏
\end{itemize}
\end{examplebox}

\subsection{标准之争与概念之争}

\begin{theorembox}[title=标准之争与概念之争]
\logicemph{别名}:第三种论争有时被称为\logicterm{标准之争}或\logicterm{概念之争}。

\logicemph{核心特征}:
\begin{itemize}
  \item 论争双方对某个关键词语的运用有着\logicterm{不同标准}
  \item 该词语指谓的是\logicterm{不同概念}
  \item 各方在不同标准的明智或正确性问题上处于尖锐冲突
\end{itemize}

\logicemph{深层冲突}:即使歧义得到阐明和区分,各方仍可能声称对手误用了标准。
\end{theorembox}

\begin{examplebox}[title=色情作品争议的深层分析]
即使双方都认识到他们有歧义地使用了"色情作品"这个词,甚至词语的歧义已经得到阐明和区分,但冲突仍然存在:

\logicemph{标准冲突}:
\begin{itemize}
  \item \logicterm{甲方标准}:如果影片包含露骨的性活动场景,便可以将它划归为色情作品
  \item \logicterm{乙方回应}:那种划归是一种概念错误
\end{itemize}

\logicwarn{论争本质}:这种论争表面上仅仅是言辞之争,但在表面之下,却是非常实质的论争。
\end{examplebox}

\subsection{识别不同类型的论争}

\begin{theorembox}[title=论争类型识别流程]
为帮助人们辨识和理解论争的不同种类,我们可以使用一个有用的\logicterm{诊断流程}:

\logicemph{第一步}:确定存在某种论争后,问:\logicwarn{"出现歧义了吗?"}
\begin{itemize}
  \item 如果回答"没有" → \logicterm{类型一}(明显的实质论争)
  \item 如果回答"出现了" → 进入第二步
\end{itemize}

\logicemph{第二步}:问:\logicwarn{"清除歧义可以消除对立吗?"}
\begin{itemize}
  \item 如果回答"可以" → \logicterm{类型二}(纯粹言辞之争)
  \item 如果回答"不可以" → \logicterm{类型三}(表面言辞实质论争)
\end{itemize}
\end{theorembox}

\begin{theorembox}[title=三种论争类型的系统总结]
\logicemph{类型一:明显的实质论争}
\begin{itemize}
  \item \logicterm{歧义状况}:不存在言辞歧义
  \item \logicterm{歧见性质}:争论双方确实有歧见(态度上或信念上)
  \item \logicterm{解决方法}:通过解决实质性的信念或态度歧见
\end{itemize}

\logicemph{类型二:纯粹言辞之争}
\begin{itemize}
  \item \logicterm{歧义状况}:存在言辞歧义
  \item \logicterm{歧见性质}:根本没有实质歧见
  \item \logicterm{解决方法}:通过明确定义和消除歧义
\end{itemize}

\logicemph{类型三:表面言辞实质论争}
\begin{itemize}
  \item \logicterm{歧义状况}:既存在言辞歧义
  \item \logicterm{歧见性质}:又有实质歧见(态度上或信念上,关于事实或语词运用标准)
  \item \logicterm{解决方法}:需同时解决语言问题和实质分歧
\end{itemize}
\end{theorembox}

\chaptersummary{
论争的类型分析是有效解决分歧的前提,不同类型的论争需要采用不同的解决策略。

\logicemph{三种基本类型}:
\begin{itemize}
  \item \logicterm{明显的实质论争}:不存在言辞歧义,存在实质性歧见,需通过解决信念或态度歧见来解决
  \item \logicterm{纯粹的言辞之争}:存在言辞歧义但无实质歧见,可通过明确定义和消除歧义来解决
  \item \logicterm{表面言辞实质论争}:既有言辞歧义又有实质歧见,需同时解决语言问题和实质分歧
\end{itemize}

\logicemph{识别方法}:
\begin{itemize}
  \item 使用两步诊断流程:先判断是否存在歧义,再判断清除歧义是否能消除对立
  \item 特别注意标准之争和概念之争的复杂性
  \item 威廉·詹姆士的松鼠例子展示了纯粹言辞之争的典型特征
\end{itemize}

\logicwarn{实践意义}:准确识别论争类型有助于避免在虚假争议上浪费时间,将精力集中在真正的分歧上,提高交流和解决问题的效率。
}
\section{定义的类型和论争的解决}

\begin{logicbox}[title=引言]
\textit{定义是解决语言争端的重要工具,理解不同类型的定义及其应用场景,有助于我们更准确地表达思想,避免不必要的争论。}
\end{logicbox}

\begin{theorembox}[title=定义的基本概念与功能]
\logicemph{基本定义}:定义是对词项意义的解说。

\logicemph{核心功能}:
\begin{itemize}
  \item 减少或消除由词项意义的不确定性引起的困难
  \item 减少或消除由词项意义的模糊性引起的困难
  \item 促进清晰的思考和有效的交流
\end{itemize}

\logicemph{两种基本方向}:
\begin{itemize}
  \item \logicterm{阐释既有意义}:解释一个既有词项的既有意义
  \item \logicterm{赋予新意义}:给一个新词项赋予新意义
\end{itemize}
\end{theorembox}

\begin{theorembox}[title=报告与规定的根本区别]
\logicemph{报告性定义}(词典定义):
\begin{itemize}
  \item \logicterm{目的}:报告一种既定的用法
  \item \logicterm{评价标准}:可以是真的或假的
  \item \logicterm{准确性}:报告的意义可以是准确的也可以是不准确的
\end{itemize}

\logicemph{规定性定义}:
\begin{itemize}
  \item \logicterm{目的}:规定一种用法
  \item \logicterm{评价标准}:\logicwarn{不能是真的或假的}
  \item \logicterm{实用性}:规定的意义可以是有用的也可以是无用的,可以是方便的也可以是不方便的
\end{itemize}

\logicwarn{根本区别}:在真假与有用无用之间存在根本区别。
\end{theorembox}

\subsection{报告性定义与规定性定义}

\begin{examplebox}[title=报告性定义的典型例子]
\logicemph{"奇数"的定义}:
\begin{itemize}
  \item \logicterm{定义内容}:"任何不能被2整除的整数"
  \item \logicterm{定义性质}:报告性定义
  \item \logicterm{理由}:报道了"奇数"这个词的既定用法
  \item \logicterm{真假评价}:这个定义是真的
\end{itemize}
\end{examplebox}

\begin{examplebox}[title=规定性定义的典型例子]
\logicemph{"圆方形"的定义}:
\begin{itemize}
  \item \logicterm{定义内容}:"既是圆又是正方形的图形"
  \item \logicterm{定义性质}:规定性定义
  \item \logicterm{理由}:规定这个新词的意义,而不是报道既有意义
  \item \logicterm{实用性评价}:在应用上无用(因为不存在圆方形)
  \item \logicterm{逻辑性评价}:逻辑上不可能(结合了两个不相容的属性)
  \item \logicwarn{真假评价}:既非真也非假
\end{itemize}
\end{examplebox}

\subsection{精确定义的作用}

\begin{theorembox}[title=精确定义的必要性]
\logicwarn{问题识别}:当我们面对歧义时,词项的既定意义并不总是很清楚的。

\logicemph{解决需求}:在这种情况下,仅仅报道词项的既有意义是不够的,需要更精确的定义。
\end{theorembox}

\begin{examplebox}[title="安乐死"的定义演进]
\logicemph{报告性定义}(常见用法):
\begin{itemize}
  \item \logicterm{定义内容}:"有意致病人于死地,而又没有该病人同意的医疗行为"
  \item \logicterm{问题}:这个定义存在模糊性,需要澄清
\end{itemize}

\logicemph{精确定义}:
\begin{itemize}
  \item \logicterm{定义内容}:"经过病人的自由和知情的同意,由医生有意地导致的一个不能忍受持久痛苦的病人无痛苦的死亡"
  \item \logicterm{作用}:澄清词项的模糊性,使标准更为精确
\end{itemize}

\logicemph{功能关系}:
\begin{itemize}
  \item 报告性定义的作用是建立一种标准
  \item 精确定义则使这种标准更为精确
\end{itemize}
\end{examplebox}

\begin{theorembox}[title=精确定义的特殊性质]
\logicemph{与规定定义的区别}:
\begin{itemize}
  \item 精确定义:针对模糊的既有词项
  \item 规定定义:给新词项赋予意义
\end{itemize}

\logicemph{精确定义的约束}:
\begin{itemize}
  \item \logicwarn{非武断性}:精确定义并不是武断的
  \item \logicterm{既有用法}:所定义的是已经具有固定用法的词项
  \item \logicterm{保持原则}:必须尽可能地保持这种固定用法
\end{itemize}

\logicemph{双重考虑}:
\begin{itemize}
  \item 使词项摆脱模糊性
  \item 与现有用法的融贯性
\end{itemize}

\logicemph{评价标准}:精确定义仍然可以被评价为真的或假的,但这要根据它是否符合现有用法而定。
\end{theorembox}

\subsection{规定定义的应用场景}

\begin{theorembox}[title=规定定义的基本特征]
\logicwarn{评价标准}:真或假的问题对于规定定义来说是完全无关的。

\logicemph{应用场景}:
\begin{itemize}
  \item 当一个新词项引进时
  \item 当一个既有词项要在新语境中使用时
\end{itemize}

\logicwarn{根本原因}:规定定义不能是真的或假的,因为它们定义的是一些此前没有定义的词项。
\end{theorembox}

\begin{examplebox}[title=规定定义的科学应用]
\logicemph{物理学例子}:
\begin{itemize}
  \item \logicterm{新词项}:"焦耳"
  \item \logicterm{定义内容}:"1牛顿的力使物体在力的方向上移动1米所做的功"
  \item \logicterm{目的}:表示功或能的单位
\end{itemize}

\logicemph{计算机科学例子}:
\begin{itemize}
  \item \logicterm{词项}:"硬盘"
  \item \logicterm{用法}:用于新型存储装置
  \item \logicterm{性质}:规定性使用
\end{itemize}
\end{examplebox}

\begin{theorembox}[title=规定定义的动机]
\logicemph{常见动机}:
\begin{itemize}
  \item \logicterm{便利性}:为了方便交流和使用
  \item \logicterm{命名需求}:因为有些新事物需要命名
  \item \logicterm{概念简化}:用简短的词来指称复杂的概念
\end{itemize}
\end{theorembox}

\begin{examplebox}[title="因特网"的规定定义]
\logicemph{词项特征}:
\begin{itemize}
  \item \logicterm{新词性质}:相对较新的词
  \item \logicterm{定义内容}:"一个全球性的计算机网络系统,它使用TCP/IP协议族来连接全世界数以百万计的计算机"
  \item \logicterm{定义目的}:用一个简短的词来指称一个复杂的概念
\end{itemize}
\end{examplebox}

\subsection{理论定义与说服定义}

\begin{theorembox}[title=理论定义]
\logicemph{基本特征}:
\begin{itemize}
  \item \logicterm{目的}:出于理论的目的
  \item \logicterm{功能}:旨在阐明一个理论概念
  \item \logicterm{性质}:也是规定性的,赋予词项新的、更专门的意义
\end{itemize}
\end{theorembox}

\begin{examplebox}[title=经济学中的理论定义]
\logicemph{"效用"的理论定义}:
\begin{itemize}
  \item \logicterm{定义内容}:"一种衡量消费者从消费一种商品或服务中获得的满足程度的指标"
  \item \logicterm{理论框架}:经济学理论
  \item \logicterm{目的}:在经济理论框架内阐明"效用"这个概念
\end{itemize}
\end{examplebox}

\begin{theorembox}[title=说服定义]
\logicwarn{基本特征}:
\begin{itemize}
  \item \logicterm{目的}:出于说服的目的
  \item \logicterm{功能}:影响人们的态度或行为
  \item \logicterm{性质}:也是规定性的,赋予词项带有情感色彩的意义
\end{itemize}

\logicwarn{警惕性}:说服定义可能包含偏见和操纵意图。
\end{theorembox}

\begin{examplebox}[title=说服定义的典型例子]
\logicemph{政治领域的例子}:
\begin{itemize}
  \item \logicterm{词项}:"爱国主义"
  \item \logicterm{说服定义}:"对自己国家盲目的、不加批判的忠诚"
  \item \logicterm{目的}:贬低那些持有不同政见的人
\end{itemize}

\logicemph{商业领域的例子}:
\begin{itemize}
  \item \logicterm{词项}:某种产品
  \item \logicterm{说服定义}:"成功的象征"
  \item \logicterm{目的}:吸引消费者购买
\end{itemize}
\end{examplebox}

\subsection{定义在解决论争中的应用}

\begin{theorembox}[title=定义在解决论争中的策略应用]
\logicwarn{基本原则}:在解决论争时,识别和理解不同类型的定义是至关重要的。

\logicemph{针对性解决策略}:
\begin{itemize}
  \item \logicterm{意义分歧}:如果论争是由于对词项意义存在分歧而引起的,通过提供清晰的、双方都能接受的定义来解决
  \item \logicterm{模糊性问题}:如果论争涉及词项的模糊性,精确定义可以帮助澄清问题
  \item \logicterm{新概念理论}:如果论争涉及新的概念或理论,规定定义或理论定义可以帮助阐明相关思想
  \item \logicwarn{说服定义}:如果论争涉及说服定义,需要警惕定义中可能包含的情感偏见
\end{itemize}
\end{theorembox}

\begin{theorembox}[title=定义的工具价值]
\logicemph{核心价值}:定义是解决论争和促进清晰思考的重要工具。

\logicemph{实践意义}:通过仔细地考察和运用不同类型的定义,我们可以更有效地进行交流和推理。
\end{theorembox}

\chaptersummary{
定义作为语言分析和论争解决的核心工具,具有多种类型和功能,理解其特征和应用条件对于有效交流至关重要。

\logicemph{五种主要定义类型}:
\begin{itemize}
  \item \logicterm{报告性定义}:阐释既有词项的既有意义,可评价为真或假,如词典定义
  \item \logicterm{精确定义}:消除既有词项的模糊性,需兼顾与现有用法的融贯性,仍可评价真假
  \item \logicterm{规定性定义}:赋予新词项以意义或赋予既有词项以新意义,不能评价为真或假
  \item \logicterm{理论定义}:为阐明理论概念而提出的规定定义,具有专门的学术用途
  \item \logicterm{说服定义}:为影响态度或行为而提出的带有情感色彩的定义,需要警惕其偏见
\end{itemize}

\logicemph{核心区别}:
\begin{itemize}
  \item 报告性定义与规定性定义的根本区别在于真假评价的适用性
  \item 精确定义介于两者之间,既要保持既有用法又要消除模糊性
  \item 理论定义和说服定义是规定性定义的特殊应用
\end{itemize}

\logicwarn{实践指导}:在论争解决中,必须根据争议的性质选择合适的定义类型,特别要警惕说服定义中的情感操纵。
}
\section{外延和内涵}

\begin{logicbox}[title=引言]
\textit{在理解词项意义时,外延与内涵是两个不可或缺的概念。正确把握二者的关系有助于我们更精确地定义概念、避免语义混淆,从而提高思维和论证的准确性。}
\end{logicbox}

\begin{theorembox}[title=意义概念的复杂性]
\logicemph{基本问题}:定义旨在表明一个词项的\logicterm{意义}(meaning),但是"意义"这个词却有不同\logicterm{含义}(sense)。

\logicemph{理论发展}:
\begin{itemize}
  \item 前面已区分了词项的描述(字面)意义与表达性意义
  \item 现在需要更仔细地考察字面意义,尤其是普遍词项的字面意义
\end{itemize}

\logicemph{普遍词项的重要性}:
\begin{itemize}
  \item \logicterm{定义}:可以运用于多于一个对象的类(class)的词项
  \item \logicterm{重要性}:在推理中,普遍词项的定义是特别重要的
\end{itemize}
\end{theorembox}

\subsection{外延意义与内涵意义}

\begin{examplebox}[title="行星"的外延意义分析]
\logicemph{适用对象}:普遍词项"行星"对水星、金星、地球、火星、土星等都是在同等含义上适用的。

\logicemph{外延构成}:
\begin{itemize}
  \item 在一种含义上,词项"行星"意谓所有这些不同对象
  \item 所有行星的汇集(collection)就构成"行星"的意义
\end{itemize}

\logicemph{逻辑分析}:
\begin{itemize}
  \item 如果我说"所有行星都有椭圆轨道"
  \item 那么我所断定的包括:火星有椭圆轨道、金星有椭圆轨道,等等
  \item 词项"行星"的意义便是由它适用的那些对象构成的
\end{itemize}
\end{examplebox}

\begin{theorembox}[title=外延的基本概念]
\logicemph{外延意义}:词项意义的这种含义被称做\logicterm{外延意义}。

\logicemph{指谓关系}:
\begin{itemize}
  \item 普遍词项或类词项\logicterm{指谓}(denote)其可以正确适用的那些对象
  \item 一个普遍词项可以正确适用的对象的汇集构成那个词项的\logicterm{外延}
\end{itemize}
\end{theorembox}

\begin{theorembox}[title=内涵的基本概念]
\logicemph{理解与使用的区别}:
\begin{itemize}
  \item 理解普遍词项的意义就是知道怎样正确使用它
  \item 但这\logicwarn{并不一定要知道}它可以正确适用的所有对象
\end{itemize}

\logicemph{共同属性的作用}:
\begin{itemize}
  \item 对一个给定词项,其外延内的所有对象具有某些共同的性质或属性
  \item 这些性质或属性引导我们使用同一词项来指谓它们
  \item 因此,我们可以知道一个词项的意义而无须知道其外延
\end{itemize}

\logicemph{内涵意义}:
\begin{itemize}
  \item 在第二种含义上,"意义"设定了决定任一对象是否属于那个词项外延的某种标准
  \item 这种含义被称做词项的\logicterm{内涵意义}
  \item \logicterm{内涵}(intension):普遍词项指谓的所有对象并且仅仅那些对象共同拥有的属性集
\end{itemize}
\end{theorembox}

\subsection{外延与内涵的关系}

\begin{theorembox}[title=普遍词项的双重意义]
\logicemph{基本原理}:每个普遍或类词项都既有一个内涵意义又有一个外延意义。
\end{theorembox}

\begin{examplebox}[title="摩天大厦"的内涵与外延分析]
\logicemph{内涵}:
\begin{itemize}
  \item 包括所有超过一定高度的建筑物的共同和特有性质
  \item 提供了判断标准:什么样的建筑物可以被称为摩天大厦
\end{itemize}

\logicemph{外延}:
\begin{itemize}
  \item 是一个类,包括该词项适用对象的汇集
  \item 具体例子:纽约的世贸中心(World Trade Center)、芝加哥的希尔斯塔(Sears Tower)、上海世界金融中心(Shanghai World Financial Center)、吉隆坡的国油双峰塔(Petronas Twin Towers)等等
\end{itemize}
\end{examplebox}

\begin{theorembox}[title=外延变化的误解与澄清]
\logicwarn{常见误解}:有时人们断言一个词项的外延不时发生变化,尽管它的内涵没有变化。

\logicemph{典型例子}:
\begin{itemize}
  \item \logicterm{错误观点}:词项"人"的外延,正如人的死亡和婴儿的降生一样,持续变化
  \item \logicterm{混淆根源}:这个说法源于一种概念混淆
\end{itemize}

\logicemph{正确分析}:
\begin{itemize}
  \item 词项"人"用来指谓所有的人,包括死去的以及尚未出生的
  \item 它并没有一个不确定的外延
  \item \logicwarn{真正变化的}是词项"活着的人"的外延
\end{itemize}

\logicemph{深层原理}:
\begin{itemize}
  \item "活着的人"这个词项的外延具有"现在活着的人"这种含义
  \item 其中"现在"这个词是指不断变化的现时
  \item 因此,词项"活着的人"的内涵在不同的时候也是不同的
\end{itemize}

\logicwarn{重要结论}:任何具有变化外延的词项必定也有一个变化的内涵,二者是同等恒定的。
\end{theorembox}

\subsection{内涵决定外延而非相反}

\begin{theorembox}[title=内涵与外延的决定关系]
\logicemph{基本原理}:当一个词项的内涵固定下来时,它的外延也就固定了。

\logicwarn{重要原则}:词项的外延由它的内涵决定,但是反过来说却不对。
\end{theorembox}

\begin{examplebox}[title=等边三角形与等角三角形的对比分析]
\logicemph{"等边三角形"的分析}:
\begin{itemize}
  \item \logicterm{内涵}:由三条等长的直线所围成的平面图形的性质
  \item \logicterm{外延}:所有那些并且仅仅那些具有这种性质的对象的类
\end{itemize}

\logicemph{"等角三角形"的分析}:
\begin{itemize}
  \item \logicterm{内涵}:由三条相互相交而形成等角的直线所围成的平面图形的性质
  \item \logicterm{外延}:与"等边三角形"的外延完全相同
\end{itemize}

\logicwarn{关键发现}:两个词项具有不同的内涵,但外延却相同。
\end{examplebox}

\begin{theorembox}[title=决定关系的逻辑分析]
\logicemph{逻辑推论}:
\begin{itemize}
  \item 确认了这些词项其中一个词项的外延,而它的内涵却处于不确定状态
  \item \logicwarn{外延不决定内涵},但是,\logicterm{内涵却必定决定外延}
\end{itemize}

\logicemph{重要结论}:
\begin{itemize}
  \item 词项可以具有\logicterm{不同的内涵但外延却相同}
  \item 而具有\logicterm{不同外延的词项却不可能有同样的内涵}
\end{itemize}
\end{theorembox}

\subsection{内涵与外延的反变关系}

\begin{theorembox}[title=内涵增加的定义]
\logicemph{基本概念}:当给一个词项的内涵添加性质时,我们就说该内涵增加了。
\end{theorembox}

\begin{examplebox}[title=内涵与外延反变关系的典型例子]
\logicemph{词项序列}(按内涵增加次序排列):
\begin{enumerate}
  \item "人"
  \item "活着的人"
  \item "活着的二十岁以上的人"
  \item "活着的二十岁以上有红发的人"
\end{enumerate}

\logicemph{内涵分析}:
\begin{itemize}
  \item 每个词项的内涵都包含其后相随的词项的内涵
  \item 每个词项的内涵都比其前的那些词项的内涵多
  \item 这些词项是按照内涵增加的次序来排列的
\end{itemize}

\logicemph{外延分析}:
\begin{itemize}
  \item 如果我们倒过来看这些词项的外延,就会发现情况相反
  \item "人"的外延比"活着的人"的外延大,等等
  \item 这些词项是按照外延减少的次序排列的
\end{itemize}
\end{examplebox}

\begin{theorembox}[title=反变规律的修正]
\logicwarn{传统观点}:有些逻辑学家得出一条公式化的\logicterm{"反变规律"},断言外延与内涵总是反向变化。

\logicemph{评价}:这种断言具有启发性,但并不完全正确。
\end{theorembox}

\begin{examplebox}[title=反变规律的反例]
\logicemph{反例序列}(按内涵增加次序排列):
\begin{enumerate}
  \item "活着的人"
  \item "活着的有脊骨的人"
  \item "活着的有脊骨的不超过一千岁的人"
  \item "活着的有脊骨的不超过一千岁的没有读完国会图书馆里所有书的人"
\end{enumerate}

\logicemph{分析结果}:
\begin{itemize}
  \item 这些词项的次序是增加内涵
  \item 但是它们每个的外延都是相同的,完全没有减少
\end{itemize}
\end{examplebox}

\begin{theorembox}[title=修正的反变规律]
\logicemph{正确表述}:如果词项按照内涵增加的次序排列,那么它们的外延将处于\logicterm{非递增的次序}。

\logicemph{精确含义}:也就是说,如果外延变化,那么它们将是沿着内涵的反向变化。

\logicwarn{重要说明}:外延可能减少,也可能保持不变,但不会增加。
\end{theorembox}

\subsection{外延为空的词项与意义歧义}

\begin{theorembox}[title=空外延词项的存在]
\logicemph{重要事实}:有些词项的外延,例如"独角兽"的外延,可能是空的。

\logicemph{理论价值}:认识到这一点,并运用我们对内涵与外延的区分,就可以把玩弄"意义"歧义的谬误论证揭露出来。
\end{theorembox}

\begin{examplebox}[title=上帝存在论证的谬误分析]
\logicemph{谬误论证}:
\begin{displayquote}
"上帝"这个词不是无意义的,因此它有意义。但是按照定义,"上帝"这个词的意思是全能的至善的存在(being)。因此,全能的至善的存在,即上帝,必然存在(exist)。
\end{displayquote}

\logicwarn{歧义分析}:
\begin{itemize}
  \item 歧义在于"意义"和"无意义"这两个词
  \item "意义"在一种含义上指的是内涵,在另一种含义上指的却是外延
\end{itemize}

\logicemph{正确分析}:
\begin{itemize}
  \item "上帝"这个词不是无意义的,因此可以肯定,存在一个内涵是它的意义
  \item 但是,由此并不能得出:一个具有内涵的词项,其内涵一定指谓一个存在物\cite{gombocz1997}
\end{itemize}
\end{examplebox}

我们在下面这个语段中也发现了一个类似的谬误:

\begin{examplebox}[title=乌托邦论证的谬误分析]
\logicemph{另一个类似谬误}:
\begin{displayquote}
kitsch (低劣作品) 以展示粗鄙、卑劣、下贱、脆弱和邪恶信仰来表现并败坏人类境况。这就是乌托邦之所以能被定义为kitsch 这一词项已消失的状况的原因, 因为在乌托邦中该词项已没有所指了。\cite{sisk1988}
\end{displayquote}

\logicwarn{谬误根源}:作者没能在意义与所指(referent)之间做出区分。

\logicemph{正确观点}:
\begin{itemize}
  \item 许多有价值的词项(例如,那些命名希腊神话中的动物的词项)都不存在所指
  \item 但是,我们并不要求或期望这样的词项消失
  \item 具有内涵但没有外延的词项是非常有用的
\end{itemize}

\logicemph{实用价值}:如果有一天乌托邦变成了现实,我们也许想要表达对减少或消除"低劣作品"或"粗鄙"等的庆幸。而要这样做,我们就需要能够有意义地使用这些词项。
\end{examplebox}

\begin{theorembox}[title=章节过渡]
\logicemph{前面内容回顾}:在前面的几节中,我们考察了定义的种类和它们的用途:
\begin{itemize}
  \item 词典定义和规定定义可消除或避免歧义
  \item 精确定义可以减少模糊性
\end{itemize}

\logicemph{后续内容预告}:在随后的几节中,我们将考察构建定义的方法:
\begin{itemize}
  \item 有些定义通过外延或所指来处理普遍词项
  \item 其他定义则通过内涵来处理
  \item 我们将会看到,每种处理方法都既有优点又有缺点
\end{itemize}
\end{theorembox}

\chaptersummary{
外延与内涵是理解词项意义的两个基本维度,它们之间的关系揭示了语言和思维的深层结构。

\logicemph{基本概念}:
\begin{itemize}
  \item \logicterm{外延}:指词项可以正确适用的所有对象的集合,体现了词项的指谓范围
  \item \logicterm{内涵}:指词项所表示的属性或特征的集合,提供了判断标准
  \item 每个普遍词项都既有外延意义又有内涵意义
\end{itemize}

\logicemph{决定关系}:
\begin{itemize}
  \item \logicwarn{内涵决定外延},而非相反:当内涵固定时,外延也就固定了
  \item 词项可以有相同外延但内涵不同(如"等边三角形"与"等角三角形")
  \item 具有不同外延的词项不可能有同样的内涵
\end{itemize}

\logicemph{反变规律}:
\begin{itemize}
  \item 当词项内涵增加时,其外延处于非递增状态(可能减少或保持不变,但不会增加)
  \item 传统的"反变规律"过于绝对,修正版本更为准确
\end{itemize}

\logicemph{空外延现象}:
\begin{itemize}
  \item 词项可以有内涵而无外延(如"独角兽"),这不影响词项的有意义性
  \item 理解这一点有助于识别和避免关于"意义"的歧义谬误
  \item 具有内涵但没有外延的词项在语言中是非常有用的
\end{itemize}

\logicwarn{理论意义}:外延与内涵的区分为构建不同类型的定义提供了理论基础,是逻辑学和语言哲学的重要工具。
}
\section{外延定义}

\begin{logicbox}[title=引言]
\textit{外延定义是通过列举词项所适用的对象来界定其意义的方法。尽管这种定义方式直观明了,但它存在一定的局限性,理解这些局限性有助于我们更合理地运用外延定义。}
\end{logicbox}

\begin{theorembox}[title=外延定义的基本概念]
\logicemph{基本定义}:\logicterm{外延定义}就是指被定义的普遍词项所适用对象的汇集。

\logicemph{操作方法}:告诉某人词项外延的最方便有效的方法就是列举出其指谓的那些对象。

\logicwarn{重要提醒}:我们须认清这种方法的局限性。
\end{theorembox}

\subsection{外延定义的基本局限}

\begin{theorembox}[title=局限一:相同外延的不同内涵]
\logicemph{理论基础}:上一节我们曾指出(以"等边三角形"和"等角三角形"为例),具有不同意义即不同内涵的两个词项可以具有恰好相同的外延。

\logicwarn{问题后果}:
\begin{itemize}
  \item 即使我们能够完全列举出其中一个词项指谓的对象
  \item 由此而得出的外延定义也不能把它与另一个指谓同样对象的词项区分开来
  \item 这样的两个词项不是同义词,但外延定义却不能在它们之间做出区分
\end{itemize}
\end{theorembox}

\begin{theorembox}[title=局限二:完全列举的不可能性]
\logicwarn{更严重的问题}:这还不是问题的棘手之处,因为外延可以完全列举出来的词项是极少的。

\logicemph{典型例子}:
\begin{itemize}
  \item \logicterm{"数"}:要列举出这个词所指谓的所有数字是完全不可能的
  \item \logicterm{"恒星"}:要列举出这个词所指谓的天文数字的对象,实际上也是绝对不可能的
  \item \logicterm{其他大多数普遍词项}:完全列举其外延都是不可能的
\end{itemize}
\end{theorembox}

\subsection{部分列举的困难}

\begin{theorembox}[title=局限三:部分列举的严重困难]
\logicwarn{现实约束}:外延定义通常就要限定于所指谓对象的\logicterm{部分列举},这是个引起严重困难的局限。

\logicemph{根本问题}:任何给定对象都具有许许多多性质,因而被包括在许许多多不同的普遍词项的外延之中。
\end{theorembox}

\begin{examplebox}[title=约翰·多伊的多重身份问题]
\logicemph{问题分析}:
\begin{itemize}
  \item 当约翰·多伊(John Doe)作为一个词项的外延定义实例而被给出时
  \item 在许多其他词项的外延定义中,他也可以作为实例而被同样恰当地提及
\end{itemize}

\logicemph{多重归属}:约翰·多伊是以下词项的实例:
\begin{itemize}
  \item "人"、"动物"、"哺乳动物"
  \item 或许也是"丈夫"、"父亲"和"学生"等等
\end{itemize}

\logicwarn{区分失效}:因此提到他,并不能帮助我们在这些词项的意义之间做出区分。即使我们给出两个、三个或更多实例,也会遇到同样的困难。
\end{examplebox}

\begin{examplebox}[title=摩天大厦定义的歧义问题]
\logicemph{定义尝试}:在定义"摩天大厦"这个词时,我们可以使用明显的范例:
\begin{itemize}
  \item 帝国大厦(Empire State Building)
  \item 克莱斯勒大厦(Chrysler Building)
  \item 沃尔华斯大厦(Woolworth Building)
\end{itemize}

\logicwarn{歧义问题}:这三座大厦作为实例同样也完全可以作为以下词项的外延:
\begin{itemize}
  \item "二十世纪的伟大建筑"
  \item "曼哈顿的昂贵房地产"
  \item "纽约城的地面标志物"
\end{itemize}

\logicemph{更严重的问题}:每个这些普遍词项都指谓其他词项不指谓的对象,因此通过使用部分列举,我们甚至不能在具有不同外延的词项之间做出区分。
\end{examplebox}

\begin{theorembox}[title=反面事例的局限性]
\logicemph{改进尝试}:引进\logicterm{"反面事例"},例如"不是泰姬陵"(Taj Mahal)或"不是五角大楼"(the Pentagon),可以帮助说明被定义项的意义。

\logicwarn{根本局限}:但是,否定事例也必定是不完善的,这个基本局限仍然存在。
\end{theorembox}

\subsection{通过子类列举的方法}

\begin{theorembox}[title=子类列举方法]
\logicemph{方法改进}:我们可以设法换一种列举方式,即不是每次列举一例,而是列举一整组例子。

\logicemph{定义方式}:使用这种方法,也就是通过\logicterm{子类来定义},有时可能做到完全的列举。
\end{theorembox}

\begin{examplebox}[title=脊椎动物的子类定义]
\logicemph{定义实例}:我们把"脊椎动物"定义为:
\begin{itemize}
  \item 两栖动物
  \item 鸟类
  \item 鱼类
  \item 哺乳动物
  \item 爬行动物
\end{itemize}

\logicemph{优势}:这种方法实现了完全列举,避免了遗漏。
\end{examplebox}

\begin{theorembox}[title=列举定义的总体评价]
\logicwarn{逻辑不充分性}:通过列举定义,无论完全列举还是部分列举,无论是列举类的个体元素还是列举子类,虽然都具有某些心理上的用处,但是要完全确定被定义项的意义,在逻辑上都是不充分的。

\logicemph{心理价值}:尽管逻辑上不充分,但在帮助理解方面仍有价值。
\end{theorembox}

\subsection{实指定义及其局限}

\begin{theorembox}[title=实指定义的基本概念]
\logicemph{基本定义}:用列举法下定义的一个特殊类型被称做\logicterm{"实指定义"}或\logicterm{"示范定义"}。

\logicemph{与一般外延定义的区别}:
\begin{itemize}
  \item \logicterm{实指定义}:通过用手或其他姿势指着对象来定义
  \item \logicterm{一般外延定义}:通过命名或描述被定义词项指谓的对象来定义
\end{itemize}
\end{theorembox}

\begin{examplebox}[title=实指定义的典型例子]
\logicemph{定义表述}:"'桌子'这个词意指这",伴随着一个姿势如用手指指着桌子的方向。

\logicemph{特征}:直接的物理指示行为构成了定义的核心部分。
\end{examplebox}

\begin{theorembox}[title=实指定义的双重局限性]
\logicwarn{局限性来源}:实指定义既有其自身的某些特殊局限性,也有前面所提到的各种局限性。
\end{theorembox}

\begin{theorembox}[title=局限一:地域限制]
\logicwarn{显而易见的地域局限}:一个人只能指着看得见的东西。

\logicemph{典型例子}:
\begin{itemize}
  \item 他不能在乡村来实指地定义"摩天大厦"
  \item 也不能在内陆山谷中去实指地定义"海洋"
\end{itemize}

\logicemph{根本原因}:物理指示需要对象的在场和可见性。
\end{theorembox}

\begin{theorembox}[title=局限二:指示歧义]
\logicwarn{更为严重的问题}:姿势也有着不可避免的歧义。

\logicemph{歧义的复杂性}:指着一张桌子也是指着:
\begin{itemize}
  \item 它的一部分
  \item 它的颜色、大小、形状、质料等等
  \item 位于桌子所在的方向上的所有东西
  \item 包括它后面的墙壁或者更远处的花园
\end{itemize}

\logicwarn{根本问题}:指示行为本身具有内在的多义性。
\end{theorembox}

\begin{theorembox}[title=准实指定义的改进尝试]
\logicemph{解决方案}:这种歧义有时可以通过给定义项增加一些描绘的短语而得到解决,其结果被称做\logicterm{准实指定义}。

\logicemph{改进例子}:"桌子"这个词意指"这件"家具(伴随以相应的姿势)。

\logicwarn{新问题}:因为这种附加假设了对"家具"这个短语的事先理解,就使实指定义的宗旨难以达到。
\end{theorembox}

\subsection{实指定义的非原初性}

\begin{theorembox}[title=关于基本性的误解]
\logicwarn{常见误解}:实指定义历来被某些人视为"基本"或"原初"定义,其意思是说:我们最初都是凭借这种方式来理解词项意义的,而其他定义都依赖于这种理解。

\logicemph{错误性质}:但是,这种原初性断言是错误的。
\end{theorembox}

\begin{theorembox}[title=姿势理解的前提性]
\logicwarn{根本依赖}:人们必须理解姿势本身的意义。

\logicemph{婴儿实验的启示}:
\begin{itemize}
  \item 当我们用手指指向婴儿床沿时,如果那个婴儿的注意力被吸引
  \item 其注意力可能放到被指向的东西上,也可能放到我们的手指上
  \item 如果我们想用其他姿势来定义一个姿势,也会出现同样的困难
\end{itemize}

\logicwarn{符号理解的循环性}:如果你要理解任何符号(sign)的定义,那么某些符号就必须已经得到理解。

\logicemph{语言学习的真实途径}:我们学习使用语言的根本途径是经过观察和模仿,而不是经过定义。
\end{theorembox}

\begin{theorembox}[title=术语使用的混淆]
\logicwarn{宽泛解释的问题}:一如有些逻辑学家所做的那样,人们可以很宽泛地解说"实指定义"这个术语,甚至包括"当这个词指谓的对象出现时,不断地听到这个词"的过程。

\logicemph{概念澄清}:
\begin{itemize}
  \item 这种过程根本不是定义
  \item 与我们这里对词项"定义"的使用不一样
  \item 它是学习使用语言的基本的和前定义的(predefinitional)方式
\end{itemize}
\end{theorembox}

\subsection{空外延词项的问题}

\begin{theorembox}[title=空外延词项的根本挑战]
\logicwarn{最终局限}:虽然有些词意义丰富,但是它们完全不指谓任何事物,因此不能从外延上定义它们。

\logicemph{典型例子}:当我们说不存在独角兽时,我们在断定"独角兽"这个词没有所指,具有一个\logicterm{"空"外延}。
\end{theorembox}

\begin{theorembox}[title=空外延词项的理论意义]
\logicemph{双重启示}:这类词项不仅仅展示了外延定义的局限性,也显示出"意义"的确更适用于内涵而不是外延。

\logicemph{逻辑分析}:
\begin{itemize}
  \item 虽然"独角兽"这个词的外延是空的,但由此肯定不是说它就是无意义的
  \item 的确,它不指谓任何事物,因为根本就没有独角兽
  \item 但是,如果"独角兽"这个词项毫无意义,那么说"不存在独角兽"也就是无意义的
  \item 然而,这个陈述并不是没有意义的,我们完全理解它的意义,而且它是真的
\end{itemize}

\logicwarn{理论结论}:显然,内涵对定义来说是真正的关键所在,我们下一节即转向讨论内涵。
\end{theorembox}

\chaptersummary{
外延定义作为通过列举对象来说明词项意义的方法,虽然直观易懂,但存在多重根本性局限。

\logicemph{外延定义的基本方法}:
\begin{itemize}
  \item \logicterm{完全列举}:列举词项所适用的所有对象
  \item \logicterm{部分列举}:列举词项的部分适用对象
  \item \logicterm{子类列举}:通过列举子类而非个体对象
  \item \logicterm{实指定义}:通过指示动作来定义词项
\end{itemize}

\logicemph{系统性局限分析}:
\begin{itemize}
  \item \logicwarn{完全列举的不可能性}:大多数词项无法完全列举其所有适用对象,如"数"、"恒星"等
  \item \logicwarn{部分列举的歧义性}:难以准确区分不同词项,同一对象可能是多个不同词项的实例
  \item \logicwarn{相同外延的不同内涵}:具有相同外延的词项可能有不同内涵,外延定义无法区分
  \item \logicwarn{实指定义的特殊问题}:地域限制、指示歧义、非原初性等问题
  \item \logicwarn{空外延词项}:某些有意义的词没有所指对象,根本无法通过外延定义
\end{itemize}

\logicemph{理论启示}:
\begin{itemize}
  \item 外延定义虽然具有心理上的用处,但在逻辑上是不充分的
  \item 空外延词项的存在表明"意义"更适用于内涵而不是外延
  \item 内涵对定义来说是真正的关键所在
\end{itemize}

\logicwarn{过渡意义}:这些局限性的分析为转向内涵定义提供了充分的理论依据。
}
\section{内涵定义}

\begin{logicbox}[title=引言]
\textit{内涵定义通过词项所表示的属性或特征来解释其意义,是定义概念的核心方法。理解各种内涵定义的方式有助于我们更准确地表达概念,并有效地解决因概念混淆而产生的争议。}
\end{logicbox}

如前所述,词项的内涵,由词项指谓的所有对象共有且仅为这些对象特有的属性构成。例如,如果"椅子"的内涵由属性"单个的座位并且有一个靠背"构成,那么就意味着每一张椅子都是具有靠背的单个座位,并且只有椅子才是具有靠背的单个座位。

\subsection{内涵的三种含义}

但是,内涵定义这个概念被三种不同含义的内涵区别弄得复杂了,这三种不同含义的内涵是\textbf{主观内涵}、\textbf{客观内涵}和\textbf{规约内涵}。

对说话者来说,词的主观内涵就是他认为该词指谓对象所具有的属性集。这种集合显然是因人而异的,甚至对同一个人也因时而异。毕竟,我们所感兴趣的是词语的公共意义,而不是它们的私人解释。

客观内涵是词项外延的所有对象共同拥有的属性全集。例如,"圆"这个词的客观内涵可以拥有圆的各种普遍特性(例如,圆包围的面积比其他任何封闭的与其具有相等周长的平面图包围的面积都大),而我们很多人在运用这个词时完全没有注意到这些普遍属性。要知道大多数词项的指谓对象所共同拥有的全部属性,就要求完完全全的全知,而由于没有人能够具有这样的全知,所以客观内涵就不是我们所追求的公共意义的解释。

然而,对大多数普遍词项来说,很显然,必定存在可为公众使用并广泛理解的内涵,即既不是主观的也不是客观的内涵。因为我们确实可以相互交流,而且我们的确常常是在一般的用法上来理解词项的。词项之所以具有稳定的意义,乃是因为对任何对象来说,在决定其是否某词项外延的一部分时,我们都同意使用同样的标准。比如,从规约的角度看,按照 "圆"这个词的通常用法,圆之所以为圆,就在于它是这样一种封闭的平面曲线,其线上所有的点到一个叫做圆心的点的距离都相等。通过非正式的承诺,我们建立了普遍词项的规约内涵。就定义之目的而言,这是内涵的最为重要的含义,因为它既是公共的,也不为使用它而要求全知。实际上,"内涵"这个词通常就是用来指"规约内涵"的——这也将是我们的用法,除非另有说明。

\subsection{识别规约内涵的方法}

人们实际上是怎样定义一个词的呢?识别词的规约内涵,即这个词的指谓对象为人们认同的共同与特有属性,要运用哪些方法呢?常用的方法有如下几种。

\subsection{同义定义}

最简单且最常用的方法(但功能有限)就是提供另一个意义已经被理解的词,而且它与被定义的词具有相同的意义。两个具有相同意义的词被称做\textbf{"同义词"},因此这种定义就被称做\textbf{同义定义}。词典,尤其是较小的词典,就主要依靠这种方法来定义词项。例如,袖珍词典可以将"谚"(adage)定义为"谚语"(proverb),"腼腆"(bashful)定义为"害羞"(shy),等等。当需要解释另一种语言的词义时,同义定义特别有用,往往是不可或缺的。在法语中,"chat"意指"猫";在西班牙语中,"amigo"意指"朋友",等等。人们学习外语词汇要依赖于同义定义。

同义定义是一种定义词项的好方法,它容易、方便而实用,但它也有很大局限性。很多词汇并没有真正的同义词,因而同义定义就常常不够完全精确并引人误解。有一句意大利谚语就是基于这种认识而来的:翻译者就是窜改者。

同义定义的一个更严重的局限是:如果我们寻求定义的词所表示的概念对我们来说完全是外来的和令人费解的,那么,其任何简单的同义词都将像被定义项本身一样令人费解。例如,需要"tylotoxea"的定义而又完全不熟悉这个名词指什么的人,当给出的定义是一个简单的同义词时,就不会有多大帮助。你问道,什么是"tylotoxea"?哦,它不是别的,就是 tylostyle。显然,在这个事例中,还需要提供比单独一个词所能提供的更多解释。当要给一个为人所知的但又模糊不清的词项提供一个比较精确或充分的定义时,也会出现同样的困难。因此,当寻求的是一个理论定义或精确定义(3.2节已解释过)时,同义词是不可能满足要求的。

\subsection{操作定义}

\textbf{"操作定义"}(operational definition),是诺贝尔奖得主物理学家 P.W.布里奇曼在他那本有影响的《现代物理学的逻辑》(1927)一书中首次使用的一个术语;为了把被定义项与一组可描述的动作或操作联系在一起,一些科学家就引进了它。例如,在爱因斯坦的相对论获得成功并被广泛接受之后,"空间"和"时间"就不能再按照牛顿所用的那种抽象方式来定义了。于是,有人提出操作地定义它们,即以在测量距离和时间中所使用的操作方法来定义之。词项的操作定义就是指这个词项被正确地运用到某个给定场合,当且仅当在那个场合中,特有的操作行为会产生特有结果。于是,给定的长度数值就可以通过参考特有测量程序的结果而操作地定义出来,如此等等。在操作定义中,仅仅涉及公共的可重复的操作。

有些社会科学家也试图把这种定义方法结合到他们的研究领域中去以避免混淆和分歧,而这些混淆和分歧已经使一些关键术语的传统定义备受质疑。例如,有些心理学家已经寻求用仅仅涉及行为或者心理学的观察的操作定义来替代"感觉"和"心灵"的抽象定义;在心理学和其他社会科学中,依靠操作定义已经有了与行为主义相联系的倾向。有的极端经验主义者坚持认为,一个词项有意义仅当它能够操作定义;但评价这种断言已超出了本书的范围。

\subsection{属加种差定义}

在不可用同义定义也不适合用操作定义的地方,我们通常可以使用\textbf{"属加种差定义"}来解释一个词项的规约内涵。这种方法也被称做"划分定义"、"分析定义"、"属种定义",或者直接称做"内涵定义"。有人错误地认为这是唯一的一种"真实"定义,但这种方法确实比任何其他方法都有更广泛的可应用性。

通过属加种差来定义词项的可能性取决于有很多属性的复杂事实,即它们可以分析为两个或更多的其他属性。这种复杂性和可分析性可以依据类这个概念而得到最好的说明。

\subsection{属和种的关系}

具有多个元素的类可以把它们的元素分为子类。例如,所有三角形这个类可以分为三个非空的子类:等边三角形、二等边的三角形和不等边三角形。\textbf{"属"}和\textbf{"种"}这两个词常常用于这种关系:被分为子类的类是属,而各种各样的子类都是种。就我们这里的用法而言,"属"与"种"这两个词是相对的,就像"父母"与"子女"一样。在关系上,相同的人是他们孩子的父母亲,但又是他们自己父母亲的子女;同样,同一个类在关系上可以是它的子类的属,也可以是它所从属的更大类的一个种。这样,所有三角形的类,相对于不等边三角形这个种就是一个属,而相对于多边形这个属它则是一个种。逻辑学家对"属"和"种"这两个词作为相对术语的用法,与生物学家把它们作为严格术语的用法是不同的,我们不应当混淆二者。

由于一个类就是具有某些共同特征事物的一个汇集,所以给定的属的所有元素都具有某些共同特征。例如,多边形这个属的所有元素都具有这样的特征,即由直线线段连接而成的封闭平面图形。这个属可以分成不同的种或子类,因此每个子类的所有元素都具有更进一步的共同属性,而这些共同属性却不为任何其他子类的元素所共享。多边形这个属分为三角形、四边形、五边形和六边形等等。多边形这个属的每个种都与其他所有的种不同;六边形这个子类的元素与任何其他子类的元素之间的特有差异是,仅有六边形这个子类的元素恰好具有六条边。一般的,一个给定的属的所有种的元素共享某些属性,这些共享属性使它们成为该属的元素;但是,任何一个种的元素都共享某些更进一步的属性,而这些属性将它们与该属的任何其他种的元素区分开来。那种用来区分它们的性质叫\textbf{"种差"}。如,具有六条边就是六边形这个种与多边形这个属的所有其他种之间的种差。

\subsection{属加种差定义的应用}

在这个含义上,六边形的属性可以分析为多边形的属性和六条边的属性。对于不知道"六边形"这个词或任何它的同义词的意义,但又的确知道"多边形"、"边"和"六"等词的意义的人来说,"六边形"这个词的意义就可以用"属加种差"的定义而得到解释:

\begin{displayquote}
"六边形"这个词意思是"具有六条边的多边形"。
\end{displayquote}

另一个例子是"质数"的定义:

\begin{displayquote}
质数就是任何大于 1 而且又仅能为它自己或 1 整除的自然数。
\end{displayquote}

可见,通过属加种差来定义一个词项要经过两步:首先,必须找出一个属,即包括被定义的那个种的较大的类;接着必须找出种差,即将被定义的那个种的元素与那个属的其他所有种的元素区分开来的性质。在上例中,属就是一个比 1 大的自然数的类:$2,3,4$,等等;其种差是仅能为它自己或 1 整除的性质。属加种差定义可以非常简明并且常常是极其有用的。

属加种差定义的另一个例子是古代人将"人"的意义定义为"有理性的动物"。这里,属是"动物","人"是其下的种,通过理性把人与其他所有的种区别开来。在这种情况下,人们也可以把"所有的有理性的生物"类看做属,而把"动物"类看做种差。对于把某个类而不是把另一个类看做属来说,虽然可能存在超逻辑的原因,但从逻辑的观点看,这种顺序并不是绝对的。

\subsection{属加种差定义的局限性}

属加种差来定义的方法也有其局限性。首先,这种方法仅能运用于那些暗含有复杂属性的词汇。如果是简单得不可再分析的属性,那么暗含这些属性的词汇就不能由属加种差来定义。有人提出,人们所感知的具体光谱段的颜色性质就是这种简单属性的范例。是否存在这样不可再分析的属性仍然是一个未解决的问题,但是如果这种属性存在,那就限制了属加种差定义的运用。第二种局限性与表达"大全"(universal)性质的词汇有关,如"存在"、"本体"、"存在物"和"客体"等。这些词都不能通过属加种差的方法来定义;例如,所有本体的类就不是某个更大的属的一个种;大全类(universal class)是最高的类,或者有人所谓的最高的属。这同样适用于那些指称形而上学的最终范畴的词汇,诸如"物质"或"性质",等等。然而从这种定义方法的实际运用角度看,这些局限性不是很重要的。

内涵定义,尤其属加种差定义,可以满足构造定义的各种目的:它们可以帮助人们消除歧义、减少模糊、阐释理论,甚至影响态度。它们也可用于增加和丰富人们的词汇。在 3.2 节,我们注意到,在达到这些不同目标的过程中,要区分五种不同的定义:词典定义、规定定义、精确定义、理论定义和说服定义。对这些种类的每一个来说,都可以运用内涵定义的方法。

\begin{center}
\begin{tabular}{|l|l|}
\hline
\multicolumn{2}{|c|}{五种定义} \\
\hline
\multicolumn{2}{|l|}{\begin{tabular}{l}
1.规定定义 \\
2.词项定义 \\
3.精确定义 \\
4.理论定义 \\
5.说服定义 \\
\end{tabular}} \\
\hline
\multicolumn{2}{|c|}{定义词项的六种方法} \\
\hline
\begin{tabular}{l}
A.外延方法 \\
1.示例定义 \\
2.实指定义 \\
3.准实指定义 \\
\end{tabular} & \begin{tabular}{l}
B.内涵方法 \\
4.同义定义 \\
5.操作定义 \\
6.属加种差定义 \\
\end{tabular} \\
\hline
\end{tabular}
\end{center}

\chaptersummary{

  内涵的三种含义:主观内涵(个人理解)、客观内涵(全部共有属性)、规约内涵(公共约定);\\
  同义定义:通过具有相同意义的词来定义,简单但局限性大;\\
  操作定义:通过可观察、可重复的操作来定义,在科学中广泛应用;\\
  属加种差定义:通过属(上位类)和种差(区别性特征)来定义,应用最广;\\
  局限:不能定义简单不可分析的属性和表示大全性质的词汇。

} 
\section{属加种差定义的五条规则}

\begin{logicbox}[title=引言]
\textit{属加种差定义作为最常用的内涵定义方法,遵循一套传统规则。这些规则不仅有助于构建清晰、准确的定义,也是评价已有定义质量的重要标准。理解这些规则能够帮助我们避免定义中的常见错误。}
\end{logicbox}

前一节我们详细讨论了属加种差定义的本质和应用。为了确保这种定义方法发挥最大效用,逻辑学家们总结了五条传统规则,用以指导属加种差定义的构建和评价。这五条规则既是构建定义的指南,也是判断定义优劣的标准。

\subsection{规则一:定义应当揭示种的本质属性}

第一条规则要求,定义揭示的应当是被定义事物的\textbf{本质属性},而非偶然属性。本质属性是事物必然具有的、缺少它事物就不成其为此种事物的属性。例如,人的理性能力是人的本质属性,而身高、体重则是偶然属性。因此,将"人"定义为"有理性的动物"比定义为"会使用工具的生物"更符合这一规则。

本质属性的选择不仅取决于事物本身,也取决于定义的目的。在科学定义中,本质属性往往是事物最深层次的特征,而在日常生活中,则可能是事物最易识别的特征。无论哪种情况,定义都应当努力揭示事物区别于其他事物的最重要、最根本的特征。

偶然地使用一种不是本质的属性可能会导致定义失去普遍性和稳定性。例如,如果我们将"大学生"定义为"穿校服的年轻人",就没有捕捉到大学生的本质特征,因为不是所有大学生都穿校服,而且有些非大学生也可能穿校服。

\subsection{规则二:定义不能循环}

第二条规则禁止在定义中直接或间接地使用被定义项本身。这种错误被称为\textbf{循环定义}。例如,"教育是教育人的活动"就是一个明显的循环定义,因为它在定义"教育"时又使用了"教育"这个词。

间接的循环定义更加隐蔽,例如:"老师是教育者,教育者是从事教学的人,教学的人是老师。"在这个定义链条中,"老师"最终通过一系列中间环节被定义为它自己。

循环定义之所以是错误的,是因为它没有提供任何新的信息,无法帮助人们理解被定义项的意义。良好的定义应当使用已知的、更基本的概念来解释未知的概念,而不是用未知解释未知。

\subsection{规则三:定义既不能过宽又不能过窄}

第三条规则要求定义的外延必须与被定义词项的外延完全一致,即定义不能太宽也不能太窄。

当定义\textbf{过宽}时,定义包含了不属于被定义项的对象。例如,将"鸟"定义为"会飞的动物"就过于宽泛,因为蝙蝠、昆虫等非鸟类动物也会飞。这种定义违反了第三条规则,因为它使用的种差(会飞)不足以将鸟与其他动物区分开来。

当定义\textbf{过窄}时,定义排除了属于被定义项的一些对象。例如,将"鸟"定义为"会飞的有羽毛的脊椎动物"就过于狭窄,因为企鹅、鸵鸟等不会飞的鸟类被排除在外。这种定义同样违反了第三条规则。

良好的定义应当使用足够特定的种差,使得定义项的外延与被定义项的外延完全一致。例如,"鸟是有羽毛、下颌骨形成喙的脊椎动物"就更为准确,因为它既包含了所有鸟类,又排除了所有非鸟类动物。

\subsection{规则四:定义不能用歧义的、晦涩的或比喻的语言}

第四条规则要求定义使用清晰、精确、直接的语言,避免使用歧义的、晦涩的或比喻性的表达。

\textbf{歧义}指的是一个词或短语有多种可能的解释。例如,将"银行"定义为"由河边延伸的倾斜地形或处理金钱的机构"就是歧义的,因为它没有明确指出是哪一种"银行"。良好的定义应当消除歧义,而不是引入新的歧义。

\textbf{晦涩}指的是使用难以理解的术语或表达。例如,将"水"定义为"氢氧化二氢"对于不懂化学的人来说就是晦涩的。虽然这个定义在技术上是准确的,但对于一般受众来说并不起到澄清意义的作用。

\textbf{比喻性语言}使用一种事物来象征或代表另一种事物。例如,将"时间"定义为"生命的河流"就是使用比喻。虽然比喻有时可以增加表达的生动性,但在定义中使用比喻往往会导致不精确,因为比喻依赖于个人解释。

良好的定义应当使用直接、确切的表达,以最大程度地减少误解的可能性。

\subsection{规则五:定义在可以用肯定的地方就不应当用否定定义}

第五条规则建议,只要有可能,定义就应当采用肯定的形式,避免使用否定形式。这是因为否定定义往往不能充分说明被定义项是什么,而只说明它不是什么。

例如,将"和平"定义为"没有战争的状态"就是一个否定定义。虽然这个定义在某种程度上是正确的,但它并没有积极地说明和平的特征是什么。相比之下,将"和平"定义为"人们和谐相处、互相尊重的社会状态"就是一个更好的肯定定义。

有些情况下,当被定义项本身是否定性质时,使用否定定义是适当的。例如,"无神论"可以恰当地定义为"不相信神存在的哲学立场"。但即使在这些情况下,如果可能的话,也应该尝试提供一些肯定的特征。

否定定义的问题在于,它们经常过于宽泛。例如,"非鱼"包括了除鱼以外的所有事物,这样的定义几乎没有信息价值。良好的定义应该提供足够的积极信息,使人们能够识别和理解被定义的概念。

\chaptersummary{
属加种差定义作为最重要的内涵定义方法,需要遵循五条传统规则以确保定义的准确性和有效性。

\logicemph{五条基本规则}:
\begin{itemize}
  \item \logicterm{规则一}:定义应当揭示种的本质属性,而非偶然属性,确保定义的根本性和稳定性
  \item \logicterm{规则二}:定义不能循环,即不能直接或间接地用被定义项来定义自身,避免逻辑循环
  \item \logicterm{规则三}:定义既不能过宽也不能过窄,其外延必须与被定义项完全一致,确保精确对应
  \item \logicterm{规则四}:定义不能用歧义的、晦涩的或比喻的语言,应当清晰精确,便于理解
  \item \logicterm{规则五}:定义在可以用肯定的地方就不应当用否定定义,应积极揭示事物特征
\end{itemize}

\logicemph{规则的重要性}:
\begin{itemize}
  \item 这些规则既是构建定义的指南,也是判断定义优劣的标准
  \item 违反任何一条规则都可能导致定义失效或产生误解
  \item 掌握这些规则有助于提高逻辑思维和表达能力
\end{itemize}

\logicwarn{实践意义}:理解和应用这五条规则对于学术写作、科学研究和日常交流都具有重要价值。
}
\section*{第3章概要}
解释词项的意义就是给出它的定义。在本章中,我们讨论了几种定义及其用法,以及构建定义的方法和运用这些方法的规则。\\\\
3.1 节解释了三种论争:

1.明显的实质争论,其中没有语词歧义,而且论争双方的确在态度上或信念上对立。

2.纯粹言辞之争,其中出现语词歧义,但根本没有实质歧见。\\\\
3.表面上是言辞的但实际上是实质的论争,其中既存在语词歧义,也存在论争双方在态度上或在信念上的歧见。

3. 2 节首先解释了定义总是符号的定义,并且引进了术语被定义项 (被定义的符号)和定义项(用来解释被定义项意义的符号)。还在五种定义及其基本用法中进行了区分:

1.规定定义,把一个意义指派给某个符号。规定定义不是报道,因而既不真也不假;它是运用被定义项来意指定义项指谓事物的建议、解决、请求或工具。

2.词典定义,它报道被定义项已经具有的意义,因而它可以或对或错。

3.精确定义,它超出了平常用法,用于消除与临界状况有关的麻烦的不确定性。其被定义项有一个现存的意义,但这个意义是模糊的;增添什么可以达至精确性,部分上是个规定问题。

4.理论定义,它寻求对它的适用对象精确表述一个理论上足够或科学上有用的描述。

5.说服定义,它运用表达性语言而不是信息性语言来寻求影响态度或激发情感。

在这五种定义中,前两种(规定定义和词典定义)主要用于消除歧义;第三种(精确定义)主要用于降低模糊性;第四种(理论定义)用于促进理论理解;而第五种(说服定义)用于影响行为。\\\\
3.3 节解释了普遍词项指谓其可以正确适用的多个对象。这些对象的汇集构成该词项的外延。说明了为词项外延中的所有对象并且仅为那些对象所共有的属性集就是该词项的内涵。词项的内涵决定其外延,但外延却不能决定内涵;因此,几个词项可以具有不同内涵而外延却相同;但外延不同的词项却不可能具有相同内涵。\\\\
3.4 节解释了怎样利用普遍词项的外延来构造外延定义;外延定义有几种类型,其局限性也被揭示出来:

1.列举定义,即在定义中列出或给出词项指谓对象的范例。\\\\
2.实指定义,在定义时,我们用手指出或以姿势标明被定义项的外延。

3.准实指定义,在定义中,姿势或手指的指示伴有一些其意义被认为是已为人所知的描述短语。\\\\
3.5 节解释了怎样利用普遍词项的内涵来构建内涵定义;内涵定义也有几种类型,其局限性也被揭示出来:

1.同义定义,在定义中提供另一个其意义已为人所知的词,这个词与被定义的词具有相同意义。

2.操作定义,它表明词项正确运用于一个给定场合,当且仅当,在该场合下特有的操作行为产生特有结果。

3.属加种差定义,首先要找出一个属,被定义项所指代的种是该属的一个子类;然后找出属性(或种差),即把该种的分子与属的所有其他种的分子区分开来的那种属性。

内涵定义的方法可以用于构建 3.2 节中五种定义的任何一种:规定定义、词典定义、精确定义、理论定义和说服定义。\\\\
3.6 节明确表述和解释了传统的属加种差定义的五条规则:

1.定义应当揭示种的本质属性。\\\\
2.定义不能循环。\\\\
3.定义既不能过宽又不能过窄。\\\\
4.定义不能用歧义的、䀲涩的或比喻的语言来表述。\\\\
5.定义在可以用肯定的地方就不应当用否定定义。
\section*{【注释】}
[1]A 是正确的,巴拿马运河的太平洋入口确实是在其大西洋人口的东面。\\
[2]William James,Pragmatism(1907).\\\\
[3]lbid.\\\\
[4]为掩盖实质论争,语词有时也被故意地用做两种含义以避免争执。拉比- A•J•鲁丁(Rabbi A.J.Rudin)把"有趣的"(interesting)释义为"英语中有争议的最该县咒的词",经常被大批的宗教会众使用以掩盖说话者的真正意见。鲁丁写道: "当用于布道时,'有趣的'的通常意思是'我患有失眠症',或者'我认为你说得精彩极了'。在其最为隐䀲的意义上,'有趣的'意指职员鲁葬……表达说话者不赞同的观点。"Religious News Service,January 1992。\\\\
[5]1991年,度量衡总委员会(General Committee on Weights and Measures)对它们进行了规定定义。该委员会是一家国际机构,其管理领域是科学的单位。另外,一千亿亿也叫一"zepto",一万亿亿也叫一"yocto"。\\\\
[6]这个新术语是在纽约城由普林斯顿大学的约翰•阿奇贝尔德•威勒(John Archibald Wheeler)博士在1967年的空间研究组织的一次会议上引进的。\\\\
[7]"夸克"出现在詹姆斯-乔伊斯(James Joyce)的小说(Finnegan's Wake)的 "Three Quark for Muster Mark"一行文字中;但是,盖尔曼博士报告说,他在看到那个名字之前就已经选择了它,他仅仅是根据乔伊斯拼写了它。\\\\
[8]See The Chronicle of Higher Education, 30 May 1993.\\\\
[9]一匹 600 千克(1 323 磅)重的真马的功率要比这个数大得多,估计大约为 18000 瓦。因此,一辆 200 马力的汽车大约相当于 8 匹真马的功率。\\\\
[10]与 1 升水的质量相同的单位长期被接受为 1 "千克"的定义。但是 1 千克现在已经更精确地定义为"与巴黎附近的保险库内的金属块具有相同质量的单位"。然而,人们仍在为"千克"寻找更加精确的定义,一种以一定数量的某种原子质量为依据的精确定义。\\\\
[11]Cali fornia v.Hodari D., 499 U.S.621, 1991.\\\\
[12]American Civil Liberties Union v.Reno, 929 Fed.Supp.824, 11 June 1996.\\\\
[13]"Defining Abortion a Tricky Business,"Honolulu Advertise, 14 February 1970.\\\\
[14]逻辑学家有时用"含义"(connotation)这个词来取代内涵,并较为普遍地使用"指称"(denotation)这个词来取代外延。但是,"指称"在日常话语中有其他更加普通的用法;大多数时候,它意指一个词项的情感意义,因此,此处引进它并无帮助。正如我们这里所做,通过运用词项内涵和外延来处理这种关键的区分,什么也不会丢失,并且可以避免一些混淆。\\\\
[15]内涵与外延之间非常有用的区分是由坎特伯雷的圣安瑟伦(St.Anselm of Canterbury,1033~1109)引进并强调的,他以他的"本体论论证"而著称,上述那个谬误的论证与他的论证并不相同。请参见 Wolfgang L.Gombocz,"Logik and Existenz in Mittelater",Philosophische Rundschau(1997)。\\\\
[16]John P.Sisk,"Art,Kitsch and Politics",Commentary,May 1988.\\\\
[17]Jay Livingston,Compulsive Gamblers(New York:Harper \& Row,1974), p. 2.\\\\
[18]W.H.Voge,"Strees-The Neglected Variable in Experimental Pharmacology and Toxicology,"Trends in Pharmacological Science,January 1987.\\\\
[19]Herbert Spencer,Principles of Biology, 1864.\\\\
[20]Samuel Johnson,Dictionary of the English Language, 1755.\\\\
[21]Ambrose Bierce,The Devil's Dictionary, 1911. 

% 第四部分
\chapter{谬误}
\section{什么是谬误?}

\begin{logicbox}[title=引言]
\textit{逻辑学中的谬误指的是一种特定类型的推理错误,它们表面上看似合理,实则存在逻辑缺陷。识别和理解谬误对于提高批判性思维和避免错误推理至关重要。}
\end{logicbox}

\subsection{谬误的定义与本质}

\begin{theorembox}[title=论证失败的两种情形]
\logicemph{论证目标}:一个论证,无论其主题或领域是什么,一般都是为证明其结论为真而构建的。

\logicemph{失败情形一:前提虚假}:
\begin{itemize}
  \item 将一个虚假命题设定为论证的前提之一
  \item 如果论证的前提不真,那么就不能确立其结论的真,即使从前提到结论的推理是正确的
  \item 检验前提的真与假并不是逻辑学家的特殊职责,而是所有研究工作的共同任务
\end{itemize}

\logicemph{失败情形二:推理无效}:
\begin{itemize}
  \item 论证所依赖的前提并不蕴涵结论
  \item 这才是逻辑学家的特殊领地
  \item 逻辑学家主要关心的是结论与前提之间的逻辑关系
\end{itemize}
\end{theorembox}

\begin{theorembox}[title=谬误的核心定义]
\logicemph{基本定义}:一个论证的前提不支持它的结论,即使它的所有前提都是真的,它的结论也可能是假的。在这种情况下,其推理便是糟糕的,而这种论证就称为\logicterm{谬误}。

\logicterm{谬误}就是推理错误。

\logicwarn{精确含义}:逻辑学家所用的"谬误"这个词,并不指称所有过失推理或虚假信念,而是指称一种典型错误,即经常出现在日常话语中破坏论证的错误。
\end{theorembox}

\begin{theorembox}[title=谬误的类型特征]
\logicemph{类型性质}:
\begin{itemize}
  \item 每个谬误都是不正确论证的一种类型
  \item 若论证中出现了一个特定类型的错误,就称为犯有那种谬误
  \item 因为每个谬误都是一种类型,两个或更多的不同论证可以包含或犯有相同谬误
\end{itemize}

\logicemph{实例关系}:
\begin{itemize}
  \item 不同论证在推理中可以表现为同一种错误
  \item 包含或犯有特定类型谬误的一个论证,也可以被称为是一个谬误
  \item 即那种类型错误的一个实例
\end{itemize}
\end{theorembox}

\subsection{谬误的心理说服力}

\begin{theorembox}[title=谬误的欺骗性特征]
\logicemph{错误的多样性}:推理进入歧路的方式可以有很多种,也就是说,论证错误有很多种。

\logicemph{心理说服力}:习惯上,人们将"谬误"这个词用在那些虽然不正确但在心理上具有一定说服力的论证。

\logicwarn{明显错误与隐蔽错误的区别}:
\begin{itemize}
  \item 有些论证错误是非常明显的,不能欺骗和说服任何人
  \item 但是,谬误却是危险的,因为我们大都会偶尔被某些谬误所愚弄
\end{itemize}

\logicemph{操作性定义}:我们将谬误定义为一种\logicterm{看似正确但经过检验可证其为错误的论证类型}。

\logicemph{研究价值}:研究这些错误论证是非常有益的,因为当我们明确理解它们后,就可以最有效地避开它们布下的陷阱。有备无患!
\end{theorembox}

\subsection{谬误分析的复杂性}

\begin{theorembox}[title=语境依赖性问题]
\logicwarn{解释依赖性}:特定的论证是否事实上真是谬误,可能取决于其作者对词项的解释。

\logicwarn{语境重要性}:
\begin{itemize}
  \item 看来是谬误的语段,若脱离语境,就可能难以确定作者使用的词项打算意味什么
  \item 有时,"谬误"的指责就会不公平地对准这样的语段
  \item 作者想要表达的观点却被批评者漏掉了(或许,作者甚至是开玩笑的)
\end{itemize}

\logicemph{分析原则}:当我们将对谬误论证的分析运用到实际谈话中时,应当注意这种不可避免的复杂情况。

\logicemph{平衡标准}:我们的逻辑标准应当高,但将这些标准运用到日常生活的论证中时,也应当宽宏大量和公平。
\end{theorembox}

\subsection{谬误的分类体系}

\begin{theorembox}[title=谬误研究的历史发展]
\logicemph{历史起源}:亚里士多德是第一位对谬误有系统研究的逻辑学家,他曾列举出13种\cite{aristotle}。

\logicemph{现代发展}:近来,超过100种的谬误名单被列了出来\cite{fearnside1959}。

\logicwarn{数量的不确定性}:谬误并没有一个精确的可以确定下来的数目,因为在列举它们时,在很大程度上取决于所使用的分类体系。
\end{theorembox}

\begin{theorembox}[title=本书的分类体系]
\logicemph{选择标准}:在此,我们挑出17种谬误,即推理中最普通且最有欺骗性的错误,分成三大组:

\begin{itemize}
  \item \logicterm{相干谬误}(fallacy of relevance)
  \item \logicterm{预设谬误}(fallacy of presumption)
  \item \logicterm{含混谬误}(fallacy of ambiguity)\cite{joseph1916}
\end{itemize}
\end{theorembox}

\begin{theorembox}[title=分类的局限性与价值]
\logicwarn{分类的任意性}:
\begin{itemize}
  \item 谬误的分组总有某种程度的任意性
  \item 一种错误会与另一种错误具有密切的相似性,有时还是相重合的
  \item 一个给定的谬误语段应属于哪个特定组别也常常引起人们的争论
  \item 语段中可能会有一个以上的推理错误
\end{itemize}

\logicemph{实用价值}:
\begin{itemize}
  \item 理解三种主要种类的每一种本质特征及其各种子类别的特别特征,将具有很大的实际用处
  \item 当推理中最难缠的错误出现于通常话语中时,这些理解就能够使人们发觉这些错误
  \item 辨识这些相互联系的谬误也有益于提高我们的逻辑敏感性
  \item 这种敏感性也有益于我们识别那些在三大组中未能包含的谬误
\end{itemize}
\end{theorembox}

\chaptersummary{
谬误是逻辑学研究的重要内容,理解谬误的本质和特征对于提高批判性思维能力具有重要意义。

\logicemph{谬误的核心概念}:
\begin{itemize}
  \item \logicterm{基本定义}:看似正确但经过检验可证其为错误的推理类型
  \item \logicterm{本质特征}:前提不支持结论的无效论证,但具有表面的合理性
  \item \logicterm{类型性质}:每个谬误都是不正确论证的一种类型,可以在不同论证中重复出现
\end{itemize}

\logicemph{谬误的欺骗性}:
\begin{itemize}
  \item \logicterm{心理说服力}:谬误具有一定的心理说服力,容易欺骗和误导人们
  \item \logicterm{隐蔽性}:与明显的论证错误不同,谬误需要仔细检验才能识别
  \item \logicterm{危险性}:正是这种表面的正确性使得谬误具有欺骗性和危险性
\end{itemize}

\logicemph{谬误的分类体系}:
\begin{itemize}
  \item \logicterm{相干谬误}:前提与结论缺乏相关性的推理错误
  \item \logicterm{预设谬误}:基于错误假设或预设的推理错误
  \item \logicterm{含混谬误}:由于语言歧义或模糊性导致的推理错误
\end{itemize}

\logicwarn{分析原则}:
\begin{itemize}
  \item 需要考虑语境和作者意图,避免脱离语境的机械判断
  \item 保持高逻辑标准,但在实际应用中要宽宏大量和公平
  \item 认识到分类的相对性和复杂性,一个论证可能包含多种谬误
\end{itemize}
}
\section{相干谬误}

\begin{logicbox}[title=引言]
\textit{相干谬误是逻辑推理中最常见的一类错误,它表现为论证前提与结论之间缺乏必要的逻辑联系。识别这类谬误不仅有助于避免错误推理,也能帮助我们构建更有说服力的论证。}
\end{logicbox}

\subsection{相干谬误的本质}

\begin{theorembox}[title=相干谬误的定义]
\logicemph{基本定义}:当一个论证所依据的前提与其结论不相干因而不可能确立结论之真时,其所犯的就是\logicterm{相干谬误}。

\logicwarn{命名问题}:或许,称之为不相干谬误更贴切,但是,(在实际论证中)这种论证的前提常常在心理上与结论是相干的,而正是这种相干性使得它们似乎正确和有说服力。
\end{theorembox}

\begin{theorembox}[title=心理相干与逻辑相干的混淆]
\logicemph{混淆机制}:\logicterm{心理的相干}怎么会与\logicterm{逻辑的相干}相混淆,可以用我们在第2章讨论的语言的不同用法进行部分阐释。

\logicemph{分析价值}:这些混淆的机制在随后的分析中将变得更加清晰。

\logicwarn{欺骗性根源}:正是心理相干与逻辑相干的混淆,使得相干谬误具有强烈的欺骗性和说服力。
\end{theorembox}

\begin{theorembox}[title=谬误的命名传统]
\logicemph{拉丁传统}:很多谬误传统上都有个拉丁名称,这些名称承载着丰富的逻辑学历史。

\logicemph{语言融合}:有些拉丁名称,像\textit{ad hominem}(人身攻击),已经进入普通英语语言之中。

\logicemph{双重命名}:我们在这里将既使用拉丁名称又使用英语名称,以便读者更好地理解和记忆。
\end{theorembox}

\begin{theorembox}[title=相干谬误的主要类型]
\logicemph{分类体系}:相干谬误包括以下主要类型:

\begin{enumerate}
  \item \logicterm{诉诸无知论证}(Argumentum ad Ignorantiam)
  \item \logicterm{诉诸不当权威}(Argumentum ad Verecundiam)
  \item \logicterm{人身攻击论证}(Argumentum ad Hominem)
  \item \logicterm{诉诸情感}(Argumentum ad Populum)
  \item \logicterm{诉诸同情}(Argumentum ad Misericordiam)
  \item \logicterm{诉诸暴力}(Argumentum ad Baculum)
  \item \logicterm{不相干结论}(Ignoratio Elenchi)
\end{enumerate}

\logicemph{学习重点}:每种谬误都有其独特的表现形式和识别方法,需要通过具体例子来深入理解。
\end{theorembox}
\subsection{R1.诉诸无知论证(The Argument from Ignorance:Argument Ad Ignorantiam)}

\begin{theorembox}[title=诉诸无知论证的定义]
\logicterm{诉诸无知论证}犯的是这样的错误,它辩称一个命题是\logicemph{真的},其依据仅仅是该命题并没有被证明为\logicwarn{假},或者辩称一个命题是\logicwarn{假的},仅仅因为并没有证明其为\logicemph{真}。稍一思考,我们就知道,许多假命题没有被证出是假的,许多真命题也没被证出是真的,因而人们对怎样证明或否证一个命题的无知并不能证实它的真或假。
\end{theorembox}

但这种\logicwarn{诉诸无知谬误},伴随着科学发展中的错误理解,是经常冒出的。在这种错误理解中,那些其真还没有得到证实的命题就因此而被某些人主张是假的;在伪科学领域中,关于通灵(psychic)及相似现象的命题被认为是真的,其理由只是它们的假并没有得到证实。

\paragraph{历史案例:伽利略的望远镜}
当伽利略用他的望远镜看到了月亮上的山脉和山谷,并力图向他那个时代的主要天文学家们进行证实时,他的批评者给出了一个在科学史上著名的诉诸无知论证。那时的一些学者绝对相信月亮是一个完美的球形,正如神学和亚里土多德学说所长期教导的那样,他们争论说,虽然我们看到的那些东西好像是山脉和山谷,但月亮实际上必定仍是一个完美的球形,因为它所有明显不规则的地方都一定充满着一种看不见的水晶般的物质——这是保全天体完美性的一种假说,而伽利略并不能证明它是假的!

据说,伽利略为了揭露这种诉诸无知论证的荒谬,仿照它提出了另一个同类的诉诸无知的论证。由于并不能证明那种设想的充满山谷的透明水晶物质的存在,他提出了一个同等可能的假说:那种看不见的水晶覆盖物上存在更高的山峰——但它是由水晶构成的,因此是看不到的!他指出,他的批评者也不能证明这个假说是假的。

\paragraph{政治变革中的诉诸无知}
那些强烈反对某种重大变革的人,常常试图以这种变革还没有被证明可行或安全为根据而反驳它。这种证明通常不可能先行给出,而诉诸这种反驳却通常是无知混合着恐惧。这种诉诸经常采取\logicterm{修辞问句}的形式来暗示(但不直接断定)所提议的变革充满着未知的危险。政治变革既可以为诉诸无知所反对,有时也可以为它所支持。

1992年,当联邦政府宣布弃权,允许威斯康星州减少曾经为多于一个孩子的母亲提供的额外利益时,有人问威斯康星州长是否有证据表明有多个孩子的未婚母亲仅仅是为了得到额外收人。他回答道(诉诸无知):"不,没有。确实没有,但是也没有反面证据。"${ }^{[4]}$

\paragraph{合理应用的情况}
当然,在某些情况下,如果人们以适当方式来积极地寻找并揭示证据或结果,但之后却没有得到特定证据或结果,那么人们对这个事实就可能有实质性的争论。例如,人们通常用老鼠或其他动物实验对象对新药进行长期的安全性检验,如果对动物没有任何毒性影响,那么也就被认为是对人可能无毒的证据,尽管这不是最后结论。消费者保护就经常依赖这类证据。

在与此相似的环境中,我们依赖的不是无知而是我们的如下知识或者信念:假如会出现我们关心的结果,那么它在某些实验中就可能已经出现。这种以未能否证去确定真的证明,设定了研究者具有高度技巧:假如有那种证据的话,他们就非常可能已经发现了它。在这种情况下,有时也可能发生悲剧性错误。但是,如果标准设得过高,如果要求的证明是实际上不可能给出的最终无害的证明,那么消费者就无法享用那些可以被证明有价值的甚至挽救生命的医药治疗。

相似的,当安全性研究没有发现实验对象产生不适当行为的证据时,做出该研究使我们一无所得的结论也可能是错误的。彻底研究的正确结果将可能是"清除"原来研究的结论。在某些情况下,不做出结论与做出一个错误结论一样违反正确推理的法则。

\paragraph{法律中的诉诸无知}
\begin{examplebox}[title=法律中的合理应用]
诉诸无知在刑事法庭上是常用而适当的方法。美国法理学和英国普通法体系中,在证明一个在刑事法庭上受指控的人有罪之前,必须先假定他\logicemph{无罪}。我们支持这个原则,因为我们认识到,宣判无罪者有罪的错误远比开释犯罪者的错误更为严重。因此在刑事案件中,辩护律师可以有权合法要求,如果被告除了合理怀疑外没有被证明有罪,那么就应裁决无罪。
\end{examplebox}

美国高等法院坚定地重申了这种证明标准,它说:

合理怀疑的(限制)标准……是降低真正错误定罪危险的主要工具。该标准为无罪推定提供了坚固基石:这种基本的公理化原则是我们的刑法得以执行的基础。 ${ }^{[5]}$

但是,这种诉诸无知只适用于此类因不能证明有罪而不得不采用无罪假定的情形,在其他语境中,这样的诉诸就是诉诸无知(谬误)论证。
\subsection{R2.诉诸不当权威(The Appeal to Inappropriate Authority:Argument Ad Verecundiam)}

在试图对某些困难或复杂问题做出决定时,受公认专家判断引导是完全\logicemph{合理的}。当我们争辩说一特定结论是\logicemph{正确的}因为专家权威已经得出那个判断时,我们并没有犯谬误。的确,对我们大多数人来说,对权威的这种依赖在很多事情上来说都是必需的。当然,专家的判断也不能构成最终证明,专家的意见之间也可能对立;即使一致,他们也可能出错;但是,专家意见确实是支持结论的一种合理方式。

\begin{theorembox}[title=诉诸不当权威谬误的定义]
然而,当诉诸的对象对所讨论问题不能合理地宣称权威时,就会产生\logicwarn{诉诸不当权威谬误}。因而,正像诉诸伟大的艺术家如毕加索的意见来解决经济争论一样,在关于道德的论证中,诉诸生物学杰出权威达尔文的意见也是\logicwarn{谬误论证}。
\end{theorembox}

但是,在决定谁的权威可以合理地依赖和拒绝上必须\logicwarn{小心}。毕加索不是经济学家,但在属于艺术杰作经济价值的争论上,他的判断就可以合理地给予某种分量;如果争论的是道德问题中的生物学作用,那么达尔文的确可以是一位适当的权威。

\paragraph{广告中的不当权威诉诸}
\begin{examplebox}[title=广告中的不当权威诉诸]
错置诉诸权威的最为明显的例子出现在广告的"证言"中。有人力劝我们开某一牌子的汽车,因为一位著名的高尔夫球员或者网球员断言了它的优越性;有人力劝我们饮用某种牌子的饮料,因为某电影明星或足球教练表达了对它的爱好。无论何处,如果对某命题为真的断定以某人的权威为依据,而他在那个领域并没有特殊的能力,那么这种\logicwarn{错置权威诉诸}就犯有\logicwarn{谬误}。
\end{examplebox}

\paragraph{政治和国际关系中的权威问题}
这好像是容易避免的头脑简单的错误,但由于存在着各种引发这种谬误诉诸的环境,这仍是一种危险的思维陷阱。这里有两个例子。在国际关系领域中,武器和战争扮演着不愉快的重要角色,对各种意见的支持经常诉诸这些人:他们对武器的技术设计和构造有特殊能力。例如,对于某些武器可以怎样或不能怎样起作用,物理学家诸如罗伯特•奥本海默(Robert Oppenheimer)或爱德华•泰勒(Edward Teller)可能的确具有给出权威判断的知识,但是他们在这个领域内的专业知识却不能在决定重大政治目标时赋予他们特殊的智慧。把一位杰出物理学家的强有力判断诉诸为批准某些国际条约的理据,就可能是诉诸不当权威的论证。

相似的,我们羡慕小说杰作的深度和洞识,比如,亚历山大•索尔仁尼琴(Alexander Solzhenitsyn)或索尔•贝娄(Saul Bellow)的小说中的洞识,但在某些政治争论中,诉诸他们的判断以决定真正的战争罪犯,就可能是诉诸不当权威。${ }^{[7]}$

\paragraph{确定合理权威的标准}
许多人都提出(或者由他人介绍)自己是某个领域的"专家",然而,决定谁的权威真正值得依赖却往往是个难题。假定我们想要知道某命题$p$是否真,假定某人A被认为是关于$p$或类似于$p$命题的专家,并且A说$p$是真的,那么,A的说法在什么条件下能够真正地给予我们充分的理由以接受$p$为真呢?当然,在真实事例中,答案取决于$p$断定的是什么,还取决于A和类似于$p$的诸多命题之间的关系。

\begin{theorembox}[title=确定合理权威的标准]
一般的,我们必须回答的问题是:按照知识、经验、训练或总体环境,A比我们这些正在讨论该问题、判断$p$是否为真的人\logicemph{更有能力}吗?如果是那样,那么,对我们来说,作为关于$p$为真的证据,A的判断就具有某种价值;当然,尽管A的判断可能还不是充分证据,但它却或许比其他考虑更平衡全面,或许比其他人的证词更重要,而这些人也比我们关于$p$有更多知识。
\end{theorembox}

\logicwarn{诉诸不当权威论证}就是诉诸这样的人,他无权声称比我们自身有更大能力来判断$p$的真。当然,即使一个人的确具有合法声称的权威,也很可能会被证明\logicwarn{出错},而我们以后可能后悔我们对专家的选择。但是,如果我们选择的专家无愧于对事物情况如$p$(无论$p$可能是什么)的知识名声,那么依赖他们并没有\logicwarn{谬误},即使他们是\logicwarn{错误的}。如果我们的结论以权威意见为基础,但该意见在那个问题上不能合理地宣称是专门知识,那么我们的错误就是一种推理错误即\logicwarn{谬误}。${ }^{[8]}$
\subsection{R3.人身攻击论证(Argument Ad Hominem)}

\begin{theorembox}[title=人身攻击论证的定义]
短语\logicterm{"ad hominem"}译做"人身攻击"。它命名的是一种\logicwarn{谬误性反驳},即它的抨击不是指向结论,而是指向断定结论或为结论辩护的人。当一个论证攻击提出主张的人而非主张本身时,就犯了\logicwarn{人身攻击谬误}。
\end{theorembox}

\begin{theorembox}[title=A. 诽谤型人身攻击]
\logicemph{常见表现}:在激烈的论辩中,参与者有时贬低对手的品格,否认他们的智力或推理能力,质疑他们的正直,等等。

\logicwarn{逻辑错误}:个人的品格与他主张的命题的真假或推理的正误在逻辑上并无关联。

\logicemph{典型例子}:如果认为某种意见是糟糕的或断定是错误的,而其原因却只是它们是由"激进派"或"极端派"提出的,那么这就构成了人身攻击谬误的一种典型特例:\logicwarn{诽谤}。

\logicwarn{心理机制}:
\begin{itemize}
  \item 诽谤的前提与结论是不相干的
  \item 然而它却可能通过转移心理进路来说服人
  \item 可以鼓动对一个人的反对态度
  \item 情感上的反对范围甚至扩展得与鼓动者做出的判断也相对立
\end{itemize}
\end{theorembox}

\begin{examplebox}[title=哲学辩论中的诽谤现象]
\logicemph{实际案例}:几位当代美国哲学家之间的一场尖锐论争就例示了这种谬误攻击。

\logicwarn{论辩者A的攻击}:
\begin{quote}
"被体面对手以体面方式抨击的事情,在哲学中一直出现。但是,在我看来,索莫斯(Sommers)的智力方法是不诚实的。她无视哲学争论的最基本礼仪。"$^{[9]}$
\end{quote}

\logicwarn{论辩者B的反击}:
\begin{quote}
"几个诋毁我的人所用的一个不诚实和毫无价值的策略是,认为我从没有做过的抱怨是我所做的,然后把这些'抱怨'作为'我不负责任的和轻率不公正的证据'来打发。"$^{[10]}$
\end{quote}

\logicemph{问题分析}:但冲突双方所居地位的应有美德,却没有在这种论证中显示出来。

\logicwarn{诽谤的多种变形}:
\begin{itemize}
  \item 对手可能被诽谤为巧舌如簧
  \item 被标签为"孤立主义者"或"干涉主义者"
  \item 被归类为"极右"或"极左"分子
  \item 等等
\end{itemize}

\logicterm{遗传谬误}:当诽谤性攻击论证采用攻击对立方出身(这当然与真假无关)的形式时,就称之为\logicterm{"遗传谬误"}。
\end{examplebox}

\begin{theorembox}[title=连带罪恶与证人可信度]
\logicwarn{连带罪恶的定义}:有时,一个结论或它的拥护者可能会因为拥护其观点者都是那些被广泛认为品质不好的人而受到指责。

\logicemph{历史案例}:在其臭名昭著的审判中,苏格拉底被判决不敬之罪,部分原因就是他与那些被广泛认为对雅典不忠和品行上贪婪的人有联系。

\logicemph{现代案例}:1997年,克莱德·柯林斯·斯诺(Clyde Collins Snow)因为他在科学研究中所得出的结论而被指责为种族主义者,他回答如下:

\begin{quote}
"在过去十年中,我的工作倾注于调研许多国家的失踪、毒打和超越法律迫害的人权受害者,这使我成了公众批评和政府撒气的靶子。然而,直到今天没有一个批评我的人把我视为种族主义者。对我的诋毁,有阿根廷(Argentina)的野蛮的军事政务会辩护者、智利(Chile)的皮诺切特(Pinochet)将军的军事代表、危地马拉的(Guatemalan)国防部长以及塞尔维亚(Serbian)政府的说客。因而,古德曼(Goodman)先生(斯诺的指责者)发现他自己处于有趣的伙伴中。"$^{[11]}$
\end{quote}

\logicwarn{谬误类型}:
\begin{itemize}
  \item 不公平指责是人身诽谤的极其普通的形式
  \item \logicterm{连带罪恶}是诽谤的另一种方式,它不那么广泛但却是同等谬误
\end{itemize}

\logicemph{法律程序中的特殊考虑}:
\begin{itemize}
  \item 在法律程序中,有时禁止不可靠者及"存疑证人"作证乃是可取的
  \item 如果不诚实在其他问题上已显示出来并因而破坏了信用,那么在法律程序中,这种存疑在这种背景下可能不是谬误
  \item 但是,由此却不能简单地断定这种证人说的是谎话
  \item 我们必须禁止各种不诚实或欺骗,也必须揭露与过去证词的矛盾
\end{itemize}

\logicwarn{关键原则}:即使在这种特殊背景下,攻击品格也不能确立所给出的证词是假的;如果那样,推理便是谬误。
\end{theorembox}

\begin{theorembox}[title=B. 背景谬误]
\logicwarn{背景谬误的定义}:\logicterm{背景谬误}是人身攻击谬误的一种形式。引起背景谬误的是,在本不相干的信念与该信念持有者的背景之间加以牵连。

\logicemph{核心原则}:人们做出或拒绝某个主张的背景并不承载该主张为真。

\logicwarn{谬误机制}:如果仅仅因为对手的职业、国籍、政治联系或其他背景,就固执地迫使对手接受或拒绝某个结论,那么这样的论证就是谬误的。

\logicemph{典型例子}:
\begin{itemize}
  \item \logicwarn{宗教背景}:如果认为圣职人员必须接受某个给定观点,因为否定它就与《圣经》相矛盾,那么这是不公平的
  \item \logicwarn{政治背景}:若认为政党候选人必须支持某项政策,因为它是其所属政党的纲领中公开宣示的,这也是不公正的
\end{itemize}

\logicwarn{逻辑错误}:这样的论证与所论及的命题真假无关,它仅仅是力促某人接受背景。
\end{theorembox}

\paragraph{tu quoque与偏见指控}
\begin{examplebox}[title=tu quoque谬误示例]
有人指责猎人毫无用途地屠杀没有惹人的动物,而猎人有时却通过指出其批评者食用无害牲畜来回应。这样的回应显然是人身攻击,批评者食肉的事实与证明猎人为娱乐而猎杀动物合理性根本不沾边。拉丁术语\logicterm{"tu quoque"}(意思是"你是另一个"),有时被用来命名这种人身攻击论证的背景谬误。
\end{examplebox}

\begin{theorembox}[title=背景谬误的心理机制与偏见指控]
\logicemph{心理作用}:在严肃论证中,对手的背景并不是重要问题,要求注意它们可能在取赞扬或说服他人方面起心理作用。

\logicwarn{本质特征}:但是,无论多么有说服力,这种论证本质上都是谬误的。

\logicwarn{偏见指控的谬误}:
\begin{itemize}
  \item 有时,背景谬误被用来表明应当拒绝对手的结论
  \item 指责导致他们做出判断的是他们的特殊处境而不是推理或证据
  \item 所以他们的判断是\logicterm{有偏见的}
\end{itemize}

\logicemph{正确原则}:
\begin{itemize}
  \item 一个有利于某团体的论证,并非就没有讨论价值
  \item 若仅仅依据其被该团体成员提出从而为该团体服务为由而非难之,就是犯了背景谬误
\end{itemize}

\logicemph{具体例子}:赞成保护关税的论证可能是糟糕的,但它们之所以糟糕,却并不是因为它们是由从关税保护中获得好处的制造商提出的。
\end{theorembox}

\begin{examplebox}[title=污泉谬误的经典案例]
\logicwarn{污泉谬误的定义}:背景谬误论证之一,称做\logicterm{"污泉"}(poisoning the well),尤为悖理。

\logicemph{历史案例}:产生这个名字的事件典型地例示了这种论证:

\logicwarn{金斯利的攻击}:英国小说家和教士查尔斯·金斯利(Charles Kingsley)攻击著名的天主教智者约翰·亨利·卡迪拉尔·纽曼(John Henry Cardinal Newman)说,卡迪拉尔·纽曼的主张是不能信任的,因为作为一名罗马天主教的牧师,他首先要忠诚的不是真理。

\logicemph{纽曼的反驳}:纽曼反驳道,这种人身攻击使他并且也使全体天主教徒的进一步论辩成为不可能,因为他们为自己辩护所说的任何东西都可以因被他人指责为根本不关心真理而遭到拒斥。

\logicterm{污泉比喻}:卡迪拉尔·纽曼说,金斯利"污染了对话之泉"。
\end{examplebox}

\begin{theorembox}[title=人身攻击论证的内在联系]
\logicemph{谬误类型的关系}:人身攻击论证的诽谤谬误和背景谬误之间,存在一种清晰的联系:背景谬误可以被看做诽谤谬误的一种特殊情况。

\logicwarn{具体表现}:
\begin{itemize}
  \item 当使用背景谬误明显或暗含地指责对手缺乏一贯性(在他们的信念中,或者在他们的言行之间),它很明显就是一种诽谤
  \item 而用背景谬误指责对手由于其属于某集团或具有集团信仰而缺乏信任价值,显然也是指责对手具有自利偏见的诽谤手段
\end{itemize}

\logicwarn{总体特征}:无论何种形式,人身攻击论证都是对论辩对手的谬误性诋毁。
\end{theorembox}
\subsection{R4.诉诸情感(The Appeal to Emotion:Argument Ad Populum)}

\begin{logicbox}[title=引言]
\textit{这种常见谬误和接下来的两种谬误都非常明显,只需稍加说明。在三种情形中,前提都明显地与结论不相干,都是故意用来操纵听者或读者信念的工具。}
\end{logicbox}

\paragraph{诉诸情感的本质}
\begin{theorembox}[title=诉诸情感谬误的定义]
\logicterm{诉诸情感}(Argument Ad Populum)的字面意义是"诉诸人群",意蕴诉诸情感容易激动的无序民众。当论证试图通过激发听众的情感而非提供理性证据来说服时,就犯了\logicwarn{诉诸情感谬误}。这类论证之所以是\logicwarn{谬误},是因为它用表达性语言和其他有计划的手段以博取情感,激起兴奋、愤怒或憎恨,而不是致力于提出证据和合理论证。
\end{theorembox}

阿道夫•希特勒的讲演,激发德国听众达到一种狂热爱国状态,可以作为一种经典范例。爱国是一种可敬的高尚情感,通过不适宜地诉诸它来操控听众,在智力上是\logicwarn{低劣的}——萨缪尔•约翰森挖苦地说:"爱国主义是恶棍的最后避难所。"

\paragraph{广告中的诉诸情感}
\begin{examplebox}[title=商业广告中的诉诸情感]
最为严重的诉诸情感可以在商业广告中找到,那里的运用几乎达到出神入化的境地。广告的产品都或明或暗地与我们渴望的或惹人好感的事物相联系。早餐的麦片粥与健美年轻、体魄健壮和精力充沛相联系,威士忌与豪华和成就相联系,啤酒与崇高冒险相联系,汽车与浪漫、富有和性感相联系。广告产品描绘出的男人一般都是英俊而杰出,女人精明而迷人—或者干脆一丝不挂。我们这个时代广告艺术家的聪明和持之以恒足以使我们全部都在某种程度上受了影响,尽管我们决心抵制。几乎各种想象不到的手段都可以用来支配我们的注意力,甚至渗透到我们的潜意识之中。我们不断地被各种\logicwarn{诉诸情感谬误}所操纵。
\end{examplebox}

\paragraph{产品与情感的隐性联系}
就其本身来说,产品与情感的纯粹联系并不是论证,但是,诉诸情感论证通常就存在于那种表面现象之下的不远处。当广告者声称他们的产品设计是为赢得我们的情感赞赏时,当它表明我们应该购买因为这些产品性感或畅销或者与财富或能力相联系时,它就隐含地断言了该结论来自这种前提,这种断言就显然是谬误的。

关于诉诸情感论证,有些例子是厚颜无耻的。下面是最近ABC-TV广告的原有语句:

为什么庞蒂克汽车大奖吸引如此多的人们?是因为庞蒂克大奖吸引了如此多的人们,是因为一大奖吸引了如此多的人们!

\paragraph{民意调查中的情感影响}
在民意调查中,诉诸大众热情尤其有害;在那里,众所周知的某些特定词汇${ }^{[12]}$的情感影响(消极的或积极的)可以使所设计的问题本身就产生出要寻找的答案。例如,美国公众对减税和联邦政府如何支配即将出现的预算盈余持什么态度呢?这取决于你怎么询问。这个问题在2000年1月呈给了随机抽样的民众——两种不同的措辞${ }^{[13]}$:

措辞1:"(在行将盈余的资金中)其大部分应当用于減税,还是应当用做新的政府计划的资金呢?"

对这个问题的这种措辞,60\%的被抽查者回答说"减税",而$25\%$回答"新计划"。毫不惊奇:"新的政府计划"明显不受很多人欢迎。

措辞2:"(在行将盈余的资金中)其大部分应当用于减税,还是应当花费在教育、环境、保健、打击犯罪和军事防御等新计划上?"

对该问题的这种措辞,$22\%$的被抽查者回答"减税",$69\%$回答"新计划"。再次毫无惊奇:"教育、保健和打击犯罪等等"都是人们熟知的深受广泛欢迎的词汇。

\paragraph{大众接受不等于真理}
当然,一项政策的广为接受并不代表它明智,很多人都持某种观点这种事实也不能证明它就是真的。罗素曾以有点过头的语言抨击了这种论证:

一个为人们广泛持有的观点这种事实,无论如何都并非它不是完全荒谬的证据;实际上,在多数人的愚蠢观念中的广为接受的信念,更可能是愚笨多于明智。[14]
\subsection{R5.诉诸同情(The Appeal to Pity:Argument Ad Misericordiam)}

\textbf{诉诸同情}(misericordiam的字面意思是"同情心")可以看做是诉诸情感的一种特殊情况,其中听众的利他主义和怜悯之心是其所诉诸的特殊情感。当论证试图利用人们的怜悯和同情心而非提供相关证据来支持结论时,就犯了\textbf{诉诸同情谬误}。

\paragraph{法庭辩护中的同情诉求}
在法庭上,原告的律师为寻求伤害赔偿金,常常以某种极其悲苦的方式安排展示委托人的伤残情况。${ }^{[15]}$在刑事审判中,虽然陪审团对被指控者无论有罪或无辜都不应带有同情,但是高效的辩护律师常常设法激起陪审团的同情,有时诉诸同情被做得神不知鬼不觉。

\paragraph{苏格拉底对诉诸同情的批判}
在其雅典审判中,苏格拉底轻蔑地提到其他被告人由他们的子女和家人陪伴出现在陪审团面前,以寻求激起怜悯之情而免除责任。苏格拉底接着说道:\\
$\cdots \cdots$我,生命处于危险中的我,将不会做任何这种事情。这种对比可能出现在他(陪审团成员)的头脑中,他可能反对我,愤怒地投票,因为他为此对我不高兴。现在,如果你们之中有这样的人,注意,我不是说确有,那么我可以诚实地回答他:我的朋友,我是人,和别人一样,一个有血有肉的生物,不是像荷马(Homer)所说的那种"木石之躯";我也有一个家庭,有孩子,噢,雅典人啊,有三个儿子,一个几乎成人,另两个还年幼;但是,我将不带他们任何人到这里以请求你们判我无罪。${ }^{[16]}$

\paragraph{荒谬的同情诉求案例}
有很多方法可以拨动心弦,而且事实上,也都为人们一再使用。在一次指控一个年轻人用斧头杀害了他父母的审判中,出现了最荒谬的诉诸同情论证:面对其罪恶的大量证据,他请求宽大处理,理由是他现在成了一个\textbf{孤儿}。 
\subsection{R6.诉诸暴力(The Appeal to Force:Argument Ad Baculum)}

\begin{theorembox}[title=诉诸暴力谬误的定义]
\logicterm{诉诸暴力}以达到接受某种结论,乍看起来好像是一种如此清楚的\logicwarn{谬误},完全用不着讨论。当一个论证依靠威胁或强制而非理性证据来支持结论时,就犯了\logicwarn{诉诸暴力谬误}。当证据或合理论证失败时,"暴力方法"的使用或威胁使用以强制对手,看起来是最后的手段——一种"方便实用"的手段。"强权就是公理"的道理并非难以捉摸。
\end{theorembox}

\paragraph{暴力的多种形式}
\begin{examplebox}[title=法律威胁案例]
当然,暴力威胁不必是武力。近来,博伊斯州立大学(Boise State University)的两位法学教授在丹佛大学(University of Denver)法律杂志上发表了一篇文章,严厉批评了博伊斯瀑布公司(Boise Cascade Corporation)——世界上纸张和木制品生产者之一。结果,这所大学发布了一个正式的"更正"声明,"这篇文章因其缺乏学术性及其错误内容已经被撤销"。

博伊斯瀑布公司威胁起诉该大学了吗?"噢,"该大学的总法律顾问说,"'\logicwarn{威胁}'是一个有趣的词。让我们这样说吧,他们指出他们受到的批评的确达到了可以提出诉讼的地步。"结果,该大学收到了一份那篇文章的复制件,它来自博伊斯瀑布的总法律顾问,附信说:"如果其中被标明之处以任何形式被丹佛大学继续发行,我已得到了对丹佛大学提起法律诉讼的建议。"${ }^{[17]}$
\end{examplebox}

\paragraph{隐蔽的暴力威胁}
但是,也有比较含蓄地使用诉诸暴力的场合。论证者可以用精心设计的方式,不是直接地而是隐蔽地传达一种可能威胁,使对方为势所迫不得不赞同,或者至少是附和。当里根政府的司法部长处于报刊引导的强大攻击下时,当时的白宫参谋长霍华德•贝克(Howard Baker),召集工作人员开会说:

\begin{displayquote}
总统仍然信任司法部长,我也信任司法部长,而且你们也应当信任司法部长,因为我们本来都是在为总统工作。谁若对此有不同意见,或者有与此不同的动机、野心或打算,那么他可以告诉我,因为我们将不得不讨论他的去留。[18]
\end{displayquote}

\paragraph{理性与强制的对立}
可以说,没有人会被这种论证愚弄;被胁迫方可以适当地做出行动,但最后不必接受强加的结论为\logicemph{真}。对此,20世纪的意大利法西斯主义代表回答说,真正的说服可以通过许多不同工具来进行,讲道理是一种而大棒是另一种;但是,一旦对手被真正说服,他们就会坚持它,而说服的工具却可能被忘却了。这种法西斯主义观似乎引导着当今世界上许多政府;但是,诉诸暴力的论证——依赖大棒或各种形式的暴力威胁——从理性上说都是\logicwarn{不可接受的}。诉诸暴力是对\logicwarn{理性的抛弃}。
\subsection{R7.不相干结论(Irrelevant Conclusion:Ignoratio Elenchi)}

\begin{theorembox}[title=不相干结论谬误的定义与特征]
\logicwarn{基本定义}:当一个论证声称要确证一个特定的结论,但却去证明另一个与之不同的结论时,就犯有\logicterm{不相干结论谬误}。

\logicemph{词汇来源}:Ignoratio elenchi的字面意义是"错误证明"。

\logicwarn{谬误特征}:
\begin{itemize}
  \item 它的前提"不得要领"
  \item 它的推理本身可能并非不合理
  \item 但它在争论所需结论的辩护却没有效力
\end{itemize}

\logicemph{核心问题}:论证者证明了某个命题,但这个命题与原本要证明的结论不相关。
\end{theorembox}

\begin{examplebox}[title=政策辩论中的不相干结论]
\logicwarn{常见领域}:社会法律领域中的论证经常犯有这种谬误。

\logicemph{典型情况}:
\begin{itemize}
  \item 一个特殊方案的确是为某种被广泛支持的更大目标服务的
  \item 但为该方案进行论证的前提所提供的理由却只能支持那个大目标
  \item 而没有告诉我们关于那个特定方案的任何东西
\end{itemize}

\logicwarn{产生原因}:
\begin{itemize}
  \item 有时这种方式是故意为之的
  \item 有时则是由于过于热情关心那种更大目标
  \item 而认识不到现有前提与特定方案的结论并不相干
\end{itemize}
\end{examplebox}

\begin{examplebox}[title=不相干结论的常见形式]
\logicwarn{普遍性}:这种谬误在日常生活和政治辩论中非常普遍。

\logicemph{具体例子}:
\begin{itemize}
  \item \logicwarn{就业政策例子}:有人主张某政策会增加就业机会,因此应当实施,然而其论证却只证明了就业增长是件好事,而没有表明该政策实际上能够增加就业
  \item \logicwarn{技术研究例子}:有人反对某项技术研究,理由是科技发展可能带来危害,但其论证却没有指出这项特定研究会如何造成危害
\end{itemize}
\end{examplebox}

\begin{theorembox}[title=不相干结论谬误的欺骗性与识别方法]
\logicwarn{欺骗性特征}:不相干结论谬误之所以具有欺骗性,是因为它的前提往往确实支持了某个结论,只是那个结论与原本要证明的命题并不相干。

\logicemph{识别关键}:仔细辨识讨论的实际焦点,是避免这种谬误的关键。

\logicemph{防范方法}:
\begin{itemize}
  \item 明确论证要证明的具体结论是什么
  \item 检查前提是否真正支持这个特定结论
  \item 区分一般性目标与具体方案的差别
  \item 避免被表面上合理的论证所误导
\end{itemize}
\end{theorembox>
\section{预设谬误}

\begin{quotation}
\textit{预设谬误是一种隐藏在论证中的特殊错误,它暗含了某些未经证明且无根据的假设。识别这类谬误能够帮助我们避免被表面上合理的论证所误导,培养更加批判性的思维方式。}
\end{quotation}

\textbf{预设谬误}通常只有在论证的精确表述中才能显示出来。一段话的作者、讲者,或读者、听者,都有可能会假定某些未经证明的和无根据的前提为真,无论是出于疏忽还是故意设计。而当掩藏在论证里的这种可疑假设对支持结论非常关键时,论证就是糟糕的并可使人陷人误区。这类无根据的跳跃就被称为预设谬误。

在这类论证谬误中,前提也常常与结论不相干。的确,在大多数谬误中都存在前提与结论之间不相干的缺口,但是,预设谬误展示出一种特殊的错误:那种不为人支持甚至是不可支持的\textbf{暗含假定}。要揭露这样的谬误,注意留心那种偷偷溜进的假设及其可疑与虚假性,通常是很奏效的。 
\subsection{P1.复杂问语(Complex Question)}

\begin{theorembox}[title=复杂问语谬误的定义]
\logicwarn{复杂问语谬误}发生在提问的方式预设了某些未被证实的假设为\logicemph{真}的情况。这种\logicwarn{谬误}的目的是使听者或读者接受事先被预设的假设,而无需对这些假设进行质疑或分析。当一个问题含有一个或多个预设,而这些预设尚未被证明或接受时,就犯了\logicwarn{复杂问语谬误}。
\end{theorembox}

\paragraph{经典的复杂问语例子}
\begin{examplebox}[title=经典的复杂问语例子]
最有名的复杂问语例子是:"你什么时候停止殴打你的妻子?"无论对方回答"昨天"、"上个月"或者"去年",他都会承认自己曾经殴打妻子。即使他回答"我从未殴打我的妻子",这种回答也会显得不自然和可疑。问题预设了"他曾经殴打妻子"这一命题,而这一命题正是需要被证实的。
\end{examplebox}

\paragraph{司法询问中的复杂问语}
\begin{examplebox}[title=司法询问中的复杂问语]
在法庭审讯中,律师常常试图诱使证人回答复杂问题。例如,原告的律师可能问被告:"你在事故发生前喝了多少酒?"这种问题预设被告事故前喝了酒。如果证人回答说他没喝酒,律师可以说:"我没问你是否喝了酒,我问你喝了多少。"或者会问:"你多久打你的妻子一次?"而不是先问:"你是否曾经打过你的妻子?"这种询问方式,预设了被告确实曾经殴打过自己的妻子。
\end{examplebox}

\paragraph{政治辩论中的预设问题}
\begin{examplebox}[title=政治辩论中的预设问题]
政治辩论中的问题经常预设争议性的假设为\logicemph{真}。例如,可能会问:"你认为哪些人应该承担我们国家衰退的责任?"这种问题预设了"我们的国家正在衰退",而这个假设可能本身就是争议的焦点。类似地,针对在任官员的问题如"你准备采取什么措施来避免再次犯重大错误?"也预设了官员已经犯了重大错误。
\end{examplebox}

\paragraph{识别和避免复杂问语}
要避免被复杂问语所误导,我们应当始终审视问题中的隐含假设,而不是立即着手回答问题。复杂问题往往可以分解为更基本的问题,例如:"你有妻子吗?如果有,你是否曾经殴打过她?如果是,你什么时候停止这样做的?"这种分解可以使人避免无意中认可未经证实的预设。

复杂问语的效力来自于它的心理压力——它迫使回答者处于不利的防御位置。在逻辑学中,复杂问语谬误属于\logicterm{预设谬误},因为它的错误在于预设了未经证实的命题为\logicemph{真}。识别这类\logicwarn{谬误}的关键是识别问题中所隐含的所有假设,并对其真实性提出质疑。
\subsection{P2.虚假原因(False Cause)}

\begin{theorembox}[title=虚假原因谬误的定义]
显然,把实际上不是某情形或事件的原因当做原因,任何依赖于此的推理就必定是\logicwarn{严重错误的}。但是,我们常常倾向于假设或者被引导假设,我们理解了事实上我们并不理解的某些特定的原因—结果关系。设定一个并不真实存在的因果联系,是一种常见的\logicwarn{错误}。在拉丁语中,这种错误被称为"\logicterm{无因之因}"(non causa pro causa)的\logicwarn{谬误},我们简单地称之为\logicwarn{虚假原因谬误}。
\end{theorembox}

原因与结果之间的联系本性,以及我们怎样确定这样的联系是否存在或缺乏,都是归纳逻辑和科学方法论的中心问题。这些问题在本书的第三部分将进行详细讨论。

\paragraph{因果关系与争议}
\begin{examplebox}[title=因果关系与争议]
所断定的因果联系是否的确\logicwarn{错误},有时可能是有争议的问题。有人辩说,有些大学教员评分宽松,是因为他们担心严格评分会导致学生降低对他们的评价,因而不利于他们的职业。逐渐的"评分膨胀"据说就是这种担心的结果。一位大学教授写道:

现在,很多学校都要求由学生来完成课程评价表,并且薪水受这些结果的影响。30年前,我来密歇根大学时,我的薪水比人类学系任何人的都高,他们今天都还很活跃。我的评分标准没有追随膨胀潮流。学生对评分的抱怨增加了,而现在我的薪水就处在教授工资单的底层。${ }^{[24]}$

你认为这段话犯有\logicwarn{虚假原因谬误}吗?
\end{examplebox}

\paragraph{时间连续性与因果错误}
有时会发生这种现象:人们假定一事件是另一事件的原因,只因为另一事件在时间上紧随着前者。我们当然知道,纯粹的时间连续并不能确证一种因果联系,但是很容易被欺骗。在对外政策中,如果一项挑战性动议之后跟随出现了一件与其并不相关的国际事件,那么有人就可能\logicwarn{错误地}得出结论:挑战性动议就是那个事件的原因。在原始科学中,这样的\logicwarn{错误}是常见的;现在,我们把这种事件作为\logicwarn{荒谬的}说法来拒斥:敲锣打鼓是日食之后太阳重又出现的原因,因为不可否认在日食时每次敲锣打鼓之后太阳的确又会复现。

这类推理\logicwarn{错误}仍然广泛存在:反常的天气状况被归咎于某些发生在前的不相关天象;实际上由病毒引起的感染,却被认为是伤风或湿脚所使然,等等。这种虚假原因被称为"\logicterm{缘出前物}"(post hoc ergo propter $h o c)$;近来,一名通讯员在给《纽约时报》的一封信中就出现了一个这样的例子。他写道:

\begin{quote}
在工业世界中,美国的死刑带给我们的是,每100000人中最高的犯罪率和数量最多的囚犯。${ }^{[25]}$
\end{quote}

当"\logicterm{缘出前物}"非常明显时,它是一种容易发现的\logicwarn{谬误};但是,甚至最伟大的科学家和政治家偶尔也会被它误导。
\input{chapter4/4-3-P3 乞题.tex}
\subsection{P4.和 P5.偶然和逆偶然(Accident and Converse Accident)}

\begin{theorembox}[title=偶然和逆偶然谬误的定义]
\logicterm{偶然}和\logicterm{逆偶然谬误}${ }^{(1)}$,可能出于无心之失,也可能是故意而做出误导他人的概括。在许多重要情形中,尤其在政治或伦理论证中,我们都要依赖关于事物的概括陈述,关于人的概括陈述,等等。但是,即使概括断言完全可行的地方,我们也必须小心不要把它们机械地或僵硬地运用于特殊事例。环境改变事例情况。一个总体上\logicemph{真}的概括,由于给定事例的特殊或偶然环境,可能不能运用于该事例。当我们把一个概括运用于个别事例中而该事例并不适于这种运用时,我们就犯了\logicwarn{偶然谬误}。反之,当我们无心或故意地把对一个特殊事例为\logicemph{真}的东西直接看做对大量事例为\logicemph{真},我们就犯了\logicwarn{逆偶然谬误}。
\end{theorembox}

\paragraph{概括的例外性}
\begin{examplebox}[title=概括的例外性]
经验教导我们,概括,即使是那些广泛合适和有用的概括,往往也有例外,我们必须对之保持警惕。在法律中,总体上良好的规则有时却发现有非常特别的例外。例如,按照法理,传闻信息不能接受为庭审证据,但如果说这番话的一方已死,或报告传闻的一方是在与自身利益有很大冲突的情况下做此报告的,则这条规则就是不适用的。几乎所有良好的法律规则都有其相应的例外;当我们假设某些规则具有普遍的适用性并以该假设进行推理时,我们就可能犯\logicwarn{谬误}论证。
\end{examplebox}

\paragraph{苏格拉底与欧西德姆斯的对话}
\begin{examplebox}[title=苏格拉底与欧西德姆斯的对话]
欧西德姆斯(Euthydemus)想成为一名政治家。在与其对话中,苏格拉底从欧西德姆斯那里获得了后者对很多习俗上接受的伦理真理的承诺,如欺骗是\logicwarn{错误的},偷盗是不正义的,等等。接着,正如色诺芬(Xenophon)在其记载中所详述的那样,苏格拉底提出一系列假设事例,使欧西德姆斯不得不同意欺骗有时(为拯救我们的同胞)是\logicemph{正当的},以及偷盗有时(为挽救一个朋友的生命)是\logicemph{正义的},等等。对那些通过机械地诉诸概括规则来试图决定特定和复杂问题的人来说,\logicwarn{偶然谬误}是一种真实而严重的威胁。逻辑学家约瑟夫(H.W.B.Joseph)观察到"如果对待一个在很多方面都不令人误解的陈述,就好像它总是\logicemph{正确的}没有限制条件的一样,那么没有比这种\logicwarn{谬误}更暗中为害的啦"。
\end{examplebox}

\paragraph{由特殊到一般的错误推理}
\logicwarn{偶然谬误}是当我们轻率地从一个概括转移到(特殊问题)时所犯的\logicwarn{谬误},而\logicwarn{逆偶然谬误}是当我们轻率地(从特殊问题)转移到概括时所犯的\logicwarn{谬误}。我们都熟悉这样一些人,由于某情形对一给定类型的一个或几个人是\logicemph{真的},他们就对那种类型的所有人做出结论。我们知道,并且需要记住,虽然在某些情况下一定的药物或食物可以是无害的,但是,它并不因而在所有的情况下就都是无害的。例如,食用油炸食物总体来说对一个人的胆固醇水平具有不利影响,但是,那种坏结果在某些人身上可能不会出现。近来,英国的一位"炸鱼片和炸土豆条"店主用如下论证为其油炸烹调方法的正当做了辩护:

\begin{logicbox}[title=逆偶然谬误的例子]
以我的儿子马丁为例。他一生一直吃炸鱼片和炸土豆条,他刚进行了胆固醇测试,他的胆固醇水平低于国家平均水平。还有什么比一个油炸食品店主的儿子是更好的证据呢?${ }^{[27]}$
\end{logicbox}

\logicwarn{逆偶然谬误}作为推理\logicwarn{谬误}的一种,其\logicwarn{错误}一旦揭露出来,对每个人来说都是明显易懂的,然而,它却可以用做一种方便的欺骗方法。当人们漫不经心地或充满感情地进行论证时,就很可能落人这种\logicwarn{谬误}的陷阱。

\footnotetext{(1)在中文文献中,这两种谬误亦称为"以全概偏"和"以偏概全"。
}
\section{含混谬误}

\begin{logicbox}[title=引言]
\textit{含混谬误是由词语或短语的意义变化而导致的推理错误,这类谬误可能出于疏忽,也可能是刻意为之。识别和理解这类谬误有助于我们避免在论证中被词语的多义性所误导。}
\end{logicbox}

由于用心不专或故意操作,在论证过程中,词或短语的意义可能会变化。一个词项在前提中可能具有一种意义,但是在结论中却是另一种相当不同的意义。当推论依赖这样的变化时,当然就是谬误。这种错误称做\textbf{含混谬误},有时或称为\textbf{诡论}(sophisms)。故意使用这样的方法常常是粗糙的和易于发现的,但是,有时(虽并非经常)这种含混是隐蔽的、难以把捉的。我们在下面区分出它的五种类型。 
\subsection{A1.歧义}

\begin{theorembox}[title=歧义谬误的定义与机制]
\logicemph{语言的多义性}:大多数词汇都有多于一个的字面意义,但在多数情况下,通过注意语境和利用我们良好的感觉,我们在阅读和听讲时不难将这些意义分辨开来。

\logicwarn{歧义的产生}:但是,当人们有意无意地混淆一个词或短语的几个意义时,就是在\logicterm{歧义}地使用这个词或短语。

\logicterm{歧义谬误}:如果在论证中这样做,就犯了\logicterm{歧义谬误}。

\logicwarn{谬误机制}:
\begin{itemize}
  \item 在论证的不同部分使用同一词汇的不同含义
  \item 利用词汇的多义性来混淆推理
  \item 前提中的词汇含义与结论中的不同
  \item 破坏了论证的逻辑一致性
\end{itemize}

\logicemph{识别关键}:识别歧义谬误的关键在于检查关键词汇在论证中是否保持了一致的含义。
\end{theorembox>

\paragraph{文学中的歧义}
\begin{examplebox}[title=文学中的歧义]
\logicwarn{文学中的巧妙运用}:有时,这种歧义谬误非常明显,在某些玩笑的字里行间使用。

\logicemph{卡罗尔的例子}:刘易斯·卡罗尔(Lewis Carroll)在《爱丽丝镜中奇遇记》(\textit{Through the Looking Glass})中对爱丽丝的奇遇的讲述,就包含着机智和逗乐的歧义。其中一个如下:

\logicwarn{对话内容}:
\begin{quote}
"你们谁走过这条路?"国王继续走着,并向送信人伸出手要些千草。

"没有人(nobody)。"送信人说。

"很对,"国王说,"这位年轻的女子也看到过他(him)。所以,当然 Nobody 比你们走得更慢。"
\end{quote>

\logicemph{歧义分析}:在这段话中,歧义谬误其实用得是相当巧妙的:

\logicwarn{词汇转换过程}:
\begin{itemize}
  \item \logicemph{第一次使用}:"nobody"这个词仅仅是指"没有人"(no person)的意思
  \item \logicemph{代词指称}:接着用代词"他(him)"来指称,就好像"nobody"这个词命名了一个人
  \item \logicemph{名词化}:当相同的词被大写并明显地用做一个名字"Nobody"时,它就显然命名了一个人
  \item \logicemph{特性转移}:这个人具有没有走过这条路的特性,而这个特性又是从该词的第一次运用中得来的
\end{itemize>

\logicemph{文学价值}:有时,歧义是机智的工具,刘易斯·卡罗尔就是一位非常机智的逻辑学家。$^{[28]}$
\end{examplebox}

\paragraph{相对词的歧义问题}
\begin{logicbox}[title=歧义论证的严肃性]
歧义论证总是谬误的,但它们却不总是愚蠢和滑稽的,这一点将在下面节录的例子中看出来。
\end{logicbox}

\begin{examplebox}[title=相对词的歧义问题]
\logicwarn{特殊类型}:有一种歧义谬误特别值得一提。这是一种由错误使用\logicterm{相对性}(relative)词项而来的错误。

\logicemph{相对词的特征}:在不同语境中,相对词具有不同意义。

\logicwarn{"高"的例子}:例如,"高"就是一个相对词:
\begin{itemize}
  \item 高个子人与高建筑物就处于非常不同的类别
  \item 一个高的人是一个比大部分人都高的人
  \item 而一座高的建筑物是一座比大部分建筑物都高的建筑物
\end{itemize}

\logicemph{论证形式的对比}:
\begin{itemize}
  \item \logicterm{有效论证}:"象是动物,因此灰色的象是灰色的动物"这个论证是完全有效的。"灰色"这个词不是相对的。
  \item \logicwarn{无效论证}:"象是动物,因此小象是小动物"这个论证却是荒唐的。
\end{itemize}

\logicwarn{关键分析}:这里的关键之点是,"小"是个相对词:小象是非常大的动物。这个谬误就是一个关于相对词"小"的一种歧义谬误。

\logicemph{更隐蔽的例子}:然而,并非所有的有关相对词的歧义谬误都是这样显然。

\logicwarn{"好"的歧义}:"好"这个词是个相对词,关于它,经常出现歧义谬误:
\begin{itemize}
  \item 有人论证说某某是一位好将军,因此也会成为一位好总统
  \item 或者是一位好学者,从而也一定是一位好教师
\end{itemize}

\logicemph{问题分析}:这些论证忽略了"好"在不同领域中的不同标准和要求。
\end{examplebox>
\subsection{A2.双关}

\begin{theorembox}[title=双关谬误的定义]
由于前提的语法结构原因,会导致前提的表达\logicterm{歧义}。当人们从这样的前提出发来论证时,就会出现\logicterm{双关}(amphiboly)\logicwarn{谬误}。"双关"这个词来源于希腊语,它的意思实质是"一团两个",或一团的"两倍"。一个陈述是\logicterm{双关的},是指由于它的词汇组合松散或笨拙导致它的意义不确定。一个\logicterm{双关}陈述可能在一种解释下可能是\logicemph{真的},而在另一种解释下却是\logicwarn{假的}。当以使其为\logicemph{真}的解释来表述论证前提,而以使其为\logicwarn{假}的解释得出结论时,那么就犯了\logicwarn{双关谬误}。
\end{theorembox}

\paragraph{政治中的双关现象}
\begin{examplebox}[title=政治中的双关现象]
在指导选举策略时,\logicterm{双关}既可以迷惑人也可以误导人。20世纪90年代,当众议员托尼•科埃略(Tony Coelho)作为来自加利福尼亚州的一位民主党员而进入美国白宫代表中时,据报道,他说:"Women prefer Democrats to men."${ }^{(1)}$ \logicterm{双关}陈述构成危险前提,但是,在严肃的话题中人们很少遭遇它。
\end{examplebox}

\paragraph{垂悬分词与短语}
\begin{examplebox}[title=垂悬分词与短语]
文法家所谓的"垂悬"分词和短语经常有娱乐类型的\logicterm{双关}出现。《纽约客》(The New Yorker)中的小栏新闻就曾给粗心忽视了\logicterm{双关}的作者和编辑开了一个讽刺玩笑:

"Leaking badly,manned by a skeleton crew,one infirmity after another overtakes the little ship."${ }^{(2)}$(The Herald Tribune, book section)

这些游戏几乎没有缺陷![30]
\end{examplebox}

\footnotetext{(1)这句话可以有两种解释:女人比男人更喜欢民主党;女人更喜欢民主党而不是男人。\\
(2)这句话可有两种解释:小船突然出现严重泄漏、配备人员最少等一个接一个的缺陷;严重泄漏、配备人员最少等一个接一个的(游戏)缺陷,突然降临小船。前者是指小船的缺陷,而后者意指游戏自身的缺陷。
}
\subsection{A3.重读}

\begin{theorembox}[title=重读谬误的定义]
\logicwarn{基本机制}:当论证的意义变化源于对其词汇或组成部分的强调的变动时,该论证就可以证明是欺骗性的和无效的。

\logicterm{重读谬误}:若前提的明显意义依赖于一个可能的强调,但是,得出的结论却依赖于对相同词汇不同的重读意义,这时就犯了\logicterm{重读}(accent)谬误。

\logicemph{谬误特征}:
\begin{itemize}
  \item 通过改变语音重音或视觉强调来改变意义
  \item 在前提和结论中使用不同的强调方式
  \item 利用强调的变化来误导听众或读者
  \item 破坏了论证的逻辑一致性
\end{itemize}

\logicwarn{表现形式}:
\begin{itemize}
  \item 口语中的语音重音变化
  \item 书面语中的字体大小、颜色、位置变化
  \item 标点符号的不同使用
  \item 上下文环境的操纵
\end{itemize}
\end{theorembox}

\paragraph{重音与语气的影响}
\begin{examplebox}[title=重音与语气的影响]
\logicwarn{语音重音的例子}:作为示例,请考虑我们可以把不同的意义给予如下陈述:

\begin{quote}
我们不应当说朋友的坏话(We should not speak ill of our friends)。
\end{quote}

\logicemph{不同重音的含义}:
\begin{itemize}
  \item 强调"我们":暗示别人可以说朋友的坏话
  \item 强调"不应当":强调道德义务
  \item 强调"朋友":暗示可以说非朋友的坏话
  \item 强调"坏话":暗示可以说朋友的好话
\end{itemize}

\logicwarn{印刷媒体中的误导}:在印刷字体及图片方面,有很多伎俩常常是通过强调某处而起误导之效。

\logicemph{新闻标题的操纵}:
\begin{itemize}
  \item 出现在新闻报道标题中的大号字敏感词汇,故意向那些匆匆浏览的人暗示错误的结论
  \item 该标题后面却很可能用其他词汇以很小的字来加以限制
  \item 为避免在看新闻报道或在签订合同时被欺骗,我们力劝人们注意"小字印刷"
\end{itemize}

\logicwarn{政治宣传中的应用}:在政治宣传中,特别是在声称所谓事实报道中:
\begin{itemize}
  \item 选择令人误解的敏感标题
  \item 选择使用部分省略的图片
  \item 都是对重读谬误的精心使用,力图使读者得出宣传者明知为假的结论
\end{itemize}

\logicemph{歪曲与谎言的区别}:解说可能不是彻底的谎言,但它也可以利用故意或虚假的重读方式来歪曲事实。
\end{examplebox}

\paragraph{广告中的重读谬误}
\begin{examplebox}[title=广告中的重读谬误]
\logicwarn{广告中的普遍现象}:在广告中,这样做的也很多。

\logicemph{价格广告的伎俩}:
\begin{itemize}
  \item 非常低的价格往往以非常大的字出现
  \item 而后面却跟随着字体极小的"以及完全说明"
\end{itemize}

\logicwarn{机票广告的例子}:
\begin{itemize}
  \item 飞机票价打折的通告后面都跟有一个星号
  \item 以远远的一个脚注说明该价格仅仅可用于提前三个月预订星期四的飞行航班
  \item 或可能还会有其他"适用限制"
\end{itemize}

\logicemph{商品广告的策略}:
\begin{itemize}
  \item 名牌昂贵商品都以非常低的价格做广告
  \item 在广告某处附有一个小注解"所列价格存货数量有限"
  \item 读者被吸引到商店,但可能以广告价格买不到商品
\end{itemize}

\logicwarn{谬误的界定}:
\begin{itemize}
  \item 重读语段本身并不是严格谬误
  \item 源于重读的语段解释,当它依赖一个非常可疑的结论暗示时,即当其采用令人误解的重读来解释时,重读语段就变成了谬误
  \item 例如:飞机票或品牌商品可以按照所列价格优先购买
\end{itemize}
\end{examplebox}

\paragraph{通过位置操纵的重读}
\begin{examplebox}[title=通过位置操纵的重读]
\logicwarn{位置操纵的威力}:甚至字面上为真的语段,也可以通过操纵其位置而以重读来欺骗人。

\logicemph{船长与助手的故事}:
\begin{itemize}
  \item \logicwarn{背景}:一位船长厌恶他的首席助手上班时再三喝醉
  \item \logicemph{船长的记录}:在该船的航行日记上,他几乎每天都记上:"助手今天喝醉了。"
  \item \logicwarn{助手的报复}:愤怒的助手进行报复。一天,船长病了,助手就自己保管日志
  \item \logicemph{助手的记录}:他在上面记着:"船长今天清醒了。"
\end{itemize}

\logicwarn{重读效果分析}:
\begin{itemize}
  \item 船长的记录暗示助手经常喝醉(通过频繁记录)
  \item 助手的记录暗示船长经常不清醒(通过强调"今天清醒了")
  \item 两个记录在字面上都可能是真实的
  \item 但通过上下文和频率的操纵,传达了误导性的印象
\end{itemize}

\logicemph{教训}:这个例子说明了语境和时机在传达信息中的重要作用,以及如何通过巧妙的位置安排来改变信息的含义。
\end{examplebox}
\subsection{A4.合成}

\begin{theorembox}[title=合成谬误的定义]
\logicwarn{合成谬误}是不正当地从部分到整体的推理,我们可以区分两类\logicwarn{错误}。第一类\logicwarn{合成谬误}是从部分的性质\logicwarn{错误地}推出整体的性质。每一个砖块都很轻,所以由砖块砌成的墙很轻,这就是这种\logicwarn{合成谬误}的简单形式。这种\logicwarn{谬误}可能完全显而易见,因而不会欺骗任何人,但是,也有些推论,虽然犯同样的\logicwarn{错误},却能导致\logicemph{正确的}结论,例如:"每个砖块都是红的,所以砌成的墙是红的。"这样,属上次推理与此次推理存在一种模式相似性,却一个是非常\logicwarn{错误的}另一个却是\logicemph{合理的},这既有趣也有啓发性。
\end{theorembox}

\paragraph{性质的转移问题}
\begin{examplebox}[title=性质的转移问题]
使这些推理看起来类似的形式是:每个部分都具有性质 P,所以整体也具有性质 P。归纳出来的规则是:判定一个特定的推理形式是否会犯\logicwarn{合成谬误},取决于所涉及的具体性质(这里是 P)。那些只为部分所有,却不为整体所有的性质,转移到整体上就是\logicwarn{合成谬误},而那些为部分所有,也可能为整体所有的性质,转移到整体上则可能是\logicemph{正当的}。例如"轻"这个性质只属于部分,不属于整体,所以就出现\logicwarn{谬误};而红色这个性质,属于部分,也可能属于整体,所以该推论不是\logicwarn{谬误}。为了辨识可能的\logicwarn{合成谬误},我们必须研究并理解特定性质能否从部分转移到整体。
\end{examplebox}

\paragraph{个体与集体的性质差异}
\begin{examplebox}[title=个体与集体的性质差异]
第二类\logicwarn{合成谬误}是不正当地从所有个别成员的性质(这些成员彼此分离或做单独考虑时)到该集体做一个整体时的性质的推理。例如,某位运动员观察到的事实:任何足球队员,单独看时都容易被击垮,因而断言任何足球队都可以轻易地被击垮,他就犯了这种\logicwarn{合成谬误}。团体有时候所具有的凝聚力是其个别成员所不具有的;因此单独个体的易碎性(可击垮性)不是一组受训练选手的特性。所以,从个别成员的特性到整体的特性的推理,确实容易误入歧途。那些归因于大学、公司、军队或体育团队的特性,不能轻易地由所有个别而分离的成员的特性来推出。
\end{examplebox}

\paragraph{两种谬误形式的区别}
\begin{examplebox}[title=两种谬误形式的区别]
这两类\logicwarn{合成谬误}虽然是平行的,但却是根本上有别的,因为元素的纯粹汇集与那些元素所构成的整体是不同的。例如,机器的各部分的纯粹汇集不是机器;砖头的纯粹汇集既不是房子也不是墙壁。整体,比如机器、房子或墙壁,是将其部分以某种特定方式组织或安排起来的。正由于组织的整体与纯粹的汇集是截然不同的,所以这两种形式的\logicwarn{合成谬误}也是如此,一种是从部分到整体的\logicwarn{无效}推广,另一种是从分子或元素到汇集的\logicwarn{无效}推广。
\end{examplebox}
\subsection{A5.分解}

\begin{theorembox}[title=分解谬误的定义]
\logicterm{分解谬误}:是合成谬误的简单颠倒;在分解谬误中,存在相同的混淆,但推论是以相反方向进行的。

\logicwarn{两种分解谬误}:与合成的情形相应,我们也可以区分出两种分解谬误。

\logicemph{第一种分解谬误}:断言对一个整体为真的东西一定对它的部分也真。

\logicwarn{公司官员的例子}:
\begin{itemize}
  \item \logicemph{前提}:某公司非常重要,并且某先生是那个公司的官员
  \item \logicwarn{错误结论}:因此某先生就是非常重要的
  \item \logicemph{谬误分析}:这个论证就犯了分解谬误
\end{itemize}

\logicwarn{机器部件的例子}:从某机器沉重、复杂或者贵重这个前提而得出该机器的任何部分都一定沉重、复杂或者贵重,这个结论也属于分解谬误。

\logicemph{房间大小的例子}:一个学生一定住着一个大房间,因为该房间位于一座大楼中,这也是这种分解谬误的实例。

\logicwarn{核心错误}:
\begin{itemize}
  \item 错误地假设整体的性质必然属于部分
  \item 忽略了整体与部分之间的质的差异
  \item 混淆了不同层次的属性归属
  \item 违反了系统论的基本原理
\end{itemize}
\end{theorembox}

\paragraph{从整体到部分的错误推理}
\begin{examplebox}[title=从整体到部分的错误推理]
\logicwarn{第二种分解谬误}:是从元素的汇集性质而得出元素自身的性质。

\logicemph{大学生专业的例子}:
\begin{itemize}
  \item \logicwarn{前提}:大学生学习医学、法律、工程、牙科和建筑学
  \item \logicwarn{错误结论}:所以任何大学生都学习医学、法律、工程、牙科和建筑学
  \item \logicemph{谬误分析}:这个论证就犯了这种分解谬误
\end{itemize}

\logicwarn{汇集与分布的区别}:
\begin{itemize}
  \item \logicemph{汇集地看}:大学生学习所有这些科目是真的
  \item \logicwarn{分布地看}:大学生学习所有这些科目却是假的
\end{itemize}

\logicemph{谬误的欺骗性}:这种分解谬误的例子常常看起来好像是有效论证,因为对一个类分布地为真的东西,肯定对其每一成员也是真的。

\logicwarn{概念澄清}:
\begin{itemize}
  \item \logicterm{汇集性质}:整个群体作为一个整体所具有的性质
  \item \logicterm{分布性质}:群体中每个成员都具有的性质
  \item \logicterm{集合谬误}:将汇集性质错误地归属于个体成员
  \item \logicterm{量词混淆}:混淆"所有"与"每个"的不同含义
\end{itemize}
\end{examplebox}

\paragraph{有效推理与分解谬误的区别}
\begin{examplebox}[title=有效推理与分解谬误的区别]
\logicemph{有效推理的例子}:

\begin{quote}
狗是肉食的。

阿富汗猎犬都是狗。

因此,阿富汗猎犬都是肉食的。
\end{quote}

\logicwarn{有效性分析}:这个论证是有效的,因为:
\begin{itemize}
  \item "狗是肉食的"是一个分布性陈述,适用于所有狗
  \item 阿富汗猎犬属于狗这个类别
  \item 因此可以正确地推出阿富汗猎犬具有狗的分布性质
\end{itemize}

\logicemph{区别关键}:
\begin{itemize}
  \item \logicterm{有效推理}:从类的分布性质推向成员的性质
  \item \logicwarn{分解谬误}:从类的汇集性质推向成员的性质
\end{itemize}

\logicwarn{判断标准}:
\begin{itemize}
  \item 检查前提中的性质是分布性的还是汇集性的
  \item 分析该性质是否可以合理地归属于个体成员
  \item 考虑整体与部分之间的逻辑关系
  \item 避免混淆不同层次的属性归属
\end{itemize}

\logicemph{实践指导}:
\begin{itemize}
  \item 明确区分集体属性和个体属性
  \item 注意量词的准确使用
  \item 分析属性的可传递性
  \item 考虑系统的层次结构
\end{itemize}
\end{examplebox}
\chaptersummary{
在本章中,我们看到,\logicterm{谬误}是那种看起来\logicemph{正确}但经过考察而证明并非如此的论证。我们对常见的欺骗性推理\logicwarn{错误}类型给出了传统名称,区分出三大类非形式\logicwarn{谬误}:\logicterm{相干谬误}、\logicterm{预设谬误}和\logicterm{含混谬误}。
}

\subsection*{相干谬误}
在这类\logicwarn{谬误}中,\logicwarn{错误}论证依赖于看起来可能与结论相关但事实上无关的前提。我们分七种\logicterm{相干谬误}来解释这类推理\logicwarn{错误}。

\paragraph{R1.诉诸无知论证}
当以一命题没有被证明是\logicwarn{假的}为理由来论证该命题是\logicemph{真的},或当论证一命题是\logicwarn{假的}因为它没有被证明是\logicemph{真的}。

\paragraph{R2.诉诸不当权威}
一个论证的前提诉诸某方或多方判断,而它或它们却不能合法地声称对手头问题具有权威。

\paragraph{R3.人身攻击论证}
攻击不是针对所做的主张或针对论证的优点,而是针对对手本身。

人身攻击论证有两种形式。当攻击直接针对人,以寻求诋毁和侮辱他们时,就称做"\logicterm{诽谤性人身攻击论证}"。当攻击间接地对准人,暗示他们坚持他们的观点主要是因为他们的特殊环境或利益时,就称做"\logicterm{背景性人身攻击论证}"。

\paragraph{R4.诉诸情感}
细心推理被激起狂热或情感来支持预先结论的精心策划所取代。

\paragraph{R5.诉诸同情}
细心推理被激起听者同情来达到说者所关注目标的精心策划所取代。

\paragraph{R6.诉诸武力}
为了得到对某些结论的承诺,细心推理被直接或含沙射影的威胁所取代。

\paragraph{R7.不相干结论}
前提不得要领,声称支持一个结论而事实上却支持或证实另一个结论。

\subsection*{预设谬误}
在这类\logicwarn{谬误}中,\logicwarn{错误}论证源于依赖于某些被假定为\logicemph{真}的命题,而这些命题实际上是\logicwarn{假的}、可疑的或没有得到证明的。我们分五种\logicterm{预设谬误}来解释这类推理\logicwarn{错误}。

\paragraph{P1.复杂问语}
以问句预设了某些假设为\logicemph{真}的方式来询问问题。

\paragraph{P2.虚假原因}
把一个东西当做一个事物的原因而它实际上并不是那个事物的原因,或更一般地说,在以因果关系为基础的推理中犯\logicwarn{错}。

\paragraph{P3.乞题}
在某个论证前提中假定了结论要寻求确证的东西。

\paragraph{P4.偶然}
把某个概括运用于它不能适当管辖的个别情况。

\paragraph{P5.逆偶然}
粗心大意地从单个情况转移到一个无辩护余地的广泛概括。

\subsection*{含混谬误}
在这类\logicwarn{谬误}中,\logicwarn{错误}论证的形成方式是,它依赖于词或短语从在前提中的用法到在结论中的用法的意义变化。我们分五种\logicterm{含混谬误}来解释这类推理\logicwarn{错误}。

\paragraph{A1.歧义}
在论证的明确表述中,有意或无意地使用同一个词或短语的两个或更多意义。

\paragraph{A2.双关}
因为陈述中的词或短语结合得松散或笨拙,论证中的这个陈述具有多于一个合理意义。

\paragraph{A3.重读}
意义的变化作为对论证的词或短语的强调改变的结果而源于该论证之内。

\paragraph{A4.合成}
(a)\logicwarn{错误地}从部分性质到整体性质进行推理,(b)或者,\logicwarn{错误地}从某汇集的个别分子性质到整个汇集的性质进行推理。

\paragraph{A5.分解}
(a)\logicwarn{错误地}从整体性质到它的一个部分的性质进行推理,(b)或者,\logicwarn{错误地}从某些实体汇集的某个全体性质到该汇集的个别实体性质进行推理。

\begin{center}
\fbox{\parbox{0.9\textwidth}{
  \centering
  \textbf{谬误分类总结}\\
  \logicterm{相干谬误}:前提与结论不相干,但表面上看似有关\\
  \logicterm{预设谬误}:依赖未经证实或可疑的假设\\
  \logicterm{含混谬误}:依赖词语或短语意义的变化\\
}}
\end{center}

% 参考文献将在主文档末尾统一显示

% 第五部分
\chapter{演绎推理}
\section{演绎理论}

\begin{quotation}
\textit{演绎推理是逻辑学的核心内容,它探讨如何从已知前提得出必然的结论。本章将介绍演绎理论的基本概念和方法,帮助读者理解逻辑推理的本质和规则。}
\end{quotation}

\subsection{演绎论证的本质}

前面几章探讨的主要是语言及其对论证的影响,现在我们来讨论论证本身。首先来分析一种特殊的论证——\textbf{演绎}。\textbf{演绎论证}是这样一种论证,其前提被要求为结论的真提供决定性基础。如果前提之真确实能够决定其结论为真,那么,这个论证就是\textbf{有效的}。任何一个演绎论证都或者有效或者无效:如果不可能出现前提真而结论假的情况,那么论证就是有效的,否则就是无效的。

\textbf{演绎理论}旨在阐明有效论证的前提与结论之间的关系,为评估演绎论证提供方法。也就是说,演绎理论要给出区别有效演绎与无效演绎的方法。为此,历史上出现了两种杰出的理论。第一种被称为\textbf{古典逻辑}(classical)或\textbf{亚里士多德型逻辑},开创这种理论的是古希腊大哲学家亚里士多德。另一种称为\textbf{现代逻辑}或\textbf{现代符号逻辑}。本章与接下来的两章(即$5、6、7$三章)主要探讨古典逻辑问题,而$8、9、10$三章主要探讨现代逻辑问题。

\subsection{亚里士多德的逻辑贡献}

亚里士多德是古代伟大智者之一。在柏拉图学园钻研20年之后,他成为亚历山大大帝的家庭教师,后来建立了自己的学园:Lyceum(吕克昂),在那里他做出了许多杰出贡献,几乎涵盖了人类知识的所有领域。亚里士多德去世以后,他关于推理的论述被收集成册,称为《工具论》(Organon)。虽然一直到公元2世纪"逻辑"这个词才获得它的现代含义,但逻辑学的主题早已在《工具论》中确定了。 

\begin{center}
\fbox{\parbox{0.95\textwidth}{
\textbf{本节要点}
\begin{itemize}
\item \textbf{演绎论证}是前提为结论的真提供决定性基础的论证
\item 论证的\textbf{有效性}:不可能前提为真而结论为假
\item \textbf{演绎理论}:阐明有效论证中前提与结论之间的关系,提供评估方法
\item 历史上两种主要的演绎理论:
  \begin{itemize}
  \item \textbf{古典逻辑}(亚里士多德型逻辑)
  \item \textbf{现代逻辑}(现代符号逻辑)
  \end{itemize}
\item 亚里士多德在《工具论》中奠定了逻辑学的基础,涵盖了广泛的知识领域
\end{itemize}
}}
\end{center} 
\section{直言命题及其类别}

\begin{logicbox}[title=引言]
\textit{直言命题是演绎理论的基石,它们关注类与类之间的关系。本节将介绍直言命题的基本类型及其特征,为理解演绎论证奠定基础。}
\end{logicbox}

\begin{theorembox}[title=直言命题的基本概念]
\logicemph{研究焦点}:亚里士多德对演绎的研究主要集中在由一种特殊命题组成的论证上,这种命题是关于范畴(categories)和类(classes)的,被称为\logicterm{直言命题}(categorical proposition)。

\logicwarn{重要地位}:直言命题是演绎理论的基石。要了解这种关于类的演绎理论,必须首先对直言命题进行非常精细的分析。

\logicemph{典型例子}:请考虑如下论证:

\begin{quote}
没有运动员是素食主义者,所有足球队员都是运动员,

所以,没有足球队员是素食主义者。
\end{quote}

\logicwarn{结构分析}:
\begin{itemize}
  \item 这个论证中的三个命题都是直言命题,包括两个前提、一个结论
  \item 这些命题肯定或否定某个类$\boldsymbol{S}$全部或部分地包含于另一个类$\boldsymbol{P}$之中
  \item 三个命题涉及的是运动员的类、素食者的类和足球队员的类
\end{itemize}

\logicemph{核心特征}:直言命题表达的是类与类之间的包含、排斥或部分重叠关系。
\end{theorembox}

\begin{theorembox}[title=类之间的关系]
\logicemph{类的定义}:有关类的知识在第3章讨论定义时已经简要地说明,一个\logicterm{类}就是共有某种特定属性的所有对象(objects)的汇集。

\logicwarn{关系类型}:两个类之间有着多种不同的关系:

\logicemph{三种基本关系}:
\begin{enumerate}
  \item \logicterm{完全包含关系}:如果一个类的所有元素(member)都是另一个类的元素,例如狗的类与哺乳动物的类,则称第一个类包含于(be included)或包括在(be contained)第二个类之中

  \item \logicterm{部分包含关系}:如果一个类中有元素是另一个类的元素,但并非其所有元素都是另一个类的元素,例如女人的类和运动员的类,则称第一个类部分地包含于第二个类之中

  \item \logicterm{相互排斥关系}:如果两个类没有共同的元素,例如三角形的类和圆形的类,则称这两个类之间是相互排斥(exclude)的
\end{enumerate}

\logicwarn{重要性}:这些关系是直言命题表达的核心内容,理解它们对于掌握演绎推理至关重要。

\logicemph{图示理解}:
\begin{itemize}
  \item 完全包含:一个圆完全在另一个圆内部
  \item 部分包含:两个圆部分重叠
  \item 相互排斥:两个圆完全分离
\end{itemize}
\end{theorembox}

\begin{theorembox}[title=直言命题的四种标准形式]
\logicemph{形成原理}:类与类之间的这些关系被直言命题所肯定或否定,其结果是恰好能形成直言命题的四种标准形式。

\logicwarn{标准例示}:可分别由如下标准命题例示:

\begin{enumerate}
  \item 所有政客是说谎者。
  \item 没有政客是说谎者。
  \item 有政客是说谎者。
  \item 有政客不是说谎者。
\end{enumerate}

\logicemph{分析维度}:这四种形式可以从两个维度来理解:
\begin{itemize}
  \item \logicterm{量的维度}:全称(所有/没有)vs. 特称(有些)
  \item \logicterm{质的维度}:肯定(是)vs. 否定(不是)
\end{itemize}

\logicwarn{系统性}:下面我们就细致地考察直言命题这四种标准形式。
\end{theorembox}

\begin{examplebox}[title=全称肯定命题(A命题)]
\logicemph{典型例子}:第一个例子——所有政客是说谎者——是一个\logicterm{全称肯定命题}。

\logicwarn{结构分析}:
\begin{itemize}
  \item 其中涉及两个类,即政客的类和说谎者的类
  \item 它说的是第一个类包含于或包括在第二个类中
  \item 全称肯定命题断言第一个类中所有元素都是第二个类的元素
\end{itemize}

\logicemph{术语说明}:
\begin{itemize}
  \item 主项"政客"指称(designate)政客的类
  \item 谓项"说谎者"指称说谎者的类
\end{itemize}

\logicterm{标准形式}:所有全称肯定命题都可以写成如下形式:

\begin{center}
所有$S$是$P$。
\end{center}

其中字母$S$和$P$分别代表主项和谓项。

\logicwarn{名称合理性}:"全称肯定命题"这一名称是恰当的,因为:
\begin{itemize}
  \item 这个命题肯定了两个类之间的包含于关系
  \item 并且是完全或者说全部包含于关系
  \item 断言$S$的所有元素同时都是$P$的元素
\end{itemize}
\end{examplebox}

\begin{examplebox}[title=全称否定命题(E命题)]
\logicemph{典型例子}:第二个例子——没有政客是说谎者——是一个\logicterm{全称否定命题}。

\logicwarn{逻辑含义}:
\begin{itemize}
  \item 它是对全部政客而言,否定他们是说谎者
  \item 就这样两个类来说,全称否定命题断言第一个类与第二个类是完全排斥的
  \item 也就是说第一个类中没有元素是第二个类的元素
\end{itemize}

\logicterm{标准形式}:所有全称否定命题都可以写成如下形式:

\begin{center}
没有$S$是$P$。
\end{center}

其中$S$和$P$也分别代表主项和谓项。

\logicwarn{名称合理性}:"全称否定命题"这一名称是恰当的,因为:
\begin{itemize}
  \item 这个命题否定了这两个类之间的包含于关系
  \item 并且是全部否定
  \item 断言在$S$的所有元素中,没有一个是$P$的元素
\end{itemize}

\logicemph{注释}:我国逻辑教材中一般把全称否定命题的形式写为:"所有$S$不是$P$",其与"没有$S$是$P$"同义。但根据英语语法,"All $S$ are not $P$"并不与"No $S$ are $P$"同义,而等义于"Not all $S$ are $P$"。故英文著作一般将"No $S$ is(are)$P$"作为全称否定命题的形式。
\end{examplebox}

\paragraph{特称肯定命题}
第三个例子—有政客是说谎者——是一个\textbf{特称肯定命题}。显然,这个例子肯定的是政客类中有元素(也)是说谎者类的元素。但并没有对政客类作全部断言:它说的并不是所有政客,而是某个或某些政客是说谎者。此命题既没有肯定也没有否定所有政客是说谎者,对此并没有给出主张。从字面含义看,它并没有断言有政客不是说谎者,尽管在某些语境中它可能暗含这样的意思。这个命题的字面含义或者说最小的(minimal)解释,即政客的类和说谎者的类之间有某个或某些元素是共同的。为确定性起见,我们这里采取最小解释。

"有"(some)这个词的含义是不确定的。它指的是"至少有一个"、"至少有两个",还是"至少有一百个"呢?到底有多少个?尽管与某些场合中的通常用法不太一致,但为了保持确定性,我们一般把"有"看做"至少有一个"的意思。这样,特称肯定命题可以写成如下形式:

有$S$是$P$。

它断言的是,主项$\boldsymbol{S}$指称的类中至少有一个元素是谓项$\boldsymbol{P}$指称的类的元素。"特称肯定命题"这个名称是恰当的,因为这种命题肯定了类之间具有某种包含于关系,但不是全部而只是部分地(partially)肯定第一个类${ }^{(1)}$中的某个或某些元素包含于第二个类。

\paragraph{特称否定命题}
第四个例子——有政客不是说谎者———是一个\textbf{特称否定命题}。这个例子,正如上面的例子一样,谈论的并不是全部政客,而只是政客类中某个或某些元素,因而是特称的。不同于第三个例子的是,它并非肯定第一类中的某部分包含于第二个类中,相反,它是否定的。所有特称否定命题可以写成如下形式:

有$S$不是$P$。

它断言的是,主项$\boldsymbol{S}$指称的类中至少有一个元素被谓项$\boldsymbol{P}$指称的类的全体所排斥。

\subsection{标准式直言命题的多样性}

并非所有标准式直言命题都像以上四个例子那样简单明了。标准式命题的主项、谓项指称的都是类,但这些词项可能是复杂的表达式而非一个单词。举例来说,在命题"所有这个职位的候选人都是诚实而正直的人"中,主项是"这个职位的候选人",谓项是"诚实而正直的人"。

\begin{theorembox}[title=直言命题的传统符号系统]
\logicemph{历史观点}:曾经有一种传统观点,认为所有演绎论证都可以用类或范畴以及它们之间的关系加以分析。

\logicwarn{基石地位}:这样,如上说明的直言命题的四种标准形式,就被认为是所有演绎论证的基石:

\logicterm{四种标准符号}:
\begin{itemize}
  \item \logicterm{A命题}:全称肯定命题(所有$S$是$P$)
  \item \logicterm{E命题}:全称否定命题(没有$S$是$P$)
  \item \logicterm{I命题}:特称肯定命题(有$S$是$P$)
  \item \logicterm{O命题}:特称否定命题(有$S$不是$P$)
\end{itemize}

\logicemph{符号来源}:这些字母来自拉丁词:
\begin{itemize}
  \item A和I来自"Affirmo"(我肯定)
  \item E和O来自"Nego"(我否定)
\end{itemize}

\logicwarn{理论价值}:尽管这种传统观点是不正确的,但的确有许多逻辑理论——正如我们将要看到的——就是以这四种命题为基础建立起来的。

\logicemph{现代意义}:这个符号系统至今仍在逻辑学教学和研究中广泛使用。
\end{theorembox}

\footnotetext{(1)我国逻辑教材中一般把全称否定命题的形式写为:"所有$S$不是$P$",其与"没有$S$是$P$"同义。但根据英语语法,"All $S$ are not $P$"并不与"No $S$ are $P$"同义,而等义于"Not all $S$ are $P$"。故英文著作一般将"No $S$ is(are)$P$"作为全称否定命题的形式。}

\footnotetext{(1)括号内的话为译者所加。}

\chaptersummary{
\logicterm{直言命题}是关于范畴和类的特殊命题,是演绎理论的基石。类之间存在\logicemph{完全包含}、\logicemph{部分包含}和\logicemph{互相排斥}三种基本关系。直言命题有四种标准形式:\logicterm{全称肯定命题}(A命题)、\logicterm{全称否定命题}(E命题)、\logicterm{特称肯定命题}(I命题)和\logicterm{特称否定命题}(O命题)。
}
\section{质、量与周延性}

\begin{quotation}
标准式直言命题的分析需要掌握其基本特征。本节将详细介绍直言命题的质、量以及词项的周延性,这些概念对于理解命题间的关系和评估推理的有效性至关重要。
\end{quotation}

\subsection{质}
每个标准式直言命题或是\textbf{肯定的}或是\textbf{否定的},这叫做命题的\textbf{质}。如果一个命题肯定了类与类间的包含于关系,不管是全部地还是部分地肯定,那么,它的质就是肯定的。因此\textbf{全称肯定命题}和\textbf{特称肯定命题}的质都是肯定的。它们的简写名称,即 \textbf{A} 和 \textbf{I},分别来自于拉丁文"AffIrmo",该词的意思是"我肯定"。如果一个命题否定类与类间的包含关系,不管是全部地还是部分地否定,那么,它的质就是否定的。因此\textbf{全称否定命题}和\textbf{特称否定命题}的质都是否定的。它们的简写名称,即 \textbf{E} 和 \textbf{O},分别来自于拉丁文"nEgO",该词的意思是"我否定"。

\subsection{量}
每个标准式直言命题或是\textbf{全称的}或是\textbf{特称的},这称为直言命题的\textbf{量}。如果一个命题述及主项所指称的类的所有元素,那么,它的量就是全称的。因此 \textbf{A 命题}和 \textbf{E 命题}的量都是全称的。如果一个命题只述及主项所指称的类的某些元素,那么,它的量就是特称的。因此 \textbf{I 命题}和 \textbf{O 命题}的量都是特称的。

每个标准式直言命题都以"\textbf{所有}"、"\textbf{没有}"或者"\textbf{有}"等词开头,这些词表明了命题的量。"所有"和"没有"表示命题是全称的,"有"表示命题是特称的。另外,"没有"还表明了 E 命题的质是否定的。

我们发现"全称肯定"、"全称否定"、"特称肯定"和"特称否定"这几个名称都是先描述量再描述质,从而唯一地描述了每一种标准式直言命题。

\subsection{标准式直言命题的一般模式}
每个标准式直言命题的主项、谓项之间都有一个动词形式"\textbf{是}"(O命题需在"是"前面加上一个"不"字),这个动词把主项和谓项联结起来,称为\textbf{联项}(copula)。前一节给出的公式中的联项只有"是"和"不是"两种,但依据不同的措辞需要,有时可能用其他形式的联项更为适当。例如,下面三个命题中:

\begin{quote}
有罗马统治者曾经是(were)独裁者。\\
所有正方形均为(are)四边形。\\
有士兵不会成为(will not be)英雄。
\end{quote}

联项分别是"曾经是"、"均为"和"不会成为"。标准式直言命题的一般模式由四个部分组成:首先是\textbf{量项},其次是\textbf{主项},再次是\textbf{联项},最后是\textbf{谓项}。可以记为:

量项(主项)联项(谓项)。

\subsection{周延性}
基于类的解释,标准式直言命题指称的都是对象的类,而命题被看做是关于这些类的。当然,命题谈及类的方式不尽相同。一个命题可能谈及一个类的全部元素,也可能只谈及这个类的一些元素。这样一来,下面这个命题:

\begin{quote}
所有参议员是公民。
\end{quote}

述及或曰关乎全部参议员,但没有述及所有公民。它断定的是参议员类的任何一个元素都是公民,但并没有就所有公民做出断言。它既没有肯定,也没有否定所有公民都是参议员。这样一来,任何一个有如下形式的 A命题:

所有$S$是$P$。

都述及了主项 $S$ 指称的类的全部元素,但并没有述及谓项 $P$ 指称的类的全部元素。

我们引入"\textbf{周延}"这个技术性术语,用以刻画出现于直言命题中的主谓项的性质。如果一个命题述及了某个词项所指称的类的全部元素,则称该词项在这个命题中是\textbf{周延的}。我们来考察一下各种标准式直言命题,看看其中哪些词项周延、哪些词项不周延。

\subsection{各类命题的周延性分析}

\subsubsection{A命题的周延性}
首先来看 A 命题。仍以"所有参议员都是公民"为例。A 命题的主项(在命题中)是\textbf{周延的},而谓项(在命题中)是\textbf{不周延的}。

\subsubsection{E命题的周延性}
接下来看 E 命题,比如:

\begin{quote}
没有运动员是素食主义者。
\end{quote}

这样的 E 命题,断定了任何一个运动员都不是素食主义者。整个运动员的类都被排除在素食主义者的类之外。由于 E 命题述及了主项指称的类的全部元素,因此可以说 E 命题的主项是周延的。同时,由于断定了整个运动员的类被排除在素食主义者的类之外,这个命题也就断定了整个素食主义者的类也被排除在整个运动员的类之外。上述例句显然断定了任何一个不是运动员的素食者,因此,它就涉及了谓项指称的类的全部元素,所以说,E命题的谓项也是周延的。总之,E命题的主项周延,谓项也周延。

\subsubsection{I命题的周延性}
说到 I 命题,情况就有所不同了。例如:

\begin{quote}
有士兵是胆小鬼。
\end{quote}

既没有对所有士兵进行断定,也没有对所有胆小鬼进行断定。不能说一个类完全包含于另一个类之中,也不能说完全排除在外。在任何特称肯定命题中,主项、谓项都是\textbf{不周延的}。

\subsubsection{O命题的周延性}
特称否定命题或者说 O 命题与特称肯定命题一样,主项不周延。例如:

\begin{quote}
有些马不是良种马。
\end{quote}

并不言说所有的马,而只述及主项指称的类的一些元素。它说的是所有马中被排除在良种马之外的那一部分,亦即这部分被排除在后一个类的全体之外。假如谈的只是特定的这部分马,那么,任何一个是良种马的元素都不在这部分之中。说某事物被排除在一个类之外,也就述及了这个类的全部。正像说一个人被排除在某个国家之外,就等于说这个国家的任何地方都不接纳此人一样。特称否定命题的谓项是\textbf{周延的},但主项\textbf{不周延}。

\subsection{周延性总结}
周延性问题可以总结如下:\textbf{全称命题},包括肯定的和否定的,其主项是周延的,而\textbf{特称命题},不管是肯定的还是否定的,其主项都是不周延的。也就是说,标准式直言命题的\textbf{量决定了主项的周延情况}。\textbf{肯定命题},无论全称的还是特称的,其谓项都是不周延的,而\textbf{否定命题},包括全称的和特称的,其谓项都是周延的。也就是说,标准式直言命题的\textbf{质决定了谓项的周延情况}。

\begin{center}
\includegraphics[width=\textwidth]{images/2025_05_15_6a28331d5e7c993ad07ag-234.jpg}

图 5-2 周延性总结
\end{center}

\begin{center}
\fbox{\parbox{0.95\textwidth}{
\textbf{本节要点}
\begin{itemize}
\item \textbf{质}:命题或肯定或否定,决定谓项的周延性
  \begin{itemize}
  \item 肯定命题(A和I):谓项不周延
  \item 否定命题(E和O):谓项周延
  \end{itemize}
\item \textbf{量}:命题或全称或特称,决定主项的周延性
  \begin{itemize}
  \item 全称命题(A和E):主项周延
  \item 特称命题(I和O):主项不周延
  \end{itemize}
\item \textbf{标准式直言命题的一般模式}:量项(主项)联项(谓项)
\item \textbf{周延性}:一个词项是周延的,当且仅当命题述及该词项所指称的类的全部元素
\item 四种标准式命题的周延性:
  \begin{itemize}
  \item A命题:主项周延,谓项不周延
  \item E命题:主项周延,谓项周延
  \item I命题:主项不周延,谓项不周延
  \item O命题:主项不周延,谓项周延
  \end{itemize}
\end{itemize}
}}
\end{center} 
\input{chapter5/5-4 传统对当方阵.tex}
\input{chapter5/5-5 其他直接推论.tex}
\section{存在含义与直言命题的解释}

\begin{quotation}
直言命题的解释关系到逻辑推理的正确性。本节讨论一个关键问题:直言命题是否含有存在预设?通过分析传统解释与现代布尔解释的区别,我们将了解不同解释如何影响对当方阵及直接推论的有效性,从而为后续对三段论的分析奠定基础。
\end{quotation}

进一步分析和评估由直言命题构成的论证,需要对它们进行图示与符号化。但是,把 A、E、I、O 命题符号化,必定会遇到而且必须解决一个深层的逻辑问题——一个上千年长期争论的问题。本节我们就来说明这个问题,同时提供一种解决方案。以此为基础,也可以对三段论做出融贯的分析。

首先要说明的是,这并不是一个简单的问题。但只要我们弄清如下关于直言命题的解释(称为\textbf{布尔解释}[Boolean interpretation]),则后面关于三段论的分析并不需要对有关争议的深度把握。如果能掌握本节最后所总结的讨论结果,就可以顺利越过此前的复杂讨论。

要理解这个结果,必须弄清有些命题有\textbf{存在含义}(existential import),有些则没有。如果说出一个命题就肯定了某种对象的存在,那么就说这个命题有存在含义。

为什么初学逻辑就要关心这个看上去很深奥的问题呢?这是因为,特定论证中所用的命题中是否有存在含义,将直接影响到该论证中推理的正确性。对直言命题必须有一个清晰、融贯的解释,以便能确定什么东西可以从它们正确地推出,同时避免错误推论。

\subsection{特称命题的存在含义}

先看 I 命题和 O 命题,它们肯定有存在含义。例如 I 命题"有士兵是英雄"说的是至少存在一个是英雄的士兵。而 O 命题"有狗不是同伴"说的是至少存在一个不是同伴的狗。特称命题 I 和 O,一般说来,确实断定了主项(例句中的士兵和狗)指称的类不为空——士兵的类和狗的类(如果给出的例子为真的话)中至少有一个元素。\cite{russell1905}

\subsection{传统解释的困境}

如果确实如此,即如果 I 和 O 命题有存在含义(没人会否认),会有什么问题呢?问题在于这种状况的后果令人十分不安。先前我们已经说过,通过差等关系推论,I 命题可以从相应的 A 命题有效地推出,也就是说,从"所有蜘蛛都是八脚动物"可以有效地推出"有蜘蛛是八脚动物"。同样,我们说 O 命题可以有效地从 E 命题推得。但如果 I 和 O 命题有存在含义,而它们分别是从 A 和 E 命题得到的,那么 A 和 E 命题必定也要有存在含义。因为一个有存在含义的命题不可能有效地从另一个没有同样含义的命题得到。\cite{russell1905}

这种结果造成了一个严重的问题。我们知道在传统逻辑方阵中,A 和 O 命题是矛盾关系。"所有丹麦人都说英语"与"有丹麦人不说英语"是互为矛盾的。具有矛盾关系的命题不可同真,因为其中必有一假。两者也不可同假,因为其中必有一真。但如果像上文总结的那样,对应的 A 和 O命题确实有存在含义的话,那么,两个矛盾命题就可能同时为假!举例来说,A命题"所有火星人都是金发碧眼的"与其对应的 O 命题"有火星人不是金发碧眼的"互为矛盾,如果它们都有存在含义的话——即我们要把它们看做都断言存在火星人的话——那么,如果火星上没有居民则两个命题都是假的。我们当然知道火星上没有人,火星人的类是空类,据此上述例子中给出的两个命题都是假的。而如果两者都是假的,它们就不可能是矛盾关系!

由此看来,传统对当方阵是有不妥之处的。假如它所说 A 和 E 命题有效地蕴涵相应的 I 和 O 命题是正确的话,那么,它断言 A 和 O 命题之间有矛盾关系就不正确了,同样,认为 I 和 O 命题为下反对关系也是不正确的。

\subsection{全面存在预设的尝试}

那么我们该怎么办呢?传统逻辑方阵还能否加以挽救?挽救是可以的,但代价很高。我们可以引入\textbf{预设}(presupposition)概念来恢复逻辑方阵的地位。我们早已注意到(见 4.3 节),对于一些复杂问语,只有已经预设了先行问题的答案,才能适当地回答"是"或"否"。只有预设了你偷过钱是真的,才能用"是"或"否"来回答"你把偷来的钱花光了吗"这样的问题,否则是不合理的。

现在,为挽救传统逻辑方阵,我们可以主张所有直言命题,即四种标准式命题 A、E、I、O——都预设(在上述含义下)它们涉及的类均不为空,即都有元素。也就是说,要使命题的真假情况以及它们之间的逻辑关系都成立并可以得到合理的解答(在这种解释下),就必须预设它们绝不涉及空类。这样,就可以保留传统对当方阵中构建的各种关系:A 与 E 仍是反对关系,I 与 O 仍是下反对关系,A与 O、E 与 I 仍是矛盾关系。然而,为了保证这个结果,必须诉诸其\textbf{全面存在预设}(blanket presupposition),即预设全部词项指称的类(及其补类)都有元素,都不为空。\cite{wiebe1991}

那么,我们为什么不能就此罢休呢?存在预设对于挽救亚里士多德型逻辑既是必要的也是充分的。而且,预设在很多情况下与现代语言的日常用法是一致的。如果有人告诉你说"桶里的苹果都是甜的",而你向桶里一看,却什么都没有,那你会怎么说?你可能不会说刚才的话是假的或真的,而是指出这里没有苹果。你会解释说,说话人犯了一个错误,即当时的存在预设(桶里有苹果)是假的。事实上,这种纠正已经表明我们理解并基本接受了日常语言中的预设。

\subsection{为何全面存在预设不可接受}

然而不幸的是,用来挽救逻辑方阵的这种全面预设却要付出一个过重的代价,是我们不能接受的。我们有充分的理由不这样做,在此列举三条理由。

首先,引人预设确实能够保留 A、E、I 和 O 之间的对当关系,但却付出了不能刻画某些我们需要的断言的代价,即不能再刻画那些否定有元素存在的命题了。而这样的否定有时非常重要,是必须明确的。

其次,即使是日常语言的用法,也并不完全与全面存在预设一致,有时我们说的话并不假定所谈的类中有元素。例如,你说"所有非法侵入者都要被起诉",这句话根本不预设非法侵人者的类中已经有元素,相反,你这样说正是为了保证这个类维持空类。

再次,在科学界及其他理论界,我们通常希望进行没有任何存在预设的推理。例如牛顿第一运动定律断定的是不受任何外力作用的物体必然保持静止状态或匀速直线运动。这种定律可以是真的,而物理学家表述它并为它辩护的时候,并没有预设不受任何外力作用的物体存在。

这些问题的存在使得上述全面存在预设不能为现代逻辑学家所接受。我们应当放弃曾长期被认为是正确的亚里士多德型解释,而采用关于直言命题的现代解释。

\subsection{布尔解释}

直言命题的现代解释不再假定我们言说的类中必定有元素。拒绝这种假定的解释称为\textbf{布尔解释}。英国逻辑学家、数学家乔治•布尔(George Boole,1815-1864)是现代符号逻辑奠基人之一,这种新的解释就是以他的名字命名的。\cite{boole1854}

在本书以下部分,我们均采纳关于直言命题的布尔解释。现在我们就来阐明这种解释:

1.在某些方面,传统解释仍然成立。$\mathbf{I}$ 和 $\mathbf{O}$ 命题在布尔解释中仍然有存在含义。所以,如果 $S$ 类为空,那么,命题"有 $S$ 是 $P$"为假,命题 "有 $S$ 不是 $P$"也为假。

2.全称命题 $\mathbf{A}$ 和 $\mathbf{E}$ 与特称命题 $\mathbf{O}$ 和 $\mathbf{I}$ 之间的矛盾关系也保持为真。也就是说,命题"所有人是会死的"与"有人不是会死的"互为矛盾,而命题"没有神灵是会死的"与"有神灵是会死的"亦互为矛盾。

3.在布尔解释中上述关系是完全融贯的,这是因为,\textbf{全称命题被解释为没有存在含义}。因此,即使 $S$ 类为空,命题"所有 $S$ 是 $P$"仍可以为真,"没有 $S$ 是 $P$"也可以为真。例如,即使独角兽不存在,"所有独角兽是有角的"与"没有独角兽是有翅膀的"都可以为真。而如果不存在独角兽,I 命题"有独角兽是有角的"就是假的,O 命题"有独角兽不是有翅膀的"同样为假。

4.在日常话语中,有时我们说出一个全称命题,确实假定了某事物的存在。当然,布尔解释也允许有这种表述,但要求用两个命题来表述,一个是有存在含义的特称命题,加之一个没有存在含义的全称命题。

5.采纳布尔解释会带来一些重要变化。相应的 $\mathbf{A、E}$ 命题可以同真,因此它们之间不再是反对关系。这似乎有点怪异,在后面 10.2 和 10.3 部分将给出详细的说明。现在弄清如下这点就足够了:在布尔解释中,"所有独角兽是有翅膀的"断言的是"如果有独角兽,那么,它是有翅膀的"。而"没有独角兽是有翅膀的"断定的是"如果有独角兽,那么,它是没有翅膀的"。如果确实不存在独角兽,这两个"如果……那么……"型的命题都可以为真。

6.类似的,在布尔解释中,因为 I 和 O 命题确实有存在含义。所以,如果主项指称的类为空,相应的 I 和 O 命题都是假的,因此相应的 I 和 O命题之间也不再是下反对关系。

7.在布尔解释中,差等关系——从 A 命题推出相应的 I 命题,从 E命题推出相应的 O 命题——不是普遍有效的。从一个没有存在含义的命题当然不能得出一个有存在含义的命题。

8.布尔解释保留了一些直接推理:$\mathbf{E}$ 命题和 $\mathbf{I}$ 命题的换位推理,$\mathbf{A}$ 命题和 O 命题的换质位推理,所有命题的换质推理。但限制换位、限制换质位推理不再有效。

9.在布尔解释下,逻辑方阵转变为如下情形:方阵周边的关系不再成立,而对角线上的矛盾关系保持不变。

简言之,现代逻辑学家否定了全面存在预设。对于一个不能明确断定其中有元素的类,我们就不能假定它有元素,否则就是错的。任何依据这种错误假定的论证都会产生\textbf{存在预设谬误},简称为\textbf{存在谬误}。现在有了清晰的布尔解释,我们就可以构造一个有力的体系,将标准式直言命题推理符号化、图示化。



\begin{center}
\fbox{\parbox{0.95\textwidth}{
\textbf{本节要点}
\begin{itemize}
\item \textbf{存在含义}:指命题肯定了某种对象的存在
  \begin{itemize}
  \item 特称命题(I和O)肯定有存在含义
  \item 如果主项指称的类为空,特称命题为假
  \end{itemize}
\item \textbf{传统解释的困境}:
  \begin{itemize}
  \item 若A和E命题有存在含义,当主项为空类时,A和O命题可同假
  \item 导致对当方阵中的矛盾关系不成立
  \end{itemize}
\item \textbf{布尔解释}的主要特点:
  \begin{itemize}
  \item 全称命题(A和E)没有存在含义
  \item 如果主项指称的类为空,全称命题可为真
  \item 保留了对角矛盾关系,但取消了对当方阵周边关系
  \item 差等关系(从A到I,从E到O)不再普遍有效
  \item 保留了E和I命题的换位推理,A和O命题的换质位推理,以及所有命题的换质推理
  \end{itemize}
\item \textbf{存在谬误}:依据错误的存在预设进行推理的谬误
\end{itemize}
}}
\end{center} 
\section{直言命题的符号系统与图解}

\begin{logicbox}[title=章节导读]
布尔解释为直言命题提供了严谨的基础,但为了更直观地理解命题间的关系,我们需要一套符号系统和图解方法。本节介绍如何用代数符号表示直言命题,以及如何用文恩图直观展示命题的含义和相互关系,为下一章分析三段论有效性奠定基础。
\end{logicbox}

直言命题的布尔解释在很大程度上以\textbf{空类}概念为基础,为方便起见,可用一个特殊的符号来表示空类。此处我们用数字"0"来代表空类。说词项$S$指称的类没有元素,就在$S$和0之间划上等号。也就是说,$S=0$表示$S$没有元素($S$的元素$s$简记为$S^{\prime}s$,说不存在$S^{\prime}s$亦即$S=0$)。

说$S$指称的类确实有元素就是否定$S$为空。断定"存在$S$′$s$"就是对$S=0$所表示的命题的否定。我们在等号上加一条斜线表示这种否定式。就是说,$S \neq 0$表示存在$S$′$s$,是对$S$为空的否定。

\subsection{直言命题的符号表示}

标准式命题都涉及两个类,所以表示它们的等式要复杂一些。两个类都分别由一个符号代表,因此可以把两个符号并排在一起,用以表示那些由同时属于两个类的元素组成的类。比如,如果$S$代表所有"讽刺作品"组成的类,$P$代表所有"诗"组成的类,那么,既是讽刺作品又是诗的东西组成的类就可以用符号$SP$表示,它代表的就是所有讽刺诗(或者说诗式讽刺作品)组成的类。两个类的共同部分或全体共同元素称为两个类的\textbf{积}(product)或\textbf{交}(intersection)。两个类的积是所有同时属于这两个类的东西组成的类。所有美国人的类与所有作家的类之积就是所有美国作家的类。(此处必须与自然语言的某些特定用法区分开来。例如,西班牙人的类与舞蹈家的类之积,不是西班牙舞蹈家的类,因为通常说的西班牙舞蹈家不是西班牙的舞蹈家,而是表演西班牙舞蹈的人。同样,抽象画家、英语课程、古董商人等等也都是这样的用法。)

使用这种新记法,我们也可以用等式和不等式将E和I命题符号化。E命题"没有$S$是$P$"说的是$S$类中没有元素是$P$类的元素,即没有东西同时属于两者。换言之,两个类的积为空,可用等式符号表示为:$SP=0$。I命题"有$S$是$P$"说的是$S$类中至少有一个元素也是$P$类的元素。这意味着$S$类和$P$类的积不空,可用不等式符号表示为:$SP \neq 0$。

对于A命题和O命题,需要引人一个表示\textbf{补类}的新方法。如5.5节所说明,一个类的补类就是所有那些不属于原类的东西的类或汇集。例如,士兵的类的补类就是所有不是士兵的东西组成的类,即非士兵的类。若用$S$代表士兵的类,则把非士兵的类记为:$\bar{S}$(读做:$S$杠),即在原来的类之上加一横杠。A命题"所有$S$是$P$"说的是$S$类的所有元素都是$P$类的元素,也就是说,没有$S$类的元素不是$P$类的元素。或者说(据换质法)"没有$S$是非$P$",像任何E命题一样,这个命题说的是,主项指称的类与谓项指称的类的积为空,可用等式符号表示为:$S\bar{P}=0$。O命题"有$S$不是$P$"换质后得逻辑等价式I命题"有$S$是非$P$",可用等式符号表示为:$S\bar{P} \neq 0$。

使用这些符号公式,就能很清晰地显示四种标准式直言命题之间的相互关系。既然A命题和O命题的符号公式分别为$S\bar{P}=0$和$S\bar{P} \neq 0$,它们显然是互为矛盾的。E命题和I命题的符号形式分别为:$SP=0$和$SP \neq 0$,显然也是互为矛盾的。布尔解释下的对当方阵可以重新表示为图5-2。

\subsection{文恩图及其应用}

命题可以用所涉及的类的图示来表达。我们用一个圆代表一个类,用指称类的词项标注它。这样$S$类可以表示为图5-3。

\begin{center}
\includegraphics[width=\textwidth]{images/2025_05_15_6a28331d5e7c993ad07ag-257(2).jpg}

图5-2 布尔解释下的对当方阵
\end{center}

\begin{center}
\includegraphics[width=\textwidth]{images/2025_05_15_6a28331d5e7c993ad07ag-257.jpg}

图5-3 类的图示
\end{center}

上图表示的是一个类,而不是命题。它只代表$S$类,而对这个类无所言说。要图示命题"$S$没有元素"或"不存在$S$′$s$",我们就在代表$S$的圆中加上阴影,来表示$S$中什么都没有,$S$为空类。要图示"存在$S$′$s$"这个命题,我们就在代表$S$的圆中写一个$x$,用来表示其中有东西,$S$不是空类。这样,"不存在$S$′$s$"和"存在$S$′$s$"这两个命题就可以用图5-4来表示。

\begin{center}
\includegraphics[width=\textwidth]{images/2025_05_15_6a28331d5e7c993ad07ag-257(1).jpg}

图5-4 空类与非空类的图示
\end{center}

实际上,表示$S$的图示也可以表示$\bar{S}$,因为圆中的部分代表的是$S$的所有元素,而圆外的部分恰好就是$\bar{S}$。

要图示标准式直言命题,一个圆不够,而需要两个圆。标准式直言命题的主、谓项分别记为$S$和$P$,而后画两个相交的圆,如图5-5。

\begin{center}
\includegraphics[width=\textwidth]{images/2025_05_15_6a28331d5e7c993ad07ag-258.jpg}

图5-5 两个类的图示
\end{center}

图示只表示出了$S$和$P$两个类,而没有表示它们形成的命题。既没有肯定也没有否定其中一个或两个类有元素。实际上,两个相交的圆表示出的类不只是$S$和$P$两个。标有$S$的圆中与$P$不重叠的部分代表的是所有不是$P$′$s$的$S$′$s$,即代表了$S$类与$\bar{P}$类的积,这一部分标记为$S\bar{P}$。两圆相交的部分代表$S$类与$P$类的积,标记为$SP$。标有$P$的圆中与$S$不重叠的部分代表的是所有不是$S$′$s$的$P$′$s$,即代表了$\bar{S}$类与$P$类的积,标记为$\bar{S}P$。最后,两个圆之外的部分,代表既不在$S$类也不在$P$类之中的东西,标记为第四个类$\overline{SP}$。加上这些标记,图5-5就成了图5-6:

\begin{center}
\includegraphics[width=\textwidth]{images/2025_05_15_6a28331d5e7c993ad07ag-258(1).jpg}

图5-6 文恩图中的四个类区域
\end{center}

可以用各种不同的类来解释上图。例如设西班牙人的类为$S$、画家的类为$P$,则$SP$就是两个类的积,由所有同时属于两个类的东西组成。因为$SP$的每个元素必须既是$S$类也是$P$类的元素,所以每个元素既是西班牙人又是画家。两个类的积就是西班牙画家的类,其中包括委拉斯开兹(Velásquez)和戈雅(Goya)等人。$S\bar{P}$是第一个类与第二个类之补的积,包括且只包括属于$S$类但不属于$P$类的对象,也就是不是画家的西班牙人组成的类,即所有非画家西班牙人,委拉斯开兹不在其中,戈雅也不在其中,但却包括小说家塞万提斯(Cervantes)和独裁者佛朗哥(Franco)及其他西班牙人。$\bar{S}P$是第二个类与第一个类之补的积,是那些不是西班牙人的画家组成的类,这个类包括荷兰画家伦布兰特(Rembrandt)、美国画家乔治亚•奥基夫(Georgia O'Keeffe)等。最后,$\overline{SP}$是原来两个类的补的积,包括而且只包括那些既不是西班牙人也不是画家的对象。这可是一个很大的类,包括的不只是英国海军上将和瑞士登山运动员们,还包括诸如密西西比河、珠穆朗玛峰这样的东西。如果对$S$和$P$进行这样的解释,那么,以上说的所有类都在图5-6中有所表示。

这就是\textbf{文恩图}(Venn diagram),得名于英国数学家、逻辑学家约翰•文恩(John Venn,1834-1923),他首先使用这种方法表示类和命题。像图5-6这样的带有几处标记的双圆图,所代表的仍只是类,尚不表示任何命题。整个圆或其中留做空白的部分既不表示类中有元素,也不表示没有。

但是,再加上一定条件,我们就能用文恩图来表示命题。通过给某些部分加上阴影,或者标上"$x$",就能准确地将四种标准式直言命题图示化。文恩图(带有标记的)能够全面、简明地表示命题,所以,它已经被公认为评价三段论论证的最有力、使用最广泛的方法。下面就来说明如何用文恩图表示这四种标准式命题。

A命题"所有$S$是$P$"即$S\bar{P}=0$,用文恩图图示之,可把代表$S\bar{P}$的那部分加上阴影,即表示其中没有元素。E命题"没有$S$是$P$"即$SP=0$,可把图中代表$SP$的那部分加上阴影,以示其中没有元素。I命题"有$S$是$P$"即$SP \neq 0$,可在图中$SP$类部分标上一个$x$,表示两个类的积不是空的,其中至少有一个元素。最后,O命题"有$S$不是$P$"即$S\bar{P} \neq 0$,可加一个$x$在$S\bar{P}$部分,表示其中至少有一个元素而不是空的。将以上四个图示列在一起,就能十分清晰地展现出四种命题的不同含义,见下图5-7。

我们已经用文恩图表示出"没有$S$是$P$"和"有$S$是$P$",而它们换位后分别得到一个等价命题:"没有$P$是$S$"和"有$P$是$S$",因此后面两个命题在图中也就表示出来了。要图示A命题"所有$P$是$S$"即$P\bar{S}=0$,循同样路径,可把代表$P\bar{S}$的部分加上阴影。显然,$P\bar{S}$与$\bar{S}P$是相同的,如果一下子不能明白,就回想一下是画家而非西班牙人的类,与非西班牙人中是画家的类。前一个类中的对象必定也是后一个类的对象——即所有是画家而非西班牙人的人与所有非西班牙人的画家,反之亦然。要图示命题"有$P$不是$S$"即$P\bar{S} \neq 0$,可在$P\bar{S}(=\bar{S}P)$部分加上一个$x$。图5-8展示的正是这几个命题。

\begin{center}
\includegraphics[width=\textwidth]{images/2025_05_15_6a28331d5e7c993ad07ag-260(1).jpg}

图5-7 四种标准式命题的文恩图
\end{center}

\begin{center}
\includegraphics[width=\textwidth]{images/2025_05_15_6a28331d5e7c993ad07ag-260.jpg}

图5-8 更多命题的文恩图表示
\end{center}

双圆图的这种灵活运用,在本书下一章中起着重要作用。任给一对带有给定标记——比如$S$和$M$——的交叉圆,就能将任何一个含有$S$和$M$的标准式直言命题图示化,无论$S$和$M$出现的顺序如何。

文恩图是标准式直言命题的肖像,将空间的包含与排斥与类间非空间的包含与排斥对应起来,是一种极为清晰的记法。下一章将会看到,这也是检验直言三段论的有效性的一种最简单、最直接的方法。

\begin{center}
\fbox{\parbox{0.95\textwidth}{
\textbf{本节要点}
\begin{itemize}
\item \textbf{直言命题的符号表示}
  \begin{itemize}
  \item 空类用"0"表示,$S=0$表示类$S$没有元素
  \item 非空类用$S \neq 0$表示,表示类$S$至少有一个元素
  \item 类的积(或交)记为$SP$,表示同时属于$S$和$P$的元素组成的类
  \item 类的补记为$\bar{S}$,表示不属于类$S$的所有元素
  \item 四种标准式命题的符号表示:
    \begin{itemize}
    \item A命题:"所有$S$是$P$"$\Rightarrow$ $S\bar{P}=0$
    \item E命题:"没有$S$是$P$"$\Rightarrow$ $SP=0$
    \item I命题:"有$S$是$P$"$\Rightarrow$ $SP \neq 0$
    \item O命题:"有$S$不是$P$"$\Rightarrow$ $S\bar{P} \neq 0$
    \end{itemize}
  \end{itemize}
\item \textbf{文恩图}:用图形直观表示命题的方法
  \begin{itemize}
  \item 圆形代表类,重叠部分代表类的交
  \item 阴影表示该区域为空类
  \item 带"$x$"标记表示该区域非空
  \item 文恩图能直观展示命题间的关系,是检验三段论有效性的有力工具
  \end{itemize}
\end{itemize}
}}
\end{center}
\section{第5章概要}
本章介绍并讨论的是\textbf{古典逻辑}即\textbf{亚里士多德型逻辑}的基本构件,也是\textbf{演绎逻辑}的基本构件。

\subsection{类与标准式直言命题}
5.2节介绍了\textbf{类}的概念。传统逻辑正是以类为基础建立起来的。我们阐明了四种基本的标准式直言命题:

\begin{itemize}
  \item \textbf{A命题}:全称肯定命题
  \item \textbf{E命题}:全称否定命题
  \item \textbf{I命题}:特称肯定命题
  \item \textbf{O命题}:特称否定命题
\end{itemize}

\subsection{命题的质与量}
5.3节更加详细地考察这四种命题。探讨了命题的\textbf{质},即肯定和否定,以及命题的\textbf{量},即全称和特称。说明了\textbf{周延的项}与\textbf{不周延的项}。

\subsection{对当关系}
5.4节探讨这几种直言命题之间的\textbf{对当关系}的种类:命题之间的\textbf{矛盾关系}、\textbf{反对关系}、\textbf{下反对关系}以及上位式与下位式之间的\textbf{差等关系}。并用一个\textbf{对当方阵}图示了这几种关系,进而说明了一些基于传统方阵的直接推理。

\subsection{直接推论方法}
5.5节阐明其他三种直接推论:\textbf{换位法}、\textbf{换质法}和\textbf{换质位法}。

\subsection{存在含义问题}
5.6节探讨\textbf{存在含义}问题。要保留传统对当方阵,只有做出一种假定,即全盘假定命题主项所指称的类总是有元素的——这是现代逻辑极不赞同的。然后,我们又对本书通篇采用的\textbf{布尔解释}作了说明。布尔解释能保留传统逻辑对当方阵中的大部分内容,同时又避免了非空类的假定。在布尔解释中,特称命题,即称为$\mathbf{I}$和$\mathbf{O}$的命题之中有存在含义,但全称命题,即$\mathbf{A}$和$\mathbf{E}$则没有存在含义。我们也很细致地说明了采用这种解释的结果。

\subsection{命题的符号化与图示化}
5.7节介绍将直言命题符号化与图示化的方法,包括使用\textbf{文恩图},用交叉的圆加以恰当的标记或阴影来刻画类与类之间的关系。

\begin{center}
\fbox{\parbox{0.9\textwidth}{
  \centering
  \textbf{第5章要点总结}\\
  \textbf{演绎逻辑基础}:古典逻辑的基本构件与概念\\
  \textbf{命题分类}:A全称肯定、E全称否定、I特称肯定、O特称否定\\
  \textbf{推理方法}:直接推论、对当关系、布尔解释、文恩图\\
}}
\end{center}

有了这些必要的工具,我们就可以考察——在接下来的两章中——建基于标准式直言命题之上的\textbf{三段论},以及传统演绎逻辑在日常语言中的其他主要用途。 

% 第六部分
\chapter{直言三段论}
\section{标准式直言三段论}

\begin{logicbox}[title=引言]
\logicterm{三段论}是演绎推理的经典形式,由三个直言命题组成,包含三个词项,每个词项恰好出现两次。本节将介绍标准式直言三段论的基本要素,包括词项、式和格的概念,为后续讨论三段论的\logicemph{有效性}奠定基础。
\end{logicbox}

\logicterm{三段论}是从两个前提推得一个结论的演绎论证。\logicterm{直言三段论}是由三个直言命题组成的演绎论证,其中包含且仅包含三个词项,每个词项在其构成命题中恰好出现两次。

如果一个直言三段论的前提、结论都是标准式的直言命题(A、E、I、O),并且以特定的标准顺序组合在一起,就称为\logicterm{标准式直言三段论}。要确定标准式直言三段论的顺序,必须首先说明直言三段论的词项和前提的特定名称,为简便起见,本章将直言三段论简称为"\logicterm{三段论}"。其他三段论将会在后面的章节进行讨论。

\subsection{大项、小项和中项}
\begin{theorembox}[title=三段论词项的定义]
标准式三段论的结论是一个标准式直言命题,三段论的三个词项有两个会在其中出现。因此,通过结论可以识别三段论的词项。

结论的谓项称为三段论的\logicterm{大项}。

结论的主项称为三段论的\logicterm{小项}。

在结论中不出现,而在前提中出现两次的项,即三段论的第三个项,称为\logicterm{中项}。
\end{theorembox}

\begin{examplebox}[title=三段论词项识别示例]
例如下面这个标准式三段论中:

没有英雄是胆小鬼,

有士兵是胆小鬼,

所以,有士兵不是英雄。

"士兵"是\logicterm{小项},"英雄"是\logicterm{大项},结论中没有出现的"胆小鬼"是\logicterm{中项}。
\end{examplebox}

标准式三段论的前提因其中出现的项而得名。大项和小项必定出现在不同的前提中,包含大项的前提称为\logicterm{大前提},包含小项的前提称为\logicterm{小前提}。在上述三段论中,大前提是"没有英雄是胆小鬼",小前提是"有士兵是胆小鬼"。

如前所述,如果前提以特定的标准顺序排列,就称这个三段论为\logicterm{标准式}。现在即可描述这个顺序:在标准式三段论中,\logicterm{大前提}处在第一位,\logicterm{小前提}处在第二位,结论在最后。应当强调的是,大前提不是根据其位置而确定的,而是因为其中包含\logicterm{大项},大项又是由结论的谓项定义的。同样的,小前提也不是根据其位置而确定的,而是因为其中包含\logicterm{小项},小项又是由结论的主项定义的。

\subsection{式}
\begin{theorembox}[title=三段论的式]
标准式三段论的\logicterm{式}由所含直言命题的类型而定(以字母 A、E、I、O为标志)。每个三段论的式都由三个按特定顺序排列的字母组成。第一个字母指的是大前提的类型,第二个字母指的是小前提的类型,第三个字母指的是结论的类型。
\end{theorembox}

例如,在上述作为例子的三段论中,大前提是一个 E命题,小前提是一个 I 命题,结论是一个 O 命题,所以,这个三段论的式就是 \logicterm{EIO 式}。

\subsection{格}
只有\logicterm{式},还不能完全刻画标准三段论的形式。考虑下面两个三段论:

\begin{examplebox}[title=相同式但不同格的三段论]
(A)所有大科学家都是大学毕业生,

有专业运动员是大学毕业生,

所以,有专业运动员是大科学家。

和

(B)所有艺术家都是自我主义者,

有艺术家是乞丐,

所以,有乞丐是自我主义者。

两个三段论的式都是 AII,但它们的形式并不相同。
\end{examplebox}

如果我们展示出它们的逻辑"骨架",就能十分清楚地揭示出其形式上的不同之处。把小项记为 $S$ ,大项记为 $P$ ,中项记为 $M$ ,并用"$\therefore$"表示"所以",这两个三段论的形式或"骨架"分别是:

(A)所有 $P$ 是 $M$ ,

有 $S$ 是 $M$ ,

$\therefore$ 有 $S$ 是 $P$ 。

(B)所有 $M$ 是 $P$ ,

有 $M$ 是 $S$ ,

$\therefore$ 有 $S$ 是 $P$ 。

在记为(A)的第一个三段论中,\logicterm{中项}在两个前提中都做谓项,而记为(B)的第二个三段论,\logicterm{中项}在两个前提中都做主项。这两个例子表明,尽管三段论的形式可以部分地由\logicterm{式}来描述,但相同式的三段论还是有区别的,这就要看中项的相对位置。\cite{lukasiewicz1957} 然而,我们可以通过陈述一个三段论的格和式来完整地描述其形式,它的\logicterm{格}表明了中项在前提中的位置。

\begin{theorembox}[title=三段论的四格]
显然,三段论有且只有四种不同的\logicterm{格}。\logicterm{中项}可能在大前提中做主项、在小前提中做谓项,或者在两个前提中都做谓项,或者在两个前提中都做主项,或者在大前提中做谓项、在小前提中做主项。中项的这些可能组合分别构成了三段论的第一、第二、第三和第四格。
\end{theorembox}

四个格的模式可依次排列如下,其中只显示了中项的相对位置,而隐藏了它们的式,也就是说既不出现量项也不出现联项:

$$
\begin{array}{llll}
M-P & P-M & M-P & P-M \\
\frac{S-M}{\therefore S-P} & \frac{S-M}{\therefore S-P} & \frac{M-S}{\therefore S-P} & \frac{M-S}{\therefore S-P} \\
\begin{array}{l}
\text { 第一格 }
\end{array} & \begin{array}{l}
\text { 第二格 } \\
\end{array} & \begin{array}{l}
\text { 第三格 }
\end{array} & \begin{array}{l}
\text { 第四格 }
\end{array}
\end{array}
$$

要完整地描述一个标准三段论的形式,只要指明其\logicterm{式}和\logicterm{格}即可。例如,任何一个第二格 AOO 式(简记为 AOO-2)的三段论都有如下形式:

所有 $P$ 是 $M$ ,

有 $S$ 不是 $M$ ,

所以,有 $S$ 不是 $P$ 。

从无限多样的不同题材中把形式抽象出来,我们会得到许多不同的标准三段论的形式。假如把它们排列一下,从AAA、AAE、AAI、AAO、AEA、AEE、AEI、AEO、AIA $\cdots \cdots$ 以此类椎,直到 OOO 式,共可列举出 64 个不同的式。由于每个式都可以与四个不同的格进行组合,于是,标准式的三段论就必然呈现出 \logicemph{256 个不同的形式}。但正如我们将要看到的,其中只有少数形式是\logicemph{有效的}。


\section{三段论论证的形式性质}

\begin{logicbox}[title=引言]
三段论的\logicemph{有效性}完全取决于其逻辑形式,而非内容。本节将阐明三段论的形式性质,解释为什么相同形式的三段论具有相同的\logicemph{有效性}或\logicwarn{无效性},并介绍如何使用\logicterm{逻辑类推法}来检验和论证三段论的有效性。
\end{logicbox}

\begin{theorembox}[title=三段论的形式性质]
三段论的形式由\logicterm{格}与\logicterm{式}唯一确定——从逻辑的观点讲,这种形式是三段论最重要的方面。三段论的\logicemph{有效性}与\logicwarn{无效性}(其构成命题都是偶真的)仅仅依赖于形式,而完全独立于具体内容和题材。
\end{theorembox}

例如,任何形式为 AAA-1 的三段论:

\begin{examplebox}[title=AAA-1形式的三段论]
所有 $M$ 是 $P$ ,

所有 $S$ 是 $M$ ,

所以,所有 $S$ 是 $P$ 。

无论其题材是什么,它都是\logicemph{有效的}论证。这就是说,无论用什么词项代替这种形式或结构中的字母 $S$、$P$ 和 $M$,得到的论证总是\logicemph{有效的}。
\end{examplebox}

例如用这几个字母分别代表"雅典人"、"人"和"希腊人",代入后就得到这样一个\logicemph{有效}论证:

所有希腊人是人,

所有雅典人是希腊人,

所以,所有雅典人是人。

同样,下面这个论证也是\logicemph{有效的}:

所有钠盐是水溶性物质,

所有肥皂是钠盐,

所以,所有肥皂是水溶性物质。

说一个\logicemph{有效的}三段论是\logicemph{有效的}论证,是仅就其形式而言的。这说明如果某个三段论是\logicemph{有效的},那么,任何与它形式相同的三段论也是\logicemph{有效的}。如果一个三段论是\logicwarn{无效的},那么,任何与它形式相同的三段论也是\logicwarn{无效的}。$^{[2]}$ 这是人们在实际论辩中经常使用\logicterm{逻辑类推法}而获得的共识。

\begin{examplebox}[title=逻辑类推法的应用]
假如有人提出下面这个论证:

所有自由主义者都是国家健康保险的支持者,

有行政人员是国家健康保险的支持者,

所以,有行政人员是自由主义者。

我们会感觉到,无论其构成命题的真假,这个论证是\logicwarn{无效的}。揭示这种三段论荒谬性的最好方式,是构造一个形式相同但其\logicwarn{无效性}可直接显示出来的论证。比如,我们可以这样去问,你是否也可以说:

所有兔子都是跑得很快的,

有马是跑得很快的,

所以,有马是兔子。

我们可以补充说明:你不可能为后面这个论证作辩护,因为毫无疑问,其前提明显为\logicemph{真}但结论明显为\logicwarn{假}。你刚才的论证与这个马兔论证的形式完全相同。马兔论证是\logicwarn{无效的},所以你刚才的论证也是\logicwarn{无效的}。
\end{examplebox}

\logicterm{逻辑类推}是一种很好的论辩方法,是用于争辩的有力武器之一。

这种\logicterm{逻辑类推法}的根据是:直言三段论的\logicemph{有效性}或\logicwarn{无效性}是纯形式问题。要证明任何荒谬论证\logicwarn{无效},都可以找另一个论证,使之与一个明显\logicwarn{无效的}即其前提明显为\logicemph{真}而结论明显为\logicwarn{假}的论证有着相同形式。(不过应当牢记,\logicwarn{无效}论证也可能得到为\logicemph{真}的结论——说推理是\logicwarn{无效的},只是意味着结论与前提之间不构成逻辑蕴涵关系,或者说它们之间的关系不是必然联系。)

\logicwarn{但是,这种检验论证有效性的方法有很大的局限性。}有时很难一下子"想出"恰当的逻辑类推。并且,三段论论证有太多\logicwarn{无效的}形式(200多个)。此外,尽管我们只要想到一个前提为\logicemph{真}而结论为\logicwarn{假}的逻辑类推,就可以证明原论证的形式\logicwarn{无效},但是,若我们不能想到这样的逻辑类推,并不就能证明该形式\logicemph{有效},因为这可能只是由于我们的思维局限性所使然。很可能实际上存在着\logicwarn{无效性}类推,只是我们没有想到而已。这就需要一种更有效力的方法,来判定形式\logicemph{有效}或\logicwarn{无效}的三段论。本章以下各节就是要介绍检验三段论的一些最有力的方法。

\footnotetext{(2)严格来说,这一论断只适用于被解释为不含存在预设的标准式三段论。对于某些其他三段论形式来说,尽管逻辑类比未必总是有效的,但如果用所依靠的前提所含的具体概念来代替这个形式中出现的词项,那么所得到的论证仍然是有效的——即如果前提为真,结论也必为真。}

\chaptersummary{
三段论的\logicemph{有效性}或\logicwarn{无效性}仅取决于其逻辑形式(\logicterm{格}与\logicterm{式}的组合),相同形式的三段论具有相同的有效性特征。\logicterm{逻辑类推法}通过构造相同形式但前提明显\logicemph{真}而结论明显\logicwarn{假}的论证,证明某个形式的三段论\logicwarn{无效}。但逻辑类推法有局限性:难以找到恰当的类推例子,无法有效证明三段论形式的\logicemph{有效性},且三段论\logicwarn{无效}形式太多(200多个)。因此需要更系统的方法来判定三段论的有效性。
}
\input{chapter6/6-3 检验三段论:文恩图解法.tex}
\input{chapter6/6-4 三段论规则和三段论谬误.tex}
\section{直言三段论的15个有效形式}

\begin{quotation}
在标准式直言三段论的256种可能形式中,只有15种是有效的。本节将介绍这些有效形式及其古典命名系统,帮助我们快速识别有效三段论并应用于实际推理中。掌握这些形式,是理解亚里士多德式推理系统的关键。
\end{quotation}

三段论的式取决于其中所含三个命题的类型(A、E、I、O)。直言三段论有64个不同的式,即这三个命题的64种可能组合:AAA、AAI、AAE等等,一直到$\cdots\cdots$EOO$、\mathrm{OOO}$。

三段论的格是其逻辑形状,由中项在前提中的不同位置决定。所以一共有四种不同的格,如果头脑中有一个图表或者用图标说明,就可以很清晰地记住这几个格,见图6—11:

\begin{center}
\includegraphics[width=\textwidth]{images/2025_05_15_6a28331d5e7c993ad07ag-288.jpg}

图6-11 三段论的四种格
\end{center}

\begin{center}
\includegraphics[width=\textwidth]{images/2025_05_15_6a28331d5e7c993ad07ag-288(1).jpg}
\end{center}

从图中可见:

\begin{itemize}
  \item 第一格的中项是大前提的主项、小前提的谓项;
  \item 第二格的中项在两个前提中都做谓项;
  \item 第三格的中项在两个前提中都做主项;
  \item 第四格的中项是大前提的谓项、小前提的主项。
\end{itemize}

64个式都可以有四个格。把两者结合起来,给定了三段论的式与格,也就唯一地确定了三段论的形式。因此,标准式直言三段论恰有$256(64 \times 4)$个可能的形式。

这些形式中绝大部分是无效的。根据前一节阐明的三段论规则,我们可以排除那些违反了一条或几条规则的形式,剩下的就是直言三段论的有效式。256个形式中,只有15个形式不能被排除,因而它们是有效的。\cite{patzig1968}

为更好地掌握三段论,古典逻辑学家给每一个有效式都起了独特的名称,每一个都完全刻画了其格与式。了解有效式的这个小集合,记住每一个有效形式的名称,对于我们实际运用三段论论证是很有帮助的。这些名称都是精心设计的,每个名称都包含了三个元音,代表着被命名三段论的式(依据标准的顺序:大前提、小前提、结论)。对于同式不同格的有效三段论形式,都分别给它们指派唯一的名称。例如,对于式为EAE的三段论,如果是第一格的就叫做Celarent,而如果是第二格的就叫做Cesare。\cite{lukasiewicz1957}

这些名称曾经有(现在仍然有)很实用的功能:如果懂得只有式与格的某些特定组合是有效的,并且通过名字就能识别那些有效论证,那么,无论给出任何式或格的三段论,就都能立即判定其正误。例如,AOO式只在第二格才是有效的。这个唯一的形式(AOO-2)就叫做Baroko。\cite{copi1980}熟悉Baroko并能很容易指认它的人就会确信这个式的其他格都是无效的,是必须拒斥的。

古典逻辑学家很细致地研究了这些形式,谙熟它们的结构和逻辑"感应"。这种精心设计好的逻辑系统,会使得一个人在言语或文本中碰到三段论论证时,能立即确切指认哪些是有效的,哪些是无效的。许多世纪以来,逻辑训练的一种常用方式,就是通过给出三段论有效形式的名称,来为三段论论证的可靠性进行辩护。而在激烈的日常论辩中具备这种迅速识别有效论证与无效论证的能力,一直被视为富有学养、思维敏锐的标志。而依赖演绎论证所建立起来的论证链条之坚固也得到了充分显示。一旦完全掌握三段论理论,这种实际论辩能力就会得到富有成效的令人愉悦的提升。

三段论论证曾经有如此广泛的应用,并被普遍视为学术论证最不可缺少的工具,因此,最先系统论述三段论理论的学术大师亚里士多德,得到了人们上千年的尊崇。他关于三段论的分析的论集迄今仍沿用着一个简单但令人肃然起敬的名字:Organon,即《工具论》。

作为这个著名逻辑体系的初学者,我们对三段论的掌握可能难以非常精通。但列出所有有效三段论形式并加以熟练掌握,无疑是最为有用的。15个有效三段论形式(布尔解释下的)可以根据格的不同分为四组:在前三个格中都有四个有效形式,第四格有三个有效形式。\cite{patzig1968}

下图即列出了15个有效形式,以及它们相应的传统名称:

\subsection{标准式直言三段论的15个有效形式}
\textbf{第一格}(中项在大前提中做主项、在小前提中做谓项):
\begin{itemize}
\item 1.AAA-1 Barbara
\item 2.EAE-1 Celarent
\item 3.AlI-1 Darii
\item 4.EIO-1 Ferio
\end{itemize}

\textbf{第二格}(中项在两个前提中都做谓项):
\begin{itemize}
\item 5.AEE-2 Camestres
\item 6.EAE-2 Cesare
\item 7.AOO-2 Baroko
\item 8.EIO-2 Festino
\end{itemize}

\textbf{第三格}(中项在两个前提中都做主项):
\begin{itemize}
\item 9.All-3 Datisi
\item 10.IAI-3 Disamis
\item 11.EIO-3 Ferison
\item 12.OAO-3 Bokardo
\end{itemize}

\textbf{第四格}(中项在大前提中做谓项、在小前提中做主项):
\begin{itemize}
\item 13.AEE-4 Camenes
\item 14.IAI-4 Dimaris
\item 15.EIO-4 Fresison
\end{itemize}

\footnotetext{(8)在传统或者亚里士多德式解释下,有效式的数目是19个或者24个,因为其中已包含了布尔解释下并不有效的弱化(weakened)格式——后者的结论比前提断定得更少。}
\footnotetext{(9)传统名称中所包含的元音是由拉丁语创造出来的一种记忆方法,其含义在拉丁语中是完备的;不过,无论说什么语言的人,都能通过观察名称中包含的元音,看出该三段论属于哪个式。}
\footnotetext{(10)传统的有记忆价值的名称是Baroco而不是Baroko;但在汉语中,似乎很难将c和k的发音分得很开,所以这里保持拼写一致用k取代c。}
\footnotetext{(11)第四格下有效三段论较少,是因为这种格式的三段论总是有点绕弯子,因而不太直接,似乎是人工制造的而不是自然思维的产物——但这也不影响其有效性。}

\begin{center}
\fbox{\parbox{0.95\textwidth}{
\textbf{本节要点}
\begin{itemize}
\item 标准式直言三段论共有\textbf{256种可能形式}(64个式 × 4个格)
\item 其中只有\textbf{15个有效形式}(布尔解释下)
\item \textbf{古典命名系统}:每个有效形式都有特定名称
  \begin{itemize}
  \item 名称中的三个元音对应三段论的式(大前提、小前提、结论)
  \item 例如:Barbara (AAA-1)、Celarent (EAE-1)、Baroko (AOO-2)
  \end{itemize}
\item \textbf{格的分布}:
  \begin{itemize}
  \item 第一格:4个有效形式
  \item 第二格:4个有效形式
  \item 第三格:4个有效形式
  \item 第四格:3个有效形式
  \end{itemize}
\item 掌握这些有效形式及其名称,有助于快速识别有效论证和无效论证
\end{itemize}
}}
\end{center} 
\section{直言三段论15个有效形式的演绎推导}

\begin{quotation}
本节通过应用三段论的六条基本规则,系统地演绎推导出标准式直言三段论的15个有效形式。我们将根据结论的不同类型(A、E、I或O命题)分类讨论,分析哪些形式符合逻辑规则而保留下来,从而完成从256个可能形式到15个有效形式的演绎。
\end{quotation}

直言三段论的15个有效形式是从256个可能形式中排除无效式以后得以确立的。我们可以通过确定哪些形式违反了三段论的基本规则来实行这种排除——演绎推导出三段论的15个有效形式。

对逻辑初学者来说,不必一定弄清如何排除无效式的细节。但对于那些从三段论分析的复杂性中获取乐趣的人而言,这应是一种虽有难度但令人愉悦的挑战。如果只想认识和把握三段论的有效式,即6.5节讲到的那些内容,就可以绕过本节不看。

这种演绎推导并不那么容易理解。从事这项工作必须清晰地记住以下两点:(1)6.4节设定的六条三段论基本规则;(2)三段论四个格的模式,即图6-11。

根据结论的不同形式,我们首先把三段论的所有可能形式分为四组。每个结论都是A、E、I、O四种直言命题之一,没有其他可能,据此可以分四种情形考察一个有效的三段论需要具备什么特性,即可以这样提问:如果结论是A命题,通过某一条或几条规则能够排除什么形式;如果结论是E命题可以排除什么形式,以此类推。下面我们就逐个进行考察。

\subsection{情形1:如果三段论的结论是$\mathbf{A}$命题}
在这种情形下,前提不可能是E命题,也不可能是O命题,因为如果前提为否定命题的话,结论就应该是否定的(规则5)。所以,两个前提必定是A命题或I命题。小前提不能是I命题,因为小项(结论的主项,也就是一个A命题的主项)在结论中是周延的,如果小前提是I命题,那么在前提中不周延的项在结论中周延,违反了规则3。两个前提,即大前提和小前提,不能是I和A,因为如果是的话,有两种可能,或者是在结论中周延的项在前提中不周延,违反规则3,或者是中项两次不周延,违反规则2。所以两个前提(结论是A命题时)必须都是A命题,这意味着唯一有效的形式是AAA式。而第二格的AAA式会使中项两次不周延,第三格和第四格的AAA式都会造成前提中不周延的项在结论中周延的错误。所以,如果三段论的结论是A命题,唯一的有效形式就是第一格的AAA式,即AAA-1,传统上称这个有效形式为Barbara。

\textbf{情形1的总结}:如果三段论的结论是$\mathbf{A}$命题,只能有一种有效形式:AAA-1-Barbara。

\subsection{情形2:如果三段论的结论是$\mathbf{E}$命题}
E命题的主项和谓项都是周延的,因此,如果结论为E命题,三段论前提中的三个项也都必须至少周延一次$^{(1)}$,这只有当前提之一也是E命题时才有可能。但不能两个前提都是E命题,因为不能允许两个否定前提(规则4),同理可知另一个前提也不能是O命题。另一个前提也不能是I命题,否则在结论中周延的项在前提中不周延,违反规则3。这样,另一个前提必须是A命题,两个前提的组合可能是AE或EA。因此,在结论是E命题的情况下,可能的正确形式为AEE和EAE。

如果是AEE式,它不能是第一格,也不能是第三格。因为如果是这两个格的话,结论中周延的项在前提中不周延。所以,有效的AEE式只能是第二格的,即AEE-2(传统上称为Camestres),或者是第四格的,即AEE-4(传统上称为Camenes)。如果是EAE式,它不能是第三格,也不能是第四格,因为那也都导致结论中周延的项在前提中不周延。所以,有效的EAE式只能或者是第一格的,即EAE-1(传统上称为Celarent),或者是第二格的,即EAE-2(传统上称为Cesare)。

\textbf{情形2的总结}:如果三段论的结论是$\mathbf{E}$命题,只能有四种有效形式:AEE-2、AEE-4、EAE-1和EAE-2——分别是Camestres、Camenes、Celarent和Cesare。

\subsection{情形3:如果三段论的结论是I命题}
在这种情形下,前提不能是E或O命题,因为如果有一个否定前提的话,结论也应该是否定的。两个前提也不能都是A命题,因为结论为特称的三段论其前提不能都是全称的(规则6)。同样,两个前提也不能都是I命题,因为中项必须至少在一个前提中周延(规则2)。这样,前提的组合必须是AI或者IA,因而结论为I命题的三段论可能的有效形式为AII和IAI。

AII在第二格和第四格中不可能有效,因为中项至少要周延一次。因此保留下来的AII式就是AII-1(传统上称为Darii)和AII-3(传统上称为Datisi)。如果是IAI式,它不能是IAI-1和IAI-2,因为这两个形式都违反中项至少在一个前提中周延的规则。剩下的有效形式就是IAI-3(传统上称为Disamis)和IAI-4(传统上称为Dimaris)。

\textbf{情形3的总结}:如果三段论的结论是I命题,只能有四种有效形式:AII-1、AII-3、IAI-3和IAI-4——分别是Darii、Datisi、Disamis和Dimaris。

\subsection{情形4:如果结论是$\mathbf{O}$命题}
在这种情形下,大前提不能是I命题,因为结论中周延的项在前提中也必须周延。所以大前提可能是A命题、E命题或者O命题。

假设大前提是A命题。这样,小前提就不能是A命题和E命题,因为结论为特称(O命题)时,前提不能都是全称的。小前提也不能是I命题,否则,或者中项一次也不周延(违反规则2),或者结论中周延的项在前提中不周延。因此,如果大前提是A命题,小前提必须是O命题,结果就是AOO式。但在第四格,AOO式不可能有效,因为中项两次不周延。在第一格和第三格也不可能有效,因为结论中周延的项在前提中不周延。因此当大前提是A命题时,AOO式保留下来的有效形式只有第二格AOO-2(传统上称为Baroko)。

再假设(如果结论是O命题)大前提是E命题。在这种情况下,小前提将不能是E命题或O命题,因为不允许两个否定前提。小前提也不能是A命题,因为结论如果为特称的,前提就不能是两个全称命题(规则6)。因而只剩下了EIO式——它在四种格中都是有效的,传统上分别叫做Ferio(EIO-1)、Festino(EIO-2)、Ferison(EIO-3)和Fresison(EIO-4)。

最后,假设大前提是O命题。同样小前提也不能是E命题或O命题,因为不能允许两个否定前提。小前提也不能是I命题,因为那样的话,或者中项一次都不周延,或者结论中周延的项在前提中不周延。因此,如果大前提是O命题,小前提必须是A命题,即必为OAO式。但要排除OAO-1,因为中项两次都不周延。也要排除OAO-2和OAO-4,因为这两种情况都会使结论中周延的项在前提中不周延。于是就只剩下一个有效形式OAO-3(传统上称为Bokardo)。

\textbf{情形4的总结}:如果结论是O命题,则有六个有效形式:AOO-2、EIO-1、EIO-2、EIO-3、EIO-4和OAO-3,分别叫做Baroko、Ferio、Festino、Ferison、Fresison和Bokardo。

以上的分析通过排除法证明了直言三段论恰有15个有效形式:结论是A命题时有1个,结论是E命题时有4个,结论是I命题时有4个,而结论为O命题时有6个。这15个有效形式中,四个是第一格的,四个是第二格的,四个是第三格的,三个是第四格的。这样,就完成了标准式直言三段论的15个有效形式的演绎推导。

\footnotetext{(1)据规则$2、3$。}

\begin{center}
\fbox{\parbox{0.95\textwidth}{
\textbf{本节要点}
\begin{itemize}
\item 通过应用六条基本规则,可以演绎推导出15个有效的三段论形式
\item 根据结论的类型可分为四种情形:
  \begin{itemize}
  \item \textbf{结论是A命题}:只有1个有效形式(AAA-1 Barbara)
  \item \textbf{结论是E命题}:有4个有效形式(AEE-2, AEE-4, EAE-1, EAE-2)
  \item \textbf{结论是I命题}:有4个有效形式(AII-1, AII-3, IAI-3, IAI-4)
  \item \textbf{结论是O命题}:有6个有效形式(AOO-2, EIO-1, EIO-2, EIO-3, EIO-4, OAO-3)
  \end{itemize}
\item 这15个有效形式在四个格中的分布:
  \begin{itemize}
  \item 第一格:4个(AAA-1, EAE-1, AII-1, EIO-1)
  \item 第二格:4个(AEE-2, EAE-2, AOO-2, EIO-2)
  \item 第三格:4个(AII-3, IAI-3, EIO-3, OAO-3)
  \item 第四格:3个(AEE-4, IAI-4, EIO-4)
  \end{itemize}
\end{itemize}
}}
\end{center} 
\section*{第6章概要}
第6章考察标准式直言三段论:组成成分、形式、有效性和制约其正确使用的规则。

6. 1 节给出了三段论大项、小项和中项的定义:

\begin{itemize}
  \item 大项:结论的谓项
  \item 小项:结论的主项
  \item 中项:两个前提中都出现,但结论中不出现的第三个项
\end{itemize}

继而又分别定义了大前提和小前提,包含大项的前提叫做大前提,包含小项的前提叫做小前提。如果几个命题出现的次序正好是:大前提在第一位、小前提在第二位、结论在最后,我们就把这样的三段论指定为标准式的。\\
6.1 节也说明了三段论的式与格是如何确定的。

三段论的式由识别三个命题类型的字母来确定,即A、E、I、O中的三个。总共有 64 个不同式。

三段论的格由中项在前提中的不同位置来确定。对四个可能的格描述并定义如下:

第一格:中项在大前提中做主项、在小前提中做谓项。\\
模式为:$M-P, S-M$ ,所以 $S-P$ 。\\
第二格:中项在两个前提中都做谓项。\\
模式为:$P-M, S-M$ ,所以 $S-P$ 。\\
第三格:中项在两个前提中都做主项。

模式为:$M-P, M-S$ ,所以 $S-P$ 。\\
第四格:中项在大前提中做谓项、在小前提中做主项。\\
模式为:$P-M, M-S$ ,所以 $S-P$ 。\\
6.2 节说明了标准式三段论的式与格如何共同地确定其逻辑形式。由于 64 个式每一个都有四个格,所以共有 256 个标准式的直言三段论,但其中只有一小部分是有效式。

6. 3 节介绍检验三段论有效性的文恩图方法,即在几个交叉的圆中,作上恰当的标记或涂上阴影以表示前提的含义。\\
6.4 节阐明标准式三段论的六条基本规则,同时定义了违反各条规则所造成的谬误。\\
-规则 1 一个有效的标准式直言三段论必须仅仅包含三个项,在整个论证中,每一个项都须在相同的意义上使用。

违反本规则所犯的错误:四项谬误。\\
-规则 2 在一个有效的标准式直言三段论中,中项必须至少在一个前提中周延。

违反本规则所犯的错误:中项不周延谬误。\\
-规则 3 在一个有效的标准式直言三段论中,在结论中周延的项在前提中也必须周延。

违反本规则所犯的错误:大项不当周延谬误,或者小项不当周延谬误。\\
-规则 4 任何有两个否定前提的标准式三段论都不是有效的。\\
违反本规则所犯的错误:排斥前提谬误。\\
-规则 5 如果一个标准式三段论有一个前提是否定的,那么结论必须是否定的。

违反本规则所犯的错误:从否定推肯定谬误。\\
-规则 6 一个有效的标准式直言三段论,如果结论为特称命题,那么其前提不能都是全称的。

违反本规则所犯的错误:存在谬误。\\
6. 5 节给出了标准式直言三段论的 15 个有效形式的说明,识别它们的格与式,并说明了它们传统的拉丁名称:

AAA-1(Barbara)、EAE-1(Celarent)、AII-1(Darii)、EIO-1(Fe- rio)、AEE-2(Camestres)、EAE-2(Cesare)、AOO-2(Baroko)、EIO-2\\
(Festino)、AII-3(Datisi)、IAI-3(Disamis)、EIO-3(Ferison)、OAO-3 (Bokardo)、AEE-4(Camenes)、IAI-4(Dimaris)、EIO-4(Fresison)。

6. 6 节展示了 15 个有效形式的演绎推导,通过排除法程序,证明了只有 15 个形式是完全遵守三段论的六条基本规则的。 

% 第七部分
\chapter{逻辑与语言}
\section{日常语言中的三段论论证}

\begin{logicbox}[title=引言]
日常语言中的三段论论证往往不像\logicterm{标准式三段论}那样整齐规范。本节将介绍如何识别和转换日常语言中的三段论论证,使其符合标准形式,便于进行\logicemph{有效性}检验。
\end{logicbox}

前几章考察的\logicterm{标准式直言三段论}往往显得生硬、不自然。它们就像 "化学纯净物"一样,不含任何杂质和不相关的东西。但是,日常语言论证并不总是这么整齐划一地出现的。在此,我们更广义的使用\logicterm{三段论}这一术语,用来指谓符合如下条件的任一论证:或者本来就是\logicterm{标准式直言三段论},或者是可以变形为\logicterm{标准式直言三段论}而没有失掉或改变原意的论证。

\begin{theorembox}[title=日常语言三段论的检验方法]
三段论论证相当常见,所以我们要设法检验其\logicemph{有效性}。但由于日常论证通常比标准形式松散,前面提到的检验方法——\logicterm{文恩图}和直言三段论的规则一一不能直接适用于它们。日常的三段论论证形式变化多样,不可能为每一种形式都发明一个特殊的检验方法,除非有一种极度复杂的逻辑工具。要检验众多三段论论证的\logicemph{有效性},最明智的方法通常是:在不改变原意的前提下,把它们\logicterm{变形}(reformulate)为\logicterm{标准式三段论}。这个方法就是向标准形式的\logicterm{化归}(reduction)或\logicterm{翻译}(translation),最后得到的三段论叫做原给定三段论的\logicterm{标准式翻版}。
\end{theorembox}

评估日常语言三段论要满足两个条件。首先,要有一种便于应用的检验方法,将\logicterm{标准式三段论}的\logicemph{有效式}和\logicwarn{无效式}区分开来,这种方法我们已经有了(前面章节中讲到的图示和规则)。其次,要有一种\logicterm{翻译}方法,将任何形式的三段论推理转变为标准形式,一旦掌握了这种方法,再用先前介绍的判定\logicemph{有效}三段论的规则或\logicterm{文恩图解}方法进行检验,我们就能评估任何三段论。

\subsection{非标准形式的三种偏离情形}

\begin{examplebox}[title=非标准形式的三种偏离情形]
要说明将日常语言中的非标准三段论论证\logicterm{翻译}为标准形式的方法,首先要区分非标准形式偏离标准形式的不同情形。下面是三种基本的偏离情形:

1.前提和结论的顺序不标准。这是小问题,因为如果仅仅是叙述的顺序不标准,很容易调整过来。

2.日常语言论证的构成命题中表面上包含不止三个项,但可以证明事实上并非如此。

3.日常语言论证的构成命题不都是\logicterm{标准式直言命题}。

第二、三种偏离情形同样有可能\logicterm{翻译}为标准形式,下面即讨论翻译方法。
\end{examplebox}

\begin{center}
\fbox{\parbox{0.95\textwidth}{
\textbf{本节要点}
\begin{itemize}
\item \logicterm{日常语言三段论}是可变形为\logicterm{标准式直言三段论}而不失去或改变原意的论证
\item 检验日常三段论\logicemph{有效性}的两个条件:
  \begin{itemize}
  \item 有将\logicterm{标准式三段论}区分\logicemph{有效}与\logicwarn{无效}的检验方法
  \item 有将任何形式的三段论推理转变为标准形式的\logicterm{翻译}方法
  \end{itemize}
\item 日常语言三段论偏离标准形式的三种情形:
  \begin{itemize}
  \item 前提和结论顺序不标准
  \item 表面上包含超过三个词项
  \item 构成命题不都是\logicterm{标准式直言命题}
  \end{itemize}
\end{itemize}
}}
\end{center}
\section{三段论词项数量的归约}

\begin{quotation}
日常语言中的三段论论证常常看似包含超过三个词项,但通过适当的转换可以归约为标准的三词项形式。本节介绍两种主要的归约方法:去除同义词和处理补类,这些方法帮助我们将复杂的日常语言论证化简为可用标准规则检验的形式。
\end{quotation}

如果日常语言中的一个论证看起来有三段论的形式,但包含着三个以上的词项,那么不应该即刻把它看成犯了\textbf{四项谬误},从而认为它是无效的。这样的论证往往能被翻译为与之逻辑上等价的只有三个词项且完全有效的标准形式三段论。完成这样的翻译要掌握两种方法:

\subsection{去除同义词}
在应用文恩图或三段论规则之前,应当去除日常语言论证中的同义词。举例来说,这样一个论证:

\begin{quote}
没有富人(wealthy)是游民(vagrant),

所有律师(lawyer)都是有钱人(rich people),

所以,没有法律代理人(attorney)是流浪者(tramps)。
\end{quote}

其中包含着"富人"、"律师"和"游民"的同义词。去除同义词之后,该论证可翻译为:

\begin{quote}
没有富人是游民,

所有律师都是富人,

所以,没有律师是游民。
\end{quote}

这个三段论是标准的 EAE-1(Celarent),很明显是有效的。

\subsection{去除补类}
有时仅仅去除同义词是不够的。来看下面这个论证,其中所有命题都是标准式直言命题:

\begin{quote}
所有哺乳动物是温血动物,

没有蜥蜴是温血动物,

所以,所有蜥蜴都是非哺乳动物。
\end{quote}

如果直接用第6章给出的三段论规则来检验,这个三段论违反了不止一个规则。一方面,它包含着四个词项:"哺乳动物"、"温血动物"、"蜥蜴"和"非哺乳动物"。另一方面,它从否定前提得到了一个肯定结论。但实际上这个推理是有效的。因为其中虽含有四个词项,但不是标准形式,不能直接用三段论规则检验。要想用第6章给出的几个规则来检验,必须首先把它翻译为标准形式。这是很容易的,因为四个词项中有两个("哺乳动物"和"非哺乳动物")互为\textbf{补类}。如果将结论进行\textbf{换质},就可以减少词项的数量——翻译的结果是原论证的一个标准式翻版:

\begin{quote}
所有哺乳动物是温血动物,

没有蜥蜴是温血动物,

所以,没有蜥蜴是哺乳动物。
\end{quote}

它与原来论证的前提相同而结论等价,所以两者在逻辑上是等价的。这个标准式翻版遵守所有规则因而是有效的。其形式为 AEE-2(Camestres)。

尽管后者是最容易得到的,但它并不是唯一的标准式翻版。还可以对第一个前提进行\textbf{换位}、对第二个前提进行换质,而不改变结论,就可以得到另一个不同(但逻辑上等价的)标准式翻版:

\begin{quote}
所有非温血动物是非哺乳动物,

所有蜥蜴是非温血动物,

所以,所有蜥蜴都是非哺乳动物。
\end{quote}

这是一个 AAA-1(Barbara),也是遵守规则的有效式。对给定的三段论论证进行翻译,并没有唯一固定的标准形式,但如果其中一个是有效的,那么其他所有翻版都应该是有效的。

如果四个词项中有两个互为补类,那么任何含有这样的四个词项的三段论都可以化归为标准形式(或逻辑上等价的标准直言三段论);如果其中两个(或三个)与另外两个(或三个)互为补类,那么任何含有五个(或六个)词项的三段论也都可以化归为标准形式。这种化归都是通过\textbf{换位法}、\textbf{换质法}、\textbf{换质位法}等\textbf{直接推论}而实现的,这些方法在5.5节都讲过。

\subsection{多重词项的归约}
一个三段论论证,其构成命题如果都是标准式直言命题,它有可能含有半打不同的词项,要把它化归为标准形式,进行一次直接推论是不够的。下面的例子就是一个六词项的三段论,但它的确是有效的:

\begin{quote}
没有非居民是公民,

所有非公民是非选举人,

所以,所有选举人都是居民。
\end{quote}

可以用两种方法进行化归,第一种方法需要用到直接推论的三种方法,或许这是最自然也最明显的方法。首先把第一个前提换位再换质,把第二个前提换质位,于是得到如下一个标准式直言三段论:

\begin{quote}
所有公民都是居民,

所有选举人都是公民,

所以,所有选举人都是居民。
\end{quote}

这也是一个 Barbara 式,用第6章阐明的任何一种方法都很容易证明它是有效的。

\begin{center}
\fbox{\parbox{0.95\textwidth}{
\textbf{本节要点}
\begin{itemize}
\item 日常语言论证看似包含超过三个词项时,可通过两种方法归约:
  \begin{itemize}
  \item \textbf{去除同义词}:识别并合并表达相同概念的不同词项
  \item \textbf{处理补类}:通过换质、换位等方法消除互为补类的词项
  \end{itemize}
\item 归约方法基于\textbf{直接推论}技术:
  \begin{itemize}
  \item 换位法:调换主谓项位置
  \item 换质法:改变命题的质(肯定变否定或否定变肯定)
  \item 换质位法:同时改变质和主谓项位置
  \end{itemize}
\item 一个论证可能有多个不同但逻辑等价的标准式翻版
\item 甚至包含五个或六个词项的论证,通过适当的归约也能检验其有效性
\end{itemize}
}}
\end{center} 
\input{chapter7/7-3 直言命题的标准化.tex}
\section{协同翻译}

\begin{logicbox}[title=引言]
本节介绍协同翻译方法,通过在三段论的构成命题中引入同一个参项,帮助我们将看似复杂的日常语言论证转换为标准的三词项形式,从而能够用正规方法检验其有效性。
\end{logicbox}

要对三段论论证进行有效性检验,其中总共只能包含三个项。有时做到这一点很难,需要比前面所述方法更细致的处理。请考虑命题"你总是与穷人为伍",显然,它既不是断言所有穷人总是在你身边,也不是说有些(特称的)穷人总是(always)在你身边。把该命题化归为标准形式的一种方法,也是一种最自然的方法,就是从其中的关键词"总是"着手分析。这个词意味着"在所有时间"(at all time),它表明原命题的一种标准式翻版为"所有时间都是你与穷人为伍的时间"。主、谓项中都出现的"时间"这个词可视为一个\textbf{参项}(parameter)。所谓参项,就是一个有助于以标准形式表达原来断言的辅助词项。

\subsection{参项的选择}

当然,绝不能机械地、不加思考地引入和使用参项,必须始终以所要翻译的那个命题为依据。命题"史密斯总是在台球比赛中获胜",显然断定的并不是史密斯从不间断地、始终在获胜!较合理的解释是,这句话是说:每当史密斯玩台球时,他就会获胜。如果这么理解,就可以直接把原句转化为"所有史密斯玩台球的时间是他获胜的时间"。并非所有参项都是时间性的。在对另一些命题进行翻译时,"地点"(place)、"情形"(case)也能被用做参项。例如"没有幻想的地方人类就会毁灭"(Where there is no vision the people perish)和"每当琼斯迟到就丢失一次推销机会"(Jones loses the sale whenever he is late)可分别译为:"所有没有幻想的地方都是人类毁灭的地方"和"所有琼斯迟到的情形都是他丢失推销机会的情形"。

\subsection{应用协同翻译}

在对三段论的三个构成命题进行协同翻译的过程中,参项的引人是必不可少的。一个直言三段论恰好包含三个项,要检验三段论就必须把它的构成命题都转化为标准式直言命题,其中只出现三个项。去除同义词以及换位法、换质法、换质位法的运用已经在7.2节讨论过。即使这样,还有很多三段论论证的项数仍然不能被缩减到三。此时,\textbf{协同翻译}就需要把同一个参项引到三个构成命题中去。请看下面这个论证:

\begin{quote}
哪里脏纸盒散落哪里就曾有不自爱者在此野餐,这里散落着脏纸盒,
\end{quote}

所以,一定有不自爱者在这里野餐过。

这个推理是完全有效的,但只有把前提和结论都翻译为标准式直言命题,并且其中只能有三个项,才能用文恩图或三段论规则来证明其有效性。第二个前提和结论能很自然地译为"有脏纸盒是散落在这里的东西"和"有不自爱者是在这里野餐过的人"。但这两个陈述句当中有四个不同的项。要把给定的论证化归为标准形式,就需要在三个命题中使用同一个参项。我们从第一个前提着手寻找这个参项,然后再用同样的参项去翻译第二个前提以及结论。"哪里"一词表明可以用"地方"作为参项。如果翻译三个命题时都用这个参项,则论证可变为:

所有散落脏纸盒的地方是不自爱者野餐过的地方,
这个地方是散落脏纸盒的地方,
所以,这个地方是不自爱者野餐过的地方。

这个标准式直言三段论的形式为 AAA-1,即 Barbara,是一个已经被证明有效的形式。

\subsection{协同翻译的进阶案例}

利用参项使表达式标准化的方法不是很容易掌握的,但有些三段论论证的确无法用其他方法进行翻译。再看一个例子有助于弄清其中的技巧:

每当狐狸经过那里,猎犬一定会发出叫声,
所以,狐狸走的一定是别的路,因为猎犬都很安静。

首先,我们必须明白上述论证说的是什么。要把"猎犬很安静"这句话理解为"猎犬此时此地没有发出叫声"。这一步是去除同义词的必需步骤,因为第一个命题说的是"猎犬发出叫声"。同样,"狐狸走的一定是别的路"的结论,应理解为断言"狐狸没有经过那里"。第一个前提中的"那里"一词表明翻译时也可用"地方"做参项。于是,可得到这样一个标准式翻版:

$$
\begin{aligned}
& \text { 所有狐狸经过的地方是猎犬发出叫声的地方, } \\
& \text { 这个地方不是猎犬发出叫声的地方, } \\
& \text { 所以, 这个地方不是狐狸经过的地方。 }
\end{aligned}
$$

这个标准式直言三段论的形式为 AEE-2,即 Camestres,其有效性很容易确定。

\begin{center}
\fbox{\parbox{0.95\textwidth}{
\textbf{本节要点}
\begin{itemize}
\item \textbf{协同翻译}是通过在三段论命题中引入同一个\textbf{参项}来实现标准化的方法
\item 常见的参项类型包括:
  \begin{itemize}
  \item "时间":用于翻译包含时间性词语(如"总是"、"每当")的命题
  \item "地点":用于翻译包含空间性词语(如"哪里"、"那里")的命题
  \item "情形":用于翻译表示条件关系的命题
  \end{itemize}
\item 协同翻译的步骤:
  \begin{itemize}
  \item 识别适当的参项(依据原命题含义)
  \item 将同一参项引入三个构成命题中
  \item 使各命题符合标准直言命题形式
  \end{itemize}
\item 协同翻译成功后,可以获得标准的三词项三段论,便于用常规方法检验其有效性
\end{itemize}
}}
\end{center}
\section{省略三段论(Enthymemes)}

\begin{quotation}
本节讨论省略三段论,即那些前提和结论未完全陈述、部分需要听者或读者"领会"的不完整论证。我们将学习如何识别、补充和评估这类在日常语言和科学中广泛使用的论证形式。
\end{quotation}

三段论论证很常用,但其前提和结论并不总是都得到明确地陈述。常常只把论证的一部分表述出来,而其余部分就要靠"领会"了。比如,只提及"琼斯是个土生土长的美国人"这个前提,就可以得到结论:"琼斯是美国公民"。上述论证的表述并不完整,但很容易根据美国宪法把省略的前提补出来。加上被省略了的前提,完整的论证就是:

\begin{quote}
所有土生土长的美国人是公民,

琼斯是土生土长的美国人,

所以,琼斯是美国公民。
\end{quote}

完整表述后,这个论证就是一个直言三段论,其形式为 AAA-1,即 Barbara,它完全有效。如果一个推理是不完整的,其中有一部分需要"领会"或仅仅"在心中",我们就称之为\logicterm{省略三段论}。省略(enthymematic)是不完整三段论的特征。

\subsection{省略三段论的原因与特点}

在日常话语甚至科学中,许多推论都是省略式的。原因不难理解,因为有相当一部分命题是公共知识。对于那些广为人知或无关紧要的真命题,听众和读者很容易就能想到并且补充完整,说话者和写作者就不再重复以减少麻烦。另外,用省略式描述推理,能够增加修辞效果,比描述出所有细节更强、更有说服力。亚里士多德在其著作《修辞学》中写道:"基于省略三段论的……演讲更受人欢迎。"

由于省略式不完整,所以要检验其有效性必须找到被省略的部分。如果省去的是一个必不可少的前提,缺了它推论就是无效的。但是如果省略的前提很容易补出来,评估时应该把它包括在论证当中才是公平的。这时,要假定论证者心中所想比明确说出的信息更多些。大多数情况下,很容易将说话人(或写作者)想到而没有表达出来的前提补充完整。举例来说,要说明"银色马"的怪事,神探福尔摩斯构造了这样一个推论,其中就省略了关键性前提,但是很容易猜到:

\begin{quote}
马厩中有一条狗。然而,尽管有人进来,并且把马牵走,狗却没有出声……显然,来者是这只狗非常熟悉的人物。
\end{quote}

我们都能很好地理解其中暗含的意思:如果来者是陌生人,狗就会发出叫声。把这个前提看做福尔摩斯论证的一部分,对作者柯南·道尔来说才是公平的。

\subsection{补充隐藏前提的原则}

补充隐藏的前提时最重要的原则是:说话人确实认为听者可以接受这个命题为真。因此,要是把结论本身当做隐藏的前提就太愚蠢了。如果论证者希望听者把它当做前提而不加证明,那就无须再作为论证的结论来表述。

\subsection{省略三段论的类型}

任何论证都能以省略式表达,但得到最广泛研究的还是三段论的省略式,本节也只限于研究三段论的省略问题。根据未表述部分的不同,传统上把省略三段论分为几种不同的省略体。

\textbf{第一种省略体}指不出现大前提的情形。上面的例子就是这种省略体。

\textbf{第二种省略体}保留大前提和结论,而不出现小前提。例如"所有学生都是反对新规则的,所以,所有大二学生都是反对新规则的",这里的小前提是个明显为真很容易补充出的命题:"所有大二学生是学生"。

\textbf{第三种省略体}中两个前提都出现,但未表述结论。下面就是这个类型的例子:

\begin{quote}
我们的观念超不出我们的经验;我们没有关于神圣的属性与作为的经验;我们用不着为我这个三段论下结论:你自己能得出推论来。\cite{hume1748}
\end{quote}

\subsection{省略三段论的检验}

检验省略三段论的有效性共需两步:首先恢复省略的部分,然后再检验。公正地表示出省去的命题,需要语境敏感性以及对说话者意图的理解。请看这样一个论证:"没有真正的基督徒是精神空虚的,但有些常去教堂礼拜的人是精神空虚的",其中没给出结论,属第三种省略体,那么,原本要得出的结论是什么呢?如果说话者是要得出"有些常去教堂的人不是真正的基督徒",那么,推理就是有效的(EIO-2,Festino);但是如果说话者想说的是"有些真正的基督徒不是常去教堂的人",那么,这个省略式就是无效的(IEO-2),犯了大项不当周延谬误。

但一般说来,语境可以无歧义地确定未表述的命题。例如根据最高法院的意见,控制州内性暴力的联邦立法("针对妇女的暴力行为法案")是违反宪法的,大多数法官的关键性论证如下:

\begin{quote}
在任何意义上,性暴力犯罪都不是经济行为……迄今为止,在美国的历史上最高法院的判例中,只有经济行为才适用控制州内行为的条款。\cite{usc2000}
\end{quote}

可以领会但没有明确表述的结论是:根据最高法院的长期实行的规则,性暴力犯罪不归国会控制。

检验第三种省略体,先要把前提和(显而易见的)结论变形为标准形式。首先陈述大前提(含有结论之谓项的前提),然后确定其式与格,如上例:

大前提:根据最高法院的规则,国会控制的所有行为是经济行为。

小前提:没有州内的性暴力犯罪是经济行为。

结论(并未表述但结合语境却很清楚):没有州内的性暴力犯罪是最高法院规定为国会控制的。

这个三段论的式为AEE,中项在两个前提中都做谓项,因此是第二个格,其形式是 Camestres——有效的三段论。

第三种省略体,在某些情况下,即使不结合语境也能看出是无效的——例如,两个前提都是否定的,或者都是特称的,或者中项不周延。如果这样的话,不可能得出有效的结论。因此,这种省略式在任何语境中都是无效的。

也可能有这样的情况,在省略的是论证的一个前提的情况下,只有加上一个高度不合理的前提,才能把论证写成有效式。此时,指出这一点就构成对省略三段论的一种合理批判(legitimate criticism)。当然,更具毁灭性的批判是:有些三段论无论补上什么样的前提(即使是不合理的前提),也不能成为有效的三段论。

省略三段论与普通三段论的区别,从本质上说是修辞上的,而不是逻辑上的。不需添加什么新规则就能处理省略式,它们终究要接受与标准直言三段论同样的检验。

\begin{center}
\fbox{\parbox{0.95\textwidth}{
\textbf{本节要点}
\begin{itemize}
\item \logicterm{省略三段论}是前提或结论部分未明确表述的不完整论证
\item 省略三段论广泛存在的原因:
  \begin{itemize}
  \item 某些命题为公共知识,无需重复陈述
  \item 增强修辞效果和说服力
  \end{itemize}
\item 省略三段论的三种主要类型:
  \begin{itemize}
  \item \textbf{第一种省略体}:省略大前提
  \item \textbf{第二种省略体}:省略小前提
  \item \textbf{第三种省略体}:省略结论
  \end{itemize}
\item 补充隐藏前提的原则:所补前提应是说话人认为听者能接受为真的命题
\item 检验省略三段论的两个步骤:
  \begin{itemize}
  \item 恢复省略的部分(依据语境和说话者意图)
  \item 对完整三段论进行有效性检验
  \end{itemize}
\item 某些省略三段论无论如何补充都不可能有效,这构成对其最有力的批判
\end{itemize}
}}
\end{center} 
\section{连锁三段论(Sorites)}

\begin{logicbox}[title=引言]
本节讨论连锁三段论,即前提多于两个的三段论论证,其中每个结论都作为下一个推理的前提。我们将学习如何将复杂的连锁推理拆解为一系列标准三段论,并通过分析每个环节来评估整体论证的有效性。
\end{logicbox}

有时会出现一个三段论论证,其前提多于两个。如果其结论是由前提依次推得的,那么它就是有效的,否则就是无效的。例如下面这个论证,它的前提有四个:

\begin{quote}
所有外交官都是机敏的人,

有外交官是欠思考的人,

所有欠思考的人都是轻率的,

没有轻率的人是谨慎的,

所以,有谨慎的人不是机敏的。
\end{quote}

这个论证可以通过一系列环环相扣的直言三段论来进行检验。如果能把这个链条上的所有三段论都写出来,那么,任何一个违反了三段论六条规则的三段论都会使整个推理无效。

\subsection{连锁三段论的拆解分析}

上述论证中的结论("有谨慎的人不是机敏的")可以由前提"没有轻率的人是谨慎的"和一个未出现的命题共同推出,这个未出现的命题就是"有轻率的人是机敏的"。这个未出现的命题,正是前面三个前提的结论。这样,我们就可以从一个论证推出另一个来。第一个论证是:

\begin{quote}
所有外交官都是机敏的人,

有外交官是欠思考的人,

所有欠思考的人都是轻率的,

所以,有轻率的人是机敏的。
\end{quote}

而第二个论证的两个前提是:第一个论证的结论,以及原论证的第四个前提("没有轻率的人是谨慎的")。第二个论证是:

\begin{quote}
有轻率的人是机敏的,

没有轻率的人是谨慎的,

所以,有谨慎的人不是机敏的。
\end{quote}

这样,就可以分别检验这两个三段论了。如果两个都有效,原论证就有效。由于第二个论证(结论是"有谨慎的人不是机敏的")的前提中,"轻率的"是中项,但两个前提中都没有出现"机敏的人"(大项)和"谨慎的人"(小项),所以,它并不符合标准形式。第二个论证的大前提(其中有大项)是"有轻率的人是机敏的",小前提是"没有轻率的人是谨慎的"。

这样,第二个论证的形式就是 IEO-3。这个形式违反了规则 3,因为大项在结论中周延而在前提中不周延,因此犯了\textbf{大项不当周延谬误}。这样,原论证的第二个环节无效,就使得整个论证无效。

\subsection{标准式连锁三段论}

这种包含几个前提和若干结论的三段论,如果每一个结论都成为下一个三段论的前提,就称为\textbf{连锁三段论}(sorites)。如果这些前提都是以标准形式排列,也就是说,每个词项(除了第一个前提的主项和最后一个前提的谓项)都分别作为前提的主项和谓项出现,这样的连锁三段论就可以看做是标准式的。如下例所示:

\begin{quote}
所有 $A$ 是 $B$,

所有 $B$ 是 $C$,

所有 $C$ 是 $D$,

没有 $D$ 是 $E$,

所以,没有 $A$ 是 $E$。
\end{quote}

任何一个标准形式的连锁三段论都可以通过依次进行的三段论推论而得到检验。例如上面的连锁三段论,就可以通过如下三个三段论进行检验:

(1)
\begin{quote}
所有 $B$ 是 $C$,

所有 $A$ 是 $B$,

所以,所有 $A$ 是 $C$。
\end{quote}

(2)
\begin{quote}
所有 $C$ 是 $D$,

所有 $A$ 是 $C$,(前一个三段论的结论)

所以,所有 $A$ 是 $D$。
\end{quote}

(3)
\begin{quote}
没有 $D$ 是 $E$,

所有 $A$ 是 $D$,(前一个三段论的结论)

所以,没有 $A$ 是 $E$。
\end{quote}

这里所有的三段论都是第一格的。第一个和第二个是 Barbara 式,第三个是 Celarent 式,它们都是有效的。因此,原连锁三段论是有效的。

\subsection{连锁三段论的检验}

一个连锁三段论的前提可以写成任何顺序,为了检验其有效性,需要先把它们整理为标准顺序。一个标准式连锁三段论的有效性(或无效性)取决于构成它的所有三段论的有效性(或无效性)。

\begin{center}
\fbox{\parbox{0.95\textwidth}{
\textbf{本节要点}
\begin{itemize}
\item \textbf{连锁三段论}(sorites)是前提多于两个的三段论论证
\item 连锁三段论的特点:
  \begin{itemize}
  \item 每个中间结论都作为下一个推理的前提
  \item 形成一系列环环相扣的标准三段论
  \end{itemize}
\item 检验连锁三段论的方法:
  \begin{itemize}
  \item 将复杂论证拆解为一系列标准三段论
  \item 按顺序检验每个三段论的有效性
  \item 只有所有环节都有效,整个连锁三段论才有效
  \end{itemize}
\item 标准式连锁三段论的特点:
  \begin{itemize}
  \item 每个词项(除首尾)都分别作为不同前提的主项和谓项
  \item 前提排列有特定顺序,便于拆解分析
  \end{itemize}
\item 任何一个环节出现无效,如犯大项不当周延谬误,都会使整个连锁三段论无效
\end{itemize}
}}
\end{center}
\section{析取三段论和假言三段论}

\begin{quotation}
本节讨论两种重要的三段论形式:析取三段论和假言三段论。这些三段论不同于直言三段论,它们涉及选言和条件关系,在日常推理中有广泛应用。我们将学习这些三段论的有效形式以及常见的相关谬误类型。
\end{quotation}

一个三段论就是包含两个前提的演绎论证。可以有许多不同种类的三段论,最常见、最重要的就是直言三段论,前面几章已经对它进行了详细的考察。本节将简要讨论其他几种三段论。

\subsection{析取三段论(Disjunctive Syllogisms)}
在一个析取(或"选言")三段论中,有一个前提是析取命题。例如:

或者傻瓜,或者无赖,\\
他不是傻瓜,\\
所以,他是个无赖。

这样的论证是有效的。\cite{boole1854} 传统上把这种论证形式称为"通过否定进行肯定"(modus tollendo ponens),即通过否定其中一个选言支来肯定另一个选言支。有效的析取三段论定义如下:

一个析取三段论是有效的,当且仅当其中一个前提是析取命题,另一个前提是对其中一个选言支的否定,而结论是未被否定的那个选言支。

这里讨论的析取命题指的是"弱的"或"可兼的"析取(见8.1节),意思是它断言的是至少有一个选言支为真,但可能两个都真。如果析取三段论的前提是一个强的(或"不可兼的")析取命题,即断言恰好只有一个选言支为真,另一个为假,那么,下面的论证形式也是有效的:

或者傻瓜,或者无赖,\\
他是个傻瓜,\\
所以,他不是无赖。

传统上把这种论证形式称为"通过肯定进行否定"(modus ponendo tollens),即通过肯定其中一个选言支来否定另一个选言支。

大多数情况下,不通过语境就无法确定析取命题到底是强的还是弱的。但是,对于有效的析取三段论的定义,并不需要这种区分。这个定义对于两种析取命题都适用。

如果一个论证具有析取三段论的形式,但并不符合这里的定义,那么,这种论证就是无效的,它所犯的错误传统上称之为\textbf{肯定选言支谬误}。这种谬误往往出现在弱析取命题的论证中,比如"或者傻瓜,或者无赖。他是个傻瓜,所以他不是无赖"。如果"或者傻瓜,或者无赖"被解释为弱析取,那么,即使两个前提都真,结论也可能为假。因此,这样的论证是无效的。

\subsection{假言三段论(Hypothetical Syllogisms)}
标准形式的假言三段论包含两个假言命题作为前提,外加一个假言命题作为结论。例如:

如果第一个土著人是政客,那么他会说谎,\\
如果他说谎,那么他会否认自己是政客,\\
所以,如果第一个土著人是政客,那么他会否认自己是政客。

这个论证叫做\textbf{纯粹假言三段论},因为它的前提和结论都是假言命题。这种论证是有效的。\cite{boole1854}

还有一种常见形式的假言三段论,它有一个假言前提,一个直言前提,以及一个直言结论。有两种这样的有效形式,在此要分别进行讨论。

第一种有效的形式,传统上称为\textbf{肯定前件}(modus ponens),例如:

如果第二个土著人说真话,那么只有他是政客,\\
第二个土著人说真话,\\
所以,只有他是政客。

这里,第一个前提是假言命题,第二个前提肯定了第一个前提的前件,而结论则肯定了第一个前提的后件。这种形式的任何论证都是有效的。

但是,如果一个论证具有与肯定前件相似的形式,但却不是有效的,这种错误就称为\textbf{肯定后件谬误}。例如:

如果培根写了《哈姆雷特》,那么培根是个大作家,\\
培根是个大作家,\\
所以,培根写了《哈姆雷特》。

显然,这个论证是无效的。第二个前提肯定了第一个前提的后件,而结论肯定了第一个前提的前件。

第二种有效的假言三段论,传统上称为\textbf{否定后件}(modus tollens),例如:

如果这位客人是陌生人,那么狗会叫,\\
狗没有叫,\\
所以,这位客人不是陌生人。

这里,第一个前提是假言命题,第二个前提否定了第一个前提的后件,而结论否定了第一个前提的前件。这种形式的任何论证都是有效的。

但是,如果一个论证具有与否定后件相似的形式,但却不是有效的,这种错误就称为\textbf{否定前件谬误}。例如:

如果洛克菲勒拥有福特汽车公司的全部黄金,那么洛克菲勒很富有,\\
洛克菲勒并不拥有福特汽车公司的全部黄金,\\
所以,洛克菲勒并不富有。

显然,这个论证是无效的。第二个前提否定了第一个前提的前件,而结论否定了第一个前提的后件。

\begin{center}
\fbox{\parbox{0.95\textwidth}{
\textbf{本节要点}
\begin{itemize}
\item \textbf{析取三段论}的特点:
  \begin{itemize}
  \item 一个前提是析取命题
  \item 有效形式:"通过否定进行肯定"(modus tollendo ponens)
  \item 对于强析取命题,"通过肯定进行否定"(modus ponendo tollens)也有效
  \item 常见错误:肯定选言支谬误
  \end{itemize}
\item \textbf{假言三段论}的类型:
  \begin{itemize}
  \item 纯粹假言三段论:两个假言前提,一个假言结论
  \item 混合假言三段论:一个假言前提,一个直言前提,一个直言结论
  \end{itemize}
\item 混合假言三段论的有效形式:
  \begin{itemize}
  \item \textbf{肯定前件}(modus ponens):如果P那么Q,P,所以Q
  \item \textbf{否定后件}(modus tollens):如果P那么Q,非Q,所以非P
  \end{itemize}
\item 混合假言三段论的常见谬误:
  \begin{itemize}
  \item \textbf{肯定后件谬误}:如果P那么Q,Q,所以P
  \item \textbf{否定前件谬误}:如果P那么Q,非P,所以非Q
  \end{itemize}
\end{itemize}
}}
\end{center} 
\section{二难推论(The Dilemma)}

\begin{logicbox}[title=引言]
本节讨论二难推论这一强有力的论证形式,它使对手面临两种选择,而无论选择哪一种,都会导致不利的结论。我们将分析二难推论的结构、类型以及应对二难推论的三种主要方法。
\end{logicbox}

没有什么特别重要的地方。但从修辞角度看,二难推论是一种最有力量的说服工具之一,可谓论战中的一种致命性武器。

不严格地说,如果一个人必须在两种选项中做出决断,但两个选项都很糟糕或令人不愉快,那么,我们就说这个人"陷入"了两难(或者说进退维谷)之中。\textbf{二难推论}就是一种旨在使对手陷入这样境地的论证方式。在争论过程中,二难推论使得对手必须做出选择,但无论选择什么,都会得出一个他不能接受的结论。

理查德·费曼(Richard Feynman)是一位著名的物理学家,他在回忆1986年"挑战者"号爆炸的调查时,猛烈地抨击了(美国)国家航空航天局(NASA)的管理失误,他用的就是下面的二难推论:

\begin{quote}
我们每次问起高层管理者,他们都会说关于手下发生的事,他们什么都不知道……或者最高领导团确实不知道,这样他们就不知道应该知道的事,或者他们知道,这样他们就在对我们说谎。\cite{feynman1988}
\end{quote}

如此的质问就将对手(此处指的是国家航空航天局的管理者们)推入两难境地,令他们无地自容。其中唯一明确表述的前提是一个析取命题,但析取支必定有一个为真,或者他们知道或者他们不知道手下发生的事。不管选择哪一方,结果对对手来说都是不利的。二难推论的结论本身也可以是一个析取命题(例如,"国家航空航天局的管理者或者不知道他们应该知道的事,或者他们说谎"),此时我们称之为\textbf{复杂式}(complex)二难推论。结论也可以是直言命题,这时就称之为\textbf{简单式}二难推论。

二难推论的结论并非总是令人不愉快的,如下简单式二难推论得出的就是个好结论:

\begin{quote}
如果天上的神明没有欲求,那么他们就会很满足,如果他们有欲求而能完全实现,那么他们也会很满足。他们或者没有欲求或者能完全实现欲求。总之,他们都会很满足。
\end{quote}

二难推论的前提并没有特殊的顺序要求,提供选项的析取前提可前可后。表述选择后果的两个条件命题可以联合表述,也可分开陈述。二难推论常用省略式表述,结论一般都是显而易见的,无须表述出来。有一个例子取自林肯总统的一封信,他为废止美国南部邦黑奴制度的宣言作了如下辩护:

\begin{quote}
此宣言如同法律一样,或者有效或者无效。如果无效,就没必要取消。如果有效,就不能取消。任何人都明白。\cite{lincoln1861}
\end{quote}

\subsection{避开二难推论的方法}

避开或驳斥二难推论的结论的方法有三种,它们也有各自的名称,都与二难的两个(或多个)"死角"有关。分别称为"绕过(或避开)死角法"、"直击(擒拿)一角法"、"构造反二难法"。它们并非证明二难推论形式无效,而是在不改变推论形式有效性的前提下,寻找避免结论的方法。

\subsubsection{绕过死角法}

\textbf{绕过死角法}是拒斥其析取前提。这是常用的最容易的避开二难的手段。除非析取前提的两个支命题是矛盾关系,否则它们很有可能是假的。常用来说明这个方法的例示是给学生分级打分的例子,有人认为好的分数能激励学生更努力地学习。但学生们想出这样一个二难推论用来驳斥上述理论:

\begin{quote}
如果学生喜爱学习,那么就不需要激励。如果学生厌烦学习,那么激励也没有用。学生或者是喜爱学习的或者是厌烦学习,所以,激励是不需要的或者没用的。
\end{quote}

该论证形式是有效的,但我们能用绕过死角法来反驳这个论证。其析取前提是假的,因为学生会有不同的学习态度:有的喜爱,有的厌烦,还有许多人不同于前两者。对于后面这些人来说,激励既是需要的也是可以发挥作用的。这种方法并不是证明结论为假,只是表明推论本身并没有给结论提供充足的理由。

\subsubsection{直击一角法}

如果析取前提穷尽了所有可能性,是不可驳倒的,就不能用上述方法了。必须有另外的方法来避开结论,其中之一就是\textbf{直击一角法},即拒斥两个假言前提中的一个。要否定两假言前提的组合,我们只需否定其中的一个即可。直击一角,就是要试图表明条件前提至少一假。刚才驳斥学校分级打分的例子,所依据的条件前提之一是"如果学生喜爱学习,就不需要激励",反驳者可以争辩说,即使一个学生喜爱学习,也需要激励,好分数会带来额外的奖励,甚至能激励最勤奋的学生更认真地学习。这样一来,就很可能得到好的回应——原来的二难的一角就被击破了。

\subsubsection{构造反二难法}

\textbf{构造反二难法}是最巧妙的方法,但并不总能令人信服,我们来看这是为什么。用这种方法驳斥给定的二难推论,需要构造另一个二难推论,它的结论与原来的结论相反。辩驳中可以使用任何一个二难推论,但最理想的反二难推论应当与原来的推论有相同的组成成分(直言命题)。

有个古老的例子能说明这种方法,相传雅典有一位母亲劝儿子不要从政时说道:

\begin{quote}
如果你主持公道,人们就会仇视你。如果你不主持公道,神灵们就会仇视你。你必定或者主持公道或者不主持公道,所以无论如何都会被仇视。
\end{quote}

他的儿子反驳说:

\begin{quote}
如果我主持公道,神灵们就会施爱于我。如果我不主持公道,人们就会施爱于我。我必定或者主持公道或者不主持公道,所以我都会被爱。
\end{quote}

在把二难推论作为强力工具的日常论辩中一般人的争论中,这种驳斥方法,从几乎相同的前提得到相反的结论,是种很不错的修辞手法。但如果更细致地研究,就会发现它们的结论并不像初看上去那样对立。

第一个二难推论的结论是儿子会被仇视(被人们或者被神灵们),而反二难的结论是儿子会被爱(被神灵们或被人们)。实际上两者完全是相容的。反驳用的反二难仅仅是建构了一个结论不同的论证而已。两个结论可能都是真的,因而这里并没有达成真正的反驳。但在唇枪舌剑的辩论中,并不需要细致分析,如果在公共争辩中出现这样的反驳,听众大多会把它当做对原论证的毁灭性攻击。

如此反驳并不能驳倒推理,而只是将注意力引向同一事情的不同方面,这从如下的二难推论可能看得更清楚。"乐观主义者"认为:

\begin{quote}
如果我工作,就能挣钱,如果赋闲在家,那么我乐得自在。我或者工作或者不工作,总之,我能挣钱或者乐得自在。
\end{quote}

而悲观主义者却会给出这样一个反二难:

\begin{quote}
如果我工作,就不能乐得自在,如果赋闲在家,就不能挣钱。或者工作或者不工作,总之,我或者不能乐得自在或者不能挣钱。
\end{quote}

这些结论只能说明看问题的视角不同,并非对事实状况的意见不一致。

\subsection{普罗塔哥拉与欧提勒士的二难困境}

通常讲二难推论,都要说到普罗塔哥拉(Protagoras)和欧提勒士(Euathlus)之间著名的讼案。普罗塔哥拉是生活在公元前5世纪的希腊的一名教师,他开设了很多课程,其中最著名的是法庭辩护术,欧提勒士想跟他学习当一名律师,但他负担不起学费。于是两人定了一个契约,普罗塔哥拉先不收学费,等欧提勒士学成并在第一场官司中获胜时,再交学费。可是,欧提勒士学成之后,迟迟没有在法庭上进行辩护,普罗塔哥拉等得不耐烦了,于是把他的学生告上法庭,要求收回学费。欧提勒士忘记了"律师为自己的案子辩护乃属愚行"的格言,决定为自己进行辩护。审理开始后,普罗塔哥拉就用一个压倒性二难推论陈述己方要求:

\begin{quote}
如果欧提勒士打输了官司,那么他必须还我学费(根据法庭的判决),如果欧提勒士打赢了官司,那么他也必须还我学费(根据我们之间的契约),或者他打输或者打赢官司,都必须还我学费。
\end{quote}

情况看来对欧提勒士而言十分不利,但他已把修辞术学得很好,于是他向法庭提出了如下相反的二难推论:

\begin{quote}
如果我打赢了官司,我不必交学费(根据法庭的判决),如果我打输了官司,我也不必交学费(根据我们之间的契约),或者我打赢或者打输,都不必交学费。
\end{quote}

如果你是法官,该如何判决呢?

注意欧提勒士的反二难的结论与普罗塔哥拉的结论的确不相容,一个确实是另一个的否定。这种相反二难推论与原来的二难推论的互相拒斥的情况并不多见。在这样的情况下,前提就是不相容的,两个二难推论可用于澄清其中蕴涵的矛盾。

\begin{center}
\fbox{\parbox{0.95\textwidth}{
\textbf{本节要点}
\begin{itemize}
\item \textbf{二难推论}的基本特征:
  \begin{itemize}
  \item 包含析取前提和两个条件前提
  \item 使对手面临两种选择,两者都导致不利的结论
  \item 可分为复杂式(结论是析取命题)和简单式(结论是直言命题)
  \end{itemize}
\item 避开或驳斥二难推论的三种方法:
  \begin{itemize}
  \item \textbf{绕过死角法}:拒斥析取前提,指出存在第三种可能性
  \item \textbf{直击一角法}:否定两个假言前提中的至少一个
  \item \textbf{构造反二难法}:构造另一个结论相反的二难推论
  \end{itemize}
\item 构造反二难法的局限性:
  \begin{itemize}
  \item 往往只是转移注意力而非真正反驳
  \item 原论证和反驳可能结论都为真,只是视角不同
  \end{itemize}
\item 普罗塔哥拉与欧提勒士案例展示了真正相互矛盾的二难推论
\end{itemize}
}}
\end{center}
\section{第7章概要}
本章考察日常语言中的三段论论证。我们看到标准式三段论的理论如何应用于这些论证。

7.1 节指出,日常语言论证很少以标准形式出现。要把它们翻译为标准形式,需要理解它们的含义。

7.2 节解释如何把一个表面上有三至六个词项的论证归约为只有三个词项的标准式三段论。这需要(1)去除同义词,以及(2)对某些词项换质以处理其补类。

7.3 节提出九种有用的方法,用以处理那些构成命题不是标准式的三段论论证。

1.单称命题,如"苏格拉底是哲学家",可当做全称命题(A 或 E)对待。

2.如果命题的谓项是形容词或形容词短语,可把它们替换为指称相应类的词项。

3.如果命题的主要动词不是标准联项"是"或"不是",可把动词及其他词语(主项与量项之外)看做类的定义特征,从而把原命题改写为标准式。

4.如果命题的各成分虽已出现但顺序不标准,找出主项,重新调整各成分的顺序。

5.处理非标准量词时,通常要把它们替换为"所有"、"没有"或"有"。要把"并非每个……"翻译为"有……不是……"。

6.排斥命题,如"只有公民是选民",一般要通过颠倒主、谓项位置翻译为 A 命题。结论通常是"所有选民是公民"。

7.不含量词的命题,要依据语境把量词"所有"或"有"补充完整。

8.有些命题的表达形式完全不像标准式直言命题,但其逻辑上等价的直言命题可以明确地表述出来。

9.除外命题,如"除了雇员之外所有人都合格",不是简单的直言命题,而要翻译为两个标准式直言命题的合取。

7.4 节说明并举例解释了协同翻译的方法,即有时需要把同一个辅助词项(参项)引入三段论的三个构成命题中,以便把整个论证翻译为标准形式。

7.5 节考察省略三段论,即前提或结论未明确表述出来的三段论。我们看到,在对省略三段论进行检验之前,如何发现并明确表述出未出现的命题。

7.6 节解释并举例说明连锁三段论。其中有些包含三个以上前提,有些则包含一系列相互关联的三段论。

7.7 节解释了析取三段论和假言三段论,指出了它们的有效形式和可能产生的谬误。

7.8 节讨论二难推论的各种形式,并说明了反驳二难推论的三种方法:抓住联言前提的虚假性、抓住析取前提的虚假性以及构造反二难推论。 

% 第八章
\chapter{现代符号逻辑}
\section{现代逻辑的符号语言}

\begin{logicbox}[title=引言]
本节介绍\logicterm{现代符号逻辑}的基本概念和优势。与古典逻辑不同,现代逻辑使用专门的符号语言来避免自然语言的缺陷,能够更精确地表述论证,并简化复杂的推理过程。我们将学习为什么\logicterm{符号逻辑}是分析演绎论证的强大工具。
\end{logicbox}

我们一直在寻求对演绎论证进行分析和评估的技术。\logicterm{演绎理论}旨在提供这样的技术,它已经发展出两个不同的分支来做这件工作:此前三章所考察的是\logicterm{经典逻辑}或\logicterm{亚里士多德型逻辑},本章和下两章的主题则是\logicterm{现代符号逻辑}。

\subsection{自然语言在逻辑分析中的局限性}

论证的分析和评估经常因其表述语言的特性(如英语或任何其他自然语言的特性)而非常困难。自然语言在逻辑分析中存在以下根本性问题:

\begin{theorembox}[title=自然语言的四大逻辑缺陷]
\textbf{1. 语义模糊性}:自然语言中的词汇往往具有多重含义或边界不清的概念。例如,"高"这个词在不同语境中有不同标准,"民主"一词在政治学、社会学中有不同内涵。

\textbf{2. 句法歧义性}:同一句子可能有多种语法分析,导致不同的逻辑结构。经典例子如"飞行的飞机的驾驶员"可以理解为"(飞行的飞机)的驾驶员"或"飞行的(飞机的驾驶员)"。

\textbf{3. 语用复杂性}:比喻、隐喻、反讽等修辞手法虽然增强了表达力,但在逻辑分析中会引起混淆。例如,"他是个狮子"在文学中是勇敢的比喻,在逻辑中却是明显的假陈述。

\textbf{4. 情感干扰}:自然语言常常承载情感色彩和价值判断,这些因素会干扰纯粹的逻辑分析。诉诸情感的论证虽然在修辞上有效,但在逻辑上可能是无效的。
\end{theorembox}

这些问题在第一部分已经探讨过了,它们构成了发展\logicterm{人工符号语言}的根本动机。要避免这些困难就要直接进入论证的逻辑核心,为此逻辑学家们构造了一种能避免自然语言缺陷的\logicterm{人工符号语言}。使用这种符号语言能\logicemph{精确地}表述论证,消除歧义,并揭示论证的真正逻辑结构。

\begin{examplebox}[title=符号语言的优势]
符号也能便利我们对论证的思考。"由于符号系统之助,"一位杰出的现代逻辑学家写道,"我们几乎用眼睛就可以机械地进行推理转换,否则,这种转换本来要求大脑有很高的智能。"\cite{quine1940} 这似乎有点悖谬,但符号语言确实可以帮助我们不需大伤脑筋就能完成某些智力活动。
\end{examplebox}

\subsection{符号逻辑的历史发展}

古代的和古典的逻辑学家们也承认某种特殊逻辑记号的价值。亚里士多德在自己的分析中就使用了变项,而如前面几章所表明,改进了的\logicterm{亚里士多德型逻辑}也以很复杂的方式使用了符号。\cite{aristotle-logic} 然而,真正的符号逻辑革命发生在19世纪末和20世纪初。

\begin{examplebox}[title=符号逻辑发展的里程碑]
\textbf{布尔代数时期(1854)}:乔治·布尔(George Boole)在《思维法则研究》中首次将逻辑运算代数化,建立了布尔代数,为现代符号逻辑奠定了基础。

\textbf{弗雷格的概念文字(1879)}:戈特洛布·弗雷格(Gottlob Frege)在《概念文字》中创建了第一个完整的形式逻辑系统,引入了量词和函数概念。

\textbf{罗素-怀特海德体系(1910-1913)}:《数学原理》建立了基于逻辑的数学基础,展示了符号逻辑的强大表达能力。

\textbf{现代发展(20世纪)}:塔尔斯基的语义理论、哥德尔的完备性和不完备性定理等进一步完善了符号逻辑体系。
\end{examplebox}

\subsection{现代逻辑与古典逻辑的根本差异}

\begin{theorembox}[title=现代逻辑的核心特征]
在\logicterm{现代逻辑}中,发生了以下根本性转变:

\textbf{1. 焦点转移}:从\logicterm{三段论}转向\logicterm{逻辑联结词}。现代逻辑认识到,逻辑联结词是所有演绎论证的基础构件,不管论证是否采用三段论形式。

\textbf{2. 结构分析}:现代逻辑关注命题和论证的\logicemph{内在逻辑结构},而不仅仅是表面的语言形式。这种结构分析能够揭示论证的深层逻辑关系。

\textbf{3. 普遍适用性}:现代符号逻辑不局限于特定的论证形式,而是提供了分析任何演绎论证的通用方法。

\textbf{4. 机械化程序}:符号逻辑允许开发机械化的决定程序,使逻辑分析更加客观和可靠。
\end{theorembox}

\logicterm{现代符号逻辑}不受演绎论证要转换成\logicterm{三段论}形式的制约(\logicterm{亚里士多德型逻辑}受这种制约)。正如我们在第 7 章所见,将复杂的日常语言论证转换为标准三段论形式是极其费力的工作,而且往往会丢失重要的逻辑信息。现代逻辑通过直接分析论证的逻辑结构,避免了这种强制性转换,使我们可以更直接、更准确地追求演绎分析的目标。

\subsection{符号逻辑的方法论优势}

现代逻辑的符号记法是分析论证的特别有力的工具,具有以下方法论优势:

\begin{examplebox}[title=符号逻辑的五大优势]
\textbf{1. 精确性}:符号语言消除了自然语言的模糊性和歧义性,使逻辑关系变得明确无误。

\textbf{2. 简洁性}:复杂的逻辑关系可以用简洁的符号公式表达,便于操作和分析。

\textbf{3. 普遍性}:符号系统不依赖于特定的自然语言,具有跨语言的普遍适用性。

\textbf{4. 可操作性}:符号公式可以进行机械化的变换和计算,支持算法化的逻辑分析。

\textbf{5. 可扩展性}:符号系统可以根据需要扩展,适应更复杂的逻辑问题。
\end{examplebox}

使用这种记法我们可以更全面地达到演绎逻辑的核心目标:\logicemph{精确地}区分\logicemph{有效}论证和\logicwarn{无效}论证,并且能够\logicemph{系统地}分析论证的逻辑结构,发现其中的逻辑错误或逻辑漏洞。

\begin{center}
\fbox{\parbox{0.95\textwidth}{
\textbf{本节要点}
\begin{itemize}
\item \textbf{演绎理论的两大分支}:
  \begin{itemize}
  \item \logicterm{经典逻辑}(亚里士多德型逻辑)
  \item \logicterm{现代符号逻辑}(本章主题)
  \end{itemize}
\item \textbf{自然语言的四大逻辑缺陷}:
  \begin{itemize}
  \item 语义模糊性:词汇多义性和概念边界不清
  \item 句法歧义性:同一句子的多种语法分析
  \item 语用复杂性:比喻、隐喻等修辞手法的干扰
  \item 情感干扰:情感色彩和价值判断的影响
  \end{itemize}
\item \textbf{符号逻辑的历史发展}:
  \begin{itemize}
  \item 布尔代数(1854):逻辑运算代数化
  \item 弗雷格概念文字(1879):完整形式逻辑系统
  \item 罗素-怀特海德《数学原理》(1910-1913)
  \item 20世纪的进一步发展和完善
  \end{itemize}
\item \textbf{现代逻辑的四大核心特征}:
  \begin{itemize}
  \item 焦点转移:从\logicterm{三段论}转向\logicterm{逻辑联结词}
  \item 结构分析:关注命题和论证的\logicemph{内在逻辑结构}
  \item 普遍适用性:提供分析任何演绎论证的通用方法
  \item 机械化程序:支持客观可靠的逻辑分析
  \end{itemize}
\item \textbf{符号逻辑的五大方法论优势}:
  \begin{itemize}
  \item 精确性:消除模糊性和歧义性
  \item 简洁性:用简洁符号表达复杂逻辑关系
  \item 普遍性:跨语言的普遍适用性
  \item 可操作性:支持机械化变换和计算
  \item 可扩展性:适应更复杂的逻辑问题
  \end{itemize}
\item \textbf{核心目标}:\logicemph{精确地}区分\logicemph{有效}论证和\logicwarn{无效}论证,\logicemph{系统地}分析论证的逻辑结构
\end{itemize}
}}
\end{center}
\input{chapter8/8-2 合取、否定和析取符号.tex}
\section{条件陈述与实质蕴涵}

\begin{logicbox}[title=引言]
本节讨论条件陈述的本质及其在逻辑中的符号表示。我们将分析"如果-那么"关系的多种含义、实质蕴涵的概念,以及条件陈述与必要条件、充分条件之间的关系。通过理解这些概念,我们能够更准确地将自然语言中的条件语句转换为形式逻辑表达式。
\end{logicbox}

当把语词"如果"放在第一个陈述之前,把语词"那么"放在第一个和第二个陈述之间来结合两个陈述时,如此构成的复合陈述就是一个\textbf{条件陈述}(也叫"假言陈述"、"蕴涵"或"蕴涵陈述")。在一个条件陈述中,跟在"如果"后面的分支陈述叫\textbf{前件}(或"蕴涵者",偶尔也叫"前式"),跟在"那么"后面的分支陈述叫\textbf{后件}(或"被蕴涵者",偶尔也叫"后式")。例如,"如果琼斯先生是那个司闸员的邻居,那么琼斯先生挣的钱是那个司闸员的三倍"是一个条件陈述,其中,"琼斯先生是那个司闸员的邻居"是前件,"琼斯先生挣的钱是那个司闸员的三倍"是后件。

一个条件陈述断言在其前件为真的任何情形下,它的后件也是真的。它并不断言其前件为真,而只是断言如果其前件为真,其后件也为真。它也并不断言其后件为真,而仅仅断言它的后件会为真,如果前件为真的

话。一个条件陈述的基本含义,是断言其前后件之间的某种关系以特定次序成立。要理解一个条件陈述的含义,我们必须理解何为蕴涵关系。\\
"蕴涵"一词不止一个含义。我们已经看到,在引进一个特殊的逻辑符号来表示日常语词"或者"的某个单一含义之前,区分它的不同含义是有用的。要是我们不这样做,日常语言的含混性就会影响我们的逻辑符号系统,妨碍我们达到所欲获得的明晰性和精确性。在我们把一个特殊的逻辑符号引入这种联系中之前,区分"蕴涵"或"如果一那么"的不同含义亦同样有用。

考查下面的四个条件陈述,它们每个都断言一种不同类型的蕴涵,都对应于一种不同含义的"如果…那么":

A.如果所有人都有死且苏格拉底是人,那么苏格拉底有死。

B.如果莱士里是单身汉,那么莱士里是未婚的。\\
C.如果把这张蓝色的石蕊纸放在酸液中,那么这张蓝色的石蕊纸会变红。

D.如果斯塔德輸掉了这场比赛,那么我就吞下我的帽子。

即使随意地观察一下这四个条件陈述也会发现,它们具有非常不同的类型。 A 的后件乃由它的前件逻辑地推出,而 B 的后件是根据其前件中的术语"单身汉"的定义而得来,而"单身汉"的定义就是未婚男人。C 的后件不是仅根据逻辑或其词项的定义从其前件推出,这种联系必须经验地发现,因为这里所陈述的蕴涵是因果关系。最后,D的后件既不是根据逻辑或定义从前件推得,也没有涉及因果性定律——就这个词的通常意义来说。大多数因果性定律,臂如物理学和化学中发现的那些定律,描述的是世界发生了什么,而不管人的希望或欲求如何。当然,没有这样一种定律和陈述 D 相联系。这个陈述表述的是说话者在某种特定的情形下以特定的方式行事的决策。

可见,这四个条件陈述的不同之处,就在于每个断言了其前件和后件之间的一种不同类型的蕴涵关系。但它们并非完全不同,它们所断言的都是蕴涵的类型。那么,它们是否存在任何可识别的共同含义,即是否存在尽管可能不是其中任何一个的完整含义,但是这些公认的不同种类蕴涵所

共有的部分含义呢?\\
关于探求共同的部分含义的重要性,我们可以回想一下对日常语词 "或"进行符号刻画的过程。那时我们是如下进行的。首先,在对比相容和不相容析取时,我们强调"或"的两种含义之间的区别。我们注意到,两个陈述的相容析取的意思是说,它们当中至少一个为真。不相容析取的意思是说,它们当中至少一个为真,但不是两者都为真。其次,我们注意到这两种类型的析取有一个共同的部分含义。这个部分的共同含义,即至少有一个析取支为真,被看做是弱的、相容的"或"的整个含义,是强的、相容的"或"的含义的一部分。然后,我们引入特殊符号"V"来表达这个共同的部分含义(它是"或"的弱意义上的整个含义)。最后,我们注意到,表达共同的部分含义的符号刻画也是对语词"或"在下述意义上的合适翻译,即可以把析取三段论作为一个有效的论证形式保留下来。我们承认把不相容的"或"翻译成符号"V",忽略和丢掉了它的部分含义。但由这种翻译所保留的那个部分含义,是析取三段论继续成为一个有效论证必需的全部东西。既然析取三段论是我们这里所关注的涉及析取的典型论证,那么,语词"或"的这种部分翻译——在某些情形,可以从它的"完全的"或"全部的"含义中抽取出来——对我们目前的目的是完全合适的。

现在,我们希望以同样的方式抽取日常语言辞组"如果一那么"的含义。第一步已经完成:我们已经强调了短语"如果一那么"对应于四种不同蕴涵的四种意义之间的区别。现在准备做第二步,即发现一个至少是所有这四种不同类型的蕴涵的含义的一部分的那种意义。

要解决这个问题,可先看什么情形足以确立一个给定条件陈述的假。在什么情形下,我们会同意下面的条件陈述为假呢?

\begin{quote}
如果把这张蓝色的石蘂纸放进那种溶液中,那么这张蓝色的石蕊纸会变红。
\end{quote}

这个条件陈述并未断言任何一张蓝色的石蕊纸实际上被放进了这种溶液中,或任何一张蓝色的石蕊纸实际上变红了,认识到这一点是很重要的。它仅仅断言如果把这张蓝色的石総纸放进那种溶液中,那么这张蓝色的石蕊纸会变红。如果这张蓝色的石䓗纸实际上被放进这种溶液中,并且它没

变红,就证明该陈述是假的。可以说,当一个条件陈述的前件为真时,就获得一个关于该条件陈述的虚假性的严峻检验,因为如果它的后件为假且前件为真,该条件陈述本身就被证明为假。

对任一条件陈述"如果 $p$ 那么 $q"$来说,如果已知合取 $p \cdot \sim q$ 为真,也就是说,如果它的前件为真且后件为假,则可知该条件陈述为假。而若一个条件陈述为真,则上面所示合取式必定为假,也就是说,它的否定 $315 \sim(p \cdot \sim q)$ 必定为真。换句话说,对任何为真的条件陈述"如果 $p$ 那么 $q"$而言,它的前件和后件的否定的合取的否定,即 $\sim(p \cdot \sim q)$ ,必定也为真。据此,我们可把~$(p \cdot \sim q)$ 当做"如果 $p$ 那么 $q"$的含义的一部分。

每个条件陈述都意谓否定其前件为真且后件为假,但这不必是其整个含义。前面的 A 那样的条件陈述还断言了其前件和后件之间的一种逻辑联系,B 那样的条件陈述还断言了一种定义性联系,C 那样的条件陈述还断言了一种因果性联系,而 D 那样的条件陈述则还断言了一种决策性联系。但不管一个条件陈述断言的是何种蕴涵,它的一部分含义是对其前件和后件的否定的合取的否定。

现在,我们引进一个特殊的符号来表达短语"如果一那么"的这种共同的部分含义。通过以 $p \supset q$ 缩写 $\sim(p \cdot \sim q)$ ,我们来定义新符号"つ" (叫"马蹄号")。符号"つ"的确切含义可以用真值表方法揭示如下:

\begin{center}
\begin{tabular}{|l|l|l|l|l|l|}
\hline
$p$ & $q$ & $\sim q$ & $p^{\cdot \sim q}$ & $\sim(p \cdot \sim q)$ & $p$ つ $q$ \\
\hline
T & T & F & F & T & T \\
\hline
T & F & T & T & F & F \\
\hline
F & T & F & F & T & T \\
\hline
F & F & T & F & T & T \\
\hline
\end{tabular}
\end{center}

其中,前两列是导引列,它们只是列出 $p$ 和 $q$ 真值组合的所有可能情形。第三列据第二列得来,第四列据第一和第三列得来,第五列据第四列得来,根据定义,第六列与第五列真值相同。

符号"コ"不应被看成是指谓"如果一那么"的某种含义,或代表 (上列蕴涵类型中的)某种蕴涵关系。那是不可能的,因为没有单一的 "如果一那么"的含义,而是有几个含义。不存在该符号所刻画的单一蕴涵关系,而是有几种不同的蕴涵关系。故符号"つ"不应被看成是代表 "如果一那么"的所有含义。这些含义各不相同,用单个逻辑符号来缩写

所有这些含义的任何企图都会使符号变得含混,正如日常语言辞组"如果一那么"或"蕴涵"一样含混。符号"つ"是完全不含混的。 $p \supset q$ 缩写的就是 $\sim(p \cdot \sim q)$ ,它的含义包含在被探讨的各种蕴涵的含义之中,但它并不构成它们中任何一个的完整含义。

既然读 $p \supset q$ 的一种方便方式是"如果 $p$ 那么 $q"$,我们也可以把符号 "$\supset$"看成表示了另一种蕴涵,而且这样做是很有好处的。但它不是与前面提到过的任何一种蕴涵相同的蕴涵,它被逻辑学家叫做\logicterm{实质蕴涵}。

\subsection{实质蕴涵的哲学意义}

给出"实质蕴涵"这个特殊的名称,就是承认它是一个特殊概念,不应该把它和其他更常见类型的蕴涵相混淆。这一概念的引入在逻辑学史上具有重要的哲学意义:

\begin{theorembox}[title=实质蕴涵的本质特征]
\textbf{1. 真值函项性}:实质蕴涵是纯粹的真值函项关系,它只依赖于前件和后件的真值,而不依赖于它们之间的任何"实在关联"。

\textbf{2. 最小性}:实质蕴涵表达了所有类型蕴涵关系的最小共同核心——即"不存在真前件假后件的情况"。

\textbf{3. 形式性}:它抽象掉了逻辑、定义、因果、决策等具体的蕴涵机制,只保留了形式的真值关系。

\textbf{4. 普遍性}:正因为它的抽象性,实质蕴涵可以统一处理各种不同类型的条件陈述。
\end{theorembox}

这种抽象化的好处在于,它使我们能够开发出适用于所有条件陈述的统一逻辑方法,而不必为每种特定类型的蕴涵关系开发专门的逻辑系统。

日常语言中的所有条件陈述并非都必须断言前面所讨论的四种蕴涵之一。实质蕴涵实际上也是日常话语中所断言的第五种蕴涵。考虑这样一个评论:"如果希特勒是军事天才,那么我是猴子的叔叔"。很显然,它不是断言逻辑的、定义性的或因果性的蕴涵。它也不表达决策性蕴涵,因为说话者并没有能力使后件为真。这里的前后件之间没有"真正的联系",不管是逻辑的、定义性的还是因果性的。这种条件陈述经常被当做一种强调或幽默的方法来使用,它否定的是其前件,其后件通常是一个滑稽的、显然为假的陈述。既然没有任何为真的条件陈述有这样的真前件和假后件,那么,肯定这样一个条件陈述就意味着否定它的前件为真。上述条件陈述的完整含义就是,只要"我是猴子的叔叔"为假,即可否定"希特勒是军事天才"为真。既然前者明显为假,该条件陈述必被理解为否定后者。

这里的关键在于,实质蕴涵没有表明前后件之间的"实在关联",实际上,它所断言的仅仅是并非后件为假时前件为真。请注意:实质蕴涵符号像合取和析取符号一样,是真值函项联结词。它可用真值表定义如下:

\begin{center}
\begin{tabular}{|ccc|}
\hline
$p$ & $q$ & $p \supset q$ \\
\hline
T & T & T \\
T & F & F \\
F & T & T \\
F & F & T \\
\hline
\end{tabular}
\end{center}

正如这个真值表定义所表明,马蹄符"つ"有几个乍看起来很奇怪的特征:假前件实质蕴涵真后件的断言是真的;假前件实质蕴涵假后件的断言也是真的。这种表面的怪异可以由下面的探讨得到部分驱散。因为数 2 比数 4 小 (用符号表示为 $2<4$ ),可以推出任何小于 2 的数都小于 4 。条件公式:

如果 $x<2$ 那么 $x<4$\\
对任一 $x$ 都是真的。我们来看数 $1 、 3$ 和 4 ,依次以它们中的每一个代人前述条件公式的数字变项 $x$ ,可以观察到如下结果:

如果 $1<2$ 那么 $1<4$\\
在这种情形下,前后件都是真的,该条件陈述当然也是真的。\\
如果 $3<2$ 那么 $3<4$\\
在这种情形下,前件为假且后件为真,该条件陈述当然也是真的。\\
如果 $4<2$ 那么 $4<4$\\
在这种情形下,前件和后件都是假的,但该条件陈述仍然是真的。这三种情形分别对应于马蹄符"コ"的真值表定义中的第一、第三和第四行。可见,在一个条件陈述的前后件皆为真、前件为假且后件为真或前后件皆为假时,该条件陈述应该为真,这一点并不特别令人奇怪或惊讶。当然,没有小于 2 且不小于 4 的数,也就是说,没有其前件为真且后件为假的真条件陈述。这恰好是"つ"的真值表定义所表明的。

现在,我们打算把词组"如果——那么"的任何一次出现翻译成逻辑符号 "つ"。这种处理方式的意思是说,在把条件陈述翻译成符号时,我们把它们都只看做是实质蕴涵。当然,大多数条件陈述断言,在前后件之间不只实质蕴涵成立。因此,这种处理方式即意味着在把一个条件陈述翻译成符号语言时,应该忽略、撇开或"抽掉"它的部分含义。怎样辩护这种处理方式呢?

前面对用符号"V"来翻译相容和不相容析取这个处理方式的辩护是基于这样的理由:即使忽略附着在不相容析取"或"之上的附加含义,析取三段论的有效性也得到了保留。我们现在提议用符号"$\supset$"把所有的条件陈述仅翻译成实质蕴涵,可用完全同样的方式得到辩护。许多论证包含各种不同类型的条件陈述,但是,即便忽略这些论证的条件陈述的附加含义,我们所关注的一般类型的有效论证的有效性也都得到了保留。当然,这一点还需要证明,这是本章下一节的主题。

条件陈述可用多种不同方式表述。如下陈述:

如果他有一个好律师,那么他会被宣判无罪。

可以不用"那么"而被同样适当地表述为:

如果他有一个好律师,他会被宣判无罪。

前件和后件的表述次序可以颠倒,此时"如果"仍应在前件之前:

他会被宣判无罪,如果他有一个好律师的话。

显然,在上面所给的任何一个例子中,语词"如果"可被诸如"一旦"、318 "假如"、"倘若"或"在……条件下"等短语代替,而含义没有任何改变。经措辞调整还可把上述条件陈述表述为:

他有一个好律师蕴涵他会被宣判无罪。

或

他有一个好律师涵衍(entail)他会被宣判无罪。

从主动语态到被动语态的转换伴随着前后件次序的颠倒,可得其逻辑等价表述:

他会被宣判无罪被他有一个好律师所蕴涵(或涵衍)。

上列表述均可符号化为 $L \supset A$ 。\\
必要条件和充分条件的观念提供了条件陈述的其他一些表述形式。对任何一个特定事件来说,它的出现需要有许多必要情境。例如,一辆正常的轿车要能行使,油箱里有油,火花塞被校准,油泵能运转等都是必要条件。因此,如果该事件出现,它的出现所必需的每个条件必定都已经得到满足。据此,下述陈述:

油箱里有油是轿车行驶的一个必要条件。

可以同样适当地表述为:

轿车行使仅当它的油箱里有油。

它是如下说法的另一方式:

如果轿车行使,那么它的油箱里有油。

这些表述形式中的任何一个都可以符号化为 $R \supset F$ 。一般地说,"$q$ 是 $p$ 的必要条件"和"$p$ 仅当 $q$"可以符号化为 $p \supset q$ 。

对某特定情形而言,会有许多备选条件,它们中的任何一个都足以产生该情形。例如,就一个钱包里不止一美元来说,它里面有 101 便士、21个五分镍币、 11 个一角的硬币、 5 个两角五分钱等都是充分条件。如果获得其中的任何一个条件,那个特定的情形就会实现。因此,说"那个钱包里有 5 个两角五分钱是它里面超过一美元的充分条件",与说"如果那个钱包里有 5 个两角五分钱,那么它里面超过一美元"是一样的。一般的, "$p$ 是 $q$ 的充分条件"被符号化为 $p \supset q$ 。

\subsection{必要条件与充分条件的深入分析}

必要条件和充分条件的概念在逻辑学、科学和日常推理中都具有根本性的重要意义。让我们深入分析这些概念之间的复杂关系:

\begin{theorembox}[title=必要条件与充分条件的逻辑关系]
\textbf{1. 基本关系}:
\begin{itemize}
\item 如果 $p$ 是 $q$ 的一个充分条件,我们就有 $p \supset q$ ,并且 $q$ 必定是 $p$ 的一个必要条件。
\item 如果 $p$ 是 $q$ 的一个必要条件,我们就有 $q \supset p$ ,并且 $q$ 必定是 $p$ 的一个充分条件。
\end{itemize}

\textbf{2. 对称性}:如果 $p$ 是 $q$ 的必要且充分条件,那么,$q$ 是 $p$的充分且必要条件。这种对称性表明了双条件关系的本质。

\textbf{3. 传递性}:如果 $p$ 是 $q$ 的充分条件,$q$ 是 $r$ 的充分条件,那么 $p$ 是 $r$ 的充分条件。
\end{theorembox}

\begin{examplebox}[title=必要条件与充分条件的实际应用]
\textbf{科学中的应用}:
\begin{itemize}
\item 在医学中:病毒感染是感冒的充分条件(但不是必要条件,因为细菌感染也可能导致感冒)
\item 在物理学中:温度达到100°C是水在标准大气压下沸腾的充分条件
\end{itemize}

\textbf{法律中的应用}:
\begin{itemize}
\item 年满18岁是获得选举权的必要条件(但不是充分条件,还需要公民身份等)
\item 故意杀人是构成谋杀罪的必要条件
\end{itemize}

\textbf{数学中的应用}:
\begin{itemize}
\item 一个数是偶数是它能被2整除的必要且充分条件
\item 三角形三边相等是它为等边三角形的必要且充分条件
\end{itemize}
\end{examplebox}

\logicwarn{常见误解}:人们经常混淆必要条件和充分条件。例如,"教育是成功的关键"这种说法往往被误解为教育是成功的充分条件,但实际上教育通常只是成功的必要条件之一。理解这种区别对于准确的逻辑推理至关重要。

并非每个含有"如果"(或类似语词)的陈述都是条件陈述。下列陈述中没有一个是条件陈述:"冰箱里有食品,如果你想吃","您的桌子准备好了,如果您乐意的话","假如感兴趣,有个消息给你","即便没得到允许,会议也会举行"。特定语词的出现与否决不是决定性的。在每种情形下,必须先理解给定语句的含义,然后用符号公式重新表述这种含义。

语词"如果"和"不确定的"之间没有必然的或逻辑的联系,尽管经常有这样一种说法:跟在语词"如果"后面的东西有点不确定。这一点可由下面的逸事所例示:

有一次,乔治•伯纳德•肖给温斯顿•丘吉尔送了两张他的新剧的首演式的票,附言"带一个朋友——如果你有的话"。对此,丘吉尔回复说他忙于出席别的首演式,但他会很感激第二场演出的票,"如果有这样一张票的话"\cite{churchill1953}。

\begin{center}
\fbox{\parbox{0.95\textwidth}{
\textbf{本节要点}
\begin{itemize}
\item \textbf{条件陈述的基本结构}:
  \begin{itemize}
  \item 由\logicterm{前件}(蕴涵者)和\logicterm{后件}(被蕴涵者)组成
  \item 断言在前件为真的情况下,后件也为真
  \item 不断言前件或后件的实际真值,只断言它们之间的条件关系
  \end{itemize}
\item \textbf{"如果-那么"关系的五种类型}:
  \begin{itemize}
  \item \textbf{逻辑蕴涵}:后件通过逻辑规则从前件推出
  \item \textbf{定义性蕴涵}:后件通过前件中术语的定义而得出
  \item \textbf{因果性蕴涵}:后件通过因果关系从前件得出
  \item \textbf{决策性蕴涵}:表达说话者的决定或承诺
  \item \textbf{实质蕴涵}:日常话语中的第五种蕴涵,无"实在关联"
  \end{itemize}
\item \textbf{实质蕴涵}($\supset$)的深入分析:
  \begin{itemize}
  \item 定义:$p \supset q$ 即 $\sim(p \cdot \sim q)$
  \item 真值条件:仅当前件为真且后件为假时为假
  \item 是所有蕴涵类型的最小共同核心
  \item \textbf{四大本质特征}:真值函项性、最小性、形式性、普遍性
  \item 哲学意义:抽象掉具体蕴涵机制,统一处理各种条件陈述
  \end{itemize}
\item \textbf{实质蕴涵的真值表特征}:
  \begin{itemize}
  \item 前件假时,无论后件真假,实质蕴涵都为真
  \item 前后件都真时,实质蕴涵为真
  \item 前件真后件假时,实质蕴涵为假
  \item 这种定义保证了条件陈述的逻辑一致性
  \end{itemize}
\item \textbf{必要条件与充分条件的深入分析}:
  \begin{itemize}
  \item \textbf{基本符号化}:"$q$是$p$的必要条件"和"$p$仅当$q$"符号化为$p \supset q$
  \item \textbf{充分条件}:"$p$是$q$的充分条件"符号化为$p \supset q$
  \item \textbf{逻辑关系}:充分条件与必要条件的相互转换关系
  \item \textbf{对称性}:必要且充分条件的双向关系
  \item \textbf{传递性}:充分条件关系的传递性质
  \item \textbf{实际应用}:科学、法律、数学中的具体例子
  \item \logicwarn{常见误解}:混淆必要条件和充分条件的危险
  \end{itemize}
\item \textbf{条件陈述的多样表达形式}:
  \begin{itemize}
  \item 语词变化:"如果"、"一旦"、"假如"、"倘若"、"在……条件下"
  \item 语态转换:主动语态与被动语态的等价表述
  \item 语序变化:前件后件位置的灵活性
  \item 非条件陈述的识别:避免误将非条件语句当作条件陈述
  \end{itemize}
\end{itemize}
}}
\end{center}
\input{chapter8/8-4 论证形式与论证.tex}
\section{陈述形式与实质等值}

\begin{logicbox}[title=引言]
本节探讨陈述形式、重言式、矛盾式与实质等值的概念及其相互关系。通过理解这些概念,我们能够辨别不同类型的逻辑真理,将复杂的逻辑关系转化为更易于分析的形式,并利用真值函项联结词准确表达命题间的各种关系。
\end{logicbox}

\subsection{陈述形式的理论基础}

现在,我们来明确一下前一节所假定的一个概念,即\logicterm{陈述形式}。以论证和论证形式之间的关系为一方,陈述和陈述形式之间的关系为另一方,这两者是完全平行的。

\begin{theorembox}[title=陈述形式的数学定义]
一个\logicterm{陈述形式}是任何一个含有陈述变元但不含陈述的符号序列,若用陈述代入这些陈述变元——用同一个陈述始终一致地代入同一个陈述变元——其结果是一个陈述。

这个定义具有以下重要特征:
\begin{itemize}
\item \textbf{形式性}:陈述形式抽象掉了具体的语义内容,只保留逻辑结构
\item \textbf{生成性}:每个陈述形式可以生成无穷多个具体陈述
\item \textbf{分类性}:陈述形式为陈述提供了逻辑分类的标准
\item \textbf{可操作性}:陈述形式支持机械化的逻辑操作和分析
\end{itemize}
\end{theorembox}

例如,$p \vee q$ 是陈述形式,因为若用陈述代入变元 $p$和 $q$ ,就会产生一个陈述。由于所产生的陈述是一个析取句,$p \vee q$ 就叫做\logicterm{析取陈述形式}。同样,$p \cdot q$ 和 $p \supset q$ 分别叫做\logicterm{合取陈述形式}和\logicterm{条件陈述形式},$\sim p$ 叫做\logicterm{否定形式}或者\logicterm{否认形式}。

\begin{examplebox}[title=陈述形式的分类体系]
\textbf{基本陈述形式}:
\begin{itemize}
\item 原子形式:$p, q, r, \ldots$
\item 否定形式:$\sim p, \sim q, \ldots$
\end{itemize}

\textbf{复合陈述形式}:
\begin{itemize}
\item 合取形式:$p \cdot q, (p \cdot q) \cdot r, \ldots$
\item 析取形式:$p \vee q, (p \vee q) \vee r, \ldots$
\item 条件形式:$p \supset q, (p \supset q) \supset r, \ldots$
\item 双条件形式:$p \equiv q, (p \equiv q) \equiv r, \ldots$
\end{itemize}

\textbf{混合形式}:$p \cdot (q \vee r), (p \supset q) \vee (r \cdot s), \ldots$
\end{examplebox}

正像某种形式的论证称为该论证形式的代入例一样,具有某种形式的任一陈述称为该陈述形式的\logicterm{代入例}。正像我们判别一个给定论证的特征形式一样,我们把一个给定陈述的\logicterm{特征形式}判别为这样一种陈述形式:通过一致地用不同的简单陈述代入每个不同的陈述变元,就可以从其产生该给定陈述。例如,$p \vee q$就是陈述"那个盲囚戴红帽子或者那个盲囚戴白帽子"的特征形式。

\subsection{逻辑真理的三重分类:重言式、矛盾式与偶真陈述形式}

尽管陈述"林肯是被暗杀的"(记为 $L$ )和"林肯或者是被暗杀的,或者不是"(记为 $L \vee \sim L$ )都是真的,但我们会非常自然地感觉到,它们是"在不同方面"为真,或有"不同种类"的真。同样,尽管陈述"华盛顿是被暗杀的"(记为 $W$ )和"华盛顿既是被暗杀的又不是被暗杀的"(记为 $W \cdot \sim W)$ 这两者都为假,但我们也会非常自然地感觉到,它们也是 "在不同方面"为假,或有"不同种类"的假。

\begin{theorembox}[title=逻辑真理与经验真理的根本区别]
这种直觉上的区别反映了逻辑学中的一个根本性分类:

\textbf{经验真理}:其真值依赖于世界的实际状况,需要通过经验观察来确定。例如,"林肯是被暗杀的"是一个历史事实,其真值取决于历史事件的实际发生。

\textbf{逻辑真理}:其真值完全由逻辑形式决定,独立于世界的实际状况。例如,"林肯或者是被暗杀的,或者不是"无论在什么可能世界中都必然为真。

这种区别在哲学上具有深远意义:它揭示了人类知识的两种不同来源——经验观察和逻辑推理。
\end{theorembox}

陈述 $L$ 为真和陈述 $W$ 为假乃属于历史事实,它们没有逻辑必然性。所有事件都有以不同方式出现的可能,因而像 $L$ 和 $W$ 这样的陈述的真值,必须通过对历史的经验研究才能被发现。而陈述 $L \vee \sim L$ 尽管是真的,但它不是历史地真,而具有\logicemph{逻辑的必然性}:事件不可能如此这般以致使它为假,它的真可以独立于任何经验研究而被知晓。

\subsection{重言式的深入分析}

陈述 $L \vee \sim L$ 是一个\logicterm{逻辑真理},或曰\logicterm{形式真理},其真仅因其形式,它是一个其所有代入例都是真陈述的陈述形式的代入例。

\begin{theorembox}[title=重言式的本质特征]
一个只有真代入例的陈述形式叫\logicterm{重言的陈述形式},或\logicterm{重言式}(Tautology)。重言式具有以下本质特征:

\textbf{1. 逻辑必然性}:重言式在所有可能的情况下都为真,体现了逻辑的必然性。

\textbf{2. 形式性}:重言式的真值完全由其逻辑形式决定,与具体内容无关。

\textbf{3. 先验性}:重言式的真值可以通过纯粹的逻辑分析确定,无需经验观察。

\textbf{4. 信息空虚性}:重言式虽然必然为真,但不提供关于世界的实质信息。
\end{theorembox}

要表明陈述形式 $p \vee \sim p$ 是一个重言式,可构造如下真值表:

\begin{center}
\begin{tabular}{|ccc|}
\hline
$p$ & $\sim p$ & $p \vee \sim p$ \\
\hline
T & F & T \\
F & T & T \\
\hline
\end{tabular}
\end{center}

这个真值表只有一个初始栏或导引栏,因为被探究的形式只含有一个陈述变元。它只有两行,代表了所有可能的代入例。被检验陈述形式下面的那一栏里只有 T ,这表明,它的所有代入例都是真的。任何一个作为重言的陈述形式的代入例的陈述,依据其形式就是真的,其本身被称为\logicterm{重言陈述},亦称为一个\logicterm{重言式}。

\subsection{矛盾式的深入分析}

\begin{theorembox}[title=矛盾式的本质特征]
一个只有假代入例的陈述形式称为\logicterm{自相矛盾的陈述形式},或\logicterm{矛盾式}(Contradiction),它是逻辑地为假的。矛盾式具有以下本质特征:

\textbf{1. 逻辑不可能性}:矛盾式在所有可能的情况下都为假,体现了逻辑的不可能性。

\textbf{2. 形式矛盾性}:矛盾式的假值完全由其逻辑形式决定,与具体内容无关。

\textbf{3. 先验可知性}:矛盾式的假值可以通过纯粹的逻辑分析确定,无需经验检验。

\textbf{4. 爆炸原理}:从矛盾式可以推出任何陈述,这在逻辑学中被称为"爆炸原理"(Principle of Explosion)。
\end{theorembox}

陈述形式 $p \cdot \sim p$ 是自相矛盾的,因为在它的真值表中只有 F 在它下面出现,这表明它的所有代入例都是假的。任何一个作为自相矛盾的陈述形式的代入例的陈述,如 $W \cdot \sim W$ ,依据其形式就是假的,其本身称为\logicterm{自相矛盾的陈述},亦称为一个\logicterm{矛盾式}。

\subsection{偶真陈述形式的深入分析}

\begin{theorembox}[title=偶真陈述形式的特征]
其代入例既有真陈述又有假陈述的陈述形式,叫做\logicterm{偶真陈述形式}(Contingent Statement Form)。偶真陈述形式具有以下特征:

\textbf{1. 真值依赖性}:其真值依赖于具体的代入内容,而不仅仅是逻辑形式。

\textbf{2. 经验可检验性}:需要通过经验观察或事实调查来确定其真值。

\textbf{3. 信息承载性}:偶真陈述承载关于世界的实质信息,具有认知价值。

\textbf{4. 可能性空间}:偶真陈述在逻辑上既可能为真也可能为假,体现了现实世界的复杂性。
\end{theorembox}

其特征形式是偶真的陈述称为\logicterm{偶真陈述}。\cite{whitehead1911}

\begin{examplebox}[title=三类陈述形式的对比分析]
\textbf{重言式例子}:
\begin{itemize}
\item $p \vee \sim p$(排中律)
\item $p \supset p$(同一律)
\item $(p \cdot q) \supset p$(简化律)
\end{itemize}

\textbf{矛盾式例子}:
\begin{itemize}
\item $p \cdot \sim p$(矛盾律)
\item $(p \supset q) \cdot (p \cdot \sim q)$
\item $p \cdot (p \supset \sim p)$
\end{itemize}

\textbf{偶真陈述形式例子}:
\begin{itemize}
\item $p, \sim p, p \cdot q, p \vee q, p \supset q$
\item 对应的具体陈述:$L, \sim L, L \cdot W, L \vee W, L \supset W$
\end{itemize}
\end{examplebox}

例如,$p, \sim p, p \cdot q, p \vee q$ 和 $p \supset q$ 都是偶真陈述形式,$L, \sim L, L \cdot W, L \vee W, L \supset W$ 这样的陈述都是偶真陈述,因为它们的真值取决于它们的内容,而不只是它们的形式。

并非所有陈述形式都如上面所引的简单例子那样,明显是重言的、自相矛盾的或者偶真的。例如,陈述形式 $[(p \supset q) \supset p] \supset p$ 就一点也不明显,虽然真值表将表明它是一个重言式。它甚至还有一个特殊的名称—— "皮尔士法则"。

\subsection{实质等值的深入分析}

正如析取和实质蕴涵一样,实质等值也是一个真值函项联结词。如前所释,任何真值函项的真值,都取决于其所联结的陈述的真或假(是它们的一个函项)。

\begin{theorembox}[title=实质等值的概念分析]
\logicterm{实质等值}是这样一种真值函项联结词:它断言它所联结的陈述有同样的真值。这个概念具有以下重要特征:

\textbf{1. 真值同步性}:两个陈述实质等值,当且仅当它们具有相同的真值。

\textbf{2. 对称性}:如果A实质等值于B,那么B也实质等值于A。

\textbf{3. 传递性}:如果A实质等值于B,B实质等值于C,那么A实质等值于C。

\textbf{4. 双向蕴涵性}:实质等值等价于双向的实质蕴涵。
\end{theorembox}

因此,两个在真值上相同的陈述,就是实质上等值的。可将之径直定义为:当两个陈述都为真或都为假时,它们就是\logicterm{实质等值的}。

\subsection{实质等值的符号表示与真值条件}

正像析取的符号是楔劈号、实质蕴涵的符号是马蹄号一样,实质等值也有一个特殊的符号,即\logicterm{三杠号}"$\equiv$"。三杠号同样也可以用真值表定义如下:

\begin{center}
\begin{tabular}{|ccc|}
\hline
$p$ & $q$ & $p \equiv q$ \\
\hline
T & T & T \\
T & F & F \\
F & T & F \\
F & F & T \\
\hline
\end{tabular}
\end{center}

任何两个真陈述彼此实质地蕴涵,这是实质蕴涵含义的一个推论;同样,任何两个假陈述也彼此实质地蕴涵。因此,任何两个实质等值的陈述必定彼此蕴涵,因为它们或者都是真的,或者都是假的。

由于任何两个实质等值的陈述 A 和 B 彼此蕴涵,故而从它们的实质等值,我们可以推断出 B 是真的,当 A 是真的;也可以推断出 B 是真的,仅当 A 是真的。由于这两种关系都被实质等值所蕴涵,我们可以把三杠号"三"读做"当且仅当"。

在日常话语中,我们只偶尔使用这种逻辑关系词。有人会说,我去看冠军赛,当且仅当,我获得人场券。当我确实获得了人场券,我会去;但仅当我获得人场券,我才能去。这就是说,我去看比赛和我获得人场券,是实质上等值的。

如前所见,每个蕴涵式都是一个条件陈述。若已知 A 和 B 两个陈述实质上等值,既可推出条件陈述 $\mathrm{A} \supset \mathrm{B}$ 的真,也可推出条件陈述 $\mathrm{B} \supset \mathrm{A}$ 的真。由于在实质等值成立时,蕴涵是双向的,故而一个形如 $\mathrm{A} \equiv \mathrm{B}$ 的陈述通常称为\textbf{双条件陈述}。

合取、析取、实质蕴涵和实质等值,就是演绎论证通常所依赖的四个真值函项联结词。我们现在已经完成了对它们的讨论。

\subsection{真值函项联结词}
真值函项联结词,就是真值函项复合命题中的逻辑联结词。真值函项复合命题,就是其真(或假)完全取决于其组成分支的真或假的复合命题。具有核心重要意义的真值函项联结词有四个:\\
-圆点号.表示合取。读做:"P且Q"。\\
$P \cdot Q$ 为真,当且仅当,$P$ 为真且 $Q$ 为真。\\
V 楔劈号.表示析取。读做:" P 或 Q "。\\
$P \vee Q$ 为真,当且仅当,$P$ 为真,或 $Q$ 为真,或 $P$ 和 $Q$ 两者都为真。\\
$\supset$ 马蹄号.表示实质蕴涵。读做:" P 蕴涵 Q "。\\
$P \supset Q$ 为真,当且仅当,并非 $P$ 为真且 $Q$ 为假,也就是,当且仅当, P 为假或 Q 为真。

三三杠号.表示实质等值。读做:" P 当且仅当 Q "。\\
$\mathrm{P} \equiv \mathrm{Q}$ 为真,当且仅当, P 和 Q 有同样的真值,也就是,当且仅当, P 为真且 Q 为真,或 P 为假且 Q 为假。

\subsection{论证、条件陈述与重言式}
每个论证都对应着这样一个条件陈述:它的前件是该论证的前提的合取,它的后件是该论证的结论。例如,一个论证若具有肯定前件式的形式:

$$
\begin{aligned}
& p \supset q \\
& p \\
& \therefore q
\end{aligned}
$$

则可以被表达成一个具有形式 $[(p \supset q) \cdot p] \supset q$ 的条件陈述。如果原论证具有有效的论证形式,即在每种情形下其结论必定可以从其前提推出,那么,可以在真值表中表明,转化后的条件陈述是一个重言式。这就是说,一个论证的前提的合取蕴涵它的结论这样一个陈述有且只有真代人例(如果该论证有效的话)。

真值表是评价论证的有力工具。一个论证形式有效,当且仅当,在真值表中,其所有前提下面都是 $\mathbf{T}$ 的每一行上,其结论栏的下面也是 $\mathbf{T}$ 。这一点可以从"有效性"的精确含义中得出。如果表达该论证形式的条件陈述成了真值表中某一栏的题头,那么,$F$ 只能出现在该栏中其所有前提下面都是 $T$ ,且结论下面是 $F$ 的那一行。但如果该论证是有效的,就不会有这样一行。因此,只有 T 会出现在与一个有效论证相对应的条件陈述的下面,从而该条件陈述必定是一个重言式。所以,我们可以断言:一个论证

形式有效,当且仅当,其条件陈述表达形式(其前件是该论证形式的前提 339 的合取,其后件是该论证形式的结论)是一个重言式。

显然,对关于真值函项的任一无效论证来说,相应的条件陈述必定不是重言式。由一个无效论证的前提的合取蕴涵其结论构成的条件陈述,或者是偶真陈述,或者是矛盾陈述。

\begin{center}
\fbox{\parbox{0.95\textwidth}{
\textbf{本节要点}
\begin{itemize}
\item \textbf{陈述形式的理论基础}:
  \begin{itemize}
  \item 含有陈述变元但不含陈述的符号序列
  \item \textbf{四大特征}:形式性、生成性、分类性、可操作性
  \item 与论证形式的完全平行关系
  \item 系统的分类体系:基本形式、复合形式、混合形式
  \end{itemize}
\item \textbf{逻辑真理的三重分类}:
  \begin{itemize}
  \item \textbf{重言式}(Tautology):所有代入例都为真
    \begin{itemize}
    \item 逻辑必然性、形式性、先验性、信息空虚性
    \item 例子:$p \vee \sim p$(排中律)、$p \supset p$(同一律)
    \end{itemize}
  \item \textbf{矛盾式}(Contradiction):所有代入例都为假
    \begin{itemize}
    \item 逻辑不可能性、形式矛盾性、先验可知性、爆炸原理
    \item 例子:$p \cdot \sim p$(矛盾律)
    \end{itemize}
  \item \textbf{偶真陈述形式}(Contingent):代入例有真有假
    \begin{itemize}
    \item 真值依赖性、经验可检验性、信息承载性、可能性空间
    \item 例子:$p, p \cdot q, p \vee q, p \supset q$
    \end{itemize}
  \end{itemize}
\item \textbf{逻辑真理与经验真理的根本区别}:
  \begin{itemize}
  \item \textbf{经验真理}:依赖世界实际状况,需经验观察确定
  \item \textbf{逻辑真理}:由逻辑形式决定,独立于世界状况
  \item 哲学意义:揭示人类知识的两种来源(经验观察与逻辑推理)
  \end{itemize}
\item \textbf{实质等值的深入分析}:
  \begin{itemize}
  \item 断言两个陈述有相同真值的真值函项联结词
  \item \textbf{四大特征}:真值同步性、对称性、传递性、双向蕴涵性
  \item 符号:三杠号"$\equiv$",读作"当且仅当"
  \item 形成双条件陈述,蕴涵关系双向成立
  \item 日常应用:条件性承诺和决策表达
  \end{itemize}
\item \textbf{四种基本真值函项联结词的系统总结}:
  \begin{itemize}
  \item 合取($\cdot$):P且Q,当且仅当P为真且Q为真
  \item 析取($\vee$):P或Q,当且仅当P为真或Q为真或两者都真
  \item 实质蕴涵($\supset$):P蕴涵Q,当且仅当P为假或Q为真
  \item 实质等值($\equiv$):P当且仅当Q,当且仅当P和Q有相同真值
  \end{itemize}
\item \textbf{论证与重言式的深层关系}:
  \begin{itemize}
  \item 每个论证对应一个条件陈述(前提合取蕴涵结论)
  \item 有效论证的对应条件陈述必定是重言式
  \item 无效论证的对应条件陈述是偶真陈述或矛盾陈述
  \item 这种对应关系为论证有效性提供了机械化检验方法
  \end{itemize}
\item \textbf{复杂陈述形式}:
  \begin{itemize}
  \item 皮尔士法则:$[(p \supset q) \supset p] \supset p$
  \item 并非所有陈述形式的逻辑性质都显而易见
  \item 真值表方法的普遍适用性
  \end{itemize}
\end{itemize}
}}
\end{center}
\input{chapter8/8-6 逻辑等价.tex}
\input{chapter8/8-7 实质蕴涵怪论.tex}
\section{三大"思想法则"}

\begin{logicbox}[title=引言]
本节探讨传统上被称为"思想法则"的三个基本逻辑原理:同一原理、不矛盾原理和排中原理。通过分析这些原理的本质及其在逻辑推理中的应用和局限,我们能够更全面地理解形式逻辑的基础以及对这些原理的误解与批评。
\end{logicbox}

\subsection{思想法则的历史背景}

一些早期思想家把逻辑定义为"关于思想法则的科学",并进一步断言:刚好有三个基本思想法则,它们如此基本以至遵从它们既是正确思维的必要条件又是其充分条件。

\begin{theorembox}[title=思想法则的哲学地位]
这种观点反映了传统逻辑学的一个重要特征:

\textbf{1. 规范性}:这些法则被视为思维必须遵循的规范,违反它们就是错误的思维。

\textbf{2. 基础性}:它们被认为是所有其他逻辑原理的基础,是不可证明的公理。

\textbf{3. 普遍性}:它们被认为适用于所有可能的思维对象和思维过程。

\textbf{4. 必然性}:它们被认为是逻辑思维的必要且充分条件。
\end{theorembox}

传统上,这三大法则叫做:

1. \textbf{同一原理}。\\
这个原理断言:如果一个陈述是真的,那么它就是真的。我们可以用符号这样重述它:同一原理断言的是每个具有 $p \supset p$ 形式的陈述必定是真的,每个这样的陈述都是重言式。\\

2. \textbf{不矛盾原理}。\\
这个原理断言:没有陈述是既真又假的。我们可以用符号这样重述它:不矛盾原理断言的是每个具有 $p \cdot \sim p$ 形式的陈述必定是假的,每个这样的陈述是自相矛盾的。\\

3. \textbf{排中原理}。\\
这个原理断言:每个陈述或者是真的或者是假的。我们可以用符号这样重述它:排中原理断言的是每个具有 $p \vee \sim p$ 形式的陈述必定是真的,每个这样的陈述都是重言式。

显然,这三大原理确实是真的,是逻辑地为真的一一但说它们具有最基本的思想法则这一特权地位,是值得怀疑的。第一个(同一原理)和第三个(排中原理)是重言式,但还有许多其他的重言形式,它们的真是同等确定的。第二个(不矛盾原理)(所排除的 $p \cdot \sim p$ )也绝不是唯一的自相矛盾的陈述形式。

在构造真值表时,我们确实使用了这几个原理。受排中原理指导,我们在真值表每一行的初始栏下填人一个 T 或 F。受不矛盾原理指导,我们不在任何地方同时既填 T 又填 F 。一旦在某个指定行中把 T 填在某个符号下面,那么(受同一原理指导),当我们在那一行的其他栏下遇到该符号时,我们把它看做仍然被赋予T。因此,我们可以把这三大思想法则看做是支配真值表构造的原理。

不过,在考虑整个演绎逻辑体系时,这三大原理并不比其他许多原理更重要或更富有成效。确实,为演绎起见,有一些比它们更有成效的重言式。在这个意义上说,它们比这三大原理更重要。更深人地讨论这一点超出了本书的范围。\cite{hamilton1833}

由于相信自己设计出了某种新的不同逻辑,一些思想家声称这三大原理实际上是不正确的,遵循它们是不必要的限制。但这些批评都是建立在误解的基础上的。

基于事物都是变化的而且一直在变化这一理由,同一原理遭到了攻击。例如,对原来的由 13 个州所组成的美国来说为真的某些陈述,对今

天有 50 个州的美国来说就不再是真的。然而,这并不能伤害同一原理。语句"美国只有 13 个州"是不完整的表述,它是陈述"1790 年的美国只有 13 个州"的一种省略表述,和它在1790年时一样,这个陈述在今天也是真的。若我们把注意力限制到命题的完整的、非省略的表述,我们就会看到,它们的真(或假)并不随时间而改变。同一原理之为真,并不妨碍我们对连续性变化的认识。

不矛盾原理受到了黑格尔主义者和马克思主义者的非难,其理由是:实际矛盾是普遍存在的,世界充满着不可避免的矛盾力量的冲突。说实在世界中存在着相冲突的力量,这当然是对的,但把这些冲突力量称为"矛盾",则是对该术语的一种不精确且令人误解的使用。劳工联盟和工厂私有者发现他们确实处于冲突之中——但私有者和劳工联盟都不是对方的 "否定"、"否认"或"矛盾"。若径直按照逻辑学家所意谓的那种意义理解,不矛盾原理是不可反驳、完全准确的。

基于其导致"二值化"这一理由,排中原理成了许多批评的靶子。 "二值化"意味着断言世界上的事物必定是"或白或黑"的,由此,它妨碍了妥协的实现,导致绝对化分层。这种反对意见也来自误解。陈述 "这是黑的"当然不能与陈述"这是白的"同时为真——假如"这"指的恰是同一事物的话。尽管这两个陈述不能同时为真,但它们却能同时为假。"这"可以既不是白的又不是黑的;这两个陈述是反对关系,而不是矛盾关系。与陈述"这是白的"有矛盾关系的陈述是"并非这是白的",并且(如果在这两个陈述中,"白的"都是在完全同样的意义上使用的话),它们当中必定有一个为真而另一个为假。排中原理是不可摆脱的。

总之,所有这三大"思想法则"都是不可驳倒的——只要它们被运用于那些使用非歧义、非省略且精确的词项的陈述。它们可能不具有某些哲学家所赋予它们的那种尊贵地位\cite{kant1781},但它们无疑都是正确的。

\begin{center}
\fbox{\parbox{0.95\textwidth}{
\textbf{本节要点}
\begin{itemize}
\item \textbf{三大思想法则}及其形式表达:
  \begin{itemize}
  \item 同一原理:$p \supset p$(陈述为真则它为真)
  \item 不矛盾原理:$\sim(p \cdot \sim p)$(陈述不能既真又假)
  \item 排中原理:$p \vee \sim p$(陈述或真或假)
  \end{itemize}
\item 这些原理在真值表构造中的应用:
  \begin{itemize}
  \item 排中原理决定每个命题必赋予T或F
  \item 不矛盾原理确保不会同时赋予T和F
  \item 同一原理保证同一符号在同一行中具有相同真值
  \end{itemize}
\item 对这些原理的误解和批评:
  \begin{itemize}
  \item 基于变化而对同一原理的批评(忽略了时间因素)
  \item 基于现实冲突而对不矛盾原理的批评(误用"矛盾"概念)
  \item 基于"二值化"对排中原理的批评(混淆反对和矛盾关系)
  \end{itemize}
\item 这些原理的地位:
  \begin{itemize}
  \item 它们是真的,但非唯一的基本逻辑原理
  \item 在整个演绎体系中,其他重言式可能更为重要
  \item 它们的正确性取决于明确、精确的陈述表达
  \end{itemize}
\end{itemize}
}}
\end{center}

% 第九章
\chapter{命题逻辑}
\section{有效性的形式证明}

\begin{logicbox}[title=引言]
本节介绍如何通过一系列已知\logicemph{有效的}基本论证形式,来证明复杂论证的\logicemph{有效性}。我们将学习\logicterm{形式证明}的方法,以及九种常用的\logicterm{推论规则},这些规则可以帮助我们构造\logicemph{有效性}的形式证明。
\end{logicbox}

\subsection{形式证明方法的理论基础}

从理论上说,\logicterm{真值表}足以检验这里所探讨的任何一般类型论证的\logicemph{有效性}。但从实际操作上考虑,随着分支陈述数量的增加,真值表判定就变得愈益笨拙。

\begin{theorembox}[title=真值表方法的局限性]
真值表方法虽然理论上完备,但在实际应用中存在显著局限:

\textbf{1. 指数增长问题}:对于包含$n$个不同简单陈述的论证,需要构造$2^n$行的真值表。当$n=10$时,需要1024行;当$n=20$时,需要超过100万行。

\textbf{2. 机械性缺陷}:真值表方法是纯机械的,不能揭示论证的逻辑结构和推理步骤,缺乏洞察力。

\textbf{3. 可读性问题}:大型真值表难以阅读和验证,容易出错,不利于理解论证的逻辑关系。

\textbf{4. 教学局限}:真值表不能展示人类实际的推理过程,不利于培养逻辑思维能力。
\end{theorembox}

判定更复杂论证之\logicemph{有效性}的更有力的方法,是运用一系列已知\logicemph{有效的}基本论证,将其结论从前提演绎出来。这一方法也与日常论证方法相当吻合。

\begin{examplebox}[title=形式证明的优势]
\textbf{1. 效率优势}:对于复杂论证,形式证明通常只需要几个步骤,而真值表可能需要数千行。

\textbf{2. 结构清晰}:形式证明清楚地展示了从前提到结论的推理路径,揭示了论证的逻辑结构。

\textbf{3. 自然性}:形式证明模拟了人类的自然推理过程,符合我们的思维习惯。

\textbf{4. 可扩展性}:形式证明方法可以处理任意复杂的论证,不受陈述数量限制。
\end{examplebox}

\begin{examplebox}[title=形式证明的应用实例]
例如,考虑下述论证:

如果安德逊被提名,那么她会去波士顿。

如果她去波士顿,那么她会在那儿竞选。

如果她在那儿竞选,她会遇到道格拉斯。

安德逊没有遇到道格拉斯。

或者安德逊被提名,或者某个更合适的人被选中。

因此,某个更合适的人会被选中。
\end{examplebox}

它的\logicemph{有效性}可能在直觉上也很显然,但我们来考虑一下证明问题。为讨论方便起见,先把该论证翻译成下列符号表达式:
$A \supset B$
$B \supset C$
$C \supset D$
$\sim D$
$A \vee E$
$\therefore E$

若用\logicterm{真值表}判定这个论证的\logicemph{有效性},要求一个有 32 行的表,因为它涉及五个不同的简单陈述。但用一个只含有四个基本\logicemph{有效}论证的序列,将其结论从前提演绎出来,就能证明该论证\logicemph{有效}。从前两个前提,$A \supset B$ 和 $B \supset C$ ,根据\logicterm{假言三段论},可\logicemph{有效地}推出 $A \supset C$ 。根据另一个\logicterm{假言三段论},从 $A \supset C$ 和第三个前提 $C \supset D$ ,可\logicemph{有效地}推出 $A \supset D$ 。根据\logicterm{否定后件式},从 $A \supset D$ 和第四个前提 $\sim D$ ,可\logicemph{有效地}推出 $\sim A$ 。从 $\sim A$ 和第五个前提 $A \vee E$ ,据\logicterm{析取三段论},可\logicemph{有效地}推出该论证的结论 $E$ 。用这四个基本的\logicemph{有效}论证,结论可以从原论证的五个前提中演绎出来,这就证明了原论证是\logicemph{有效的}。在这里,基本的\logicemph{有效}论证形式,如\logicterm{假言三段论}(H.S.)、\logicterm{否定后件式}(M.T.)和\logicterm{析取三段论}(D.S.),是作为\logicterm{推理规则}来使用的。根据它们,结论从前提中\logicemph{有效地}演绎或推论出来。

把前提及从它们推出的那些陈述写在一栏,并在每个这样的陈述右边另立一栏,写上"理由",即我们所能给出的在证明中得到它的原因,我们就给出了\logicemph{有效性}的一个更形式化的证明。可以先直接列出所有前提,然后另立一行写下结论,稍微靠右侧把结论和前提分开。若所有陈述都编了号,那么,每个陈述的"理由"都要包括该陈述从之推出的那些在先的陈述的编号,以及它所依据的\logicterm{推理规则}的缩写。上述论证的\logicterm{形式证明}可以写成:

\begin{examplebox}[title=形式证明的格式]
1.$A \supset B$

2.$B \supset C$

3.$C \supset D$

4.$\sim D$

5.$A \vee E$

$\therefore E$

6.$A \supset C \quad 1,2, H . S$ .

7.$A \supset D \quad 6,3, \mathrm{H} . \mathrm{S}$ .

8.$\sim A \quad 7,4, \mathrm{M} . \mathrm{T}$ .

9.$E$ 5,8,D.S.
\end{examplebox}

\begin{theorembox}[title=形式证明的定义]
我们把给定论证的\logicemph{有效性}的一个\logicterm{形式证明}定义为一个陈述序列,该序列中的每个陈述或者是该论证的一个前提,或者是根据一个基本\logicemph{有效}论证从该序列中在先的陈述推论出来的,而该序列的最后一个陈述,就是所欲证明其\logicemph{有效性}的那个论证的结论。
\end{theorembox}

我们把一个\logicterm{基本的有效论证}定义为:该论证是一个基本的\logicemph{有效}论证形式的代入例。要强调的一点是,一个基本\logicemph{有效}论证形式的任何代人例都是一个基本\logicemph{有效}论证。例如,如下论证:

$$
\begin{aligned}
& (A \cdot B) \supset[C \equiv(D \vee E)] \\
& A \cdot B \\
& \therefore C \equiv(D \vee E)
\end{aligned}
$$

是一个基本的有效论证,因为它是基本的有效论证形式肯定前件式\\
(M.P.)的代人例。用 $A \cdot B$ 代人 $p, C \equiv(D \vee E)$ 代人 $q$ ,它可以从下述形式产生:

$$
\begin{aligned}
& p \supset q \\
& p \\
& \therefore q
\end{aligned}
$$

因此,尽管肯定前件式不是该论证的特征形式,它仍是具有肯定前件式的有效形式。

肯定前件式无疑是一个非常基本的有效论证形式,推理规则中还包括其他哪些有效论证形式呢?

\subsection{推论规则的理论基础}

\begin{theorembox}[title=推论规则的选择原则]
构造有效性的形式证明时常用的九个推论规则的选择基于以下原则:

\textbf{1. 基础性}:这些规则都是最基本、最直观的有效论证形式,符合人类的自然推理习惯。

\textbf{2. 完备性}:这九个规则加上后续的替换规则,构成了命题逻辑的一个完全系统,能够证明所有有效的真值函项论证。

\textbf{3. 独立性}:虽然某些规则在理论上可以相互推导,但保留它们是为了提高证明的效率和直观性。

\textbf{4. 实用性}:这些规则在实际的逻辑分析和数学证明中使用频率最高,具有重要的实用价值。
\end{theorembox}

\subsection{九种基本推论规则的系统分析}

如下是构造有效性的形式证明时常用的九个推论规则:
1.\textbf{肯定前件式}(M.P.)\\
$p \supset q$\\
$p$\\
$\therefore q$\\
3.\textbf{假言三段论}(H.S.)\\
$p \supset q$\\
$q \supset r$\\
$\therefore p \supset r$\\
5.\textbf{构造式二难}(C.D.)\\
$(p \supset q) \cdot(r \supset s)$\\
$p \vee r$\\
$\therefore q \vee s$\\
7.\textbf{简化律}(Simp.)\\
$p \cdot q$\\
$\therefore p$

2.\textbf{否定后件式}(M.T.)\\
$p \supset q$\\
$\sim q$\\
$\therefore \sim p$\\
4.\textbf{析取三段论}(D.S.)\\
$p \vee q$\\
$\sim p$\\
$\therefore q$\\
6.\textbf{吸收律}(Abs.)\\
$p \supset q$\\
$\therefore p \supset(p \cdot q)$

8.\textbf{合取律}(Conj.)\\
$p$\\
$q$\\
$\therefore p \cdot q$

9.\textbf{附加律}(Add.)\\
$p$\\
$\therefore p \vee q$

\subsection{推论规则的深入分析}

这九个推论规则的有效性很容易用真值表判定。在它们的帮助下,可以为大量更复杂的论证构造有效性的形式证明。

\begin{theorembox}[title=推论规则的分类与特征]
这九个推论规则可以按其逻辑功能进行分类:

\textbf{1. 条件推理规则}:
\begin{itemize}
\item \textbf{肯定前件式}(M.P.):最基本的条件推理,体现了"如果-那么"的核心逻辑
\item \textbf{否定后件式}(M.T.):反向条件推理,是反证法的基础
\item \textbf{假言三段论}(H.S.):条件链式推理,体现了传递性
\end{itemize}

\textbf{2. 析取推理规则}:
\begin{itemize}
\item \textbf{析取三段论}(D.S.):排除法推理,体现了"非此即彼"的逻辑
\item \textbf{构造式二难}(C.D.):复合析取推理,处理多重选择情况
\end{itemize}

\textbf{3. 合取操作规则}:
\begin{itemize}
\item \textbf{简化律}(Simp.):从合取中提取单个合取支
\item \textbf{合取律}(Conj.):将多个陈述合并为合取
\end{itemize}

\textbf{4. 扩展规则}:
\begin{itemize}
\item \textbf{附加律}(Add.):向陈述添加析取支
\item \textbf{吸收律}(Abs.):条件陈述的内部扩展
\end{itemize}
\end{theorembox}

\begin{examplebox}[title=推论规则的历史发展]
\textbf{古代起源}:肯定前件式和否定后件式可以追溯到古希腊的斯多葛学派,是最早被系统化的推理形式。

\textbf{中世纪发展}:假言三段论在中世纪逻辑学中得到了详细的分析和应用。

\textbf{现代形式化}:19世纪末20世纪初,这些规则被纳入现代符号逻辑体系,获得了精确的形式化表述。

\textbf{计算机应用}:在现代计算机科学中,这些规则成为了自动定理证明和专家系统的基础。
\end{examplebox}

所列的这些名称都是标准名称,而使用它们的缩写使得用最少量的书写就可以把形式证明记录下来。这种标准化的符号系统不仅提高了效率,也促进了国际学术交流。

\begin{center}
\fbox{\parbox{0.95\textwidth}{
\textbf{本节要点}
\begin{itemize}
\item \textbf{形式证明方法的理论基础}:
  \begin{itemize}
  \item 真值表方法的四大局限性:指数增长、机械性缺陷、可读性问题、教学局限
  \item 形式证明的四大优势:效率优势、结构清晰、自然性、可扩展性
  \item 形式证明模拟人类自然推理过程,与日常论证方法相当吻合
  \end{itemize}
\item \textbf{形式证明的定义与结构}:
  \begin{itemize}
  \item 陈述序列中每个陈述要么是原论证的前提,要么是根据推理规则得出
  \item 最后一个陈述是原论证的结论
  \item 每个推导步骤都必须明确标注所使用的推理规则和前提行号
  \end{itemize}
\item \textbf{推论规则的选择原则}:
  \begin{itemize}
  \item \textbf{基础性}:最基本、最直观的有效论证形式
  \item \textbf{完备性}:与替换规则一起构成命题逻辑的完全系统
  \item \textbf{独立性}:保留冗余规则以提高证明效率和直观性
  \item \textbf{实用性}:在逻辑分析和数学证明中使用频率最高
  \end{itemize}
\item \textbf{九种推论规则的系统分类}:
  \begin{itemize}
  \item \textbf{条件推理规则}:肯定前件式(M.P.)、否定后件式(M.T.)、假言三段论(H.S.)
  \item \textbf{析取推理规则}:析取三段论(D.S.)、构造式二难(C.D.)
  \item \textbf{合取操作规则}:简化律(Simp.)、合取律(Conj.)
  \item \textbf{扩展规则}:附加律(Add.)、吸收律(Abs.)
  \end{itemize}
\item \textbf{推论规则的历史发展}:
  \begin{itemize}
  \item 古代起源:斯多葛学派的肯定前件式和否定后件式
  \item 中世纪发展:假言三段论的详细分析
  \item 现代形式化:19-20世纪的符号逻辑体系
  \item 计算机应用:自动定理证明和专家系统的基础
  \end{itemize}
\item \textbf{基本有效论证与代入例}:
  \begin{itemize}
  \item 基本有效论证形式的任何代入例都是基本有效论证
  \item 复杂陈述可以作为简单变元的代入,扩展了规则的适用范围
  \item 标准化符号系统提高效率并促进国际学术交流
  \end{itemize}
\end{itemize}
}}
\end{center}
\section{替换规则}

\begin{logicbox}[title=引言]
本节讨论如何使用替换规则来增强形式证明的能力。我们将学习十种重要的逻辑等价关系,它们可以作为替换规则使用,从而使我们能够构造更复杂论证的有效性证明。本节还会探讨形式证明的能行性及其与真值表方法的区别。
\end{logicbox}

\subsection{替换规则的必要性}

仅仅依靠前九个推论规则,某些明显有效的论证的有效性得不到证明。例如,要为下述明显有效的论证构造一个有效性的形式证明:

$$
\begin{aligned}
& A \supset B \\
& C \supset \sim B \\
& \therefore A \supset \sim C
\end{aligned}
$$

就要求增加新的规则。

\begin{theorembox}[title=替换规则的理论基础]
替换规则建立在以下重要的逻辑原理之上:

\textbf{1. 真值函项性原理}:在任何真值函项复合陈述中,如果它的一个分支陈述被另外一个有相同真值的陈述替换,该复合陈述的真值保持不变。

\textbf{2. 逻辑等价性原理}:两个逻辑等价的陈述在所有可能的真值指派下都具有相同的真值,因此可以在任何语境中相互替换。

\textbf{3. 组合性原理}:复合陈述的真值完全由其组成部分的真值和逻辑联结词决定,这保证了局部替换的全局有效性。

\textbf{4. 保真性原理}:替换操作保持论证的有效性,不会将有效论证变为无效论证。
\end{theorembox}

由于我们这里所关注的只有真值函项复合陈述,因此,我们可以把\logicterm{替换规则}作为新的推论规则接受下来,该规则允许我们对任何陈述都可以做如下替换:该陈述的任一分支陈述都可被替换为与其逻辑等价的陈述。

\begin{examplebox}[title=替换规则的应用示例]
根据断言 $p$ 逻辑地等价于 $\sim \sim p$ 的双重否定原则,通过替换,我们可以从 $A \supset \sim \sim B$ 推出下面的任何一个陈述:

$$A \supset B, \sim \sim A \supset \sim \sim B, \sim \sim(A \supset \sim \sim B) \text {, 或 } A \supset \sim \sim \sim \sim B$$

这种灵活性大大增强了形式证明的能力,使我们能够处理更复杂的逻辑结构。
\end{examplebox}

为把这项新规则加以确定,我们列出可以使用的十个重言的或逻辑地为真的双条件式。这些双条件式提供了在证明复杂论证的有效性时可使用的一些新增推论规则。我们接着 9.1 节所列的九条规则,给它们连续编号。

\subsection{替换规则列表}

下面任一逻辑等价的形式,在它们出现的任何地方,都可以相互替换:

10.\textbf{德摩根律}(De M.):$\sim(p \cdot q) \xlongequal{\mathrm{T}}(\sim p \vee \sim q)$

$$
\sim(p \vee q) \xlongequal{\mathrm{T}}(\sim p \cdot \sim q)
$$

11.\textbf{交换律}(Com.):$(p \vee q) \xlongequal{\text { T }}(q \vee p)$

$$
(p \cdot q) \stackrel{\mathrm{T}}{=}(q \cdot p)
$$

12.\textbf{结合律}(Assoc.):$[p \vee(q \vee r)] \stackrel{\mathrm{T}}{=}[(p \vee q) \vee r]$

$$
[p \cdot(q \cdot r)] \stackrel{\mathrm{T}}{=}[(p \cdot q) \cdot r]
$$

13.\textbf{分配律}(Dist.):$[p \cdot(q \vee r)] \stackrel{\mathrm{T}}{=}[(p \cdot q) \vee(p \cdot r)]$

$$
[p \vee(q \cdot r)] \stackrel{\mathrm{T}}{=}[(p \vee q) \cdot(p \vee r)]
$$

14.\textbf{双重否定律}(D.N.):$p \stackrel{\mathrm{~T}}{=} \sim p$

15.\textbf{易位律}(Trans.):$(p \supset q) \xlongequal{\text { T }}(\sim q \supset \sim p)$\\
16.\textbf{实质蕴涵律}(Impl,):( $p \supset q$ )$\xlongequal{T}(\sim p \vee q)$\\
17.\textbf{实质等值律}(Equiv.):$(p \equiv q) \stackrel{\mathrm{T}}{=}[(p \supset q) \cdot(q \supset p)]$

$$
(p \equiv q) \stackrel{T}{=}[(p \cdot q) \vee(\sim p \cdot \sim q)]
$$

18.\textbf{输出律}(Exp.):$[(p \cdot q) \supset r] \stackrel{\mathrm{T}}{=}[p \supset(q \supset r)]$\\
19.\textbf{重言律}(Taut.)\cite{wittgenstein1922}: $p \stackrel{\mathrm{~T}}{=}(p \vee p)$

$$
p \stackrel{T}{\equiv}(p \cdot p)
$$

\subsection{替换规则的深入分析}

\begin{theorembox}[title=十种替换规则的分类与特征]
这十种替换规则可以按其逻辑功能进行系统分类:

\textbf{1. 结构变换规则}:
\begin{itemize}
\item \textbf{德摩根律}:否定与合取/析取的相互转换,体现了对偶性原理
\item \textbf{交换律}:改变运算顺序而不改变逻辑含义
\item \textbf{结合律}:改变运算分组而不改变逻辑含义
\item \textbf{分配律}:不同逻辑联结词之间的分配关系
\end{itemize}

\textbf{2. 简化规则}:
\begin{itemize}
\item \textbf{双重否定律}:消除冗余的否定
\item \textbf{重言律}:消除冗余的重复
\end{itemize}

\textbf{3. 转换规则}:
\begin{itemize}
\item \textbf{易位律}:条件陈述的逆否转换
\item \textbf{实质蕴涵律}:条件陈述与析取的等价转换
\item \textbf{实质等值律}:双条件陈述的两种等价形式
\item \textbf{输出律}:复合条件的嵌套转换
\end{itemize}
\end{theorembox}

\begin{examplebox}[title=替换规则的历史意义]
\textbf{布尔代数的贡献}:德摩根律、交换律、结合律、分配律都源于乔治·布尔的代数逻辑研究,奠定了现代计算机科学的基础。

\textbf{逻辑等价的发现}:这些规则的发现过程体现了逻辑学从直觉推理向形式化系统的转变,每个规则都代表了对逻辑结构的深刻洞察。

\textbf{现代应用}:在计算机科学中,这些规则被广泛应用于电路设计、程序优化、自动推理等领域,具有重要的实用价值。
\end{examplebox}

\subsection{替换与代入的区别}

替换的过程与代人非常不同:代人是以陈述代人陈述变元,而替换是以其他陈述替换陈述。从一个陈述形式到它的代人例的过程中,或者在从一个论证形式到其代人例的过程中,只要一个陈述被代人一个陈述变元的某次出现,它也必须被代人到该陈述变元的所有其他出现;在遵守此规定的条件下,我们就能用任何陈述代人任何陈述变元。但在从一个陈述到另一个陈述的替换过程中,运用10-19 中的某个逻辑等值式,我们只用一个与之逻辑等价的陈述,就能够替换第一个陈述中的某个分支陈述,我们可以只替换该分支陈述的某次出现,而不需要替换它的任何其他出现。

这 19 个推论规则并不构成这样一个极小集,即用它足以形式地证明复杂论证的有效性,在这个意义上说,它们有点多余。例如,否定后件式可以从表中去掉而并不真正削弱我们的证明手段,因为依据否定后件式的任何一行,实际上都能由表中的其他规则给予辩护。本章第 350 页(边码)所给的第一个形式证明例子中的第 8 行,$\sim A$ ,就是根据否定后件式,从第四和第七行,即 $\sim D$ 和 $A \supset D$ ,演绎出来的。但如果不把否定后件式作为推理规则,我们仍然能从 $A \supset D$ 和 $\sim D$ 演绎出 $\sim A$ 。譬如,在它们中间插人~Dつ~A这样一行,就可以做到这一点。 $\sim D \supset \sim A$ 可以根据易位原则(Trans.)从 $A \supset D$ 推出,然后根据肯定前件式(M.P.),可以从 $\sim D$ $\supset \sim A$ 和 $\sim D$ 得到 $\sim A$ 。但否定后件式作为一个运用如此频繁且直觉上如此显明的推论规则,应该被包括在推论规则之内。这 19 个中的其他一些规则,在这个意义上也是多余的。

\subsection{推论规则表的完备性}

这个推论规则表不仅有冗余的特点,它还有某种不足。例如,尽管论证:

$$
\begin{aligned}
& \sim B \\
& \therefore A
\end{aligned}
$$

直觉上有效,但它的形式:

$$
\begin{aligned}
& p \vee q \\
& \sim q \\
& \therefore p
\end{aligned}
$$

却没有包括在推论规则之内。尽管结论 A 可以根据两个推论规则,从前提 $A \vee B$ 和 $\sim B$ 演绎出来,但它并不是根据任何单一的推论规则,从这两个前提推出来的。该论证有效性的形式证明可以写为:

1.$A \vee B$\\
2.$\sim B$\\
$\therefore A$\\
3.$B \vee A \quad 1$ ,Com.\\
4.A $3,2, \mathrm{D} . \mathrm{S}$ .\\
若在推论规则表中添加另外一个规则,我们可以消除这种不足。但是,如果我们对每个这样的情形都添加一个规则,我们最终会有一个长得多且更不易处理的规则表。

对任何有效的真值函项论证来说,目前这个有 19 个推理规则的列表使得我们都可以为之构造一个有效性的形式证明。在这个意义上说,该表构成了真值函项逻辑的一个\textbf{完全的系统}。\cite{gentzen1935}

\subsection{形式证明的能行性深入分析}

形式证明是一个\logicterm{能行的}(effective)概念。这一概念在逻辑学和计算机科学中具有根本性的重要意义。

\begin{theorembox}[title=能行性的精确定义]
所谓\logicterm{能行的},是指根据给定的推论规则表,可以在有限步骤内机械地判定一个给定陈述序列是否构成一个形式证明。能行性具有以下特征:

\textbf{1. 机械性}:整个验证过程可以完全机械化,不需要任何创造性思维或直觉判断。

\textbf{2. 有限性}:验证过程必须在有限步骤内完成,不能无限进行下去。

\textbf{3. 确定性}:对于任何给定的陈述序列,验证过程必须给出明确的"是"或"否"的答案。

\textbf{4. 算法性}:验证过程可以用算法描述,并且可以由计算机程序实现。
\end{theorembox}

这里不需要任何思维。所谓不需要思维,就是既不需要思考序列中的陈述的"意义",也不需要用逻辑直觉来检查任何步骤的有效性。

\begin{examplebox}[title=能行性验证的两个基本要求]
能行性验证只需要做两件基本的事情:

\textbf{第一件事}:\logicemph{语法匹配}——能够看出在一个地方出现的某个陈述与在另一个地方出现的一个陈述是完全相同的。这包括:
\begin{itemize}
\item 核对证明中的某些陈述是所欲证明其有效性的那个论证的前提
\item 确认证明中的最后一个陈述是该论证的结论
\item 验证引用的前提行号是否正确
\end{itemize}

\textbf{第二件事}:\logicemph{模式识别}——能够看出一个给定陈述是否有某种模式,即能看出它是否是某个陈述形式的代入例。这包括:
\begin{itemize}
\item 识别陈述是否符合某个推论规则的前提模式
\item 验证结论是否符合相应的结论模式
\item 检查替换是否遵循逻辑等价关系
\end{itemize}
\end{examplebox}

这样,关于上列陈述序列是否是一个有效性的形式证明的问题,就很容易用一种完全机械的方式来解答。一眼就可以看出,第1行和第2行是该论证的前提,第 4 行是结论。第 3 行是根据某个给定推论规则从前面几行推出的,这一点可以在有限几步内确定——即使符号"1, Com."不写在旁边。第二栏中的解释性符号起帮助作用,它应该包括在证明内。但严格说来,它本身并不是证明的一个必要部分。每一行的前面只有有限几行,并且只有有限多的推论规则或凭据形式可查。尽管费时,但通过对形式进行观察和比较,可以确定第 3 行不是根据肯定前件式、否定后件式或假言三段论等从第 1 行和第 2 行推出的。一直依照这种程序进行,直到我们碰到这样一个问题:第 3 行是否是根据交换律从第 1 行和第 2 行推出来的?此时,仅通过观察形式,我们就可以知道的确如此。任何形式证明中的任何陈述的合法性,都可用同样的方式在有限步骤内得到检验。没有哪一步涉及形式或形态比较之外的任何其他东西。为了保持这种能行性,我们要求一次只采取一个步骤。可能有人想合并几步以缩短证明,但所节约的时间和空间是微不足道的,通过每步只用一个推论规则而获得能行性才是更重要的。

尽管在有效性的形式证明能够机械地确定一个给定序列是否一个证明的意义上,形式证明是能行的,但建构一个形式证明并没有一个能行的程序。在这方面,形式证明不同于真值表方法。真值表的构造是完全机械的:给定任何一个我们现在所关注的那类论证,依照上一章规定的简单程序规则,我们总能构造一个真值表来检验其有效性。但我们没有能行的或机械的规则来构造形式证明。我们必须思考或"想出"从哪儿着手,以及怎样前进。不过,通过构造一个有效性的形式证明来证明一个论证的有效性,比纯机械构造的真值表方法要简单得多。这样的真值表可能有几百甚至几千行。

\subsection{前九条与后十条规则的区别}

前九条和后十条推论规则之间有很重要的区别。前九条规则只能运用到证明中的完整行上。例如,在有效性的形式证明中,只有当 $A \cdot B$ 构成一完整行时,陈述 $A$ 才能根据简化律从陈述 $A \cdot B$ 推出。显然,$A$ 不能有效地从 $(A \cdot B) \supset C$ 或 $C \supset(A \cdot B)$ 推出,因为在 $A$ 为假时,后两个陈述可以为真。陈述 $A \supset C$ 也不能根据简化律或任何其他推论规则,从 $(A \cdot B) \supset C$ 推出。它根本推不出,因为如果 $A$ 为真,并且 $B$ 和 $C$ 都为假,那么 $(A \cdot B) \supset C$为真,而 $A \supset C$ 为假。再如,尽管根据附加律,$A \vee B$ 可以从 $A$ 推得,但

我们不能根据附加律或其他任何推论规则,从 $A \supset C$ 推出 $(A \vee B) \supset C$ 。因为如果 $A$ 和 $C$ 都为假而 $B$ 为真,则 $A \supset C$ 为真而 $(A \vee B) \supset C$ 为假。另一方面,后十条推论规则中的任何一个都既可运用到整行,也可运用到行中的某些部分。根据输出律,不仅可以从整行 $(A \cdot B) \supset C$ 推出陈述 $A \supset$ $(B \supset C)$ ,我们还可以从行 $[(A \cdot B) \supset C] \vee D$ 推出 $[A \supset(B \supset C)] \vee D$ 。根据替换规则,逻辑等价式可以相互替换它们的每次出现,即使它们并不构成证明中的一个完整行。但前九条推论规则只能运用到证明中的完整行中,而且这些完整行是作为前提来使用的。

\subsection{构造形式证明的技巧}

尽管没有构造形式证明的纯机械性规则,但可以给出一些大略的规则,或一些关于证明进程的提示。第一个提示是,要根据给定推论规则从给定前提着手演绎结论。随着越来越多的子结论成为进一步演绎的前提,会越来越清楚该如何演绎出所欲证明为有效的那个论证的结论。另一个提示是,要努力消除那些在前提中出现而结论中不出现的陈述。当然,这种消除只能依据推论规则进行。这些推论规则中含有许多消除陈述的技巧。简化律就是这样一个规则,借此可以去掉整行中合取式右边的合取支。交换律允许我们把合取式左边的合取支换到右边,然后根据简化律就可以去掉那个合取支了。给定两个具有模式 $p \supset q$ 和 $q \supset r$ 的陈述,根据假言三段论,可以消除"中项"$q$ 。分配律是一个把形如 $p \vee(q \cdot r)$ 的析取式变换为合取式 $(p \vee q) \cdot(p \vee r)$ 的有用规则。根据简化律,就可以消除这个合取式右边的合取支。另一个值得提出的规则是,可根据附加律,引人一个结论中出现但前提中未出现的陈述。再一个常用的方法是,从结论倒澌寻找结论从中演绎出来的那个或那些陈述,然后试着从前提演绎出那些中间陈述。然而,要想熟练地掌握构造形式证明的方法,习题训练是无可替代的途径。

\begin{center}
\fbox{\parbox{0.95\textwidth}{
\textbf{本节要点}
\begin{itemize}
\item \textbf{替换规则的理论基础}:
  \begin{itemize}
  \item \textbf{四大逻辑原理}:真值函项性、逻辑等价性、组合性、保真性原理
  \item 替换规则建立在逻辑等价关系之上,保证了论证有效性的保持
  \item 大大增强了形式证明的能力,使我们能够处理更复杂的逻辑结构
  \end{itemize}
\item \textbf{十种替换规则的系统分类}:
  \begin{itemize}
  \item \textbf{结构变换规则}:德摩根律、交换律、结合律、分配律
  \item \textbf{简化规则}:双重否定律、重言律
  \item \textbf{转换规则}:易位律、实质蕴涵律、实质等值律、输出律
  \item 每类规则都有其特定的逻辑功能和应用场景
  \end{itemize}
\item \textbf{替换规则的历史意义}:
  \begin{itemize}
  \item 源于布尔代数的贡献,奠定了现代计算机科学基础
  \item 体现了逻辑学从直觉推理向形式化系统的转变
  \item 在电路设计、程序优化、自动推理等领域有重要应用
  \end{itemize}
\item \textbf{替换与代入的本质区别}:
  \begin{itemize}
  \item \textbf{代入}:以陈述代入陈述变元,必须在所有出现处一致替换
  \item \textbf{替换}:以逻辑等价的陈述替换另一陈述的部分,可仅替换某次出现
  \item 替换基于逻辑等价关系,代入基于变元的统一性要求
  \end{itemize}
\item \textbf{推论规则表的完备性}:
  \begin{itemize}
  \item 19个推理规则构成真值函项逻辑的完全系统
  \item 虽然某些规则在理论上冗余,但保留它们提高了实用性
  \item 可以为任何有效的真值函项论证构造形式证明
  \end{itemize}
\item \textbf{形式证明的能行性深入分析}:
  \begin{itemize}
  \item \textbf{能行性四特征}:机械性、有限性、确定性、算法性
  \item \textbf{验证的两个基本要求}:语法匹配和模式识别
  \item 验证形式证明是能行的,但构造形式证明不是能行的
  \item 与真值表方法的根本区别:构造vs验证的能行性差异
  \end{itemize}
\item \textbf{前九条与后十条规则的应用区别}:
  \begin{itemize}
  \item 前九条规则(推论规则)只能应用于证明中的完整行
  \item 后十条规则(替换规则)可应用于整行或行中的部分内容
  \item 这种区别反映了不同类型逻辑操作的本质差异
  \end{itemize}
\item \textbf{构造形式证明的策略技巧}:
  \begin{itemize}
  \item 从前提出发演绎结论,逐步构建推理链条
  \item 努力消除前提中出现而结论中不出现的陈述
  \item 利用各种规则的消除和引入技巧
  \item 从结论倒推寻找中间陈述,然后从前提正推
  \end{itemize}
\end{itemize}
}}
\end{center}

\begin{center}
\begin{tabular}{|l|l|}
\hline
\multicolumn{2}{|c|}{推论规则} \\
\hline
\multicolumn{2}{|r|}{我们阐述了构造有效性证明要使用的19条规则。它们是:} \\
\hline
基本有效论证形式: & 逻辑等价表达式: \\
\hline
1.肯定前件式(M.P.): & 10.德摩根律(De M.): \\
\hline
$p \supset q, \quad p, \quad \therefore q$ &  \\
\hline
 & \( \begin{aligned} & \sim(p \cdot q) \stackrel{\mathrm{T}}{=}(\sim p \vee \sim q) \\ & \sim(p \vee q) \stackrel{\mathrm{T}}{=}(\sim p \cdot \sim q) \end{aligned} \) \\
\hline
\end{tabular}
\end{center}

2.否定后件式(M.T,):\\
$p \supset q, \sim q, \therefore \sim p$

3.假言三段论(H.S.):\\
$p \supset q, q \supset r, \therefore p \supset r$

4.析取三段论(D.S.):\\
$p \vee q, \sim p, \therefore q$

5.构造式二难(C.D.):\\
$(p \supset q) \cdot(r \supset s), p \vee r, \therefore q \vee s$\\
6.吸收律(Abs.):\\
$p \supset q, \therefore p \supset(p \cdot q)$\\
7.简化律(Simp.):\\
$p \cdot q, \therefore p$

8.合取律(Conj.)\\
$p, q, \therefore p \cdot q$

9.附加律(Add.):\\
$p, \therefore p \vee q$

11.交换律(Com.):\\
$(p \vee q) \stackrel{\mathrm{T}}{=}(q \vee p)$\\
$(p \cdot q) \stackrel{\mathrm{T}}{=}(q \cdot p)$\\
12.结合律(Assoc.):\\
$[p \vee(q \vee r)] \stackrel{\mathrm{T}}{\equiv}[(p \vee q) \vee r]$\\
$[p \cdot(q \cdot r)] \stackrel{\mathrm{T}}{=}[(p \cdot q) \cdot r]$\\
13.分配律(Dist.):\\
$[p \cdot(q \vee r)] \stackrel{\mathrm{T}}{=}[(p \cdot q) \vee(p \cdot r)]$\\
$[p \vee(q \cdot r)] \stackrel{\mathrm{T}}{\equiv}[(p \vee q) \cdot(p \vee r)]$\\
14.双重否定律(D.N.):\\
$p \stackrel{\mathrm{~T}}{=} \sim p$\\
15.易位律(Trans.):\\
$(p \supset q) \stackrel{T}{\equiv}(\sim q \supset \sim p)$\\
16.实质蕴涵律(Impl.):\\
$(p \supset q) \stackrel{\mathrm{T}}{=}(\sim p \vee q)$\\
17.实质等值律(Equiv.):\\
$(p \equiv q) \stackrel{\mathrm{T}}{\equiv}[(p \supset q) \cdot(q \supset p)]$\\
$(p \equiv q) \stackrel{\mathrm{T}}{=}[(p \cdot q) \vee(\sim p \cdot \sim q)]$\\
18.输出律(Exp.):\\
$[(p \cdot q) \supset r] \stackrel{\mathrm{T}}{=}[p \supset(q \supset r)]$\\
19.重言律(Taut.):\\
$p \stackrel{\mathrm{~T}}{=}(p \vee q)$\\
$p \stackrel{\mathrm{~T}}{=}(p \cdot p)$
\section{无效性的证明}

\begin{logicbox}[title=引言]
本节讨论如何证明论证的无效性,介绍一种比真值表更简洁高效的方法。通过为论证中的简单陈述指派适当的真值,使前提为真而结论为假,我们可以直接证明一个论证形式的无效性,而无需构造完整的真值表。
\end{logicbox}

\subsection{形式证明方法的局限性}

对一个无效论证来说,当然没有其有效性的形式证明。但如果我们找不到一个给定论证的有效性的形式证明,这种失败并不就能证明该论证无效,也并不证明不可能构造出形式证明。

\begin{theorembox}[title=形式证明方法的不对称性]
形式证明方法存在以下根本性的不对称性:

\textbf{1. 正向完备性}:如果一个论证是有效的,那么存在该论证的有效性形式证明。这保证了所有有效论证原则上都是可证明的。

\textbf{2. 负向不完备性}:如果我们不能构造某个论证的有效性形式证明,这并不能证明该论证无效。失败可能有两种原因:
\begin{itemize}
\item 论证确实无效(客观原因)
\item 我们缺乏足够的聪明才智或坚持不懈的精神(主观原因)
\end{itemize}

\textbf{3. 非能行性后果}:这种不对称性是证明构造过程的非能行性的直接后果。我们没有机械的程序来构造形式证明,因此失败并不意味着不可能性。

\textbf{4. 方法论需求}:这表明我们需要一种积极的方法来证明论证的无效性,而不能仅仅依赖于构造证明的失败。
\end{theorembox}

那么,怎样构成一个给定论证\logicterm{无效性}的一个证明呢?

\subsection{无效性证明的理论基础}

如下描述的方法与真值表方法密切相关,尽管这个方法比后者简略得多。

\begin{theorembox}[title=真值指派方法的逻辑原理]
真值指派方法建立在以下逻辑原理之上:

\textbf{1. 反例原理}:要证明一个论证无效,只需要找到一个反例——即一种情况,在这种情况下前提都为真而结论为假。

\textbf{2. 存在性证明}:如果能在真值表中发现这样一个情形(或一行),即对一个论证形式中的陈述变元进行这样一种\logicterm{真值指派},使得其前提为真而结论为假,那么该论证形式就是无效的。

\textbf{3. 充分性原理}:如果我们能对一个论证的简单分支陈述进行这样的真值指派,即使得它的前提为真且结论为假,那么,这种指派就足以证明该论证无效。

\textbf{4. 效率原理}:实际上,进行这种指派正是真值表所做的。但如果我们不实际构造完整的真值表就能进行这种真值指派,那么可以省去很多工作。
\end{theorembox}

\begin{examplebox}[title=真值指派方法的优势]
相比于完整的真值表方法,真值指派方法具有以下优势:

\textbf{1. 效率优势}:对于包含$n$个简单陈述的论证,真值表需要$2^n$行,而真值指派方法只需要找到一行反例。

\textbf{2. 目标导向}:我们可以有策略地寻找使前提为真、结论为假的指派,而不是盲目地检查所有可能性。

\textbf{3. 认知友好}:这种方法更符合人类的推理习惯,我们通常通过寻找反例来质疑论证。

\textbf{4. 可扩展性}:对于非常复杂的论证,这种方法仍然可行,而完整真值表可能变得不可处理。
\end{examplebox}

考查这样一个论证:

如果地方官赞同政府为低收入者修建住房,那么他会赞成限制私有企业的规模。

如果地方官是一个社会主义者,那么他会赞成限制私有企业的规模。

因此,如果地方官赞同政府为低收入者修建住房,那么他是一个社会主义者。

它可以符号化为:

$$
\begin{aligned}
& F \supset R \\
& S \supset R \\
& \therefore F \supset S
\end{aligned}
$$

我们不必构造完整的真值表就可以证明它无效。首先可以提问:"使结论为假要求何种真值指派?"显然,一个条件陈述为假,仅当它的前件为真而后件为假。因此,给 $F$ 指派真值"真",且给 $S$ 指派真值"假",会使得结论 $F \supset S$ 为假。现在,如果把真值"真"指派给 $R$ ,那么两个前提都是真的,因为只要它的后件为真,该条件陈述就为真。于是我们可以说,如果把真值"真"指派给 $F$ 和 $R$ ,把真值"假"指派给 $S$ ,该论证就有真前提和假结论,据此,它就被证明为无效。

\subsection{与真值表方法的关系}

这种证明无效性的方法是真值表证明方法的一个变种,因而应注意这二者之间的本质联系。实际上,在我们进行如上所示的那种真值指派时,我们所做的就是构造给定论证的真值表中的一行。若我们把这种真值指派水平地写成下述形式:

\begin{center}
\begin{tabular}{|cccccc|}
\hline
$F$ & $R$ & $S$ & $F \supset R$ & $S \supset R$ & $F \supset S$ \\
\hline
真 & 真 & 假 & 真 & 真 & 假 \\
\hline
\end{tabular}
\end{center}

这种关系会看得更清楚。其中,上述真值指派构成了论证真值表中的一行 (第二行)。通过显示其真值表中至少有这样一行,即其前提都为真而结论为假,一个论证就被证明为无效。因此,要发现一个论证的无效性,我们不必检验它的真值表的每一行:只要发现有一行,它的前提都为真而结论为假,这就足够了。当前证明无效性的方法,就是一种构造这样一行而不必构造整个真值表的方法。\cite{jeffrey1967}

\subsection{真值指派的系统策略}

目前这种方法比写出整个真值表要简略,一个论证所涉及的简单分支陈述越多,这种方法相应地节省的时间和空间也越多。对一个前提相当多,或相当复杂的论证来说,进行所需的真值指派可能不太容易。虽然没有机械的处理办法,但某些系统性的策略是有帮助的。

\begin{theorembox}[title=真值指派的基本策略]
\textbf{1. 基本陈述优先原则}:如果要证明无效性,给那些立即就能看出是基本的陈述指派真值是最有效的做法。

\textbf{2. 前提约束原则}:任何仅断言某陈述 $S$ 为真的前提,立刻提示我们对 $S$指派 $T$(或 $F$ ,如果作为前提的 $S$ 已被断言为假),因为我们知道所有前提必须被处理为真。

\textbf{3. 结论否定原则}:同一原则适用于结论中的陈述,只是那里的真值指派必须使结论为假。例如:
\begin{itemize}
\item 形如 $A \supset B$ 的结论:对 $A$ 指派 $T$ ,对 $B$ 指派 $F$
\item 形如 $A \vee B$ 的结论:对 $A$ 指派 $F$ ,对 $B$ 指派 $F$
\item 形如 $A \cdot B$ 的结论:对 $A$ 指派 $F$ 或对 $B$ 指派 $F$
\end{itemize}

\textbf{4. 策略选择原则}:我们应该从寻求使前提为真出发,还是从寻求使结论为假出发,取决于这些命题的结构。一般来说,我们最好从最有把握的指派开始。
\end{theorembox}

\begin{examplebox}[title=真值指派的高级策略]
\textbf{1. 约束传播}:从一个确定的指派开始,通过逻辑约束传播到其他陈述。例如,如果$A \supset B$必须为真且$A$为真,那么$B$也必须为真。

\textbf{2. 冲突检测}:在指派过程中检测是否出现冲突。如果某个陈述被要求同时为真和假,则当前指派路径失败,需要回溯。

\textbf{3. 试探性指派}:当没有明显的强制指派时,可以进行试探性指派,然后检查是否能够完成一致的指派。

\textbf{4. 结构分析}:分析论证的逻辑结构,识别关键的"瓶颈"陈述,优先处理这些陈述的真值指派。
\end{examplebox}

当然,会有许多这样的情形,其第一次指派不得不是任意的和试探性的。一定数量的试错是必要的。但即使这样,这种证明无效性的方法,也几乎总比写出完整的真值表简略和容易。

\subsection{方法的局限性与适用范围}

\begin{theorembox}[title=真值指派方法的局限性]
虽然真值指派方法通常很有效,但它也有一些局限性:

\textbf{1. 非机械性}:与真值表方法不同,真值指派方法不是完全机械的,需要一定的策略和技巧。

\textbf{2. 试错需求}:对于复杂的论证,可能需要多次尝试不同的指派组合才能找到反例。

\textbf{3. 不完全性}:如果论证实际上是有效的,这种方法会失败,但失败本身不能证明有效性。

\textbf{4. 复杂度依赖}:对于某些特殊结构的论证,寻找反例可能仍然很困难。
\end{theorembox}

\begin{center}
\fbox{\parbox{0.95\textwidth}{
\textbf{本节要点}
\begin{itemize}
\item \textbf{形式证明方法的不对称性}:
  \begin{itemize}
  \item \textbf{正向完备性}:有效论证总是可以被证明的
  \item \textbf{负向不完备性}:不能构造证明不等于证明了无效性
  \item \textbf{非能行性后果}:构造证明的失败可能是主观或客观原因
  \item \textbf{方法论需求}:需要积极的方法来证明无效性
  \end{itemize}
\item \textbf{真值指派方法的逻辑原理}:
  \begin{itemize}
  \item \textbf{反例原理}:只需找到一个前提真结论假的情况
  \item \textbf{存在性证明}:在真值表中找到一行反例即可
  \item \textbf{充分性原理}:一个反例足以证明无效性
  \item \textbf{效率原理}:避免构造完整真值表的工作量
  \end{itemize}
\item \textbf{真值指派方法的优势}:
  \begin{itemize}
  \item \textbf{效率优势}:从$2^n$行减少到寻找1行反例
  \item \textbf{目标导向}:有策略地寻找反例而非盲目检查
  \item \textbf{认知友好}:符合人类通过反例质疑论证的习惯
  \item \textbf{可扩展性}:对复杂论证仍然可行
  \end{itemize}
\item \textbf{真值指派的系统策略}:
  \begin{itemize}
  \item \textbf{基本策略}:基本陈述优先、前提约束、结论否定、策略选择原则
  \item \textbf{高级策略}:约束传播、冲突检测、试探性指派、结构分析
  \item 从最有把握的指派开始,必要时进行试错
  \end{itemize}
\item \textbf{与真值表方法的关系}:
  \begin{itemize}
  \item 真值指派构造真值表中的一行而非整个表
  \item 只需发现一行前提真结论假的情况即可证明无效
  \item 对于含多个简单陈述的论证,效率提升尤为明显
  \item 本质上是真值表方法的优化版本
  \end{itemize}
\item \textbf{方法的局限性}:
  \begin{itemize}
  \item \textbf{非机械性}:需要策略和技巧,不是完全机械的过程
  \item \textbf{试错需求}:复杂论证可能需要多次尝试
  \item \textbf{不完全性}:失败不能证明有效性
  \item \textbf{复杂度依赖}:某些结构的论证仍然困难
  \end{itemize}
\end{itemize}
}}
\end{center}
\section{不相容性}

\begin{logicbox}[title=引言]
本节探讨前提互不相容的情况及其对演绎论证有效性的影响。我们将分析为什么不相容的前提集可以推出任何结论,即使这些结论看似无关或荒谬。通过理解这一"严格蕴涵怪论",我们能更全面地把握逻辑有效性的本质以及相容性在理性思维中的重要地位。
\end{logicbox}

\subsection{从有效性定义到不相容性问题}

如果一种真值指派能使得一个论证的所有前提为真而结论为假,那么,这表明该论证是无效的。而如果一个演绎论证不是无效的,那么,它必定是有效的。因此,如果不可能对一个论证的简单分支陈述进行这种真值指派,即不可能使得它的前提为真而结论为假,那么该论证必定有效。

\begin{theorembox}[title=有效性的逻辑结构]
这一推理过程揭示了有效性概念的深层逻辑结构:

\textbf{1. 二分法原理}:每个演绎论证要么有效,要么无效,没有第三种可能性。

\textbf{2. 反证法逻辑}:证明无效性需要找到反例(前提真结论假的情况);如果找不到反例,则论证有效。

\textbf{3. 不可能性推论}:如果使前提为真而结论为假在逻辑上不可能,那么论证必定有效。

\textbf{4. 怪异推论的产生}:这种逻辑结构会导致一些直觉上令人困惑的结果。
\end{theorembox}

尽管从"有效性"的定义可以推出这一点,但它也有一个怪异的推论。考虑下面的论证,它的前提看上去与结论完全不相干:

\begin{quote}
如果飞机的引擎出了故障,它就降落在本德了。\\
如果飞机的引擎没有出故障,它就降落在克利夫兰了。\\
飞机没有降落在本德或克利夫兰。\\
因此,飞机必定降落在丹佛了。
\end{quote}

把它翻译成符号就是:

$$
\begin{aligned}
& A \supset B \\
& \sim A \supset C \\
& \sim(B \vee C) \\
& \therefore D
\end{aligned}
$$

对其简单分支陈述进行使其所有前提为真而结论为假的真值指派的努力,都注定会失败。如果我们忽略结论,把注意力放在对其简单分支陈述进行使其所有前提都为真的真值指派上,我们也一定会失败——尽管这个计划初看上去并不难以实现。

\subsection{前提不相容与有效性的深入分析}

这里之所以不能获得前提都真而结论为假的真值指派,乃因为在任何情形下使用任何真值指派都不能使前提都真。

\begin{theorembox}[title=不相容前提的逻辑特征]
由于前提是\logicterm{互不相容}的,它们具有以下逻辑特征:

\textbf{1. 不可满足性}:没有真值指派能使所有前提都真。这种情况在逻辑学中被称为"不可满足"(unsatisfiable)。

\textbf{2. 矛盾性}:前提的合取作为一个矛盾的陈述形式的代入例,乃是自相矛盾的。它在所有可能的真值指派下都为假。

\textbf{3. 真值表特征}:如果我们构造该论证的真值表,就会发现在每一行中至少有一个前提是假的。

\textbf{4. 有效性推论}:因为没有所有前提都为真这样一行,也就没有所有前提为真而结论为假这样一行。因此,该论证的真值表确立了它的有效性。
\end{theorembox}

\begin{examplebox}[title=不相容前提的数学类比]
这种情况可以用数学中的类比来理解:

考虑方程组:$x = 1$ 和 $x = 2$。这个方程组是不相容的,因为没有任何数值可以同时满足两个方程。

类似地,不相容的前提集是一组逻辑"方程",没有任何真值指派可以同时满足所有前提。正如不相容的数学方程组可以"推出"任何结论(因为从假设出发可以证明任何东西),不相容的逻辑前提也可以推出任何结论。
\end{examplebox}

下面的形式证明也可以确立它的有效性:

1.$A \supset B$\\
2.$\sim A \supset C$\\
3.$\sim(B \vee C)$\\
$\therefore D$\\
4.$\sim B \cdot \sim C$\\
5.$\sim B$\\
6.$\sim A$\\
7.$C$\\
8.$\sim C \cdot \sim B$\\
9.$\sim C$

3,De M.\\
4,Simp.\\
$1,5, \mathrm{M} . \mathrm{T}$ .\\
2,6,M.P.\\
4,Com.\\
8,Simp.

\begin{center}
\begin{tabular}{ll}
10.$C \vee D$ & 7, Add. \\
11.$D$ & 10,9, D.S. \\
\end{tabular}
\end{center}

在这个证明中, 1 至 9 行表明了前提中隐含的不相容性。这种不相容性呈现在第 7 行和第 9 行,它们分别断言了 $C$ 和 $\sim C$ 。一旦这种明显的矛盾被表示出来,根据附加律和析取三段论原理,很快就可以推出结论。

由此可见,如果一组前提不相容,这些前提就会有效地产生任何结论,而不论它们如何不相干。下面的论证更简单地表明了这一问题的精髓,其公然不相容的前提使得我们可以有效地推出一个不相干且荒谬的结论:

今天是星期天。\\
今天不是星期天。\\
因此,月亮是鲜奶酪做的。

用符号表示就是:\\
1.$S$\\
2.$\sim S$\\
$\therefore M$\\
它的有效性的形式证明十分显然:

\begin{center}
\begin{tabular}{ll}
3.$S \vee W$ & 1, Add. \\
4.$M$ & 3,2, D.S. \\
\end{tabular}
\end{center}

\subsection{不相容前提的问题}

问题出在哪里呢?如此贫乏甚至不相容的前提怎能使得它们在其中出现的论证有效?首先要注意到,如果一个论证因其前提的不相容性而有效,那么它不可能是一个合理的论证。如果前提互不相容,它们不可能都是真的。一个前提不相容的论证不能确立任何结论的真,因为它的前提本身不可能都是真的。

\subsection{严格蕴涵怪论的深入分析}

目前情形与所谓\logicterm{实质蕴涵怪论}密切相关。在讨论后者时,我们注意到(在 8.7 节),陈述形式 $\sim p \supset (p \supset q)$ 是一个重言式,其所有代入例都为真。它的自然语言表述断言的是:"如果一个陈述为假,那么它实质蕴涵任何陈述。"

\begin{theorembox}[title=严格蕴涵怪论的逻辑结构]
当下讨论所确立的是下述论证形式有效:
$$p \cdot \sim p$$
$$\therefore q$$

这种现象被称为\logicterm{严格蕴涵怪论},它具有以下特征:

\textbf{1. 普遍性}:不管其结论是什么,任何前提不相容的论证都是有效的。

\textbf{2. 可证明性}:它的有效性可以用真值表,或者用形式证明判定。

\textbf{3. 逻辑必然性}:一个有效论证的前提蕴涵它的结论,不仅仅是"实质"蕴涵意义上的,还有逻辑的或"严格"意义上的蕴涵。

\textbf{4. 不可能性条件}:在一个有效论证中,当结论为假时,其前提为真是逻辑不可能的。只要前提为真是逻辑不可能的,即使忽略结论的真假问题,这种情形也照样成立。
\end{theorembox}

\begin{examplebox}[title=严格蕴涵怪论与实质蕴涵怪论的比较]
\textbf{实质蕴涵怪论}:
\begin{itemize}
\item 形式:$\sim p \supset (p \supset q)$
\item 含义:假陈述实质蕴涵任何陈述
\item 基础:真值函项的定义
\end{itemize}

\textbf{严格蕴涵怪论}:
\begin{itemize}
\item 形式:$(p \cdot \sim p) \supset q$
\item 含义:矛盾前提严格蕴涵任何陈述
\item 基础:有效性的定义
\end{itemize}

两者的相似性使某些逻辑学者称之为"严格蕴涵怪论"。
\end{examplebox}

\subsection{怪论的哲学解释}

然而,根据逻辑学家对"有效性"的技术性定义,它似乎并不是特别怪异的。

\begin{theorembox}[title=怪论产生的根源]
所宣称的这个怪论之所以产生,主要是由于以下原因:

\textbf{1. 术语混淆}:把一个技术性术语当成日常语言中的普通术语。

\textbf{2. 直觉冲突}:技术定义与日常直觉之间的冲突。

\textbf{3. 语境差异}:逻辑语境与日常语境的不同要求。

\textbf{4. 理解偏差}:对形式逻辑目标和方法的误解。
\end{theorembox}

\subsection{相容性的重要性}

前面的讨论有助于解释为什么对相容性评价如此之高。其基本原因当然是,两个不相容的陈述不能都是真的。这一事实乃是交互询问策略的基石。在交互询问中,律师会设法使对方证人陷人自相矛盾。如果证词肯定了不能自圆其说或不相容的断言,那么证词不能都是真的,证人的可信性就被破坏…或至少被动摇。\cite{wigmore1937} 不相容性令人如此反感的另一个原因是,任何结论都可从一些被当做前提的不相容陈述逻辑地推出。不相容陈述并不是"没有意义的",它们的麻烦正好相反:其意谓太多。在蕴涵任何东西这个意义卜说,它们意谓着所有东西。如果所有东西都被断言,那么被断言的有一半肯定是假的,因为每个陈述都有一个否定。

上面的讨论附带地为我们解答了一个古老难题:一个不可抗拒的力量遇到一个不可移动的物体,会发生什么事?这个描述含有一个矛盾。要一个不可抗拒的力量遇到一个不可移动的物体,这两者都必须存在。必定存在一个不可抗拒的力量,并且也必定存在一个不可移动的物体。但如果存在不可抗拒的力量,就不会存在不可移动的物体。在此,矛盾被表述得很清楚:存在一个不可移动的物体,并且不存在一个不可移动的物体。给定这种不相容的前提,任何结论都可有效地推出。因此,对"一个不可抗拒的力量遇到一个不可移动的物体,会发生什么事?"这一问题的正确回答是"任何事"!

\subsection{不相容性与幽默}

尽管在一个论证中发现不相容性是灾难性的,但正如伟大的棒球运动员扬基队的贝拉经常被引用的评论那样,不相容性是非常有趣的。据说,

贝拉曾宣称"那个餐馆如此拥挤以致不再有人去那儿了"。在谈到他的那段长而幸福的婚姻中的伴侣时,他说:"我们长时间待在一起,即使我们不在一起时也是如此。"

这些话语很有趣,因为它们所包含的矛盾(若照字面意义理解,这些评论都是胡说),似乎没被它们的作者意识到。因此,当我们听到学生说,澳大利亚内地的气候如此不好,以致居民不再住在那儿了,我们会暗自发笑。这种漫不经心且未意识到的不相容话语,有时被称为"Irish Bull" (爱尔兰牛皮)。

从逻辑上看,不相容的命题集不可能同时为真。但人们并非总是合乎逻辑的,有时确实会说出甚至会相信两个互相矛盾的命题。这一点似乎难以置信,但逻辑领域一个非常值得信赖的权威刘易斯•卡罗尔告诉我们,《爱丽丝漫游奇境记》中的白衣女王形成了这样一个习惯,即在早餐之前相信六件不可能的事。

\begin{center}
\fbox{\parbox{0.95\textwidth}{
\textbf{本节要点}
\begin{itemize}
\item \textbf{有效性的逻辑结构}:
  \begin{itemize}
  \item \textbf{二分法原理}:每个演绎论证要么有效,要么无效
  \item \textbf{反证法逻辑}:证明无效性需要找到反例
  \item \textbf{不可能性推论}:如果找不到反例,则论证有效
  \item \textbf{怪异推论}:这种逻辑结构导致直觉上困惑的结果
  \end{itemize}
\item \textbf{不相容前提的逻辑特征}:
  \begin{itemize}
  \item \textbf{不可满足性}:没有真值指派能使所有前提都真
  \item \textbf{矛盾性}:前提合取在所有真值指派下都为假
  \item \textbf{真值表特征}:每一行中至少有一个前提为假
  \item \textbf{有效性推论}:无法找到"前提全真而结论为假"的情况
  \end{itemize}
\item \textbf{数学类比}:
  \begin{itemize}
  \item 不相容前提类似于不相容的数学方程组
  \item 正如不相容方程组可以"推出"任何结论
  \item 不相容逻辑前提也可以推出任何结论
  \end{itemize}
\item \textbf{严格蕴涵怪论的深入分析}:
  \begin{itemize}
  \item \textbf{逻辑结构}:$(p \cdot \sim p) \supset q$ 形式的有效性
  \item \textbf{四大特征}:普遍性、可证明性、逻辑必然性、不可能性条件
  \item \textbf{与实质蕴涵怪论的比较}:形式、含义、基础的对比分析
  \item 两种怪论的相似性和差异性
  \end{itemize}
\item \textbf{怪论的哲学解释}:
  \begin{itemize}
  \item \textbf{怪论产生的四大根源}:术语混淆、直觉冲突、语境差异、理解偏差
  \item 技术性定义与日常语言用法的差异
  \item 形式逻辑目标与日常推理的不同要求
  \end{itemize}
\item \textbf{相容性的重要价值}:
  \begin{itemize}
  \item 相容性是理性思维和有效论证的基础
  \item 不相容陈述的问题不是"无意义"而是"意义过多"
  \item 不相容陈述集逻辑上蕴涵一切可能的结论
  \item 在法律中作为交互询问策略的基石
  \end{itemize}
\item \textbf{不相容性的实际应用}:
  \begin{itemize}
  \item \textbf{法律领域}:质疑证词可信度的基础
  \item \textbf{哲学难题}:解答"不可抗拒力量vs不可移动物体"等古老问题
  \item \textbf{幽默效果}:产生"爱尔兰牛皮"式的有趣表达
  \item \textbf{认知科学}:说明人类思维并非总是严格遵循逻辑规则
  \end{itemize}
\item \textbf{不相容前提论证的局限性}:
  \begin{itemize}
  \item 虽然形式上有效,但不能确立任何结论的真实性
  \item 不是合理的论证,因为前提本身不可能都为真
  \item 在实际推理中应该避免不相容的前提集
  \end{itemize}
\end{itemize}
}}
\end{center}

% 第十章
\chapter{谓词逻辑}
\section{单称命题}

\begin{logicbox}[title=引言]
本节引入了\logicterm{单称命题}的概念及其符号化方法。通过分析命题的内在逻辑结构,我们将学习如何区分\logicterm{个体}与\logicterm{属性},以及如何使用\logicterm{个体常元}和\logicterm{谓词符号}来表示各种类型的单称命题,从而为分析更复杂的论证奠定基础。
\end{logicbox}

\subsection{命题逻辑的局限性与谓词逻辑的必要性}

前两章的逻辑技术使得我们可以对某种类型的\logicemph{有效}论证和\logicwarn{无效}论证进行区分。该类型的论证可以粗略地刻画为:其\logicemph{有效性}仅取决于简单陈述通过\logicterm{真值函项}结合成复合陈述的方式。

\begin{theorembox}[title=命题逻辑的适用范围与局限]
命题逻辑具有以下特征和局限性:

\textbf{1. 适用范围}:命题逻辑能够有效处理那些有效性完全依赖于逻辑联结词(合取、析取、条件、双条件、否定)的论证。

\textbf{2. 分析单位}:命题逻辑将简单陈述视为不可分析的原子单位,不考虑其内部结构。

\textbf{3. 主要局限}:对于那些有效性依赖于陈述内部结构的论证,命题逻辑无法提供充分的分析工具。

\textbf{4. 历史意义}:这种局限性的发现推动了现代逻辑学向更精细的分析方向发展,催生了谓词逻辑的诞生。
\end{theorembox}

然而,还有另外一些类型的论证,前两章的\logicemph{有效性}标准对它们不够用。一个明显\logicemph{有效的}不同类型论证的例子是:

\begin{examplebox}[title=经典三段论实例]
所有人都是有死的。

苏格拉底是人。

因此,苏格拉底是有死的。
\end{examplebox}

如果把前面介绍的评估方法运用到这个论证上,我们可以将之符号化为:
$$
\begin{aligned}
& A \\
& H \\
& \therefore M
\end{aligned}
$$

在这种符号式中,该论证显然\logicwarn{无效}。

\begin{theorembox}[title=命题逻辑分析的失败与原因]
这个例子清楚地展示了命题逻辑的根本局限性:

\textbf{1. 分析失败}:到目前为止所介绍的符号逻辑技术不能直接运用到这种新型论证上。

\textbf{2. 有效性来源}:该论证的\logicemph{有效性}并不取决于简单陈述的复合方式,因为在该论证中没有出现任何复合陈述。

\textbf{3. 真正原因}:毋宁说它的\logicemph{有效性}取决于所涉及非复合陈述的内在逻辑结构。

\textbf{4. 解决方案}:要阐明检验这种新型论证\logicemph{有效性}的方法,就必须依据它们的内在逻辑结构,设计出一些描述和符号化非复合陈述的技术。
\end{theorembox}

\begin{examplebox}[title=谓词逻辑的历史必然性]
这种分析需求在逻辑学史上具有重要意义:

\textbf{亚里士多德的贡献}:早在古希腊时期,亚里士多德就认识到了这类论证的重要性,并发展了三段论理论来处理它们。

\textbf{现代发展}:19世纪末20世纪初,弗雷格、罗素等逻辑学家发展了现代谓词逻辑,为这类论证提供了更精确的分析工具。

\textbf{理论意义}:这标志着逻辑学从关注陈述间关系转向关注陈述内部结构的重大转变。
\end{examplebox}

\subsection{单称命题的结构}

\begin{theorembox}[title=单称命题的定义]
一种最简单的非复合陈述的例示是前述论证中的第二个前提:"苏格拉底是人。"这种类型的陈述传统上叫做\logicterm{单称命题}。一个(肯定的)\logicterm{单称命题}断言的是,一个特定\logicterm{个体}具有某种特定\logicterm{属性}。在上述例子中,日常语法和传统逻辑都一致地把"苏格拉底"划为主项,把"人"划为谓项。主项指称某特定\logicterm{个体},谓项指谓该\logicterm{个体}据称所具有的某种\logicterm{属性}。
\end{theorembox}

显然,同一主项可以在不同的\logicterm{单称命题}中出现。因此,在下述每个命题中,我们都以词项"苏格拉底"做主项:"苏格拉底是有死的","苏格拉底是胖的","苏格拉底是聪明的"和"苏格拉底是漂亮的"。当然,有些是\logicemph{真的}(第一和第三个),有些是\logicwarn{假的}(第二和第四个)。\cite{quine1953} 同一谓项显然也可以出现在不同的\logicterm{单称命题}中。因此在下述每个命题中,我们都以词项"人"做谓项:"亚里士多德是人","巴西是人","芝加哥是人"和"奥基夫是人"。当然,有些是\logicemph{真的}(第一和第四个),有些是\logicwarn{假的}(第二和第三个)。

\begin{theorembox}[title=个体概念的哲学分析]
从前所述,我们应该清楚语词"\logicterm{个体}"不仅用来指人,还可以指事物,譬如,国家、城市,实际上可以指谓像是人或有死的这样能被有意义地断言为其\logicterm{属性}的任何事物。

\textbf{个体的本体论地位}:
\begin{itemize}
\item \textbf{具体个体}:如苏格拉底、这张桌子、巴黎等具有时空位置的实体
\item \textbf{抽象个体}:如数字、概念、命题等不具有时空位置但可被指称的对象
\item \textbf{虚构个体}:如哈姆雷特、独角兽等在现实中不存在但可被谈论的对象
\end{itemize}

\textbf{个体识别的标准}:
\begin{itemize}
\item \textbf{同一性原则}:每个个体都与自身同一,与其他个体不同
\item \textbf{不可分辨者同一性}:具有完全相同属性的个体是同一个体
\item \textbf{指称的唯一性}:每个个体常元在给定语境中指称唯一确定的个体
\end{itemize}
\end{theorembox}

\begin{examplebox}[title=谓项的语法灵活性]
前面所举的例子中,有些谓项是形容词。从日常语法的观点看,形容词与名词的区分是相当重要的。但在本章中这种区别并不重要,我们并不区分"苏格拉底是有死的"和"苏格拉底是有死者",或"苏格拉底是聪明的"和"苏格拉底是一个聪明的人"。

\textbf{谓项的多样表现形式}:
\begin{itemize}
\item \textbf{形容词形式}:有死的、聪明的、漂亮的
\item \textbf{名词形式}:有死者、智者、美人
\item \textbf{动词形式}:写作、思考、存在
\item \textbf{介词短语}:在雅典、属于希腊、来自古代
\end{itemize}

一个谓项可以是一个形容词或者是一个名词,甚或是一个动词。如在"亚里士多德写作"中,它有时可以被表述为"亚里士多德是一个写作者"。这种语法灵活性表明,逻辑结构比表面语法更为根本。
\end{examplebox}

\subsection{单称命题的符号化}

\begin{theorembox}[title=符号化系统]
假定我们能区分开具有某\logicterm{属性}的\logicterm{个体}和它们所具有的\logicterm{属性},我们引进并使用两种不同的符号来指称它们。在随后的讨论中,我们将用从 $a$ 到 $w$的小写字母来指谓\logicterm{个体}。这些符号是\logicterm{个体常元}。在它们出现的任何特定上下文,每个在该整个上下文中都指称一个特定的\logicterm{个体}。用它(他,或她)的名称的第一个字母指称一个\logicterm{个体},通常是很方便的。因此在当前的上下文中,我们应分别用字母 $s 、 a 、 b 、 c 、 o$ 指称苏格拉底、亚里士多德、巴西、芝加哥和奥基夫。我们将用大写字母来符号化\logicterm{属性},在此使用同样的指导原则是很便利的。因此,我们用字母 $H 、 M 、 F 、 W 、 B$ 分别符号化\logicterm{属性}是人、有死的、胖的、聪明的、漂亮的。
\end{theorembox}

有两组符号,一组是\logicterm{个体}的符号,另一组是\logicterm{个体属性}的符号。

\begin{theorembox}[title=符号化约定的理论基础]
我们采取这样一个约定:把\logicterm{属性}符号直接写在\logicterm{个体}符号的左边,表征被命名的\logicterm{个体}具有规定的\logicterm{属性}这样一个\logicterm{单称命题}。

\textbf{符号化的语法规则}:
\begin{itemize}
\item \textbf{基本形式}:$Px$表示"$x$具有属性$P$"
\item \textbf{语序约定}:谓词符号在前,个体符号在后
\item \textbf{括号使用}:可写成$P(x)$以强调函数关系
\item \textbf{多元谓词}:$R(x,y)$表示"$x$与$y$具有关系$R$"
\end{itemize}

\textbf{约定的理论意义}:
\begin{itemize}
\item 体现了谓词逻辑的函数-论元结构
\item 与数学函数记号保持一致性
\item 便于处理复杂的逻辑表达式
\item 为计算机处理提供了标准格式
\end{itemize}
\end{theorembox}

于是,\logicterm{单称命题}"苏格拉底是人"可以符号化为 $H s$ 。上面提到的涉及谓项"人"的其他一些\logicterm{单称命题},分别可以符号化为 $H a 、 H b 、 H c$ 和 $H o$ 。

\begin{examplebox}[title=模式识别与变元引入]
我们注意到它们都有某种共同的模式,即它们不是被符号化为 $H$ 自身,而是 $H$ —。在此,"一"表示在谓项符号的右边有另一个符号即\logicterm{个体}符号出现。

\textbf{变元的引入}:习惯上用字母 $x$(这是可以的,因为我们只用从 $a$ 到 $w$ 的字母做\logicterm{个体常元})而不是用破折号("——")作替代标示。

\textbf{模式表示}:我们用 $H x$[有时写成 $H(x)$ ]来符号化所有以 "是人"作为\logicterm{个体属性}的\logicterm{单称命题}的共同模式。

\textbf{变元的本质}:被称做\logicterm{个体变元}的字母 $\boldsymbol{x}$只是一个位置标示,用来指示从 $a$ 到 $w$ 的各个字母——\logicterm{个体常元}——可以填入以便产生\logicterm{单称命题}的位置。

这种抽象化过程体现了从具体到一般的重要逻辑思维方式。
\end{examplebox}

\subsection{命题函项}

\begin{theorembox}[title=命题函项的精确定义]
各种\logicterm{单称命题} $H a 、 H b 、 H c$ 和 $H d$ 是或\logicemph{真}或\logicwarn{假}的;但由于 $H x$ 根本不是陈述或命题,它既不\logicemph{真}也不\logicwarn{假}。

表达式 $H x$ 是一个\logicterm{命题函项},它可以被定义成这样一个表达式:
\begin{enumerate}
\item 含有\logicterm{个体变元}
\item 当一个\logicterm{个体常元}代入\logicterm{个体变元}时,它就变成一个陈述
\end{enumerate}

\textbf{命题函项的本质特征}:
\begin{itemize}
\item \textbf{开放性}:含有自由变元,因此不具有确定的真值
\item \textbf{模板性}:为生成具体命题提供结构模板
\item \textbf{抽象性}:表达了一类命题的共同逻辑形式
\item \textbf{函数性}:从个体到真值的映射关系
\end{itemize}
\end{theorembox}

\begin{examplebox}[title=命题函项的数学类比]
命题函项与数学中的函数概念有深刻的类比关系:

\textbf{数学函数}:$f(x) = x^2 + 1$
\begin{itemize}
\item $x$是变元,$f(x)$不是数值
\item 代入具体值:$f(3) = 10$得到具体数值
\item 函数定义了从数到数的映射
\end{itemize}

\textbf{命题函项}:$H(x)$ = "$x$是人"
\begin{itemize}
\item $x$是个体变元,$H(x)$不是命题
\item 代入具体个体:$H(s)$ = "苏格拉底是人"得到具体命题
\item 命题函项定义了从个体到真值的映射
\end{itemize}

这种类比揭示了逻辑与数学的深层联系。
\end{examplebox}

\logicterm{个体常元}被认为是\logicterm{个体}的专名。任何\logicterm{单称命题}都是一个\logicterm{命题函项}的代入例,是用\logicterm{个体常元}代人该\logicterm{命题函项}中的\logicterm{个体变元}所产生的结果。\cite{reichenbach1947}

\begin{theorembox}[title=简单谓述的特征]
一般说来,一个\logicterm{命题函项}有\logicemph{真}代人例和\logicwarn{假}代入例。到目前为止所讨论的\logicterm{命题函项}——Hx、Mx、Fx、Bx 和 $W x$ ——都是这种类型。

为了把它们与后面几节将介绍的更复杂的\logicterm{命题函项}区分开,我们把这些\logicterm{命题函项}叫\logicterm{简单谓述}。

\textbf{简单谓述的定义}:一个\logicterm{简单谓述}是一个有一些\logicemph{真}代人例和\logicwarn{假}代入例的\logicterm{命题函项},并且每个代人例都是一个\logicterm{单称肯定命题}。

\textbf{简单谓述的重要性}:
\begin{itemize}
\item 构成谓词逻辑的基本构建块
\item 为更复杂的逻辑结构提供基础
\item 体现了属性归属的基本逻辑关系
\item 连接语言表达与逻辑形式的桥梁
\end{itemize}
\end{theorembox}

\begin{center}
\fbox{\parbox{0.95\textwidth}{
\textbf{本节要点}
\begin{itemize}
\item \textbf{命题逻辑的局限性与谓词逻辑的必要性}:
  \begin{itemize}
  \item 命题逻辑只能处理依赖于逻辑联结词的论证有效性
  \item 对于依赖于陈述内部结构的论证,命题逻辑无法提供充分分析
  \item 这种局限性推动了现代逻辑学向谓词逻辑的发展
  \item 亚里士多德的三段论理论是早期处理此类论证的尝试
  \end{itemize}
\item \textbf{单称命题的结构与特征}:
  \begin{itemize}
  \item 断言特定\logicterm{个体}具有某种\logicterm{属性}的命题
  \item 由主项(指称个体)和谓项(表示属性)构成
  \item \logicemph{有效性}取决于命题的内在逻辑结构而非复合方式
  \item 个体概念包括具体个体、抽象个体和虚构个体
  \end{itemize}
\item \textbf{符号化系统的理论基础}:
  \begin{itemize}
  \item \logicterm{个体常元}:小写字母a到w,指称特定\logicterm{个体}
  \item \logicterm{属性}符号:大写字母,表示\logicterm{个体}可能具有的\logicterm{属性}
  \item 符号化约定:谓词符号在前,个体符号在后(如$Hs$)
  \item 体现函数-论元结构,与数学函数记号保持一致性
  \end{itemize}
\item \textbf{个体概念的哲学分析}:
  \begin{itemize}
  \item \textbf{本体论地位}:具体个体、抽象个体、虚构个体的区分
  \item \textbf{识别标准}:同一性原则、不可分辨者同一性、指称唯一性
  \item 谓项的语法灵活性:形容词、名词、动词、介词短语等多种形式
  \item 逻辑结构比表面语法更为根本
  \end{itemize}
\item \textbf{命题函项的精确定义}:
  \begin{itemize}
  \item 含有\logicterm{个体变元}且代入常元后变成陈述的表达式
  \item \textbf{四大本质特征}:开放性、模板性、抽象性、函数性
  \item 与数学函数的深刻类比:从个体到真值的映射关系
  \item 不具有确定真值,但为生成具体命题提供结构模板
  \end{itemize}
\item \textbf{简单谓述的特征与重要性}:
  \begin{itemize}
  \item 有真代入例和假代入例的命题函项
  \item 每个代入例都是单称肯定命题
  \item \textbf{四大重要性}:基本构建块、提供基础、体现逻辑关系、连接桥梁
  \item 为更复杂的逻辑结构奠定基础
  \end{itemize}
\end{itemize}
}}
\end{center}
\section{量化}

\begin{logicbox}[title=引言]
本节讨论如何通过量化将命题函项转变为命题。我们将介绍全称量词和存在量词的概念及符号表示,分析它们之间的逻辑关系,并探讨如何使用量化符号来表达不同类型的普遍命题,从而为理解更复杂的逻辑结构奠定基础。
\end{logicbox}

\subsection{从命题函项到命题的两种路径}

用个体常元代人个体变元,并不是从命题函项获得命题的唯一方式。通过概括或量化程序也可以得到命题。

\begin{theorembox}[title=生成命题的两种基本方法]
从命题函项生成命题有两种根本不同的方法:

\textbf{1. 列举方法(Instantiation)}:
\begin{itemize}
\item 用个体常元代入个体变元
\item 产生关于特定个体的单称命题
\item 例如:从$H(x)$得到$H(s)$(苏格拉底是人)
\item 这种方法产生具体的、特殊的陈述
\end{itemize}

\textbf{2. 概括方法(Generalization)}:
\begin{itemize}
\item 通过量词对个体变元进行约束
\item 产生关于个体集合或类别的普遍命题
\item 例如:从$H(x)$得到$(x)H(x)$(所有事物都是人)
\item 这种方法产生一般的、普遍的陈述
\end{itemize}

这两种方法体现了逻辑思维中从特殊到一般、从一般到特殊的双向运动。
\end{theorembox}

\begin{examplebox}[title=普遍命题的特征]
谓项通常不仅仅出现在单称命题中。例如,命题"每个事物是有死的"和"有些事物是漂亮的"含有谓项,但不是单称命题,因为它们不含有任何特定个体的名称。

\textbf{普遍命题的重要特征}:
\begin{itemize}
\item \textbf{非特指性}:不特别指称任何特定个体
\item \textbf{范围性}:涉及整个个体域或其子集
\item \textbf{模式性}:表达个体之间的一般模式或规律
\item \textbf{预测性}:可以对未知个体做出预测
\end{itemize}

相反,作为普遍命题,它们不特别指称任何特定个体,而是对整个个体域做出断言。
\end{examplebox}

第一个命题可以用各种不同的逻辑等价的方式表示:或者表示为"所有事物都是有死的",或者表示为:

给定不管任何个体事物,它都是有死的。

在后一种表述中,语词"它"是一个关系代词,回指该陈述中前面的语词 "事物"。用字母 $x$ ,即个体变元,代替代词"它"及其先行词,我们可以把第一个普遍命题重写为:

给定任何 $x, x$ 是有死的。

或者,用前一节所介绍的符号,我们可以写成:

给定任何 $x, M x$ 。

尽管命题函项 $M x$ 不是一个命题,但我们这里有了一个含有它的表述式,而这个表述式是命题。

\begin{theorembox}[title=全称量词的深入分析]
短语"给定任何 $x$"习惯上用符号"$(x)$"表示,称为\logicterm{全称量词}。

\textbf{全称量词的本质特征}:
\begin{itemize}
\item \textbf{约束功能}:将自由变元转换为约束变元
\item \textbf{普遍性}:表达对整个个体域的断言
\item \textbf{逻辑强度}:全称陈述比存在陈述具有更强的逻辑承诺
\item \textbf{可证伪性}:一个反例就足以证伪全称陈述
\end{itemize}

\textbf{全称量词的哲学意义}:
\begin{itemize}
\item 体现了人类认识从个别到一般的抽象能力
\item 是科学定律和数学定理的逻辑基础
\item 反映了理性思维对普遍性的追求
\item 连接经验观察与理论概括的桥梁
\end{itemize}
\end{theorembox}

上述第一个普遍命题可以完全符号化为:
$$(x)Mx$$

这个符号表达式读作:"对于所有的$x$,$x$是有死的"或"每个$x$都是有死的"。

\subsection{存在量化与全称量化}

第二个普遍命题,即"有些事物是漂亮的",也可以表达成:

至少存在这样一个事物,它是漂亮的。

在后一个表述中,语词"它"也是一个关系代词,回指语词"事物"。用个体变元 $x$ 代替代词"它"及其先行词,我们可以把第二个普遍命题重写为:

至少存在这样一个 $x, x$ 是漂亮的。

或者,我们可以用给定符号把它写成:

至少存在这样一个 $x, B x$ 。

同样,尽管 $B x$ 是一个命题函项而不是命题,但我们这里又有一个含有它的表述式,这个表述式是命题。

\begin{theorembox}[title=存在量词的深入分析]
短语"至少存在这样一个 $x$"习惯上用符号"$(\exists x)$"表示,称为\logicterm{存在量词}。

\textbf{存在量词的本质特征}:
\begin{itemize}
\item \textbf{存在承诺}:断言至少有一个个体满足给定条件
\item \textbf{弱逻辑强度}:比全称陈述更容易满足
\item \textbf{可验证性}:找到一个正例就足以验证存在陈述
\item \textbf{开放性}:不指定具体是哪个个体满足条件
\end{itemize}

\textbf{存在量词的认识论意义}:
\begin{itemize}
\item 反映了人类对世界多样性的认识
\item 是发现和探索的逻辑基础
\item 体现了可能性思维的重要性
\item 为假设和猜想提供逻辑框架
\end{itemize}
\end{theorembox}

第二个普遍命题可以完全符号化为:
$$(\exists x) B x$$

这个符号表达式读作:"存在一个$x$使得$x$是漂亮的"或"有些$x$是漂亮的"。

\begin{examplebox}[title=两种生成方法的总结]
于是我们看到,命题可以用两种方法从命题函项生成:

\textbf{1. 列举方法}:通过用个体常元代入个体变元
\begin{itemize}
\item 产生具体的单称命题
\item 例如:$H(x) \rightarrow H(s)$
\item 适用于关于特定个体的断言
\end{itemize}

\textbf{2. 概括方法}:在命题函项前面放一个量词
\begin{itemize}
\item 产生普遍的量化命题
\item 例如:$H(x) \rightarrow (x)H(x)$ 或 $(\exists x)H(x)$
\item 适用于关于个体类别的断言
\end{itemize}

这两种方法构成了谓词逻辑的基本操作,使我们能够在特殊与一般之间自由转换。
\end{examplebox}

\subsection{量化命题的真值条件}

\begin{theorembox}[title=量化命题真值条件的精确定义]
现在请考虑量化命题的真值条件:

\textbf{全称量化的真值条件}:
一个命题函项的全称量化式 $(x) M x$ 为真,当且仅当,它的所有代入例都为真;这正是普遍性的意义之所在。

\textbf{存在量化的真值条件}:
很显然,一个命题函项的存在量化式 $(\exists x) M x$ 为真,当且仅当,它至少有一个真代入例。

\textbf{个体域的假定}:
我们假定(没人会否认这一点)至少存在一个个体。这是一个非常弱但重要的假定,被称为\logicterm{存在假定}。
\end{theorembox}

\begin{examplebox}[title=量词间的逻辑关系]
在存在假定下,每个命题函项必定至少有一个代入例,这个实例或真或假。

\textbf{重要的逻辑关系}:
如果一个命题函项的全称量化式为真,那么它的存在量化式也必定为真。

\textbf{形式表示}:$(x)Mx \rightarrow (\exists x)Mx$

\textbf{直观理解}:如果每个 $x$ 都是 $M$ ,那么,如果至少存在一个事物,则这个事物是 $M$ 。

\textbf{哲学意义}:这个关系体现了从普遍到特殊的逻辑推理,是演绎推理的基础。但反向关系不成立:从存在到全称的推理是归纳性的,不具有逻辑必然性。
\end{examplebox}

\subsection{否定与量化}

到此时为止,只举了单称肯定命题作为命题函项的代入例。

\begin{theorembox}[title=否定在命题函项中的作用]
$M x$($x$ 是有死的)是一个命题函项。$M s$ 是它的一个实例,是一个单称肯定命题,即"苏格拉底是有死的"。

\textbf{否定的引入}:
但并非所有命题都是肯定的。一个人可以否认苏格拉底是有死的,即 $\sim M s$ ,"苏格拉底不是有死的"。

\textbf{否定命题函项}:
如果 $M s$ 是 $M x$ 的一个代入例,那么,$\sim Ms$可以看成是命题函项 $\sim M x$ 的一个代入例。

\textbf{概念扩展}:
因此,我们可以超出前一节所介绍的简单谓述,把我们的命题函项概念扩大到能包括否定符"$\sim$"。

\textbf{否定的重要性}:
\begin{itemize}
\item 增强了表达能力,可以表达否定性质
\item 为量词的相互转换提供了基础
\item 使逻辑系统更加完整和对称
\item 反映了思维的批判性和辨析能力
\end{itemize}
\end{theorembox}

如下所示,使用否定符可以丰富我们对量化的理解。从下述普遍命题出发:

没有任何事物是完美的。

我们可以把它解释为:

每个事物都是不完美的。

它又可以写成:

给定不管任何个体事物,它不是完美的。

它可以改写成:

给定任何 $x, x$ 不是完美的。

如果用 $P$ 符号化属性"是完美的",用刚才给出的符号(量词和否定符),我们可以把这个命题("没有任何事物是完美的")表示为 $(x) \sim P x$ 。

\subsection{量词间的逻辑关系深入分析}

现在我们可以列出并举例说明全称量化和存在量化之间的一系列重要关系。

\begin{theorembox}[title=量词对偶关系的第一组]
\textbf{第一个重要关系}:

(全称)普遍命题"每个事物都是有死的",被(存在)普遍命题"有些事物不是有死的"否定。

\textbf{符号表示}:$(x)Mx$ 被 $(\exists x)\sim Mx$ 否定。

\textbf{逻辑分析}:
\begin{itemize}
\item 这两个命题是矛盾关系,不能同时为真,也不能同时为假
\item 一个为真当且仅当另一个为假
\item 这体现了全称断言的可证伪性:一个反例就足以推翻全称命题
\end{itemize}

因为它们每个都是另一个的否定,下述双条件陈述是必然真的、逻辑真的:

$$\sim(x) M x \stackrel{\mathrm{T}}{=} (\exists x) \sim M x$$

\textbf{直观理解}:"并非每个事物都是有死的"等价于"存在某个事物不是有死的"。
\end{theorembox}

\begin{theorembox}[title=量词对偶关系的第二组]
\textbf{第二个重要关系}:

"每个事物都是有死的"正好表示了"不存在任何不是有死的事物"所表示的东西。

$$
(x) M x \stackrel{\mathrm{T}}{=} \sim(\exists x) \sim M x
$$

\textbf{直观理解}:"所有事物都是有死的"等价于"不存在不是有死的事物"。这体现了全称肯定与存在否定的等价关系。
\end{theorembox}

\begin{theorembox}[title=量词对偶关系的第三组]
\textbf{第三个重要关系}:

很清楚,(全称)普遍命题"没有任何事物是有死的",被(存在)普遍命题"有些事物是有死的"否定。用符号我们可以说 $(x) \sim M x$被$(\exists x) M x$ 否定。

$$
\sim(x) \sim M x \stackrel{\mathrm{T}}{=} (\exists x) M x
$$

\textbf{直观理解}:"并非没有任何事物是有死的"等价于"存在某个事物是有死的"。
\end{theorembox}

\begin{theorembox}[title=量词对偶关系的第四组]
\textbf{第四个重要关系}:

"每个事物都不是有死的"正好表示了"不存在任何有死的事物"所表示的东西。

$$
(x) \sim M x \stackrel{\mathrm{T}}{=} \sim(\exists x) M x
$$

\textbf{直观理解}:"所有事物都不是有死的"等价于"不存在有死的事物"。这体现了全称否定与存在否定的等价关系。
\end{theorembox}

\begin{theorembox}[title=量词对偶关系的一般化]
这四个逻辑真的双条件陈述阐明了全称量词和存在量词的相互关系。

\textbf{重要应用}:任何一个否定符在量词之前的命题,(利用这些逻辑真的双条件陈述)我们都可以用另一个与其逻辑等价但量词前面没有否定符的命题替换之。

\textbf{一般化形式}:现在以符号 $\phi$(希腊字母 phi)替换例示谓词 $M$(有死的),$\phi$ 代表任何一个简单谓词,我们立即可列出下面这四个双条件陈述式:

$$
\begin{aligned}
& {\left[(x) \phi_{x}\right] \stackrel{\mathrm{T}}{=}\left[\sim(\exists x) \sim \phi_{x}\right]} \\
& {\left[(\exists x) \phi_{x}\right] \stackrel{\mathrm{T}}{=}\left[\sim(x) \sim \phi_{x}\right]} \\
& {\left[(x) \sim \phi_{x}\right] \stackrel{\mathrm{T}}{=}\left[\sim(\exists x) \phi_{x}\right]} \\
& {\left[(\exists x) \sim \phi_{x}\right] \stackrel{\mathrm{T}}{=}\left[\sim(x) \phi_{x}\right]}
\end{aligned}
$$

\textbf{理论意义}:
\begin{itemize}
\item 这些等价关系被称为\logicterm{德摩根定律的量词版本}
\item 它们表明全称量词和存在量词是对偶的
\item 任何量化陈述都可以用其对偶量词加否定来表达
\item 这种对偶性是谓词逻辑的基本特征之一
\end{itemize}
\end{theorembox}

\subsection{量词逻辑方阵}

全称量化和存在量化之间的一般关系,可以用图 10-1 中的方阵进行

更图示化的描述。\\
\includegraphics[width=\textwidth]{images/2025_05_15_6a28331d5e7c993ad07ag-463.jpg}

图10—1\\
继续假定至少存在一个个体,就该方阵我们可以说:\\
1.顶端的两个命题是反对关系;就是说,它们可以同时为假,但不能同时为真。

2.底端的两个命题是下反对关系;就是说,它们可以同时为真,但不能同时为假。

3.对角线相反两端的命题是矛盾关系;它们中一个为真,则另一个必定为假。

4.在方阵的每侧,下面命题的真被它正上方命题的真所蕴涵。

\begin{center}
\fbox{\parbox{0.95\textwidth}{
\textbf{本节要点}
\begin{itemize}
\item \textbf{从命题函项到命题的两种路径}:
  \begin{itemize}
  \item \textbf{列举方法}:用个体常元代入变元,产生关于特定个体的单称命题
  \item \textbf{概括方法}:通过量词约束变元,产生关于个体集合的普遍命题
  \item 体现了逻辑思维中从特殊到一般、从一般到特殊的双向运动
  \item 普遍命题具有非特指性、范围性、模式性、预测性等特征
  \end{itemize}
\item \textbf{全称量词的深入分析}:
  \begin{itemize}
  \item 符号$(x)$表示"对所有x都成立"
  \item \textbf{四大本质特征}:约束功能、普遍性、逻辑强度、可证伪性
  \item \textbf{哲学意义}:体现抽象能力、科学定律基础、理性追求、理论概括
  \item 一个反例就足以证伪全称陈述
  \end{itemize}
\item \textbf{存在量词的深入分析}:
  \begin{itemize}
  \item 符号$(\exists x)$表示"至少存在一个x使...成立"
  \item \textbf{四大本质特征}:存在承诺、弱逻辑强度、可验证性、开放性
  \item \textbf{认识论意义}:反映多样性认识、发现基础、可能性思维、假设框架
  \item 找到一个正例就足以验证存在陈述
  \end{itemize}
\item \textbf{量化命题的真值条件}:
  \begin{itemize}
  \item 全称量化为真当且仅当所有代入例都为真
  \item 存在量化为真当且仅当至少有一个代入例为真
  \item \textbf{存在假定}:至少存在一个个体的弱假定
  \item \textbf{重要关系}:$(x)Mx \rightarrow (\exists x)Mx$(从普遍到特殊)
  \end{itemize}
\item \textbf{否定在命题函项中的作用}:
  \begin{itemize}
  \item 扩展了命题函项概念,包括否定符"$\sim$"
  \item \textbf{四大重要性}:增强表达能力、量词转换基础、系统完整性、批判思维
  \item 为量词间的相互转换提供了基础
  \item 使逻辑系统更加完整和对称
  \end{itemize}
\item \textbf{量词对偶关系的深入分析}:
  \begin{itemize}
  \item \textbf{四组基本关系}:$\sim(x)\phi x = (\exists x)\sim\phi x$等
  \item 被称为\logicterm{德摩根定律的量词版本}
  \item 全称量词和存在量词是对偶的
  \item 任何量化陈述都可以用其对偶量词加否定来表达
  \end{itemize}
\item \textbf{量词逻辑方阵}:
  \begin{itemize}
  \item 描述全称肯定、全称否定、特称肯定、特称否定命题间的关系
  \item 包含反对关系、下反对关系和矛盾关系
  \item 展示量化命题间的逻辑蕴涵关系
  \item 为传统逻辑的现代化提供了精确的形式化工具
  \end{itemize}
\end{itemize}
}}
\end{center}
\section{传统主——谓命题}

\begin{logicbox}[title=引言]
本节探讨传统逻辑中的四种主谓命题形式及其符号化表示。通过分析全称肯定、全称否定、特称肯定和特称否定命题的结构,我们将学习如何用量词和命题函项准确表达它们,并理解这些命题形式之间的逻辑关系,从而增强对自然语言与符号逻辑之间转换的理解。
\end{logicbox}

\subsection{传统逻辑的四种基本命题形式}

运用存在和全称量词,以及根据对图 10-1 中对当方阵的理解,我们现在开始分析(并且在推理中准确地使用)以下四种为传统逻辑研究所注重的普遍命题。

\begin{theorembox}[title=四种基本命题形式的历史背景]
这四种命题形式构成了亚里士多德逻辑学的核心,被称为\logicterm{直言命题}(categorical propositions)。它们在西方逻辑传统中占据了两千多年的主导地位,直到现代符号逻辑的兴起。

\textbf{历史意义}:
\begin{itemize}
\item 亚里士多德在《工具论》中首次系统化这四种形式
\item 中世纪经院哲学将其发展为精密的逻辑体系
\item 现代逻辑学通过量词理论为其提供了精确的形式化
\item 至今仍是逻辑教学和日常推理的重要工具
\end{itemize}
\end{theorembox}

这四种命题的标准例子如下:

\begin{center}
\begin{tabular}{ll}
所有人是有死的。 & [全称肯定:A] \\
所有人都不是有死的。 & [全称否定:E] \\
有些人是有死的。 & [特称肯定:I] \\
有些人不是有死的。 & [特称否定:O] \\
\end{tabular}
\end{center}

\begin{examplebox}[title=命题类型标记的词源学]
每种命题通常由其字母来指称:两种肯定命题用 A 和 I(来自拉丁文 affirmo,我肯定);两种否定命题用 E 和 O (来自拉丁文 nego,我否认)。\cite{peirce1883}

\textbf{记忆口诀的历史}:
\begin{itemize}
\item 中世纪逻辑学家创造了记忆口诀:"AffIrmo, nEgO"
\item A和I取自"affirmo"的元音,表示肯定
\item E和O取自"nego"的元音,表示否定
\item 这种标记法一直沿用至今,成为逻辑学的标准术语
\end{itemize}
\end{examplebox}

\subsection{A型命题的符号化}

用量词符号化这些命题,使得我们进一步扩大了命题函项概念。

\begin{theorembox}[title=A型命题的逐步符号化过程]
首先来看 \textbf{A命题}"所有人是有死的",我们从下述命题开始逐次解释:

\textbf{第一步:自然语言分析}
给定不管任何事物,如果它是人,它是有死的。

\textbf{第二步:代词消解}
其中关系代词"它"的两次出现显然是回指它们共同的先行词"事物"。与上节的前部分一样,因为它们有同样的(不确定的)指称,从而都能用字母"$x$"替换。于是该命题可改写成:

给定任何 $x$ ,如果 $x$ 是人,那么 $x$ 是有死的。

\textbf{第三步:逻辑联结词引入}
现在,用先前引入的"如果-那么"的符号,可以把前一个命题改写成:

给定任何 $x, x$ 是人 $\supset x$ 是有死的。

\textbf{第四步:完全符号化}
最后,用我们已掌握的命题函项符号和量词,原来的 A 命题可表示为:

$$(x)(H x \supset M x)$$

\textbf{符号化的理论意义}:
\begin{itemize}
\item 揭示了全称命题的条件结构
\item 体现了"如果-那么"的逻辑本质
\item 为处理空类问题提供了基础
\item 连接自然语言与形式逻辑的桥梁
\end{itemize}
\end{theorembox}

\begin{theorembox}[title=复合命题函项的结构分析]
在我们的符号翻译中, A 命题是以一种新的命题函项的全称量化形式出现的。

\textbf{条件命题函项}:
表述式 $H x \supset M x$ 是一个命题函项,它既没有单称肯定命题又没有单称否定命题作为其代入例,而是以条件陈述作为代入例,这些条件陈述的前件和后件是具有同样主项的单称命题。命题函项 $H x \supset M x$ 的代入例有条件陈述 $H a \supset M a 、 H b \supset M b 、 H c \supset M c 、 H d \supset M d$ 等等。

\textbf{其他复合命题函项类型}:
\begin{itemize}
\item \textbf{合取命题函项}:$H x \cdot M x$ 的代入例为 $H a \cdot M a 、 H b \cdot M b 、 H c \cdot M c 、 H d \cdot M d$ 等
\item \textbf{析取命题函项}:$W x \vee B x$ 的代入例为 $W a \vee B a$ 和 $W b \vee B b$ 等
\item \textbf{双条件命题函项}:$P x \equiv Q x$ 的代入例为 $P a \equiv Q a$ 等
\item \textbf{否定命题函项}:$\sim R x$ 的代入例为 $\sim R a$ 等
\end{itemize}

\textbf{一般原理}:
实际上,任何以具有相同主项的单称命题为分支陈述的真值函项复合命题,都可以看做由某些或所有真值函项联结词(圆点号、楔劈号、马蹄号、三杠等值号和波浪号)加之简单谓词($A x 、 B x 、 C x 、 D x \cdots$ )所构成的命题函项的代入例。
\end{theorembox}

\begin{examplebox}[title=量词辖域的重要性]
在把 A 命题翻译成$(x)(H x \supset M x)$时,圆括号充当标点符号,用以表明全称量词$(x)$"作用于"整个(复合)命题函项 $H x \supset M x$ ,或命题函项 $H x \supset M x$"作为其辖域"。

\textbf{辖域概念的重要性}:
\begin{itemize}
\item 明确量词的作用范围,避免歧义
\item 确定变元的约束关系
\item 为复杂公式的解释提供基础
\item 是形式语义学的核心概念之一
\end{itemize}

\textbf{辖域标记的必要性}:
没有适当的辖域标记,复杂的逻辑表达式可能产生多种解释,导致逻辑推理的混乱。
\end{examplebox}

\begin{theorembox}[title=自然语言表达的多样性与符号逻辑的统一性]
在继续讨论直言命题的其他传统形式之前,应该注意符号公式$(x)(H x \supset M x)$不仅是对标准形式的命题"所有 $H$'s 都是 $M$'s"的翻译,而且是对任何一个有同样含义的自然语言句子的翻译。\cite{brown1954}

\textbf{A型命题的自然语言变体}:
在自然语言中,述说这同一件事有许多不同的方式。它们的部分清单如下:
\begin{itemize}
\item \textbf{直接形式}:"$H$ 是 $M$","一个 $H$ 就是一个 $M$"
\item \textbf{量词形式}:"每个 $H$ 是 $M$","每一个 $H$ 是 $M$","任何 $H$ 是 $M$"
\item \textbf{否定形式}:"没有 $H$ 不是 $M$","没有什么是 $H$ 但不是 $M$"
\item \textbf{关系形式}:"是 $H$ 的每个事物都是 $M$","是 $H$ 的任何事物都是 $M$"
\item \textbf{条件形式}:"如果任何事物是 $H$ ,那么它是 $M$","如果某事物是 $H$ ,那么它是 $M$"
\item \textbf{限制形式}:"只有 $M$'s 是 $H$'s","除了 $M$'s 以外,没什么是 $H$'s"
\item \textbf{除非形式}:"没什么是 $H$ ,除非它是 $M$"
\item \textbf{抽象形式}:"人蕴涵(或涵衍)有死"
\end{itemize}
\end{theorembox}

\begin{examplebox}[title=符号逻辑与自然语言的比较]
符号逻辑语言对相当数量的自然语言句子的共同含义有一个单一的表达式,这一点被认为是符号逻辑在认知或信息方面比自然语言优越之处。

\textbf{符号逻辑的优势}:
\begin{itemize}
\item \textbf{精确性}:消除自然语言的歧义和模糊性
\item \textbf{简洁性}:用统一的符号表达多样的语言形式
\item \textbf{可操作性}:便于进行机械化的逻辑推理
\item \textbf{普遍性}:跨越语言和文化的障碍
\end{itemize}

\textbf{自然语言的优势}:
尽管从修辞力或诗意表现的观点看,我们承认符号逻辑在表达丰富性方面是一种劣势。
\begin{itemize}
\item \textbf{表达丰富性}:能够传达情感、态度、语调等
\item \textbf{语境敏感性}:能够根据具体情境调整含义
\item \textbf{修辞效果}:具有说服力和感染力
\item \textbf{文化承载}:承载着深厚的文化内涵
\end{itemize}
\end{examplebox}

\subsection{其他类型命题的符号化}

\begin{theorembox}[title=E型命题的符号化分析]
\textbf{E命题}"所有人都不是有死的"可以被依次释为:

\textbf{逐步符号化过程}:
\begin{quote}
给定不管任何个体事物,如果它是人,那么它不是有死的。\\
给定任何 $x$ ,如果 $x$ 是人,那么 $x$ 不是有死的。\\
给定任何 $x, x$ 是人 $\supset x$ 不是有死的。
\end{quote}

\textbf{最终符号化}:
$$(x)(H x \supset \sim M x)$$

\textbf{E型命题的自然语言变体}:
这种符号翻译不仅表示了自然语言中传统的 E 形式,同样也表示了一些说同一件事的不同方式,如"没有是 $M$ 的 $H$","没有什么既是 $H$ 又是 $M$",以及"$H$ 从不是 $M$"。

\textbf{逻辑结构特征}:
E型命题与A型命题具有相同的逻辑结构(全称量词+条件),区别仅在于后件的否定。
\end{theorembox}

\begin{theorembox}[title=I型命题的符号化分析]
\textbf{I命题}"有些人是有死的"可以依次释为:

\textbf{逐步符号化过程}:
至少有一个是人且有死的事物。\\
至少有这样一个 $x, x$ 是人并且 $x$ 是有死的。\\
至少有这样一个 $x, x$ 是人$\cdot$ $x$ 是有死的。

\textbf{最终符号化}:
$$(\exists x)(H x \cdot M x)$$

\textbf{逻辑结构特征}:
I型命题使用存在量词和合取结构,直接断言存在满足两个条件的个体。
\end{theorembox}

\begin{theorembox}[title=O型命题的符号化分析]
\textbf{O命题}"有些人不是有死的"可以依次释为:

\textbf{逐步符号化过程}:
至少存在一个是人但不是有死的事物。\\
至少存在这样一个 $x, x$ 是人并且 $x$ 不是有死的。\\
至少存在这样一个 $x, x$ 是人$\cdot$ $\sim x$ 是有死的。

\textbf{最终符号化}:
$$(\exists x)(H x \cdot \sim M x)$$

\textbf{逻辑结构特征}:
O型命题与I型命题具有相同的逻辑结构(存在量词+合取),区别仅在于第二个合取支的否定。
\end{theorembox}

\subsection{命题形式间的逻辑关系}

若用希腊字母 phi( $\phi$ )和 psi( $\Psi$ )表示任何一个谓词,传统逻辑的四个主—谓型普遍命题可以在图 10—2 所示的方阵中得到表达。\\
\includegraphics[width=\textwidth]{images/2025_05_15_6a28331d5e7c993ad07ag-466.jpg}

图10-2\\
显然, A 命题和 O 命题是\textbf{矛盾关系},一个是另一个的否定; E 命题和 I 命题也是矛盾关系。

有人可能会认为,一个 I 命题可以从与之相对的 A 命题推出,一个 O命题可以从与之相对的 E 命题推出,但情况并非如此。一个 A 命题为真时,与之对应的 I 命题却可能是假的。如果 $\phi_{x}$ 是一个没有真代人例的命

题函项,那么,不管命题函项 $\Psi x$ 有何种代人例,(复合)命题函项 $\phi_{x} \supset$ $\Psi x$ 的全称量化式都是真的。例如,考虑命题函项"$x$ 是一个人首马身的怪物",我们把它简写为 $C x$ 。因为不存在人首马身的怪物,$C x$ 的每个代入例都是假的,即 $C a 、 C b 、 C c \cdots \cdots$ 都为假。因此,复合命题函项 $C x \supset B x$的每个代人例都是一个前件为假的条件陈述。这样,其代人例 $\mathrm{Ca} \supset \mathrm{Ba}$ 、 $C b \supset B b 、 C c \supset B c$ 都是真的,因为任何一个断言实质蕴涵的条件陈述,如果其前件为假,那么它必定为真。由于其所有代人例都是真的,所以,命题函项 $C x \supset B x$ 的全称量化式为真,即 A 命题 $(x)(C x \supset B x)$ 为真。但与之相对的 I 命题 $(\exists x)(C x \cdot B x)$ 却是假的,因为命题函项 $C x \cdot B x$没有真代人例。 $C x \cdot B x$ 没有真代人例可以从 $C x$ 没有真代人例推出。 $C x \cdot B x$ 的各个代人例,如 $C a \cdot B a 、 C b \cdot B b 、 C c \cdot B c \cdots \cdots$ 都是第一个合取支为假的合取式,因为 $C a 、 C b 、 C c \cdots \cdots$ 都为假。由于其所有代人例都为假,所以命题函项 $C x \cdot B x$ 的存在量化式为假,即 I 命题( $\exists x$ )( $C x \cdot B x$ )为假。因此,有可能一个 A 命题是真的,而与之对应的 I 命题却是假的。

这种分析还可表明,为什么有可能一个 E 命题是真的,而与之对应的 $O$ 命题却是假的。如果我们以命题函项 $\sim B x$ 替换前面讨论中的命题函项 $B x$ ,那么,$(x)(C x \supset \sim B x)$ 可以是真的,而 $(\exists x)(C x \cdot \sim B x)$ 却是假的。当然,这也是因为并没有人首马身的怪物。

\subsection{命题类型间的关键区别}

问题的关键在于:A 命题和 E 命题并不断言或假定任何事物存在,它们仅断言情况是这样的:如果有某事,则有另外一件事。但 I 命题和 O 命题却假定某物存在,它们断言情形是这样的:有这件事并且有另一件事。 1 命题和 $O$ 命题中的\textbf{存在量词}是区别的关键所在。从一个并不断言或假定任何事物存在的命题推出某物的存在,这显然是错误的。

如果我们假定至少有一个个体存在,那么( $x$ )( $C x \supset B x$ )确实蕴涵 ( $\exists x) ~(C x \supset B x)$ 。但后者不是一个 I 命题。 I 命题"有些人首马身的怪物是漂亮的"应符号化为 $(\exists x)(C x \cdot B x)$ ,它说的是,至少存在一个漂亮的人首马身的怪物。但在自然语言中,被符号化为( $\exists x$ )( $C x \supset B x$ )的东西,可以被理解为"至少存在一个如此这般的事物,如果它是人首马身的怪物,那么它是漂亮的"。它并没说存在一个人首马身的怪物,而只是说存在一个个体,它或者不是人首马身的怪物,或者是漂亮的。而这个命题只在两种情况下是假的:第一,如果根本不存在个体;第二,如果所有个体都是人首马身的怪物,并且它们当中没有一个是漂亮的。通过作这样

一个明确的(并且显然是真的)假定,即假定宇宙中至少存在一个个体,我们可以排除第一种情形。第二种情形是如此极端地不合理,以致与 I 形命题( $\exists x$ )( $\phi_{x} \cdot \Psi x$ )的重要性相反,任何形如( $\exists x$ )( $\phi_{x} \supset \Psi x$ )的命题都必定是非常平庸的。显而易见,尽管在自然语言中 A 命题"所有人是有死的"和 I 命题"有些人是有死的"的区别,仅在于初始词"所有"和"有些"的不同,但它们意义上的区别并不限于全称量化和存在量化,而是比这深刻得多。经量化而产生 A 命题和 I 命题的命题函项不仅在量化上有区别,而且它们还是不同的命题函项,一个含有"$\supset$",另一个含有"•"。换言之,A 命题和 I 命题并不像它们在自然语言中看起来那么相似。它们之间的区别可通过使用命题函项符号和量词符号得以彰显。

\subsection{命题变形技巧}

就逻辑操作来说,处理那些否定号的出现——如果有否定号出现的话——只作用于简单谓述的公式最为方便。因此,我们将在必要时通过替换来得到这种公式。要做到这一点很简单。从第9章所确立的推论规则可知,我们可以用另一个与之逻辑等价的表述式来替换一个表述式。而我们有四个这样的逻辑等价式(10.2节),它们当中否定号在量词之前的命题,都与另一个否定号直接作用于简单谓述的命题等价。用我们熟悉已久的推论规则,可以移动否定号,使它们最终不再作用于复合表达式,而只作用于简单谓述。譬如说,公式:

$$
\sim(\exists x)(F x \cdot \sim G x)
$$

可以依次改写。首先,如果我们用 10.2 节所给的第三个逻辑等价式,它可以变形为:

$$
(x) \sim(F x \cdot \sim G x)
$$

然后,可运用德摩根律使之变成:

$$
(x)(\sim F x \vee \sim \sim G x)
$$

再用双重否定律可得公式:

$$
(x)(\sim F x \vee G x)
$$

最后,若援引实质蕴涵定义,原公式也可以改写成下述 A 命题:

$$
(x)(F x \supset G x)
$$

在转到关于非复合陈述推论的话题之前,读者应该进行一些把非复合陈述从自然语言翻译成逻辑符号的训练。自然语言有如此之多不规则的和惯用的构造,以致不可能有把自然语言语句翻译成逻辑符号的简单规则。在任何情形下都要先理解语句的含义,然后用命题函项和量词术语予以重述。

\begin{center}
\fbox{\parbox{0.95\textwidth}{
\textbf{本节要点}
\begin{itemize}
\item \textbf{传统四种命题形式的历史背景}:
  \begin{itemize}
  \item 构成亚里士多德逻辑学的核心,被称为\logicterm{直言命题}
  \item 在西方逻辑传统中占据两千多年的主导地位
  \item 现代逻辑学通过量词理论为其提供了精确的形式化
  \item 命题类型标记来自拉丁文:A、I来自"affirmo",E、O来自"nego"
  \end{itemize}
\item \textbf{四种命题形式的符号化}:
  \begin{itemize}
  \item \textbf{A型}(全称肯定):$(x)(Sx \supset Px)$ - 使用全称量词和条件结构
  \item \textbf{E型}(全称否定):$(x)(Sx \supset \sim Px)$ - 条件后件否定
  \item \textbf{I型}(特称肯定):$(\exists x)(Sx \cdot Px)$ - 使用存在量词和合取结构
  \item \textbf{O型}(特称否定):$(\exists x)(Sx \cdot \sim Px)$ - 合取第二支否定
  \end{itemize}
\item \textbf{复合命题函项的结构分析}:
  \begin{itemize}
  \item 条件命题函项、合取命题函项、析取命题函项、否定命题函项等
  \item 量词辖域的重要性:明确作用范围,避免歧义
  \item 任何真值函项复合命题都可看作命题函项的代入例
  \item 辖域标记是形式语义学的核心概念之一
  \end{itemize}
\item \textbf{自然语言表达的多样性}:
  \begin{itemize}
  \item A型命题有8种主要的自然语言变体形式
  \item 符号逻辑的优势:精确性、简洁性、可操作性、普遍性
  \item 自然语言的优势:表达丰富性、语境敏感性、修辞效果、文化承载
  \item 符号逻辑为多样的自然语言表达提供统一的形式化工具
  \end{itemize}
\item \textbf{命题间的逻辑关系深入分析}:
  \begin{itemize}
  \item A与O、E与I构成矛盾关系(一真一假)
  \item A命题不必然蕴涵I命题(当主词类为空时)
  \item 空类问题:人首马身怪物的例子说明了存在假定的重要性
  \item 传统对当方阵在现代逻辑中的修正
  \end{itemize}
\item \textbf{存在假定的关键区别}:
  \begin{itemize}
  \item A型和E型命题不断言或假定任何事物存在
  \item I型和O型命题假定主词类有实例存在
  \item 存在量词是区别的关键所在
  \item 这一区别解释了传统逻辑与现代逻辑的重要差异
  \end{itemize}
\item \textbf{命题变形技巧}:
  \begin{itemize}
  \item 利用量词对偶关系进行命题转换
  \item 运用德摩根律、双重否定律、实质蕴涵定义等
  \item 将否定号移动到简单谓述上的技巧
  \item 自然语言翻译需要先理解含义再用逻辑术语重述
  \end{itemize}
\end{itemize}
}}
\end{center}

\input{chapter10/10-4 有效性证明.tex}
\input{chapter10/10-5 无效性证明.tex}
\section{非三段论推论}

\begin{logicbox}[title=引言]
本节将我们的分析扩展到更复杂的论证形式——非三段论推论。这类论证不限于传统直言三段论的结构,而是涉及更复杂的内部逻辑关系和命题形式。我们将学习如何正确地符号化这些复杂论证,避免常见的误解,并应用已建立的逻辑工具来判断它们的有效性。
\end{logicbox}

\subsection{超越三段论的限制}

前两节讨论的所有论证都具有传统上叫做直言三段论的形式。它们由两个前提和一个结论组成,每个前提和结论都可以分析成一个单称命题或 $A 、 E 、 I 、 O$ 中的某一种。现在我们转向评价更复杂一些的论证。评估这些论证并不需要比此前已经给出的更多的逻辑工具。这些论证称为\textbf{非三段论论证},这就是说,它们不能划归为标准形式的直言三段论。因此,评价它们就需要一种比传统上检验直言三段论所使用的更有力的逻辑。

本节我们仍关注普遍命题,它们是通过量化只含有一个个体变元的命题函项而形成的。在直言三段论中,被量化的命题函项具有 $\phi_{x} \supset \Psi x$ , $\phi_{x} \supset \sim \Psi_{x}, \phi_{x} \cdot \Psi x, \phi_{x} \cdot \sim \Psi x$ 形式。但现在我们要量化一些具有更复杂内部结构的命题函项。下述例子有助于说明问题,请考虑论证:

\begin{quote}
旅馆都是既贵又令人压抑的。\\
有些旅馆简陋。\\
因此,有些贵的东西简陋。
\end{quote}

\subsection{适当符号化的重要性}

该论证显然是有效的,但它并不能用传统方法加以分析。若分别用符号 $H x, B x, S x$ 和 $E x$ 缩写命题函项"$x$ 是旅馆","$x$ 既贵又令人压抑", "$x$ 是简陋的"和"$x$ 是贵的",该论证的确可以用 A 和 I 命题来表达。\cite{lukasiewicz1951}用这些缩写形式可把该论证符号化为:

$$
\begin{aligned}
& (x)(H x \supset B x) \\
& (\exists x)(H x \cdot S x) \\
& \therefore(\exists x)(E x \cdot S x)
\end{aligned}
$$

但以这种方式强迫该论证受传统的 A 和 I 形式的束缚,就遮蔽了它的有效性。尽管原来的论证非常有效,但刚才用符号给出的论证却是无效的。这里对直言命题所施加的符号限制遮蔽了 $B x$ 和 $E x$ 之间的逻辑联系。用如上所解释的 $H x 、 S x$ 和 $E x$ ,加上 $D x$ ,我们可以获得一个更适当的分析。在此,$D x$ 是"$x$ 是令人压抑的"的缩写。原来的论证用这些符号可以翻译成:

1.$(x)[H x \supset(E x \cdot D x)]$\\
2.$(\exists x)(H x \cdot S x)$\\
$\therefore(\exists x)(E x \cdot S x)$\\
经过如此符号化,它的有效性证明很容易构造。这样的证明可以如下进行:

\begin{center}
\begin{tabular}{ll}
3.Hw•Sw & 2,EI \\
4.Hwつ(Ew•Dw) & 1,UI \\
5.Hw & 3,Simp. \\
6.Ew•Dw & 4,5,M.P. \\
7.Ew & 6,Simp. \\
8.Sw•Hw & 3,Com. \\
9.Sw & 8, Simp. \\
10.Ew•Sw & 7,9,Conj. \\
11.$(\exists x)(E x \cdot S x)$ & 10,EG \\
\end{tabular}
\end{center}

\subsection{自然语言符号化的潜在问题}

在对经量化更复杂的命题函项而产生的普遍命题进行符号化时,必须小心不要被日常语言的表述方式所误导。我们不能依照任何形式的或机械的规则来把自然语言翻译为逻辑符号。在每种情形下,必须理解自然语言

语句的意义,然后用命题函项和量词术语加以符号化。\\
日常语言中有时令人困扰的三种表达方式是这样的。第一,像"所有运动员力气大或跑得快"这样的陈述,尽管它含有联结词"或",但它不是一个析取式。它无疑和"或者所有运动员力气大或者所有运动员跑得快"不具有同样的含义。使用缩写形式,前者可以恰当地符号化为:

$$
(x)[A x \supset(S x \vee Q x)]
$$

而后者却可以符号化为:

$$
(x)(A x \supset S x) \vee(x)(A x \supset Q x)
$$

第二,我们注意到,"牡蛎和蚌好吃"这样的陈述,可以被表述为两个普遍命题的合取,即"牡蛎好吃并且蚌好吃";但它也可被表述为一个单一的非复合普遍命题。在这种情况下,语词"和"可以用"$V$"而不是 "•"来恰当地符号化。该命题可以符号化为:

$$
(x)[(O x \vee C x) \supset D x]
$$

而不是

$$
(x)[(O x \cdot C x) \supset D x]
$$

因为说牡蛎和蚌好吃,就是说任何一个或者是牡蛎或者是蚌的东西好吃,而不是说任何一个既是牡蛎又是蚌的东西好吃。

\subsection{除外命题的处理}

第三,对所谓的\textbf{除外命题}要格外小心。如"除以前的获胜者外,都符合条件"这样的命题,可以被处理成两个普遍命题的合取。利用刚给出的那个例子,我们可以合理地把此命题理解为断言:以前的获胜者不符合条件,并且那些不是以前的获胜者的人符合条件。因此,它可以符号化为:

$$
(x)(P x \supset \sim E x) \cdot(x)(\sim P x \supset E x)
$$

但这个同样的除外命题也可以翻译成一个非复合的普遍命题,这个命题是一个含有实质等值符"三"的命题函项的全称量化式,它是一个双条件陈述,可以符号化为:

$$
(x)(E x \equiv \sim P x)
$$

这个符号表达式也可以用日常语言翻译成"任何人要符合条件,当且仅当,这个人不是以前的获胜者"。一般来说,除外命题可以最方便地看做

是量化了的双条件陈述。\\
有时很难确定一个命题事实上是否是除外命题。近期一件要求联邦法庭全体陪审员解决的纠纷说明了这种情境上的困难。《人口调查法》制定了每十年进行一次的全国普查的一些规则,它有这样一段话:

195 节.除为了在几个州中分配国会代表的席位而确定人口数量以外,[商业]部长在执行这项权利的有关规定时,有权批准使用"抽样"统计方法,如果他认为这是可行的话。

在因分配国会代表席位要确定人口数量而进行的 2000 年的普查中,普查局想使用抽样技术,但被众议院控诉。众议院宣称上面的引文禁止在这样一次普查中进行抽样。普查局对此作了辩护,认为这段话批准在某些情境中使用抽样,但在席位分配情境中却悬而末决。对法规中除外规定的哪种解释是正确的呢?

法庭认为众议院的见解正确,它写道:

考察这样一个指令,"除我祖母的结婚礼服外,把我衣榭里的东西都送到洗衣店去"。……这似乎是说,如果该孙女的指令的接受者把结婚礼服送到洗衣店去,并且随后争辩说她把这留给他作决定,那么她会气恼。产生这一结果的原因……是因为我们关于结婚礼服的背景知识:我们知道它们特别易坏,并且对家庭成员具有极深的情感价值。因此,我们不希望决定把礼服送到洗衣店是完全任意的。

各州国会代表席位的分配就是衣瀜中的那件结婚礼服……分配函数是"十年一度的普查的单调构成性函数",其执行方式不仅影响各州代表席位的分配,而且影响众议院中政治力量的平衡……本法庭认为,《人口调查法》禁止为了在州中分配代表席位而去确定人口数量时使用统计抽样法……[10]

因此,这个法规中的除外命题被理解为断定这两个命题的合取:(1)在分配席位的情境中,使用抽样是不允许的,(2)在所有其他情境中,可以任意使用抽样。一个除外形式的争议性语句必须在其情境中来理解。

\subsection{非三段论论证的有效性判断}

在 10.4 节,我们的推论规则表增加了 4 个规则,并且表明,这个扩展表足以证明有效的直言三段论的有效性。刚才已经看到,同一扩展表足以确立所描述类型的非三段论论证的有效性。现在我们可以观察到,正如扩展表足以在非三段论论证中判定有效性一样,证明三段论无效的(在 10.5 节所解释的)方法,即通过描述非空的可能域或模型,也足以证明当前这种非三段论论证的无效性。考虑下面这个非三段论论证:

经理和主管或者是有能力的员工,或者是所有者的亲属。\\
敢抱怨的人必定或者是主管,或者是所有者的亲属。\\
唯有经理和工头是有能力的员工。\\
某人敢抱怨。\\
因此,某个主管是所有者的亲属。

可以符号化为:

$$
\begin{aligned}
& (x)[(M x \vee S x) \supset(C x \vee R x)] \\
& (x)[D x \supset(S x \vee R x)] \\
& (x)(M x \equiv C x) \\
& (\exists x) D x \\
& \therefore(\exists x)(S x \cdot R x)
\end{aligned}
$$

通过描述一个只含有个体 $a$ 的可能域或模型,并对 $C a 、 D a 、 F a$ 和 $R a$ 指派真值真,对 Sa 指派真值假,我们可以证明它无效。

\begin{center}
\fbox{\parbox{0.95\textwidth}{
\textbf{本节要点}
\begin{itemize}
\item \textbf{非三段论推论的特点}:
  \begin{itemize}
  \item 超越了传统直言三段论的形式和限制
  \item 涉及更复杂内部结构的量化命题函项
  \item 可用相同的逻辑工具(四个量化规则)进行分析
  \end{itemize}
\item \textbf{符号化中的关键问题}:
  \begin{itemize}
  \item 准确符号化对判断论证有效性至关重要
  \item 过度简化可能掩盖论证的真实逻辑结构
  \item 正确分析复合概念(如"既贵又令人压抑")的内部结构
  \end{itemize}
\item \textbf{自然语言表达的陷阱}:
  \begin{itemize}
  \item "所有A都是B或C"≠"或者所有A都是B,或者所有A都是C"
  \item "A和B都是C"应译为"(x)[(Ax∨Bx)⊃Cx]",而非"(x)[(Ax·Bx)⊃Cx]"
  \item 除外命题需要根据上下文正确解释,可表示为双条件陈述或合取命题
  \end{itemize}
\item \textbf{有效性和无效性判断}:
  \begin{itemize}
  \item 证明有效性:使用相同的四个量化推论规则
  \item 证明无效性:构造具有特定真值指派的有限模型
  \item 同样的方法适用于三段论和非三段论推论
  \end{itemize}
\end{itemize}
}}
\end{center}

% 第十一章
\chapter{模态逻辑}
\section{类比论证}

\begin{logicbox}[title=引言]
本节介绍\logicterm{类比论证}的概念及其在日常推理中的重要性。我们将分析\logicterm{类比论证}的基本结构,区分它与\logicterm{演绎论证}的根本差异,探讨类比在论证和非论证语境中的多种用途,并学习如何识别和表达\logicterm{类比论证}的基本形式,从而为理解这种\logicterm{归纳推理}方式奠定基础。
\end{logicbox}

\subsection{从演绎确定性到归纳或然性}

前几章讨论的是\logicterm{演绎论证}。\logicterm{演绎论证}是否\logicemph{有效},取决于其前提是否能够证明地(demonstratively)得到结论。

\begin{theorembox}[title=演绎确定性的局限性]
然而,还有许多良好的和重要的论证,这些论证的结论不能得到确定性的证明。这种局限性体现在以下几个方面:

\textbf{1. 经验知识的或然性}:我们充分相信许多\logicterm{因果连接}(causal connections),只是基于\logicterm{盖然性}(probability)——尽管\logicterm{盖然性}程度可能非常高。

\textbf{2. 科学认识的特征}:我们能够不加迟疑地说,吸烟是癌症的一个原因,但我们不能够赋予我们的这种知识与从前提中推得一个演绎\logicemph{有效的}论证结论这种知识以相同的确定性。

\textbf{3. 专业认知的限制}:一个著名的医科专家根据演绎标准声称:"没有人将能够证明(prove)吸烟导致癌症,或者说任何事情导致任何事情。从理论上讲,你不能够证明任何事情。"\cite{surgeon1964}

\textbf{4. 认识论的现实}:的确,当我们评价我们关于世界的事实的知识时,演绎确定性的标准太高了。
\end{theorembox}

\begin{examplebox}[title=归纳推理的历史必然性]
这种从演绎确定性向归纳或然性的转变,在认识论史上具有重要意义:

\textbf{古代哲学}:亚里士多德已经认识到,除了演绎推理之外,还存在从特殊到一般的归纳推理。

\textbf{近代科学革命}:培根强调经验归纳法在科学发现中的重要性,与笛卡尔的演绎理性主义形成对比。

\textbf{现代科学方法}:波普尔的证伪主义和库恩的范式理论都承认,科学知识本质上是可错的和或然的。

\textbf{当代认知科学}:人工智能和机器学习的发展进一步证实了归纳推理在人类认知中的核心地位。
\end{examplebox}

\begin{theorembox}[title=归纳论证与演绎论证的区别]
本章及以后的各章将转向分析这样的论证:人们在这些论证中并不声称结论的真理性是从前提必然地得到,而仅仅表明,前提对结论的支持是\logicterm{或然的}(probable),或者说结论\logicterm{盖然为真}。这种论证被称为\logicterm{归纳论证},其与\logicterm{演绎论证}具有根本性差异。我们已经在第1章中讨论了演绎和归纳之间的基本区别。本书第二部分已经对演绎进行了讨论,第三部分则用来讨论归纳。
\end{theorembox}

\subsection{类比论证的本质与特征}

在\logicterm{归纳论证}中有一种被普遍使用的论证类型:\logicterm{类比}(analogy)论证。

\begin{theorembox}[title=类比论证的认知基础]
类比论证在人类认知中占据特殊地位,其重要性体现在以下几个方面:

\textbf{1. 认知经济性}:类比允许我们利用已有知识来理解新情况,避免从零开始的认知负担。

\textbf{2. 模式识别}:人类大脑天生具有识别相似模式的能力,类比论证正是这种能力的逻辑表达。

\textbf{3. 创新思维}:许多科学发现和技术创新都源于类比思维,如开普勒的行星运动类比、达尔文的自然选择类比。

\textbf{4. 社会交往}:类比论证是说服和解释的重要工具,在法律、政治、教育等领域广泛应用。

\textbf{5. 文化传承}:通过类比,复杂的文化概念和价值观念得以在代际间传递。
\end{theorembox}

下面是两个\logicterm{类比论证}的例子:

\begin{examplebox}[title=类比论证实例一:教师资格测验]
一些人认为教师资格测验是不公正的双重测试。"教师已经是大学毕业生,"他们说,"他们为什么还要被测试?"其实这很简单。律师是大学毕业生,而且还是职业学院的毕业生,但他们不得不参加律师资格考试。还有其他大量的行业,如会计、精算师、医生、建筑师等,这些行业对想成为其成员的人都要求参加并通过资格考试,以证明他们的专业素质。没有理由说明教师不应当被要求做同样的事情。\cite{davis1986}
\end{examplebox}

\begin{examplebox}[title=类比论证实例二:行星生命假说]
在我们居住的地球和其他行星(土星、木星、火星、金星和水星)之间,我们可以观察到许多类似之处。它们均如地球一样围绕太阳运行,尽管它们绕太阳的半径不同、周期也不同。它们均从太阳那里获得光,地球也是如此。我们已经知道,其中一些行星,如地球一样,围绕它们的轴自转,因而它们必定有类似白天和黑夜的更替。一些行星有卫星,当太阳不再照射时,这些卫星给行星以光亮,就如我们的月亮给我们以光一样。这些行星的运动均与地球一样受制于万有引力定律。根据所有这些类似,认为这些行星可能与我们地球一样,有不同等级的生命存在,这不是不合理的。通过\logicterm{类比}得到的这个结论具有一定程度的可能性。\cite{reid1785}
\end{examplebox}

\subsection{类比推理在日常生活中的普遍性}

我们的许多日常推论是通过\logicterm{类比}进行的。

\begin{examplebox}[title=日常类比推理的典型实例]
\textbf{消费决策}:我推论我将从一台新的计算机那里得到好的服务,根据是,我从同样的生产厂家购买的一台计算机曾给了我很好的服务。

\textbf{文化选择}:我看到某个作者的新作,根据我读过该作者的其他著作并且喜欢这些著作而推断,我将喜欢读这本新作。

\textbf{经验应用}:我们过去的经验在未来同样成立的大多数日常推论,其基础就是\logicterm{类比}。

\textbf{本能反应}:当然,我们无法给出一个清楚的公式化的论证,我们只能说,曾被烧伤的孩童躲避火的行为即涉及\logicterm{类比推论}。
\end{examplebox}

\begin{theorembox}[title=类比推理的心理学机制]
类比推理的普遍性源于其深层的心理学机制:

\textbf{1. 记忆结构}:人类记忆以相似性为基础组织,相似的经验会被归类存储,便于类比提取。

\textbf{2. 模式匹配}:大脑会自动将新情况与已存储的模式进行匹配,寻找最相似的先例。

\textbf{3. 预测机制}:基于相似性的预测是生存的基本需要,类比推理是这种预测能力的体现。

\textbf{4. 学习转移}:类比允许我们将在一个领域学到的知识转移到另一个领域,提高学习效率。

\textbf{5. 概念形成}:许多抽象概念都是通过类比具体事物而形成的,如"时间流逝"类比"河水流动"。
\end{theorembox}

\logicwarn{这些论证中没有一个是确定的或者说是证明性地\logicemph{有效的}。这些论证中的结论,没有一个能够从前提中获得逻辑必然性。}这是逻辑可能的:用来判断律师和医生资格的方法,并不适合于判断教师的资格;这也是逻辑可能的:地球可能是唯一可以居住的行星,新的计算机可能运转不灵,我喜欢的作者的新书可能无趣而难以卒读;甚至这也是逻辑可能的:一团火能够烧伤人,另外一团火则不会。没有一个\logicterm{类比论证}可以指望具有数学的那种确定性。\logicterm{类比论证}不是按\logicemph{有效}和\logicwarn{无效}来区分的,我们只能用\logicterm{概率}来刻画它们。

\subsection{类比的多元功能:论证之外的应用}

除了在论证中频繁使用类比外,人们为了描述生动,经常将类比用于非论证的活动中。

\begin{theorembox}[title=类比的非论证功能分类]
类比在非论证语境中具有多种重要功能:

\textbf{1. 文学修辞功能}:
明喻和暗喻为类比在文学中的用法,它们为作家给读者的心中创造鲜活的画面提供了莫大的帮助。例如:砧骨上做马蹄铁而产生的副产品火花一样。火花比马蹄铁更为灿烂,但它们在本质上是无意义的。\cite{chesterton1910}

\textbf{2. 教学说明功能}:
类比也用于说明,将读者不熟悉的某种东西,与读者比较熟悉的另一种东西进行对照,比较它们的类似之处,而使读者得以理解。

\textbf{3. 科学解释功能}:
在科学传播中,类比是连接专业知识与公众理解的重要桥梁。

\textbf{4. 概念建构功能}:
许多抽象概念的形成都依赖于与具体事物的类比。

\textbf{5. 情感共鸣功能}:
类比能够激发情感反应,增强表达的感染力。
\end{theorembox}

\begin{examplebox}[title=科学类比的经典案例]
麻省理工学院基因组研究中心主任埃瑞克-兰德试图说明人类基因组计划的巨大影响。为了加强那些对基因研究不熟悉的人的理解,类比是他所用的一个工具:

"基因组计划完全类似于化学中创立周期表。正如门捷列夫在周期表中安排化学元素,使得以前不相关的大量数据变得连贯,同样,当前有机体中上万的基因,将能够从较少数量的简单基因模块或单元即所谓原始基因的组合中得到。"\cite{lander1995}

\textbf{这个类比的成功之处}:
\begin{itemize}
\item 利用了听众对周期表的熟悉程度
\item 突出了组织性和系统性的共同特征
\item 暗示了基因组研究的革命性意义
\item 将复杂的生物学概念转化为可理解的化学类比
\end{itemize}
\end{examplebox}

\subsection{类比论证的结构分析}

类比在描述和说明中的使用不同于在论证中的使用,尽管在某些实例下,不容易区分是哪一种用法。但是,无论是论证地使用类比还是其他的使用法,类比都是不难定义的。在两个或更多的实体之间进行一个类比,就是表明它们在一个或多个方面(respect)是类似的(similar)。

\begin{theorembox}[title=类比与类比论证的区别]
这说明了什么是类比,但是仍然没有刻画什么是类比论证。两者的关键区别在于:

\textbf{类比}:仅仅指出相似性,不涉及推理过程。
\textbf{类比论证}:基于相似性进行推理,从已知相似性推出未知相似性。

类比论证的核心是\logicterm{推理转移}:从一个或多个已知的相似方面,推断出另一个未知方面的相似性。
\end{theorembox}

让我们考察一个类比论证事例并分析它的结构。我们选择上面引用的例子中最简单的例子:我新买的计算机将给我好的服务,因为我的一台旧计算机是从同样厂家购买的,它给了我好的服务。

\begin{examplebox}[title=类比论证的结构解析]
具有类似方面的两个事物是两台计算机。这里存在三点类比,两个事物被认为在三个方面相似:
\begin{enumerate}
\item 均为计算机
\item 均在同样厂家购买
\item 给我好的服务
\end{enumerate}

\textbf{关键观察}:类比的这三点在论证中并不起相同的作用。前两点出现在前提中,而第三点既出现在前提中又出现在结论中。

\textbf{论证结构}:该论证具有这样的前提:首先断定两个事物在两点类似,其次断定了其中的一个事物还具有另外一个特点,从而推论得出另一个事物也具有这个特点的结论。

\textbf{逻辑形式}:
\begin{itemize}
\item \textbf{前提1}:A和B在性质P和Q方面相似
\item \textbf{前提2}:A具有性质R
\item \textbf{结论}:因此,B可能也具有性质R
\end{itemize}
\end{examplebox}

\begin{theorembox}[title=法庭类比论证的特殊地位]
类比论证是法庭最基本的工具之一。法官不是事先摆出严格的法规或原理,他们往往这样推理,因为两个案件——早先的已经被判决的案件和手头上待判决的案件——有相同的特点,它们应当具有相同的判决结果。

\textbf{判例法的逻辑基础}:
\begin{itemize}
\item \textbf{先例约束原则}:相似案件应得到相似处理
\item \textbf{法律连续性}:保持法律适用的一致性
\item \textbf{可预测性}:当事人能够预期法律后果
\item \textbf{公平性}:避免任意性和歧视性判决
\end{itemize}

\textbf{实例分析}:例如,一旦做出了不能禁止3K党发表言论的判决,那么法庭可能通过类比论证而得出不能禁止纳粹党游行的结论。\cite{collin1978} 通过判例的论证一旦做出,人们将确定和强调以前的案子和手头案子之间类似的那些特点。

\textbf{法庭类比的复杂性}:
\begin{itemize}
\item 需要识别法律上相关的相似性
\item 必须区分重要特征与偶然特征
\item 涉及价值判断和政策考量
\item 可能面临多个竞争性先例
\end{itemize}
\end{theorembox}

\subsection{类比论证的一般形式}

当然,不是每个类比论证都必须精确地涉及两个事物或者精确涉及三个不同的特点。托马斯-雷德(在上面我们已经提到)认为其他行星可能有人居住,他的论证是对六个事物(当时知道的行星)的八个方面进行类比的。

\begin{theorembox}[title=类比论证的一般结构]
然而,除了这些数量存在差别外,所有的类比论证均具有相同的一般结构或模式。每个类比推理都是这样进行的:从在一个或多个方面两个或更多的事物之间的类似性,到这些事物在某个其他方面具有类似性。

\textbf{形式化表示}:设$a 、 b 、 c 、 d$ 是实体,$P 、 Q 、 R$ 是属性或"相似方面",一个类比论证可以表示成下列形式:

$$
\begin{aligned}
& a 、 b 、 c 、 d \text { 均具有属性 } P \text { 和 } Q, \\
& a 、 b 、 c \text { 均具有属性 } R,
\end{aligned}
$$

$$
\text { 因而 } d \text { 可能具有属性 } R \text { 。 }
$$

\textbf{结构要素分析}:
\begin{itemize}
\item \textbf{类比对象}:参与比较的实体(a, b, c, d)
\item \textbf{共同属性}:所有对象都具有的属性(P, Q)
\item \textbf{已知属性}:部分对象已知具有的属性(R)
\item \textbf{推断属性}:待推断对象可能具有的属性(R)
\item \textbf{或然性}:结论的不确定性("可能")
\end{itemize}

在识别并且特别是评价类比论证时,将之表示成这种形式是有帮助的。
\end{theorembox}

\begin{center}
\fbox{\parbox{0.95\textwidth}{
\textbf{本节要点}
\begin{itemize}
\item \textbf{从演绎确定性到归纳或然性}:
  \begin{itemize}
  \item 演绎确定性标准在评价经验知识时过于严格
  \item 科学认识本质上具有或然性特征,如因果关联的确立
  \item 归纳推理的历史必然性:从古代哲学到现代认知科学的发展脉络
  \item 归纳论证与演绎论证的根本差异:或然性vs必然性
  \end{itemize}
\item \textbf{类比论证的认知基础}:
  \begin{itemize}
  \item \textbf{五大重要性}:认知经济性、模式识别、创新思维、社会交往、文化传承
  \item 类比推理是人类大脑天生的认知能力
  \item 许多科学发现和技术创新都源于类比思维
  \item 在法律、政治、教育等领域广泛应用
  \end{itemize}
\item \textbf{类比推理的心理学机制}:
  \begin{itemize}
  \item \textbf{五大机制}:记忆结构、模式匹配、预测机制、学习转移、概念形成
  \item 人类记忆以相似性为基础组织,便于类比提取
  \item 基于相似性的预测是生存的基本需要
  \item 许多抽象概念都是通过类比具体事物而形成
  \end{itemize}
\item \textbf{类比的多元功能}:
  \begin{itemize}
  \item \textbf{论证功能}:基于相似性进行推理,从已知推出未知
  \item \textbf{非论证功能}:文学修辞、教学说明、科学解释、概念建构、情感共鸣
  \item 类比与类比论证的区别:指出相似性vs基于相似性推理
  \item 科学传播中类比是连接专业知识与公众理解的重要桥梁
  \end{itemize}
\item \textbf{类比论证的结构分析}:
  \begin{itemize}
  \item \textbf{核心机制}:推理转移——从已知相似性推出未知相似性
  \item \textbf{逻辑形式}:前提1(A和B在P、Q方面相似)+前提2(A具有R)→结论(B可能具有R)
  \item 不同相似方面在论证中起不同作用:基础相似性vs推断相似性
  \item 结论的或然性是类比论证的本质特征
  \end{itemize}
\item \textbf{法庭类比论证的特殊地位}:
  \begin{itemize}
  \item \textbf{判例法的逻辑基础}:先例约束原则、法律连续性、可预测性、公平性
  \item 需要识别法律上相关的相似性,区分重要特征与偶然特征
  \item 涉及价值判断和政策考量,可能面临多个竞争性先例
  \item 体现了法律推理的类比本质
  \end{itemize}
\item \textbf{类比论证的一般形式}:
  \begin{itemize}
  \item 所有类比论证都具有相同的一般结构或模式
  \item \textbf{五大结构要素}:类比对象、共同属性、已知属性、推断属性、或然性
  \item 形式化表示有助于识别和评价类比论证
  \item 数量可变但结构不变:可涉及多个对象和多个属性
  \end{itemize}
\end{itemize}
}}
\end{center}
\input{chapter11/11-2 类比论证的评价.tex}
\section{通过逻辑类推进行的反驳}

\begin{logicbox}[title=引言]
本节讨论如何使用\logicterm{逻辑类推}作为反驳论证的\logicemph{有效}工具。我们将分析\logicterm{逻辑类推}的基本原理,探讨它在\logicterm{演绎}和\logicterm{归纳论证}中的应用,并通过实际案例说明这种方法如何揭示论证的形式缺陷。通过理解\logicterm{逻辑类推}反驳的技巧,我们将能够更\logicemph{有效地}评估和批判各种复杂论证。
\end{logicbox}

\subsection{逻辑类推反驳法的文学启示}

\begin{examplebox}[title=《爱丽丝漫游奇境记》中的逻辑类推]
"你应当说出你的意思。"[野兔尖刻地责备爱丽丝。]\\
"我说了,"爱丽丝慌忙回应;"至少——至少我说的意思——那是同样的事情,你是知道的。"\\
"一点也不同!"哈特尔说。"哎呀,你不如说'我看到我吃的东西' 与 '我吃我看到的东西' 是同一回事情!"\\
"你不如说"野兔附和道,"'我喜欢我得到的东西'与'我得到我喜欢的东西'是同一回事!"

\textbf{文学中的逻辑智慧}:
野兔、哈特尔和冬眠鼠都使用逻辑类推(logical analogy)试图反驳爱丽丝的看法,即你所说的与你的意图是一回事。这个文学例子生动地展示了逻辑类推反驳法的基本技巧。
\end{examplebox}

\begin{theorembox}[title=逻辑类推反驳法的基本原理]
从逻辑的观点来看,论证的形式与论证的内容不同,形式是论证的最重要的方面。因而,我们往往通过表明另外一个被认为是错误的论证,与给定的论证有相同的逻辑形式,而证明该论证是不可靠的。

\textbf{方法的核心要素}:
\begin{itemize}
\item \textbf{形式同构性}:构造的反驳论证必须与原论证具有完全相同的逻辑形式
\item \textbf{内容差异性}:虽然形式相同,但内容要明显不同,以便暴露形式的缺陷
\item \textbf{结论明显性}:反驳论证的结论必须明显错误或不可接受
\item \textbf{普遍适用性}:一旦证明形式有缺陷,所有具有该形式的论证都被否定
\end{itemize}

\textbf{哲学基础}:
这种方法体现了形式逻辑的核心洞察——论证的有效性完全取决于其逻辑结构,而非具体的经验内容。这一原理可以追溯到亚里士多德的三段论理论。
\end{theorembox}

\subsection{逻辑类推在演绎论证中的应用}

\begin{theorembox}[title=演绎论证中的逻辑类推反驳]
在演绎情况下,对一给定论证进行反驳性的类推是这样:其形式与给定论证一样,但反驳用的类推的前提为真而结论为假。由于用来反驳的类推是无效的,因而遭攻击的论证也是无效的——因为它具有相同的形式。

\textbf{演绎反驳的逻辑机制}:
\begin{itemize}
\item \textbf{形式保持}:反驳论证必须与原论证具有完全相同的逻辑形式
\item \textbf{真假对比}:反驳论证的前提为真,结论为假,从而证明形式无效
\item \textbf{必然推论}:如果反驳论证无效,原论证也必然无效
\item \textbf{普遍否定}:所有具有该形式的论证都被同时否定
\end{itemize}

\textbf{理论基础}:这里的原理与在6.2节中阐述的作为检验直言三段论的基础的原理是一样的,该原理同样是我们在8.4节中反复强调的逻辑形式的基础。

\textbf{方法的严格性}:在演绎情况下,逻辑类推反驳具有决定性的力量——一旦成功,就能够确定地证明原论证的无效性。
\end{theorembox}

\subsection{逻辑类推在归纳论证中的应用}

\begin{theorembox}[title=归纳论证中的逻辑类推反驳]
在归纳论证情况下,我们目前所考虑的逻辑类推的反驳技术同样可以是有效力的。许多在科学中、政治中或经济中的论证并不宣称是演绎的,它们会受到这样的反驳:它们与其他的论证具有十分类似的结构,而这些其他的论证的结论是错误的,或者被普遍地认为是不可能的。

\textbf{归纳论证的特殊性}:
归纳论证本质上不同于演绎论证,差别在于前提给结论所提供的支持程度不同。但是所有的论证,无论是归纳的还是演绎的,具有同样基本的形式或模式。

\textbf{归纳反驳的策略}:
当我们要攻击一个归纳论证时,我们也可以提出另外一个具有同样形式的归纳论证,但是该论证明显有缺陷,因而结论十分可疑,这样的话,我们同样怀疑待考察论证的结论。

\textbf{归纳反驳的特点}:
\begin{itemize}
\item \textbf{或然性质}:不能确定地证明原论证错误,只能质疑其可靠性
\item \textbf{程度差异}:反驳的力度取决于类比的准确性和反例的明显性
\item \textbf{语境依赖}:在不同的学科和实践领域中,反驳的标准可能不同
\item \textbf{可辩驳性}:对手可能质疑类比的准确性或相关性
\end{itemize}
\end{theorembox}

\subsection{经典案例:滑坡论证的逻辑类推反驳}

\begin{examplebox}[title=滑坡论证的结构分析]
考虑下面的例子。反对安乐死合法化的一个通常论证被称为"滑坡"论证("slippery slope" argument)。

\textbf{滑坡论证的基本结构}:
\begin{enumerate}
\item \textbf{初始行为}:授权医生进行某种道德上可疑的行为(安乐死)
\item \textbf{因果链条}:这将使该类型的不道德行为加剧和增多
\item \textbf{不可控性}:一旦迈出第一步,将无法停止
\item \textbf{结论}:因此应当避免初始行为,即使其初衷是仁慈的
\end{enumerate}

\textbf{滑坡论证的一般形式}:
如果允许A,就会导致B,B会导致C,...,最终导致不可接受的Z。因此,不应该允许A。
\end{examplebox}

\begin{theorembox}[title=针对滑坡论证的逻辑类推反驳]
针对这个论证,一个当代的批评家是这样回应的:

"滑坡论证尽管有影响,但它是难以站住脚的。它提出,一旦我们允许医生在病人的请求下结束病人的生命,医生可能并且也会肆意杀害那些并不想死的累赘病人。这个想法得不到辩护……内科医师如果开出的药物剂量大于规定剂量,这会将病人杀害。但在实际中没有人害怕医生会使用大剂量药物而置人于死地。没有人因恐惧'滑坡'而反对医生开处方。授权医生帮助那些求援的病人以结束他们的生命,会产生医生结束那些不想死的病人的生命的结果,这无异于说,授权进行手术以切除肿瘤会导致切除病人的心脏的结果。"\cite{rachels1991}

\textbf{反驳的逻辑结构分析}:
\begin{itemize}
\item \textbf{类比对象1}:处方权的授予与可能的滥用
\item \textbf{类比对象2}:手术权的授予与可能的滥用
\item \textbf{共同形式}:授权→可能滥用→应该禁止授权
\item \textbf{反驳效果}:显示这种推理形式在其他情况下导致荒谬结论
\end{itemize}
\end{theorembox}

\begin{examplebox}[title=反驳论证的详细分析]
这是一个在归纳论证中用逻辑类推来进行反驳的极好例子。

\textbf{原论证的重构}:
先摆出要驳斥的论证:如果我们给了医生帮助病人结束他们的生命的权力,一些医生将肆意滥用这种权力。因而,这个论证得出,我们在这条路上一步都不应当迈出,我们应当拒绝给任何医生帮助病人结束他们生命的权力。

\textbf{反驳论证的构造}:
针对这个论证,一个形式一样的类比性的反驳得以提出,它建立在由归纳得来的对医生行为的公共认识之上:我们确实给了医生可能被滥用的特权。我们给了医生开出危险剂量药物的特权,而使用低剂量药物能够是有益的,当然我们知道他们能够开出伤害病人的大剂量药物。但是开出这样剂量药物的特权的滥用有这样的后果,这个事实丝毫不会使我们后悔我们授予医生这样的特权。

\textbf{反驳的逻辑力量}:
因此,可以看到的是,从授予医生以安乐死的特权的可能滥用,到授予医生以处方特权的可能滥用(反驳提出的)的这个论证表明,滑坡论证就它对于医生而言不是十分有说服力的。

\textbf{反驳成功的关键因素}:
\begin{itemize}
\item \textbf{形式同构}:两个论证具有完全相同的逻辑结构
\item \textbf{经验基础}:反驳论证基于广泛接受的经验事实
\item \textbf{结论荒谬}:反驳论证的结论明显不可接受
\item \textbf{类比恰当}:医生的不同权力之间具有相关的相似性
\end{itemize}
\end{examplebox}

\begin{theorembox}[title=第二个类推反驳的分析]
上面引用的文章中也提供了另外一个类推反驳,它在形式上十分相似:假如说,给予医生特权,协助那些企求帮助的病人结束他们的生命(根据待反驳的论证)将会导致结束那些确实不想死的病人的特权的产生,那么,在这种情况之中,不单单医生而且立法机构都在陡坡上滑动。

\textbf{第二个反驳类推}:现在普遍地给予医生切除某个身体器官的特权,当然是在病人的同意之下。得出这样的结论是荒唐的:这种特权会导致人们(立法者或医生)认为,该特权包含了在没有准许的情况下切除某个功能正常的其他器官的权力。

\textbf{多重反驳的策略意义}:
\begin{itemize}
\item \textbf{强化效果}:多个类推反驳相互支持,增强整体反驳力度
\item \textbf{覆盖面广}:不同的类比涵盖了原论证的不同方面
\item \textbf{降低反击}:即使一个类比被质疑,其他类比仍然有效
\item \textbf{形式确认}:多个成功的类比确认了对原论证形式的准确把握
\end{itemize}
\end{theorembox}

\subsection{逻辑类推反驳的方法特征与争议}

\begin{theorembox}[title=方法的核心特征与潜在争议]
在这种驳斥之中,焦点在于论证形式。滑坡论的辩护者可能这样回击我们上面引述的抨击:这里的类推反驳是不成功的。理由是,反驳的形式没有正确地反映原来论证的形式。无疑,这个争论将会继续下去。

\textbf{方法的重大意义}:但是,这里使用的逻辑技术具有重大的意义:如果一个论证确实与被反驳的另外一个论证具有相同的形式,并且,用来作为类推的论证明显是糟糕的,那么,可以确定的是,被反驳的论证遭到了破坏。

\textbf{潜在的争议点}:
\begin{itemize}
\item \textbf{形式识别}:如何准确识别和描述论证的逻辑形式
\item \textbf{类比恰当性}:反驳论证是否真正具有相同的形式
\item \textbf{相关性判断}:类比中的差异是否影响形式的同一性
\item \textbf{语境因素}:不同领域的特殊性是否影响形式的适用性
\end{itemize}

\textbf{方法的局限性}:
逻辑类推反驳的成功完全依赖于形式同构的准确性,这往往成为争议的焦点。
\end{theorembox}

\subsection{逻辑类推反驳的语言特征与识别标志}

\begin{theorembox}[title=逻辑类推反驳的语言标志]
在归纳情景下与在演绎情景下一样,一个由逻辑类推进行的反驳常常包含这样的句子"你不如说……",或其他与之有相同意思的语言。

\textbf{常见的提示性语言}:
\begin{itemize}
\item "你不如说……"
\item "这无异于说……"
\item "这就像说……"
\item "按照这种逻辑……"
\item "如果这样的话……"
\end{itemize}

\textbf{语言功能分析}:
这些提示性语言的作用是引导听众注意到论证形式的相似性,为即将展示的类比做好心理准备。
\end{theorembox}

\begin{examplebox}[title=文化论证的类推反驳案例]
在上述引用的段落中,其破坏性类推提示性的语言是"这无异于说"。一个学者攻击如下论证:因为伊斯兰文化是从外面被带到乍得,在乍得它只不过是一个装饰。他说,"[你说]乍得仅有一个'伊斯兰装饰'。人们也会敏锐地说法国仅有一个'基督教装饰'"。\cite{brenner1993}

\textbf{这个反驳的逻辑分析}:
\begin{itemize}
\item \textbf{原论证形式}:文化X从外部传入国家Y → 文化X在国家Y只是装饰
\item \textbf{反驳论证}:基督教从外部传入法国 → 基督教在法国只是装饰
\item \textbf{反驳效果}:显示原论证形式导致明显荒谬的结论
\end{itemize}

在这个类推反驳中所用的提示性语言稍有不同,但功能相同。
\end{examplebox}

\begin{examplebox}[title=政治论证的类推反驳案例]
在类推反驳很明显的地方,不需要提示性的语言。前密西西比州州长柯尔克-福迪斯争辩道:"这是一个简单的事实,美国是一个基督教国家。"因为"在美国基督教是主要的宗教"。与他进行电视辩论的记者迈克尔-金塞以生动的类推进行了回击:"本国妇女占大多数,这能够使我们得出我国是女性国家吗?再者,我们能够因我国的大多数人是白人而得出,我国是一个白人国家吗?"\cite{kinsley1992}

\textbf{这个反驳的精妙之处}:
\begin{itemize}
\item \textbf{形式完全相同}:群体X在国家Y占多数 → 国家Y是X国家
\item \textbf{反例明显}:女性占多数、白人占多数的结论都明显荒谬
\item \textbf{无需提示语}:类比如此明显,直接展示即可
\item \textbf{双重反驳}:用两个类比增强反驳力度
\end{itemize}
\end{examplebox}

\begin{center}
\fbox{\parbox{0.95\textwidth}{
\textbf{本节要点}
\begin{itemize}
\item \textbf{逻辑类推反驳法的理论基础}:
  \begin{itemize}
  \item 基于形式逻辑的核心洞察——论证有效性完全取决于逻辑结构
  \item \textbf{四大核心要素}:形式同构性、内容差异性、结论明显性、普遍适用性
  \item 体现了从亚里士多德三段论理论到现代形式逻辑的一致性原理
  \item 文学作品中的生动展示:《爱丽丝漫游奇境记》的逻辑智慧
  \end{itemize}
\item \textbf{在演绎论证中的严格应用}:
  \begin{itemize}
  \item \textbf{逻辑机制}:形式保持、真假对比、必然推论、普遍否定
  \item 构造形式相同但前提为真、结论为假的论证
  \item 具有决定性的反驳力量——一旦成功就能确定地证明原论证无效
  \item 与直言三段论检验原理和逻辑形式基础理论一致
  \end{itemize}
\item \textbf{在归纳论证中的灵活应用}:
  \begin{itemize}
  \item \textbf{特殊性质}:或然性质、程度差异、语境依赖、可辩驳性
  \item 不能确定地证明原论证错误,只能质疑其可靠性
  \item 反驳力度取决于类比的准确性和反例的明显性
  \item 在科学、政治、经济等领域广泛应用
  \end{itemize}
\item \textbf{滑坡论证反驳的经典案例分析}:
  \begin{itemize}
  \item \textbf{滑坡论证结构}:初始行为→因果链条→不可控性→禁止结论
  \item \textbf{反驳策略}:医生处方权类比、手术权类比等多重反驳
  \item \textbf{成功要素}:形式同构、经验基础、结论荒谬、类比恰当
  \item 多重反驳的策略意义:强化效果、覆盖面广、降低反击、形式确认
  \end{itemize}
\item \textbf{方法特征与潜在争议}:
  \begin{itemize}
  \item \textbf{核心争议点}:形式识别、类比恰当性、相关性判断、语境因素
  \item 方法的重大意义:如果形式确实相同且反驳论证明显糟糕,则原论证被破坏
  \item 方法的局限性:成功完全依赖于形式同构的准确性
  \item 争议的不可避免性:形式同构的判断往往成为辩论焦点
  \end{itemize}
\item \textbf{语言特征与识别标志}:
  \begin{itemize}
  \item \textbf{常见提示语}:"你不如说"、"这无异于说"、"这就像说"等
  \item 语言功能:引导听众注意论证形式的相似性
  \item 明显类比无需提示语:如政治论证中的双重反驳案例
  \item 不同领域的应用:文化论证、政治论证等多样化案例
  \end{itemize}
\item \textbf{方法的哲学意义与实用价值}:
  \begin{itemize}
  \item 体现了形式逻辑在实际论辩中的强大威力
  \item 是批判性思维和理性论辩的重要工具
  \item 在学术争论、法庭辩论、政治讨论中广泛应用
  \item 培养了识别和分析论证形式的重要能力
  \end{itemize}
\end{itemize}
}}
\end{center}

% 参考文献将在主文档末尾统一显示

% 第十二章
\chapter{逻辑悖论}
\section{因果连接:基本概念}

\begin{logicbox}[title=引言]
本节深入探讨\logicterm{因果关系}的基本概念及其在逻辑推理中的核心地位。因果关系作为人类认识世界和改造世界的基础,不仅是科学研究的核心问题,更是日常生活中推理和决策的重要依据。我们将系统分析"\logicterm{原因}"的多种含义,深入区分\logicterm{必要条件}与\logicterm{充分条件}的哲学意义,考察\logicterm{因果律}与\logicterm{自然齐一性}的认识论基础,并详细介绍\logicterm{简单枚举归纳法}作为建立\logicterm{因果关系}的基本方法。通过理解这些概念的深层内涵和相互关系,我们将能够更\logicemph{有效地}分析和评价各种\logicterm{归纳论证}中的\logicterm{因果推理},为科学方法论和实践决策提供坚实的理论基础。
\end{logicbox}

\subsection{因果关系的哲学地位与认识论意义}

\begin{theorembox}[title=因果关系在人类认识中的核心地位]
因果关系是人类认识世界的最基本范畴之一,其重要性体现在以下几个方面:

\textbf{1. 认识论基础}:因果关系是人类理解世界运行规律的基本框架,为科学知识的建构提供了根本性的概念工具。

\textbf{2. 实践指导}:通过掌握因果关系,人类能够预测未来、控制环境、实现目标,这是一切技术和工程活动的基础。

\textbf{3. 逻辑推理}:因果推理是归纳逻辑的核心内容,为从经验事实推出一般规律提供了基本方法。

\textbf{4. 科学方法}:现代科学的实验方法、假说检验、理论建构都建立在因果关系分析的基础之上。

\textbf{5. 社会实践}:法律责任、道德判断、政策制定等社会活动都离不开对因果关系的准确把握。
\end{theorembox}

\begin{examplebox}[title=因果关系在不同领域的重要性]
\textbf{医学领域}:为了治疗某种疾病,医生必须知道它的\logicterm{原因};并且,他们应当了解他们所用药物的后果(包括副作用)。

\textbf{工程技术}:工程师需要理解材料性能与处理工艺之间的因果关系,以设计出符合要求的产品。

\textbf{经济学}:经济政策的制定需要准确把握各种经济变量之间的因果关系。

\textbf{环境科学}:环境保护和治理需要深入理解人类活动与环境变化之间的因果机制。

\textbf{教育学}:有效的教学方法需要基于对学习过程中各种因素与学习效果之间因果关系的理解。
\end{examplebox}

\subsection{"\logicterm{原因}"的意义}

为了对环境进行控制性操作,我们必须拥有某种\logicterm{因果连接}的知识。因和果之间的关系其重要性非同一般。然而,这种关系因为"\logicterm{原因}"一词有多种含义而易于混淆。因而,我们先区分这些含义。

\begin{theorembox}[title=必要条件与充分条件的深入分析]
在对自然的研究中一个基本的公设是,只有在确定的条件下事件才能发生。人们习惯于区分事件发生的\logicterm{必要条件}和\logicterm{充分条件}。

\textbf{必要条件的定义与特征}:
一个特定事件发生的\logicterm{必要条件}是指,在缺乏它的情况下,该事件不能发生。例如,具有氧气是燃烧能够发生的\logicterm{必要条件}:如果燃烧发生,必须具有氧气,因为在缺乏氧气的情况下便没有燃烧。

\textbf{必要条件的逻辑形式}:
如果P是Q的必要条件,则:
\begin{itemize}
\item 逻辑表达:$Q \rightarrow P$(如果Q发生,则P必须存在)
\item 等价表达:$\neg P \rightarrow \neg Q$(如果P不存在,则Q不会发生)
\item 这体现了必要条件的"否定后件推否定前件"的逻辑特征
\end{itemize}

\textbf{必要条件的认识论意义}:
\begin{itemize}
\item \textbf{排除功能}:通过消除必要条件可以阻止不希望的事件发生
\item \textbf{诊断价值}:必要条件的缺失可以排除某种可能性
\item \textbf{预防作用}:在医学、工程等领域,控制必要条件是预防问题的重要手段
\end{itemize}
\end{theorembox}

\begin{theorembox}[title=充分条件的深入分析]
尽管具有氧气是一个必要条件,但它不是燃烧能够发生的充分条件。

\textbf{充分条件的定义与特征}:
一个事件能够发生的充分条件是,在它出现的情况下事件必定发生。因为在有氧气的情况下也可能不发生燃烧,所以,出现氧气不是燃烧的充分条件。另一方面,对几乎每一种物质而言,都存在某个温度范围,在该温度范围里具有氧气是该物质燃烧的充分条件。

\textbf{充分条件的逻辑形式}:
如果P是Q的充分条件,则:
\begin{itemize}
\item 逻辑表达:$P \rightarrow Q$(如果P存在,则Q必定发生)
\item 这体现了充分条件的"肯定前件推肯定后件"的逻辑特征
\item 充分条件保证了结果的必然性
\end{itemize}

\textbf{必要条件与充分条件的关系}:
\begin{itemize}
\item 一个事件的发生可能有多个必要条件
\item 这些必要条件均包含在充分条件里
\item 充分条件通常是多个必要条件的联合
\item 在理想情况下,充分条件等于所有必要条件的完整集合
\end{itemize}

\textbf{充分条件的认识论意义}:
\begin{itemize}
\item \textbf{预测功能}:掌握充分条件使我们能够可靠地预测结果
\item \textbf{控制作用}:通过创造充分条件可以确保期望事件的发生
\item \textbf{技术应用}:工程技术中的设计往往基于对充分条件的掌握
\end{itemize}
\end{theorembox}

\begin{examplebox}[title=法律论证中的必要条件与充分条件]
必要和充分条件的区分在法律论证中经常起关键作用。在美国高等法院中,一名法官最近争辩说,州立基金被用于资助宗教协会时必须满足两个条件:必须是公平的——它的发放是中立的,而不对任何一个宗教有所偏爱;必须是间接的——因为宪法禁止宗教协会从政府直接获得资助。

\textbf{法官的论证分析}:
"在每个资助中,资助是广泛的和中立的,这个事实是资助程序中的必要条件。但是公正的意义失去了。在每个资助情况中,我们没有说,该条件就是充分的,或者说决定性的。情况完全相反。这些资助中对我们决策起决定作用的是这样的事实:资助是间接的;资助到达宗教组织完全是受资助者的完全独立的和私人的选择。"

\textbf{案例的逻辑结构}:
这个法官做出这样的区别,是因为在手头案子中(这是关于一所州立大学拒绝了为一个学生宗教社团付印刷费的案子),争议中的州资助即使公平地给予,它也是直接的,因此他认为是不允许的。在这名法官看来,资助的接受程序是两条必要条件,其中能够满足的只有一条。

\textbf{法律推理的逻辑特征}:
\begin{itemize}
\item \textbf{必要条件的累积性}:多个必要条件必须同时满足
\item \textbf{充分条件的整体性}:仅满足部分必要条件不构成充分条件
\item \textbf{法律适用的严格性}:缺少任何一个必要条件都会导致法律后果的改变
\item \textbf{条件分析的重要性}:准确区分必要条件与充分条件对法律判决至关重要
\end{itemize}
\end{examplebox}

\subsection{原因概念的多重含义与语境依赖性}

\begin{theorembox}[title=原因概念的三种基本含义]
"原因"有时是在"必要条件"的意义上使用,而有时是在"充分条件"的意义上使用。当手边的问题是要淘汰不受欢迎的现象时,它更多地是在"必要条件"的意义上使用。

\textbf{1. 原因作为必要条件}:
为了淘汰某个现象,人们只要找到某个对该现象的存在为必需的条件,然后将该条件淘汰。医生努力寻找何种微生物是某个疾病的"原因",以便开出杀灭那些微生物的药物,从而治愈该疾病。那些微生物被认为是该疾病的原因,是说它们是疾病的必要条件——因为如果没有它们便不会有该疾病。

\textbf{2. 原因作为充分条件}:
当我们对某个希望发生的事情感兴趣的时候(而不是淘汰不希望的事情),我们是在"充分条件"的意义上使用"原因"一词的。冶金专家的目标是发现什么使金属合金具有更大的强度,如果我们找到了这样的一个热处理和冷处理的复合过程(该过程使得金属具有我们希望的结果),我们说,这样的一个过程是合金强度增高的原因。

\textbf{3. 原因作为关键因素}:
"原因"一词有另外一个普遍的但不精确的用法,该用法与充分条件的含义密切相关。一给定现象与某些后果关联,它可能便是原因。例如,我们断定"吸烟导致癌症"。当我们这样说时,可以肯定的是,我们并没有说吸烟是癌症的必要条件。因为我们知道许多癌症是在完全没有吸烟的情况下得的。同样不能说吸烟必定产生癌症,因为可能的是,某些人的长期吸烟的习惯并没有带来癌症后果。但是,吸烟,与某些生物环境相结合,在癌症的发展中频繁地发挥作用,以至于我们合理地认为吸烟为癌症的一个"原因"。
\end{theorembox}

\begin{examplebox}[title=生物学中的多重必要条件案例]
忽视这种意义的原因将导致无谓的争论。某种动物行为的真正原因是它的基因还是环境?当然,大多数情况下两者都起作用;当两者都不能独自解释该行为的时候,两者都是本质的。

\textbf{鸣鸟唱歌的复合原因分析}:
在鸣鸟群里,通常只有雄鸣鸟唱歌。当科学家使幼小的雄鸣鸟不再产生睾丸激素后,它们不再能唱歌。但是,如果它们在其幼年的某个阶段没有听到周边其他鸣鸟的鸣唱,它们也不能唱歌。一个雄鸟听到一首歌,该歌开启了一个用睾丸激素以唱歌的方式建立脑神经的过程。本性和养育两者均是鸟能够唱歌的必要条件。\cite{marler1991}

\textbf{多重必要条件的认识论启示}:
\begin{itemize}
\item \textbf{因果复杂性}:现实中的因果关系往往涉及多个必要条件
\item \textbf{简化的危险}:将复杂现象归因于单一原因可能导致错误理解
\item \textbf{系统思维}:需要从系统的角度理解多因素的相互作用
\item \textbf{实践意义}:在实际应用中需要同时控制多个必要条件
\end{itemize}
\end{examplebox}

\begin{examplebox}[title=保险调查中的原因概念]
这产生了"原因"的另外一个用法:作为某个现象发生过程中的关键因素或常常是关键因素。

\textbf{保险火灾调查案例}:
假定一家保险公司派遣调查员弄清一场神秘火灾的原因。如果调查员报告说火灾是由空气中的氧气所致,那么调查员的工作将不保。尽管他们是对的——在必要条件的含义上。因为如果不存在氧气,火灾便不可能发生。然而,保险公司派遣他们去调查,不是打算为了弄清该种含义上的原因。

保险公司也不对充分条件感兴趣。如果经过几个星期后调查员汇报说,尽管他们已经证明火是由投保的客户有意点燃的,但他们还不能够知道所有必要条件,因而仍然不能确定(充分条件含义上的)原因,此时,公司将打电话给他们,告诉他们别再浪费时间和金钱。

\textbf{实用原因概念}:
保险公司是在另外一种意义上使用"原因"一词:他们希望查找的是,在现有的条件之下造成该事件出现或不出现的差别的事件或行为是什么。

\textbf{原因概念的语境依赖性}:
\begin{itemize}
\item \textbf{理论语境}:科学研究中追求完整的必要条件或充分条件
\item \textbf{实践语境}:日常生活中关注关键的差异性因素
\item \textbf{法律语境}:法律责任认定中的原因概念有特定要求
\item \textbf{技术语境}:工程应用中的原因概念侧重于可控因素
\end{itemize}
\end{examplebox}

\subsection{遥远与最近的原因:因果链条的复杂性}

\begin{theorembox}[title=因果链条中的原因层次]
我们对第三种含义的原因做两个区分。传统上人们将它们称为遥远的(remote)和最近的(proximate)原因。

\textbf{因果链条的基本结构}:
几个事件组成的一个因果序列或链条:$A$ 引起 $B, B$ 引起 $C, C$ 引起 $D, D$ 引起 $E$,此时我们将 $E$ 称为先行事件的结果。其中最近的即 $D$,为 $E$ 的最近的原因,而其他的为 $E$ 越来越遥远的原因:$A$ 比 $B$ 遥远,$B$ 比 $C$ 遥远。

\textbf{时间与因果距离的复杂关系}:
尽管如此,由于因果链条的连接数量,在时间上与结果十分接近的原因可能在距离上是遥远的。这种现象在现代信息社会中尤为突出,全球化的经济和通信网络使得遥远的事件能够迅速产生连锁反应。

\textbf{原因层次的认识论意义}:
\begin{itemize}
\item \textbf{解释的层次性}:不同层次的原因提供不同深度的解释
\item \textbf{干预的针对性}:最近的原因通常是最容易干预的
\item \textbf{预防的根本性}:遥远的原因往往是预防问题的关键
\item \textbf{责任的分配}:法律和道德责任的认定需要考虑原因的层次
\end{itemize}
\end{theorembox}

\begin{examplebox}[title=全球经济中的快速因果传导]
下述是对出现在1996年的一件事情的真实解释:

事件的导火索起因于两年前六月份的一个早晨。巴西经过了一整夜的霜冻后,一个政府官员宣布减少计划中的咖啡生产产量。该消息立刻传到芝加哥贸易部,该处咖啡的期货价格立即攀升。大豆和其他物品的商人立刻抬高价格,导致物品价格指数上升。这一切均记录在商人们的计算机屏幕上。这些商人分布在几乎200个华尔街公司里,他们将通货膨胀情况汇报给他们的合约—贸易伙伴,这些伙伴开始抛售合约,这导致合约价格下降,合约价格下降导致合约产量上升,合约产量上升给利息率的升高增加了向上的力量,这个力量造成股票价格下降。在巴西公告发出和华尔街股票波动之间的时间间隔不会超过10分钟。\cite{nasar1996}

\textbf{现代因果链条的特征}:
\begin{itemize}
\item \textbf{传导速度}:信息技术使因果传导几乎瞬时完成
\item \textbf{全球联系}:地理上遥远的事件可以产生直接影响
\item \textbf{复杂网络}:多个中介环节形成复杂的因果网络
\item \textbf{放大效应}:小的初始变化可能产生巨大的最终影响
\end{itemize}
\end{examplebox}

\begin{examplebox}[title=教育与健康的多层次因果关系]
一件研究表明,"教育是健康最重要的关联因素。接受较多的教育的人变得更为有所了解;他们了解医学技术,医疗,保险以及保健服务系统的重要性。他们更为有能力地从医疗系统获得有价值的服务...尽管遭遇同样的疾病,接受较少教育的人往往接受较差的医疗。"\cite{kitagawa1973}

\textbf{因果层次分析}:
但是上大学不是健康的最近的原因,无知也不是疾病的最近的原因。落后的教育是在该因果链条中的一个环节,它往往造成对疾病过程不恰当的理解,因而,较好的医学后果所需的生活方式难以建立。因此人们普遍并正确地观察到,贫困,它广泛地对教育产生影响,它是缺乏健康的一个"根本原因"——个遥远的而不是最近的原因。

\textbf{社会因果关系的复杂性}:
\begin{itemize}
\item \textbf{根本原因}:贫困作为影响多个中介因素的基础性因素
\item \textbf{中介机制}:教育水平影响健康知识和行为选择
\item \textbf{直接因素}:具体的生活方式和医疗行为
\item \textbf{政策意义}:不同层次的干预策略具有不同的效果
\end{itemize}
\end{examplebox}

\begin{theorembox}[title=原因概念的逻辑总结]
我们已经看到,"原因"一词的含义存在几种。

\textbf{推理方向的逻辑限制}:
\begin{itemize}
\item 我们仅能够在"必要条件"的含义上合法地从结果中推出原因
\item 我们仅能够在"充分条件"的含义上合法地从原因中推出结果
\end{itemize}

\textbf{双向推理的条件}:
当我们从原因推论到结果并且从结果推论到原因时,原因必定是在既充分又必要条件的意义上使用的。在这种用法中,原因等同于充分条件,而充分条件被认为是所有必要条件的联合。

\textbf{概念的多义性}:
应当清楚的是,不存在符合该词的所有不同用法的单个"原因"定义。这种多义性反映了因果概念在不同语境中的丰富性和复杂性。

\textbf{哲学意义}:
原因概念的多义性提醒我们,在进行因果分析时必须明确我们所使用的"原因"概念的具体含义,避免概念混淆导致的逻辑错误。
\end{theorembox}

\subsection{因果律和自然的齐一性:从特殊到普遍的认识论问题}

\begin{theorembox}[title=自然齐一性原则的基础地位]
但是"原因"一词的每一种用法,无论是在日常生活中的还是在科学中的,都与下述原则相关,或预设了下述原则:原因和结果齐一地(uniformly)相连。

\textbf{齐一性原则的内容}:
我们说,一个特定事态造成了一个特定结果,即是说该类型的其他事态(在产生该事态充分类似的条件下)将造成与先前结果同种类型的结果。换句话说,同类原因导致同类结果。

\textbf{因果律的普遍性特征}:
我们今天使用的"原因"一词的部分意义是,一个原因产生一个结果的每一次出现,都是普遍因果律——如此的事态总是伴随着如此的现象——的一个实例或一个事例。于是,如果在另外的情形下出现了与事态$C$同类的事态,但是结果$E$并不发生,此时我们不认为事态$C$是在一个特定场合下结果$E$的原因。

\textbf{普遍性的逻辑要求}:
因为特定事态是特定现象的原因的每一个断定意味着存在某个因果律,每一个因果连接的断定都包含与普遍性(generality)有关的一个关键成分。因果律——当我们使用该术语的时候——断定,如此这般的事态下恒常地伴随着一个特定种类的现象,而无论该事态发生于何时何地。

\textbf{认识论的根本问题}:
但是我们如何知道这样普遍性的真理呢?这是归纳逻辑面临的核心问题。
\end{theorembox}

\begin{theorembox}[title=从经验到普遍性的认识论挑战]
\textbf{因果关系的经验性质}:
因果关系不是纯粹逻辑的或演绎的,它不能被任何先验的论证所发现。因果律只能经验地或后验地(即诉诸经验)发现。

\textbf{经验的局限性}:
但是我们的经验总是与特定情形、特定现象以及现象的特定次序有关。我们能够观察到一个特定事态(比如$C$)下的几个事例,我们观察到的事例也能够被一个特定种类现象(如$P$)的一个事例所伴随。

\textbf{归纳推理的挑战}:
但是我们未来能够经历的仅仅是世界上事态$C$中的一些事例,这些观察能够展示给我们的仅仅是$P$伴随着$C$的一些事例。然而,我们的目标是建立一个普遍的因果关系。我们如何能够从我们经历的特定事例中,得到$C$的所有场合下都有$P$这样普遍性的命题($C$引起$P$)?

\textbf{休谟问题的核心}:
这就是著名的"休谟问题"——从有限的经验如何推出无限的普遍性?这个问题至今仍是认识论和科学哲学的核心议题。
\end{theorembox}

\subsection{简单枚举归纳法:从特殊到普遍的基本方法}

\begin{theorembox}[title=归纳概括的理论基础]
从特定经验事实中得到一般或普遍命题的过程被称做归纳概括。这一过程体现了人类认识从感性到理性、从个别到一般的基本规律。

\textbf{归纳概括的两种结果}:
从三张蓝色石蕊试纸放到酸中都变红的前提中,我们或者会得到一个特定结论——将第四张蓝色石蕊试纸放到酸中它将发生什么样的现象,或者会得到一个普遍结论——每一张蓝色石蕊试纸放到酸中将发生什么。

\textbf{类比论证与归纳论证的区别}:
如果我们得到第一个,我们就使用了一个类比论证;如果是第二个,则为一个归纳论证。这两个论证类型的结构在下面得到分析。

\textbf{逻辑结构的对比}:
\begin{itemize}
\item \textbf{类比论证}:前提反映的是两个属性(或情形或现象)共同发生的事例,由类比我们可以推得,在具有一个属性的其他事例中也会出现另外的属性
\item \textbf{归纳概括}:我们能够推得,一个属性出现其中的每一个事例将同时也是另外属性的事例
\end{itemize}

\textbf{简单枚举归纳法的定义}:
这种形式的归纳概括就是简单枚举归纳法。简单枚举归纳法非常类似于类比论证,所不同的只是它形成的结论更为普遍。

\textbf{认识论意义}:
简单枚举归纳法代表了人类认识中从特殊到一般的最基本形式,是科学发现和日常经验总结的重要方法。
\end{theorembox}

\begin{theorembox}[title=简单枚举归纳法的形式结构]
\textbf{标准形式}:
\begin{displayquote}
现象 $E$ 的事例 1 伴随有事态 $C$\\
现象 $E$ 的事例2伴随有事态 $C$\\
现象 $E$ 的事例 3 伴随有事态 $C$\\
$\vdots$\\
现象 $E$ 的事例 n 伴随有事态 $C$
\end{displayquote}

因而现象 $E$ 的每个事例都伴随有事态 $C$。

\textbf{逻辑特征分析}:
\begin{itemize}
\item \textbf{前提的有限性}:前提只包含有限数量的观察事例
\item \textbf{结论的普遍性}:结论声称对所有可能的事例都成立
\item \textbf{推理的跳跃}:从"一些"到"所有"的逻辑跳跃
\item \textbf{或然性质}:结论具有或然性而非必然性
\end{itemize}

\textbf{因果关系的建立}:
我们经常用简单枚举法建立因果连接。当一种现象的许多事例恒常地伴随着一特定类型的事态的时候,我们自然地得出在它们之间存在一个因果关系。

\textbf{具体应用实例}:
将蓝色石蕊试纸放进酸中的情形在所有观察中都伴随有试纸变红现象,我们由简单枚举法得到,将蓝色石蕊试纸放进酸中是它变红的原因。在这样论证中的类比特征相当明显。
\end{theorembox}

\begin{theorembox}[title=简单枚举归纳法的评价标准]
由于简单枚举法和类比论证之间有很大的类似性,类似的评价标准都适合它们。

\textbf{数量标准的应用}:
某些简单枚举法论证能够比其他的论证建立较高盖然度的结论。举出的事例数越多,结论成真的概率就越高。

\textbf{确证事例的概念}:
伴随着事态 $C$ 的不同事例或场合,往往被称做断定 $C$ 引起 $E$ 的因果律的确证事例。确证事例数越多,若其他事态不变的话,因果律为真的概率越高。

\textbf{评价标准的移植}:
于是,用于类比论证的第一个标准可直接应用于简单枚举归纳法论证。

\textbf{其他相关标准}:
\begin{itemize}
\item \textbf{事例的多样性}:不同条件下的事例增强论证强度
\item \textbf{观察的精确性}:准确的观察提高论证可靠性
\item \textbf{时间的跨度}:长期观察增加论证的稳定性
\item \textbf{反例的缺失}:没有发现反例增强论证可信度
\end{itemize}
\end{theorembox}

\begin{examplebox}[title=历史推理中的简单枚举归纳法:卡那尔文伯爵的经典案例]
在历史报告中简单枚举法可以为一个因果关系的建立提供论证基础。

\textbf{历史背景}:
对某个个体或群体与其财产或权利进行暂时性的强制分离的司法行为,被称为财产和公民权剥夺法案;熟知的是,当政治权力的钟摆发生摆动时,该司法行为对该法案的鼓吹者也会造成危险;今天的原告明天会成受害人。

\textbf{卡那尔文伯爵的论证}:
卡那尔文伯爵为了指控上议院针对托马斯•奥斯本的这种法案,在1678年用下面的枚举法阐明其观点:

"大人们,从不少的英国历史中我了解到这些检举的危害以及检举人的悲惨命运。我将追溯到伊丽莎白女王统治的晚期而不是更远,当时埃塞克斯伯爵被瓦尔特•拉莱爵士所检举,大人们,你们很清楚拉莱发生了什么。培根大人检举了瓦尔特•拉莱爵士。大人们,你们清楚培根大人发生了什么。巴金汗侯爵检举了培根大人。大人们,对巴金汗侯爵的命运,你们是清楚的。托马斯•文特沃斯爵士然后是斯特拉福特伯爵,检举了巴金汗侯爵。你们都知道斯特拉福特伯爵的命运。哈瑞•凡恩爵士检举了斯特拉福特伯爵,大人们,你们知道哈瑞•凡恩爵士如何了,海德大臣检举了他。你们清楚海德大臣的命运,托马斯•奥斯本以及现在的旦比伯爵,检举了海德大臣。

旦比伯爵的命运将如何呢,大人们最好能够告诉我。但是让我们看一下,胆敢将旦比伯爵赶下台的人,他的命运将如何。"\cite{roberts1966}

\textbf{论证结构分析}:
\begin{itemize}
\item \textbf{前提模式}:检举者A检举被检举者B → 检举者A最终遭遇不幸
\item \textbf{事例枚举}:拉莱→培根→巴金汗→斯特拉福特→凡恩→海德→旦比
\item \textbf{归纳结论}:检举他人者最终必遭报应
\item \textbf{预测应用}:旦比伯爵和未来的检举者都将面临同样命运
\end{itemize}

\textbf{修辞效果与逻辑力量}:
这个论证具有强烈的修辞效果,通过历史事例的重复强化了因果关系的印象,体现了简单枚举归纳法在政治论辩中的说服力。
\end{examplebox}

\begin{theorembox}[title=简单枚举归纳法的根本局限性]
\textbf{确定性的缺失}:
事例的重复尽管有修辞效果,但它没有提供确定性的论证。恶意指控和随后的垮台之间存在因果关系的结论,诉诸六个确证事例。但是这些事例的本性阻碍了将真实因果律的确证事例和仅仅是历史巧合之间区别开来。

\textbf{反例问题的核心}:
这个困难的核心是:简单枚举法对提出的因果律的例外没有解释,而且不可能有解释。任何断言的因果律都会被一个反例所推翻,因为,任何一个反例表明,所谓的一个"规律"不是真正普遍的。例外否证了该规则。

\textbf{反例的两种形式}:
一个例外(或"反例")或者是这样一个情况:
\begin{itemize}
\item 人们发现了所断言的原因,而断言的结果并没有伴随(在该历史案例中,指控提案的提出者没有发生类似的命运)
\item 结果发生了,但所断言的原因没有发生
\end{itemize}

用前面的图式表示:$C$ 发生而 $E$ 不发生,或 $E$ 发生而 $C$ 没有发生。

\textbf{方法的内在缺陷}:
但在一个简单枚举论证中,这两个情况中的任何一个都是无效的;在这样论证中唯一合法的前提是断言的原因和断言的结果两者都出现的事例报告。

\textbf{选择性偏见的问题}:
如果我们限定我们归纳论证的视野,我们将不去寻找甚至于不去注意那些可能发现的否定的或不确证的事例,这是简单枚举论证的一个严重缺陷。

\textbf{方法的适用范围}:
正因为这一点,简单枚举归纳法尽管在因果律的建立过程中成果丰硕并且具有价值,但它不适合检验因果律。然而这样的检验是至关重要的。为了进行检验,我们必须依赖于其他类型的归纳论证。
\end{theorembox}

\begin{theorembox}[title=简单枚举归纳法的认识论意义与历史地位]
\textbf{认识发展的阶段性}:
简单枚举归纳法代表了人类认识发展的一个重要阶段,它是从经验观察向科学理论发展的必经之路。

\textbf{科学方法的起点}:
虽然存在局限性,但简单枚举归纳法仍然是科学发现的重要起点,许多重要的科学定律最初都是通过简单的事例枚举而发现的。

\textbf{日常认识的基础}:
在日常生活中,简单枚举归纳法仍然是我们形成经验判断和做出实践决策的重要依据。

\textbf{向更高级方法的过渡}:
认识到简单枚举归纳法的局限性,促使人们发展更加精密的归纳方法,如穆勒的归纳方法、统计推断等。

\textbf{哲学反思的价值}:
对简单枚举归纳法的分析揭示了归纳推理的根本问题,为认识论和科学哲学的发展提供了重要启示。
\end{theorembox}

\begin{center}
\fbox{\parbox{0.95\textwidth}{
\textbf{本节要点}
\begin{itemize}
\item \textbf{因果关系的哲学地位与认识论意义}:
  \begin{itemize}
  \item 因果关系是人类认识世界的最基本范畴,为科学知识建构提供根本性概念工具
  \item \textbf{五大重要性}:认识论基础、实践指导、逻辑推理、科学方法、社会实践
  \item 在医学、工程、经济、环境、教育等领域具有核心地位
  \item 体现了人类理解世界运行规律和改造世界的基本需要
  \end{itemize}
\item \textbf{原因概念的多重含义与语境依赖性}:
  \begin{itemize}
  \item \textbf{必要条件}:缺少则事件不能发生,具有排除功能、诊断价值、预防作用
  \item \textbf{充分条件}:出现则事件必定发生,具有预测功能、控制作用、技术应用价值
  \item \textbf{关键因素}:在现有条件下导致事件发生的差异性因素
  \item \textbf{语境依赖性}:理论语境、实践语境、法律语境、技术语境中的不同理解
  \item 推理方向的逻辑限制:必要条件用于从果推因,充分条件用于从因推果
  \end{itemize}
\item \textbf{因果关系的层次性:遥远与最近的原因}:
  \begin{itemize}
  \item 因果链条的基本结构:$A \rightarrow B \rightarrow C \rightarrow D \rightarrow E$
  \item 时间与因果距离的复杂关系:现代信息社会中的快速因果传导
  \item \textbf{原因层次的认识论意义}:解释的层次性、干预的针对性、预防的根本性、责任的分配
  \item 全球经济案例:巴西霜冻→华尔街股市波动的10分钟传导链
  \item 社会因果案例:贫困→教育→健康的多层次因果关系
  \end{itemize}
\item \textbf{因果律和自然的齐一性:认识论的根本问题}:
  \begin{itemize}
  \item \textbf{齐一性原则}:同类原因导致同类结果的普遍性假设
  \item 因果律的普遍性特征:每个特定因果关系都是普遍因果律的实例
  \item \textbf{休谟问题}:从有限经验如何推出无限普遍性的认识论挑战
  \item 因果关系的经验性质:只能后验发现,不能先验推导
  \item 经验局限性与归纳推理挑战的深层分析
  \end{itemize}
\item \textbf{简单枚举归纳法:从特殊到普遍的基本方法}:
  \begin{itemize}
  \item \textbf{理论基础}:体现人类认识从感性到理性、从个别到一般的基本规律
  \item \textbf{逻辑结构}:前提的有限性、结论的普遍性、推理的跳跃、或然性质
  \item 与类比论证的区别:结论的普遍性程度不同
  \item \textbf{评价标准}:数量标准、多样性、精确性、时间跨度、反例缺失
  \item 历史应用:卡那尔文伯爵的政治论证案例分析
  \end{itemize}
\item \textbf{简单枚举归纳法的根本局限性}:
  \begin{itemize}
  \item \textbf{反例问题}:任何反例都能推翻所断言的因果律
  \item 选择性偏见:只关注确证事例,忽视否定事例
  \item 因果关系与历史巧合难以区分
  \item \textbf{适用范围限制}:适合建立因果律,不适合检验因果律
  \item 向更高级归纳方法的过渡必要性
  \end{itemize}
\item \textbf{认识论意义与历史地位}:
  \begin{itemize}
  \item 代表人类认识发展的重要阶段,是科学方法的起点
  \item 在日常认识中仍具有重要价值
  \item 促进了更精密归纳方法的发展
  \item 为认识论和科学哲学提供重要启示
  \end{itemize}
\end{itemize}
}}
\end{center}
\section{密尔方法}

\begin{quotation}
本节介绍归纳推理中最重要的方法之一——密尔方法。我们将分析约翰·斯图亚特·密尔提出的五种归纳法则:求同法、求异法、求同求异并用法、剩余法和共变法。通过理解这些归纳方法的原理和应用,我们将能够更有效地分析因果关系,并掌握科学研究中检验假说的基本工具。
\end{quotation}

人们早已知道简单枚举法的局限。早在1605年,弗兰西斯•培根就提出了其他类型的归纳程序。他在其伟大的著作《学习的进步》中探寻改革科学研究的方法。但是,更为强大的归纳方法,其精确表述和系统化,是由另一个英国哲学家约翰•斯图亚特•密尔在其著作《逻辑系统》 (1843)(1)中所完成,并被称为"归纳推理的密尔方法"。密尔总结为五条 "教规",他称它们为:

1.求同法(The Method of Agreement)\\
2.求异法(The Method of Difference)\\
3.求同求异并用法(The Joint Method of Agreement and Difference)

4.剩余法(The Method of Residues)\\
5.共变法(The Method of Concomitant Variation)

我们将依次考察它们。我们首先分析密尔对每个方法(一个例外)的经典陈述,接着对它们进行简单的说明和分析。尽管密尔对这些方法的解\\
(1)该书有严复中文节译本《穆勒名学》,在中国学界有历史性影响,故过去通常将 Mill 译为搖勒,近年多据正确读音译为密尔或弥尔。

释现在来说十分古老,但是密尔对这些在寻找因果律过程中时时和处处使用的最基本的工具的分析是精辟的。

\subsection{求同法}
约翰•斯图亚特•密尔写道:

如果被研究的现象的两个或更多的事例只有一个共同的事态,那么,这个事态——所有事例在该事态上相契合——是给定现象的原因(或结果)。

该方法比简单枚举法优越。它不仅试图发现原因与结果重复出现的连接,而且试图确定这个唯一的事态——不变地与我们感兴趣的结果或现象关联的一个事态。这是科学探究的一个重要的也是非常普遍的工具。例如,在寻找某个致命的流行病过程中,或者在查找某些地质现象的原因的过程中,流行病专家或地质学家将选出特定的那些事态,答案就在其中;他们询问,明显不同的事态集合(答案就在其中)在什么方面相一致?

想象一下在某个公寓楼的居民当中发生消化不良,我们得了解其原因。首先要研究的自然是所有得病的人吃了什么食物?一些病人吃的而不是所有病人吃的食物不可能是得病的原因;我们希望知道什么事态是每个得病场合所共同的。当然,共同的东西可能不是一种食物;可能是受感染的器具,或者接近某种有害的污水,或其他的情况。但是,仅当我们找到了某种对所有疾病的事例都是共同的事态,我们才找对了正确解决问题的途径。

求同法可以示意如下。其中大写字母表示事态,小写字母代表现象:

$$
\begin{aligned}
& A 、 B 、 C 、 D \text { 与 } w 、 x 、 y 、 z \text { 一起发生 } \\
& A 、 E 、 F 、 G \text { 与 } w 、 t 、 u 、 v \text { 一起发生 }
\end{aligned}
$$

$$
\text { 因而 } A \text { 是 } w \text { 的原因 (或结果) }
$$

该方法在确定一种现象或者事态的一个范围方面特别有用,对之的进一步研究将产生成效。它是富有建设性的,甚至在不能有结论的地方也是如此。例如,在分子遗传学中,一个遗传疾病的可能原因其范围往往通过使用求同法而大大变小。寻找范围被锁定在由特定疾病频繁发生的那些人

或家族的独特的基因构成。阿尔茨海默病(Alzheimer,导致精神过程进一步和不可逆转的下降)被认为是遗传的。在所有得病的人的基因构成中存在某个共同的事态吗?华盛顿大学的一个研究小组首先筛选了上百个得病家族,然后,对一个阿尔茨海默病高发病率的范围相对小的家族进行了艰苦的调查,该研究负责人写道:

\begin{displayquote}
我们选取该疾病遗传明显的那些家族。我们做这样的假设,存在一个有缺陷的基因,我们的任务是找到它。我们的初始工作是在一个包含所有人类染色体的大草堆中寻找一根针。我们在第 14 号染色体上发现了一个小小的地方,在那里存在一个引起阿尔茨海默病的有缺陷的基因。\cite{tanzi1996}
\end{displayquote}

在几年前求同法的一个类似使用,产生了一项给人类带来巨大利益的发现。人们发现,在某些城市里牙齿腐烂的速度相当慢,而当时不知原因为何。研究发现,那些城市中存在一个共同的事态:在那些城市的供水中氟的含量不同寻常的高。人们得出,使用氟能够减少牙齿腐烂的发生。该结论随后得到证实。结果是人们在全球范围内的城市供水中加氟。简言之,我们找到一个对给定现象的所有事例来说都是共同的事态,此时我们可以自信地认为,我们已经发现了它的原因。

然而,求同法有严重的局限。让我们主要看一下确证事例。单单该方法往往不足以确定待寻找的原因。我们难以安排可用数据,以确定所有事例所共同的一个事态。当研究发现所有事例中共同的事态不止一个时,只使用该技术不能评判这些不同的可能性。

尽管事态和现象之间的求同经常不是结论性的,但缺乏相同点可以帮助我们确定什么不是待研究现象的原因。求同法本质上是排除法,它说明了这样的事情,在我们感兴趣的现象出现的某些场合而不是所有场合下出现的事态,不可能是该现象的原因。因而,人们否定某个声称的因果关系,可能是因为他们注意到缺乏共同点,从而推论得出所声称的原因既不是该现象的充分条件又不是它的必要条件。

例如,某些人认为,在公立学校学生的进步表现(由教育评估考试即 SAT 的分数确定)与州政府在学校上的投人之间存在一个因果关系;投人的钱越多,学习越好。该观点在一定程度上被人们所驳倒。人们指出:

在1992-1993年间教师薪水最高的五个州中,没有一个进人 SAT 最高分的 15 个州的行列之中;单位学生花费最高的 10 个州中,仅有一个州(威斯康星州)进人 10 个 SAT 最高分的州中;并且,单位学生花费最高的新泽西州在分数排行榜中位于第 34 位。——所有证据都表明高投人不是学生成绩的充分条件。但是单位学生花费最低的 10 个州中,有 4 个州(北达科他州、南达科他州、田纳西州、犹他州)其拥有的 SAT 分数位于 SAT 分数顶尖的 10 个州之列;而北达科他州花费排名为第 44 位,而 SAT 分数排名为第 2 位;南达科他州的教师薪水为倒数第一,而其 SAT分数为第 3 位。所有这些证据说明,高投人不是学生学习取得好成绩的必要条件。\cite{forbes1996} 具有讽刺意味的是,旦尼尔•帕特瑞克•莫尼汗议员通过观察得到,影响美国公立学校质量的决定性因素不是金钱,而是与加拿大的距离!该论证远不是结论性的——但是缺乏相同性、缺乏一致性确实使人们质疑所提出的因果关系。

我们了解了求同法能够告诉我们的东西之后,在寻找原因过程中我们需要其他较精致的归纳法。

\subsection{求异法}
约翰•斯图亚特•密尔写道:

象不发生,两个事例中的事态除了这一个事态不同外(该事态仅在现象发生的过程中),其他均相同,该事态(它使两个事例产生区别)便是该现象的结果或原因,或者为原因中的一个不可缺少的部分。

该方法不关注在产生结果的事例中什么是共同的,而是关注在产生结果的事例和没有产生结果的事例之间存在什么差异。当我们研究胃不适问题时,如果我们已经知道得病的所有人吃了甜点罐装梨子,而没有吃那些梨子的人没有得病,我们能相当自信地认为,我们已经找到了该病的原因。

求异法和求同法之间的差别,在最近的一份关于荷尔蒙睾丸激素在雄性好斗行为中的作用的报告中表现突出。

许多物种的彝丸在一年的大多数时间里是封存不用的,只在一个特定交配的季节期间里,精确地说是在雄性与雄性之间打斗增加的那段时间里,它们才启动并产生晖九激素。尽管它们表现明显,这些数据仅仅是相关的:打斗发生的时候经常发现草丸激素。

可以用刀来证明,委婉说法是进行摘除实验。将物种中的军丸激素之源去除,好斗程度便下降。注入合成搴丸激素使奉丸激素回到正常水平之后,好斗便得以恢复。\cite{sapolsky1997}

这个摘除和恢复方法给出了荷尔蒙与好斗之间存在关联的证明,当然这个证明方法是受揵责的。

明显的,睾丸激素造成了关键的差别,但是作者报告谨慎,没有断定睾丸激素是雄性好斗的原因;而是更准确地说,睾丸激素肯定与好斗相关。用密尔的说法,就是说荷尔蒙是雄性好斗原因中的一个不可缺少的部分。如果我们能够确定单个因素,该因素在其他一切保持不变的情况下造成了差别,即:当我们去除该因素时,待考察的现象也不再发生,当我们将该因素引进来时,考察的现象发生了,此时,我们将相当肯定地找到我们考察的现象的原因或原因的一个不可缺少的部分。

求异法可用下面的形式来刻画,其中大写字母表示事态,小写字母表示现象:

$$
A 、 B 、 C 与 w 、 x 、 y 、 z 一起发生 \\
B 、 C 、 D 与 x 、 y 、 z 一起发生
$$
求异法在几乎所有类型的科学研究中起着中心作用。该方法在医疗研究人员对特定蛋白质的效果进行的研究中得到鲜活应用,这种蛋白质被怀疑与某种疾病的发展有关联。待考察的物质是否真的是原因(或者原因的一个不可缺少的一个部分),只有在我们建立了一个该物质被排除的实验环境的时候才能确定。当然,研究人员只能是在老鼠身上而不是在人身上进行该研究:从染上同样疾病的老鼠的身上去除产生可疑蛋白质的基因。处理过的老鼠进行近亲交配,以产生后代。这些后代被称为"基因剔除老鼠"(knockout mice),在当前医学研究界是很珍贵的。人们能够在一个老鼠身上研究与该疾病有关的过程。该老鼠与其他患有该种疾病的老鼠除了由基因剔除产生的差别外其余的完全一样,老鼠身上由基因剔除而缺少的物质被假定为原因。这样的研究在医疗上产生了重大进展。

求异法的一个著名的同时也是令人感动的例证,由对黄热病真实原因的确证实验的解释所提供。黄热病是人类长期遭受的重大瘟疫之一。这里描述的实验是由美国军队医生瓦尔特•雷德、詹姆斯•卡罗尔和杰西•W•拉杰尔在1900年11月进行的。该年初,卡罗尔医生在另外的实验里故意让自己被受感染的蚊子所叮咬,从而使自己染上黄热病。不久后另外一位医生拉杰尔死于黄热病,随后进行实验所在的营地以他的名字命名以纪念他。

所设计的实验其目的是表明,蚊子传播黄热病(通过排除受感染的所有其他途径)。建造了一个小房子,绝对杜绝蚊子从窗户、门及其他可能的出口出入。一个金属丝蚊帐将房间分成两个空间,其中一个空间里15个已经叮咬过黄热病病人的蚊子在飞。一个没有免疫的志愿者进入有蚊子的房间,他被 7 个蚁子所叮咬。四天后,他感染了黄热病。另外两个没有免疫力的人在无蚊子的房间里睡了 13 个晚上,而没有任何反应。

为了表明,该疾病的传播是通过蚊子而不是通过黄热病人的排泄物或与他们接触过的东西来进行的,另外一处房子建造了起

\begin{displayquote}
来。该房屋里是无蚊子的。将黄热病人的衣物、床上用品和吃饭器具,以及被黄热病人的血液、排泄物污染的其他器具,放置于该房屋,然后,让 3 个没有免疫力的人住在该屋子里。他们所用的床单是从病人的床上取下来的,那些病人因黄热病而死去。对床单上的污染的东西没有进行清洗,也没有进行其他的处理。以不同的志愿者将实验重复了两次。在整个阶段,居住在房子里的人被严格隔离,以免遭蚊子叮咬。这些实验中的人没有一个感染上黄热病。在随后的实验中证明了他们本身不具有免疫力,因为他们中的四个或者引蚊子叮咬或者因注射了黄热病人的血液,而感染了黄热病。\cite{garrison1929}
\end{displayquote}

在上述第一段落所描述的实验中,在两个精心密闭的空间中的受试人之间制造了一个重要的差异:一个房间里有叮咬过黄热病人的蚊子,另外的房间里则没有这样的蚊子。上述第二段落中所描述的实验精心使用了求异法:两组志愿者都密切接触黄热病人,唯一重要的差别是,其中一些志愿者后来被感染的蚊子所叮咬,或者注射了感染了的血液——缺乏这种事态,便没有感染发生。

在寻找原因过程中,求异法是普遍可用的同时也是强有力的。

\subsection{求同求异并用法}
尽管密尔认为这是一个不同的和独立的方法,但该方法最好理解成求同法和求异法在同一个研究中的联合运用。该法可以图示如下(也用大写字母表示事态、小写字母表示现象):

$$
\begin{array}{ll}
A, B, C-x, y, z ; & A, B, C-x, y, z \\
A, D, E-x, t, w ; & B, C-y, z \\
\hline
\end{array}
$$

因而,$A$ 是 $x$ 的结果或原因,或原因中不可缺少的一部分

由于两个方法(左边刻画的是求同法、右边刻画的是求异法)中的每一个方法给结论以某个概率的支持,它们的联合运用给该结论提供了较高的概率。在许多科学研究中,这种联合运用成为威力强大的归纳推理模式。

最近的一个著名医学成就显示了这种并用法的威力。甲型肝炎是肝脏感染,它折磨着成千上万的美国人;它在儿童中广泛传播,主要通过受污染的食物和水进行传播。它有时是致命的。如何预防它呢?当然,理想的方法是注射有效疫苗。但是一个很大的困难是,给何人注射甲肝疫苗?难以预测何处将爆发感染。因而,通常来说,不可能通过选择实验对象以产生可靠的结果。这个困难最终被克服,方法如下。

一种被认为有效的疫苗,在纽约俄兰基县克亚斯-乔尔镇的哈西德教派的犹太人社区中进行测试。该社区不同寻常,每年都流行甲肝。在克亚斯•乔尔镇几乎无人能够逃过甲肝的感染,该社区中近 $70 \%$ 的人在 19 岁

前就感染上了。克亚斯•乔尔医学研究所的阿兰•威尔兹伯格和他的同事,在该社区中招募了年龄 2 至 16 岁的 1037 名儿童,这些儿童没有受到甲肝感染——他们血液中没有该病毒的抗体。一半儿童(519.人)注射了一种新的疫苗,这些注射了疫苗的儿童中没有发现一例甲肝。没有注射疫苗的 518 个儿童中 25 个儿童不久被甲肝病毒感染。于是人们找到了甲肝疫苗。\cite{werzberger1992}

波士顿、华盛顿的肝脏专家对该项研究表示祝贺,称赞该研究是"一个重大突破"、"医学上重要的进展"。该研究依赖于什么推论方式?求同法和求异法都用到了。在医学研究中人们普遍这样做。在该社区能够对甲肝病毒免疫的年轻人中,只有一个条件是共同的:所有免疫者都接受了新的疫苗。由此,我们肯定地认为,该疫苗确实是导致免疫的原因。求异法对结论提供了很大的支持:免疫者的事态和不免疫者的事态在每个方面均类似,只在一个方面不同,即免疫居民被注射了疫苗。

人们经常进行所谓"双管齐下"(double-arm)实验,以检验新药或新方法:一组接受新的治疗,而另外一组没有;第二阶段,对原来没有接受治疗的人进行治疗,对原来接受治疗的人不施行治疗。这样研究的基础是求同法和求异法的联合运用,该方法应用广泛并且是有威力的。

\subsection{剩余法}
约翰•斯图亚特•密尔写道:

从一个现象中减去这样一个部分,在以前的归纳中该部分被认为是某个先行事件的结果,那么该现象剩余的部分为剩余的先行事件的结果。

前面的三个方法似乎假定了,我们能够整个地淘汰或产生某个现象的原因(或结果),有时我们确实能够这样。然而在某些情况下,我们只能通过观察一组事态中的变化——我们已经部分地知道该变化的原因——而推论得某个现象的因果性作用。

该方法关注剩余物。用于称货车上货物的重量的特殊装置可以很好地说明该方法。已知空车的重量。为了测定货物的重量,称出货与车一起的重量,然后我们就知道了货物的重量:整个重量减去车的重量。用密尔的术语来说,已知的"先行事件"是已经记录的空车重量——它必须从总数中减去;总数和已知的先行事件之间的差值,其原因明显地应归因于剩余的"先行事件",即货物本身。

剩余法可以表示如下:

$$
A, B, C-x, y, z
$$

已知 $B$ 是 $y$ 的原因 $C$ 是 $z$ 的原因

因而,$A$ 是 $x$ 的原因

天文学史上的一个伟大章节,即海王星的发现,给我们提供了剩余法威力的一个极好案例:

1821 年,巴黎的波瓦尔德发表了行星包括天王星的运动数据表。在准备天王星数据的时候,他遇到了很大的困难:根据 1800 年以后得到的位置数据而计算出来的轨道,与根据该行星刚刚被发现之后所观察到的数据所计算出来的轨道不协调。他对以前的观察数据完全置之不理,他的图表建立在新近观察的数据之上。然而,在后来的几年里,根据该表而计算出来的位置与该行星观察的数据存在不一致;到 1844 年差值总计达 2 分钟弧度。由于所有其他已知行星的运动位置与计算出来的位置一致,天王星中出现的差值引发了大讨论。

1845 年,勒维烈——那时还是一个年轻人——着手解决该问题。他检查了波瓦尔德的计算,发现计算是正确的。他感到,该问题的唯一满意的解释是,在天王星周围的某个地方存在一个干犹它运动的行星。到1846年的中期,他完成了他的计算, 9月他写信给柏林的迦勒(Galle),请求他在天空的一特定位置寻找一个新的行星。因为在德国已经绘制出了包含新的恒星的图表,而勒维烈当时还没有获得这些图表。在9月23日,迦勒开始寻找,在不到一小时的时间里他找到了一个物体,而这个物体是新图表中所没有的。到第二晚,该物体发生略微的移动,这个新的行星——后来被命名为海王星——在预测的位置的 1 度内被发现。该发现被认为是数理天文学中一个巨大的成就。\cite{berry1961}

这里,待研究的现象是天王星的运动。当时人们能够对该现象一一天王星绕太阳运行的轨道一一的大部分有很好的理解。天王星的观察数据近似于计算的轨道,但是一个难解之谜是,它们之间的差值。这个差值已经计算出来,需要我们进一步解释。一个附加的"先行事件"(一个存在的能够对这种差别进行说明的附加因素),被假设为另外一个(未发现的)星球,它的引力与已知的天王星轨道有关的假说一起,对这个差值进行说明。一472旦做出这样的假设,那个新的行星很快就得以发现。

使用;而其他方法要求考察至少两个事例。并且与其他方法不同的是,剩余法依赖于预先建立的因果律,而其他方法(如密尔描述的那样)则不是。尽管如此,剩余法是归纳的,而非演绎的。因为它产生的结论仅仅是或然的,而不能从前提中有效演绎出来。一个或两个附加的前提会使剩余法的推理转变成一个有效的演绎论证——但是也能够将之说成是另外的归纳方法。

\subsection{共变法}
已经讨论的四个方法本质上都是排除法的。通过剔除出给定现象的某个或某些可能原因,这些方法对其他的某个假定的因果解释提供支持。求同法排除掉那些不可能为原因的事态——在该事态缺乏的情况下该现象仍然能够发生;求异法通过剔除关键的一个先行因素而排除某个或某些可能原因;求同求异并用法也是排除法,它同时使用上面的两种方法;而剩余法努力排除那些不可能为原因的事态——这些事态的结果已经通过归纳预先建立起来。

但是存在这些方法都不可用的许多情形,因为存在不可能排除的事态。这经常发生在经济学、物理学、医学,以及在一个因素的增或减导致相伴随的另外一个因素的增或减的任何地方。此时,完全排除一个因素不可行。

约翰•斯图亚特•密尔写道:

一个现象随着另外一个现象以某种方式变化而发生变化,此时另外一个现象或者是该现象的一个原因,或者是一个结果,或者它通过某个作为原因的事实与之相连接。

例如,共变法对于研究某种食物的因果作用是重要的。无论我们吃什么食物,我们都不能排除疾病。我们几乎不能从大量人口的食物中排除掉某种食物,但是我们能够注意到,在特定人群中增加或减少某种食物量对某种疾病发生频率的影响。该种方法的一个最近的研究是,考察心脏病发生的频率,并与吃鱼的人心脏病发病的频率相对比。归纳出来的结论是惊人的:一周吃一次鱼肉,患心脏病的危险降低了 50 个百分点;一个月吃

两次鱼肉,患心脏病的危险降低了 30 个百分点。在某个范围内,在心脏患病和吃鱼之间似乎存在显著的共同变化。\cite{stampfer1994}

用加或减的符号表示一个变化的现象出现在一个给定情形中较高或较低的程度,共变法能够表示如下:

$$
\begin{aligned}
& A B C-x y z \\
& A+B C-x+y z \\
\hline & \text { 因而 } A \text { 与 } x \text { 因果地连接在一起 }
\end{aligned}
$$

该方法有广泛的应用。农民通过对不同的土地施不同数量的肥料,观察到肥料用量与产量之间的变化关系,而得出所施的肥料与庄稼收成之间的因果连接。商人在不同的时间段播放不同的广告,以观察那些时间段生意的好坏,从而确定不同种类的广告的功效。

当一个现象的增加对应于另外一个现象的增加时,我们说这些现象之间是直接相关的。但是该方法可以以任何方式来使用。当现象间是反方向变化的时候——一个现象的增加导致另外一个现象的减少,我们同样可以推论出一个因果关系。经济学家经常说,假定其他事物基本保持不变,在非计划的市场中某种货物(如原油)供应量的增加,将导致其价格发生相应的降低。该关系确实显示出真正的共变:当国际局势紧张、原油供应面临短缺的威胁的时候,我们注意到石油价格就无例外地上升。

当然,一些共同变化完全是偶然的。我们必须谨慎,不能从完全偶然的事件关系中推论出一个因果连接。但是某些变化看上去是偶然的(否则是令人费解的),但可能具有一个隐蔽的因果解释。人们发现,在英国乡村筑巢的鹳的数量与在每个乡村出生的婴儿之间存在高度相关;鹳越多,婴儿越多。这肯定不可能 $\cdots \cdots$ 是的,这不可能。具有高出生率的乡村具有更多新婚夫妇,因而具有更多的新建房屋。巧的是,鹳喜欢在以前没有被其他鹳用过的烟囱旁边筑巢。\cite{matthews1999} 追寻共同变化的现象的因果链条,我们可以找到共同的环节,这就是密尔所要表达的意思——他说这些现象可能是 "通过某个作为原因的事实……而连接起来"。

因为共变法允许我们举出例证,说明事态和现象之间出现的程度之间的变化关系,它大大加强我们的归纳技术。它是归纳推理的定量方法,而前面讨论的那些方法本质上是定性的。使用共变法预设了,存在对现象变

化的程度进行测量或估计的方法,哪怕仅仅是大致的。 

\begin{center}
\fbox{\parbox{0.95\textwidth}{
\textbf{本节要点}
\begin{itemize}
\item \textbf{密尔归纳方法概述}:
  \begin{itemize}
  \item 约翰·斯图亚特·密尔系统化提出五种归纳推理方法
  \item 这些方法用于寻找因果关系,是科学研究的基本工具
  \item 克服了简单枚举归纳法的局限性
  \end{itemize}
\item \textbf{五种密尔归纳方法}:
  \begin{itemize}
  \item 求同法:寻找现象所有事例中共同的事态作为可能原因
  \item 求异法:寻找导致现象存在与否的关键事态差异
  \item 求同求异并用法:结合两种方法提高因果推理的可靠性
  \item 剩余法:通过排除已知原因的影响发现新的因果关系
  \item 共变法:研究现象与事态之间的量化变化关系
  \end{itemize}
\item \textbf{方法应用特点}:
  \begin{itemize}
  \item 求同法、求异法、求同求异并用法和剩余法是定性方法
  \item 共变法是定量方法,研究因素变化的程度与结果变化的关系
  \item 多数方法基于排除法原理,通过淘汰不可能的原因确定真正原因
  \end{itemize}
\item \textbf{方法在科学研究中的应用}:
  \begin{itemize}
  \item 医学研究中的实验设计:如疫苗测试、基因剔除实验
  \item 天文学发现:如海王星的发现应用了剩余法
  \item "双管齐下"实验:在新药和新疗法研究中的应用
  \end{itemize}
\end{itemize}
}}
\end{center} 
\section{对密尔方法的批评:局限性与价值的哲学反思}

\begin{logicbox}[title=引言]
本节深入评估\logicterm{密尔归纳方法}的局限与价值,进行系统的哲学反思。我们将从认识论角度分析这些方法在实际应用中面临的根本困难,包括识别相关事态的理论问题、证明\logicterm{因果关系}的逻辑挑战,以及归纳推理的本质局限。同时,我们也将重新认识\logicterm{密尔方法}作为科学研究中检验假说工具的真正价值,理解它们在现代科学实验设计中的核心地位和方法论意义。这种批判性分析将帮助我们更准确地理解密尔方法在科学认识中的适当位置。
\end{logicbox}

\subsection{密尔方法的理论基础问题:相关性识别的困境}

\begin{theorembox}[title=密尔方法的根本局限性]
密尔本人相信,上面分析的技术可以用做发现因果关系的工具,并且能够用做证明因果连接的准则。在这两点上他都错了。这些方法确实意义重大,但是它们在科学中的地位并不像他认为的那样至高无上。

\textbf{理想化假设的问题}:
在密尔对这些方法的阐述中,他涉及"只有一个事态相同"的场合和"除了一个事态外其余的每个事态都相同"的场合。不能从字面上理解这些表述;任何两个物体无论它们多么不同,它们均具有许多相同的方面;没有两件事物只在一个方面不同——人们离北方越远,越靠近太阳;等等。我们甚至不能检查所有可能的事态,以确定是否它们只在一个方面存在差别。

\textbf{相关性识别的循环问题}:
简言之,密尔陈述这些方法时用到了所有相关事态的集合,这些事态与待研究的因果连接有关。但是哪些是相关事态?只用密尔方法我们不能知道哪些因素是相关的。我们必须求助于这些方法所应用的背景,此时我们已经分析了因果因素(哪些是有关的、哪些是无关的)。

\textbf{认识论的循环性}:
这里存在一个深层的认识论问题:为了应用密尔方法,我们需要预先知道哪些因素是相关的;但确定因素的相关性正是我们希望通过密尔方法来解决的问题。这种循环性表明密尔方法不能作为独立的发现工具。
\end{theorembox}

\begin{examplebox}[title="科学的酗酒者":相关性识别失败的经典案例]
关于"科学的酗酒者"的讽刺表明了这个问题:什么东西使酗酒者多次喝醉?他仔细观察,第一晚他喝的是苏格兰酒和苏打,第二晚喝的是波旁酒和苏打,接着是白兰地和苏打,朗姆酒和苏打,杜松子酒和苏打。他发誓再不碰苏打!

\textbf{案例分析}:
\begin{itemize}
\item \textbf{方法应用}:酗酒者正确应用了求同法,寻找所有醉酒事例的共同因素
\item \textbf{逻辑错误}:将苏打水而非酒精确定为共同因素
\item \textbf{根本问题}:缺乏关于酒精作用的背景知识
\item \textbf{理论意义}:说明密尔方法依赖于预先的理论框架
\end{itemize}

\textbf{深层启示}:
这个案例揭示了密尔方法的一个根本缺陷:它们无法自动识别真正相关的因素。在现实世界中,事态并没有贴有"有关的"或"无关的"标签。成功应用密尔方法需要研究者具备相关的理论知识和洞察力。
\end{examplebox}

\begin{examplebox}[title=黄热病研究:理论洞察与方法应用的结合]
前面讨论求异法时举了寻找黄热病原因的英雄行为,他们的研究证实了黄热病是由于受感染的蚊子叮咬而传染的,我们现在知道了这点,正如我们知道使人醉的是酒精而非苏打。

\textbf{成功的关键因素}:
\begin{itemize}
\item \textbf{理论洞察}:研究者具备了关于疾病传播的理论框架
\item \textbf{科学勇气}:愿意进行危险的人体实验
\item \textbf{系统设计}:精心设计的对照实验
\item \textbf{因果分析}:对蚊子叮咬的检验之前需要因果分析
\end{itemize}

\textbf{方法论启示}:
但是黄热病实验需要洞察力也需要勇气,在现实世界中的事态并没有贴有"有关的"或"无关的"标签。对蚊子叮咬的检验之前需要因果分析,以便接下来能够使用密尔方法。当我们手边有了这样的分析之后,这些方法才是十分有帮助的。

\textbf{结论}:
但是密尔方法作为科学发现的工具并不足够。它们需要与理论洞察、背景知识和科学直觉相结合才能发挥作用。
\end{examplebox}

\subsection{作为证明方法的根本缺陷:归纳推理的本质局限}

\begin{theorembox}[title=密尔方法作为证明方法的根本缺陷]
同样,密尔方法不能构成证明的规则。

\textbf{1. 预设假说的依赖性}:
因为我们总是根据关于因果事实的预先假说(刚才已经说明)而使用这些方法,并且由于我们不能考虑所有的事态,我们的注意力将限定在那些认为可能的原因上。

\textbf{2. 判断错误的可能性}:
但是判断可能是错误的,比如,医学家起初没有考虑到脏手在传播着疾病,或者当科学家因某种原因没能将在他们面前的事态分解成恰当的单元的时候。由于应用这些方法所预设的这些分析本身,可能是不适当的或不正确的,基于这些分析上的推理同样会是错误的。

\textbf{3. 观察的欺骗性}:
此外,所有的密尔方法依赖观察到的相关性,然而即使观察是十分精确的,这样的观察也可能是欺骗人的。我们寻求因果规律——普遍的关系,而我们迄今拥有的机会所观察到的东西可能不会告诉我们整个事情。

\textbf{4. 归纳推理的本质局限}:
我们的观察数量越大,我们记载的关联为真正的规律的可能性就越大——但是无论数量多大,我们不能在没有观察的事例中确定地得到一个因果连接。

\textbf{5. 归纳与演绎的根本鸿沟}:
理解这里的意思的关键是:在归纳和演绎之间存在一个巨大的鸿沟。一个有效的演绎推理构成一个证明或论证;但是任何一个归纳论证至多是高度可靠的,绝不能成为证明的(demonstrative)。

\textbf{结论}:
因而,密尔声称的他的方法是"证明的方法"(method of proof)的观点,连同他的它们是"发现方法的全部"的观点一起,都必须被拒绝。
\end{theorembox}

\begin{examplebox}[title=医学史中的认识局限:塞麦尔维斯的悲剧]
19世纪匈牙利医生塞麦尔维斯发现,医生洗手可以大大降低产褥热的发病率。他通过统计数据证明了这一点,但当时的医学界拒绝接受这个发现,因为它与当时占主导地位的疾病理论不符。

\textbf{案例启示}:
\begin{itemize}
\item \textbf{观察的准确性}:塞麦尔维斯的观察和统计是准确的
\item \textbf{理论的局限}:当时缺乏细菌理论的支持
\item \textbf{接受的困难}:即使有了正确的观察,理论框架的缺失使得发现难以被接受
\item \textbf{方法论意义}:说明密尔方法的有效性依赖于更广泛的理论背景
\end{itemize}

这个案例说明,即使密尔方法得出了正确的结论,如果缺乏适当的理论框架,这些结论也可能被拒绝或误解。
\end{examplebox}

\subsection{密尔方法的真正价值:假说检验工具的重新定位}

\begin{theorembox}[title=密尔方法的真正威力:假说检验的逻辑工具]
在本章中讨论的这些方法,尽管有局限,但是它们在科学方法中处于中心地位并且确实十分有效。

\textbf{与假说结合的必要性}:
由于绝对不可能将所有事态考虑进去,我们必须把密尔方法与关于被考察的事态的一个或更多的因果假说一起来使用,正如我们已经看到的那样。我们通常相当不自信,因而提出不同的假说,在这些假说下不同的因素暂时地作为被研究现象的原因。

\textbf{演绎推理的有效性}:
作为淘汰方法的密尔法能够使我们演绎地得到:如果对先行事态的某个特定分析是正确的,那么这些因素中的一个因素不能是(或必定是)被研究的现象的原因(或部分原因)。这个演绎是有效的——但是我们再一次强调,论证的稳固性建立在先行事态的分析的正确性之上。

\textbf{可靠性的条件}:
仅当形成的假说确实正确地识别出因果关联的事态的时候,这些方法才能产生可靠的结果;并且仅当假说被加在论证中作为一个前提的时候,结果才能通过这些方法演绎出来。

\textbf{方法的本质重新定义}:
现在我们能够明白这些方法提供给我们的力量的本质。它们不是发现的通路,也不是证明规则。它们是检验假说的工具。

\textbf{现代科学的基础}:
这些方法描述了受控实验的普遍方法——在所有现代科学中普遍和不可缺少的工具。
\end{theorembox}

\begin{examplebox}[title=现代科学实验设计中的密尔方法]
\textbf{药物试验中的应用}:
在现代药物试验中,密尔方法的原理被系统地应用:
\begin{itemize}
\item \textbf{求异法}:对照组与实验组的设计
\item \textbf{求同法}:寻找所有有效案例的共同特征
\item \textbf{求同求异并用法}:双盲随机对照试验
\item \textbf{共变法}:剂量-效应关系的研究
\end{itemize}

\textbf{方法论价值}:
\begin{itemize}
\item \textbf{假说检验}:不是用来发现新假说,而是检验既有假说
\item \textbf{排除法}:通过排除不可能的因素来缩小可能性范围
\item \textbf{控制变量}:为现代实验设计提供逻辑基础
\item \textbf{因果推断}:在有限条件下进行可靠的因果推断
\end{itemize}

\textbf{现代意义}:
密尔方法为现代科学的实验方法论奠定了逻辑基础,虽然它们不能独立发现或证明因果关系,但作为假说检验工具具有不可替代的价值。
\end{examplebox}

\begin{center}
\fbox{\parbox{0.95\textwidth}{
\textbf{本节要点}
\begin{itemize}
\item \textbf{密尔方法的理论基础问题}:
  \begin{itemize}
  \item \textbf{理想化假设的困难}:不能从字面上理解"只有一个事态相同"或"除一个事态外都相同"
  \item \textbf{相关性识别的循环问题}:为了应用密尔方法需要预先知道哪些因素相关,但确定相关性正是方法要解决的问题
  \item \textbf{认识论的循环性}:存在深层的认识论循环,表明密尔方法不能作为独立的发现工具
  \item \textbf{"科学的酗酒者"案例}:说明方法无法自动识别真正相关的因素,需要理论知识和洞察力
  \item \textbf{黄热病研究启示}:成功应用需要理论洞察、科学勇气和系统设计的结合
  \end{itemize}
\item \textbf{作为证明方法的根本缺陷}:
  \begin{itemize}
  \item \textbf{预设假说的依赖性}:总是根据预先假说使用方法,注意力限定在认为可能的原因上
  \item \textbf{判断错误的可能性}:预设分析可能不适当或不正确,导致基于此的推理错误
  \item \textbf{观察的欺骗性}:即使精确观察也可能欺骗人,不能告诉我们完整的情况
  \item \textbf{归纳推理的本质局限}:无论观察数量多大,都不能确定地得到因果连接
  \item \textbf{归纳与演绎的根本鸿沟}:归纳论证至多高度可靠,绝不能成为证明性的
  \item \textbf{塞麦尔维斯案例}:说明即使正确观察,缺乏理论框架也难以被接受
  \end{itemize}
\item \textbf{密尔方法的真正价值重新定位}:
  \begin{itemize}
  \item \textbf{假说检验工具}:不是发现通路或证明规则,而是检验假说的有效工具
  \item \textbf{与假说结合的必要性}:必须与因果假说一起使用才能发挥作用
  \item \textbf{演绎推理的有效性}:在正确分析基础上能够进行有效的演绎推理
  \item \textbf{现代科学实验基础}:描述了受控实验的普遍方法,是现代科学不可缺少的工具
  \item \textbf{药物试验应用}:在现代药物试验中系统应用各种密尔方法原理
  \end{itemize}
\item \textbf{哲学反思的意义}:
  \begin{itemize}
  \item \textbf{方法论的准确定位}:帮助我们准确理解密尔方法在科学认识中的适当位置
  \item \textbf{科学方法的理解}:深化对科学方法本质和局限性的理解
  \item \textbf{批判性思维}:培养对方法论的批判性思维和理性反思能力
  \item \textbf{现代科学基础}:为理解现代科学实验设计提供重要的逻辑基础
  \end{itemize}
\end{itemize}
}}
\end{center}
% 参考文献将在主文档末尾统一显示

% 第十三章
\chapter{非经典逻辑}
\section{科学的价值:人类文明进步的双重动力}

\begin{logicbox}[title=引言]
本节深入探讨\logicterm{科学}在人类文明发展中的根本价值和多重意义。我们将从哲学、社会学、认识论等多个维度分析\logicterm{科学}的实践价值与认识价值,深入理解\logicterm{科学}如何通过技术革命改变人类生活方式,以及如何通过理论建构满足人类认识世界的根本需求。通过系统分析\logicterm{科学}不仅是事实的收集,更是对现象的理论解释和规律发现,我们将全面认识\logicterm{科学理论}、\logicterm{自然定律}和\logicterm{科学方法}在现代文明中的核心地位,以及科学精神对人类思维方式的深刻影响。
\end{logicbox}

\subsection{科学的实践价值:技术革命与社会变迁}

\begin{theorembox}[title=科学实践价值的历史演进]
现代科学几乎改变了我们生活的每个方面。它的实践价值在于,它使得更方便、更健康和更丰富的生活成为可能。

\textbf{科学技术革命的历史阶段}:
\begin{itemize}
\item \textbf{第一次科学革命(16-17世纪)}:以伽利略、牛顿为代表,建立了现代科学方法论基础
\item \textbf{第一次工业革命(18-19世纪)}:蒸汽机技术改变了生产方式和社会结构
\item \textbf{第二次工业革命(19-20世纪)}:电力和化学工业的发展
\item \textbf{第三次科学技术革命(20世纪中后期)}:信息技术、生物技术、新材料技术
\item \textbf{第四次工业革命(21世纪)}:人工智能、物联网、大数据等新兴技术
\end{itemize}

\textbf{科学应用的广泛领域}:
尽管人们对于它的一些成果颇感忧虑,然而,大多数人同意,科学进步及其在通信、运输、制造、种植、娱乐和公共卫生等等方面的技术应用,总的来说大大地有益于人类。

\textbf{科学价值的辩证性}:
科学的实践价值具有双重性——既带来巨大福祉,也可能产生负面影响。这要求我们以理性和负责任的态度对待科学技术的发展和应用。
\end{theorembox}

\begin{examplebox}[title=科学技术改变人类生活的具体实例]
\textbf{医学领域的革命性进步}:
\begin{itemize}
\item \textbf{疫苗技术}:天花的根除、小儿麻痹症的控制、COVID-19疫苗的快速研发
\item \textbf{抗生素发现}:青霉素的发现拯救了数百万生命
\item \textbf{现代外科技术}:器官移植、微创手术、机器人辅助手术
\item \textbf{基因治疗}:CRISPR技术为遗传疾病治疗带来希望
\end{itemize}

\textbf{信息技术的社会变革}:
\begin{itemize}
\item \textbf{互联网革命}:改变了信息传播、商业模式、社交方式
\item \textbf{移动通信}:智能手机使全球即时通信成为现实
\item \textbf{人工智能}:在医疗诊断、自动驾驶、语言翻译等领域的应用
\item \textbf{大数据分析}:为科学研究、商业决策、公共政策提供支持
\end{itemize}

\textbf{环境科学与可持续发展}:
\begin{itemize}
\item \textbf{清洁能源技术}:太阳能、风能、核能技术的发展
\item \textbf{环境监测}:卫星遥感、环境传感器网络
\item \textbf{生态保护}:生物多样性保护、生态系统修复技术
\item \textbf{气候科学}:全球气候变化的监测和预测
\end{itemize}
\end{examplebox}

\subsection{科学的认识价值:人类理性精神的最高体现}

\begin{theorembox}[title=科学认识价值的哲学基础]
科学在实现认识愿望方面也实现了内部价值。这种认识价值体现了人类作为理性存在的本质特征。

\textbf{古典哲学的认识论传统}:
很久以前,亚里士多德写道:"认识某个事情(不仅哲学方面的,而且人类的其余方面的)是最大的乐趣。"\cite{aristotle1950} 这一观点奠定了西方理性主义传统的基础,强调了知识本身的内在价值。

\textbf{现代科学家的认识追求}:
爱因斯坦则代表所有时代的科学家写道:

"什么因素推动我们发明一个又一个理论?我们为什么要发明理论?答案很简单:因为我们乐于'理解'(comprehending),即,通过逻辑过程将现象归约为已经知道或有(明显)证据的事物。"\cite{einstein1935}

\textbf{科学认识的独特性}:
\begin{itemize}
\item \textbf{系统性}:科学不满足于零散的知识,追求系统化的理解
\item \textbf{普遍性}:寻求适用于所有情况的普遍规律
\item \textbf{精确性}:通过数学化和量化实现精确描述
\item \textbf{可验证性}:理论必须能够接受经验检验
\item \textbf{预测性}:能够对未来现象进行准确预测
\end{itemize}
\end{theorembox}

\begin{examplebox}[title=科学认识价值的历史见证]
\textbf{天文学的认识革命}:
\begin{itemize}
\item \textbf{哥白尼革命}:日心说改变了人类对宇宙的认识
\item \textbf{开普勒定律}:揭示了行星运动的数学规律
\item \textbf{哈勃发现}:宇宙膨胀理论改变了宇宙观
\item \textbf{引力波探测}:验证了爱因斯坦广义相对论的预言
\end{itemize}

\textbf{生物学的认识突破}:
\begin{itemize}
\item \textbf{达尔文进化论}:揭示了生命演化的机制
\item \textbf{DNA双螺旋结构}:解开了遗传的分子基础
\item \textbf{基因组计划}:绘制了人类遗传信息的完整图谱
\item \textbf{CRISPR技术}:使精确基因编辑成为可能
\end{itemize}

\textbf{物理学的理论建构}:
\begin{itemize}
\item \textbf{牛顿力学}:建立了经典物理学的理论框架
\item \textbf{相对论}:革命性地改变了时空观念
\item \textbf{量子力学}:揭示了微观世界的奇异性质
\item \textbf{标准模型}:统一描述了基本粒子和相互作用
\end{itemize}
\end{examplebox}

\subsection{科学的本质:从事实到理论的认识飞跃}

\begin{theorembox}[title=科学认识的层次结构]
科学的目标就是发现普遍真理。这一目标体现了科学认识的层次性和系统性特征。

\textbf{事实与理论的辩证关系}:
当然单个事实是关键的;用事实建造科学大厦,如同用石头建造房屋。采集了石头不等于建成房屋,仅仅事实的收集更不能成为科学。科学家寻求理解现象,为此,他们努力揭示现象发生的方式以及它们之间的系统的关系。

\textbf{科学解释的必要性}:
仅仅知道事实是不够的,对它们进行说明是科学的任务。因而,这需要理论(如爱因斯坦所说),以及与理论相连的支配事实的自然定律和基础性原理。

\textbf{科学认识的三个层次}:
\begin{itemize}
\item \textbf{经验层次}:观察和实验获得的事实材料
\item \textbf{理论层次}:对事实的概念化和系统化解释
\item \textbf{哲学层次}:对科学理论的反思和世界观建构
\end{itemize}
\end{theorembox}

\begin{examplebox}[title=从事实到理论的科学发现过程]
\textbf{开普勒行星运动定律的发现}:
\begin{itemize}
\item \textbf{观察事实}:第谷·布拉赫精确观测行星位置数据
\item \textbf{数据分析}:开普勒分析火星轨道数据
\item \textbf{理论建构}:提出椭圆轨道假说,建立三大定律
\item \textbf{普遍化}:定律适用于所有行星运动
\end{itemize}

\textbf{门捷列夫元素周期律的建立}:
\begin{itemize}
\item \textbf{事实收集}:已知元素的原子量和化学性质
\item \textbf{规律发现}:按原子量排列发现周期性规律
\item \textbf{理论预测}:预言未知元素的存在和性质
\item \textbf{实验验证}:镓、钪、锗的发现验证了理论
\end{itemize}

\textbf{达尔文进化论的形成}:
\begin{itemize}
\item \textbf{博物学观察}:物种分布、化石记录、人工选择
\item \textbf{理论综合}:自然选择机制的提出
\item \textbf{解释力}:统一解释生物多样性和适应性
\item \textbf{现代发展}:与遗传学结合形成现代综合理论
\end{itemize}
\end{examplebox}

\subsection{科学精神与人类文明:价值观念的深层影响}

\begin{theorembox}[title=科学精神的文明意义]
科学不仅改变了人类的物质生活和认识能力,更深刻地影响了人类的思维方式和价值观念。

\textbf{科学精神的核心要素}:
\begin{itemize}
\item \textbf{理性主义}:相信理性思维和逻辑推理的力量
\item \textbf{经验主义}:重视观察、实验和经验证据
\item \textbf{批判精神}:质疑权威,追求真理,勇于修正错误
\item \textbf{开放态度}:接受新思想,欢迎不同观点的挑战
\item \textbf{合作精神}:科学共同体的协作和知识共享
\end{itemize}

\textbf{科学对现代文明的深层影响}:
\begin{itemize}
\item \textbf{民主制度}:科学精神促进了理性讨论和民主决策
\item \textbf{教育理念}:强调批判思维和创新能力的培养
\item \textbf{法律制度}:证据为本的司法理念
\item \textbf{社会治理}:基于数据和科学分析的政策制定
\item \textbf{国际合作}:全球性科学合作促进人类命运共同体建设
\end{itemize}

\textbf{科学伦理与责任}:
科学的巨大力量也带来了相应的责任。科学家和社会都需要思考科学技术的伦理问题,确保科学发展服务于人类福祉。
\end{theorembox}

\begin{center}
\fbox{\parbox{0.95\textwidth}{
\textbf{本节要点}
\begin{itemize}
\item \textbf{科学的实践价值:技术革命与社会变迁}:
  \begin{itemize}
  \item \textbf{历史演进}:从第一次科学革命到第四次工业革命的技术发展历程
  \item \textbf{生活改善}:在医学、信息技术、环境科学等领域的革命性进步
  \item \textbf{社会变革}:科学技术应用改变了生产方式、社会结构和生活方式
  \item \textbf{辩证性质}:科学价值具有双重性,既带来福祉也可能产生负面影响
  \item \textbf{具体实例}:疫苗技术、互联网革命、清洁能源、人工智能等重大突破
  \end{itemize}
\item \textbf{科学的认识价值:人类理性精神的最高体现}:
  \begin{itemize}
  \item \textbf{哲学基础}:从亚里士多德到爱因斯坦的认识论传统
  \item \textbf{独特性质}:系统性、普遍性、精确性、可验证性、预测性
  \item \textbf{历史见证}:天文学革命、生物学突破、物理学理论建构的认识价值
  \item \textbf{理性追求}:满足人类理解世界的根本需求和认识乐趣
  \item \textbf{文明意义}:体现了人类作为理性存在的本质特征
  \end{itemize}
\item \textbf{科学的本质:从事实到理论的认识飞跃}:
  \begin{itemize}
  \item \textbf{层次结构}:经验层次、理论层次、哲学层次的认识体系
  \item \textbf{辩证关系}:事实是基础,理论是升华,两者相互依存
  \item \textbf{解释功能}:科学不满足于事实收集,追求系统化解释
  \item \textbf{发现过程}:从开普勒定律、元素周期律到进化论的理论建构实例
  \item \textbf{普遍真理}:通过理论和定律揭示现象间的系统关系
  \end{itemize}
\item \textbf{科学精神与人类文明:价值观念的深层影响}:
  \begin{itemize}
  \item \textbf{核心要素}:理性主义、经验主义、批判精神、开放态度、合作精神
  \item \textbf{文明影响}:促进民主制度、教育理念、法律制度、社会治理的现代化
  \item \textbf{思维方式}:科学精神深刻影响人类的认识方法和价值观念
  \item \textbf{国际合作}:全球性科学合作促进人类命运共同体建设
  \item \textbf{伦理责任}:科学发展需要承担相应的社会责任和伦理考量
  \end{itemize}
\item \textbf{科学价值的现代意义}:
  \begin{itemize}
  \item 科学是人类文明进步的双重动力:既推动技术发展,又提升认识能力
  \item 科学精神成为现代文明的重要组成部分,影响社会制度和文化观念
  \item 科学发展需要在追求真理与承担责任之间保持平衡
  \item 科学教育应培养批判思维和创新能力,传承科学精神
  \end{itemize}
\end{itemize}
}}
\end{center}
\section{说明:科学的和非科学的——认识论的根本分野}

\begin{logicbox}[title=引言]
本节深入探讨\logicterm{科学说明}与\logicterm{非科学说明}的本质区别,这一区别构成了现代认识论的根本分野。我们将从逻辑学、认识论、科学哲学等多个角度分析什么构成一个\logicemph{有效的}说明,深入辨别\logicterm{科学态度}与\logicterm{非科学态度}之间的关键差异,并系统理解\logicterm{科学说明}的\logicterm{可检验性}、\logicterm{可证伪性}等核心特征。通过理解\logicterm{科学说明}必须是可证实的、探索性的、开放的,而非教条主义的、封闭的,我们将能够准确区分\logicemph{真正的}科学解释与仅凭权威、传统或习俗的\logicwarn{非科学说法},从而建立科学理性的认识基础。
\end{logicbox}

\subsection{说明的逻辑结构:从推理到解释的认识转换}

\begin{theorembox}[title=说明的逻辑本质]
当要对某个事情进行说明(explanation)时,我们需要什么?

\textbf{说明的定义}:
一个被寻求的解释(account)就是对世界的某个陈述集合,或某个叙说(story),从该解释中能够逻辑地推导出需要解释的事情。该解释能够对需要解释的有疑问的问题进行消解或者简约。

\textbf{说明与推理的逻辑关系}:
说明和推论可以被看成是同一个过程,只是方向相反:
\begin{itemize}
\item \textbf{演绎推理}:从前提到结论($P \rightarrow Q$)
\item \textbf{科学说明}:从结论到前提(已知$Q$,寻找$P$使得$P \rightarrow Q$)
\end{itemize}

\textbf{逻辑形式的双重功能}:
在本书第1章(1.6节)中我们阐述了,当我们要推得$Q$时,"由$P$得$Q$"如何表达一个论证;如果我们所进行的是从一个已经建立的$Q$到能够对之说明的前提的推理,它也可以表达一个说明。

\textbf{说明的认识论意义}:
说明不仅是逻辑操作,更是人类理解世界的基本认识活动。通过说明,我们将未知归约为已知,将复杂归约为简单,将特殊归约为普遍。
\end{theorembox}

\begin{examplebox}[title=说明逻辑结构的具体实例]
\textbf{物理学中的说明}:
\begin{itemize}
\item \textbf{待解释现象}:苹果从树上掉下来
\item \textbf{说明前提}:万有引力定律、地球质量、苹果质量
\item \textbf{逻辑推导}:从引力定律推导出苹果必然下落
\item \textbf{说明效果}:特殊现象被普遍定律所解释
\end{itemize}

\textbf{生物学中的说明}:
\begin{itemize}
\item \textbf{待解释现象}:长颈鹿的长脖子
\item \textbf{说明前提}:自然选择理论、环境压力、遗传变异
\item \textbf{逻辑推导}:从进化理论推导出长脖子的适应优势
\item \textbf{说明效果}:生物特征被进化机制所解释
\end{itemize}

\textbf{心理学中的说明}:
\begin{itemize}
\item \textbf{待解释现象}:学习效果的差异
\item \textbf{说明前提}:认知负荷理论、工作记忆容量、注意资源分配
\item \textbf{逻辑推导}:从认知理论推导出学习效果的差异
\item \textbf{说明效果}:心理现象被认知机制所解释
\end{itemize}
\end{examplebox}

\subsection{有效说明的标准:相关性、真实性与普遍性}

\begin{theorembox}[title=有效说明的基本要求]
自然,每一个好的说明必须满足几个基本标准。

\textbf{1. 相关性(Relevance)}:
如果我解释说,我上班迟到是因为在中部非洲发生持续的政治混乱,那么这会被认为什么都没有说明;它是不相关的——因为需要说明的我迟到的事实,不能从中被推论出来。

\textbf{2. 真实性(Truth)}:
当然每个真正的说明不仅是相关的而且是真实的。虚假的前提无法提供有效的说明。

\textbf{3. 普遍性(Universality)}:
无论我迟到的正确的说明是什么,之所以需要这个说明,是因为在我迟到的这个事件上存在疑问。然而,科学的说明除了相关和真实外,必须超越特定事件,而能够对给定种类的所有事件提供解释。

\textbf{科学说明的普遍性特征}:
科学说明的力量在于其普遍适用性。它不仅解释特定现象,更重要的是能够解释同类的所有现象。
\end{theorembox}

\begin{examplebox}[title=牛顿万有引力定律的说明力量]
牛顿力学的伟大在于万有引力定律。牛顿写道:

"宇宙中每个质点以一个力吸引另外一个质点。该力正比于质点质量的乘积,反比于它们间距离的平方。"

\textbf{万有引力定律的说明范围}:
\begin{itemize}
\item \textbf{地面现象}:物体下落、抛物运动
\item \textbf{天体运动}:行星轨道、月球运动、彗星轨道
\item \textbf{潮汐现象}:海洋潮汐的周期性变化
\item \textbf{双星系统}:恒星的相互绕转
\end{itemize}

\textbf{普遍性的认识价值}:
一个定律能够统一解释如此广泛的现象,这体现了科学理论的巨大解释力和预测力。这种普遍性使得科学说明超越了特定事件的描述,达到了对自然规律的深层理解。

\textbf{数学表达的精确性}:
$$F = G\frac{m_1 m_2}{r^2}$$
这个简洁的数学公式包含了对宇宙中所有引力现象的完整描述,体现了科学说明追求精确性和普遍性的特征。
\end{examplebox}

\subsection{科学说明与非科学说明的根本区别}

\begin{theorembox}[title=非科学说明的特征分析]
非科学的说明也可以是相关的和普遍的,但它们缺乏科学说明的核心特征。

\textbf{非科学说明的典型例子}:
\begin{itemize}
\item \textbf{机械故障的神秘解释}:用神秘的小鬼动了手脚,来解释引擎不能启动
\item \textbf{疾病的超自然解释}:疾病可以解释成邪恶的精灵侵入人体所引起的
\item \textbf{天体运动的拟人化解释}:在长达数个世纪的时间里,人们一直用在行星上生活并控制它们运动的"智慧生物"来解释行星的规则运动
\end{itemize}

\textbf{非科学说明的共同特征}:
\begin{itemize}
\item \textbf{不可检验性}:无法通过经验观察或实验来验证或证伪
\item \textbf{特设性}:为了解释特定现象而临时构造,缺乏独立的证据支持
\item \textbf{封闭性}:不接受批评和修正,拒绝与其他理论对话
\item \textbf{模糊性}:概念定义不清,缺乏精确的预测能力
\end{itemize}

\textbf{历史上的非科学说明}:
这些非科学说明在历史上曾经广泛存在,反映了人类在科学方法建立之前对世界的理解方式。它们虽然满足了人类解释世界的心理需求,但无法提供可靠的知识基础。
\end{theorembox}

但是我们对真正科学的说明感兴趣。科学的说明与非科学的说明在两个相互关联的方面相区别:

\begin{theorembox}[title=科学态度与非科学态度的根本区别]
\textbf{第一个区别:认识态度的差异}

\textbf{非科学态度的特征}:
接受非科学说明的人是教条的,解释被认为是绝对真的,是不能改进的。亚里士多德的观点在几个世纪里被非科学地接受成对事实的最终权威。尽管亚里士多德本人是谦虚的,但是一些中世纪的学者却以僵化的、非科学的态度对待他的观点。\cite{crombie1960}

\textbf{科学态度的特征}:
相反,真正科学的态度则与之十分不同。每个提出的说明都是探索性的或暂时的。科学说明被认为是假说——它们在现有证据下具有不同的可靠程度。

\textbf{态度差异的深层含义}:
\begin{itemize}
\item \textbf{开放性vs封闭性}:科学态度对新证据开放,非科学态度拒绝质疑
\item \textbf{可错性vs绝对性}:科学承认理论的可错性,非科学声称绝对真理
\item \textbf{进步性vs静态性}:科学追求不断改进,非科学维持现状
\item \textbf{批判性vs权威性}:科学鼓励批判思维,非科学依赖权威
\end{itemize}
\end{theorembox}

\begin{theorembox}[title=证据基础:科学与非科学的根本分野]
\textbf{第二个区别:证据基础的差异}

科学说明和非科学说明之间的第二个同时也是最根本的区别是,接受或拒绝某个观点所基于的基础。

\textbf{非科学说明的证据基础}:
一个非科学的说明被简单地认为是真的,因为"每个人知道"它如此。一个非科学信念之被坚持,不依赖于有利于它的证据之上。

\textbf{科学说明的证据基础}:
但是在科学中,一个假说仅仅在存在支持它的证据的条件下才值得接受,人们总是对它的真或假保持怀疑,寻找证据的过程永不停止。

\textbf{科学的经验性质}:
科学是经验的——真理的检验在于经验之中,因而科学说明的本质是,它是可检验的。

\textbf{证据标准的对比}:
\begin{itemize}
\item \textbf{科学标准}:经验证据、逻辑一致性、可重复性、可预测性
\item \textbf{非科学标准}:权威、传统、直觉、信仰、"常识"
\item \textbf{检验方式}:科学通过实验和观察检验,非科学拒绝检验
\item \textbf{修正机制}:科学有自我修正机制,非科学缺乏修正能力
\end{itemize}
\end{theorembox}

\subsection{可检验性:科学说明的核心特征}

\begin{theorembox}[title=直接检验与间接检验]
真理的检验可以是直接的也可以是间接的。

\textbf{直接检验}:
为了弄清外面是否下雨,我只要看一下外面。这种检验直接、简单、确定。

\textbf{间接检验的必要性}:
但是用做说明的假说是普遍性命题,它们不能是直接可检验的。科学理论通常涉及不可直接观察的实体或过程,因此需要间接检验。

\textbf{间接检验的逻辑结构}:
如果我对我上班迟到的解释是交通事故,我的老板如果对之怀疑,他能够借助于警察的事故报告而间接地检验我的解释。

一个间接的检验从待检验的命题(如我遭遇到一次交通事故)演绎出其他某个能够被直接检验的命题(如我提交了一个事故报告)。

\textbf{间接检验的逻辑后果}:
\begin{itemize}
\item 如果那个演绎出来的命题是错的,包含这个命题的说明必定是错的
\item 如果演绎出来的命题是真的,它提供了某个证据证明这个说明是真的、已经被间接证实
\item 但是这个证据不是结论性的
\end{itemize}
\end{theorembox}

\begin{theorembox}[title=间接检验的局限性与复杂性]
间接检验永远不会是确定的。它总是依赖于某些辅助的前提。

\textbf{辅助前提的依赖性}:
比如这样的前提:我对我的老板描述的该起事故与警察记载的一样。但是警察部门应当对我所涉案的事故的记录进行备案,但可能还没有备案;缺乏该记录并不证明我的说明是假的。

\textbf{证据的非结论性}:
并且,某个附加前提即使是真的,它并不给说明赋予确定性——尽管演绎出的结论(本例中事故报告的真实性)得到成功检验确实加固了它的前提。

\textbf{杜恒-奎因论题}:
这种复杂性反映了科学哲学中著名的杜恒-奎因论题:我们无法孤立地检验单个假说,而总是在检验一个假说网络。这意味着:
\begin{itemize}
\item 检验结果的解释具有多义性
\item 理论的修正可能涉及多个层面
\item 科学知识具有整体性特征
\item 确定性在科学中是相对的而非绝对的
\end{itemize}
\end{theorembox}

\begin{theorembox}[title=可证实性:科学说明的本质特征]
即使一个非科学的说明也有某个它喜爱的证据,即用它来解释的那个事实。

\textbf{非科学说明的证据局限}:
行星上居住着"智慧生物",他们使行星沿着我们观察到的轨道运动,这个非科学理论能够称这个事实——行星确实在它们的轨道上运动——为证据。

\textbf{科学与非科学说明的根本差别}:
但是,在该假说和关于行星运动的可靠的天文学说明之间存在巨大的差别:
\begin{itemize}
\item \textbf{非科学假说}:不能够从中演绎出其他的可直接检验的命题
\item \textbf{科学说明}:能够演绎出可直接检验的命题,而不是陈述待解释事实的命题
\end{itemize}

\textbf{经验可证实性的含义}:
这就是当我们说一个说明是经验可证实的时所要表达的意思。这样的可证实性是科学说明最本质的特征。\cite{popper1935}

\textbf{可证实性的深层意义}:
\begin{itemize}
\item \textbf{预测能力}:科学理论能够预测新的、未观察到的现象
\item \textbf{风险承担}:科学理论敢于做出可能被证伪的预测
\item \textbf{内容丰富}:可检验的预测越多,理论的经验内容越丰富
\item \textbf{进步机制}:通过检验和修正实现科学进步
\end{itemize}
\end{theorembox}

\begin{examplebox}[title=科学说明可证实性的历史实例]
\textbf{爱因斯坦相对论的预测与验证}:
\begin{itemize}
\item \textbf{理论预测}:光线在强引力场中会发生弯曲
\item \textbf{检验方法}:1919年日食观测
\item \textbf{验证结果}:观测结果与理论预测精确吻合
\item \textbf{科学意义}:确立了广义相对论的科学地位
\end{itemize}

\textbf{门捷列夫元素周期律的预测}:
\begin{itemize}
\item \textbf{理论预测}:预言了镓、钪、锗等未知元素的存在和性质
\item \textbf{检验方法}:化学实验和元素发现
\item \textbf{验证结果}:预言的元素相继被发现,性质与预测高度一致
\item \textbf{科学意义}:证明了周期律的科学价值
\end{itemize}

\textbf{达尔文进化论的预测}:
\begin{itemize}
\item \textbf{理论预测}:应该存在连接不同物种的过渡化石
\item \textbf{检验方法}:古生物学研究和化石发现
\item \textbf{验证结果}:大量过渡化石的发现支持了进化理论
\item \textbf{科学意义}:为进化论提供了强有力的证据支持
\end{itemize}
\end{examplebox}

\begin{center}
\fbox{\parbox{0.95\textwidth}{
\textbf{本节要点}
\begin{itemize}
\item \textbf{说明的逻辑结构:从推理到解释的认识转换}:
  \begin{itemize}
  \item \textbf{说明的本质}:从中能逻辑推导出待解释现象的陈述集合
  \item \textbf{逻辑关系}:说明与推论是相反方向的逻辑过程(演绎推理:$P \rightarrow Q$;科学说明:已知$Q$,寻找$P$)
  \item \textbf{认识意义}:通过说明将未知归约为已知,将复杂归约为简单,将特殊归约为普遍
  \item \textbf{跨学科应用}:物理学、生物学、心理学等各领域的说明实例
  \end{itemize}
\item \textbf{有效说明的标准:相关性、真实性与普遍性}:
  \begin{itemize}
  \item \textbf{相关性要求}:说明前提必须能够逻辑地推导出待解释现象
  \item \textbf{真实性要求}:说明的前提必须是真实的,虚假前提无法提供有效说明
  \item \textbf{普遍性特征}:科学说明超越特定事件,能对同类事件提供普遍解释
  \item \textbf{牛顿万有引力定律}:统一解释地面现象、天体运动、潮汐等广泛现象的典型例子
  \end{itemize}
\item \textbf{科学说明与非科学说明的根本区别}:
  \begin{itemize}
  \item \textbf{非科学说明特征}:不可检验性、特设性、封闭性、模糊性
  \item \textbf{历史实例}:神秘小鬼、邪恶精灵、行星智慧生物等非科学解释
  \item \textbf{认识态度差异}:科学态度的开放性vs非科学态度的教条性
  \item \textbf{证据基础差异}:科学依赖经验证据,非科学依赖权威传统
  \end{itemize}
\item \textbf{科学态度与非科学态度的深层对比}:
  \begin{itemize}
  \item \textbf{开放性vs封闭性}:科学对新证据开放,非科学拒绝质疑
  \item \textbf{可错性vs绝对性}:科学承认理论可错性,非科学声称绝对真理
  \item \textbf{进步性vs静态性}:科学追求不断改进,非科学维持现状
  \item \textbf{批判性vs权威性}:科学鼓励批判思维,非科学依赖权威
  \end{itemize}
\item \textbf{可检验性:科学说明的核心特征}:
  \begin{itemize}
  \item \textbf{直接vs间接检验}:科学理论通常需要间接检验,涉及辅助前提和假说网络
  \item \textbf{杜恒-奎因论题}:无法孤立检验单个假说,科学知识具有整体性特征
  \item \textbf{可证实性的深层意义}:预测能力、风险承担、内容丰富、进步机制
  \item \textbf{历史验证实例}:相对论光线弯曲预测、元素周期律预言、进化论化石预测
  \end{itemize}
\item \textbf{认识论的根本分野}:
  \begin{itemize}
  \item 科学说明与非科学说明的区别构成现代认识论的根本分野
  \item 可检验性是科学说明最本质的特征,决定了科学知识的可靠性
  \item 科学方法为人类认识世界提供了最可靠的途径
  \item 理解这一区别对建立科学理性的认识基础具有重要意义
  \end{itemize}
\end{itemize}
}}
\end{center}
\section{对科学说明的评价:理论选择的哲学基础}

\begin{logicbox}[title=引言]
本节深入探讨如何评价相互竞争的\logicterm{科学说明},这一问题构成了科学哲学的核心议题。我们将从认识论、方法论和科学史等多个角度分析科学家用于判断假说优劣的三个关键标准:与已有\logicterm{理论}的协调性、\logicterm{预测力}或\logicterm{说明力},以及\logicterm{简单性}。通过深入理解这些评价标准的哲学基础、历史演变和实际应用,我们将能够全面认识\logicterm{科学知识}是如何在\logicterm{理论}竞争中发展的,科学进步的内在逻辑是什么,以及如何在相互矛盾的科学解释之间做出\logicemph{合理选择}。这种理解对于把握科学理性的本质和科学方法的有效性具有重要意义。
\end{logicbox}

\subsection{理论竞争的认识论背景:科学进步的动力机制}

\begin{theorembox}[title=理论竞争的普遍性与必然性]
对同样的现象,人们往往会提出不同的、相互不协调的科学说明以对之进行解释。

\textbf{日常生活中的理论竞争}:
我的同事动作生硬,可以解释成她生气了,也可以解释成她害羞。在刑事调查中,对犯罪的认定有两个相互不协调的假说,都对犯罪事实有很好的解释。但是当两个假说不能都真的时候,我们将如何在其中做出选择?

\textbf{科学中的理论选择问题}:
这里,我们在做的是评价相互竞争的科学说明。我们假定两个(或所有的)假说都是相关的并且是可检验的。我们应当采用什么标准以便从手边的理论中选择出最好的理论?

\textbf{发现与证实的区别}:
我们不能指望存在这样的规则,它们引导我们发现假说;发现假说是科学事业的创造方面,它体现了天才和想象力,在某些方面类似于艺术工作。尽管不存在发现新假说的公式,但存在比相关性和可检验性更进一步的标准,我们可以用这些标准对可接受的假说进行确证(conform)。

\textbf{理论竞争的认识论意义}:
\begin{itemize}
\item \textbf{知识增长的机制}:理论竞争是科学知识增长的重要机制
\item \textbf{理性选择的基础}:为科学理性提供了客观的选择标准
\item \textbf{真理逼近的过程}:通过理论竞争逐步逼近真理
\item \textbf{科学进步的动力}:推动科学不断向前发展
\end{itemize}
\end{theorembox}

\begin{examplebox}[title=科学史上的重大理论竞争]
\textbf{天体运动理论的竞争}:
\begin{itemize}
\item \textbf{托勒密地心说}:地球为宇宙中心,行星在本轮上运动
\item \textbf{哥白尼日心说}:太阳为中心,地球和行星绕太阳运动
\item \textbf{第谷体系}:地球静止,太阳绕地球,行星绕太阳
\item \textbf{开普勒椭圆轨道}:行星沿椭圆轨道绕太阳运动
\end{itemize}

\textbf{光的本性理论竞争}:
\begin{itemize}
\item \textbf{牛顿微粒说}:光是由微小粒子组成的
\item \textbf{惠更斯波动说}:光是一种波动现象
\item \textbf{麦克斯韦电磁说}:光是电磁波
\item \textbf{爱因斯坦光量子说}:光具有波粒二象性
\end{itemize}

\textbf{生物进化理论竞争}:
\begin{itemize}
\item \textbf{拉马克获得性遗传}:后天获得的性状可以遗传
\item \textbf{达尔文自然选择}:通过自然选择实现进化
\item \textbf{现代综合理论}:结合遗传学的进化理论
\item \textbf{中性进化理论}:分子水平的中性突变和随机漂变
\end{itemize}
\end{examplebox}

\subsection{理论评价的三大标准:科学理性的操作化}

人们在评判竞争的科学假说的优缺点时普遍地使用三个标准:

\subsubsection{标准一:与原有已确立假说的协调性}

\begin{theorembox}[title=协调性标准的理论基础]
科学的目标是获得一个说明性的假说系统。当然,这样的系统必须是自我相容的,因为没有一个自相矛盾的命题集合能够是真的。

\textbf{渐进式科学进步}:
进步之得出是通过渐渐发展假说以理解越来越多的事实,这样的进步要求每个新假说应与已经得到证实的那些假说相一致。

\textbf{海王星发现的经典案例}:
例如,在天王星轨道外面存在另外一个未知的行星的假说,与天文学理论的主要部分吻合完美,它导致海王星的发现(1846年)。\cite{kuhn1957} 科学中的进步是有序的,这要求任何新的理论与以前的理论相一致。

\textbf{协调性的认识论价值}:
\begin{itemize}
\item \textbf{知识的累积性}:确保科学知识的累积性增长
\item \textbf{理论的一致性}:维护科学理论体系的逻辑一致性
\item \textbf{预测的可靠性}:基于一致的理论体系进行可靠预测
\item \textbf{解释的统一性}:实现对自然现象的统一解释
\end{itemize}
\end{theorembox}

\begin{theorembox}[title=科学革命与理论替代的复杂性]
科学的理想是通过一个又一个新理论的增加而使知识发生渐渐地增长,但是科学进步的实际历史不总是遵循这种有序的方式。

\textbf{革命性理论变革}:
有时重要的新假说与已有理论不相容,它直接替代了已有理论,而不是努力与旧理论相一致。

\textbf{相对论革命}:
爱因斯坦的相对论就是这种假说,它突破了旧的牛顿理论中的许多原有概念。

\textbf{放射性发现的冲击}:
19世纪后期放射性物质的发现推翻了物质守恒原则。物质守恒原则断定,物质既不能被创造也不能被消灭。镭原子发生自发衰变的假说直接与这旧的、已被接受的原则不相容,最终这个旧的原则不得不被抛弃。

\textbf{理论替代的特征}:
\begin{itemize}
\item \textbf{非连续性}:科学进步有时表现为非连续的跳跃
\item \textbf{概念革命}:涉及基本概念和世界观的根本改变
\item \textbf{范式转换}:从一个理论框架转向另一个理论框架
\item \textbf{解释力提升}:新理论必须具有更强的解释力
\end{itemize}
\end{theorembox}

\begin{theorembox}[title=理论继承与发展的辩证关系]
科学中旧理论被抛弃,较新的和较好的理论被接受,这个过程不是很快或者无抵抗的。

\textbf{理论的继承性}:
事实上,旧理论不是被认为一无是处地被抛弃。爱因斯坦自己总是坚持,他自己的工作是对牛顿工作的一个修正,而非抛弃。

\textbf{概念的扩展}:
物质守恒原则被修正成更为广泛的质能守恒原则。

\textbf{理论替代的条件}:
一个理论之被建立,因它显示出能够解释大量的数据或已知事实的能力。它不能被某些新假说所废弃,除非新假说对同样的事实能够进行解释甚至更好。

\textbf{对应原理}:
新理论必须能够在适当的条件下还原为旧理论,这确保了科学知识的连续性和累积性。例如:
\begin{itemize}
\item 相对论在低速情况下还原为牛顿力学
\item 量子力学在宏观尺度下还原为经典力学
\item 质能守恒在普通化学反应中表现为质量守恒
\end{itemize}
\end{theorembox}

因此,科学通过采用更为广泛因而更为相关的说明而得以进步。通过

说明,世界将它展现在我们的经验面前。这种进步不会是反复无常的。当不相容产生的时候,一个假说的年岁较长不能自动证明它是正确的,但是这个假定(年岁较长)有利于旧假说——如果旧假说已经得到广泛的确证。如果与它发生冲突的新假说同样获得广泛的确证,考虑假说的年岁或提出的先后是不合适的。当两个假说发生冲突的时候,为了在它们间做出选择,我们必须求助于可观察的事实。上诉的最终法庭是经验。

这个标准一一与先前良好建立的假说相协调——所要达到的最终结果是,任何时候所接受的假说全体必须是相互融贯的。\cite{blanshard1939} 在其他情况一样的条件下,与已接受的科学理论吻合得较好的假说应当被偏爱。与"其他情况一样"有关联的问题将我们带到第二个标准。

\subsection{预测力或说明力}
正如我们已经看到的,每个科学假说必须是可检验的;如果某个可观察的事实能够从中演绎出来,它就是可检验的。当我面临两个可检验的假说,其中一个比另外一个演绎出更大范围的事实,我们说该假说具有较大的预测力或解释力。

举例来说明。伽利略(1564-1642)建立了落体定律公式,该定律对靠近地球表面的物体的行为给出了一个十分普遍的解释。差不多同时,德国天文学家乔哈恩斯•开普勒,用丹麦的第谷•布拉赫收集的天文数据建立了行星运动定律,该定律描述了行星绕日运行的椭圆轨道。这些科学家将各自研究领域里(伽利略——陆地上的力学,开普勒——天体力学)的不同现象统一起来。这些发现自然是辉煌的成就,但是它们是各自分离的。艾萨克-牛顿提出了三大运动定律和万有引力理论,将这些分离理论统一了起来并给予了解释。牛顿万有引力解释了所有伽利略和开普勒解释的结果,以及除此之外更多的事实。从一个给定假说中演绎出一个可观察的事实,我们说该事实被该假说所说明,并且我们也能够说该事实被该假说所预测。牛顿理论具有巨大的预测力。一个假说预测力越大,它解释得越多,并且它对我们理解它所涉及的现象的贡献越大。\cite{braithwaite1960}

这第二个标准具有否证作用。如果一个假说与某个得到证实的观察不一致,该假说便是错的,必须被拒绝。当两个不同假说都能完全解释某个事实集合,都是可预测的,并且都与已经构建的整个科学理论相协调,此时,在它们之间做出选择是可能的:从它们推出可检验的但相互不协调的命题。为了在冲突的理论中做出选择,可以建立一个判决性实验。根据第

一个假说,在确定条件下一个给定结果将发生;而根据第二个假说,在那些同样的条件下给定结果将不发生,我们可以通过观察该结果发生还是不发生,而在两个假说中做出选择:它的发生否证了第二个假说,它的不发生则否证了第一个假说。

对两种相竞争的假说进行判决的这种判决性实验也许不容易实现。原因在于,制造那些关键性事件是困难的或者不可能的。牛顿理论和爱因斯坦广义相对论之间的决策不得不等到日全食的发生——这是一个明显超出我们自己能够创立的事件。\cite{eddington1919} 在其他情况,判决性实验可能要等到新工具的发明:这些新工具或者是为了创造所需的条件,或者是为了对已经做出预测的现象进行观察或测量。因此,竞争的天文假说的支持者有时必须等待时机,等待建造出新的、威力更强大的望远镜。对于判决性实验,在 13.6 节中我们将进一步讨论。

\subsection{简单性}
两个竞争性假说可能是相关的和可检验的,可能与已有理论吻合得同样好,甚至可能具有大致相当的预测力。在这样的条件下,我们可能支持两个中比较简单的那个。关于天体运动的托勒密理论(地球中心)和哥白尼理论(太阳中心)之间的冲突就是如此。两者都与早先的理论吻合良好,它们都同等程度地预测天体运动。两个假说都依赖于一个笨拙的(自然是错误的)工具——假想的本轮(较小的圆在较大圆上运动),以解释已做出的天文观察。但是哥白尼系统依赖这样的本轮更少,因而它更简单,这个较大的简单性是后来的天文学家接受该理论的主要原因。\cite{kuhn1957b}

简单性似乎是一个可以求助的"自然"标准。在日常生活中我们同样趋向于接受符合所有事实的最简单的理论。在刑事法庭上对一犯罪行为会提出两种观点,最终在该案子上更简单、更自然的观点可能被支持(或应当被支持)。

但是"简单性"是一个不好捉摸的观念;只有在非常少量的情况下,如在托勒密和哥白尼的冲突中,我们根据较少的实体数的要求选择比较简单的理论。在两个竞争的理论中可能是,在不同的方面一个比另外一个简单。一些人可能依赖于比较少的实体数量,而其他人可能基于较简单的数学方程。甚至"自然"(naturalness)可能是欺骗人的。许多人会更"自然"地相信,明显不运动的地球事实上是不动的,而明显运动着的太阳确实环绕我们在运行。简单性是一个重要的标准,有时甚至是决定性的。但

是它是难以公式化,并且不总是易于应用的。

\begin{center}
\fbox{\parbox{0.95\textwidth}{
\textbf{本节要点}
\begin{itemize}
\item \textbf{理论竞争的认识论背景}:
  \begin{itemize}
  \item \textbf{理论竞争的普遍性}:同一现象常有多种相互竞争的科学解释
  \item \textbf{发现与证实的区别}:发现假说依赖创造力,评价假说需要严格标准
  \item \textbf{认识论意义}:理论竞争是知识增长机制、理性选择基础、真理逼近过程
  \item \textbf{历史实例}:天体运动、光的本性、生物进化等重大理论竞争
  \end{itemize}
\item \textbf{标准一:与原有已确立假说的协调性}:
  \begin{itemize}
  \item \textbf{理论基础}:科学追求自洽的假说系统,确保知识累积性增长
  \item \textbf{渐进式进步}:新假说应与已证实理论相一致,如海王星发现案例
  \item \textbf{科学革命的复杂性}:重大理论变革涉及概念革命和范式转换
  \item \textbf{理论继承的辩证关系}:新理论通过对应原理保持与旧理论的连续性
  \item \textbf{经验的最终裁决}:理论冲突时必须求助于可观察事实
  \end{itemize}
\item \textbf{标准二:预测力或说明力}:
  \begin{itemize}
  \item \textbf{预测力的定义}:能从假说演绎出更广泛事实的理论具有更大预测力
  \item \textbf{牛顿理论的统一}:统一并解释了伽利略和开普勒的分离理论
  \item \textbf{否证作用}:若理论与观察不符,必须被拒绝
  \item \textbf{判决性实验}:可以在竞争理论间做出选择的关键实验
  \item \textbf{实验的挑战}:判决性实验可能需要等待特殊条件或新技术
  \end{itemize}
\item \textbf{标准三:简单性原则}:
  \begin{itemize}
  \item \textbf{简单性的价值}:在其他条件相同时,应选择更简单的理论
  \item \textbf{历史案例}:哥白尼日心说比托勒密地心说更简单
  \item \textbf{概念的复杂性}:简单性标准难以精确定义,不同理论可能在不同方面简单
  \item \textbf{应用的挑战}:简单性是重要标准,但难以公式化且应用存在困难
  \item \textbf{自然性的欺骗}:直觉上的"自然"可能是误导性的
  \end{itemize}
\item \textbf{理论选择的哲学意义}:
  \begin{itemize}
  \item 三大标准构成了科学理性的操作化体现
  \item 理论评价是科学方法的核心组成部分
  \item 标准的应用需要科学共同体的集体判断
  \item 理论选择体现了科学进步的内在逻辑
  \end{itemize}
\end{itemize}
}}
\end{center}
\section{科学研究的七个阶段:系统化探索的方法论框架}

\begin{logicbox}[title=引言]
本节深入描述\logicterm{科学研究}过程的七个关键阶段,这一框架构成了现代科学方法论的核心。我们将从认识论、方法论和科学史等多个角度分析从确定问题到应用\logicterm{理论}的完整\logicterm{科学研究}流程,深入理解科学家如何通过系统化、规范化的方法从初始观察发展到成熟\logicterm{理论},以及这一过程如何体现了科学理性的本质特征。通过全面了解这一过程的逻辑结构、认识价值和实践意义,我们将认识到\logicterm{科学}不仅是一种知识体系,更是一种\logicemph{严谨的}、可重复的探索方法,是人类认识世界最可靠的途径。
\end{logicbox}

\begin{theorembox}[title=科学研究方法论的理论基础]
科学研究的七个阶段框架体现了现代科学方法论的系统化特征。

\textbf{历史发展脉络}:
\begin{itemize}
\item \textbf{培根的归纳法}:强调从观察到理论的系统归纳
\item \textbf{笛卡尔的演绎法}:重视从基本原理出发的逻辑推演
\item \textbf{牛顿的综合方法}:结合归纳和演绎的科学方法
\item \textbf{现代假说-演绎法}:以假说为核心的科学研究模式
\end{itemize}

\textbf{方法论特征}:
\begin{itemize}
\item \textbf{系统性}:各阶段相互关联,形成完整的认识循环
\item \textbf{可重复性}:方法可以被其他研究者重复应用
\item \textbf{自我修正性}:通过检验和反馈不断完善理论
\item \textbf{开放性}:接受新证据和理论修正
\end{itemize}

\textbf{认识论意义}:
这一框架体现了科学认识的基本规律,是人类理性认识世界的最有效途径。
\end{theorembox}

\subsection{第一阶段:确定问题——科学探索的起点}

\begin{theorembox}[title=问题确定的认识论分析]
科学研究开始于某个问题。一个问题可以表示成一个或一组没有可接受的说明的事实。

\textbf{问题的本质特征}:
\begin{itemize}
\item \textbf{认识缺口}:现有知识无法解释的现象
\item \textbf{理论冲突}:不同理论预测的矛盾
\item \textbf{异常现象}:不符合预期的观察结果
\item \textbf{实践需求}:现实问题需要科学解决方案
\end{itemize}

\textbf{问题发现的方式}:
例如,侦探面临一个案子,他的问题是如何将之侦破,即确定犯罪人并给予证明。在某些情况下,如在柯南•道尔关于伟大的夏洛克-福尔摩斯的故事中,问题产生于尚未发生犯罪的特定事件或环境之中。

科学家的研究可能开始于十分明确的问题;然而,更为普遍的是,他们是渐渐地发现了不相容或奇怪之处,这些不相容或奇怪之处演化成一个特定问题。

\textbf{问题的重要性}:
如果不存在可思考的事情,甚至夏洛克•福尔摩斯或者爱因斯坦也不能从事深刻的思考。一个天才必须面对一个问题。

\textbf{杜威的问题解决理论}:
正如约翰•杜威和许多其他现代哲学家正确认为的那样,反思性的思考——从犯罪侦查到物理学、数学的抽象思考这个范围广泛的活动——是解决问题的(problem-solving)活动。

\textbf{问题确定的必要性}:
科学家开始工作之前,问题必须被确定,或者至少以模糊的形式被确定。
\end{theorembox}

\begin{examplebox}[title=科学史上的重大问题确定实例]
\textbf{物理学领域}:
\begin{itemize}
\item \textbf{黑体辐射问题}:经典物理学无法解释黑体辐射谱,导致量子力学的诞生
\item \textbf{水星近日点进动}:牛顿力学无法完全解释,促成广义相对论
\item \textbf{以太漂移实验}:迈克尔逊-莫雷实验的零结果引发相对论革命
\end{itemize}

\textbf{生物学领域}:
\begin{itemize}
\item \textbf{物种起源问题}:生物多样性和适应性的解释需求
\item \textbf{遗传机制问题}:性状遗传的物质基础探索
\item \textbf{DNA结构问题}:遗传信息存储和传递的分子机制
\end{itemize}

\textbf{医学领域}:
\begin{itemize}
\item \textbf{传染病病因}:细菌理论的建立过程
\item \textbf{抗生素耐药性}:新的治疗策略需求
\item \textbf{癌症机制}:细胞恶性转化的分子基础
\end{itemize}
\end{examplebox}

\subsection{第二阶段:形成初始假说——理论建构的起点}

\begin{theorembox}[title=初始假说的认识论功能]
对手边的问题的哪怕最初始的思考都要求某个初步理论。

\textbf{假说的指导作用}:
最初的尝试不可能产生最后的答案,但是需要某个理论以便能够知道,需要收集何种类的证据,到哪里寻找它们,以及如何寻找它们。

\textbf{事实与理论的关系}:
侦探考察犯罪现场,询问嫌疑人,并寻找线索,但是赤裸裸的事实不是线索。只有当线索能够被安排进某个融贯的模式,哪怕是粗糙的和临时的模式之中时,它们才有意义。

\textbf{科学研究中的假说功能}:
科学家与此相同,他们用某个初始假说开始收集证据。这个假说是关于待寻求的说明的本质的。

\textbf{知识的累积性}:
科学家必须依赖于某些以前的知识,科学不会从绝对无知中开始。事实上,如果被说明的事实出现真正的问题,必定存在某些先验的信念。

\textbf{选择性注意的必要性}:
对于任何一个严肃的问题,世界上存在太多的相关的事实、太多的数据、以至于人们不能将它们全部收集起来。一些事实将被注意并被观察,另外的事实则没有。

\textbf{假说的筛选功能}:
最耐心和最全面的研究者必须选择:被发现的事实中哪些要研究,哪些要放弃。这需要某个假说来工作:为了这个假说或者根据这个假说而收集相关的证据。

\textbf{初始假说的特征}:
该假说不必是完善的理论——但是在它那里至少显示出理论的轮廓。否则的话,研究者不能确定从整个事实全体中挑选出何种事实来。一个临时的初始假说不管是如何的不完善,任何严格的探究开始的时候它都是必需的。
\end{theorembox}

\subsection{第三阶段:收集额外事实——假说与证据的互动循环}

\begin{theorembox}[title=证据收集的系统性特征]
一般来说,最初令人迷惑不解的事实似乎太多了,以至于不能提出一个对它们非常满意的说明;如果情况不是这样的话,这些事实不可能表现出问题来。

\textbf{专业知识的重要性}:
但是,特别的,对那些熟悉普遍种类(如天体现象、社会现象或历史现象)的事实或事件的科学家而言,原初的问题会激发出一个初始假说,该假说引导他们寻找额外的相关事实。

\textbf{证据的指导功能}:
这个额外证据可以起到引导作用,引导我们得出较完全和较接近的合适答案。

\textbf{科学工作的现实性}:
收集证据的任务,既艰辛又耗时,经常是失望和沮丧。好的科学意味着艰巨的工作。这个费力的收集过程是许多科学工作的主要内容。

\textbf{证据收集的特征}:
\begin{itemize}
\item \textbf{目标导向性}:由假说指导的有针对性收集
\item \textbf{系统性}:按照科学方法进行规范化收集
\item \textbf{持续性}:需要长期坚持和反复验证
\item \textbf{批判性}:对证据质量进行严格评估
\end{itemize}
\end{theorembox}

\begin{theorembox}[title=假说与证据收集的动态关系]
在实际的科学活动中,步骤2和步骤3自然不是完全分离的。它们紧密相连、相互依赖。

\textbf{循环互动过程}:
在开始收集证据时,某个初始假说是必需的;使用该假说来收集证据的过程,也是调整和精练假说本身的过程,这又引导我们进一步地寻找……也许导致新的发现……它又使我们更加精练假说,等等。

\textbf{动态关系的特征}:
\begin{itemize}
\item \textbf{相互修正}:证据修正假说,假说指导证据收集
\item \textbf{螺旋上升}:在循环中逐步提高认识水平
\item \textbf{开放性}:随时准备接受新证据和修正理论
\item \textbf{创新性}:在互动中可能产生意外发现
\end{itemize}

\textbf{认识论意义}:
这种动态关系体现了科学认识的辩证性质,既不是纯粹的归纳,也不是纯粹的演绎,而是两者的有机结合。
\end{theorembox}

\subsection{4.进行预测}
在任何成功的研究中迟早会达到这样一点,研究者(科学家、侦探,甚至普通人)将最终相信,解决原初问题所需要的所有事实都已经获得。一个难题,更可能的是一组难题,摆在他或她的面前,其任务是将它们组装成一个可以理解的整体。这样思考的终结产品——如果成功的话,是这样的假说:它将解释所有数据、产生问题的原有事实集,以及初始假说所涉及的额外事实。

不存在实现某个完善理论的机械方法。对真正说明性的假说的实际发

现或发明是一个创造性的过程,在这个过程中需要想象,也需要知识。某些研究者,如夏洛克•福尔摩斯和阿尔伯特•爱因斯坦,在对存在的现象进行说明的"逆向推理"的过程中展示了其才能。但是每一个成功的科学家必须完成智力整合这个挑战性的任务:对激发研究兴趣的成问题事实进行解释,以构造和形成最终假说。\cite{peirce1958}

\subsection{5.进行实验}
我们已经看到预测力是评价说明的标准之一。一个真正富于成果的假说将不仅说明激发假说形成的原初事实,而且解释许多其他的事实。好的假说超越初始的事实,它涉及新的和不同的事实——这些事实较早没有被怀疑。假说所引出的这些事实被证实,使得假说得以确证——当然不能给予确定性的证明。

被称做"大爆炸的"宇宙学理论可以看做对这样的预测进行阐述的例子。这个理论认为,如果目前的宇宙开始于一个大爆炸事件,最初的火球应是平稳的和均匀的,而没有任何结构。与此对照的是,目前的宇宙具有大量的结构,是多块状的;可见物质组成星系、星系群,等等。这样的结构对生命的起源和演化是基本的。但是该结构是何时产生并如何产生的?通过观察膨胀的宇宙中那些遥远的天体,天文学家能够"回顾过去"。通过这样的观察,他们最终必定能够找到目前结构的原初证据。如果如此早的证据不能通过最敏感的仪器来探测到,大爆炸理论将是不可辩护的。如果这样的结果被探测到,大爆炸理论得到确证,尽管不是被证明。

\subsection{对结果进行检验}
在生物学领域里我们可以提出这个假说:在哺乳动物中蛋白质之产生是为了对抗特定的酶,而这种酶是在一个特定基因引导下产生的。从该假说中我们可以推论出进一步的结论:缺少该基因的地方,该蛋白质将不出现或者蛋白质数量不足。

为了检验该生物学假说是否正确,我们构造某个特定基因的作用能够被测定的实验。经常的做法是,将去除特定基因的老鼠进行繁殖——被称为"基因剔除老鼠"。如果在这样的老鼠中被研究的酶以及与之有关的蛋白质发生缺失,我们的假说将得到证实。\cite{capecchi1994} 在医学中许多有价值的信息正是以这种方法获得的。这种实验是广泛的生物学研究中典型的实验。我们设计实验,以弄清我们认为是对的东西,在如此这般的条件得到满足的情况下,是否确实是真的。为此,我们必须构造这样或那样的特定的条件。\\
"一个实验",正如伟大的物理学家马克斯•普朗克说,"是科学给自然提出的一个问题,而测量是对自然回答的记录。"

对某些预测的结果,如同夏洛克-福尔摩斯的许多预测的结果,其检验可能是直接的。银行窃贼将打破拱顶而人?福尔摩斯和华生等待他们,并且他们确实来了。\cite{doyle1927} 医生将避开从假的通风口进来的毒蛇吗?福尔摩斯和华生从躲藏的地方观看,发现医生避开了。\cite{doyle1927b} 那些说明性的理论直接地被检验,并牢固地得到了证实。

当然,许多科学理论不能被简单的观察所检验。早期宇宙的结构不可能被直接观察到。但是,如果存在某个早期的结构,如大爆炸理论预测的那样,那么,由于目前的背景辐射根源于早期,在它之中将必定存在不规则、不均匀。在原则上测量背景中的微波辐射是可能的,因而我们能够以这种方式间接地确定在大爆炸之后十分短暂的时刻是否存在这样的不规则性。几年后,一个卫星被设计来探测这些不规则性——如果它们存在的话。该卫星(宇宙背景探测者 COBE)的观察对大爆炸理论是至关重要的。如果最终没有探测到长期寻找的宇宙中早期结构的证据,膨胀宇宙的大爆炸解释将遭受严重的质疑。然而,1992年春天,预测到的不规则性被 COBE 探测到并被测量出来。这些不规则性来自于最遥远的过去,一直存在到宇宙学家回顾它们的今天。这个成功检验尽管不能证明该理论是正确的,但的确给人印象深刻地确证了大爆炸理论。

\subsection{应用该理论}
通过科学,我们的目的是说明我们观察到的现象,但是我们另一个目的是控制这些现象,为我们所用。牛顿和爱因斯坦的抽象理论在太阳系的现代探索中发挥中心作用。举一个不同种类的例子,假定我们面临的问题是某个疾病,发明的说明性假说是某个特定细菌引起该疾病。假定通过给老鼠或啮齿动物注射该病菌,对该理论进行检验,并假定在进行实验的动物中产生了该种同样的疾病,这些检验给说明性假说以强的支持。我们当然试图在临床医学中使用该理论。做法是,通过消灭患有该病的病人身上的细菌而将病治愈——先在实验人群中进行,然后再按照常规来进行。正是按照这种方法,我们学会了如何与许多可怕的人类疾病进行战斗,在一些情况下甚至完全消灭这些疾病。通过科学,我们试图理解世界;同样通过科学,我们使用一些手段,对世界给予我们的危险进行控制。

\section{科学研究的七个阶段}
1.确定问题\\
2.选择初始假说\\
3.收集额外事实\\
4.形成说明性假说\\
5.推导进一步结果\\
6.检验结果\\
7.应用理论

\begin{center}
\fbox{\parbox{0.95\textwidth}{
\textbf{本节要点}
\begin{itemize}
\item \textbf{科学研究方法论的理论基础}:
  \begin{itemize}
  \item \textbf{历史发展}:从培根归纳法、笛卡尔演绎法到现代假说-演绎法的演进
  \item \textbf{方法论特征}:系统性、可重复性、自我修正性、开放性
  \item \textbf{认识论意义}:体现科学认识的基本规律,是人类理性认识世界的最有效途径
  \item \textbf{框架价值}:七个阶段构成现代科学方法论的核心
  \end{itemize}
\item \textbf{第一阶段:确定问题——科学探索的起点}:
  \begin{itemize}
  \item \textbf{问题本质}:认识缺口、理论冲突、异常现象、实践需求
  \item \textbf{问题重要性}:天才必须面对问题,反思性思考是问题解决活动
  \item \textbf{杜威理论}:从犯罪侦查到物理学的广泛问题解决活动
  \item \textbf{历史实例}:黑体辐射、水星进动、物种起源、传染病病因等重大科学问题
  \end{itemize}
\item \textbf{第二阶段:形成初始假说——理论建构的起点}:
  \begin{itemize}
  \item \textbf{指导作用}:确定收集何种证据、到哪里寻找、如何寻找
  \item \textbf{事实与理论关系}:赤裸裸的事实不是线索,需要融贯的模式
  \item \textbf{知识累积性}:科学不从绝对无知开始,依赖先验信念
  \item \textbf{筛选功能}:从无限事实中选择相关数据,显示理论轮廓
  \end{itemize}
\item \textbf{第三阶段:收集额外事实——假说与证据的互动循环}:
  \begin{itemize}
  \item \textbf{系统性特征}:目标导向性、系统性、持续性、批判性
  \item \textbf{专业知识重要性}:熟悉相关领域的科学家更能有效收集证据
  \item \textbf{动态关系}:假说与证据收集相互修正、螺旋上升、开放创新
  \item \textbf{认识论意义}:体现科学认识的辩证性质,归纳与演绎的有机结合
  \end{itemize}
\item \textbf{后续阶段的系统性}:
  \begin{itemize}
  \item \textbf{形成说明性假说}:创造性的智力整合过程,需要想象和知识
  \item \textbf{进行预测和实验}:检验假说的预测力,通过实验验证理论
  \item \textbf{检验结果}:直接或间接检验,如COBE卫星验证大爆炸理论
  \item \textbf{应用理论}:理解世界并控制现象,服务于人类实践需求
  \end{itemize}
\item \textbf{七个阶段的整体特征}:
  \begin{itemize}
  \item 各阶段相互关联,形成完整的科学认识循环
  \item 实际研究中各阶段交错进行,体现科学方法的灵活性
  \item 体现了科学作为系统化、可重复探索方法的本质特征
  \item 是人类认识世界最可靠途径的方法论体现
  \end{itemize}
\end{itemize}
}}
\end{center}
\section{实际工作中的科学家:科学研究的模式——DNA双螺旋发现的案例分析}

\begin{logicbox}[title=引言]
本节通过DNA双螺旋结构发现这一20世纪最重要的科学突破,深入展示\logicterm{科学研究}的七个阶段如何在真实科学探索中体现。我们将从科学史、认识论和方法论等多个角度分析这一著名科学发现的完整过程,深入理解科学家如何从初始问题出发,通过\logicterm{假说}形成、证据收集、理论建构、实验验证最终达成重大突破。通过这一经典案例研究,我们将全面认识\logicterm{科学研究}的理论模式如何在实践中应用,\logicemph{创造性思维}与\logicemph{系统方法}如何共同推动科学进步,以及科学发现的社会性、竞争性和合作性特征。
\end{logicbox}

\subsection{科学研究模式的实践价值与案例选择}

\begin{theorembox}[title=科学研究模式的实践价值]
通过上面概述的七个阶段,我们可以更好地理解科学研究的一般过程。但是,这样一个模式的价值不在于抽象的描述,而在于它指出了所有成功的科学研究所共有的特征。

\textbf{理论与实践的关系}:
将这一模式与成功的科学研究进行对照,能很好地说明这一点。理论模式为我们提供了分析和理解科学发现过程的框架,而实际案例则验证了这一模式的有效性。

\textbf{科学方法的普遍性}:
渗透所有科学探究的模式可以表述成前面一节中给予解释的七个步骤。当然不是说只有科学家在使用科学方法;任何人只要遵循从可观察的事实和证据,推论出可通过经验检验的结论这样的一般的推理模式,都可以说成是科学地工作。

\textbf{科学思维的广泛应用}:
训练有素的侦探是这种意义上的科学家,我们中的大多数人有时也是如此。这表明科学方法不仅是专业科学家的工具,更是人类理性思维的基本模式。

\textbf{案例选择的意义}:
现在我们考察一个遵循这个合理探究模式的例子。我们跟随当代科学家,看一看他们最近是如何对脱氧核糖核酸即DNA结构进行破解的。\cite{watson1968}

\textbf{DNA发现的重要性}:
\begin{itemize}
\item \textbf{科学意义}:解开了生命遗传的分子基础
\item \textbf{方法论价值}:完美体现了科学研究的七个阶段
\item \textbf{历史地位}:20世纪生物学最重要的发现之一
\item \textbf{社会影响}:开启了分子生物学和基因工程时代
\end{itemize}
\end{theorembox}

1.问题。所有生物从一个单细胞开始生长,并且所有生物使自身进行再生产。因而动植物遗传的特点必定隐藏在它们最初的细胞之中,但是它到底隐藏在哪里?基因信息是如何代代相传的?每个发育的有机

体为什么各部分最终均发展为复杂的形式?在 20 世纪中期这个深奥的难解之谜(即解答"生命的秘密"),困扰了相互合作同时也相互竞争的科学家。寻找这种基因是最近的科学史上一个最激动人心的篇章之一。

答案必定存在于构成活细胞的四种物质的一个之中:(1)脂肪(油脂);(2)糖和淀粉(多聚糖);(3)蛋白质;(4)核酸。前两个在该研究开始很久之前就被人们很肯定地排除掉了。第四个,即核酸,它们的化学成分为人们所知晓,它们的构造相当简单,它们各个部件位置固定,次序重复。这些部件中一个部件是糖,被称为核糖;包含糖的这种核酸无所不在,其中一种核酸缺一个氧原子,因而被称为脱氧核酸,或 DNA。那时,人们普遍相信 DNA 是一个"愚蝵的"物质,在细胞中只不过起到使结构坚硬的作用——如同使新衬农保持形状的硬纸板一样。于是人们没有将之当成构成基因的候选材料。

如果 DNA 不携带遗传信息,遗传必定是通过某种还没有找到的蛋白质进行传输的。但是,到1944年有坚实的证据表明,携带基因信息的无论是什么东西,都不可能是蛋白质。然而,在林纳斯•泡林于1949年利用 X 射线的探测技术发现了 $\alpha$ —螺旋(蛋白质中的一个关键组成成分)之后,蛋白质再次成为兴奋的焦点。此外,基因信息的巨大复杂性一一无数细节和特征需要从一代传递到另一代——使得许多科学家相信,基因的秘密可能存在于某些结构极其复杂的大的蛋白质分子之中。人们根据这个思路,广泛地猎求这种基因。即使该思路正确,许多蛋白质的存在也令人无所适从。但最终,在蛋白质中寻找基因的努力都没有成功。

2.初始假说。英国剑桥的卡文迪许实验室的詹姆斯•华生和弗兰西斯•克瑞克1951年开始他们的基因寻找工作。他们的数据令人困惑,也不完整。在信念和广泛的接受的理论之间的不一致更加加大了他们的迷恫。如果1944年奥斯瓦尔德•阿伟力进行的观察数据和排除论证是可靠的,在蛋白质中寻找基因的努力注定要失败。如果这样,正如华生后来所写的,"DNA 将必须充当这个关键钥匙"\cite{watson1968b}。这就是华生和克瑞克开始研究的初始假说:遗传信息是在 DNA 结构中得以携带的。对他们的研究进行指导的另外两个初始假说是:第一,根据罗萨林德•弗兰克林和毛里斯•威尔金斯(他后来与华生、克瑞克一起分享诺贝尔奖)拍下的 X 光衍射照片,DNA 的结构是有规则的。华生写道:

\begin{displayquote}
突然间我对化学感到十分兴奋……我对基因可能是无规则的可能性感到怀疑。尽管如此, 我知道基因能够被弄清, 因而它们必定具有一个规则的结构, 该结构能够被易于理解的方式解开。\cite{watson1968c}
\end{displayquote}

第二,他们假定 DNA 丝——根据它们的强度和衍射图像来看——可能的形状是螺旋形状,或双螺旋形状,或者也许类似于泡林早期在某些蛋白质中发现的 $\alpha$ —螺旋形状。

3.收集额外事实。为了将这些初始假说与已知的但令人迷惘的关于 DNA 组成的事实相协调,人们必须知道更多的东西一其中一些隐藏在科学文献中,一些则刚刚被发现。

核酸被认为有一条长长的"脊椎",它由一个由糖(核糖)和交替出现的磷酸盐(带四个氧原子的磷)所构成。在该"脊椎"的每个关节处,一个被称为一个基(base)的第三个分子单位,被粘在这条链上。每个基是四种中的一种:腺嘌呤(ademine)、鸟嘌呤(guanine)、氧氨嘧啶(cy- tosine)和胸腺嘧啶(thymine),用它们的第一个字母表示,即为A、G、 C、 T 。在脊椎上四种物质出现的次序是个谜,甚至人们不知道这些基是如何和脊椎相连接的。随着更多的数据被收集到,以及随着初始假说得以精练,问题转变成如何将 DNA 片段组合起来。该链条上每个三片的单位 (糖,加上磷酸,加上一个基)被叫做"核苷"(nucleotide)。核苷如何组合在一起以形成被称做 DNA 的酸呢?迷惑他们的更一般的问题("什么是一个基因"),被华生和克瑞克精练成易处理的结构问题。

起初的进展十分缓慢。他们使用硬纸和金属丝组成超大尺寸的模型,尝试着构造各种他们能够发明的链条形状的结构。某些特别的条件必须得以满足:水的含量,倾斜的角度,特定的化学结合方式。每件事情必须与以前发现的事实、最近的 X 光图像和已有理论相一致。已经知道四个基 (A、G、C 和 T)是平躺着。华生和克瑞克尝试着将它们看成盘子般连接在螺旋脊椎的内部或外部或者相互连接的模型。螺旋的角度被调整,糖分子之间的连接理论被重新考察。然而各种尝试均无法形成一个合理的结构。

4.形成精练的说明性假说。一个重大问题的解决经常有赖于来自不同领域里的成果;它往往是一个合作性的事业,但有时是高度竞争的。其

他科学家同样赛着解决 DNA 的结构。威尔金斯和弗兰克林在他们伦敦的实验室里获得较好的 X 射线衍射图像。泡林描述了他所认为的 DNA 结构——三条链的螺旋,但是华生和克瑞克根据足够的信息认识到,泡林的解释有一个致命的错误,他们既失望又高兴。泡林的手稿一经发表,华生写道:

\begin{displayquote}
仅几天时间,错误就被发现。在林纳斯再次用全部时间追寻 DNA 之前的六周里,我们没有开始我们的研究……我让弗兰西斯(克瑞克)给我买了威士忌。林纳斯仍没有赢得诺贝尔奖。\cite{watson1968d}
\end{displayquote}

能够解决问题的这个精练的假说,必须对基因的两个不同的能力做出解释:(1)生命结构中无数的细节是如何在遗传信息中被传输的?以及 (2)遗传信息在下一代中如何使自己得以复制的?所需要的是一个与已知事实和理论相一致的三维结构,它为生命的所有细节提供编码,并且它能够一代接一代地复制自身。

哥伦比亚大学的爱尔文•查尔格夫的探究帮助他们走上了正确的轨道。查尔格夫做出了一个震惊的发现:在 DNA 的所有测试样品中,四个基——腺嘌呤、乌嘌呤、氧氨嘧啶和胸腺嘧啶——的相对数量是固定的。其中两个 $\mathrm{A} 、 \mathrm{G}$ 被称为嘌呤(purines),另外的两个 C 和 T 被称为嘧啶酮(pyrimidines)。查尔格夫已经证明 A 分子的数量总是与 T 分子的数量相等,$G$ 分子的数量总是与 $C$ 分子的数量相等。嘌呤(A 和 $G$ )的数量总是与嘧啶酮(T 和 C )的数量相等。但是没有人能够解释为什么这样。

通过计算和模型处理,克瑞克确定出 A 和 T 的结构是这样的,它们总是自然地粘在一起,并且,将 G 和 C 相互吸引在一起的力也能够被找到。如果 DNA 链条是构造成这样的,对任何一个 A 都存在一个对应的 $T$ ,并且对于每个 $G$ 都有相应的 $C$ ,那么,这个链条——如果在中间裂开的话——将能够给出一个自我复制的绝妙系统:这链条的每一边可以看做一把锁,而另外一边是它的钥匙;每个链条是一个建造新的匹配的钥匙的模板。如果含有匹配基的链条很长,它们的次序和数字能够解释所要求的关于细节的遗传编码问题。他们假设的答案是某种双螺旋。他们尝试将脊椎安排在中心、基向外伸出的模型;他们尝试着将脊椎安排在外面、基向

内射出的模型。他们仍然没有成功。然而他们相信已经接近弄清 DNA 的结构了。

问题出在基(A、G、T 和 C )是如何地相互连接的这个已经接受的理论之中吗?如果已经接受的理论有缺陷,能够被基之间相互的化学连接的新的理论所代替,那么,发明一个双螺旋模型可能是可行的。他们探索了该可能性。各个谜开始在华生的脑海里组合:

第二天早上我来到仍然是空空荡荡的办公室,我迅速将我桌子上面的论文挪到一边,以使我能够有大片的地方将基对通过氧结合物而组合在一起。 ${ }^{[19]}$

仍然没有成功。一边的 A 与另外一边的 A 匹配, C 和 C 相匹配,等等,基指向内部,并在链条的空的中间相互交叉地相互连接,然而它们不能被直接安排进双螺旋中去。

华生修改这个结构以便该理论的各个部分能够相协调,终于,华生能够建立一个完全精练的并被证明是正确的假说:DNA 分子确实是一个双螺旋,其中基确实是指向内一一但是基对的匹配是互补的:每个 A 匹配一个 T ,每个 G 匹配一个 C 。

我……开始将基转到朝向内部,排除掉其他不同的匹配可能性。突然,我知道由两个氢结合物握住的腺嘌呤一胸腺嘧啶对,在形状上与由至少两个氢结合物握住的鸟嘌呤一氧氨嘧啶对一样。所有的氢结合物似乎自然地形成,不需要捏造使得两个类型的基对(base pairs)在形状上一样......

对于为什么这个嘌呤( A 和 G )数量与嘧啶酮( C 和 T )的数量完全相等这个谜,我们有了答案。两个不规则的基序列能够被规则地压缩进[双]螺旋的中心……腺嘌呤总是与胸腺嘧啶配成对,而鸟嘌呤总是与氧氨嘧啶配成对……两个互相缠绕的链条的基序列是互补的。给定一个链条的基序列,它的伙伴的基序列可以自动地得以确定。

于是,理论上我们能够十分容易地形象化地表示,单个的链条如何成为合成它的互补序列链条的模板。\cite{watson1968e}(见图13-1)

当弗兰西斯•克瑞克在剑桥的鹰酒馆吃午餐时告诉每个人,"我们已经找到生命的秘密"时,华生写道:"我感到有点不安"\cite{watson1968f}。\\
\includegraphics[width=\textwidth]{images/2025_05_15_6a28331d5e7c993ad07ag-590.jpg}

图 13-1 一个互补的双螺旋的示意图。两个糖一磷酸盐的脊椎在外面卷曲着。平躺着的基组成核,它们成对出现——A 总是和 $\mathbf{T}$ 连接, $\mathbf{C}$ 总是和 $\mathbf{G}$ 连接。该结构形成一个螺旋楼梯,成对的基形成胗梯。\\
摘自于 J.D.Watson,The Double Helix.p.130,这里的引用得到版权所有人 G.S.斯坦特的同意。

5 和 6.演绎和检验结果。假说已经形成,接着要对它进行检验。首先,直接的推论是:如果华生和克瑞克提出的双螺旋确实是对 DNA 的正确解释,那么,构造这样一个三维的双螺旋模型是可能的,即所有基都被安排在内部,并且螺旋的角度以及链条的其他特点应当符合以前的 X 光图像及其他的实验结果。这个模型很快就得到了。

许多其他的理论推论被得到,每一个推论的检验都获得了成功。令分子生物学家长期困扰和沮丧的某些数据,由于这样的分析而得以理解。人们了解到,来自于父母的生殖细胞中的 DNA 数量,在普通细胞中只发现了一半。现在我们清楚了个中原因:如果双螺旋在生殖过程中裂开,来自于双亲的裂开的细胞自然只包含正常 DNA 数量的一半。华生一克瑞克对 DNA 结构的解决的正确性的证据迅速增加;不久,他们的假说非常充分

地得到证实。\\
7.应用。华生-克瑞克关于 DNA 结构 128 页的报告\cite{watson1953} 创造了科学史,它对生物学进程的改变是巨大的也是持久的。该知识的广泛且威力强大的运用将他们的成就推到顶峰。在随后的几十年里,人们认识了 DNA序列中使用的编码;整个人类基因组的完整图实际上是完全的,不久人类将得到该图。将 DNA 链进行切割、重组的技术已经得到发展,它们已经在新药、疫苗和人造荷尔蒙的制造中普遍地得到应用。重新组合 DNA 技术仅在 DNA 结构被最终解决的条件下才是可能的,该技术的应用使生物学和医学发生革命,并且仍保持着旺盛的应用生命力。

\begin{center}
\fbox{\parbox{0.95\textwidth}{
\textbf{本节要点}
\begin{itemize}
\item \textbf{科学研究模式的实践价值与案例选择}:
  \begin{itemize}
  \item \textbf{理论与实践关系}:理论模式提供分析框架,实际案例验证模式有效性
  \item \textbf{科学方法普遍性}:不仅限于专业科学家,是人类理性思维的基本模式
  \item \textbf{DNA发现重要性}:科学意义、方法论价值、历史地位、社会影响的完美结合
  \item \textbf{案例选择意义}:20世纪生物学最重要发现,完美体现科学研究七个阶段
  \end{itemize}
\item \textbf{第一阶段:问题确定——生命遗传的根本谜题}:
  \begin{itemize}
  \item \textbf{核心问题}:遗传特点隐藏在哪里?基因信息如何代代相传?
  \item \textbf{候选物质}:脂肪、糖类、蛋白质、核酸四种可能的遗传物质
  \item \textbf{理论困境}:DNA被认为是"愚蠢"物质,蛋白质成为主要候选
  \item \textbf{证据转折}:1944年证据表明遗传物质不可能是蛋白质
  \end{itemize}
\item \textbf{第二阶段:初始假说——DNA作为遗传信息载体}:
  \begin{itemize}
  \item \textbf{华生-克里克假说}:遗传信息在DNA结构中被携带
  \item \textbf{结构假说}:DNA具有规则结构,可能是螺旋或双螺旋形状
  \item \textbf{X射线证据}:弗兰克林和威尔金斯的衍射照片支持规则结构
  \item \textbf{类比推理}:参考泡林在蛋白质中发现的α-螺旋结构
  \end{itemize}
\item \textbf{第三阶段:收集额外事实——多源证据的系统整合}:
  \begin{itemize}
  \item \textbf{分子组成}:糖-磷酸脊椎和四种碱基(A、G、C、T)的结构
  \item \textbf{查尔格夫规则}:A=T、G=C的数量相等关系的发现
  \item \textbf{模型构建}:使用硬纸和金属丝构造超大尺寸模型
  \item \textbf{竞争压力}:泡林等其他科学家的同步研究形成竞争
  \end{itemize}
\item \textbf{第四阶段:精练假说——互补碱基配对的突破}:
  \begin{itemize}
  \item \textbf{关键洞察}:A-T和G-C的互补配对而非同类配对
  \item \textbf{结构解释}:双螺旋中碱基指向内部,形成互补配对
  \item \textbf{功能解释}:解释了遗传信息存储和自我复制机制
  \item \textbf{创造性突破}:华生的"突然知道"时刻体现科学发现的顿悟性
  \end{itemize}
\item \textbf{第五、六阶段:演绎检验——理论验证的系统过程}:
  \begin{itemize}
  \item \textbf{模型验证}:构造三维双螺旋模型验证结构可行性
  \item \textbf{数据解释}:解释生殖细胞DNA数量只有普通细胞一半的现象
  \item \textbf{多重检验}:多个理论推论的成功检验增强理论可信度
  \item \textbf{证据积累}:假说正确性的证据迅速增加并得到充分证实
  \end{itemize}
\item \textbf{第七阶段:理论应用——分子生物学革命的开端}:
  \begin{itemize}
  \item \textbf{历史影响}:128页报告创造科学史,对生物学产生巨大持久改变
  \item \textbf{技术发展}:DNA切割重组技术、基因工程、人类基因组计划
  \item \textbf{医学应用}:新药、疫苗、人造荷尔蒙制造的革命性进展
  \item \textbf{持续影响}:重组DNA技术使生物学和医学发生革命,保持旺盛生命力
  \end{itemize}
\item \textbf{案例的方法论启示}:
  \begin{itemize}
  \item 科学发现遵循系统化的逻辑过程,但充满创造性和偶然性
  \item 理论与实验、合作与竞争、直觉与逻辑的有机结合
  \item 科学研究的社会性特征和知识累积性特征
  \item 重大发现对整个学科和社会发展的深远影响
  \end{itemize}
\end{itemize}
}}
\end{center}
\input{chapter13/13-6 判决性实验和特设性假说.tex}
\input{chapter13/13-7 作为假说的分类.tex}

% 第十四章
\chapter{归纳逻辑}
\input{chapter14/14-1 概率与归纳逻辑.tex}
\section{概率计算}

\begin{logicbox}[title=引言]
\textit{概率计算帮助我们确定复合事件的可能性,通过理解单元事件如何组合,我们能够处理日常生活和科学研究中的不确定性。}
\end{logicbox}

我们来确定一个\textbf{复合事件}的概率。复合事件可以被看做由多个事件构成的整体。例如,我们问:从一副牌中连续抽出两张黑桃的概率是多少?连续抽两张牌这样的复合事件是一个由两个部分组成的整体。这两个部分是,第一次抽出黑桃的事件,和第二次抽出黑桃的事件。再举一个例子,新娘和新郎活到庆祝金婚纪念日的复合事件,是由新娘再活 50 年的事件和新郎再活 50 年的事件,以及不发生离婚的事件组成的。当人们知道各个组成事件是如何相互关联的时候,人们能够根据单个事件的概率而求得该复合事件的概率。因而,我们把"概率计算"——用单元事件的概率计算出复合事件的概率——规定为纯数学的一个分支。

\subsection{概率计算的实用性}

概率计算在日常生活中是极其有用的。知道某个结果的可能性可以帮助我们进行决策,而使我们做事谨慎。因而,其基本定理的掌握和运用是逻辑研究最有用的结果之一。

概率计算最容易用\textbf{机会游戏}(games of chance)——掷骰子、玩扑克等等——的术语来解释。原因是,这些游戏所限定的人工世界使概率定理的直接使用成为可能。因此,尽管概率计算有广泛的应用范围,在这一章中,我们通过赌博中引申出来的问题,初步地阐明概率计算。在阐释过程中我们使用了概率的先验理论,当然,所有结果经过少量的重新解释后也能够用相对频率理论来表述和分析。

\subsection{复合事件的类型}

在接下来的小节中,我们将讨论两种主要的复合事件:
\begin{itemize}
\item \textbf{事件的共同发生}是指所有被考虑的单元事件均发生,例如,连续掷三次硬币得到三次正面的概率为多少?
\item \textbf{事件的替代性发生}是指至少有一个被考虑的单元事件发生,例如,掷两次骰子,至少得到一次6点的概率是多少?
\end{itemize}

我们要先讨论事件的共同发生的概率,然后讨论事件的替代性发生的概率。

\subsection{事件的关系}

为了计算复合事件的概率,我们需要知道单元事件之间的关系:它们是独立的还是非独立的。

\begin{itemize}
\item \textbf{独立事件}是指其中一个事件是否发生,不影响另一个事件发生的概率。例如,连续掷两次硬币,第一次是否出现正面,不影响第二次出现正面的概率。
\item \textbf{非独立事件}是指一个事件的发生会影响另一个事件发生的概率。例如,从一副牌中抽两张牌(不放回),第一次抽出黑桃的事件会影响第二次抽出黑桃的概率。
\end{itemize}

这种独立性或非独立性的区分对于正确计算复合事件的概率至关重要,我们将在后续章节中详细探讨。

\begin{center}
\fbox{\parbox{0.95\textwidth}{
\textbf{本节要点}
\begin{itemize}
\item \textbf{概率计算的基本概念}:
  \begin{itemize}
  \item 概率计算是用单元事件的概率计算复合事件概率的数学分支
  \item 复合事件由多个单元事件组成,可能是共同发生或替代性发生
  \item 概率值总在0到1之间,不可能事件为0,必然事件为1
  \end{itemize}
\item \textbf{事件的关系类型}:
  \begin{itemize}
  \item 独立事件:一个事件的发生不影响另一个事件的概率
  \item 非独立事件:一个事件的发生改变另一个事件的概率
  \item 互斥事件:两个事件不能同时发生
  \end{itemize}
\item \textbf{实际应用价值}:
  \begin{itemize}
  \item 帮助评估科学假说的可靠性
  \item 为日常决策提供理性基础
  \item 在不确定性中识别最优选择
  \end{itemize}
\end{itemize}
}}
\end{center} 
\section{共同发生的概率:乘法定理的数学基础与应用}

\begin{logicbox}[title=引言]
共同发生的概率是概率论中最重要的概念之一,它涉及多个事件同时发生的可能性计算。本节将深入探讨乘法定理的数学基础、逻辑原理和实际应用,分析事件独立性对概率计算的决定性影响。通过掌握独立事件和非独立事件的不同计算方法,我们将能够解决从简单的机会游戏到复杂的医疗决策、工程可靠性分析等各种现实问题。这种理解对于科学研究、风险评估和理性决策具有重要意义。
\end{logicbox}

\subsection{共同发生概率的数学定义与理论基础}

\begin{theorembox}[title=共同发生的数学定义]
\textbf{共同发生}(joint occurrences)是指某个复合事件的单元事件中的两个或两个以上事件的发生。

\textbf{数学表示}:
对于事件$A_1, A_2, \ldots, A_n$,它们的共同发生记为:
$$P(A_1 \cap A_2 \cap \cdots \cap A_n)$$
或简记为:
$$P(A_1 \text{ 且 } A_2 \text{ 且 } \cdots \text{ 且 } A_n)$$

\textbf{实际问题实例}:
我们希望知道从一副牌中连续抽出3张黑桃的概率,或者在一场赛马中喜爱的两匹马都使我输钱的概率,或者将一枚硬币扔十次得到十次正面向上的概率。

\textbf{基本情况分析}:
假定我们正考察的是只有两个单元事件$a$和$b$的发生。当我们要得到$a$并且$b$两者的概率时,我们便要求它们的共同发生。

\textbf{共同发生的特征}:
\begin{itemize}
\item \textbf{交集性质}:共同发生对应集合论中的交集运算
\item \textbf{概率单调性}:$P(A \cap B) \leq \min(P(A), P(B))$
\item \textbf{逻辑合取}:对应逻辑学中的合取(AND)运算
\item \textbf{条件依赖}:计算方法取决于事件间的独立性关系
\end{itemize}
\end{theorembox}

\subsection{独立与非独立事件}

一个困难立即出现了:两个事件中的一个出现或不出现对另外一个事件的出现或不出现产生影响吗?如果存在这样的影响,单元事件就不独立;如果不存在这样的影响,它们就是独立的。如果两个事件中的一个的发生或不发生对另外一个事件的发生或不发生,不产生任何影响,我们说两个事件是\textbf{独立的}。例如,如果我们掷两枚硬币,无论一枚硬币是正面朝上还是反面朝上,不会影响另外一枚硬币是正面朝上还是反面朝上;它们是独立的事件。

\subsection{独立事件的共同发生}

为了讨论事件共同发生的概率,我们先分析比较容易的情况:独立事件的共同发生。考虑这样一个简单问题:掷两枚硬币,两枚均正面朝上的概率是多少?掷两枚硬币有三个可能结果:或两个正面,或两个反面,或一正一反。但是它们不是等可能的。因为,有两种方式发生一正一反,而只有一种方式得到两个正面。第一枚硬币出现正面,第二枚硬币出现反面;或者第一枚硬币出现反面,第二枚硬币出现正面,它们是不同的情况。因而,当我们掷出两枚硬币时,可能出现 4 个不同的可能事件。将之列表如下:

\begin{center}
\begin{tabular}{|c|c|}
\hline
\textbf{第一枚硬币} & \textbf{第二枚硬币} \\
\hline
正 & 正 \\
正 & 反 \\
反 & 正 \\
反 & 反 \\
\hline
\end{tabular}
\end{center}

没有理由期望其中的任何一个情况比其他情况更可能发生,因而我们认为它们是等可能的。两枚正面朝上的特别情形只是 4 个等可能的事件之一,因此,掷出两枚硬币,得到两次正面的概率是 $1 / 4$ 。这个复合事件的概率可以通过两个独立的单元事件的概率而求得。该复合事件由第一次掷出正面和第二次掷出正面,这两个事件的共同发生所构成。第一次掷出正面的概率为 $1 / 2$ ,第二次掷出正面的概率也为 $1 / 2$ 。这两个事件是独立的,因而我们可以用概率计算的\textbf{乘法定理}来计算它们共同发生的概率。根据独立事件的乘法定理,两个独立事件共同发生的概率等于它们各自概率的乘积。这个一般公式可以写成:

$$
P(a \text { 且 } b)=P(a) \times P(b)
$$

这里,$a$、$b$ 为两个独立事件,$P(a)$ 和 $P(b)$ 为它们的概率,而 $P(a$ 且 $b)$为 $a$、$b$ 共同发生的概率。本例中,$a$ 为第一次出现正面的事件,$b$ 为第二次得到正面的事件,这样,$P(a)=1 / 2, P(b)=1 / 2$ ;因此,$P(a$ 且 $b)=$ $1 / 2 \times 1 / 2=1 / 4$ 。

\subsection{骰子问题的概率}

考虑第二个问题。我们摇两个骰子,得到 12 点的概率为多少?只有当每个骰子都为 6 点,两个骰子才出现 12 点。每个骰子有 6 面,摇后每一面向上与其他面向上的可能性相同。假定 $a$ 为第一个骰子出现 6 点的事件,$P(a)=1 / 6$ ;假定 $b$ 为第二个骰子出现 6 点的事件,$P(b)=1 / 6$ 。 $a$ 和 $b$ 的共同发生构成了两个骰子出现 12 点的复合事件。根据乘法定理, $\mathrm{P}(\mathrm{a}$ 且 b$)=1 / 6 \times 1 / 6=1 / 36$ 。 $1 / 36$ 即为要两个骰子得到 12 点的概率。我们也可以通过列举摇两个骰子时所有可能发生的事件,而求得同样的结果。有 36 个等可能事件,列表如下。在表中,每一对数字中的第一个数字代表第一个骰子向上的数字,第二个数字代表第二个骰子向上的数字。

\begin{center}
\begin{tabular}{|c|c|c|c|c|c|}
\hline
1-1 & 2-1 & 3-1 & 4-1 & 5-1 & 6-1 \\
\hline
1-2 & 2-2 & 3-2 & 4-2 & 5-2 & 6-2 \\
\hline
1-3 & 2-3 & 3-3 & 4-3 & 5-3 & 6-3 \\
\hline
1-4 & 2-4 & 3-4 & 4-4 & 5-4 & 6-4 \\
\hline
1-5 & 2-5 & 3-5 & 4-5 & 5-5 & 6-5 \\
\hline
1-6 & 2-6 & 3-6 & 4-6 & 5-6 & 6-6 \\
\hline
\end{tabular}
\end{center}

在 36 个等可能的情况中,只有 1 个为我们希望的(出现 12 点),因而,我们直接得到概率为 $1 / 36$ 。

\subsection{乘法定理的推广}

我们可以将乘法定理一般化,以便涵盖任意多个独立事件的共同发生。如果我们从一副牌中抽出一张牌,将之放回并抽第二次牌,再放回去并抽第三次牌,那么抽出三次黑桃的事件,为第一次抽出黑桃的事件、第二次抽出黑桃的事件和第三次抽出黑桃的事件共同发生所构成。这三个事件用 $a$、$b$、$c$ 来表示,它们共同发生的概率 $P(a$ 且 $b$ 且 $c)$ 等于三个事件各自概率的乘积:$P(a) \times P(b) \times P(c)$ 。这个概率容易计算出来。一副扑克有 52 张牌,其中 13 张为黑桃。因此,抽出一张黑桃的概率为 $13 / 52=1 / 4$ 。由于再次抽牌之前原先抽出的牌被放了回去,第二次抽牌的情况与第一次的一样,因而,$P(a)$、$P(b)$、$P(c)$ 均为 $1 / 4$ 。它们共同发生的概率为 $P(a$ 且 $b$ 且 $c)=1 / 4 \times 1 / 4 \times 1 / 4=1 / 64$ 。我们可以用通用乘法定理计算任意多个独立事件共同发生的概率。

\subsection{非独立事件的共同发生}

现在我们转向分析不独立的事件。将独立事件的概率简单相乘,如上面的例子中所做的,没有考虑单元事件之间的关系。如果那些事件是有关联的,我们需要将这种关系考虑进来,以便精确计算这样的事件的共同发生。我们经常能够这样做。将上述例子做些修改。假定我要求从一副洗好的扑克牌中连续抽三张黑桃的概率,但抽出的牌不放回去。如果每一次抽出的牌在下次抽牌之前不放回去,前面的抽牌结果确实对后面的抽牌结果产生影响。如果抽出的第一张牌是一张黑桃,那么第二次抽牌过程中总的牌数为 51 张牌,剩下的黑桃有 12 张。而如果第一次抽出的不是一张黑桃,那么,剩下的 51 张牌中有 13 张黑桃。假定 $a$ 是从一副牌中抽出一张黑桃并且不放回去的事件,$b$ 为从剩下的牌中抽取另外一张黑桃的事件,那么 $b$ 的概率,即 $P(b \mid a)$为 $12 / 51$ ,即 $4 / 17$ 。如果 $a$ 和 $b$ 都发生,第三次抽牌是在只有 11 张黑桃的 50 张牌中进行。如果 $c$ 是最后的事件,那么 $P(c \mid a$ 且 $b)$为 $11 / 50$ 。于是,从一副牌中抽取三张牌、抽完不放回去,根据乘法定理,三张均是黑桃的概率为 $13 / 52 \times 12 / 51 \times 11 / 50$ ,即 $11 / 850$ 。这个值小于抽三张牌、但每次抽牌后放回去的概率。这也是我们能够预知的,原因是将抽出的牌放回去增加下次抽到黑桃的概率。

\subsection{条件概率与通用乘法定理}

我们来看另外与不独立事件共同发生的概率有关的一个例子。假定有一个袋子,袋子里面有 2 个白球和 1 个黑球。如果我们连续摸两个球,并且第一次摸到的球在第二次摸球之前不放回去,两次摸到的均是白球的概率是多少?假定 $a$ 为第一次摸到白球的事件。有三个等可能性,每个可能性对应于其中一个球。由于两个球为白色的,其中两个可能性能得到白球。因而,第一次摸到白球的概率为 $2 / 3$ 。如果 $a$ 事件发生了,袋中只剩下了两个球,一白一黑。明显的,第二次摸到白球(我们用 $b$ 表示)的概率为 $1 / 2$ ,即 $p(b \mid a)=1 / 2$ 。据通用的乘法定理,摸到两次白球的概率为 $a$ 和 $a$ 条件下 $b$ 共同发生的概率,其值为它们各自发生的概率值的乘积, $2 / 3 \times 1 / 2=1 / 3$ 。通用公式为:

$$
P(a \text { 且 } b)=P(a) \times P(b \mid a)
$$

在这个简单的情况下,我们可以通过计算各个可能的情形而确定连续两次摸到两个白球的概率。我们用 $W_{1}$ 表示一个白球,$W_{2}$ 表示另外一个白球,$B$ 表示黑球,下表列举了所有可能的等可能情况:

\begin{center}
\begin{tabular}{|c|c|}
\hline
\textbf{第一次摸球} & \textbf{第二次摸球} \\
\hline
$W_{1}$ & $W_{2}$ \\
$W_{1}$ & $B$ \\
$W_{2}$ & $W_{1}$ \\
$W_{2}$ & $B$ \\
$B$ & $W_{1}$ \\
$B$ & $W_{2}$ \\
\hline
\end{tabular}
\end{center}

在这 6 个等可能的事件中,两种情形是我们需要的(第一和第三)。连续两次摸球、第一次摸到的球不放回去的概率,我们可以直接求得为 $1 / 3$ 。

\subsection{现实应用:医疗决策中的概率}

通用乘法定理可以用于对现实世界问题的后果估计,下面就是一个例子。一个加利福尼亚少女受慢性白血病的折磨。如果不治疗,它将因白血病而死去。只有找到匹配的骨髓捐赠者,她才能得救。当她的父母寻找这样的捐赠人的所有努力均失败之后,他们决定再生一个小孩,以希望能够成功进行骨髓移植。但她的父亲首先得将切断的输精管接通,这只有 $50\%$ 的成功率。如果成功了,她的母亲因当时有 45 岁,她怀孕的机会也只有 $0.73$ 。如果她确实受孕成功,婴儿骨髓与受病痛折磨的女儿匹配的机会也只有四分之一( $0.25$ )。并且即使匹配成功,白血病病人经过必需的化疗和骨髓移植后活下来的机会为 $0.70$ 。

可以看到的是,结果成功的概率很低,但不是低到毫无希望。输精管成功得到接通,母亲也确实怀孕了,至此,希望增加了。巧的是,婴儿拥有能够匹配的骨髓。1992年进行了艰巨的骨髓移植手术。手术获得巨大成功\cite{ayala1993}。这个美满结果其概率在她的父母做决策的时候有多大呢?

\section*{乘法定理}
为了计算两个或更多事件共同发生的概率:

\begin{center}
\begin{tabular}{|p{0.95\textwidth}|}
\hline
\textbf{A.如果这些事件(如 $a$\zhtext{、}$b$ )是独立的:} \\
它们共同发生的概率为其概率的简单乘积:
$P(a \text { 且 } b)=P(a) \times P(b)$ \\
\hline
\textbf{B.如果这些事件(如 $a$\zhtext{、}$b$\zhtext{、}$c$ 等)是不独立的:} \\
它们共同发生的概率为第一个事件的概率乘以第一个事件发生的条件下第二个事件的概率,乘以第一和第二个事件发生的条件下第三个事件的概率,等等。 \\
$P(a \text{ 且 } b \text{ 且 } c)=P(a) \times P(b \mid a) \times P(c \mid a \text{ 且 } b)$ \\
\hline
\end{tabular}
\end{center}

\begin{center}
\fbox{\parbox{0.95\textwidth}{
\textbf{本节要点}
\begin{itemize}
\item \textbf{共同发生概率的数学定义与理论基础}:
  \begin{itemize}
  \item \textbf{数学表示}:$P(A_1 \cap A_2 \cap \cdots \cap A_n)$,对应集合论的交集运算
  \item \textbf{逻辑性质}:对应逻辑学中的合取(AND)运算,体现多条件同时满足
  \item \textbf{概率特征}:$P(A \cap B) \leq \min(P(A), P(B))$,共同发生概率不超过单个事件概率
  \item \textbf{条件依赖性}:计算方法完全取决于事件间的独立性关系
  \end{itemize}
\item \textbf{独立事件的共同发生与乘法定理}:
  \begin{itemize}
  \item \textbf{独立性定义}:一个事件的发生不影响其他事件的概率
  \item \textbf{简单乘法定理}:$P(a \text{ 且 } b) = P(a) \times P(b)$
  \item \textbf{经典实例}:硬币投掷、骰子摇动等相互无关的随机实验
  \item \textbf{定理推广}:可扩展到任意多个独立事件的共同发生
  \end{itemize}
\item \textbf{非独立事件的共同发生与条件概率}:
  \begin{itemize}
  \item \textbf{非独立性特征}:一个事件的发生改变其他事件的概率分布
  \item \textbf{通用乘法定理}:$P(a \text{ 且 } b) = P(a) \times P(b|a)$
  \item \textbf{条件概率}:$P(b|a)$表示在事件$a$发生条件下事件$b$的概率
  \item \textbf{实际应用}:抽牌不放回、连续选取、医疗决策等复杂情境
  \end{itemize}
\item \textbf{乘法定理的数学结构与应用}:
  \begin{itemize}
  \item \textbf{独立事件公式}:$P(a \text{ 且 } b) = P(a) \times P(b)$
  \item \textbf{非独立事件公式}:$P(a \text{ 且 } b \text{ 且 } c) = P(a) \times P(b|a) \times P(c|a \text{ 且 } b)$
  \item \textbf{现实应用价值}:医疗决策、工程可靠性、风险评估等领域
  \item \textbf{计算验证}:可通过列举所有等可能情况进行验证
  \end{itemize}
\item \textbf{共同发生概率的认识论意义}:
  \begin{itemize}
  \item 为复杂系统的可靠性分析提供数学工具
  \item 体现了概率论在处理多重不确定性中的重要作用
  \item 连接了数学理论与现实决策,具有重要的实践指导价值
  \item 是科学推理和风险评估的重要理论基础
  \end{itemize}
\end{itemize}
}}
\end{center}

% The rest of the file, including exercises, is removed.
\input{chapter14/14-4 替代性发生的概率.tex}
\section{期望值:不确定性下的理性决策理论}

\begin{logicbox}[title=引言]
期望值是概率论中衡量随机事件平均结果的核心概念,它为不确定性环境下的理性决策提供了重要的数学基础。本节将深入探讨期望值的数学定义、计算方法和理论意义,分析其在赌博、投资、保险、职业选择等各个领域的应用。通过理解期望值最大化原则及其局限性,我们将能够在面临不确定性时做出更加理性和科学的决策,同时认识到风险偏好、心理因素等在实际决策中的重要作用。
\end{logicbox}

我们经常必须在几个可能的行为之间做出选择。当在这些行为的结果中包含有不确定性时,概率计算可以帮助我们做出最好的选择。我们如何才能在这些不确定的结果中做出选择呢?

一个被广泛接受的规则是:我们应该以这样一种方式行动,以使我们的\textbf{期望值}(expected value)最大。期望值是指,在一个赌博或商业冒险中,一个人平均期望获得的价值。在一个游戏的特定场合下,一个人赢或者输,他不可能刚好赢得其期望值。但是,如果他多次参加这个游戏,他可以期望获得所有赢的次数和所有输的次数的平均值,这个平均值等于他的期望值。许多人认为,当在不确定的选项之间进行选择时,一个理性的人将选择期望值最高的那个选项。

\subsection{期望值的基本概念}

我们可以通过一个简单的例子来说明期望值是什么。假定你持有 1000张已售出的彩票中的一张,头奖为 500 美元。这张彩票的期望值是多少?如果我们多次参加这个游戏,我们将在 1000 次中赢一次。这意味着,在 1000 次中,我们所获得的钱数为 500 美元;平均下来,每次获得的钱数为 $500 / 1000$ 美元,即 0.5 美元。因而,这张彩票的期望值为 50美分。

一般地,一个特定彩票的期望值等于任何奖金的概率乘以该奖金的价值。如果用符号 $E$ 代表期望值, $P$ 代表获得奖金的概率, $V$ 代表奖金的价值,我们可以将规则表示为:

$$
E=P \times V
$$

这可以推广到多个奖金的情况。假定我持有上述彩票中的一张,头奖为 500 美元,二等奖为 100 美元,三等奖为 20 美元。假定在 1000 张已售出的彩票中,一张彩票将获得头奖,一张彩票将获得二等奖,三张彩票将获得三等奖。现在我的彩票的期望值是什么?获得头奖的概率是 $1 / 1000$ ,头奖的价值是 500 美元,头奖的期望值为 $1 / 1000 \times 500$ 美元,即 0.50 美元。获得二等奖的概率是 $1 / 1000$ ,二等奖的价值为 100 美元,二等奖的期望值为 $1 / 1000 \times 100$ 美元,即 0.10 美元。获得三等奖的概率是 $3 / 1000$ ,三等奖的价值为 20 美元,三等奖的期望值为 $3 / 1000 \times 20$ 美元,即 0.06 美元。我这张彩票的总的期望值是这三个期望值之和: $0.50+0.10+0.06=0.66$ 美元。这比只有头奖时的期望值要高一些。因而,在这样的多奖项的彩票中,值得多花几分钱购买。

\subsection{公平赌博与期望值}

期望值的概念在赌博中非常重要。如果一个赌博的期望值为 0 ,它便是公平的赌博。这意味着,赌博者平均下来既不赢也不输。一个赌徒必须支付的费用应等于他获胜的概率乘以他获胜的价值。这就是为什么当在赌场掷骰子时,如果掷出 7 点或 11 点,庄家支付一赔一的赌注。如果他支付更多,他将会输钱;如果他支付更少,赌徒们将发现这个赌博不公平而拒绝参加。如果一场赌博的期望值为正,它对赌徒有利;如果期望值为负,则对庄家有利。

\subsection{期望值在决策中的应用}

如果人们普遍遵循最大化期望值的规则,那么在日常事务中运用概率将是非常有用的。例如,假定你面对在两个工作中选择一个。一个工作稳定,每年薪水 3 万美元。另外一个工作风险较大,每年薪水 5 万美元,但如果你所在的公司破产,你将失业,每年薪水只有 1.5 万美元(失业救济金)。根据你对该公司前景的评估,你估计公司成功的概率为 $80\%$ ,破产的概率为 $20\%$ 。你应该选择哪个工作?为了遵循最大化期望值的规则,我们应该计算每个选项的期望值,然后选择期望值高的那个。第一个工作的期望值很容易计算,因为它是确定的: $1.00 \times 30000=30000$ 美元。第二个工作的期望值是: $(0.80 \times 50000)+(0.20 \times 15000)=40000+3000=43000$ 美元。由于第二个工作的期望值更高,根据最大化期望值的规则,你应该选择风险较大的那个工作。

当然,在现实生活中,还有其他因素需要考虑,例如一个人对风险的承受能力,以及对稳定性的偏好。然而,期望值的计算提供了一个有用的工具,可以帮助我们在不确定的情况下做出更明智的决策。

\subsection{期望值与保险}

期望值这个概念也可以用来解释为什么保险是合理的。例如,假定一所价值 10 万美元的房子每年被烧毁的概率为 $1 / 500$ 。房主每年支付 250 美元的保险费。这是否是一个好的交易?我们来计算一下不买保险和买保险的期望值。

如果不买保险,期望损失是 $1 / 500 \times 100000 = 200$ 美元。这意味着,平均而言,房主每年会因为火灾损失 200 美元。

如果购买保险,房主每年固定支出 250 美元。如果发生火灾,保险公司将赔偿损失,所以房主的损失为 0。因此,购买保险的期望"损失"(支出)是 $1.00 \times 250 = 250$ 美元。

乍一看,不买保险的期望损失 (200美元) 低于购买保险的固定支出 (250美元)。那么为什么还要买保险呢?这里的关键在于"风险规避"。对于大多数人来说,一次性损失 10 万美元的灾难性后果,远比每年多支付 50 美元的成本更难以承受。保险通过将个体的巨大风险分摊给大量的投保人,从而降低了个体面临的风险。虽然从纯粹的期望值来看,保险公司会盈利 (因为保费总额大于预期的赔付总额),但对于个体投保人而言,购买保险是为了规避那种虽然概率低但一旦发生就无法承受的巨大损失。

\begin{center}
\fbox{\parbox{0.95\textwidth}{
\textbf{本节要点}
\begin{itemize}
\item \textbf{期望值的概念}:
  \begin{itemize}
  \item 期望值是在多次尝试中平均期望获得的价值
  \item 用概率乘以价值计算:$E = P \times V$
  \item 多项奖励的期望值是各项期望值的总和
  \end{itemize}
\item \textbf{期望值在赌博中的应用}:
  \begin{itemize}
  \item 公平赌博的期望值为0
  \item 期望值为正有利于赌徒,为负有利于庄家
  \item 期望值可用于评估赌博的公平性
  \end{itemize}
\item \textbf{期望值在决策中的作用}:
  \begin{itemize}
  \item 理性决策应选择期望值最高的选项
  \item 在不确定情况下帮助比较多个选择
  \item 提供风险评估的量化方法
  \end{itemize}
\item \textbf{期望值的局限性}:
  \begin{itemize}
  \item 没有考虑风险偏好和风险承受能力
  \item 保险等情况中,需考虑风险规避因素
  \item 实际决策中应结合其他因素一起考虑
  \end{itemize}
\end{itemize}
}}
\end{center}
\section*{第14章概要}
在所有归纳论证中,前提只是以某个概率度对结论进行支持,在科学假说中我们只是简单地把这个度描述成"更"可能或"不太"可能。本章

说明了如何能够将一个定量的概率(表示为 0 与 1 之间的小数)分派给归纳结论。

14. 1 节给出两种概率概念,它们都可以给予定量配置:(1)相对频率理论,根据这个理论,概率被定义成一个类的成员出现一个特定属性的相对频率。(2)先验理论,根据这个理论,一个事件发生的概率,由事件能够发生的途径数除以等可能的后果数来确定。

这两个理论均与 14.2 节介绍的概率计算相协调。如果复杂事件的各单元事件的概率能够确定,复杂事件的概率就能够计算出。在概率计算中使用两个基本的定理:乘法定理和加法定理。

如果复杂事件是一个共同发生的事件,两个或更多的单元事件均发生的概率可用乘法定理得到, 14.3 节给出说明。乘法定理断定,如果单元事件是独立的,它们共同发生的概率等于它们各自的概率的积。但如果单元事件是不独立的,可以运用通用乘法定理:( $a$ 且 $b$ )的概率等于 $a$ 的概率乘以在 $a$ 发生的条件下 $b$ 的概率。

如果复杂事件是替代性发生的(两个或更多事件中至少一个发生的概率),可应用加法定理,在 14.4 节得到说明。加法定理断定,如果单元事件是相互排斥的,它们的概率之和给出了替代性发生的概率。但如果单元事件不是相互排斥的,它们替代性发生的概率可以这样计算:(1)通过将所需要的场合分解成相互独立的事件,然后将他们的概率相加;或者 (2)确定至少替代性发生事件将不发生的概率,然后用 1 减去这个数。

为了计算一项投资或赌博的预期值(14.5节的内容),我们既要考虑可能后果的概率,又要考虑每个可能事件下获得的收益。先将每个后果预期回报与该回报发生的概率相乘,然后将这些乘积相加便得到投资的预期值。 
\section*{【注释】}
[1]病人爱丽莎•爱亚拉(Anissa Ayala)在手术成功的一年后结了婚,救了她的命的妹妹玛丽莎-爱亚拉(Marissa Ayala)在婚礼上为她撒花。该例子的具体细节见 1993 年第 12 月 Life 的报道。\\
[2]关于该问题的讨论参见:L.E.Rose,"Countering a Counter-Intuitive Proba- bility",Philosophy of Science 39 (1972):523-524;A.I.Dale,"On a Problem in Con- ditional Probability",Philosophy of Science 41 (1974):202-206;R.Faber,"Re-En-\\
countering a Counter-Intuitive Probability",Philosophy of Science 43 (1976):283- 285;S.Goldberg,"Copi's Conditional Probability Problem",Philosophy of Science 43 (1976):286-289。\\
[3]尽管下注于"每天 3 个数字"是不明智的,但它十分受欢迎,以至于现在一天开奖两次:中午和晚上。人们可能会认为,不是购买该奖券的那些人没有计算他们下注的期望值,就是这样的赌博给了他们满足,这个满足与他们下注的金钱期望值无关。\\
[4]事实上,持续出现一个结果(正面或反面)的情况包含在一个长的正面和反面(或者转轮中黑色和红色,等等)的随机序列之中,其频率比我们普遍认为的要高得多。出现一打正面不是十分稀奇的。如果赌博者从 $\$ 1$ 开始下注在反面,如果出现正面就持续加倍下注,在第 12 局它要求赌博者下 $\$ 2048$ 。第 12 局之后,第 13 局为反面的机会还是 $1 / 2$ ! 

% 后记部分
\backmatter
\chapter*{参考文献}
\printbibliography

\end{document}