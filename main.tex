% This LaTeX document needs to be compiled with XeLaTeX.
\documentclass[10pt]{article}
\usepackage[utf8]{inputenc}
\usepackage{amsmath}
\usepackage{amsfonts}
\usepackage{amssymb}
\usepackage[version=4]{mhchem}
\usepackage{extpfeil}
\usepackage{stmaryrd}
\usepackage{graphicx}
\usepackage[export]{adjustbox}
\graphicspath{ {./images/} }
\usepackage{multirow}
\usepackage{fvextra, csquotes}
\usepackage{xeCJK}
\usepackage{polyglossia}
\usepackage{fontspec}
\usepackage{enumitem}
% 使用思源宋体 (衬线字体)
\setCJKmainfont{Noto Serif CJK SC}

% 或者,使用思源黑体 (无衬线字体)
% \setCJKmainfont{Noto Sans CJK SC}

% 或者,使用文泉驿微米黑
% \setCJKmainfont{WenQuanYi Micro Hei}

\setmainlanguage{english}
\IfFontExistsTF{CMU Serif}
{\setmainfont{CMU Serif}}
{\IfFontExistsTF{DejaVu Sans}
  {\setmainfont{DejaVu Sans}}
  {\setmainfont{Georgia}}
}


%New command to display footnote whose markers will always be hidden
\let\svthefootnote\thefootnote
\newcommand\blfootnotetext[1]{%
  \let\thefootnote\relax\footnote{#1}%
  \addtocounter{footnote}{-1}%
  \let\thefootnote\svthefootnote%
}

%Overriding the \footnotetext command to hide the marker if its value is `0`
\let\svfootnotetext\footnotetext
\renewcommand\footnotetext[2][?]{%
  \if\relax#1\relax%
    \ifnum\value{footnote}=0\blfootnotetext{#2}\else\svfootnotetext{#2}\fi%
  \else%
    \if?#1\ifnum\value{footnote}=0\blfootnotetext{#2}\else\svfootnotetext{#2}\fi%
    \else\svfootnotetext[#1]{#2}\fi%
  \fi
}

\begin{document}
\input{序}
\input{1.逻辑学的基本概念}
\section*{2.1 语言的三种基本功能}
\section*{2.2 经功能话语}
\section*{2.3 话语形式}
\section*{2.4 情感词汇}
\section*{2.5 一致与歧见的种类}
\section*{2.6 情感中性语言}
第2章概要\\
\includegraphics[max width=\textwidth, center]{2025_05_15_6a28331d5e7c993ad07ag-096}

\section*{2.1 语言的三种基本功能}
语言是一种非常精细和复杂的工具,以至于我们可能会忽视它的用法 (uses)的多样性。我们会自然而然地去追求简单化,而不注意语言使用的语境及其不同使用目的,从而就可能被我们遇到的语词或者话语形式所误导。

语词并不总是服务于它们的表面诉求。在非正式谈话中,"你好吗?"这个问题并不是真正地在询问他人的健康状况。虽然这句话好像是在进行信息请求(a request for information),但是我们知道,它通常仅仅是一句友好的问候。那些通过描述其健康状况来回答该问题的人,就很可能被认为是愚義的。请求、报道和问候等仅是语言所具有的一些比较明显的功能。

哲学家乔治•贝克莱(George Berkeley)在他的《人类知识原理》 (Treatise Concerning the Principle of Human Knowledge)(1710)中说:\\
……思想交流……并非像通常想象的那样是语言的首要的和唯一的目的。使用语言还有许多其他宗旨,诸如引起某些情感、鼓动或者抑制行动、使人专心于某些特定的安排等等。前者(交流思想)在很多情况下都基本上是从属性的;当没有前者也能实现那些宗旨时,前者甚或完全被忽略,我认为,这种情况在人们熟悉的语言用法中经常发生。

20世纪的哲学家们非常详尽地阐明了多种多样的语言用法。路德维希•维特根斯坦(Ludwig Wittgenstein)在其《哲学研究》(Philosophi- cal Investigations)(1953)中正确地主张,"我们称之为‘符号’、‘语词’和‘语句’的东西有无数不同种类的用法。"在维特根斯坦所列举的例子中有:发出命令、描述物体的外表或者给出它的测量结果、报道事件、推测事件、提出和检验假说、结合图表提交实验的结果、编故事、演戏、唱歌、猜谜、开玩笑、解算术题、语言翻译、询问、思考、诅咒、问候和祈祷等等。

语言的用法多得令人吃惊,通过将它们分为三种非常笼统的种类,即

信息性用法(the informative)、表达性用法(the expressive)和指令性用法(the directive),可以使之具有一定的条理。诚然,这种三重划分是一种简化,甚或过于简化。但是,许多逻辑和语言学者发现这是一种非常有用的划分。

1.语言的第一种用法是用于信息交流。通常,它是通过明确表述并肯定(或者否定)命题来完成的。能被用于肯定或否定命题,或者能为此提出论证,称为语言的信息性功能。这里使用的"信息"一词也包括错误信息,即既包括真命题也包括假命题,既包括错误的论证也包括正确的论证。信息性话语用来描述世界和进行有关世界的推理。无论其所报道的事实重要与否、是普遍还是特殊,用来描述和报道的语言都是用来提供信息的。下面就是一个语言的信息性用法的简单例子,它出自最近佛罗里达高等法院的一篇报道:

\begin{displayquote}
2000年11月7日,星期二,佛罗里达州,和美国其他州一道,进行美国总统普选。11月8日,星期三,该选举分区(佛罗里达州)报道说,共和党候选人乔治•布什获得 2909135 张选票,民主党候选人小艾伯特•戈尔获得 2907351 张选票。因为投给他们的全部票数的总差(1784张),低于该选区全部投票票数的百分之一的一半,所以根据佛罗里达州法律规定进行了自动重新计票。 ${ }^{[1]}$
\end{displayquote}

2.正如信息性话语的最为清晰的例示来自于法院或者实验室的报道一样,语言用做表达性用法的最好的例子来自抒情诗。面对令人惊奇的古城帕特拉(Petra)遗迹,约翰•W•伯根(John W.Burgon)的诗句:

\begin{displayquote}
如此奇迹今我惊叹,它保留在东部的风情中——玫瑰一样红的城市——"几乎和时间一样永恒"!
\end{displayquote}

并不是意欲告诉我们关于世界的任何事实和理论,而是要表达诗人的赞赏和敬畏之情。这些诗句告诉了我们他眼前的一些真实景色,但是它们的主要目的并不是为了报道信息。诗句表达了作者感受到的强烈情感,其目的在于在读者心灵中激起类似的情感。无论何时,如果语言被用来发泄和激

发情感,那么它就具有表达性的功能。\\
"表达"这个词在这里的运用范围比通常要狭窄。很自然,我们可以说表达感情、情感或态度等。但是通常情况下,人们也可以说表达见解、信念或者信仰。为了避免混淆语言的信息性的和表达性的功能,我们将说陈述或表明见解或信念,而在本章中将"表达"这个词留用于揭示或交流情感、感受和态度。

并非所有的表达性语言都是诗歌。我们用"真糟糕"或"真遗憾"来表达悲伤,用"好极了"或"太妙了"来表达兴奋。强烈的情感可以通过恋人爱慕的喃喃私语来表达。崇信上帝者,通过诵读主祷文或大卫王的赞美诗第 23 篇,可以表达他对广袤无垠和神秘莫测的宇宙的敬畏和惊叹之情。语言的这种用法不在于交流信息,而在于表达情感、感受或态度。正因为表达性话语只是表达性的,故而其既不真也不假。若把真与假、正确与错误作为衡量抒情诗之类的表达性话语的标准,那就会文不对题,使其价值丧失殆尽。谁要是因为知道是巴尔波(Balboa)而不是柯尔特斯 (Cortés)发现了太平洋,而降低对济慈(Keats)的十四行诗《初读查普曼译荷马》(On First Looking into Chapman's Homer)的欣赏,那他就是一个蹩脚的读者。这首诗的宗旨并不是教给人们历史知识。当然,有些诗的确含有重要的信息性内容。在伟大的诗人那里,有些诗的确是很好的 "生活评判"。但是,这种诗就不仅仅是表达性的了。可以说,这种诗具有 "混合用途"(mixed usage)或曰包括多种功能。这个概念将在后面作进一步讨论。

表达可以分成两种情形。当一个人独自发泄,写诗却不视之以人,或者孤独地祷告时,他的语言就起到表达说话者或写作者情感的功能,但不是为了在别人心中引起共鸣。另一方面,当演讲者要寻求感染他人时,当恋人在求爱中使用诗歌语言时,以及当人群为运动队欢呼时,语言就不但被用来表示说话者的情感,而且还意欲在听者心中引发共鸣。总之,表达性话语或者用来表达说话者的情感,或者用来激发听讲者的情感共鸣。当然,它也可以同时具有这两方面的功用。

3.当语言意欲引起或阻止明显的行动时,它就具有了第三种功能,即指令性功能。其最显然的例子就是命令和请求。当父母告诉孩子洗手吃饭时,其意图不是为了交流任何信息或者表达或激发任何特殊情感。这种语言是为了获得其指令结果。当你去电影院对售票员说"请给两张"时,

语言也是被指令性地使用以产生行动。命令和请求之间的差别是微妙的,因为通过语调的适当转换,或仅仅是加上"请"这个词,几乎所有的命令都可以变为请求。通常,当提出问题以寻求回答时,该问题也被归类于指令性话语。

在单纯的祈使形式中,指令性话语既不真也不假。一个命令,比如 "关上窗户",既不能是真的也不能是假的。对一个命令是否应当遵从,我们可能会有不同意见;但是,对命令是否有真假,我们不会有分歧,因为真假这样的词语不能直接运用于命令句。不过,命令和请求还具有其他属性——合理或不合理、适当或不适当——这些属性与信息性话语的真或假有相类似之处。在第1章中我们看到,我们可以为履行某个行动给出理由,而如果陈述这些理由时伴随着命令,那么我们就可以把整个过程视为论证。例如:

\begin{displayquote}
小心驾驶!谨记墓地中满是守法的公民,他们有走路的权利。 ${ }^{[2]}$
\end{displayquote}

在把这种话语处理为论证时,我们可以把其中包含的命令视做命题。命令的接受者从中被告知他们应当或应该履行所命令的行动。有些学者研究了这类问题,已经发展出了"祈使逻辑"(logic of imperatives),但对它的讨论超出了本书的范围。 ${ }^{[3]}$

\section*{2.2 多功能话语}
前面所给出的信息性的、表达性的和指令性的话语的例子,打个比方说,都像纯正的化学标本。语言交流的这种三重划分是启发性的,有价值的,但是不能机械运用,因为几乎任何一种正常交流都可能会表现出语言的这三种用法。例如,一首诗可能主要是表达性话语,但也可能会有教育意义并因而也可以引导读者走向不同的生活方式。华兹华斯(Words- worth)写道:

我们身边的世界丰富精彩:迟早,无论是得到和失去,我们都要浪费掉自己的精力:

显然,诗歌也可以包含一定数量的信息。\\
再如,尽管布道主要是指令性的,希冀在会众中带来一定令人称赞的行动(抛弃罪恶,乐善好施),但是它也可以表达并激发情感,因而具有表达性功能,而且还可以包括一些信息,比如福音(the Gospels)的好消息。科学论文虽然本质上是信息性的,但是也可以表达作者的理性激情,而且还可以至少是含蓄地请读者去独立地证明作者的结论。语言的大多数平常的用法都是混合的。

语言的这种混合的或多功能的用法,并不是因为说话者或作者混淆了它们。相反,成功的交流都要求一定的功能结合。除清晰的语境和正式的关系——一父母与子女,雇主与雇员——之外,人们不能简单地发布命令并希望它得到执行。赤裸裸的命令会引起反感和敌对,并且经常是自生自灭。因此,必须使用一定的间接方式。通常,为了追求我们要引起的行动,我们并不直截了当地发布命令,而需要使用比较委婉的方法。

行动常常具有非常复杂的原因。与逻辑学家相比,心理学家更为适合研究动机,但是行动通常既涉及行动者的欲望又涉及他的信念,这是常识。除非饥饿的人相信他们面前的东西是食物,否则,他们就不会把它放进自己的嘴里;除非想吃,否则,即使人们毫不怀疑面前就是食物,他们也不会去碰它。

欲望是我们所谓的"态度"或"情感"的特殊类型,而信念通常会受到所接收到的信息的影响。因此,我们有时是通过激起他人的适当态度而成功地引起他们的行动,而有时则是通过提供信息以影响他们的相关信念而做到这一点。

假设你的目标是使你的听众向某个特定的慈善组织提供捐助,假定你的听众态度是助人为乐的,你就可以通过给他们提供该慈善机构的良好工作信息而促使他们行动。你的语言是指令性的,目的是引起行动,但你是通过提供信息,而不是通过发布一条毫无掩饰的命令或不客气的要求,而达到你的诉求的。再如,假定你的听众已经被深深地说服,相信我们谈论的那个慈善机构确实信誉良好,但对捐献的鲁莽要求仍然可能失败;但是,通过在一定程度上激发他们的乐善好施情感,你就可能会成功地使他们向该慈善机构提供捐助。在这种情况下,你就通过使用表达性的话语而

达到了自己的目的:你实现了一个"动人诉求"(moving appeal)。这样,你的语言自然而然地就具有了混合用法,既有表达性功能又有指令性功能。

最后,再假设你要向那些既缺乏乐善好施的态度,也缺乏对你推荐的慈善组织之信誉的认识的人进行捐献动员,那么你就必须一并使用语言的三种功能,既要有表达与信息功能,又旨在引发行动。这种一并使用并不是可偶尔为之的用法,而是必须为成功交流而精心准备的基本手段。

语言的三种基本用法是:信息性用法、表达性用法和指令性用法。但值得一提的是,语言在某些特殊语境中还具有一些特殊用法,而这些用法并不能完好地归属这三重划分。

语言的礼仪性用法是很普遍的,在有些场合下它是一种表达性的和指令性的话语的混合。社交中的问候、赴宴邀请、雇用告知等表述,都是体现礼仪功能的例子;语言还有很多其他相关的不确定用法,它们主要服务于使人们之间的互动变得融洽。礼仪功能还是宗教场合中一种庄重的语言用法。给人留下深刻印象的结婚仪式的语言,既要突出场合的庄重性(表达性的功能),又要使新郎和新娘提高对严肃的结婚誓言的正确理解而引起新的角色行为(指令性的功能)。

与礼仪性用法相近,语言还有其他一些用法,它们也不仅仅是那些主要用法的混合。假如有人邀请你在某一时间和地点去参加会议,你回答 "好,我答应你",那么你就不仅是以此表明了你的态度和预告了你的行为,你同时还用语言来许诺。相似的,在婚礼结束时,司仪或主持人说 "我宣布你们是夫妻",这也不仅是表明了说话者在干什么。在有些语境中,说出某些话实际上包含了一种重要行动。这些都是语言的践行性 (performative)用法的例子。

践行性话语实际上就是实施一种行动,即其所报告或描述的行动。践行性动词是一种特殊的种类,它们代表行动,这种行动通常是以第一人称使用动词而(在适当的情况下)完成的。这里可以再举出一些例子,如 "我祝贺你……"、"我向你道歉,我……"、"我建议……"、"我将这艘船命名为 $\cdots \cdots$"、"我接受你建议......"等等。

语词和语句的这些及其他的特殊用法显示了自然语言的丰富性,它们的诸多复杂功能难以归纳为任何一个单独的分类系统。

\section*{2.3 话语形式}
语句常常被定义为表达一个完整思想的语言单位。在语法教科书中,语句一般被分为四种类型,即陈述句、疑问句、祈使句和感叹句。但是,这四种语法分类与陈述、询问、命令和惊叹并不完全对应。我们可能会尝试把形式等同于功能,即认为陈述句和信息性话语是对应的,感叹句仅仅适用于表达性话语,或者我们会认为指令性话语完全包括祈使句或疑问语气的句子(把询问总是当做寻求回答的请求)。假如这种整齐的等同是可能的,那么交流问题就会大大地简单化了,因为这样我们就可以仅仅通过一段话的形式而说出它的原有功能,而其形式是很容易直接检查的。但是,那些把形式与功能等同的人会错误理解别人的话,而且或许会漏掉别人要传达的很多要点。

设想任何具有陈述句形式的事物都是信息性话语,真的就予以好评, 77 假的就予以拒斥,这显然是不正确的。"我在你的聚会上度过了一段美好的时光"是个陈述句,但它的功能完全不必是信息性的;倒不如说,它是礼仪性的或者表达性的,表达了友好和欣慰的感情。尽管很多诗歌和祷文的功能都不是信息性的,但它们却具有陈述句的形式。简单地认为它们是信息性的并简单地将它们评价为真的或假的,将会把自己排除在富有价值的审美和宗教体验之外。同样,许多请求和命令都是间接地——或许更加委婉地———用陈述句来表述的。"我喜欢咖啡"这个判断句不应当被侍者仅仅当做是顾客的表白,而应当被视为针对行动的指令或请求。对于陈述句,诸如"对这些帮助,我将非常感谢"或"我希望你课后能够在图书馆见到我",假如我们僵硬地判别它们的真与假,只是将它们视为信息报道,那么我们很快就会没朋友了。这些例子向我们表明,陈述的形式并不一定就标朋信息性功能。陈述句在每种话语类型的表达方式中都能见到。

其他语句形式也是如此。"你意识到我们几乎要迟到了吗?"这个疑问句,不必是在询问你的大脑状态的有关信息,而可能是要求抓紧时间。疑问句"1939年,俄国和德国签署了一个条约,它导致了第二次世界大战,不是吗?"可能完全不是询问,而可能是交流信息的间接方式或者是企图表达并激发一种对俄国的敌对情感;在第一种情况下起的是信息性功能,

而在第二种情况下起的是表达性功能。甚至语法上的祈使句,如在公文中以"兹请周知 $\cdots \cdots$"开头,可能并不是命令,在其所断言的东西中是信息性话语;在其激发神圣和庄重的适当情感的语言用法中是表达性话语。就感叹句来说,尽管它与表达性话语的功能关系紧密,但也可以有相当不同的功能。感叹句"天啊,要迟到了!"在语境中可以表示抓紧时间的请求。而房地产经纪人对潜在的顾客说出"多么美好的景色啊!"这个感叹句,其祈使性功能比表达性功能更浓重。

许多话语都企图同时达到语言的两种或者可能是其全部的三种功能。在这种情况下,给定语段的每一方面或功能都从属于它自己的适当标准。一个具有信息性功能的语段可以具有能够评价为真或假的方面。同样的语段,也可以具有指令性功能,能够以恰当或不恰当、对或错等进行评价。而假如这个语段还具有表达性功能,那么其组成成分还可以评价为真诚或虚假、是否宝贵等。正确地评价某一给定语段,需要把握语言的不同功能以及该语段本身的宗旨。

对逻辑学家来说,真与假,以及与之相关的论证正确与错误的概念,是最重要的。因此,作为学习逻辑的学生,我们必须能够将信息性功能与非信息性功能的话语区分开来。进而,我们还必须能够理顺给定语段所具有的信息性功能与其可能具有的所有其他功能之间的关系。为了做到"理顺",我们必须知道语言可以具有哪些功能,以及必须能够将它们区分开来。语段的语法结构常常能够标示其功能,但是,功能与语法形式之间并没有必然联系。功能与语段表面上断言的内容之间也不存在严格的关系。关于这一点,一位大语言学家在他对"意义"(meaning)的讨论中给出了例证说明:

\begin{displayquote}
一个顽皮的孩子,该上床睡觉时却说"我饿了",他母亲知道他的把戏,就以打发他去睡觉来回答他。这是移位语(dis- placed speech)(1)的一个例子。 ${ }^{[4]}$
\end{displayquote}

在这里,这个孩子的话是指令性的一一虽然没有成功地实现其所希望的改变。关于语段的功能,我们一般是指它想要达到的那种功能。但不幸的

\footnotetext{(1)日常会话中的"托词"是"移位语"的一种典型,本处即为托词之例。
}是,那并不是总能够轻而易举地判定的。\\
当孤立地引用一个语段时,要判定该语段最初欲要达到的功能就特别困难。其原因在于,语境在判定功能的过程中极其重要。某些本身是祈使性或者普通信息性的语句,如果在实际语境中将它们安排到一个具有诗化效果的整体之中,就可以变为表达性语句。例如,孤立的:

到窗口来吧,

是一个起指令性功能的祈使句。而:

\section*{今夜海上风平浪静}
是陈述句,起着信息性功能。它们好像都没有较大的表达性威力,但是在马修•阿诺德(Matthew Arnod)的诗歌《多弗海滨》(Dover Beach)的语境中,二者都主要用为诗的表达性功能,而且效果显著。很多诗歌完全依赖于语段的表达性用法,而在其他语境中,这些语段却具有根本不同的功能。

下面的区分也是重要的,即句子表示的命题与关于说话者的事实(句子的说出就是证据)之间的区分。当有人说"天在下雨"时,这个被断言的命题是关于天气的,而不是关于说话者的。但做出断言却构成了说话者相信天在下雨的证据,这就是关于说话者的事实。也可能发生下述情况:人们做出的陈述只是表面上关乎他们的信念,实际上并不是为了给出关于他们自己的信息,而完全是一种述说其他事情的方式。如说"我认为黄金不应该用做货币的标准",通常并不能理解为关于说话者信念的心理自我表白,而只能理解为断言不应该使用黄金作为货币标准的一种方式。而当说话者发出命令时,由此推断说话者希望有人完成某事却是合理的;的确,在有些环境中,只要断言某人有某种特殊渴望,实际上就是给出了一个命令或者做出了一个请求。快乐地欢呼证明说话者非常愉快,即使他在整个过程中没有做出任何断言。但是,作为心理报道,断言说话者快乐就是肯定一个命题,这与只是快乐地欢呼相当不同。

在1.6节中我们注意到,论证与说明之间的差别常常取决于语段的说话者或作者的目的。现在,对语言的不同功能的探讨允许我们可以对该问

题进行更深人的考察。\\
当说话者论及某个有争议问题时,如果他说"我强烈反对什么什么",那么我们就理解,这样一句话的目的通常不是为了报道说话者的观点(除非这样的话是一位公职候选人或者其观点代表了公众利益的知名人士所讲的)。实际上,这种自我报道的表达形式是述说什么什么是个坏主意并且我们都应当反对它的常用方式。当说话者不断地证明我们所持的观点是正确的时候,他并不是在说明他的判断,而是有意的论证,以说服别人相信他的判断是正确的。对于有些争论性问题,通过陈述自己的观点而展开论证,并不是有意欺骗;在这种情况下,即使把判断和自我报道混合在一块儿,也不是有意欺骗。

在一个单独语段中,一种以上重要功能的组合可能会成为问题。思想表达,受我们宪法第一修正案的保护,可能会包含极具冒犯性的语言;在这种情况下,认识到冒犯言语中信息性与情感性功能的融合(integra- tion),对保护言论自由可能是极其关键的。在洛杉矶地区法院,一个年轻人身穿故意装饰有亵渎之物的夹克来抗议越战时期的军事法案;根据加利福尼亚刑法典,他以冒犯行为而被判处有罪。高等法院推翻了对他的指控,并雄辩地精确表述了这个问题:

我们不能忽视这个事实,这里牵涉的事件极好地表明,许多语言表达都具有双重交流功能:加上附带说明,它们不仅可以传达相当精确的思想,还可以传达其他不可表达的情感。实际上,与其情感力量一样,语词也常常因其认知力量而被选用。我们不能赞同这样的观点:宪法关注个人言辞的认知内容,而没有或不关心其情感功能。的确,情感功能可能常常是在寻求交流的信息的更重要因素……同样,我们也不能纵容这种轻率假定:可以禁止特殊言辞,但在这个过程中又不遭遇压制思想的真正危险。 ${ }^{[5]}$

区别语言的信息性以及论证性功能与语言的其他功能,并没有一个机械的方法。在后面几章中,我们发展的逻辑技术可以相当机械地运用于检验论证的有效性,但是没有一个机械技术可以识别论证的出现。识别在一给定语境中的话语的不同功能,要对语言的灵活性和其用法的复杂性熟虑

而敏感。\\
80

\begin{center}
\begin{tabular}{|l|l|}
\hline
语言的主要用法 & 语言的语法形式 \\
\hline
 & 陈述可 \\
\hline
信息性用法 &  \\
\hline
 & 疑问句 \\
\hline
表达性用法 &  \\
\hline
 & 祈使句 \\
\hline
指令性用法 &  \\
\hline
 & 感叹句 \\
\hline
 & 语法形式常常是功能的一个标志,但语法形式与其使用目的之间并没 \\
\hline
 & 定的联系。用做三种主要功能(左侧纵列)的任何一个功能的语言, \\
\hline
能会采用四种语法形式(右侧纵列)中的任何一种。 &  \\
\hline
\end{tabular}
\end{center}

\section*{练习题}
I.下列每个语段例示了哪种语言功能?\\
*1.请在第 6 a 行方框内打叉,除非你父母(或其他人)声明确认你是非独立的所得税申报人。\\
-U.S.Internal Revenue Service,"Instruc- tions",Form 1040, 1999\\
2.Twas brillig,and the slithy toves\\
Did gyre and gimble in the wabe;\\
All mimsy were the borogoves,\\
And the mome raths outgrabe.\\
(风怒兮阴露满空,\\
滚滚兮布于四方,\\
雾霚笼罩兮翻腾,\\
怒号兮直达上苍。)\\
-Lewis Carroll,Through the Looking

3.站在迦太基(Carthage)、帕密拉(Palmyra)、波斯波利斯(Per- sepolis)或者罗马(Rome)的遗迹上,想到王国和人的易逝,想到强壮和富有的生活现已远逝,哪个游者不被刺激得墥然神伤……\\
-G.W.F.Hegel,Lectures on the Philoso- phy of History, 1823\\
4.在这五颗外围的行星中,木星、土星、天王星和海王星都比地球大得多;但最外围的冥王星是所有行星中最小的,甚至比水星还要小。\\
*5.She was a child and I was a child,\\
In this kingdom by the sea,\\
But we loved with a love that was more than love-\\
I and my Annabel Lee-\\
(她还是孩子,我也是,\\
在大海边的王国里,\\
可我们相爱,爱超越了爱——\\
我和我的安娜贝-李—)\\
--Edgar Allan Poe,"Annabel Lee"\\
6.抛弃传教士们的缺点,他们既不教诲爱心也不教海兄弟情谊,却主要是从资本方面教导利私德行,而这些资本却是从你们的土地和劳动中偷偷捿取的。醒来吧,非洲!穿上漂亮的泛非社会主义(Pan-African So- cialism)长袍!\\
—W.E.B.Dubois,"Pan-Africa", 1958\\
7.如果我仅以人的甚或天使的语调说话,但却没有爱心,那么我无异于一面嘈杂的铜锣或者一只叮当响的铁钹。

\section*{——I Corinthians 13: 1}
8.我因此向你们通报,我即日通过这个文件辞去由选举产生的共和国总统职务。\\
-President Fernando Collor De Mello,in a letter to the Senate of Brazil, 29 Decem- ber 1992\\
9.美国生活是一种强大的溶剂。它好像可以中和任何智力因素,无论这些因素多么坚韧和格格不人,它都将其溶解在本土的良好愿望、自

大、自利和乐观之中。\\
-George Santayana,Character and Opin- ion in the United States, 1934\\
${ }^{\cdot} 10$ .美国陆地的最北端、最西端以及最东端都在阿拉斯加州。\\
II.下列每个语段最可能被打算用做语言的什么功能?\\
-1.我们这里没有社会等级。我们的宪法是色盲,既不承认也不能容忍公民的等级划分。就公民权而言,法律面前人人平等。最卑微的人与最有权力的人都是同等的。\\
--Justice John Harlan,dissenting in Plessy v.Ferguson, 163 U.S.537, 1896\\
2.法官们不知道怎样恢复罪犯的社会名誉——因为根本就没人知道。\\
-Andrew Von Hirsch,Doing Justice- The Choice of Punishment, 1976\\
3.耕作开始时,其他技艺就随之出现。因此农民是人类文明的奠基者。\\
—Daniel Webster,"On Agriculture", 1840\\
4.好人不作为,罪恶就得逞。\\
--Edmund Burke,letter to William Smith, 1795\\
*他们当中没有律师,因为他们认为律师的职业就是颠倒黑白。\\
--Sir Thomas More,Utopia, 1516\\
6.快乐是一种实际而且合理的目标。但是,如果有人说它是人们唯一感兴趣的东西,他就将招致一个古老而合理的回答:大多数快乐人们都不可能实际地得到,除非其另有所图。如果人们在猎狐中获得快乐,那仅是因为他们当时能够忘记狩猎的快乐。\\
-Brand Blanshard,The Nature of Thought, 1939

7.在许多工业部门中构成多数的技术差的工人都坚决主张,他们应该与技术好的工人拿同样的工资。\\
——John Stuart Mill,On Liberty, 1859\\
8.战争是折磨人类的最大瘟疫;它毁灭宗教,毁灭国家,毁灭家庭。它是苦难中的苦难。

9.人类历史越来越成为教育与灾难之间的竞赛。\\
--H.G.Wells,The Outline of History, 1920\\
*10.谁坚持在做出决定之前看得完全清楚,谁就永远也做不出决定。\\
——Henri-Frederic Amiel,Amiel's Journal, 1885

11.(君主的)另一耻祘是因为不整军经武,从而遭人藐视。\\
-_Niccolò Machiavelli,The Prince, 1515\\
12.永久的和平只是个梦,甚至连好梦都不是。战争是上帝的世界秩序的一部分,它演化出人的最高贵的美德:勇敢和克制,以及尽职尽责和自我牺牲。没有战争,世界就将沉沦到物质主义之中去。\\
——Helmuth Von Moltke, 1892\\
13.语言!灵魂的血液。阁下,我们的思想流淌其内,成长其上。\\
-Oliver Wendell Holmes,The Autocrat of the Break fast-Table, 1858\\
14.在过去 133 年中,超过 7500 名科学家,包括社会科学家,被选人美国国家科学院;其中只有 3 位黑人。\\
-The Journal of Blacks in Higher Educa- tion,Summer 1996\\
*15.初涉哲学,人们的思想倾向于无神论;但深人哲学,人们的思想会倾向于宗教。\\
-Francis Bacon,Essays\\
16.不将爱国主义剔除出人类,世界将永无宁日。\\
-George Bernard Shaw,O'Flaherty,V.C\\
17.假如(他)真的认为美德与邪恶之间没有区别,那么,阁下,他离开我们的屋子时,让我们数一数我们的汤匙好了。\\
——Samuel Johnson, 1763\\
18.在(给动物)配对之前,人们极其认真地审视马、牛和狗的性情 83和血统;但是,对待自己的婚姻,人们却极少或者从不这样谨慎。\\
-Charles Darwin,The Descent of Man, 1871

19.鲸鱼吞掉约拿的事,即使鲸鱼足够大,也令人非常惊异;但假如

是约拿吞掉了鲸鱼,那简直就是奇迹了。\\
-Thomas Paine,The Age of Reason, 1796\\
*20.种族概念是个多头怪物,它在我们的大多美梦还没做之前就窒息了它们,它使我们不去关注正常的人际互动所面临的挑战,在狂妄无度的追求中,走向怀疑和仇恨的龌龊。\\
-C.Eric Lincoln,Coming Through the Fire,Duke University Press, 1996\\
21.白人社会已经深深地影响着少数民族聚居区。这种聚居区由白人建造,由白人维护,而且也要由白人认可。\\
-The National Commission on Civil Disor- ders(Kerner Commission), 1968

22.伊丽莎白,你面临着痛苦的抉择。从今天起,你将必定是你父母之一的陌路人。如果你不嫁给柯林斯先生,你母亲将再也不见你;但是假如你嫁给他,我将再也不见你。\\
--Jane Austen,Pride and Prejudice, 1892\\
23.关于匹克威克这个人,我不想说什么;这话题提了出来但没有吸引力;先生们,我不是他,你们也不是;先生们,这样的人,在令人讨厌的冷酷和蓄意的邪恶沉思中获得乐趣。\\
——Charles Dickens,Pickwick Papers, 1870\\
24.你赞扬那些人,他们设宴向我们的公民提供其所欲望的美味佳肴。人们说是他们让城邦变得非同一般,但没有意识到,城邦的腐败和溃烂状况,正应归咎于这些过去的政治家;因为他们使城邦塞满了港口、船坞、城墙、税收以及类似的垃圾,但没有给正义和节制留下任何空间。\\
--Plato,Gorgias\\
*25.在(给我的)很多公开信中,最鼓舞人心和使人坚强的东西是,它们清楚地表示了坚决反对暴政、中伤和谋杀所必需的那种意志:意志必胜。\\
-Salman Rushdie,The Rushdie Letters\\
III.分析下列语段打算断定的是什么命题,或打算引起什么行动,或读者可能会把它们视做提供了什么证据。\\
*1.即使提名我,我也不接受;即使选举我,我也不供职。\\
-William Tecumseh Sherman,message to the Republican National Convention, 1884

2.政府竟英明地认为冰是一种"食物制品",这就意味着南极洲是世界上最主要的食物生产基地之一了。\\
--George P.Will\\
3.批评意见就如同魔棒:要用榛木棒去引出宝藏,不要用桦木棒招致对冒犯者的反击。\\
----Arthur Symons,An Introduction to the Study of Browning, 1886

4.没有音乐,地球就像一座没有住户的、空荡的和还未竣工的房子。因此,希腊和《圣经》的最早历史,以及各国的历史,都始于音乐。\\
——Ludwig Tieck,quoted in Paul Henry Lang,Music in Western Civilization, 1941\\
*5.研究是一种基本的心灵状态,包括对原则和公理的连续不断的反复检验,而当前的思想和行动都以原则和公理为基础。因此,它对现行实践是批判性的。\\
-Theobald Smith,American Journal of Medical Science,vol.178, 1929

6.我不懈地努力,使自己不去嘲笑人们的行为,不为他们感到悲哀,也不去嫌恶他们,而去理解他们。\\
-Baruch Spinoza,Tractatus Theologico- politicus, 1670\\
7.对于那些连面包都没有的人,政治自由有什么用?它仅仅对雄心勃勃的理论家们和政治家们有价值。\\
——Jean-Paul Marat\\
8.只要有低级阶层,我就是其中的一员;只要有犯罪分子,我就是其中的一位;只要还有一个人在蹲监狱,我就是不自由的。\\
——Eugene Debs\\
9.假如有诸神的王国,那么它应当是民主政府;但是,如此美妙的政府不适于人类。\\
— Jean-Jacques Rousseau,The Social Con-\\
tract\\
*10.公民有三个阶层。第一阶层是富人,他们懒情但又总是贪求较

多。第二阶层是穷人,他们一无所有,充满妒忌,痛恨富人,而且易为薥惑人心的政客所利用。在这两种极端之间存在的那些人,担负着国家安全和支撑法律的重任。\\
——Euripides,The Suppliant Women\\
11.我相信,这个罪恶时代的所有罪恶趋向和骚乱,都不是属于下级阶层而是属于中间阶层,而我们愚蠢得非常习惯于夸耀的阶层就是中间阶层。\\
-Lord Robert Cecil,Diary in Australia\\
12.上帝将会注意到,战争应当周而复始,就像对病人下猛药一样。\\
——Heinrich Von Treitschke,Politik, 1916\\
13.我宁愿人民对我为何没做总统而感到惊奇,而不是对我为何做了总统而感到惊奇。\\
-Salmon P.Chase\\
14.他(本杰明 • 迪斯雷利)是一个靠自己努力而成功的人,他热爱他的创造。\\
-John Bright\\
*15.我们听到过宪法权利、自由言论和自由出版。每次听到这些词语我就对自己说:"那人是个红色分子,那人是个共产主义者。"你从来没有听到过一个真正的美国人以那种方式讲话。\\
-Frank Hague,speech before the Jersey City Chamber of Commerce, 12 January 1938\\
16.即使是笨伯,当他保持沉默时,也会被认为是聪明人:\\
闭口不言者就是智者。\\
—Proverbs 17: 28\\
17.一句恰当的话\\
如同在银饰中的金苹果一样。\\
——Proverbs 25: 11\\
18.我站在上帝的祭坛前发誓,永远反对钳制人们思想的任何形式。\\
——Thomas Jefferson, 1800\\
19.自由人对死亡思考得最少,他的智慧不是对死亡而是对生活的沉思。\\
*20.之前,我就已经看到和听到过很多伦敦佬(Cockney)的厚颜无耻的言行,但我从未想到一个花花公子当众胡乱画画就要 200 基尼 (guinea)。\\
--John Ruskin,on Whistler's painting "Nocturne in Black and Gold"\\
21.当人们的命运从表面上看还算可以时,他们并没有在生活中发现足够的乐趣并使之对他们有价值,其中的原因通常是除了自己他们谁都不关心。

\section*{——John Stuart Mill,Utilitarianism}
22.年轻人不适合做政治科学演说的听众,因为他对生活中的行动没有经验,但政治科学的讨论却是从这些行动开始,并且与这些行动紧密相关的;更进一步讲,由于他容易为自己的情感所左右,所以他的学习就将是徒劳无功的,因为学习的最终目的不是知识而是行动。

\section*{-Aristotle,Nichomachean Ethics}
23.当人们自由地讨论问题时,绝不可能恰当地解决它。\\
-Thomas Babington Macaulay,"Southey's Colloquies on Society", 1830\\
24.在永恒斗争中,人类得到成长;在永久和平中,就将只有消亡。

\section*{-Adolph Hitler,Mein Kampf, 1925}
*25.但是,在他们所讲的许多谎言中,有一个令我非常惊奇——我指的是他们对你们说应当防范我,不要被我的雄辩力量所欺骗。言下之意,就是说我只是一个摇唇弄舌的娴熟演说家。在我看来,讲这些话的人真是极端厚颜无耻——除非他们所谓的雄辩力量意思是真话的力量,如果他们的意思讲的是这个,那么我就承认我是雄辩家。但是,这与他们的想法根本不同!

\section*{-Plato,Apology}
\section*{2.4 情感词汇}
现在,我们从讨论语句和较复杂的语段转向探讨构成它们的词汇。正如 2.2 节所见,一个单独的语句,可以同时具有信息性的和表达性的用法。要具有前者,句子必须明确表述一个命题,而要做到这一点,它的词

汇必须具有字面的或描述性的意义,以指示客体或事件以及它们的性质或关系。而当句子表达态度或感情时,其词汇就会具有情感的暗示或影响。一个语词或短语可以既具有字面意义又具有情感影响。后者通常被称为词汇的情感意义。

词汇的字面意义和情感意义在很大程度上是各自独立的。例如,"官僚"(bureaucrat)、"政府官员"(government official)与"公仆"(public servant)的字面意义几乎一样,但它们的情感意义却很有区别。"官僚"倾向于表达厌恶和反对,而作为敬语的"公仆"则倾向于表达尊重和赞赏。"政府官员"则更接近中性。

显然,我们用以指示事物的词汇会明显地影响我们对事物的态度。花的实际芳香不会因其名称而改变。正如莎士比亚所写的那样,一朵玫瑰不论使用其他任何名字,闻起来都是香甜的。然而,假如有人告诉我们有一种称做"臭菘"(skunkweed)的玫瑰,我们对它的反应就可能会受到影响。在华尔街(Wall Street),谨慎地选择语言可以促使人们在股票市场上采取行动。某几天是"回升",它意味着价格上涨;另几天是"取短期利息",这意味着价格在下降,因为很多人都在抛售股票,但这个词语仍然好听。如今大公司极少再进行"破产",但它们可以"重组",这听起来要好多了。

这种态度上的影响可以说明委婉语增多的现象,委婉语就是用温和的词汇表示严酷的现实。在战争中,己方军队的失败可能被称做容易接受的 "暂时撤退",而重大的撤退可能被报道为"兵力的有序集结"。在越南战争中,正在竞选总统提名的参议员尤金-麦卡锡(Eugene McCarthy)曾对美国的军事干预政策及公众不愿坦诚地面对它的状况进行了颇具讽刺性的批评:"我们不再宣称战争",他说,"我们宣称国家防御。"[6]

我们在不断地创造新词汇以替代那些不再令人满意的旧词汇。"殡葬人员"(undertake)变成了"殡仪员"(mortician),"看门者"(janitors)变为"守护员"(maintenance men),"老人"(old people)变成了"年长公民"(senior citizens)。但是,与旧的实际情况相联系的新替换的词汇最终也会失去它们的吸引力;"守护员"结果被"保卫者"(custodian)所代替,"殡仪员"为"殡仪主管"(funeral director)所取代,等等。杰梅恩•格瑞尔(Germaine Greer)写道:

\begin{displayquote}
经过与其指代的实际相联系,委婉语便迅速地失去了它们的功能,这是它们的宿命。因此,它们必须经常被它们自己的委婉语所取代。 ${ }^{[7]}$
\end{displayquote}

据说,杜鲁门总统的夫人贝丝(Bess)的朋友请求她阻止杜鲁门再说"大粪"(manure),她回答说,她花费了四十年时间才使他开始说"大粪"。

语言的确有它自己的生命,独立于它用以描绘的事实。有些包含生殖和排泄的生理活动可以用医学词汇不带情感地进行描述,而不至于引起神经质的不快;但是使用下流粗俗的词汇描述同样的活动却可以震惊除最麻木不仁者之外的所有听众。用我们的术语来说,可以说这两种词汇具有相同的字面或描述意义,但是在它们的情感意义上却有或缓和或激烈的对立。

在某个具体的人的思想中,词或短语有时可以产生情感意义,那不是产生于其字面上指代的东西,而是产生于第一次学习或遇到它的语境。一个作者曾描述说:

\begin{displayquote}
这是一个有启发意义的故事。一个小姑娘近来学会了阅读,正在拼读报纸上的一篇政治性文章。"爸爸,"她问,"什么是坦曼尼协会(Tammany Hall)?"她爸爸用通常社交禁语的口吻回答说:"亲爱的,你长大了就会明白的。"按照这种奇怪的成年人遁词,她终止了询问;但在她父亲的语气中,有某些东西使她相信坦曼尼协会必定与不正当的性事有关,以至于多年来她只要一听到这个政治机构 ${ }^{(1)}$ 就会体验到一种神秘的非政治性震颤。 ${ }^{[8]}$
\end{displayquote}

对于很多人来说,一定的词汇或短语,由于与我们的生活具有某些特殊联系,可以携带某种我们或许不愿公开承认的隐私情感。

字面意义和情感意义之间的对比,以及它们不同的可操作性用法,促使哲学家伯特兰•罗素设计了一种寓教于乐的游戏。他这样"调配"出一

\footnotetext{(1)坦曼尼协会最初是旨在通过捐赠与赞助进行控制的纽约市民主党执行委员会,成立于 1789 年。1805年转型为有明确政治宗旨的"慈着机构"。
}类"不规则动词"(irregular verb):

我坚定;你倔强;他头脑呆板。

伦敦的《新政治家和国家》(New Statesman and Nation)杂志随后举行了一次竞赛,以征求这样的不规则词汇表,获胜者如下:

我义愤;你生气;\\
他破口大骂。

我重新考虑过了;你改变了想法;\\
他食言了。

这种游戏证实了普通经验的教益:相同事物可以被情感色彩非常不同的词汇所指称。

\section*{练习题}
造出五组创新的"不规则动词",每组词字面意义相同,在第一人称中给出一个赞美的描述,在第二人称中给出一个相对中性的,而在第三人称中是贬义的。

\section*{2.5 一致与歧见的种类}
任何事物或行动都可以选择不同的短语来描述:传达赞许或反对意见,或者中立意见;不同类型的一致(agreement)和歧见(disagree- ment)可以就任何事情展开交流。

两个人可能会在某事情是否已经实际地发生上意见相左,这种情况可称为"信念歧见"。另一方面,他们也可能都同意已经事实上发生了一件事,因而是信念一致的;但对那件事,他们仍可以具有不同的或者甚至相反的态度。你可以用语言描述那件事来表达赞许,别人却可以用语言来表达反对。这里也存在歧见,但不是信念歧见。这是对这件事的感受不同,

是态度歧见。 ${ }^{[9]}$\\
明确了这两种歧见,我们就可以区分出两个人(让我们把他们称做 A和 B)之间的四种关系,来讨论某些给定事件或者其他事实情况。

第一,他们可能达到充分一致,即他们对于事件发生的信念和对事件的态度都是一致的。

第二,他们可能对于事件的信念是一致的,但在态度上却对立,一个人认为是好事,而另一个人却认为是坏事。设想我们讨论的事件是:对于某个有争议问题,一位政治候选人的立场改变。对于的确发生了这个改变,A 和 B 可能意见一致;但是,A 认为好极了,而 B 则发现它令人担忧。如果像 A 所理解的那样,这位候选人就会被称赞为"倾听了理性的呼声";如果像 B 所理解的那样,这位候选人就将被谴责为"机会主义的反复无常"。

第三,他们可能态度一致,但对于引起态度的事实,他们却可能有信念歧见。因此,A 和 B 都可能热情地赞扬所论及的那位候选人,而他们对该候选人的实际立场却有不同理解。A 可能认为,由于"倾听了理性的呼声",那位候选人确实改变了他的立场;而 B 则可能认为,由于"坚定不移地拒斥了为奉承所左右",他根本没有改变自己的立场。乍一看,这个第三种可能性好像不合理,但经过思考之后就会被认为是平常的;我们知道,在政治选举中,同一候选人常常可能有不同的支持原因,这些原因不但不同而且有时还不相容。

第四,这两个人可能会处于一种完全对立的状态,他们不但在事实上有歧见而且对事实的态度也对立。由于相信那位候选人改变了立场,A 可能会非常热烈地称赞这种改变是"明智的重新考虑"的结果;由于认为候选人的立场保持未变, B 可能会激烈地贬斥他"顽固地拒绝承认错误"。

当我们以解决歧见为目标时,我们就必须既要关心给定情况下的事实,又要关心争论者对这些事实的不同态度。不同类型的歧见需要不同的解决方法。因此,如果我们不清楚所存在的歧见是什么类型,那么我们就不清楚去使用什么方法。信念歧见可以通过确认事实而得到最好的解决。为了明确这些事实,如果它足够重要,可以询问证人、查阅文本和检查记录等等。当事实得到了确证、解决了事实问题时,歧见就会得到解决。科学的探究方法在这里都可以用到,这将足以指导他们直面有关信念歧见的事实问题。

另一方面,如果是态度歧见而不是信念歧见,那么适于解决它的方法就有相当大的差别,难以如此径情直遂。以确证事件发生与否为目的,召唤证人、查阅文本,诸如此类,对解决这样的争论不会有什么效果,因为争论的问题并不是事实,这种歧见不是关于事实是什么而是关于怎样评价它们的对立。解决这种态度歧见的努力可能会涉及有关事实问题,但不是那种存在态度冲突的事实。或许考虑那种引出愉快或不愉快的结论的事情若不发生将会怎样,可能是有益的。动机和目的也可能具有重要性。诚然,它们都是事实问题,但如果歧见在信念上而不是在态度上,它们就都不会成为争论的主题。其他一些方法有时也可以解决态度歧见,你可以大量地使用表达性语言来尝试说服方法;在凝结团体意志和取得统一态度中,修辞艺术也可能富有成效(当然,它在解决事实问题上完全没有价值)。\\
"好"和"坏","对"和"错",诸如此类的语词,在严格的伦理用法中,往往具有非常强烈的情感色彩。无疑,当我们把某行动描述为对或把某情形描述为好时,我们就对它表达了一种赞赏态度。有些伦理学者认为,这些语词"没有"词汇意义或认知意义,而只有适合它们的情感意义;而另一些伦理学者则坚持这些语词的确具有认知意义,它们指称所讨论事物的客观特征。在这种争论中,学习逻辑的学生不必偏向某一方。但显然,不是所有的赞同或不赞同态度都蕴涵道德判断,因为还有审美价值的考虑,还有个人偏好或口味的影响。对某些事物(例如有些食品或服装款式)的否定态度,不必牵涉伦理或审美判断,但它也可以由情感色彩强烈的语言来表达。

当歧见是在态度上而不是在信念上时,最强烈的(当然也是实质的)歧见可以用朴实无华的真实陈述来表达。当双方针锋相对并以逻辑上相容的陈述来明确表述他们的分歧时,若认为双方的歧见不是"实质的"或者是"纯粹言辞的"就是错误的。他们并不是仅仅"用不同的言辞说相同的事物"。当然,他们可以用不同言辞来断言词汇意义上相同的事实,但是,他们也可以用不同言辞来表达对相同事实的矛盾态度。在这样的情况下,虽然他们的歧见不是"词汇意义上的",然而却是实质的。这不是"纯粹言辞的"歧见,因为语词既具有表达性功能也具有信息性功能。如果我们有兴趣解决歧见,我们就必须弄清楚其本性,因为适于解决一种歧见的技巧,正如我们已看到的,可能对另一种毫无用处。

有时,我们难以确定一种歧见是信念的或是态度的,或者既是信念的又是态度的;这可能取决于争论者对词汇的某种解说。冲突意见的表达方式可能会把不同态度之间的区别以及不同信念之间的区别弄得模糊,因此争论的关键核心就会难以把握。当两个人对在一个事物是否比另一个"更好"或"更重要"上意见对立时,他们都可能认为,真实情况很可能是不同信念使他们产生了分歧。但是在某些情况下,一个其表面形式是关于所谓的事实问题的差别论争,实际上是一个关于态度的实质争论,当争论的东西是事物或行为的价值时尤其如此。

在关于赢球的重要性上,一位著名体育运动作家和一位著名足球教练产生了深刻的分歧。新闻记者格兰特兰德•赖斯(Grantland Rice)写道:

\begin{displayquote}
当球星走进球场,为其声名留下光芒,这光芒不论输与赢,只记录场上飞奔的身影。
\end{displayquote}

文斯•隆巴蒂(Vince Lombardi)教练却说:

\begin{displayquote}
赢球不是别的,它就是唯一(竞赛目标)。
\end{displayquote}

很明显,这两位对赢球的态度是冲突的。你相信这种态度上的歧见的根源是信念歧见吗?

尽管有这些不可避免的困难,态度歧见与信念歧见之间的区分还是非常有用的;留心语言的不同用法有助于理解我们可能遭遇的种种歧见。当然,找出区别本身并不能解决问题或消除歧见。但是,由此可以澄清讨论的问题,揭示出它们的类型并找出冲突的所在。我们越充分地理解歧见的本性,我们就越能够更好地去解决歧见。

\section*{练习题}
识别下列各组语段中最可能表示的一致和歧见类型。\\
*1.a.按照蛀人的䅗话来回答他自己,免得他自以为有智慧。\\
-Proverbs 26: 5\\
b.不要照䘄人的盘话来回答他自己,\\
免得你也像他一样。\\
—Proverbs 26: 4\\
2.a.阿布哈兹人说土耳其语,大部分人是穆斯林;一千年前,阿布哈兹开始受格鲁吉亚统治。19世纪,格魯吉亚本身也被纳入了俄罗斯帝国;当斯大林,一个格鲁吉亚出身的共产主义者,主宰克里姆林宫时,格鲁吉亚各种族被迫重做调整。去年(1991),格鲁吉亚重获独立……但7月(1992),阿布哈兹分裂势力也宣布独立,尽管现在居住在阿布哈兹的人只有 $18 \%$ 是阿布哈兹族人。\\
-_Editorial,"Abkhazia:Small War,Big Risk",New York Times, 8 October 1992\\
b.你把阿布哈兹人描绘成"说土耳其语,大部分是穆斯林"是令 92 人气愤的。阿布哈兹人民有他们自己的语言,土耳其人完全不橫!你不断地把阿布哈兹人描绘成分裂势力和脱离主义者是非常错误的。阿布哈兹人并没有主张不属于他们自己的领土。千百年来,阿布哈兹就是阿布哈兹人的领土……如果格鲁吉亚人民可以宣布独立,为什么阿布哈兹人不能那样做呢?"自治"这个词只能由格鲁吉亚人使用吗?\\
-Y.Kazan,letter to New York Times, 22\\
October 1992\\
3.a.我想澄清的是,你所提及的我校以前的两位教师,艾尔斯教授和杰弗里博士,并不是像你所说的那样被"行政性暂停职务"。事实是,他们接到了行政部门不再重新聘用的意向通知,从而取消了他们继续履行义务的契约……我们深思熟虑后认定,在他们的案例中,学校没有侵犯学术自由或者违背学术正当程序。\\
--Edward H.Pauley,Vice President for Ac- ademic Affairs,Dallas Baptist University, in a letter to the editor of Measure,Au- gust 1992\\
b.你在信中说,因为艾尔斯教授和杰弗里博士及时收到了不再重

新聘用的通知,所以在他们的案例中随之而来的是"正当程序"。我们不同意这种说法。艾尔斯教授和杰弗里博士签订了1992-1993学年的契约,并被安排在秋季学期教授课程,看起来这些并没有争议。因此,我们仍然认为,行政部门解除艾尔斯教授和杰弗里博士的教职,随后既没有恢复他们原来的职务,也没有给他们一个听证会的机会,这构成了轻率的解聘。\\
-Jonathan Knight,Associate Secretary,A- merican Association of University Profes- sors,in a letter to Vice President Pauley, Dallas Baptist University,Measure,Oc- tober 1992

4.a.及时行事,事半功倍。\\
b.迟做总比不做强。\\
*5.a.距离产生美。\\
b.眼不见,心不想。\\
6.a.跑得快者未必能赢得比赛,实力强者未必能赢得战斗。\\
——Ecclesiastes 9: 11\\
b.但打赌须据此(实力)下注。\\
-Jimmy The Greek\\
7.a.既然有人应当统治而其他人应当被统治不但必要而且可取,那么,有些人天生就注定要服从,其他人则天生要统治 $\cdots \cdots$ 显然,有些人天生是自由的,而其他人则是奴隶,并且对后者而言,奴隶制度既是可取的也是正确的。\\
--Aristotle,Politics\\
b.如果存在某些天生的奴隶,乃正是因为过去曾违背天性地迫使人成为奴隶。暴力造就了第一批奴隶;而由于他们的怯懦与困顿,使奴隶制度对他们的束缚永久化了。\\
---Jean-Jacques Rousseau,The Social Contract\\
8.a.只有战争才能把人类的所有能力提升到最高程度,并给勇于面对战争的民族打上高贵的印记。\\
-Benito Mussolini,Encyclopedia Italiana\\
b.战争以它血腥的铁蹄踏碎所有公平、所有幸福,以及人类所有神圣的东西。在我们这个时代,所有的和平都是光荣的,所有的战争都是

罪恶的。\\
-Charles Sumner,Addresses on War, 1904\\
9.a.自由和公平之下,重要的是公众教育;没有它,不论是自由还是公平都不会保持长久。\\
——James A.Garfield\\
b.教育对任何具有艺术感火花的人都是致命的。教育应当局限于职员,即使他们也要限于必需。这个世界认识到了我们从不学习我们以前不知道的任何东西吗?\\
-George Moore,Confessions of a Young Man, 1888\\
*10.a.相信上帝存在毫无用途,同样,也没有根据。在无神论普及之前,世界上不会有快乐。\\
-J.O.La Mettrie,L'Homme Machine, 1865\\
b.几乎所有的记录在案的无神论者都是行为放荡和卑鄙的人。\\
-J.P.Smith,Instructions on Christian Theory\\
11.a.我知道,对任何国家来说,没有比为改善农业、牧业和为农村劳力关心的其他方面所提供的服务更真实和更重要的了。\\
-George Washington,letter to John Sin- clair\\
b.随着农业的引进,人类便进人了漫长的艰辛、痛苦和疯狂的时期;直到今天,人类才被机器的良好运作所解放。\\
-Bertrand Russell,The Conquest of Hap- piness, 1930\\
12.a.在任何国家,无论何时,只要存在未耕作的土地和未被雇用的穷人,那么显然,不断扩展的财产法违背了自然法权。\\
——Thomas Jefferson\\
b.人天生拥有自己的财产权。这是人与低等动物之间的重要区别之一。\\
-Pope Leo XIII,Rerum Novarum\\
13.a.革命是一种固有权利。当人民被政府压迫时,如果他们足够

强大,那么通过或者从政府那里收回权利,或者推翻它,并更换一个比较可接受的政府,将自己从压迫中解放出来,这是他们的天然权利。

b.煽动革命是叛逆,不但反人类而且反上帝。

\section*{--Pope Leo XIII,Immortalie Dei}
14.a.语言是人类思想的武器库,它包含自己过去的战利品,同时还包含自己征服未来的武器。

\section*{--Samuel Taylor Coleridge}
b.语言,人类的语言,与家禽的呱呱和咯咯叫以及其他野兽的自然发声(有时不是那么充分)相比,毕竟好不到哪儿去。

\section*{--Nathaniel Hawthorne,American Note- books}
*15.a.今天,人们对美国政府感觉如何呢?我回答说,与它联系在一起,让人们感到蒙羞。\\
--Henry David Thoreau,An Essay on Civil Disobedience\\
b.就我们(美国)现有政府的所有缺点而言,它是无与伦比的最好的政府,或者迄今为止存在过的最好政府。\\
——Thomas Jefferson\\
16.a.务农是一种毫无意思的职业,仅仅是循环劳作:播种,收获;收获,播种。永远没有结果。\\
-Joannes Stobaeus,Florilegium\\
b.对我来说,没有比修理地球这个职业更令我快乐的了。\\
——Thomas Jefferson\\
17.a.我们的国家:祝愿她在与外国的交往中总是正确的;可是我们的国家,(究竟)对还是错?!\\
--Stephen Decatur,toast at a dinner in Nor- folk,Virginia,April 1816\\
b.我们的国家,或对或错。对了,就坚持;错了,就改正。 ${ }^{[10]} \quad 95$ -Carl Schurz,speech in the U.S.Senate,

18.a.糟糕的和平甚至比战争更糟糕。\\
-Tacitus,Annals b.\\
b.即使最糟糕的和平也比最正义的战争好。\\
-Desiderius Erasmus,Adagia\\
19.a.无论判决你去农场或者去乡村监狱,都没有什么区别。\\
——Henry David Thoreau,Walden\\
b.我不知道有什么事物比一座良好运作的、精耕细作的农场更能令人赏心悦目、心满意足。\\
-Edward Everett\\
*20.a.思维,与所有强大武器一样,如果使用错误,就是极其危险的。因此清晰的理性思维是人渴望的东西,不仅是为了充分发挥大脑的潜力,而且也是为了避免灾祸。\\
--Giles St.Aubyn,The Art of Argument\\
b.理性是信仰的最大敌人:它从不有助于精神的东西,而是在多数情况下,反对精神世界,蔑视一切来自上帝的东西。\\
-Martin Luther,Table Talk

\section*{2.6 情感中性语言}
语言的表达性用法与信息性用法一样是正当的。情感语言本身没有什么不当,非情感语言或曰中性语言也没有什么不当。这就如同我们可以说枕头没什么错,锤子也没有什么错一样毋庸置疑。但是我们不能用枕头钉钉子,也不能枕在锤子上舒服地休息。同样,当我们用实话实说的话语来替换诗人的情感语言时,尽管可以保留罗曼蒂克诗句的字面意义,但它会失去很多兴味。在某些类型的诗歌中,情感色彩浓郁的语言比中性语言更受喜爱;而在另一些领域中,中性语言则比情感色彩浓郁的语言更为可取。

那是些什么领域呢?如果把探求现实真理作为我们的目标,那么中性语言就应更受重视。当我们试图了解事实的真相所在,或者试图加以论证时,心猿意马就会招致失败,而情感因素正是一种分散注意力的力量。激情倾向于掩盖理性,"冷静"(dispassionate)与"客观"(objective)两词接近同义的用法就反映了这个道理。因此,当我们试图以冷静和客观的方式推论事实时,使用强烈的情感语言便是有害而无益的。

当我们处理某些冲突的话题时,使用完全中性的、不受感情影响的语言或许并不可行。例如,在流产是否正当的论辩中,论辩对手(无论哪一方)所使用的关键词汇就非常可能为情感所扭曲,因为此时并不存在完全不带情感色彩的所有各方都接受的价值中性词汇。在这种情况下,如果真正的目标仍然是求真,那么就应当尽可能地减少所用词汇的情感负荷。情感中性的目标不可能完全达到,但是我们至少可以尽力使用这种语言:它仅预设论辩者都赞同的信念(无论何种)。情感色彩的语言肯定是分散注意力的,情感意义负荷过重的语言是不可能促进求真的。

在论文《决定论的两难》(The Dilemma of Determinism)中,威廉•詹姆士(William James)提出"希望消灭"自由’ 这个词",其理由是"它的歌功颂德的联想……使它的所有其他意义黯然失色"。他更喜欢恰当地使用词汇"决定论"和"非决定论"来讨论"意志自由",他说,这是因为"它们冰冷的和数学的面孔没有情感牵连,而情感牵连可以预先以各种方式来贿赂我们"。詹姆士的做法为我们树立了榜样。

从事专业民意调查的采访者必须非常谨慎,以免对调查询问中使用情感措辞而收到的反馈产生偏见。如果忽视这种慎重,调查结果就可能是没有价值的。1993年,《时代》(Time)和有线电视新闻网(Cable News Network)在一项大型民意调查中询问道:"应当通过立法来禁止利益集团赞助竞选吗?或者,利益集团的确应当拥有赞助他们支持的候选人的权利吗?"在所有回复者中, $40 \%$ 的人说他们赞同禁止利益集团的赞助, $55 \%$ 的人回答利益集团有赞助的权利。同年,罗斯-佩罗(Ross Perot),一位非常富有的总统候选人,组织了其自己的民意测验,其中如此询问: "应该通过立法来消除所有特殊利益者给候选人大笔金钱的可能性吗?"冊庸惊奇,对这个问题 $80 \%$ 的回答是"是的",这样的赞助应当禁止。包含像"特殊利益"和"大笔金钱"这样的短语无疑妨害了了解人们对这种事情的真实看法。 ${ }^{[11]}$ 可以说,这两个民意测验问的不是同样的问题。但即使如此,逻辑要点仍然是:如果我们的目标是交流信息,如果我们希望避免误会,那么我们就应该尽可能少地使用情感色彩浓厚的语言。玩弄情感,而不是诉诸理性,是那些从歪曲真相中获得好处的人的常用手段。这种操纵的努力最公开的展示是广告世界,那里压倒一切的目的总是说服、销售并且常常是获利。我们必须经常警惕这些情感负荷的语言用法,也要警惕它们在政治活动中的使用,几乎每一种修辞手段都在政治活动中一再使

用。本杰明•迪斯雷利说:"我们能利用语词来支配人。"我们最好的防御就是对语言及其不同的用法要多思而敏感,并具备识别那些不讲道德原则的人强词夺理的伎俩的能力。

\section*{练习题}
从近期期刊上选择一段简洁而又具有高度情感的文章,并进行如下翻译:保留其信息性意义,而将它的表达性意义降至最低。

\section*{第2章概要}
本章阐释语言的多种用法和形式,以及可能由于没有认识到这些复杂性而引起的错解和滥用。\\
2.1 节区分了语言的三种基本功能:信息性功能、表达性功能和指令性功能。\\
2.2 节展示了一个给定语段可能行使的多种功能的方式:同时行使两种甚或所有三种功能。\\
2.3 节表明了标准语法形式的句子,即陈述句、疑问句、祈使句和感叹句,并不总是行使与其名称相关的功能。陈述句可以用做指令性的或者表达性的功能;疑问句可以具有信息性的或指令性的功能,等等。语法形式不决定语言功能。\\
2.4 节考察了构成句子的词汇的用法,也讨论了情感语言。\\
2.5 节区分了信念歧见与态度歧见。冲突双方可以既在事实是什么上一致也在对事实的态度上一致,或者在两方面都对立。他们可能在事实上一致,而在对事实的态度上对立。他们还可能在事实是什么上对立,但在对他们所相信的事实的态度上却一致。要解决歧见问题,了解其真正本性是极其重要的。

2. 6 节讨论当辩论的目的是求真时,尽可能地将负载情感的语段归约为情感中性语言的重要性。

\section*{【注释】}
[1]Palm Beach County Canvassing Board v.Katherine Harris,decided 21 No-\\
vember 2000.\\
[2]Ann Lander,"You Could Be Dead Right!"syndicated column, 26 August 1988.\\
[3]关于对这个主题的介绍,有兴趣的读者可以参考 Nicholas Rescher 的 The Logic of Commands(London:Routledge \textbackslash &Kegan Paul,1996)。\\
[4]Leonard Bloomfield,Language(New York:Henry Holt,1933).\\
[5]Cohen v.California, 403 U.S.15,at p.26, 1971.\\
[6]In a speech at the National Convention of the Democratic Party,in Chicago,Ju- ly 1968.\\
[7]In The Female Eunuch(New York:McGraw-Hill,1971).\\
[8]Margaret Schlauch,The Gift of Tongues(New York:Viking Press,1942).\\
[9]关于短语"在信念上"和"在态度上"的一致和歧见,以及我们在第 4 章讨论的概念的"说服定义"(persuasive definition),我们感激我们的同事和朋友斯蒂文森(Charles L.Stevenson)教授。参见他的《伦理学和语言》(Ethics and Language) (New Haven,CT:Yale University Press,1944)。\\
[10]对于这种歧见,切斯特顿(G.K.Chesterton)评论道:""我们的国家,或对或错’就像说‘我的母亲,或陶醉或哭泣’。"\\
[11]佩罗民意测验的 17 个问题,大多都包含情感负荷的词汇;它们以"公民选票"的形式登在1993年3月20日的TVGuide上;1993年3月21日 NBC TV 也在全国范围内进行征询。1993年3月,New York Times 报道了 Time/CNN 的民意测验。

\end{document}