\section{密尔方法}

\begin{logicbox}[title=引言]
本节深入介绍\logicterm{归纳推理}中最重要的方法体系——\logicterm{密尔方法}。作为现代科学方法论的重要基石,密尔方法不仅克服了简单枚举归纳法的根本局限性,更为科学研究提供了系统的因果分析工具。我们将详细分析约翰·斯图亚特·密尔在《逻辑系统》中提出的五种\logicterm{归纳法则}:\logicterm{求同法}、\logicterm{求异法}、\logicterm{求同求异并用法}、\logicterm{剩余法}和\logicterm{共变法}。通过深入理解这些\logicterm{归纳方法}的哲学基础、逻辑原理和实际应用,我们将能够更\logicemph{有效地}分析\logicterm{因果关系},掌握科学研究中检验假说的基本工具,并理解现代实验科学的方法论基础。
\end{logicbox}

\subsection{密尔方法的历史背景与理论意义}

\begin{theorembox}[title=从培根到密尔:归纳方法的历史发展]
人们早已知道\logicterm{简单枚举法}的局限。早在1605年,弗兰西斯·培根就提出了其他类型的\logicterm{归纳程序}。他在其伟大的著作《学习的进步》中探寻改革科学研究的方法。

\textbf{培根的贡献与局限}:
\begin{itemize}
\item \textbf{方法论革命}:培根首次系统提出了经验归纳法,强调观察和实验的重要性
\item \textbf{排除法思想}:提出了通过排除不相关因素来发现真正原因的基本思路
\item \textbf{历史局限}:培根的方法缺乏精确的逻辑表述和系统化的理论框架
\end{itemize}

\textbf{密尔的理论贡献}:
但是,更为强大的\logicterm{归纳方法},其精确表述和系统化,是由另一个英国哲学家约翰·斯图亚特·密尔在其著作《逻辑系统》(1843)中所完成,并被称为"\logicterm{归纳推理的密尔方法}"。

\textbf{密尔方法的历史地位}:
\begin{itemize}
\item \textbf{方法论的成熟}:将培根的归纳思想发展为精确的逻辑方法
\item \textbf{科学实践的指导}:为19世纪以来的科学研究提供了系统的方法论指导
\item \textbf{现代科学的基础}:成为现代实验科学和统计学的重要理论基础
\item \textbf{跨学科影响}:在自然科学、社会科学、医学等领域都有广泛应用
\end{itemize}
\end{theorembox}

\begin{theorembox}[title=密尔的五种归纳法则及其理论基础]
密尔总结为五条"教规",他称它们为:

1. \logicterm{求同法}(The Method of Agreement)
2. \logicterm{求异法}(The Method of Difference)
3. \logicterm{求同求异并用法}(The Joint Method of Agreement and Difference)
4. \logicterm{剩余法}(The Method of Residues)
5. \logicterm{共变法}(The Method of Concomitant Variation)

\textbf{方法体系的理论特征}:
\begin{itemize}
\item \textbf{系统性}:五种方法相互补充,形成完整的因果分析体系
\item \textbf{逻辑性}:每种方法都有明确的逻辑结构和推理规则
\item \textbf{实用性}:直接指导科学实验和观察的设计
\item \textbf{普遍性}:适用于各种类型的因果关系研究
\end{itemize}

\textbf{与简单枚举法的根本区别}:
密尔方法通过主动的排除和控制,而非被动的事例收集,来确定因果关系,这标志着从经验归纳向实验科学的重要转变。
\end{theorembox}

我们将依次考察它们。我们首先分析密尔对每个方法的经典陈述,接着对它们进行深入的说明和分析。尽管密尔对这些方法的解释现在来说十分古老,但是密尔对这些在寻找因果律过程中时时和处处使用的最基本的工具的分析是精辟的。

\footnote{该书有严复中文节译本《穆勒名学》,在中国学界有历史性影响,故过去通常将Mill译为穆勒,近年多据正确读音译为密尔或弥尔。}

\subsection{求同法:寻找共同因素的逻辑}

\begin{theorembox}[title=求同法的基本原理]
约翰•斯图亚特•密尔写道:

"如果被研究的现象的两个或更多的事例只有一个共同的事态,那么,这个事态——所有事例在该事态上相契合——是给定现象的原因(或结果)。"

\textbf{方法的逻辑基础}:
求同法基于一个重要的逻辑假设:如果某个因素是现象的原因,那么该因素必须在现象出现的所有场合都存在。这体现了原因作为必要条件的逻辑特征。

\textbf{排除法原理}:
该方法的核心是排除法——通过排除那些不在所有事例中都出现的因素,来确定真正的原因。这种排除过程体现了科学研究中控制变量的基本思想。

\textbf{与简单枚举法的优越性}:
该方法比简单枚举法优越。它不仅试图发现原因与结果重复出现的连接,而且试图确定这个唯一的事态——不变地与我们感兴趣的结果或现象关联的一个事态。
\end{theorembox}

\begin{examplebox}[title=求同法在科学探究中的应用]
这是科学探究的一个重要的也是非常普遍的工具。例如,在寻找某个致命的流行病过程中,或者在查找某些地质现象的原因的过程中,流行病专家或地质学家将选出特定的那些事态,答案就在其中;他们询问,明显不同的事态集合(答案就在其中)在什么方面相一致?

\textbf{流行病学调查案例}:
想象一下在某个公寓楼的居民当中发生消化不良,我们得了解其原因。首先要研究的自然是所有得病的人吃了什么食物?一些病人吃的而不是所有病人吃的食物不可能是得病的原因;我们希望知道什么事态是每个得病场合所共同的。当然,共同的东西可能不是一种食物;可能是受感染的器具,或者接近某种有害的污水,或其他的情况。但是,仅当我们找到了某种对所有疾病的事例都是共同的事态,我们才找对了正确解决问题的途径。

\textbf{方法应用的关键步骤}:
\begin{itemize}
\item \textbf{事例收集}:收集所有出现目标现象的事例
\item \textbf{因素分析}:分析每个事例中存在的各种可能因素
\item \textbf{共同性识别}:找出在所有事例中都存在的共同因素
\item \textbf{排除过程}:排除不在所有事例中都出现的因素
\item \textbf{因果推断}:将共同因素确定为可能的原因
\end{itemize}
\end{examplebox}

\begin{theorembox}[title=求同法的形式结构]
求同法可以示意如下。其中大写字母表示事态,小写字母代表现象:

$$
\begin{aligned}
& A 、 B 、 C 、 D \text { 与 } w 、 x 、 y 、 z \text { 一起发生 } \\
& A 、 E 、 F 、 G \text { 与 } w 、 t 、 u 、 v \text { 一起发生 }
\end{aligned}
$$

$$
\text { 因而 } A \text { 是 } w \text { 的原因 (或结果) }
$$

\textbf{逻辑结构分析}:
\begin{itemize}
\item \textbf{前提1}:在现象w出现的第一个事例中,存在事态A、B、C、D
\item \textbf{前提2}:在现象w出现的第二个事例中,存在事态A、E、F、G
\item \textbf{共同因素}:只有事态A在两个事例中都出现
\item \textbf{结论}:因此A是w的原因(或结果)
\end{itemize}

\textbf{方法的建设性价值}:
该方法在确定一种现象或者事态的一个范围方面特别有用,对之的进一步研究将产生成效。它是富有建设性的,甚至在不能有结论的地方也是如此。
\end{theorembox}

\begin{examplebox}[title=分子遗传学中的求同法应用:阿尔茨海默病研究]
例如,在分子遗传学中,一个遗传疾病的可能原因其范围往往通过使用求同法而大大变小。寻找范围被锁定在由特定疾病频繁发生的那些人或家族的独特的基因构成。

\textbf{研究背景}:
阿尔茨海默病(Alzheimer,导致精神过程进一步和不可逆转的下降)被认为是遗传的。在所有得病的人的基因构成中存在某个共同的事态吗?

\textbf{研究过程}:
华盛顿大学的一个研究小组首先筛选了上百个得病家族,然后,对一个阿尔茨海默病高发病率的范围相对小的家族进行了艰苦的调查,该研究负责人写道:

"我们选取该疾病遗传明显的那些家族。我们做这样的假设,存在一个有缺陷的基因,我们的任务是找到它。我们的初始工作是在一个包含所有人类染色体的大草堆中寻找一根针。我们在第14号染色体上发现了一个小小的地方,在那里存在一个引起阿尔茨海默病的有缺陷的基因。"\cite{tanzi1996}

\textbf{求同法的应用}:
\begin{itemize}
\item \textbf{事例收集}:筛选上百个患病家族
\item \textbf{共同性寻找}:在所有患病个体中寻找共同的基因特征
\item \textbf{范围缩小}:将搜索范围锁定在第14号染色体
\item \textbf{因果确定}:确定特定基因缺陷为疾病原因
\end{itemize}

\textbf{科学意义}:
这一发现不仅为阿尔茨海默病的诊断和治疗提供了新的途径,更展示了求同法在现代分子生物学研究中的强大威力。
\end{examplebox}

\begin{examplebox}[title=公共卫生史上的重大发现:氟化物防龋]
在几年前求同法的一个类似使用,产生了一项给人类带来巨大利益的发现。

\textbf{观察现象}:
人们发现,在某些城市里牙齿腐烂的速度相当慢,而当时不知原因为何。

\textbf{求同法应用}:
研究发现,那些城市中存在一个共同的事态:在那些城市的供水中氟的含量不同寻常的高。

\textbf{因果推断}:
人们得出,使用氟能够减少牙齿腐烂的发生。该结论随后得到证实。

\textbf{实践应用}:
结果是人们在全球范围内的城市供水中加氟。

\textbf{方法总结}:
简言之,我们找到一个对给定现象的所有事例来说都是共同的事态,此时我们可以自信地认为,我们已经发现了它的原因。

\textbf{历史意义}:
这一发现被认为是20世纪公共卫生领域的十大成就之一,展示了求同法在流行病学研究中的重要价值。
\end{examplebox}

\begin{theorembox}[title=求同法的严重局限性]
然而,求同法有严重的局限。

\textbf{1. 确证事例的不充分性}:
单单该方法往往不足以确定待寻找的原因。我们难以安排可用数据,以确定所有事例所共同的一个事态。

\textbf{2. 多重共同因素问题}:
当研究发现所有事例中共同的事态不止一个时,只使用该技术不能评判这些不同的可能性。

\textbf{3. 排除功能的价值}:
尽管事态和现象之间的求同经常不是结论性的,但缺乏相同点可以帮助我们确定什么不是待研究现象的原因。

\textbf{4. 本质上的排除法特征}:
求同法本质上是排除法,它说明了这样的事情,在我们感兴趣的现象出现的某些场合而不是所有场合下出现的事态,不可能是该现象的原因。

\textbf{5. 充分条件与必要条件的区分}:
因而,人们否定某个声称的因果关系,可能是因为他们注意到缺乏共同点,从而推论得出所声称的原因既不是该现象的充分条件又不是它的必要条件。
\end{theorembox}

\begin{examplebox}[title=求同法的反驳应用:教育投入与学业成绩关系的质疑]
例如,某些人认为,在公立学校学生的进步表现(由教育评估考试即SAT的分数确定)与州政府在学校上的投入之间存在一个因果关系;投入的钱越多,学习越好。该观点在一定程度上被人们所驳倒。

\textbf{反驳证据}:
人们指出:

在1992-1993年间教师薪水最高的五个州中,没有一个进入SAT最高分的15个州的行列之中;单位学生花费最高的10个州中,仅有一个州(威斯康星州)进入10个SAT最高分的州中;并且,单位学生花费最高的新泽西州在分数排行榜中位于第34位。——所有证据都表明高投入不是学生成绩的充分条件。

但是单位学生花费最低的10个州中,有4个州(北达科他州、南达科他州、田纳西州、犹他州)其拥有的SAT分数位于SAT分数顶尖的10个州之列;而北达科他州花费排名为第44位,而SAT分数排名为第2位;南达科他州的教师薪水为倒数第一,而其SAT分数为第3位。所有这些证据说明,高投入不是学生学习取得好成绩的必要条件。\cite{forbes1996}

\textbf{讽刺性观察}:
具有讽刺意味的是,丹尼尔•帕特瑞克•莫尼汗议员通过观察得到,影响美国公立学校质量的决定性因素不是金钱,而是与加拿大的距离!

\textbf{方法论意义}:
该论证远不是结论性的——但是缺乏相同性、缺乏一致性确实使人们质疑所提出的因果关系。这个案例展示了求同法在反驳虚假因果关系方面的重要价值。

\textbf{教育政策启示}:
这一分析提醒我们,教育质量的影响因素可能比简单的资金投入更加复杂,需要考虑文化、社会、地理等多重因素。
\end{examplebox}

我们了解了求同法能够告诉我们的东西之后,在寻找原因过程中我们需要其他较精致的归纳法。

\subsection{求异法}
约翰•斯图亚特•密尔写道:

象不发生,两个事例中的事态除了这一个事态不同外(该事态仅在现象发生的过程中),其他均相同,该事态(它使两个事例产生区别)便是该现象的结果或原因,或者为原因中的一个不可缺少的部分。

该方法不关注在产生结果的事例中什么是共同的,而是关注在产生结果的事例和没有产生结果的事例之间存在什么差异。当我们研究胃不适问题时,如果我们已经知道得病的所有人吃了甜点罐装梨子,而没有吃那些梨子的人没有得病,我们能相当自信地认为,我们已经找到了该病的原因。

求异法和求同法之间的差别,在最近的一份关于荷尔蒙睾丸激素在雄性好斗行为中的作用的报告中表现突出。

许多物种的彝丸在一年的大多数时间里是封存不用的,只在一个特定交配的季节期间里,精确地说是在雄性与雄性之间打斗增加的那段时间里,它们才启动并产生晖九激素。尽管它们表现明显,这些数据仅仅是相关的:打斗发生的时候经常发现草丸激素。

可以用刀来证明,委婉说法是进行摘除实验。将物种中的军丸激素之源去除,好斗程度便下降。注入合成搴丸激素使奉丸激素回到正常水平之后,好斗便得以恢复。\cite{sapolsky1997}

这个摘除和恢复方法给出了荷尔蒙与好斗之间存在关联的证明,当然这个证明方法是受揵责的。

明显的,睾丸激素造成了关键的差别,但是作者报告谨慎,没有断定睾丸激素是雄性好斗的原因;而是更准确地说,睾丸激素肯定与好斗相关。用密尔的说法,就是说荷尔蒙是雄性好斗原因中的一个不可缺少的部分。如果我们能够确定单个因素,该因素在其他一切保持不变的情况下造成了差别,即:当我们去除该因素时,待考察的现象也不再发生,当我们将该因素引进来时,考察的现象发生了,此时,我们将相当肯定地找到我们考察的现象的原因或原因的一个不可缺少的部分。

求异法可用下面的形式来刻画,其中大写字母表示事态,小写字母表示现象:

$$
A 、 B 、 C 与 w 、 x 、 y 、 z 一起发生 \\
B 、 C 、 D 与 x 、 y 、 z 一起发生
$$
求异法在几乎所有类型的科学研究中起着中心作用。该方法在医疗研究人员对特定蛋白质的效果进行的研究中得到鲜活应用,这种蛋白质被怀疑与某种疾病的发展有关联。待考察的物质是否真的是原因(或者原因的一个不可缺少的一个部分),只有在我们建立了一个该物质被排除的实验环境的时候才能确定。当然,研究人员只能是在老鼠身上而不是在人身上进行该研究:从染上同样疾病的老鼠的身上去除产生可疑蛋白质的基因。处理过的老鼠进行近亲交配,以产生后代。这些后代被称为"基因剔除老鼠"(knockout mice),在当前医学研究界是很珍贵的。人们能够在一个老鼠身上研究与该疾病有关的过程。该老鼠与其他患有该种疾病的老鼠除了由基因剔除产生的差别外其余的完全一样,老鼠身上由基因剔除而缺少的物质被假定为原因。这样的研究在医疗上产生了重大进展。

求异法的一个著名的同时也是令人感动的例证,由对黄热病真实原因的确证实验的解释所提供。黄热病是人类长期遭受的重大瘟疫之一。这里描述的实验是由美国军队医生瓦尔特•雷德、詹姆斯•卡罗尔和杰西•W•拉杰尔在1900年11月进行的。该年初,卡罗尔医生在另外的实验里故意让自己被受感染的蚊子所叮咬,从而使自己染上黄热病。不久后另外一位医生拉杰尔死于黄热病,随后进行实验所在的营地以他的名字命名以纪念他。

所设计的实验其目的是表明,蚊子传播黄热病(通过排除受感染的所有其他途径)。建造了一个小房子,绝对杜绝蚊子从窗户、门及其他可能的出口出入。一个金属丝蚊帐将房间分成两个空间,其中一个空间里15个已经叮咬过黄热病病人的蚊子在飞。一个没有免疫的志愿者进入有蚊子的房间,他被 7 个蚁子所叮咬。四天后,他感染了黄热病。另外两个没有免疫力的人在无蚊子的房间里睡了 13 个晚上,而没有任何反应。

为了表明,该疾病的传播是通过蚊子而不是通过黄热病人的排泄物或与他们接触过的东西来进行的,另外一处房子建造了起

\begin{displayquote}
来。该房屋里是无蚊子的。将黄热病人的衣物、床上用品和吃饭器具,以及被黄热病人的血液、排泄物污染的其他器具,放置于该房屋,然后,让 3 个没有免疫力的人住在该屋子里。他们所用的床单是从病人的床上取下来的,那些病人因黄热病而死去。对床单上的污染的东西没有进行清洗,也没有进行其他的处理。以不同的志愿者将实验重复了两次。在整个阶段,居住在房子里的人被严格隔离,以免遭蚊子叮咬。这些实验中的人没有一个感染上黄热病。在随后的实验中证明了他们本身不具有免疫力,因为他们中的四个或者引蚊子叮咬或者因注射了黄热病人的血液,而感染了黄热病。\cite{garrison1929}
\end{displayquote}

在上述第一段落所描述的实验中,在两个精心密闭的空间中的受试人之间制造了一个重要的差异:一个房间里有叮咬过黄热病人的蚊子,另外的房间里则没有这样的蚊子。上述第二段落中所描述的实验精心使用了求异法:两组志愿者都密切接触黄热病人,唯一重要的差别是,其中一些志愿者后来被感染的蚊子所叮咬,或者注射了感染了的血液——缺乏这种事态,便没有感染发生。

在寻找原因过程中,求异法是普遍可用的同时也是强有力的。

\subsection{求同求异并用法}
尽管密尔认为这是一个不同的和独立的方法,但该方法最好理解成求同法和求异法在同一个研究中的联合运用。该法可以图示如下(也用大写字母表示事态、小写字母表示现象):

$$
\begin{array}{ll}
A, B, C-x, y, z ; & A, B, C-x, y, z \\
A, D, E-x, t, w ; & B, C-y, z \\
\hline
\end{array}
$$

因而,$A$ 是 $x$ 的结果或原因,或原因中不可缺少的一部分

由于两个方法(左边刻画的是求同法、右边刻画的是求异法)中的每一个方法给结论以某个概率的支持,它们的联合运用给该结论提供了较高的概率。在许多科学研究中,这种联合运用成为威力强大的归纳推理模式。

最近的一个著名医学成就显示了这种并用法的威力。甲型肝炎是肝脏感染,它折磨着成千上万的美国人;它在儿童中广泛传播,主要通过受污染的食物和水进行传播。它有时是致命的。如何预防它呢?当然,理想的方法是注射有效疫苗。但是一个很大的困难是,给何人注射甲肝疫苗?难以预测何处将爆发感染。因而,通常来说,不可能通过选择实验对象以产生可靠的结果。这个困难最终被克服,方法如下。

一种被认为有效的疫苗,在纽约俄兰基县克亚斯-乔尔镇的哈西德教派的犹太人社区中进行测试。该社区不同寻常,每年都流行甲肝。在克亚斯•乔尔镇几乎无人能够逃过甲肝的感染,该社区中近 $70 \%$ 的人在 19 岁

前就感染上了。克亚斯•乔尔医学研究所的阿兰•威尔兹伯格和他的同事,在该社区中招募了年龄 2 至 16 岁的 1037 名儿童,这些儿童没有受到甲肝感染——他们血液中没有该病毒的抗体。一半儿童(519.人)注射了一种新的疫苗,这些注射了疫苗的儿童中没有发现一例甲肝。没有注射疫苗的 518 个儿童中 25 个儿童不久被甲肝病毒感染。于是人们找到了甲肝疫苗。\cite{werzberger1992}

波士顿、华盛顿的肝脏专家对该项研究表示祝贺,称赞该研究是"一个重大突破"、"医学上重要的进展"。该研究依赖于什么推论方式?求同法和求异法都用到了。在医学研究中人们普遍这样做。在该社区能够对甲肝病毒免疫的年轻人中,只有一个条件是共同的:所有免疫者都接受了新的疫苗。由此,我们肯定地认为,该疫苗确实是导致免疫的原因。求异法对结论提供了很大的支持:免疫者的事态和不免疫者的事态在每个方面均类似,只在一个方面不同,即免疫居民被注射了疫苗。

人们经常进行所谓"双管齐下"(double-arm)实验,以检验新药或新方法:一组接受新的治疗,而另外一组没有;第二阶段,对原来没有接受治疗的人进行治疗,对原来接受治疗的人不施行治疗。这样研究的基础是求同法和求异法的联合运用,该方法应用广泛并且是有威力的。

\subsection{剩余法}
约翰•斯图亚特•密尔写道:

从一个现象中减去这样一个部分,在以前的归纳中该部分被认为是某个先行事件的结果,那么该现象剩余的部分为剩余的先行事件的结果。

前面的三个方法似乎假定了,我们能够整个地淘汰或产生某个现象的原因(或结果),有时我们确实能够这样。然而在某些情况下,我们只能通过观察一组事态中的变化——我们已经部分地知道该变化的原因——而推论得某个现象的因果性作用。

该方法关注剩余物。用于称货车上货物的重量的特殊装置可以很好地说明该方法。已知空车的重量。为了测定货物的重量,称出货与车一起的重量,然后我们就知道了货物的重量:整个重量减去车的重量。用密尔的术语来说,已知的"先行事件"是已经记录的空车重量——它必须从总数中减去;总数和已知的先行事件之间的差值,其原因明显地应归因于剩余的"先行事件",即货物本身。

剩余法可以表示如下:

$$
A, B, C-x, y, z
$$

已知 $B$ 是 $y$ 的原因 $C$ 是 $z$ 的原因

因而,$A$ 是 $x$ 的原因

天文学史上的一个伟大章节,即海王星的发现,给我们提供了剩余法威力的一个极好案例:

1821 年,巴黎的波瓦尔德发表了行星包括天王星的运动数据表。在准备天王星数据的时候,他遇到了很大的困难:根据 1800 年以后得到的位置数据而计算出来的轨道,与根据该行星刚刚被发现之后所观察到的数据所计算出来的轨道不协调。他对以前的观察数据完全置之不理,他的图表建立在新近观察的数据之上。然而,在后来的几年里,根据该表而计算出来的位置与该行星观察的数据存在不一致;到 1844 年差值总计达 2 分钟弧度。由于所有其他已知行星的运动位置与计算出来的位置一致,天王星中出现的差值引发了大讨论。

1845 年,勒维烈——那时还是一个年轻人——着手解决该问题。他检查了波瓦尔德的计算,发现计算是正确的。他感到,该问题的唯一满意的解释是,在天王星周围的某个地方存在一个干犹它运动的行星。到1846年的中期,他完成了他的计算, 9月他写信给柏林的迦勒(Galle),请求他在天空的一特定位置寻找一个新的行星。因为在德国已经绘制出了包含新的恒星的图表,而勒维烈当时还没有获得这些图表。在9月23日,迦勒开始寻找,在不到一小时的时间里他找到了一个物体,而这个物体是新图表中所没有的。到第二晚,该物体发生略微的移动,这个新的行星——后来被命名为海王星——在预测的位置的 1 度内被发现。该发现被认为是数理天文学中一个巨大的成就。\cite{berry1961}

这里,待研究的现象是天王星的运动。当时人们能够对该现象一一天王星绕太阳运行的轨道一一的大部分有很好的理解。天王星的观察数据近似于计算的轨道,但是一个难解之谜是,它们之间的差值。这个差值已经计算出来,需要我们进一步解释。一个附加的"先行事件"(一个存在的能够对这种差别进行说明的附加因素),被假设为另外一个(未发现的)星球,它的引力与已知的天王星轨道有关的假说一起,对这个差值进行说明。一472旦做出这样的假设,那个新的行星很快就得以发现。

使用;而其他方法要求考察至少两个事例。并且与其他方法不同的是,剩余法依赖于预先建立的因果律,而其他方法(如密尔描述的那样)则不是。尽管如此,剩余法是归纳的,而非演绎的。因为它产生的结论仅仅是或然的,而不能从前提中有效演绎出来。一个或两个附加的前提会使剩余法的推理转变成一个有效的演绎论证——但是也能够将之说成是另外的归纳方法。

\subsection{共变法}
已经讨论的四个方法本质上都是排除法的。通过剔除出给定现象的某个或某些可能原因,这些方法对其他的某个假定的因果解释提供支持。求同法排除掉那些不可能为原因的事态——在该事态缺乏的情况下该现象仍然能够发生;求异法通过剔除关键的一个先行因素而排除某个或某些可能原因;求同求异并用法也是排除法,它同时使用上面的两种方法;而剩余法努力排除那些不可能为原因的事态——这些事态的结果已经通过归纳预先建立起来。

但是存在这些方法都不可用的许多情形,因为存在不可能排除的事态。这经常发生在经济学、物理学、医学,以及在一个因素的增或减导致相伴随的另外一个因素的增或减的任何地方。此时,完全排除一个因素不可行。

约翰•斯图亚特•密尔写道:

一个现象随着另外一个现象以某种方式变化而发生变化,此时另外一个现象或者是该现象的一个原因,或者是一个结果,或者它通过某个作为原因的事实与之相连接。

例如,共变法对于研究某种食物的因果作用是重要的。无论我们吃什么食物,我们都不能排除疾病。我们几乎不能从大量人口的食物中排除掉某种食物,但是我们能够注意到,在特定人群中增加或减少某种食物量对某种疾病发生频率的影响。该种方法的一个最近的研究是,考察心脏病发生的频率,并与吃鱼的人心脏病发病的频率相对比。归纳出来的结论是惊人的:一周吃一次鱼肉,患心脏病的危险降低了 50 个百分点;一个月吃

两次鱼肉,患心脏病的危险降低了 30 个百分点。在某个范围内,在心脏患病和吃鱼之间似乎存在显著的共同变化。\cite{stampfer1994}

用加或减的符号表示一个变化的现象出现在一个给定情形中较高或较低的程度,共变法能够表示如下:

$$
\begin{aligned}
& A B C-x y z \\
& A+B C-x+y z \\
\hline & \text { 因而 } A \text { 与 } x \text { 因果地连接在一起 }
\end{aligned}
$$

该方法有广泛的应用。农民通过对不同的土地施不同数量的肥料,观察到肥料用量与产量之间的变化关系,而得出所施的肥料与庄稼收成之间的因果连接。商人在不同的时间段播放不同的广告,以观察那些时间段生意的好坏,从而确定不同种类的广告的功效。

当一个现象的增加对应于另外一个现象的增加时,我们说这些现象之间是直接相关的。但是该方法可以以任何方式来使用。当现象间是反方向变化的时候——一个现象的增加导致另外一个现象的减少,我们同样可以推论出一个因果关系。经济学家经常说,假定其他事物基本保持不变,在非计划的市场中某种货物(如原油)供应量的增加,将导致其价格发生相应的降低。该关系确实显示出真正的共变:当国际局势紧张、原油供应面临短缺的威胁的时候,我们注意到石油价格就无例外地上升。

当然,一些共同变化完全是偶然的。我们必须谨慎,不能从完全偶然的事件关系中推论出一个因果连接。但是某些变化看上去是偶然的(否则是令人费解的),但可能具有一个隐蔽的因果解释。人们发现,在英国乡村筑巢的鹳的数量与在每个乡村出生的婴儿之间存在高度相关;鹳越多,婴儿越多。这肯定不可能 $\cdots \cdots$ 是的,这不可能。具有高出生率的乡村具有更多新婚夫妇,因而具有更多的新建房屋。巧的是,鹳喜欢在以前没有被其他鹳用过的烟囱旁边筑巢。\cite{matthews1999} 追寻共同变化的现象的因果链条,我们可以找到共同的环节,这就是密尔所要表达的意思——他说这些现象可能是 "通过某个作为原因的事实……而连接起来"。

因为共变法允许我们举出例证,说明事态和现象之间出现的程度之间的变化关系,它大大加强我们的归纳技术。它是归纳推理的定量方法,而前面讨论的那些方法本质上是定性的。使用共变法预设了,存在对现象变

化的程度进行测量或估计的方法,哪怕仅仅是大致的。

\begin{center}
\fbox{\parbox{0.95\textwidth}{
\textbf{本节要点}
\begin{itemize}
\item \textbf{密尔方法的历史背景与理论意义}:
  \begin{itemize}
  \item 从培根的归纳思想到密尔的系统化:方法论革命的完成
  \item 密尔方法的历史地位:现代科学方法论的重要基石
  \item 跨学科影响:在自然科学、社会科学、医学等领域的广泛应用
  \item 与简单枚举法的根本区别:主动排除控制vs被动事例收集
  \end{itemize}
\item \textbf{五种密尔归纳方法的深入分析}:
  \begin{itemize}
  \item \textbf{求同法}:基于必要条件逻辑,通过排除法寻找共同因素
  \item \textbf{求异法}:关注差异性,通过控制变量确定关键因素
  \item \textbf{求同求异并用法}:双重验证,提高因果推理的可靠性
  \item \textbf{剩余法}:依赖预先建立的因果律,发现新的因果关系
  \item \textbf{共变法}:定量方法,研究变量间的数量关系
  \end{itemize}
\item \textbf{求同法的深入分析}:
  \begin{itemize}
  \item \textbf{逻辑基础}:原因作为必要条件的逻辑特征
  \item \textbf{形式结构}:通过共同因素识别确定可能原因
  \item \textbf{建设性价值}:在确定研究范围方面特别有用
  \item \textbf{经典应用}:阿尔茨海默病基因研究、氟化物防龋发现
  \item \textbf{严重局限性}:确证事例不充分、多重共同因素问题
  \item \textbf{反驳功能}:通过缺乏共同性质疑虚假因果关系
  \end{itemize}
\item \textbf{方法体系的理论特征}:
  \begin{itemize}
  \item \textbf{系统性}:五种方法相互补充,形成完整的因果分析体系
  \item \textbf{逻辑性}:每种方法都有明确的逻辑结构和推理规则
  \item \textbf{实用性}:直接指导科学实验和观察的设计
  \item \textbf{普遍性}:适用于各种类型的因果关系研究
  \end{itemize}
\item \textbf{现代科学研究中的应用价值}:
  \begin{itemize}
  \item 分子遗传学:基因-疾病关系的确定
  \item 公共卫生:流行病学调查和预防措施制定
  \item 教育研究:政策效果评估和因果关系质疑
  \item 实验设计:为现代科学实验提供方法论指导
  \end{itemize}
\item \textbf{认识论意义}:
  \begin{itemize}
  \item 体现了从经验归纳向实验科学的重要转变
  \item 为现代实验科学和统计学提供了重要理论基础
  \item 展示了逻辑方法在科学发现中的重要作用
  \item 强调了排除法和控制变量在因果分析中的核心地位
  \end{itemize}
\end{itemize}
}}
\end{center}