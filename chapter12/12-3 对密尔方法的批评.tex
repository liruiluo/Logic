\section{对密尔方法的批评}

\begin{logicbox}[title=引言]
本节评估\logicterm{密尔归纳方法}的局限与价值。我们将分析这些方法在实际应用中面临的困难,包括识别相关事态的问题和证明\logicterm{因果关系}的挑战。同时,我们也将认识到\logicterm{密尔方法}作为科学研究中检验假说工具的重要价值,理解它们在现代科学实验设计中的核心地位。
\end{logicbox}

\subsection{密尔方法的局限}
密尔本人相信,上面分析的技术可以用做发现因果关系的工具,并且能够用做证明因果连接的准则。在这两点上他都错了。这些方法确实意义重大,但是它们在科学中的地位并不像他认为的那样至高无上。

在密尔对这些方法的阐述中,他涉及"只有一个事态相同"的场合和 "除了一个事态外其余的每个事态都相同"的场合。不能从字面上理解这些表述;任何两个物体无论它们多么不同,它们均具有许多相同的方面;没有两件事物只在一个方面不同——人们离北方越远,越靠近太阳;等等。我们甚至不能检査所有可能的事态,以确定是否它们只在一个方面存在差别。简言之,密尔陈述这些方法时用到了所有相关事态的集合,这些事态与待研究的因果连接有关。

但是哪些是相关事态?只用密尔方法我们不能知道哪些因素是相关的。我们必须求助于这些方法所应用的背景,此时我们已经分析了因果因素(哪些是有关的、哪些是无关的)。关于"科学的酗酒者"的讽刺表明了这个问题:什么东西使酗酒者多次喝醉?他仔细观察,第一晚他喝的是苏格兰酒和苏打,第二晚喝的是波旁酒和苏打,接着是白兰地和苏打,浪姆酒和苏打,杜松子酒和苏打。他发誓再不碰苏打!

科学的酗酒者正确运用了这些方法的规则,但是它们被证明是无效力的。因为在先行事态中的相关因素没有被揭示出来。如果酒精被确定为这所有事例中共同的一个事态,用差异法很快将苏打淘汰,这是可能的。

前面讨论求异法时举了寻找黄热病原因的英雄行为,他们的研究证实了黄热病是由于受感染的蚊子叮咬而传染的,我们现在知道了这点,正如我们知道使人醉的是酒精而非苏打。但是黄热病实验需要洞察力也需要勇气,在现实世界中的事态并没有贴有"有关的"或"无关的"标签。对蚊子叮咬的检验之前需要因果分析,以便接下来能够使用密尔方法。当我们手边有了这样的分析之后,这些方法才是十分有帮助的。但是密尔方法作为科学发现的工具并不足够。

同样,密尔方法不能构成证明的规则。因为我们总是根据关于因果事实的预先假说(刚才已经说明)而使用这些方法,并且由于我们不能考虑

所有的事态,我们的注意力将限定在那些认为可能的原因上。但是判断可能是错误的,比如,医学家起初没有考虑到脏手在传播着疾病,或者当科学家因某种原因没能将在他们面前的事态分解成恰当的单元的时候。由于应用这些方法所预设的这些分析本身,可能是不适当的或不正确的,基于这些分析上的推理同样会是错误的。这种依赖性表明密尔的方法不能用做证明。

此外,所有的密尔方法依赖观察到的相关性,然而即使观察是十分精确的,这样的观察也可能是欺骗人的。我们寻求因果规律——普遍的关系,而我们迄今拥有的机会所观察到的东西可能不会告诉我们整个事情。我们的观察数量越大,我们记载的关联为真正的规律的可能性就越大——但是无论数量多大,我们不能在没有观察的事例中确定地得到一个因果连接。

理解这里的意思的关键是:在归纳和演绎之间存在一个巨大的鸿沟。一个有效的演绎推理构成一个证明或论证;但是任何一个归纳论证至多是高度可靠的,绝不能成为证明的(demonstrative)。因而,密尔声称的他的方法是"证明的方法"(method of proof)的观点,连同他的它们是 "发现方法的全部"的观点一起,都必须被拒绝。

\subsection{密尔方法的威力}
在本章中讨论的这些方法,尽管有局限,但是它们在科学方法中处于中心地位并且确实十分有效。由于绝对不可能将所有事态考虑进去,我们必须把密尔方法与关于被考察的事态的一个或更多的因果假说一起来使用,正如我们已经看到的那样。我们通常相当不自信,因而提出不同的假说,在这些假说下不同的因素暂时地作为被研究现象的原因。作为淘汰方法的密尔法能够使我们演绎地得到:如果对先行事态的某个特定分析是正确的,那么这些因素中的一个因素不能是(或必定是)被研究的现象的原因(或部分原因)。这个演绎是有效的——但是我们再一次强调,论证的稳固性建立在先行事态的分析的正确性之上。

仅当形成的假说确实正确地识别出因果关联的事态的时候,这些方法才能产生可靠的结果;并且仅当假说被加在论证中作为一个前提的时候,结果才能通过这些方法演绎出来。现在我们能够明白这些方法提供给我们的力量的本质。它们不是发现的通路,也不是证明规则。它们是检验假说的工具。这些方法描述了受控实验的普遍方法——在所有现代科学中普遍

和不可缺少的工具。

\begin{center}
\fbox{\parbox{0.95\textwidth}{
\textbf{本节要点}
\begin{itemize}
\item \logicterm{密尔方法}的局限性:
  \begin{itemize}
  \item 无法识别哪些事态与\logicterm{因果关系}相关,需要预先的理论分析
  \item 不能从字面上理解"只有一个事态相同"或"除一个事态外都相同"
  \item 不可能考虑所有可能的事态,只能关注预设的可能原因
  \item "科学的酗酒者"例子说明方法本身不能揭示\logicemph{真正的}相关因素
  \end{itemize}
\item 作为科学发现工具的局限:
  \begin{itemize}
  \item 不能作为独立的发现\logicterm{因果关系}的工具
  \item 需要先有关于可能\logicterm{因果连接}的假说才能\logicemph{有效}应用
  \item 缺乏识别相关事态的内在机制
  \item \logicemph{有效性}取决于应用者对因果因素的预先分析
  \end{itemize}
\item 作为证明方法的局限:
  \begin{itemize}
  \item \logicterm{归纳推理}本质上无法达到\logicterm{演绎推理}的确定性
  \item 观察到的关联可能是\logicwarn{欺骗性的}或不完整的
  \item 所有\logicterm{归纳论证}最多只能是高度可靠而非证明性的
  \item 密尔\logicwarn{错误地}认为这些方法是"证明的方法"
  \end{itemize}
\item \logicterm{密尔方法}的\logicemph{真正价值}:
  \begin{itemize}
  \item 作为检验假说的\logicemph{有效}工具,而非发现或证明工具
  \item 构成现代科学中受控实验的基本方法论
  \item 与因果假说结合时能\logicemph{有效}排除不可能的原因
  \item 在科学研究中占据中心地位且实际效用显著
  \end{itemize}
\end{itemize}
}}
\end{center}
\printbibliography[heading=subbibliography,title={第12章参考文献}]