\section{对密尔方法的批评:局限性与价值的哲学反思}

\begin{logicbox}[title=引言]
本节深入评估\logicterm{密尔归纳方法}的局限与价值,进行系统的哲学反思。我们将从认识论角度分析这些方法在实际应用中面临的根本困难,包括识别相关事态的理论问题、证明\logicterm{因果关系}的逻辑挑战,以及归纳推理的本质局限。同时,我们也将重新认识\logicterm{密尔方法}作为科学研究中检验假说工具的真正价值,理解它们在现代科学实验设计中的核心地位和方法论意义。这种批判性分析将帮助我们更准确地理解密尔方法在科学认识中的适当位置。
\end{logicbox}

\subsection{密尔方法的理论基础问题:相关性识别的困境}

\begin{theorembox}[title=密尔方法的根本局限性]
密尔本人相信,上面分析的技术可以用做发现因果关系的工具,并且能够用做证明因果连接的准则。在这两点上他都错了。这些方法确实意义重大,但是它们在科学中的地位并不像他认为的那样至高无上。

\textbf{理想化假设的问题}:
在密尔对这些方法的阐述中,他涉及"只有一个事态相同"的场合和"除了一个事态外其余的每个事态都相同"的场合。不能从字面上理解这些表述;任何两个物体无论它们多么不同,它们均具有许多相同的方面;没有两件事物只在一个方面不同——人们离北方越远,越靠近太阳;等等。我们甚至不能检查所有可能的事态,以确定是否它们只在一个方面存在差别。

\textbf{相关性识别的循环问题}:
简言之,密尔陈述这些方法时用到了所有相关事态的集合,这些事态与待研究的因果连接有关。但是哪些是相关事态?只用密尔方法我们不能知道哪些因素是相关的。我们必须求助于这些方法所应用的背景,此时我们已经分析了因果因素(哪些是有关的、哪些是无关的)。

\textbf{认识论的循环性}:
这里存在一个深层的认识论问题:为了应用密尔方法,我们需要预先知道哪些因素是相关的;但确定因素的相关性正是我们希望通过密尔方法来解决的问题。这种循环性表明密尔方法不能作为独立的发现工具。
\end{theorembox}

\begin{examplebox}[title="科学的酗酒者":相关性识别失败的经典案例]
关于"科学的酗酒者"的讽刺表明了这个问题:什么东西使酗酒者多次喝醉?他仔细观察,第一晚他喝的是苏格兰酒和苏打,第二晚喝的是波旁酒和苏打,接着是白兰地和苏打,朗姆酒和苏打,杜松子酒和苏打。他发誓再不碰苏打!

\textbf{案例分析}:
\begin{itemize}
\item \textbf{方法应用}:酗酒者正确应用了求同法,寻找所有醉酒事例的共同因素
\item \textbf{逻辑错误}:将苏打水而非酒精确定为共同因素
\item \textbf{根本问题}:缺乏关于酒精作用的背景知识
\item \textbf{理论意义}:说明密尔方法依赖于预先的理论框架
\end{itemize}

\textbf{深层启示}:
这个案例揭示了密尔方法的一个根本缺陷:它们无法自动识别真正相关的因素。在现实世界中,事态并没有贴有"有关的"或"无关的"标签。成功应用密尔方法需要研究者具备相关的理论知识和洞察力。
\end{examplebox}

\begin{examplebox}[title=黄热病研究:理论洞察与方法应用的结合]
前面讨论求异法时举了寻找黄热病原因的英雄行为,他们的研究证实了黄热病是由于受感染的蚊子叮咬而传染的,我们现在知道了这点,正如我们知道使人醉的是酒精而非苏打。

\textbf{成功的关键因素}:
\begin{itemize}
\item \textbf{理论洞察}:研究者具备了关于疾病传播的理论框架
\item \textbf{科学勇气}:愿意进行危险的人体实验
\item \textbf{系统设计}:精心设计的对照实验
\item \textbf{因果分析}:对蚊子叮咬的检验之前需要因果分析
\end{itemize}

\textbf{方法论启示}:
但是黄热病实验需要洞察力也需要勇气,在现实世界中的事态并没有贴有"有关的"或"无关的"标签。对蚊子叮咬的检验之前需要因果分析,以便接下来能够使用密尔方法。当我们手边有了这样的分析之后,这些方法才是十分有帮助的。

\textbf{结论}:
但是密尔方法作为科学发现的工具并不足够。它们需要与理论洞察、背景知识和科学直觉相结合才能发挥作用。
\end{examplebox}

\subsection{作为证明方法的根本缺陷:归纳推理的本质局限}

\begin{theorembox}[title=密尔方法作为证明方法的根本缺陷]
同样,密尔方法不能构成证明的规则。

\textbf{1. 预设假说的依赖性}:
因为我们总是根据关于因果事实的预先假说(刚才已经说明)而使用这些方法,并且由于我们不能考虑所有的事态,我们的注意力将限定在那些认为可能的原因上。

\textbf{2. 判断错误的可能性}:
但是判断可能是错误的,比如,医学家起初没有考虑到脏手在传播着疾病,或者当科学家因某种原因没能将在他们面前的事态分解成恰当的单元的时候。由于应用这些方法所预设的这些分析本身,可能是不适当的或不正确的,基于这些分析上的推理同样会是错误的。

\textbf{3. 观察的欺骗性}:
此外,所有的密尔方法依赖观察到的相关性,然而即使观察是十分精确的,这样的观察也可能是欺骗人的。我们寻求因果规律——普遍的关系,而我们迄今拥有的机会所观察到的东西可能不会告诉我们整个事情。

\textbf{4. 归纳推理的本质局限}:
我们的观察数量越大,我们记载的关联为真正的规律的可能性就越大——但是无论数量多大,我们不能在没有观察的事例中确定地得到一个因果连接。

\textbf{5. 归纳与演绎的根本鸿沟}:
理解这里的意思的关键是:在归纳和演绎之间存在一个巨大的鸿沟。一个有效的演绎推理构成一个证明或论证;但是任何一个归纳论证至多是高度可靠的,绝不能成为证明的(demonstrative)。

\textbf{结论}:
因而,密尔声称的他的方法是"证明的方法"(method of proof)的观点,连同他的它们是"发现方法的全部"的观点一起,都必须被拒绝。
\end{theorembox}

\begin{examplebox}[title=医学史中的认识局限:塞麦尔维斯的悲剧]
19世纪匈牙利医生塞麦尔维斯发现,医生洗手可以大大降低产褥热的发病率。他通过统计数据证明了这一点,但当时的医学界拒绝接受这个发现,因为它与当时占主导地位的疾病理论不符。

\textbf{案例启示}:
\begin{itemize}
\item \textbf{观察的准确性}:塞麦尔维斯的观察和统计是准确的
\item \textbf{理论的局限}:当时缺乏细菌理论的支持
\item \textbf{接受的困难}:即使有了正确的观察,理论框架的缺失使得发现难以被接受
\item \textbf{方法论意义}:说明密尔方法的有效性依赖于更广泛的理论背景
\end{itemize}

这个案例说明,即使密尔方法得出了正确的结论,如果缺乏适当的理论框架,这些结论也可能被拒绝或误解。
\end{examplebox}

\subsection{密尔方法的真正价值:假说检验工具的重新定位}

\begin{theorembox}[title=密尔方法的真正威力:假说检验的逻辑工具]
在本章中讨论的这些方法,尽管有局限,但是它们在科学方法中处于中心地位并且确实十分有效。

\textbf{与假说结合的必要性}:
由于绝对不可能将所有事态考虑进去,我们必须把密尔方法与关于被考察的事态的一个或更多的因果假说一起来使用,正如我们已经看到的那样。我们通常相当不自信,因而提出不同的假说,在这些假说下不同的因素暂时地作为被研究现象的原因。

\textbf{演绎推理的有效性}:
作为淘汰方法的密尔法能够使我们演绎地得到:如果对先行事态的某个特定分析是正确的,那么这些因素中的一个因素不能是(或必定是)被研究的现象的原因(或部分原因)。这个演绎是有效的——但是我们再一次强调,论证的稳固性建立在先行事态的分析的正确性之上。

\textbf{可靠性的条件}:
仅当形成的假说确实正确地识别出因果关联的事态的时候,这些方法才能产生可靠的结果;并且仅当假说被加在论证中作为一个前提的时候,结果才能通过这些方法演绎出来。

\textbf{方法的本质重新定义}:
现在我们能够明白这些方法提供给我们的力量的本质。它们不是发现的通路,也不是证明规则。它们是检验假说的工具。

\textbf{现代科学的基础}:
这些方法描述了受控实验的普遍方法——在所有现代科学中普遍和不可缺少的工具。
\end{theorembox}

\begin{examplebox}[title=现代科学实验设计中的密尔方法]
\textbf{药物试验中的应用}:
在现代药物试验中,密尔方法的原理被系统地应用:
\begin{itemize}
\item \textbf{求异法}:对照组与实验组的设计
\item \textbf{求同法}:寻找所有有效案例的共同特征
\item \textbf{求同求异并用法}:双盲随机对照试验
\item \textbf{共变法}:剂量-效应关系的研究
\end{itemize}

\textbf{方法论价值}:
\begin{itemize}
\item \textbf{假说检验}:不是用来发现新假说,而是检验既有假说
\item \textbf{排除法}:通过排除不可能的因素来缩小可能性范围
\item \textbf{控制变量}:为现代实验设计提供逻辑基础
\item \textbf{因果推断}:在有限条件下进行可靠的因果推断
\end{itemize}

\textbf{现代意义}:
密尔方法为现代科学的实验方法论奠定了逻辑基础,虽然它们不能独立发现或证明因果关系,但作为假说检验工具具有不可替代的价值。
\end{examplebox}

\begin{center}
\fbox{\parbox{0.95\textwidth}{
\textbf{本节要点}
\begin{itemize}
\item \textbf{密尔方法的理论基础问题}:
  \begin{itemize}
  \item \textbf{理想化假设的困难}:不能从字面上理解"只有一个事态相同"或"除一个事态外都相同"
  \item \textbf{相关性识别的循环问题}:为了应用密尔方法需要预先知道哪些因素相关,但确定相关性正是方法要解决的问题
  \item \textbf{认识论的循环性}:存在深层的认识论循环,表明密尔方法不能作为独立的发现工具
  \item \textbf{"科学的酗酒者"案例}:说明方法无法自动识别真正相关的因素,需要理论知识和洞察力
  \item \textbf{黄热病研究启示}:成功应用需要理论洞察、科学勇气和系统设计的结合
  \end{itemize}
\item \textbf{作为证明方法的根本缺陷}:
  \begin{itemize}
  \item \textbf{预设假说的依赖性}:总是根据预先假说使用方法,注意力限定在认为可能的原因上
  \item \textbf{判断错误的可能性}:预设分析可能不适当或不正确,导致基于此的推理错误
  \item \textbf{观察的欺骗性}:即使精确观察也可能欺骗人,不能告诉我们完整的情况
  \item \textbf{归纳推理的本质局限}:无论观察数量多大,都不能确定地得到因果连接
  \item \textbf{归纳与演绎的根本鸿沟}:归纳论证至多高度可靠,绝不能成为证明性的
  \item \textbf{塞麦尔维斯案例}:说明即使正确观察,缺乏理论框架也难以被接受
  \end{itemize}
\item \textbf{密尔方法的真正价值重新定位}:
  \begin{itemize}
  \item \textbf{假说检验工具}:不是发现通路或证明规则,而是检验假说的有效工具
  \item \textbf{与假说结合的必要性}:必须与因果假说一起使用才能发挥作用
  \item \textbf{演绎推理的有效性}:在正确分析基础上能够进行有效的演绎推理
  \item \textbf{现代科学实验基础}:描述了受控实验的普遍方法,是现代科学不可缺少的工具
  \item \textbf{药物试验应用}:在现代药物试验中系统应用各种密尔方法原理
  \end{itemize}
\item \textbf{哲学反思的意义}:
  \begin{itemize}
  \item \textbf{方法论的准确定位}:帮助我们准确理解密尔方法在科学认识中的适当位置
  \item \textbf{科学方法的理解}:深化对科学方法本质和局限性的理解
  \item \textbf{批判性思维}:培养对方法论的批判性思维和理性反思能力
  \item \textbf{现代科学基础}:为理解现代科学实验设计提供重要的逻辑基础
  \end{itemize}
\end{itemize}
}}
\end{center}
% 参考文献将在主文档末尾统一显示