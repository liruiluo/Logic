\section{因果连接:基本概念}

\begin{quotation}
本节探讨因果关系的基本概念及其在逻辑推理中的重要性。我们将分析"原因"的多种含义,区分必要条件与充分条件,考察因果律与自然齐一性的关系,并介绍简单枚举归纳法作为建立因果关系的基本方法。通过理解这些概念,我们将能够更有效地分析和评价各种归纳论证中的因果推理。
\end{quotation}

\subsection{"原因"的意义}

为了对环境进行控制性操作,我们必须拥有某种因果连接的知识。例如,为了治疗某种疾病,医生必须知道它的原因;并且,他们应当了解他们所用药物的后果(包括副作用)。因和果之间的关系其重要性非同一般。然而,这种关系因为"原因"一词有多种含义而易于混淆。因而,我们先区分这些含义。

在对自然的研究中一个基本的公设是,只有在确定的条件下事件才能发生。人们习惯于区分事件发生的必要条件和充分条件。一个特定事件发生的必要条件是指,在缺乏它的情况下,该事件不能发生。例如,具有氧气是燃烧能够发生的必要条件:如果燃烧发生,必须具有氧气,因为在缺乏氧气的情况下便没有燃烧。

尽管具有氧气是一个必要条件,但它不是燃烧能够发生的充分条件。一个事件能够发生的充分条件是,在它出现的情况下事件必定发生。因为在有氧气的情况下也可能不发生燃烧,所以,出现氧气不是燃烧的充分条件。另一方面,对几乎每一种物质而言,都存在某个温度范围,在该温度范围里具有氧气是该物质燃烧的充分条件。明显的是,一个事件的发生可能有多个必要条件,并且这些必要条件均包含在充分条件里。

必要和充分条件的区分在法律论证中经常起关键作用。在美国高等法院中,一名法官最近争辩说,州立基金被用于资助宗教协会时必须满足两个条件:必须是公平的一一它的发放是中立的,而不对任何一个宗教有所偏爱;必须是间接的——因为宪法禁止宗教协会从政府直接获得资助。这是两个早期得到认同的资助程序,受到经费资助的人自由地将之用于宗教组织之中。对于这两个条件,该法官写道:

在每个资助中,资助是广泛的和中立的,这个事实是资助程序中的必要条件。但是公正的意义失去了。在每个资助情况中,我们没有说,该条件就是充分的,或者说决定性的。情况完全相反。这些资助中对我们决策起决定作用的是这样的事实:资助是间接的;资助到达宗教组织完全是受资助者的完全独立的和私人的选择。

这个法官做出这样的区别,是因为在手头案子中(这是关于一所州立大学拒绝了为一个学生宗教社团付印刷费的案子),争议中的州资助即使公平地给予,它也是直接的,因此他认为是不允许的。在这名法官看来,资助的接受程序是两条必要条件,其中能够满足的只有一条。

"原因"有时是在"必要条件"的意义上使用,而有时是在"充分条件"的意义上使用。当手边的问题是要淘汰不受欢迎的现象时,它更多地是在"必要条件"的意义上使用。为了淘汰某个现象,人们只要找到某个对该现象的存在为必需的条件,然后将该条件淘汰。医生努力寻找何种微生物是某个疾病的"原因",以便开出杀灭那些微生物的药物,从而治愈该疾病。那些微生物被认为是该疾病的原因,是说它们是疾病的必要条件一一因为如果没有它们便不会有该疾病。

忽视这种意义的原因将导致无谓的争论。某种动物行为的真正原因是它的基因还是环境?当然,大多数情况下两者都起作用;当两者都不能独自解释该行为的时候,两者都是本质的。在鸣鸟群里,通常只有雄鸣鸟唱歌。当科学家使幼小的雄鸣鸟不再产生睾丸激素后,它们不再能唱歌。但是,如果它们在其幼年的某个阶段没有听到周边其他鸣鸟的鸣唱,它们也不能唱歌。一个雄鸟听到一首歌,该歌开启了一个用睾丸激素以唱歌的方式建立脑神经的过程。本性和养育两者均是鸟能够唱歌的必要条件。\cite{marler1991}

当我们对某个希望发生的事情感兴趣的时候(而不是淘汰不希望的事情),我们是在"充分条件"的意义上使用"原因"一词的。冶金专家的目标是发现什么使金属合金具有更大的强度,如果我们找到了这样的一个热处理和冷处理的复合过程(该过程使得金属具有我们希望的结果),我们说,这样的一个过程是合金强度增高的原因。

"原因"一词有另外一个普遍的但不精确的用法,该用法与充分条件的含义密切相关。一给定现象与某些后果关联,它可能便是原因。例如,我们断定"吸烟导致癌症"。当我们这样说时,可以肯定的是,我们并没有说吸烟是癌症的必要条件。因为我们知道许多癌症是在完全没有吸烟的情况下得的。同样不能说吸烟必定产生癌症,因为可能的是,某些人的长期吸烟的习惯并没有带来癌症后果。但是,吸烟,与某些生物环境相结合,在癌症的发展中频繁地发挥作用,以至于我们合理地认为吸烟为癌症的一个"原因"。

这产生了"原因"的另外一个用法:作为某个现象发生过程中的关键因素或常常是关键因素。假定一家保险公司派遣调查员弄清一场神秘火灾的原因。如果调查员报告说火灾是由空气中的氧气所致,那么调查员的工作将不保。尽管他们是对的一一在必要条件的含义上。因为如果不存在氧气,火灾便不可能发生。然而,保险公司派遣他们去调查,不是打算为了弄清该种含义上的原因。保险公司也不对充分条件感兴趣。如果经过几个星期后调査员汇报说,尽管他们已经证明火是由投保的客户有意点燃的,但他们还不能够知道所有必要条件,因而仍然不能确定(充分条件含义上的)原因,此时,公司将打电话给他们,告诉他们别再浪费时间和金钱。保险公司是在另外一种意义上使用"原因"一词:他们希望查找的是,在现有的条件之下造成该事件出现或不出现的差别的事件或行为是什么。

\subsection{遥远与最近的原因}

我们对第三种含义的原因做两个区分。传统上人们将它们称为遥远的(remote)和最近的(proximate)原因。几个事件组成的一个因果序列或链条:$A$ 引起 $B, B$ 引起 $C, C$ 引起 $D, D$ 引起 $E$ ,此时我们将 $E$ 称为先行事件的结果。其中最近的即 $D$ ,为 $E$ 的最近的原因,而其他的为 $E$ 越来越遥远的原因:$A$ 比 $B$ 遥远,$B$ 比 $C$ 遥远。尽管如此,由于因果链条的连接数量,在时间上与结果十分接近的原因可能在距离上是遥远的。下述是对出现在1996年的一件事情的真实解释:

事件的导火索起因于两年前六月份的一个早晨。巴西经过了一整夜的霜冻后,一个政府官员宣布減少计划中的咖啡生产产量。该消息立刻传到芝加哥贸易部,该处咖啡的期货价格立即攀升。大豆和其他物品的商人立刻抬高价格,导致物品价格指数上升。这一切均记录在商人们的计算机屏幕上。这些商人分布在几乎 200 个华尔街公司里,他们将通货膨胀情况汇报给他们的合约一贸易伙伴,这些伙伴开始抛售合约,这导致合约价格下降,合约价格下降导致合约产量上升,合约产量上升给利息率的升高增加了向上的力量,这个力量造成股票价格下降。在巴西公告发出和华尔街股票波动之间的时间间隔不会超过 10 分钟。\cite{nasar1996}

一件研究表明,"教育是健康最重要的关联因素。接受较多的教育的人变得更为有所了解;他们了解医学技术,医疗,保险以及保健服务系统的重要性。他们更为有能力地从医疗系统获得有价值的服务...尽管遭遇同样的疾病,接受较少教育的人往往接受较差的医疗。"\cite{kitagawa1973} 但是上大学不是健康的最近的原因,无知也不是疾病的最近的原因。落后的教育是在该因果链条中的一个环节,它往往造成对疾病过程不恰当的理解,因而,较好的医学后果所需的生活方式难以建立。因此人们普遍并正确地观察到,贫困,它广泛地对教育产生影响,它是缺乏健康的一个"根本原因"一一个遥远的而不是最近的原因。

我们已经看到,"原因"一词的含义存在几种。我们仅能够在"必要条件"的含义上合法地从结果中推出原因。并且,我们仅能够在"充分条件"的含义上合法地从原因中推出结果。当我们从原因推论到结果并且从结果推论到原因时,原因必定是在既充分又必要条件的意义上使用的。在这种用法中,原因等同于充分条件,而充分条件被认为是所有必要条件的联合。应当清楚的是,不存在符合该词的所有不同用法的单个"原因"定义。

\subsection{因果律和自然的齐一性}

但是"原因"一词的每一种用法,无论是在日常生活中的还是在科学中的,都与下述原则相关,或预设了下述原则:原因和结果齐一地(uniformly)相连。我们说,一个特定事态造成了一个特定结果,即是说该类型的其他事态(在产生该事态充分类似的条件下)将造成与先前结果同种类型的结果。换句话说,同类原因导致同类结果。我们今天使用的"原因"一词的部分意义是,一个原因产生一个结果的每一次出现,都是普遍因果律——如此的事态总是伴随着如此的现象一一的一个实例或一个事例。于是,如果在另外的情形下出现了与事态 $C$ 同类的事态,但是结果 $\boldsymbol{E}$ 并不发生,此时我们不认为事态 $C$ 是在一个特定场合下结果 $E$ 的原因。

因为特定事态是特定现象的原因的每一个断定意味着存在某个因果律,每一个因果连接的断定都包含与普遍性(generality)有关的一个关键成分。因果律——当我们使用该术语的时候——断定,如此这般的事态下恒常地伴随着一个特定种类的现象,而无论该事态发生于何时何地。但是我们如何知道这样普遍性的真理呢?

因果关系不是纯粹逻辑的或演绎的,它不能被任何先验的论证所发现。因果律只能经验地或后验地(即诉诸经验)发现。但是我们的经验总是与特定情形、特定现象以及现象的特定次序有关。我们能够观察到一个特定事态(比如 $C$ )下的几个事例,我们观察到的事例也能够被一个特定种类现象(如 $P$ )的一个事例所伴随。但是我们末来能够经历的仅仅是世界上事态 $C$ 中的一些事例,这些观察能够展示给我们的仅仅是 $P$ 伴随着 $C$ 的一些事例。然而,我们的目标是建立一个普遍的因果关系。我们如何能够从我们经历的特定事例中,得到 $C$ 的所有场合下都有 $P$ 这样普遍性的命题( $C$ 引起 $P$ )?

\subsection{简单枚举归纳法}

从特定经验事实中得到一般或普遍命题的过程被称做归纳概括。从三张蓝色石蕊试纸放到酸中都变红的前提中,我们或者会得到一个特定结论一一将第四张蓝色石蕊试纸放到酸中它将发生什么样的现象,或者会得到一个普遍结论-每一张蓝色石蕊试纸放到酸中将发生什么。如果我们得到第一个,我们就使用了一个类比论证;如果是第二个,则为一个归纳论证。这两个论证类型的结构在下面得到分析。前提反映的是两个属性(或情形或现象)共同发生的事例,由类比我们可以推得,在具有一个属性的其他事例中也会出现另外的属性;而由归纳概括我们能够推得,一个属性出现其中的每一个事例将同时也是另外属性的事例。这种形式的归纳概括就是简单枚举归纳法。简单枚举归纳法非常类似于类比论证,所不同的只是它形成的结论更为普遍。

\begin{displayquote}
现象 $E$ 的事例 1 伴随有事态 $C$\\
现象 $E$ 的事例2伴随有事态 $C$\\
现象 $E$ 的事例 3 伴随有事态 $C$
\end{displayquote}

因而现象 $E$ 的每个事例都伴随有事态 $C$ 。

我们经常用简单枚举法建立因果连接。当一种现象的许多事例恒常地伴随着一特定类型的事态的时候,我们自然地得出在它们之间存在一个因果关系。将蓝色石惢试纸放进酸中的情形在所有观察中都伴随有试纸变红现象,我们由简单枚举法得到,将蓝色石蕊试纸放进酸中是它变红的原因。在这样论证中的类比特征相当明显。

由于简单枚举法和类比论证之间有很大的类似性,类似的评价标准都适合它们。某些简单枚举法论证能够比其他的论证建立较高盖然度的结论。举出的事例数越多,结论成真的概率就越高。伴随着事态 $C$ 的不同事例或场合,往往被称做断定 $C$ 引起 $E$ 的因果律的确证事例。确证事例数越多,若其他事态不变的话,因果律为真的概率越高。于是,用于类比论证的第一个标准可直接应用于简单枚举归纳法论证。

在历史报告中简单枚举法可以为一个因果关系的建立提供论证基础。举一个例子:对某个个体或群体与其财产或权利进行暂时性的强制分离的司法行为,被称为财产和公民权剥夺法案;熟知的是,当政治权力的钟摆发生摆动时,该司法行为对该法案的鼓吹者也会造成危险;今天的原告明天会成受害人;卡那尔文伯爵为了指控上议院针对托马斯•奥斯本的这种法案,在1678年用下面的枚举法阐明其观点:

大人们,从不少的英国历史中我了解到这些检举的危害以及检举人的悲惨命运。我将追溯到伊丽莎白女王统治的晚期而不是更远,当时埃塞克斯伯爵被瓦尔特•拉莱爵士所检举,大人们,你们很清楚拉莱发生了什么。培根大人检举了瓦尔特•拉莱爵士。大人们,你们清楚培根大人发生了什么。巴金汗侯爵检举了培根大人。大人们,对巴金汗侯爵的命运,你们是清楚的。托马斯•文特沃斯爵士然后是斯特拉福特伯爵,检举了巴金汗侯爵。你们都知道斯特拉福特伯爵的命运。哈瑞•凡恩爵士检举了斯特拉福特伯爵,大人们,你们知道哈瑞•凡恩爵士如何了,海德大臣检举了他。你们清楚海德大臣的命运,托马斯•奥斯本以及现在的旦比伯爵,检举了海德大臣。

旦比伯爵的命运将如何呢,大人们最好能够告诉我。但是让我们看一下,胆敢将旦比伯爵赶下台的人,他的命运将如何。\cite{roberts1966}

事例的重复尽管有修辞效果,但它没有提供确定性的论证。恶意指控和随后的垮台之间存在因果关系的结论,诉诸六个确证事例。但是这些事例的本性阻碍了将真实因果律的确证事例和仅仅是历史巧合之间区别开来。

这个困难的核心是:简单枚举法对提出的因果律的例外没有解释,而且不可能有解释。任何断言的因果律都会被一个反例所推翻,因为,任何一个反例表明,所谓的一个"规律"不是真正普遍的。例外否证了该规则。因为一个例外(或"反例")或者是这样一个情况:人们发现了所断言的原因,而断言的结果并没有伴随(在该历史案例中,指控提案的提出者没有发生类似的命运);或者是这样的情况:结果发生了,但所断言的原因没有发生。(如果用前面的图式)$C$ 发生而 $E$ 不发生,或 $E$ 发生而 $C$ 没有发生。但在一个简单枚举论证中,这两个情况中的任何一个都是无效的;在这样论证中唯一合法的前提是断言的原因和断言的结果两者都出现的事例报告。

如果我们限定我们归纳论证的视野,我们将不去寻找甚至于不去注意那些可能发现的否定的或不确证的事例,这是简单枚举论证的一个严重缺陷。正因为这一点,简单枚举归纳法尽管在因果律的建立过程中成果丰硕并且具有价值,但它不适合检验因果律。然而这样的检验是至关重要的。为了进行检验,我们必须依赖于其他类型的归纳论证,下面我们就转向它们。 

\begin{center}
\fbox{\parbox{0.95\textwidth}{
\textbf{本节要点}
\begin{itemize}
\item \textbf{"原因"的多种含义}:
  \begin{itemize}
  \item 必要条件:缺少则事件不能发生(如燃烧需要氧气)
  \item 充分条件:出现则事件必定发生
  \item 关键因素:在现有条件下导致事件发生的差异性因素
  \end{itemize}
\item \textbf{因果关系的类型区分}:
  \begin{itemize}
  \item 遥远的原因vs最近的原因:在因果链条中的位置
  \item 根本原因:处于因果链条起始位置的关键因素
  \item 各类原因在不同语境下的理解和使用
  \end{itemize}
\item \textbf{因果律与自然齐一性}:
  \begin{itemize}
  \item 同类原因导致同类结果的普遍性假设
  \item 因果关系只能通过经验后验发现,而非先验推理
  \item 从特定事例推断普遍规律的归纳问题
  \end{itemize}
\item \textbf{简单枚举归纳法}:
  \begin{itemize}
  \item 从观察到的事例归纳出普遍因果关系
  \item 与类比论证的相似性及区别
  \item 局限性:无法处理反例,不适合严格检验因果律
  \end{itemize}
\end{itemize}
}}
\end{center} 