\subsection{P1.复杂问语(Complex Question)}

\textbf{复杂问语谬误}发生在提问的方式预设了某些未被证实的假设为真的情况。这种谬误的目的是使听者或读者接受事先被预设的假设,而无需对这些假设进行质疑或分析。当一个问题含有一个或多个预设,而这些预设尚未被证明或接受时,就犯了复杂问语谬误。

\paragraph{经典的复杂问语例子}
最有名的复杂问语例子是:"你什么时候停止殴打你的妻子?"无论对方回答"昨天"、"上个月"或者"去年",他都会承认自己曾经殴打妻子。即使他回答"我从未殴打我的妻子",这种回答也会显得不自然和可疑。问题预设了"他曾经殴打妻子"这一命题,而这一命题正是需要被证实的。

\paragraph{司法询问中的复杂问语}
在法庭审讯中,律师常常试图诱使证人回答复杂问题。例如,原告的律师可能问被告:"你在事故发生前喝了多少酒?"这种问题预设被告事故前喝了酒。如果证人回答说他没喝酒,律师可以说:"我没问你是否喝了酒,我问你喝了多少。"或者会问:"你多久打你的妻子一次?"而不是先问:"你是否曾经打过你的妻子?"这种询问方式,预设了被告确实曾经殴打过自己的妻子。

\paragraph{政治辩论中的预设问题}
政治辩论中的问题经常预设争议性的假设为真。例如,可能会问:"你认为哪些人应该承担我们国家衰退的责任?"这种问题预设了"我们的国家正在衰退",而这个假设可能本身就是争议的焦点。类似地,针对在任官员的问题如"你准备采取什么措施来避免再次犯重大错误?"也预设了官员已经犯了重大错误。

\paragraph{识别和避免复杂问语}
要避免被复杂问语所误导,我们应当始终审视问题中的隐含假设,而不是立即着手回答问题。复杂问题往往可以分解为更基本的问题,例如:"你有妻子吗?如果有,你是否曾经殴打过她?如果是,你什么时候停止这样做的?"这种分解可以使人避免无意中认可未经证实的预设。

复杂问语的效力来自于它的心理压力——它迫使回答者处于不利的防御位置。在逻辑学中,复杂问语谬误属于\textbf{预设谬误},因为它的错误在于预设了未经证实的命题为真。识别这类谬误的关键是识别问题中所隐含的所有假设,并对其真实性提出质疑。 