\section*{4. 2 相干谬误}
当一个论证所依据的前提与其结论不相干因而不可能确立结论之真时,其所犯的就是相干谬误。或许,称之为不相干谬误更贴切,但是, (在实际论证中)这种论证的前提常常在心理上与结论是相干的,而正是这种相干性使得它们似乎正确和有说服力。心理的相干怎么会与逻辑的相干相混淆,可以用我们在第 2 章讨论的语言的不同用法进行部分阐释;这些混淆的机制在随后的分析中将变得更加清晰。

很多谬误传统上都有个拉丁名称;有些拉丁名称,像 ad hominem (人身攻击),已经进入普通英语语言之中。我们在这里将既使用拉丁名称又使用英语名称。 