\subsection{A3.重读}

\begin{theorembox}[title=重读谬误的定义]
\logicwarn{基本机制}:当论证的意义变化源于对其词汇或组成部分的强调的变动时,该论证就可以证明是欺骗性的和无效的。

\logicterm{重读谬误}:若前提的明显意义依赖于一个可能的强调,但是,得出的结论却依赖于对相同词汇不同的重读意义,这时就犯了\logicterm{重读}(accent)谬误。

\logicemph{谬误特征}:
\begin{itemize}
  \item 通过改变语音重音或视觉强调来改变意义
  \item 在前提和结论中使用不同的强调方式
  \item 利用强调的变化来误导听众或读者
  \item 破坏了论证的逻辑一致性
\end{itemize}

\logicwarn{表现形式}:
\begin{itemize}
  \item 口语中的语音重音变化
  \item 书面语中的字体大小、颜色、位置变化
  \item 标点符号的不同使用
  \item 上下文环境的操纵
\end{itemize}
\end{theorembox}

\paragraph{重音与语气的影响}
\begin{examplebox}[title=重音与语气的影响]
\logicwarn{语音重音的例子}:作为示例,请考虑我们可以把不同的意义给予如下陈述:

\begin{quote}
我们不应当说朋友的坏话(We should not speak ill of our friends)。
\end{quote}

\logicemph{不同重音的含义}:
\begin{itemize}
  \item 强调"我们":暗示别人可以说朋友的坏话
  \item 强调"不应当":强调道德义务
  \item 强调"朋友":暗示可以说非朋友的坏话
  \item 强调"坏话":暗示可以说朋友的好话
\end{itemize}

\logicwarn{印刷媒体中的误导}:在印刷字体及图片方面,有很多伎俩常常是通过强调某处而起误导之效。

\logicemph{新闻标题的操纵}:
\begin{itemize}
  \item 出现在新闻报道标题中的大号字敏感词汇,故意向那些匆匆浏览的人暗示错误的结论
  \item 该标题后面却很可能用其他词汇以很小的字来加以限制
  \item 为避免在看新闻报道或在签订合同时被欺骗,我们力劝人们注意"小字印刷"
\end{itemize}

\logicwarn{政治宣传中的应用}:在政治宣传中,特别是在声称所谓事实报道中:
\begin{itemize}
  \item 选择令人误解的敏感标题
  \item 选择使用部分省略的图片
  \item 都是对重读谬误的精心使用,力图使读者得出宣传者明知为假的结论
\end{itemize}

\logicemph{歪曲与谎言的区别}:解说可能不是彻底的谎言,但它也可以利用故意或虚假的重读方式来歪曲事实。
\end{examplebox}

\paragraph{广告中的重读谬误}
\begin{examplebox}[title=广告中的重读谬误]
\logicwarn{广告中的普遍现象}:在广告中,这样做的也很多。

\logicemph{价格广告的伎俩}:
\begin{itemize}
  \item 非常低的价格往往以非常大的字出现
  \item 而后面却跟随着字体极小的"以及完全说明"
\end{itemize}

\logicwarn{机票广告的例子}:
\begin{itemize}
  \item 飞机票价打折的通告后面都跟有一个星号
  \item 以远远的一个脚注说明该价格仅仅可用于提前三个月预订星期四的飞行航班
  \item 或可能还会有其他"适用限制"
\end{itemize}

\logicemph{商品广告的策略}:
\begin{itemize}
  \item 名牌昂贵商品都以非常低的价格做广告
  \item 在广告某处附有一个小注解"所列价格存货数量有限"
  \item 读者被吸引到商店,但可能以广告价格买不到商品
\end{itemize}

\logicwarn{谬误的界定}:
\begin{itemize}
  \item 重读语段本身并不是严格谬误
  \item 源于重读的语段解释,当它依赖一个非常可疑的结论暗示时,即当其采用令人误解的重读来解释时,重读语段就变成了谬误
  \item 例如:飞机票或品牌商品可以按照所列价格优先购买
\end{itemize}
\end{examplebox}

\paragraph{通过位置操纵的重读}
\begin{examplebox}[title=通过位置操纵的重读]
\logicwarn{位置操纵的威力}:甚至字面上为真的语段,也可以通过操纵其位置而以重读来欺骗人。

\logicemph{船长与助手的故事}:
\begin{itemize}
  \item \logicwarn{背景}:一位船长厌恶他的首席助手上班时再三喝醉
  \item \logicemph{船长的记录}:在该船的航行日记上,他几乎每天都记上:"助手今天喝醉了。"
  \item \logicwarn{助手的报复}:愤怒的助手进行报复。一天,船长病了,助手就自己保管日志
  \item \logicemph{助手的记录}:他在上面记着:"船长今天清醒了。"
\end{itemize}

\logicwarn{重读效果分析}:
\begin{itemize}
  \item 船长的记录暗示助手经常喝醉(通过频繁记录)
  \item 助手的记录暗示船长经常不清醒(通过强调"今天清醒了")
  \item 两个记录在字面上都可能是真实的
  \item 但通过上下文和频率的操纵,传达了误导性的印象
\end{itemize}

\logicemph{教训}:这个例子说明了语境和时机在传达信息中的重要作用,以及如何通过巧妙的位置安排来改变信息的含义。
\end{examplebox}