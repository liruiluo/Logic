\section{含混谬误}

\begin{quotation}
\textit{含混谬误是由词语或短语的意义变化而导致的推理错误,这类谬误可能出于疏忽,也可能是刻意为之。识别和理解这类谬误有助于我们避免在论证中被词语的多义性所误导。}
\end{quotation}

由于用心不专或故意操作,在论证过程中,词或短语的意义可能会变化。一个词项在前提中可能具有一种意义,但是在结论中却是另一种相当不同的意义。当推论依赖这样的变化时,当然就是谬误。这种错误称做\textbf{含混谬误},有时或称为\textbf{诡论}(sophisms)。故意使用这样的方法常常是粗糙的和易于发现的,但是,有时(虽并非经常)这种含混是隐蔽的、难以把捉的。我们在下面区分出它的五种类型。 