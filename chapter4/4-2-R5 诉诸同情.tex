\subsection{R5.诉诸同情(The Appeal to Pity:Argument Ad Misericordiam)}

\textbf{诉诸同情}(misericordiam的字面意思是"同情心")可以看做是诉诸情感的一种特殊情况,其中听众的利他主义和怜悯之心是其所诉诸的特殊情感。当论证试图利用人们的怜悯和同情心而非提供相关证据来支持结论时,就犯了\textbf{诉诸同情谬误}。

\paragraph{法庭辩护中的同情诉求}
在法庭上,原告的律师为寻求伤害赔偿金,常常以某种极其悲苦的方式安排展示委托人的伤残情况。${ }^{[15]}$在刑事审判中,虽然陪审团对被指控者无论有罪或无辜都不应带有同情,但是高效的辩护律师常常设法激起陪审团的同情,有时诉诸同情被做得神不知鬼不觉。

\paragraph{苏格拉底对诉诸同情的批判}
在其雅典审判中,苏格拉底轻蔑地提到其他被告人由他们的子女和家人陪伴出现在陪审团面前,以寻求激起怜悯之情而免除责任。苏格拉底接着说道:\\
$\cdots \cdots$我,生命处于危险中的我,将不会做任何这种事情。这种对比可能出现在他(陪审团成员)的头脑中,他可能反对我,愤怒地投票,因为他为此对我不高兴。现在,如果你们之中有这样的人,注意,我不是说确有,那么我可以诚实地回答他:我的朋友,我是人,和别人一样,一个有血有肉的生物,不是像荷马(Homer)所说的那种"木石之躯";我也有一个家庭,有孩子,噢,雅典人啊,有三个儿子,一个几乎成人,另两个还年幼;但是,我将不带他们任何人到这里以请求你们判我无罪。${ }^{[16]}$

\paragraph{荒谬的同情诉求案例}
有很多方法可以拨动心弦,而且事实上,也都为人们一再使用。在一次指控一个年轻人用斧头杀害了他父母的审判中,出现了最荒谬的诉诸同情论证:面对其罪恶的大量证据,他请求宽大处理,理由是他现在成了一个\textbf{孤儿}。 