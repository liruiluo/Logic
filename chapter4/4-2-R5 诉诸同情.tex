\subsection{R5.诉诸同情(The Appeal to Pity:Argument Ad Misericordiam)}

\begin{theorembox}[title=诉诸同情谬误的定义与特征]
\logicemph{词汇来源}:\logicterm{诉诸同情}(misericordiam的字面意思是"同情心")可以看做是诉诸情感的一种特殊情况。

\logicemph{特殊情感}:其中听众的利他主义和怜悯之心是其所诉诸的特殊情感。

\logicwarn{谬误定义}:当论证试图利用人们的怜悯和同情心而非提供相关证据来支持结论时,就犯了\logicterm{诉诸同情谬误}。

\logicwarn{逻辑错误}:
\begin{itemize}
  \item 同情心虽然是高尚的情感,但不能替代理性证据
  \item 情感诉求与论证的逻辑有效性无关
  \item 怜悯不能改变事实的真假
\end{itemize}
\end{theorembox}

\paragraph{法庭辩护中的同情诉求}
\begin{examplebox}[title=法庭中的同情诉求]
\logicwarn{民事诉讼中的应用}:在法庭上,原告的律师为寻求伤害赔偿金,常常以某种极其悲苦的方式安排展示委托人的伤残情况。$^{[15]}$

\logicwarn{刑事审判中的问题}:
\begin{itemize}
  \item 在刑事审判中,虽然陪审团对被指控者无论有罪或无辜都不应带有同情
  \item 但是高效的辩护律师常常设法激起陪审团的同情
  \item 有时诉诸同情被做得神不知鬼不觉
\end{itemize}

\logicemph{法律与情感的冲突}:
\begin{itemize}
  \item 法律要求基于事实和证据进行判断
  \item 情感诉求可能干扰理性判断
  \item 同情心虽然可贵,但不应影响司法公正
\end{itemize}
\end{examplebox}

\begin{examplebox}[title=苏格拉底对诉诸同情的批判]
\logicemph{历史背景}:在其雅典审判中,苏格拉底轻蔑地提到其他被告人由他们的子女和家人陪伴出现在陪审团面前,以寻求激起怜悯之情而免除责任。

\logicemph{苏格拉底的立场}:苏格拉底接着说道:

\begin{quote}
"……我,生命处于危险中的我,将不会做任何这种事情。这种对比可能出现在他(陪审团成员)的头脑中,他可能反对我,愤怒地投票,因为他为此对我不高兴。现在,如果你们之中有这样的人,注意,我不是说确有,那么我可以诚实地回答他:我的朋友,我是人,和别人一样,一个有血有肉的生物,不是像荷马(Homer)所说的那种'木石之躯';我也有一个家庭,有孩子,噢,雅典人啊,有三个儿子,一个几乎成人,另两个还年幼;但是,我将不带他们任何人到这里以请求你们判我无罪。"$^{[16]}$
\end{quote}

\logicwarn{哲学意义}:
\begin{itemize}
  \item 苏格拉底拒绝使用情感操控手段
  \item 坚持理性论证的原则
  \item 体现了对真理和正义的尊重
\end{itemize}
\end{examplebox}

\begin{examplebox}[title=荒谬的同情诉求案例]
\logicwarn{极端案例}:有很多方法可以拨动心弦,而且事实上,也都为人们一再使用。

\logicwarn{最荒谬的例子}:在一次指控一个年轻人用斧头杀害了他父母的审判中,出现了最荒谬的诉诸同情论证:面对其罪恶的大量证据,他请求宽大处理,理由是他现在成了一个孤儿。

\logicemph{逻辑分析}:
\begin{itemize}
  \item 这个案例显示了诉诸同情谬误的极端形式
  \item 被告试图利用自己造成的后果来获得同情
  \item 完全颠倒了因果关系和道德责任
\end{itemize}
\end{examplebox>