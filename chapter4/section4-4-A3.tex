\subsection{A3.重读}

当论证的意义变化源于对其词汇或组成部分的强调的变动时,该论证就可以证明是欺骗性的和无效的。若前提的明显意义依赖于一个可能的强调,但是,得出的结论却依赖于对相同词汇不同的重读意义,这时就犯了\textbf{重读}(accent)谬误。

\paragraph{重音与语气的影响}
作为示例,请考虑我们可以把不同的意义给予如下陈述:

我们不应当说朋友的坏话(We should not speak ill of our friends)。

在印刷字体及图片方面,有很多伎俩常常是通过强调某处而起误导之效。出现在新闻报道标题中的大号字敏感词汇,故意向那些勿勿浏览的人暗示错误的结论,而该标题后面却很可能用其他词汇以很小的字来加以限制。为避免在看新闻报道或在签订合同时被欺骗,我们力劝人们注意"小字印刷"。在政治宣传中,特别是在声称所谓事实报道中,选择令人误解的敏感标题或选择使用部分省略的图片,都是对重读谬误的精心使用,力图使读者得出宣传者明知为假的结论。解说可能不是彻底的谎言,但它也可以利用故意或虚假的重读方式来歪曲事实。

\paragraph{广告中的重读谬误}
在广告中,这样做的也很多。非常低的价格往往以非常大的字出现,而后面却跟随着字体极小的"以及完全说明"。飞机票价打折的通告后面都跟有一个星号,以远远的一个脚注说明该价格仅仅可用于提前三个月预订星期四的飞行航班,或可能还会有其他"适用限制"。名牌昂贵商品都以非常低的价格做广告,在广告某处附有一个小注解"所列价格存货数量有限"。读者被吸引到商店,但可能以广告价格买不到商品。重读语段本身并不是严格谬误;源于重读的语段解释,当它依赖一个非常可疑的结论暗示时,即当其采用令人误解的重读来解释时,重读语段就变成了\textbf{谬误}(例如,飞机票或品牌商品可以按照所列价格优先购买)。

\paragraph{通过位置操纵的重读}
甚至字面上为真的语段,也可以通过操纵其位置而以重读来欺骗人。

一位船长厌恶他的首席助手上班时再三喝醉,在该船的航行日记上,他几乎每天都记上:"助手今天喝醉了。"愤怒的助手进行报复。一天,船长病了,助手就自己保管日志,他在上面记着:"船长今天清醒了。" 