\subsection{A4.合成}

\begin{theorembox}[title=合成谬误的定义]
\logicterm{合成谬误}:是不正当地从部分到整体的推理,我们可以区分两类错误。

\logicwarn{第一类合成谬误}:从部分的性质错误地推出整体的性质。

\logicemph{错误例子}:每一个砖块都很轻,所以由砖块砌成的墙很轻,这就是这种合成谬误的简单形式。

\logicwarn{谬误的复杂性}:这种谬误可能完全显而易见,因而不会欺骗任何人,但是,也有些推论,虽然犯同样的错误,却能导致正确的结论。

\logicemph{正确例子}:"每个砖块都是红的,所以砌成的墙是红的。"

\logicwarn{模式相似性的问题}:这样,上次推理与此次推理存在一种模式相似性,却一个是非常错误的另一个却是合理的,这既有趣也有启发性。

\logicemph{关键洞察}:
\begin{itemize}
  \item 相同的推理形式可能产生不同的结果
  \item 谬误的判定取决于具体涉及的性质
  \item 需要分析性质是否可以从部分转移到整体
  \item 形式逻辑的局限性在此显现
\end{itemize}
\end{theorembox}

\paragraph{性质的转移问题}
\begin{examplebox}[title=性质的转移问题]
\logicwarn{推理形式}:使这些推理看起来类似的形式是:每个部分都具有性质P,所以整体也具有性质P。

\logicemph{判定规则}:归纳出来的规则是:判定一个特定的推理形式是否会犯合成谬误,取决于所涉及的具体性质(这里是P)。

\logicwarn{性质分类}:
\begin{itemize}
  \item \logicterm{不可转移性质}:那些只为部分所有,却不为整体所有的性质,转移到整体上就是合成谬误
  \item \logicterm{可转移性质}:那些为部分所有,也可能为整体所有的性质,转移到整体上则可能是正当的
\end{itemize}

\logicemph{具体分析}:
\begin{itemize}
  \item \logicwarn{"轻"的例子}:"轻"这个性质只属于部分,不属于整体,所以就出现谬误
  \item \logicemph{"红色"的例子}:红色这个性质,属于部分,也可能属于整体,所以该推论不是谬误
\end{itemize}

\logicwarn{识别方法}:为了辨识可能的合成谬误,我们必须研究并理解特定性质能否从部分转移到整体。

\logicemph{性质类型}:
\begin{itemize}
  \item \logicterm{累积性质}:如重量、体积等,部分的总和构成整体
  \item \logicterm{分布性质}:如颜色、材质等,可以从部分扩展到整体
  \item \logicterm{涌现性质}:如功能、结构等,整体具有而部分不具有
  \item \logicterm{平均性质}:如密度、温度等,需要特殊计算方法
\end{itemize}
\end{examplebox}

\paragraph{个体与集体的性质差异}
\begin{examplebox}[title=个体与集体的性质差异]
\logicwarn{第二类合成谬误}:是不正当地从所有个别成员的性质(这些成员彼此分离或做单独考虑时)到该集体做一个整体时的性质的推理。

\logicemph{足球队的例子}:
\begin{itemize}
  \item \logicwarn{观察事实}:某位运动员观察到的事实:任何足球队员,单独看时都容易被击垮
  \item \logicwarn{错误结论}:因而断言任何足球队都可以轻易地被击垮
  \item \logicemph{谬误分析}:他就犯了这种合成谬误
\end{itemize}

\logicwarn{集体凝聚力}:团体有时候所具有的凝聚力是其个别成员所不具有的;因此单独个体的易碎性(可击垮性)不是一组受训练选手的特性。

\logicemph{推理的危险性}:所以,从个别成员的特性到整体的特性的推理,确实容易误入歧途。

\logicwarn{组织特性的复杂性}:那些归因于大学、公司、军队或体育团队的特性,不能轻易地由所有个别而分离的成员的特性来推出。

\logicemph{集体涌现特性}:
\begin{itemize}
  \item \logicterm{协调能力}:个体缺乏但团队具有的配合能力
  \item \logicterm{分工效率}:通过专业化分工产生的效率提升
  \item \logicterm{集体智慧}:群体决策可能优于个体决策
  \item \logicterm{组织文化}:超越个人特质的集体价值观和行为模式
  \item \logicterm{系统效应}:整体功能大于部分功能之和
\end{itemize}
\end{examplebox}

\paragraph{两种谬误形式的区别}
\begin{examplebox}[title=两种谬误形式的区别]
\logicwarn{两类谬误的关系}:这两类合成谬误虽然是平行的,但却是根本上有别的,因为元素的纯粹汇集与那些元素所构成的整体是不同的。

\logicemph{汇集与整体的区别}:
\begin{itemize}
  \item \logicwarn{机器例子}:机器的各部分的纯粹汇集不是机器
  \item \logicwarn{建筑例子}:砖头的纯粹汇集既不是房子也不是墙壁
\end{itemize}

\logicterm{整体的特征}:整体,比如机器、房子或墙壁,是将其部分以某种特定方式组织或安排起来的。

\logicwarn{根本区别}:正由于组织的整体与纯粹的汇集是截然不同的,所以这两种形式的合成谬误也是如此:
\begin{itemize}
  \item \logicterm{第一种}:从部分到整体的无效推广
  \item \logicterm{第二种}:从分子或元素到汇集的无效推广
\end{itemize}

\logicemph{概念澄清}:
\begin{itemize}
  \item \logicterm{汇集}(Collection):元素的简单聚合,无特定组织结构
  \item \logicterm{整体}(Whole):有组织的系统,具有特定的结构和功能
  \item \logicterm{部分}(Parts):构成整体的组成要素
  \item \logicterm{成员}(Members):属于某个集合的个体
\end{itemize}

\logicwarn{实践意义}:
\begin{itemize}
  \item 分析组织时要考虑结构和关系
  \item 评估集体时要考虑协作和互动
  \item 避免简单的线性推理
  \item 重视系统性思维
\end{itemize}
\end{examplebox}