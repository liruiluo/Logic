\subsection{A4.合成}

\textbf{合成谬误}是不正当地从部分到整体的推理,我们可以区分两类错误。第一类合成谬误是从部分的性质错误地推出整体的性质。每一个砖块都很轻,所以由砖块砌成的墙很轻,这就是这种合成谬误的简单形式。这种谬误可能完全显而易见,因而不会欺骗任何人,但是,也有些推论,虽然犯同样的错误,却能导致正确的结论,例如:"每个砖块都是红的,所以砌成的墙是红的。"这样,属上次推理与此次推理存在一种模式相似性,却一个是非常错误的另一个却是合理的,这既有趣也有啓发性。

\paragraph{性质的转移问题}
使这些推理看起来类似的形式是:每个部分都具有性质 P,所以整体也具有性质 P。归纳出来的规则是:判定一个特定的推理形式是否会犯合成谬误,取决于所涉及的具体性质(这里是 P)。那些只为部分所有,却不为整体所有的性质,转移到整体上就是合成谬误,而那些为部分所有,也可能为整体所有的性质,转移到整体上则可能是正当的。例如"轻"这个性质只属于部分,不属于整体,所以就出现谬误;而红色这个性质,属于部分,也可能属于整体,所以该推论不是谬误。为了辨识可能的合成谬误,我们必须研究并理解特定性质能否从部分转移到整体。

\paragraph{个体与集体的性质差异}
第二类合成谬误是不正当地从所有个别成员的性质(这些成员彼此分离或做单独考虑时)到该集体做一个整体时的性质的推理。例如,某位运动员观察到的事实:任何足球队员,单独看时都容易被击垮,因而断言任何足球队都可以轻易地被击垮,他就犯了这种合成谬误。团体有时候所具有的凝聚力是其个别成员所不具有的;因此单独个体的易碎性(可击垮性)不是一组受训练选手的特性。所以,从个别成员的特性到整体的特性的推理,确实容易误入歧途。那些归因于大学、公司、军队或体育团队的特性,不能轻易地由所有个别而分离的成员的特性来推出。

\paragraph{两种谬误形式的区别}
这两类合成谬误虽然是平行的,但却是根本上有别的,因为元素的纯粹汇集与那些元素所构成的整体是不同的。例如,机器的各部分的纯粹汇集不是机器;砖头的纯粹汇集既不是房子也不是墙壁。整体,比如机器、房子或墙壁,是将其部分以某种特定方式组织或安排起来的。正由于组织的整体与纯粹的汇集是截然不同的,所以这两种形式的合成谬误也是如此,一种是从部分到整体的无效推广,另一种是从分子或元素到汇集的无效推广。 