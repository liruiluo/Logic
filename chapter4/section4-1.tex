\section*{藛紷密}
\section*{4.1 什么是谬误?}
一个论证,无论其主题或领域是什么,一般都是为证明其结论为真而构建的。但是,在两种情形下,任何论证都不能实现这一宗旨。一种情形是将一个虚假命题设定为论证的前提之一。在第1章中我们看到,每个论证都断言其结论之真是从前提到结论的真推导出来或为前提之真所蕴涵。因此,如果论证的前提不真,那么就不能确立其结论的真,即使从前提到结论的推理是正确的。然而,检验前提的真与假并不是逻辑学家的特殊职责,那是所有研究工作的共同任务,因为前提可以牵涉任何研究主题。

论证不能确立结论之真的另一种情形是,其所依赖的前提并不蕴涵结论。这才是逻辑学家的特殊领地。逻辑学家主要关心的是结论与前提之间的逻辑关系。一个论证的前提不支持它的结论,即使它的所有前提都是真的,它的结论也可能是假的。在这种情况下,其推理便是糟糕的,而这种论证就称为谬误。谬误就是推理错误。

然而,逻辑学家所用的"谬误"这个词,并不指称所有过失推理或虚假信念,而是指称一种典型错误,即经常出现在日常话语中破坏论证的错误。每个谬误都是不正确论证的一种类型。若论证中出现了一个特定类型的错误,就称为犯有那种谬误。因为每个谬误都是一种类型,故而我们可以说,两个或更多的不同论证可以包含或犯有相同谬误;也就是说,它们在推理中表现为同一种错误。包含或犯有特定类型谬误的一个论证,也可以被称为是一个谬误,也即那种类型错误的一个实例。

推理进入歧路的方式可以有很多种,也就是说,论证错误有很多种。习惯上,人们将"谬误"这个词用在那些虽然不正确但在心理上具有一定说服力的论证。有些论证错误是非常明显的,不能欺骗和说服任何人。但是,谬误却是危险的,因为我们大都会偶尔被某些谬误所愚弄。因此,我们将谬误定义为一种看似正确但经过检验可证其为错误的论证类型。研究这些错误论证是非常有益的,因为当我们明确理解它们后,就可以最有效地避开它们布下的陷阱。有备无患!

特定的论证是否事实上真是谬误,可能取决于其作者对词项的解释。看来是谬误的语段,若脱离语境,就可能难以确定作者使用的词项打算意味什么。有时,"谬误"的指责就会不公平地对准这样的语段,而其作者

想要表达的观点却被批评者漏掉了(或许,作者甚至是开玩笑的)。当我们将对谬误论证的分析运用到实际谈话中时,应当注意这种不可避免的复杂情况。我们的逻辑标准应当高,但将这些标准运用到日常生活的论证中时,也应当宽宏大量和公平。

我们可以在论证中区分出多少种不同谬误呢?亚里士多德是第一位对它有系统研究的逻辑学家,他曾列举出 13 种 ${}^{[1]}$ ;近来,超过 100 种的谬误名单被列了出米。 ${ }^{[2]}$ 然而,谬误并没有一个精确的可以确定下来的数目,因为在列举它们时,在很大程度上取决于所使用的分类体系。在此,我们挑出 17 种谬误,即推理中最普通且最有欺骗性的错误,分成三大组,分别称为:(a)相干谬误(fallacy of relevance);(b)预设谬误(fallacy of presumption);(c)含混谬误(fallacy of ambiguity)。 ${}^{[3]}$

谬误的分组总有某种程度的任意性,因为一种错误会与另一种错误具有密切的相似性,有时还是相重合的。一个给定的谬误语段应属于哪个特定组别也常常引起人们的争论,因为语段中可能会有一个以上的推理错误。如果人们谨记这种不可避免的不精确性,那么理解三种主要种类的每一种本质特征及其各种子类别的特别特征,将具有很大的实际用处。当推理中最难缠的错误出现于通常话语中时,这些理解就能够使人们发觉这些错误。辨识这些相互联系的谬误也有益于提高我们的逻辑敏感性,而这种敏感性也有益于我们识别那些在三大组中未能包含的谬误。 