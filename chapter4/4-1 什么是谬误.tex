\section{什么是谬误?}

\begin{logicbox}[title=引言]
\textit{逻辑学中的谬误指的是一种特定类型的推理错误,它们表面上看似合理,实则存在逻辑缺陷。识别和理解谬误对于提高批判性思维和避免错误推理至关重要。}
\end{logicbox}

\subsection{谬误的定义与本质}

\begin{theorembox}[title=论证失败的两种情形]
\logicemph{论证目标}:一个论证,无论其主题或领域是什么,一般都是为证明其结论为真而构建的。

\logicemph{失败情形一:前提虚假}:
\begin{itemize}
  \item 将一个虚假命题设定为论证的前提之一
  \item 如果论证的前提不真,那么就不能确立其结论的真,即使从前提到结论的推理是正确的
  \item 检验前提的真与假并不是逻辑学家的特殊职责,而是所有研究工作的共同任务
\end{itemize}

\logicemph{失败情形二:推理无效}:
\begin{itemize}
  \item 论证所依赖的前提并不蕴涵结论
  \item 这才是逻辑学家的特殊领地
  \item 逻辑学家主要关心的是结论与前提之间的逻辑关系
\end{itemize}
\end{theorembox}

\begin{theorembox}[title=谬误的核心定义]
\logicemph{基本定义}:一个论证的前提不支持它的结论,即使它的所有前提都是真的,它的结论也可能是假的。在这种情况下,其推理便是糟糕的,而这种论证就称为\logicterm{谬误}。

\logicterm{谬误}就是推理错误。

\logicwarn{精确含义}:逻辑学家所用的"谬误"这个词,并不指称所有过失推理或虚假信念,而是指称一种典型错误,即经常出现在日常话语中破坏论证的错误。
\end{theorembox}

\begin{theorembox}[title=谬误的类型特征]
\logicemph{类型性质}:
\begin{itemize}
  \item 每个谬误都是不正确论证的一种类型
  \item 若论证中出现了一个特定类型的错误,就称为犯有那种谬误
  \item 因为每个谬误都是一种类型,两个或更多的不同论证可以包含或犯有相同谬误
\end{itemize}

\logicemph{实例关系}:
\begin{itemize}
  \item 不同论证在推理中可以表现为同一种错误
  \item 包含或犯有特定类型谬误的一个论证,也可以被称为是一个谬误
  \item 即那种类型错误的一个实例
\end{itemize}
\end{theorembox}

\subsection{谬误的心理说服力}

\begin{theorembox}[title=谬误的欺骗性特征]
\logicemph{错误的多样性}:推理进入歧路的方式可以有很多种,也就是说,论证错误有很多种。

\logicemph{心理说服力}:习惯上,人们将"谬误"这个词用在那些虽然不正确但在心理上具有一定说服力的论证。

\logicwarn{明显错误与隐蔽错误的区别}:
\begin{itemize}
  \item 有些论证错误是非常明显的,不能欺骗和说服任何人
  \item 但是,谬误却是危险的,因为我们大都会偶尔被某些谬误所愚弄
\end{itemize}

\logicemph{操作性定义}:我们将谬误定义为一种\logicterm{看似正确但经过检验可证其为错误的论证类型}。

\logicemph{研究价值}:研究这些错误论证是非常有益的,因为当我们明确理解它们后,就可以最有效地避开它们布下的陷阱。有备无患!
\end{theorembox}

\subsection{谬误分析的复杂性}

\begin{theorembox}[title=语境依赖性问题]
\logicwarn{解释依赖性}:特定的论证是否事实上真是谬误,可能取决于其作者对词项的解释。

\logicwarn{语境重要性}:
\begin{itemize}
  \item 看来是谬误的语段,若脱离语境,就可能难以确定作者使用的词项打算意味什么
  \item 有时,"谬误"的指责就会不公平地对准这样的语段
  \item 作者想要表达的观点却被批评者漏掉了(或许,作者甚至是开玩笑的)
\end{itemize}

\logicemph{分析原则}:当我们将对谬误论证的分析运用到实际谈话中时,应当注意这种不可避免的复杂情况。

\logicemph{平衡标准}:我们的逻辑标准应当高,但将这些标准运用到日常生活的论证中时,也应当宽宏大量和公平。
\end{theorembox}

\subsection{谬误的分类体系}

\begin{theorembox}[title=谬误研究的历史发展]
\logicemph{历史起源}:亚里士多德是第一位对谬误有系统研究的逻辑学家,他曾列举出13种\cite{aristotle}。

\logicemph{现代发展}:近来,超过100种的谬误名单被列了出来\cite{fearnside1959}。

\logicwarn{数量的不确定性}:谬误并没有一个精确的可以确定下来的数目,因为在列举它们时,在很大程度上取决于所使用的分类体系。
\end{theorembox}

\begin{theorembox}[title=本书的分类体系]
\logicemph{选择标准}:在此,我们挑出17种谬误,即推理中最普通且最有欺骗性的错误,分成三大组:

\begin{itemize}
  \item \logicterm{相干谬误}(fallacy of relevance)
  \item \logicterm{预设谬误}(fallacy of presumption)
  \item \logicterm{含混谬误}(fallacy of ambiguity)\cite{joseph1916}
\end{itemize}
\end{theorembox}

\begin{theorembox}[title=分类的局限性与价值]
\logicwarn{分类的任意性}:
\begin{itemize}
  \item 谬误的分组总有某种程度的任意性
  \item 一种错误会与另一种错误具有密切的相似性,有时还是相重合的
  \item 一个给定的谬误语段应属于哪个特定组别也常常引起人们的争论
  \item 语段中可能会有一个以上的推理错误
\end{itemize}

\logicemph{实用价值}:
\begin{itemize}
  \item 理解三种主要种类的每一种本质特征及其各种子类别的特别特征,将具有很大的实际用处
  \item 当推理中最难缠的错误出现于通常话语中时,这些理解就能够使人们发觉这些错误
  \item 辨识这些相互联系的谬误也有益于提高我们的逻辑敏感性
  \item 这种敏感性也有益于我们识别那些在三大组中未能包含的谬误
\end{itemize}
\end{theorembox}

\chaptersummary{
谬误是逻辑学研究的重要内容,理解谬误的本质和特征对于提高批判性思维能力具有重要意义。

\logicemph{谬误的核心概念}:
\begin{itemize}
  \item \logicterm{基本定义}:看似正确但经过检验可证其为错误的推理类型
  \item \logicterm{本质特征}:前提不支持结论的无效论证,但具有表面的合理性
  \item \logicterm{类型性质}:每个谬误都是不正确论证的一种类型,可以在不同论证中重复出现
\end{itemize}

\logicemph{谬误的欺骗性}:
\begin{itemize}
  \item \logicterm{心理说服力}:谬误具有一定的心理说服力,容易欺骗和误导人们
  \item \logicterm{隐蔽性}:与明显的论证错误不同,谬误需要仔细检验才能识别
  \item \logicterm{危险性}:正是这种表面的正确性使得谬误具有欺骗性和危险性
\end{itemize}

\logicemph{谬误的分类体系}:
\begin{itemize}
  \item \logicterm{相干谬误}:前提与结论缺乏相关性的推理错误
  \item \logicterm{预设谬误}:基于错误假设或预设的推理错误
  \item \logicterm{含混谬误}:由于语言歧义或模糊性导致的推理错误
\end{itemize}

\logicwarn{分析原则}:
\begin{itemize}
  \item 需要考虑语境和作者意图,避免脱离语境的机械判断
  \item 保持高逻辑标准,但在实际应用中要宽宏大量和公平
  \item 认识到分类的相对性和复杂性,一个论证可能包含多种谬误
\end{itemize}
}