\section{预设谬误}

\begin{quotation}
\textit{预设谬误是一种隐藏在论证中的特殊错误,它暗含了某些未经证明且无根据的假设。识别这类谬误能够帮助我们避免被表面上合理的论证所误导,培养更加批判性的思维方式。}
\end{quotation}

\textbf{预设谬误}通常只有在论证的精确表述中才能显示出来。一段话的作者、讲者,或读者、听者,都有可能会假定某些未经证明的和无根据的前提为真,无论是出于疏忽还是故意设计。而当掩藏在论证里的这种可疑假设对支持结论非常关键时,论证就是糟糕的并可使人陷人误区。这类无根据的跳跃就被称为预设谬误。

在这类论证谬误中,前提也常常与结论不相干。的确,在大多数谬误中都存在前提与结论之间不相干的缺口,但是,预设谬误展示出一种特殊的错误:那种不为人支持甚至是不可支持的\textbf{暗含假定}。要揭露这样的谬误,注意留心那种偷偷溜进的假设及其可疑与虚假性,通常是很奏效的。 