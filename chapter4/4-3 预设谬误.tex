\section{预设谬误}

\begin{logicbox}[title=引言]
\textit{预设谬误是一种隐藏在论证中的特殊错误,它暗含了某些未经证明且无根据的假设。识别这类谬误能够帮助我们避免被表面上合理的论证所误导,培养更加批判性的思维方式。}
\end{logicbox}

\begin{theorembox}[title=预设谬误的本质特征]
\logicemph{显现条件}:\logicterm{预设谬误}通常只有在论证的精确表述中才能显示出来。

\logicemph{产生原因}:
\begin{itemize}
  \item 一段话的作者、讲者,或读者、听者,都有可能会假定某些未经证明的和无根据的前提为真
  \item 无论是出于疏忽还是故意设计
  \item 当掩藏在论证里的这种可疑假设对支持结论非常关键时,论证就是糟糕的并可使人陷入误区
\end{itemize}

\logicwarn{定义}:这类无根据的跳跃就被称为预设谬误。
\end{theorembox}

\begin{theorembox}[title=预设谬误的特殊性]
\logicemph{与相干谬误的关系}:在这类论证谬误中,前提也常常与结论不相干。

\logicemph{独特之处}:
\begin{itemize}
  \item 在大多数谬误中都存在前提与结论之间不相干的缺口
  \item 但是,预设谬误展示出一种特殊的错误:那种不为人支持甚至是不可支持的\logicterm{暗含假定}
\end{itemize}

\logicemph{识别方法}:要揭露这样的谬误,注意留心那种偷偷溜进的假设及其可疑与虚假性,通常是很奏效的。
\end{theorembox}

\begin{theorembox}[title=预设谬误的主要类型]
\logicemph{五种主要类型}:预设谬误包括以下主要类型:

\begin{enumerate}
  \item \logicterm{复杂问语}(Complex Question)
  \item \logicterm{虚假原因}(False Cause)
  \item \logicterm{乞题}(Begging the Question)
  \item \logicterm{偶然}(Accident)
  \item \logicterm{逆偶然}(Converse Accident)
\end{enumerate}

\logicemph{学习重点}:每种谬误都有其独特的表现形式和识别方法,需要通过具体例子来深入理解。
\end{theorembox}