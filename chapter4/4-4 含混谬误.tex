\section{含混谬误}

\begin{logicbox}[title=引言]
\textit{含混谬误是由词语或短语的意义变化而导致的推理错误,这类谬误可能出于疏忽,也可能是刻意为之。识别和理解这类谬误有助于我们避免在论证中被词语的多义性所误导。}
\end{logicbox}

\begin{theorembox}[title=含混谬误的本质特征]
\logicemph{产生原因}:由于用心不专或故意操作,在论证过程中,词或短语的意义可能会变化。

\logicemph{基本机制}:
\begin{itemize}
  \item 一个词项在前提中可能具有一种意义
  \item 但是在结论中却是另一种相当不同的意义
  \item 当推论依赖这样的变化时,当然就是谬误
\end{itemize}

\logicwarn{定义}:这种错误称做\logicterm{含混谬误},有时或称为\logicterm{诡论}(sophisms)。

\logicemph{识别难度}:
\begin{itemize}
  \item 故意使用这样的方法常常是粗糙的和易于发现的
  \item 但是,有时(虽并非经常)这种含混是隐蔽的、难以把捉的
\end{itemize}
\end{theorembox}

\begin{theorembox}[title=含混谬误的五种主要类型]
\logicemph{分类体系}:我们在下面区分出含混谬误的五种类型:

\begin{enumerate}
  \item \logicterm{歧义谬误}(Equivocation)
  \item \logicterm{双关谬误}(Amphiboly)
  \item \logicterm{重读谬误}(Accent)
  \item \logicterm{合成谬误}(Composition)
  \item \logicterm{分解谬误}(Division)
\end{enumerate}

\logicemph{学习重点}:
\begin{itemize}
  \item 每种谬误都涉及不同类型的语言歧义或意义变化
  \item 需要通过具体例子来理解各种含混谬误的表现形式
  \item 培养对语言精确性的敏感度是避免这类谬误的关键
\end{itemize}

\logicwarn{实用价值}:理解含混谬误有助于我们在日常交流和学术写作中避免被词语的多义性所误导。
\end{theorembox}