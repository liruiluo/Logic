\subsection{P3.乞题(Begging the Question:Petitio Principii)}

\begin{theorembox}[title=乞题谬误的定义与特征]
\logicwarn{重要性}:在所有非形式谬误中,\logicterm{乞题谬误}可能是最多讨论、最多批评和最被滥用的一种。

\logicemph{术语澄清}:作为一个技术术语,它并不意味着它有时在现代用法中被赋予的意思:"引出一个问题"或"表明需要讨论一个问题"。

\logicwarn{传统定义}:而是有它的传统含义:一个论证犯有乞题谬误,当且仅当该论证所使用的前提蕴涵、寄生于或以某种方式预设了它所要确证的结论时。

\logicemph{拉丁词源}:正如它的拉丁名(Petitio Principii)所暗示的,这种逻辑错误是"恳求"问题的原则(或开端),即恳求得到确证该结论的许可。

\logicwarn{核心问题}:
\begin{itemize}
  \item 前提中已经包含了结论
  \item 论证变成了循环推理
  \item 没有提供独立的证据支持
  \item 假设了需要证明的东西
\end{itemize}
\end{theorembox}

\paragraph{循环论证}
\begin{examplebox}[title=循环论证的典型例子]
\logicwarn{最明显的情形}:用同一个命题断言作为前提和结论的愚蠢尝试,是一种乞题的最明显的情形。

\logicemph{经典例子}:如果有人想证明上帝存在,而断言"上帝存在,因为圣经如是说",那么:

\logicwarn{循环推理的结构}:
\begin{itemize}
  \item \logicemph{前提1}:圣经说上帝存在
  \item \logicemph{隐含前提}:圣经是上帝的话语(因此是可靠的)
  \item \logicemph{结论}:上帝存在
\end{itemize}

\logicwarn{逻辑问题}:只要他同意"圣经是上帝的话语",他就做了一个圆圈般的论证,其中的结论已经被假定在前提之中。

\logicemph{关键矛盾}:在宣称圣经是确证上帝存在的权威证据之前,他必须首先被确信有这么一个上帝。

\logicwarn{本质分析}:本质上,他就一直在断言,上帝的确存在,因为上帝说他存在。
\end{examplebox}

\begin{theorembox}[title=恶性循环论证的特征]
\logicterm{术语定义}:这种乞题形式被称做"\logicterm{恶性循环}"论证,即结论作为假定在前提中自现而不为人所注意,因而论证无法保护结论免于质疑。

\logicwarn{普遍性问题}:可惜地是,我们在日常谈话中经常不加心思地使用这种论证形式的一些变形,以至于甚至许多谨慎的思想家也掉入了它的陷阱。

\logicemph{历史警告}:哲学家弗朗西斯·培根曾就一个几乎陷入这种谬误的自然哲学例子发出警告(他在他的《新工具》或《新方法》一书中指出了它)。
\end{theorembox}

\begin{logicbox}[title=培根的警告:认知偏见与循环推理]
\logicwarn{人类理智的局限}:人的理智(此处是指日常思维方式)不是纯粹的光明;受到意志和情感的浸染;使这些特性遂其所欲。

\logicemph{愿望思维}:因为一个人希望为真的,他便很容易相信是真的。

\logicwarn{选择性观察}:他所观察的特例在他之前泛滥变多或者缺乏变少,这取决于那些特例是否会导出他先前决定的结论。$^{[23]}$

\logicemph{现代意义}:这种观察揭示了确认偏误和循环推理之间的深层联系。
\end{logicbox>

\paragraph{归纳原理的证明困境}
\begin{examplebox}[title=归纳原理的证明困境]
\logicwarn{更细微的乞题形式}:但是,更细微且更容易误导的乞题谬误形式,也可以借助另一种所提出的论证来讲解,该论证旨在确证\logicterm{归纳原理}。

\logicemph{归纳原理的定义}:这种原理不是关于过去经验如何引导现在的行为,而是关于将过去经验视为一个关于未来的可靠依据的原理。

\logicwarn{循环论证的结构}:任何这样的论证都企图,通过再次假定该原理为真,来寻求确证归纳程序的真实性。

\logicemph{原理的内容}:这种原理是:
\begin{itemize}
  \item 自然法则像它们操控今天一样也会操控明天
  \item 本质上自然法则在基本方面是无变化的
  \item 因而我们可以依赖过去的经验来指导我们未来的行为
\end{itemize}

\logicwarn{问题的焦点}:"未来本质上像过去一样"的断言是问题的焦点,但是,这个断言——在平常生活中从未遭到质疑,结果非常难以证明。

\logicemph{常见的错误论证}:有些思想家断言,通过表明当我们过去依赖归纳原理时,我们总是发现这种方法能够帮助我们获取目标,这样就可以证明它。

\logicwarn{循环推理}:他们问:"为什么得出未来将与过去一样?"回答道:"因为它总是与过去一样。"

\logicemph{休谟的批评}:但是,正如大卫·休谟所指出的那样,这种常见论证是一个"petitio",它犯了乞题谬误。

\logicwarn{逻辑错误}:因为所讨论问题的焦点正是,自然将是否继续有规律地运行;它过去如此不能作为它未来还将如此的证据,除非一个人事先假定了正在讨论的那种原则:未来将与过去一样。

\logicemph{休谟问题}:因而,休谟承认过去中的未来的确都与过去一样,但他问道(这个著名的休谟问题哲学家们仍在争论):

\begin{quote}
"我们怎么能够知道未来的未来将与过去一样呢?当然它们可能一样,但是,我们不能为了证明它们而假定它们。"$^{[26]}$
\end{quote}

\logicwarn{哲学意义}:这个问题揭示了归纳推理的根本困境,至今仍是哲学讨论的重要话题。
\end{examplebox}