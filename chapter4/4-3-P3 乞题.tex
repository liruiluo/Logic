\subsection{P3.乞题(Begging the Question:Petitio Principii)}

\begin{theorembox}[title=乞题谬误的定义]
在所有非形式谬误中,\logicwarn{乞题谬误}可能是最多讨论、最多批评和最被滥用的一种。作为一个技术术语,它并不意味着它有时在现代用法中被赋予的意思:"引出一个问题"或"表明需要讨论一个问题";而是有它的传统含义:一个论证犯有\logicwarn{乞题谬误},当且仅当该论证所使用的前提蕴涵、寄生于或以某种方式预设了它所要确证的结论时。正如它的拉丁名所暗示的,这种逻辑错误是"恳求"问题的原则(或开端),即恳求得到确证该结论的许可。
\end{theorembox}

\paragraph{循环论证}
\begin{examplebox}[title=循环论证的典型例子]
用同一个命题断言作为前提和结论的愚蠢尝试,是一种\logicwarn{乞题}的最明显的情形。如果有人想证明上帝存在,而断言"上帝存在,因为圣经如是说",那么,只要他同意"圣经是上帝的话语",他就做了一个圆圈般的论证,其中的结论已经被假定在前提之中。在宣称圣经是确证上帝存在的权威证据之前,他必须首先被确信有这么一个上帝。本质上,他就一直在断言,上帝的确存在,因为上帝说他存在。
\end{examplebox}

这种\logicwarn{乞题}形式被称做"\logicterm{恶性循环}"论证,即结论作为假定在前提中自现而不为人所注意,因而论证无法保护结论免于质疑。可惜地是,我们在日常谈话中经常不加心思地使用这种论证形式的一些变形,以至于甚至许多谨慎的思想家也掉入了它的陷阱。哲学家弗朗西斯•培根曾就一个几乎陷入这种\logicwarn{谬误}的自然哲学例子发出警告(他在他的《新工具》或《新方法》一书中指出了它):

\begin{logicbox}[title=培根的警告]
人的理智(此处是指日常思维方式)不是纯粹的光明;受到意志和情感的浸染;使这些特性遂其所欲;因为一个人希望为\logicemph{真}的,他便很容易相信是\logicemph{真}的……他所观察的特例在他之前泛滥变多或者缺乏变少,这取决于那些特例是否会导出他先前决定的结论。[23]
\end{logicbox}

\paragraph{归纳原理的证明困境}
\begin{examplebox}[title=归纳原理的证明困境]
但是,更细微且更容易误导的\logicwarn{乞题谬误}形式,也可以借助另一种所提出的论证来讲解,该论证旨在确证\logicterm{归纳原理}——这种原理不是关于过去经验如何引导现在的行为,而是关于将过去经验视为一个关于未来的可靠依据的原理。任何这样的论证都企图,通过再次假定该原理为\logicemph{真},来寻求确证归纳程序的真实性。这种原理是,自然法则像它们操控今天一样也会操控明天,本质上自然法则在基本方面是无变化的,因而我们可以依赖过去的经验来指导我们未来的行为。"未来本质上像过去一样"的断言是问题的焦点,但是,这个断言——在平常生活中从未遭到质疑,结果非常难以证明。有些思想家断言,通过表明当我们过去依赖归纳原理时,我们总是发现这种方法能够帮助我们获取目标,这样就可以证明它。他们问:"为什么得出未来将与过去一样?"回答道:"因为它总是与过去一样。"

但是,正如大卫•休谟所指出的那样,这种常见论证是一个"\logicterm{petitio}",它犯了\logicwarn{乞题谬误}。因为所讨论问题的焦点正是,自然将是否继续有规律地运行;它过去如此不能作为它未来还将如此的证据,除非一个人事先假定了正在讨论的那种原则:未来将与过去一样。因而,休谟承认过去中的未来的确都与过去一样,但他问道(这个著名的休谟问题哲学家们仍在争论):"我们怎么能够知道未来的未来将与过去一样呢?当然它们可能一样,但是,我们不能为了证明它们而假定它们。"[26]
\end{examplebox}