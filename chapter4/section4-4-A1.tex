\subsection{A1.歧义}

大多数词汇都有多于一个的字面意义,但在多数情况下,通过注意语境和利用我们良好的感觉,我们在阅读和听讲时不难将这些意义分辨开来。但是,当人们有意无意地混淆一个词或短语的几个意义时,就是在\textbf{歧义}地使用这个词或短语。如果在论证中这样做,就犯了\textbf{歧义谬误}。

\paragraph{文学中的歧义}
有时,这种歧义谬误非常明显,在某些玩笑的字里行间使用。刘易斯•卡罗尔(Lewis Carroll)在《爱丽丝镜中奇遇记》(\textit{Through the Looking Glass})中对爱丽丝的奇遇的讲述,就包含着机智和逗乐的歧义。其中一个如下:\\
"你们谁走过这条路?"国王继续走着,并向送信人伸出手要些千草。\\
"没有人(nobody)。"送信人说。\\
"很对,"国王说,"这位年轻的女子也看到过他(him)。所以,当然 Nobody 比你们走得更慢。"

在这段话中,歧义谬误其实用得是相当巧妙的。第一次使用时"nobody"这个词仅仅是指"没有人"(no person)的意思。但是,接着用代词"他(him)"来指称,就好像"nobody"这个词命名了一个人。结果,当相同的词被大写并明显地用做一个名字"Nobody"时,它就显然命名了一个人,这个人具有没有走过这条路的特性,而这个特性又是从该词的第一次运用中得来的。有时,歧义是机智的工具,刘易斯-卡罗尔就是一位非常机智的逻辑学家。${ }^{[28]}$\\

\paragraph{相对词的歧义问题}
歧义论证总是谬误的,但它们却不总是愚蠢和滑稽的,这一点将在下面节录的例子中看出来:

有一种歧义谬误特别值得一提。这是一种由错误使用\textbf{相对性}(relative)词项而来的错误;在不同语境中,相对词具有不同意义。例如,"高"就是一个相对词,高个子人与高建筑物就处于非常不同的类别。一个高的人是一个比大部分人都高的人,而一座高的建筑物是一座比大部分建筑物都高的建筑物。某些论证形式可以对没有相对性的词有效,但当用相对词来代替那些词的时候,这种论证就垮掉了。"象是动物,因此灰色的象是灰色的动物"这个论证是完全有效的。"灰色"这个词不是相对的。但是,"象是动物,因此小象是小动物"这个论证却是荒唐的。这里的关键之点是,"小"是个相对词:小象是非常大的动物。这个谬误就是一个关于相对词"小"的一种歧义谬误。然而,并非所有的有关相对词的歧义谬误都是这样显然。"好"这个词是个相对词,关于它,经常出现歧义谬误。例如,有人论证说某某是一位好将军,因此也会成为一位好总统,或者是一位好学者,从而也一定是一位好教师。 