\subsection{R6.诉诸暴力(The Appeal to Force:Argument Ad Baculum)}

\begin{theorembox}[title=诉诸暴力谬误的定义]
\logicterm{诉诸暴力}以达到接受某种结论,乍看起来好像是一种如此清楚的\logicwarn{谬误},完全用不着讨论。当一个论证依靠威胁或强制而非理性证据来支持结论时,就犯了\logicwarn{诉诸暴力谬误}。当证据或合理论证失败时,"暴力方法"的使用或威胁使用以强制对手,看起来是最后的手段——一种"方便实用"的手段。"强权就是公理"的道理并非难以捉摸。
\end{theorembox}

\paragraph{暴力的多种形式}
\begin{examplebox}[title=法律威胁案例]
当然,暴力威胁不必是武力。近来,博伊斯州立大学(Boise State University)的两位法学教授在丹佛大学(University of Denver)法律杂志上发表了一篇文章,严厉批评了博伊斯瀑布公司(Boise Cascade Corporation)——世界上纸张和木制品生产者之一。结果,这所大学发布了一个正式的"更正"声明,"这篇文章因其缺乏学术性及其错误内容已经被撤销"。

博伊斯瀑布公司威胁起诉该大学了吗?"噢,"该大学的总法律顾问说,"'\logicwarn{威胁}'是一个有趣的词。让我们这样说吧,他们指出他们受到的批评的确达到了可以提出诉讼的地步。"结果,该大学收到了一份那篇文章的复制件,它来自博伊斯瀑布的总法律顾问,附信说:"如果其中被标明之处以任何形式被丹佛大学继续发行,我已得到了对丹佛大学提起法律诉讼的建议。"${ }^{[17]}$
\end{examplebox}

\paragraph{隐蔽的暴力威胁}
但是,也有比较含蓄地使用诉诸暴力的场合。论证者可以用精心设计的方式,不是直接地而是隐蔽地传达一种可能威胁,使对方为势所迫不得不赞同,或者至少是附和。当里根政府的司法部长处于报刊引导的强大攻击下时,当时的白宫参谋长霍华德•贝克(Howard Baker),召集工作人员开会说:

\begin{displayquote}
总统仍然信任司法部长,我也信任司法部长,而且你们也应当信任司法部长,因为我们本来都是在为总统工作。谁若对此有不同意见,或者有与此不同的动机、野心或打算,那么他可以告诉我,因为我们将不得不讨论他的去留。[18]
\end{displayquote}

\paragraph{理性与强制的对立}
可以说,没有人会被这种论证愚弄;被胁迫方可以适当地做出行动,但最后不必接受强加的结论为\logicemph{真}。对此,20世纪的意大利法西斯主义代表回答说,真正的说服可以通过许多不同工具来进行,讲道理是一种而大棒是另一种;但是,一旦对手被真正说服,他们就会坚持它,而说服的工具却可能被忘却了。这种法西斯主义观似乎引导着当今世界上许多政府;但是,诉诸暴力的论证——依赖大棒或各种形式的暴力威胁——从理性上说都是\logicwarn{不可接受的}。诉诸暴力是对\logicwarn{理性的抛弃}。