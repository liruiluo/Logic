\subsection{R2.诉诸不当权威(The Appeal to Inappropriate Authority:Argument Ad Verecundiam)}

\begin{theorembox}[title=权威诉诸的合理性]
\logicemph{合理的权威诉诸}:在试图对某些困难或复杂问题做出决定时,受公认专家判断引导是完全合理的。

\logicemph{非谬误情况}:当我们争辩说一特定结论是正确的因为专家权威已经得出那个判断时,我们并没有犯谬误。

\logicemph{必要性}:对我们大多数人来说,对权威的这种依赖在很多事情上来说都是必需的。

\logicwarn{权威的局限性}:
\begin{itemize}
  \item 专家的判断也不能构成最终证明
  \item 专家的意见之间也可能对立
  \item 即使一致,他们也可能出错
  \item 但是,专家意见确实是支持结论的一种合理方式
\end{itemize}
\end{theorembox}

\begin{theorembox}[title=诉诸不当权威谬误的定义]
\logicemph{谬误产生条件}:当诉诸的对象对所讨论问题不能合理地宣称权威时,就会产生\logicwarn{诉诸不当权威谬误}。

\logicemph{典型例子}:
\begin{itemize}
  \item 诉诸伟大的艺术家如毕加索的意见来解决经济争论
  \item 在关于道德的论证中,诉诸生物学杰出权威达尔文的意见
\end{itemize}

这些都是\logicwarn{谬误论证}。
\end{theorembox}

\begin{theorembox}[title=权威判断的复杂性]
\logicwarn{判断的谨慎性}:在决定谁的权威可以合理地依赖和拒绝上必须小心。

\logicemph{边界情况的考虑}:
\begin{itemize}
  \item 毕加索不是经济学家,但在属于艺术杰作经济价值的争论上,他的判断就可以合理地给予某种分量
  \item 如果争论的是道德问题中的生物学作用,那么达尔文的确可以是一位适当的权威
\end{itemize}

\logicemph{关键原则}:权威的适当性取决于其专业知识与讨论问题的相关程度。
\end{theorembox>

\paragraph{广告中的不当权威诉诸}
\begin{examplebox}[title=广告中的不当权威诉诸]
错置诉诸权威的最为明显的例子出现在广告的"证言"中。有人力劝我们开某一牌子的汽车,因为一位著名的高尔夫球员或者网球员断言了它的优越性;有人力劝我们饮用某种牌子的饮料,因为某电影明星或足球教练表达了对它的爱好。无论何处,如果对某命题为真的断定以某人的权威为依据,而他在那个领域并没有特殊的能力,那么这种\logicwarn{错置权威诉诸}就犯有\logicwarn{谬误}。
\end{examplebox}

\paragraph{政治和国际关系中的权威问题}
这好像是容易避免的头脑简单的错误,但由于存在着各种引发这种谬误诉诸的环境,这仍是一种危险的思维陷阱。这里有两个例子。在国际关系领域中,武器和战争扮演着不愉快的重要角色,对各种意见的支持经常诉诸这些人:他们对武器的技术设计和构造有特殊能力。例如,对于某些武器可以怎样或不能怎样起作用,物理学家诸如罗伯特•奥本海默(Robert Oppenheimer)或爱德华•泰勒(Edward Teller)可能的确具有给出权威判断的知识,但是他们在这个领域内的专业知识却不能在决定重大政治目标时赋予他们特殊的智慧。把一位杰出物理学家的强有力判断诉诸为批准某些国际条约的理据,就可能是诉诸不当权威的论证。

相似的,我们羡慕小说杰作的深度和洞识,比如,亚历山大•索尔仁尼琴(Alexander Solzhenitsyn)或索尔•贝娄(Saul Bellow)的小说中的洞识,但在某些政治争论中,诉诸他们的判断以决定真正的战争罪犯,就可能是诉诸不当权威。${ }^{[7]}$

\paragraph{确定合理权威的标准}
许多人都提出(或者由他人介绍)自己是某个领域的"专家",然而,决定谁的权威真正值得依赖却往往是个难题。假定我们想要知道某命题$p$是否真,假定某人A被认为是关于$p$或类似于$p$命题的专家,并且A说$p$是真的,那么,A的说法在什么条件下能够真正地给予我们充分的理由以接受$p$为真呢?当然,在真实事例中,答案取决于$p$断定的是什么,还取决于A和类似于$p$的诸多命题之间的关系。

\begin{theorembox}[title=确定合理权威的标准]
一般的,我们必须回答的问题是:按照知识、经验、训练或总体环境,A比我们这些正在讨论该问题、判断$p$是否为真的人\logicemph{更有能力}吗?如果是那样,那么,对我们来说,作为关于$p$为真的证据,A的判断就具有某种价值;当然,尽管A的判断可能还不是充分证据,但它却或许比其他考虑更平衡全面,或许比其他人的证词更重要,而这些人也比我们关于$p$有更多知识。
\end{theorembox}

\logicwarn{诉诸不当权威论证}就是诉诸这样的人,他无权声称比我们自身有更大能力来判断$p$的真。当然,即使一个人的确具有合法声称的权威,也很可能会被证明\logicwarn{出错},而我们以后可能后悔我们对专家的选择。但是,如果我们选择的专家无愧于对事物情况如$p$(无论$p$可能是什么)的知识名声,那么依赖他们并没有\logicwarn{谬误},即使他们是\logicwarn{错误的}。如果我们的结论以权威意见为基础,但该意见在那个问题上不能合理地宣称是专门知识,那么我们的错误就是一种推理错误即\logicwarn{谬误}。${ }^{[8]}$