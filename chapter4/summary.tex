\section*{第4章概要}
在本章中,我们看到,\textbf{谬误}是那种看起来正确但经过考察而证明并非如此的论证。我们对常见的欺骗性推理错误类型给出了传统名称,区分出三大类非形式谬误:\textbf{相干谬误}、\textbf{预设谬误}和\textbf{含混谬误}。

\subsection*{相干谬误}
在这类谬误中,错误论证依赖于看起来可能与结论相关但事实上无关的前提。我们分七种相干谬误来解释这类推理错误。

\paragraph{R1.诉诸无知论证}
当以一命题没有被证明是假的为理由来论证该命题是真的,或当论证一命题是假的因为它没有被证明是真的。

\paragraph{R2.诉诸不当权威}
一个论证的前提诉诸某方或多方判断,而它或它们却不能合法地声称对手头问题具有权威。

\paragraph{R3.人身攻击论证}
攻击不是针对所做的主张或针对论证的优点,而是针对对手本身。

人身攻击论证有两种形式。当攻击直接针对人,以寻求诋毁和侮辱他们时,就称做"诽谤性人身攻击论证"。当攻击间接地对准人,暗示他们坚持他们的观点主要是因为他们的特殊环境或利益时,就称做"背景性人身攻击论证"。

\paragraph{R4.诉诸情感}
细心推理被激起狂热或情感来支持预先结论的精心策划所取代。

\paragraph{R5.诉诸同情}
细心推理被激起听者同情来达到说者所关注目标的精心策划所取代。

\paragraph{R6.诉诸武力}
为了得到对某些结论的承诺,细心推理被直接或含沙射影的威胁所取代。

\paragraph{R7.不相干结论}
前提不得要领,声称支持一个结论而事实上却支持或证实另一个结论。

\subsection*{预设谬误}
在这类谬误中,错误论证源于依赖于某些被假定为真的命题,而这些命题实际上是假的、可疑的或没有得到证明的。我们分五种预设谬误来解释这类推理错误。

\paragraph{P1.复杂问语}
以问句预设了某些假设为真的方式来询问问题。

\paragraph{P2.虚假原因}
把一个东西当做一个事物的原因而它实际上并不是那个事物的原因,或更一般地说,在以因果关系为基础的推理中犯错。

\paragraph{P3.丕题}
在某个论证前提中假定了结论要寻求确证的东西。

\paragraph{P4.偶然}
把某个概括运用于它不能适当管辖的个别情况。

\paragraph{P5.逆偶然}
粗心大意地从单个情况转移到一个无辩护余地的广泛概括。

\subsection*{含混谬误}
在这类谬误中,错误论证的形成方式是,它依赖于词或短语从在前提中的用法到在结论中的用法的意义变化。我们分五种含混谬误来解释这类推理错误。

\paragraph{A1.歧义}
在论证的明确表述中,有意或无意地使用同一个词或短语的两个或更多意义。

\paragraph{A2.双关}
因为陈述中的词或短语结合得松散或笨拙,论证中的这个陈述具有多于一个合理意义。

\paragraph{A3.重读}
意义的变化作为对论证的词或短语的强调改变的结果而源于该论证之内。

\paragraph{A4.合成}
(a)错误地从部分性质到整体性质进行推理,(b)或者,错误地从某汇集的个别分子性质到整个汇集的性质进行推理。

\paragraph{A5.分解}
(a)错误地从整体性质到它的一个部分的性质进行推理,(b)或者,错误地从某些实体汇集的某个全体性质到该汇集的个别实体性质进行推理。

\begin{center}
\fbox{\parbox{0.9\textwidth}{
  \centering
  \textbf{谬误分类总结}\\
  \textbf{相干谬误}:前提与结论不相干,但表面上看似有关\\
  \textbf{预设谬误}:依赖未经证实或可疑的假设\\
  \textbf{含混谬误}:依赖词语或短语意义的变化\\
}}
\end{center} 

\printbibliography[heading=subbibliography,title={第4章参考文献}] 