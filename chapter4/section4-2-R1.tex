\subsection{R1.诉诸无知论证(The Argument from Ignorance:Argument Ad Ignorantiam)}

\textbf{诉诸无知论证}犯的是这样的错误,它辩称一个命题是真的,其依据仅仅是该命题并没有被证明为假,或者辩称一个命题是假的,仅仅因为并没有证明其为真。稍一思考,我们就知道,许多假命题没有被证出是假的,许多真命题也没被证出是真的,因而人们对怎样证明或否证一个命题的无知并不能证实它的真或假。但这种诉诸无知谬误,伴随着科学发展中的错误理解,是经常冒出的。在这种错误理解中,那些其真还没有得到证实的命题就因此而被某些人主张是假的;在伪科学领域中,关于通灵(psychic)及相似现象的命题被认为是真的,其理由只是它们的假并没有得到证实。

\paragraph{历史案例:伽利略的望远镜}
当伽利略用他的望远镜看到了月亮上的山脉和山谷,并力图向他那个时代的主要天文学家们进行证实时,他的批评者给出了一个在科学史上著名的诉诸无知论证。那时的一些学者绝对相信月亮是一个完美的球形,正如神学和亚里土多德学说所长期教导的那样,他们争论说,虽然我们看到的那些东西好像是山脉和山谷,但月亮实际上必定仍是一个完美的球形,因为它所有明显不规则的地方都一定充满着一种看不见的水晶般的物质——这是保全天体完美性的一种假说,而伽利略并不能证明它是假的!

据说,伽利略为了揭露这种诉诸无知论证的荒谬,仿照它提出了另一个同类的诉诸无知的论证。由于并不能证明那种设想的充满山谷的透明水晶物质的存在,他提出了一个同等可能的假说:那种看不见的水晶覆盖物上存在更高的山峰——但它是由水晶构成的,因此是看不到的!他指出,他的批评者也不能证明这个假说是假的。

\paragraph{政治变革中的诉诸无知}
那些强烈反对某种重大变革的人,常常试图以这种变革还没有被证明可行或安全为根据而反驳它。这种证明通常不可能先行给出,而诉诸这种反驳却通常是无知混合着恐惧。这种诉诸经常采取\textbf{修辞问句}的形式来暗示(但不直接断定)所提议的变革充满着未知的危险。政治变革既可以为诉诸无知所反对,有时也可以为它所支持。

1992年,当联邦政府宣布弃权,允许威斯康星州减少曾经为多于一个孩子的母亲提供的额外利益时,有人问威斯康星州长是否有证据表明有多个孩子的未婚母亲仅仅是为了得到额外收人。他回答道(诉诸无知):"不,没有。确实没有,但是也没有反面证据。"${ }^{[4]}$

\paragraph{合理应用的情况}
当然,在某些情况下,如果人们以适当方式来积极地寻找并揭示证据或结果,但之后却没有得到特定证据或结果,那么人们对这个事实就可能有实质性的争论。例如,人们通常用老鼠或其他动物实验对象对新药进行长期的安全性检验,如果对动物没有任何毒性影响,那么也就被认为是对人可能无毒的证据,尽管这不是最后结论。消费者保护就经常依赖这类证据。

在与此相似的环境中,我们依赖的不是无知而是我们的如下知识或者信念:假如会出现我们关心的结果,那么它在某些实验中就可能已经出现。这种以未能否证去确定真的证明,设定了研究者具有高度技巧:假如有那种证据的话,他们就非常可能已经发现了它。在这种情况下,有时也可能发生悲剧性错误。但是,如果标准设得过高,如果要求的证明是实际上不可能给出的最终无害的证明,那么消费者就无法享用那些可以被证明有价值的甚至挽救生命的医药治疗。

相似的,当安全性研究没有发现实验对象产生不适当行为的证据时,做出该研究使我们一无所得的结论也可能是错误的。彻底研究的正确结果将可能是"清除"原来研究的结论。在某些情况下,不做出结论与做出一个错误结论一样违反正确推理的法则。

\paragraph{法律中的诉诸无知}
诉诸无知在刑事法庭上是常用而适当的方法。美国法理学和英国普通法体系中,在证明一个在刑事法庭上受指控的人有罪之前,必须先假定他\textbf{无罪}。我们支持这个原则,因为我们认识到,宣判无罪者有罪的错误远比开释犯罪者的错误更为严重。因此在刑事案件中,辩护律师可以有权合法要求,如果被告除了合理怀疑外没有被证明有罪,那么就应裁决无罪。

美国高等法院坚定地重申了这种证明标准,它说:

合理怀疑的(限制)标准……是降低真正错误定罪危险的主要工具。该标准为无罪推定提供了坚固基石:这种基本的公理化原则是我们的刑法得以执行的基础。 ${ }^{[5]}$

但是,这种诉诸无知只适用于此类因不能证明有罪而不得不采用无罪假定的情形,在其他语境中,这样的诉诸就是诉诸无知(谬误)论证。 