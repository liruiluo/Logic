\subsection{R7.不相干结论(Irrelevant Conclusion:Ignoratio Elenchi)}

\begin{theorembox}[title=不相干结论谬误的定义与特征]
\logicwarn{基本定义}:当一个论证声称要确证一个特定的结论,但却去证明另一个与之不同的结论时,就犯有\logicterm{不相干结论谬误}。

\logicemph{词汇来源}:Ignoratio elenchi的字面意义是"错误证明"。

\logicwarn{谬误特征}:
\begin{itemize}
  \item 它的前提"不得要领"
  \item 它的推理本身可能并非不合理
  \item 但它在争论所需结论的辩护却没有效力
\end{itemize}

\logicemph{核心问题}:论证者证明了某个命题,但这个命题与原本要证明的结论不相关。
\end{theorembox}

\begin{examplebox}[title=政策辩论中的不相干结论]
\logicwarn{常见领域}:社会法律领域中的论证经常犯有这种谬误。

\logicemph{典型情况}:
\begin{itemize}
  \item 一个特殊方案的确是为某种被广泛支持的更大目标服务的
  \item 但为该方案进行论证的前提所提供的理由却只能支持那个大目标
  \item 而没有告诉我们关于那个特定方案的任何东西
\end{itemize}

\logicwarn{产生原因}:
\begin{itemize}
  \item 有时这种方式是故意为之的
  \item 有时则是由于过于热情关心那种更大目标
  \item 而认识不到现有前提与特定方案的结论并不相干
\end{itemize}
\end{examplebox}

\begin{examplebox}[title=不相干结论的常见形式]
\logicwarn{普遍性}:这种谬误在日常生活和政治辩论中非常普遍。

\logicemph{具体例子}:
\begin{itemize}
  \item \logicwarn{就业政策例子}:有人主张某政策会增加就业机会,因此应当实施,然而其论证却只证明了就业增长是件好事,而没有表明该政策实际上能够增加就业
  \item \logicwarn{技术研究例子}:有人反对某项技术研究,理由是科技发展可能带来危害,但其论证却没有指出这项特定研究会如何造成危害
\end{itemize}
\end{examplebox}

\begin{theorembox}[title=不相干结论谬误的欺骗性与识别方法]
\logicwarn{欺骗性特征}:不相干结论谬误之所以具有欺骗性,是因为它的前提往往确实支持了某个结论,只是那个结论与原本要证明的命题并不相干。

\logicemph{识别关键}:仔细辨识讨论的实际焦点,是避免这种谬误的关键。

\logicemph{防范方法}:
\begin{itemize}
  \item 明确论证要证明的具体结论是什么
  \item 检查前提是否真正支持这个特定结论
  \item 区分一般性目标与具体方案的差别
  \item 避免被表面上合理的论证所误导
\end{itemize}
\end{theorembox>