\subsection{R7.不相干结论(Irrelevant Conclusion:Ignoratio Elenchi)}

\begin{theorembox}[title=不相干结论谬误的定义]
当一个论证声称要确证一个特定的结论,但却去证明另一个与之不同的结论时,就犯有\logicwarn{不相干结论谬误}(Ignoratio elenchi的字面意义是"错误证明")。它的前提"不得要领";它的推理本身可能并非不合理,但它在争论树结论的辩护却没有效力。
\end{theorembox}

\paragraph{政策辩论中的不相干结论}
社会法律领域中的论证经常犯有这种\logicwarn{谬误}。一个特殊方案的确是为某种被广泛支持的更大目标服务的,但为该方案进行论证的前提所提供的理由却只能支持那个大目标,而没有告诉我们关于那个特定方案的任何东西。有时这种方式是故意为之的,有时则是由于过于热情关心那种更大目标,而认识不到现有前提与特定方案的结论并不\logicwarn{相干}。

\paragraph{不相干结论的常见形式}
\begin{examplebox}[title=不相干结论的常见形式]
这种\logicwarn{谬误}在日常生活和政治辩论中非常普遍。例如,有人主张某政策会增加就业机会,因此应当实施,然而其论证却只证明了就业增长是件\logicemph{好事},而没有表明该政策实际上能够增加就业。又如,有人反对某项技术研究,理由是科技发展可能带来危害,但其论证却没有指出这项特定研究会如何造成危害。
\end{examplebox}

不相干结论谬误之所以具有\logicwarn{欺骗性},是因为它的前提往往确实支持了某个结论,只是那个结论与原本要证明的命题并不\logicwarn{相干}。仔细辨识讨论的实际焦点,是避免这种谬误的关键。