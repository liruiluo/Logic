\subsection{A5.分解}

\begin{theorembox}[title=分解谬误的定义]
\logicterm{分解谬误}:是合成谬误的简单颠倒;在分解谬误中,存在相同的混淆,但推论是以相反方向进行的。

\logicwarn{两种分解谬误}:与合成的情形相应,我们也可以区分出两种分解谬误。

\logicemph{第一种分解谬误}:断言对一个整体为真的东西一定对它的部分也真。

\logicwarn{公司官员的例子}:
\begin{itemize}
  \item \logicemph{前提}:某公司非常重要,并且某先生是那个公司的官员
  \item \logicwarn{错误结论}:因此某先生就是非常重要的
  \item \logicemph{谬误分析}:这个论证就犯了分解谬误
\end{itemize}

\logicwarn{机器部件的例子}:从某机器沉重、复杂或者贵重这个前提而得出该机器的任何部分都一定沉重、复杂或者贵重,这个结论也属于分解谬误。

\logicemph{房间大小的例子}:一个学生一定住着一个大房间,因为该房间位于一座大楼中,这也是这种分解谬误的实例。

\logicwarn{核心错误}:
\begin{itemize}
  \item 错误地假设整体的性质必然属于部分
  \item 忽略了整体与部分之间的质的差异
  \item 混淆了不同层次的属性归属
  \item 违反了系统论的基本原理
\end{itemize}
\end{theorembox}

\paragraph{从整体到部分的错误推理}
\begin{examplebox}[title=从整体到部分的错误推理]
\logicwarn{第二种分解谬误}:是从元素的汇集性质而得出元素自身的性质。

\logicemph{大学生专业的例子}:
\begin{itemize}
  \item \logicwarn{前提}:大学生学习医学、法律、工程、牙科和建筑学
  \item \logicwarn{错误结论}:所以任何大学生都学习医学、法律、工程、牙科和建筑学
  \item \logicemph{谬误分析}:这个论证就犯了这种分解谬误
\end{itemize}

\logicwarn{汇集与分布的区别}:
\begin{itemize}
  \item \logicemph{汇集地看}:大学生学习所有这些科目是真的
  \item \logicwarn{分布地看}:大学生学习所有这些科目却是假的
\end{itemize}

\logicemph{谬误的欺骗性}:这种分解谬误的例子常常看起来好像是有效论证,因为对一个类分布地为真的东西,肯定对其每一成员也是真的。

\logicwarn{概念澄清}:
\begin{itemize}
  \item \logicterm{汇集性质}:整个群体作为一个整体所具有的性质
  \item \logicterm{分布性质}:群体中每个成员都具有的性质
  \item \logicterm{集合谬误}:将汇集性质错误地归属于个体成员
  \item \logicterm{量词混淆}:混淆"所有"与"每个"的不同含义
\end{itemize}
\end{examplebox}

\paragraph{有效推理与分解谬误的区别}
\begin{examplebox}[title=有效推理与分解谬误的区别]
\logicemph{有效推理的例子}:

\begin{quote}
狗是肉食的。

阿富汗猎犬都是狗。

因此,阿富汗猎犬都是肉食的。
\end{quote}

\logicwarn{有效性分析}:这个论证是有效的,因为:
\begin{itemize}
  \item "狗是肉食的"是一个分布性陈述,适用于所有狗
  \item 阿富汗猎犬属于狗这个类别
  \item 因此可以正确地推出阿富汗猎犬具有狗的分布性质
\end{itemize}

\logicemph{区别关键}:
\begin{itemize}
  \item \logicterm{有效推理}:从类的分布性质推向成员的性质
  \item \logicwarn{分解谬误}:从类的汇集性质推向成员的性质
\end{itemize}

\logicwarn{判断标准}:
\begin{itemize}
  \item 检查前提中的性质是分布性的还是汇集性的
  \item 分析该性质是否可以合理地归属于个体成员
  \item 考虑整体与部分之间的逻辑关系
  \item 避免混淆不同层次的属性归属
\end{itemize}

\logicemph{实践指导}:
\begin{itemize}
  \item 明确区分集体属性和个体属性
  \item 注意量词的准确使用
  \item 分析属性的可传递性
  \item 考虑系统的层次结构
\end{itemize}
\end{examplebox}