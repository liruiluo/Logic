\subsection{A5.分解}

\begin{theorembox}[title=分解谬误的定义]
\logicwarn{分解谬误}是\logicwarn{合成谬误}的简单颠倒;在\logicwarn{分解谬误}中,存在相同的混淆,但推论是以相反方向进行的。与合成的情形相应,我们也可以区分出两种\logicwarn{分解谬误}。第一种\logicwarn{分解谬误}断言对一个整体为\logicemph{真}的东西一定对它的部分也\logicemph{真}。因为某公司非常重要,并且某先生是那个公司的官员,因此某先生就是非常重要的,这个论证就犯了\logicwarn{分解谬误}。同样,从某机器沉重、复杂或者贵重这个前提而得出该机器的任何部分都一定沉重、复杂或者贵重,这个结论也属于\logicwarn{分解谬误}。一个学生一定住着一个大房间,因为该房间位于一座大楼中,这也是这种\logicwarn{分解谬误}的实例。
\end{theorembox}

\paragraph{从整体到部分的错误推理}
\begin{examplebox}[title=从整体到部分的错误推理]
第二种\logicwarn{分解谬误}是从元素的汇集性质而得出元素自身的性质。因为大学生学习医学、法律、工程、牙科和建筑学,所以任何大学生都学习医学、法律、工程、牙科和建筑学,这个论证就犯了这种\logicwarn{分解谬误}。汇集地看,大学生学习所有这些科目是\logicemph{真的},但分布地看,大学生学习所有这些科目却是\logicwarn{假的}。这种\logicwarn{分解谬误}的例子常常看起来好像是\logicemph{有效}论证,因为对一个类分布地为\logicemph{真}的东西,肯定对其每一成员也是\logicemph{真的}。例如如下论证:
\end{examplebox}

\paragraph{有效推理与分解谬误的区别}
\begin{examplebox}[title=有效推理与分解谬误的区别]
狗是肉食的。

阿富汗猎犬都是狗。

因此,阿富汗猎犬都是肉食的。
\end{examplebox}