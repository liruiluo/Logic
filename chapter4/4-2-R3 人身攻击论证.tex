\subsection{R3.人身攻击论证(Argument Ad Hominem)}

\begin{theorembox}[title=人身攻击论证的定义]
短语\logicterm{"ad hominem"}译做"人身攻击"。它命名的是一种\logicwarn{谬误性反驳},即它的抨击不是指向结论,而是指向断定结论或为结论辩护的人。当一个论证攻击提出主张的人而非主张本身时,就犯了\logicwarn{人身攻击谬误}。
\end{theorembox}

\paragraph{A.诽谤型人身攻击}

在激烈的论辩中,参与者有时贬低对手的品格,否认他们的智力或推理能力,质疑他们的正直,等等。但是,个人的品格与他主张的命题的\logicemph{真假}或推理的\logicemph{正误}在逻辑上并无关联。如果认为某种意见是\logicwarn{糟糕的}或断定是\logicwarn{错误的},而其原因却只是它们是由"激进派"或"极端派"提出的,那么这就构成了人身攻击谬误的一种典型特例:\logicwarn{诽谤}。

诽谤的前提与结论是不相干的,然而它却可能通过转移心理进路来说服人,可以鼓动对一个人的反对态度,情感上的反对范围甚至扩展得与鼓动者做出的判断也相对立。

\paragraph{哲学辩论中的诽谤现象}
几位当代美国哲学家之间的一场尖锐论争就例示了这种谬误攻击。其中一位论辩者写道:

被体面对手以体面方式抨击的事情,在哲学中一直出现。但是,在我看来,索莫斯(Sommers)的智力方法是不诚实的。她无视哲学争论的最基本礼仪。${ }^{[9]}$

争论的对方回答道:

几个诋毁我的人所用的一个不诚实和毫无价值的策略是,认为我从没有做过的抱怨是我所做的,然后把这些"抱怨"作为"我不负责任的和轻率不公正的证据"来打发。${ }^{[10]}$

但冲突双方所居地位的应有美德,却没有在这种论证中显示出来。诽谤性人身攻击有很多种变形。对手可能被诽谤为巧舌如簧,"孤立主义者"或"干涉主义者","极右"或"极左"分子,如此等等。当诽谤性攻击论证采用攻击对立方出身(这当然与真假无关)的形式时,就称之为\textbf{"遗传谬误"}。

\paragraph{连带罪恶与证人可信度}
有时,一个结论或它的拥护者可能会因为拥护其观点者都是那些被广泛认为品质不好的人而受到指责。在其臭名昭著的审判中,苏格拉底被判决不敬之罪,部分原因就是他与那些被广泛认为对雅典不忠和品行上贪婪的人有联系。1997年,克莱德•柯林斯•斯诺(Clyde Collins Snow)因为他在科学研究中所得出的结论而被指责为种族主义者,他回答如下:

\begin{displayquote}
在过去十年中,我的工作倾注于调研许多国家的失踪、毒打和超越法律迫害的人权受害者,这使我成了公众批评和政府撒气的靶子。然而,直到今天没有一个批评我的人把我视为种族主义者。对我的诋毁,有阿根廷(Argentina)的野蛮的军事政务会辩护者、智利(Chile)的皮诺切特(Pinochet)将军的军事代表、危地马拉的(Guatemalan)国防部长以及塞尔维亚(Serbian)政府的说客。因而,古德曼(Goodman)先生(斯诺的指责者)发现他自己处于有趣的伙伴中。${ }^{[11]}$
\end{displayquote}

不公平指责是人身诽谤的极其普通的形式;\textbf{连带罪恶}是诽谤的另一种方式,它不那么广泛但却是同等谬误。

不过,在法律程序中,有时禁止不可靠者及"存疑证人"作证乃是可取的。如果不诚实在其他问题上已显示出来并因而破坏了信用,那么在法律程序中,这种存疑在这种背景下可能不是谬误。但是,由此却不能简单地断定这种证人说的是谎话。我们必须禁止各种不诚实或欺骗,也必须揭露与过去证词的矛盾。即使在这种特殊背景下,攻击品格也不能确立所给出的证词是假的;如果那样,推理便是谬误。

\paragraph{B.背景谬误}

\begin{theorembox}[title=背景谬误的定义]
\logicwarn{背景谬误}是人身攻击谬误的一种形式。引起背景谬误的是,在本不相干的信念与该信念持有者的背景之间加以牵连。人们做出或拒绝某个主张的背景并不承载该主张为\logicemph{真}。
\end{theorembox}

\paragraph{职业与身份的背景影响}
因此,如果仅仅因为对手的职业、国籍、政治联系或其他背景,就固执地迫使对手接受或拒绝某个结论,那么这样的论证就是谬误的。如果认为圣职人员必须接受某个给定观点,因为否定它就与《圣经》相矛盾,那么这是不公平的。再如,若认为政党候选人必须支持某项政策,因为它是其所属政党的纲领中公开宣示的,这也是不公正的。这样的论证与所论及的命题真假无关,它仅仅是力促某人接受背景。

\paragraph{tu quoque与偏见指控}
\begin{examplebox}[title=tu quoque谬误示例]
有人指责猎人毫无用途地屠杀没有惹人的动物,而猎人有时却通过指出其批评者食用无害牲畜来回应。这样的回应显然是人身攻击,批评者食肉的事实与证明猎人为娱乐而猎杀动物合理性根本不沾边。拉丁术语\logicterm{"tu quoque"}(意思是"你是另一个"),有时被用来命名这种人身攻击论证的背景谬误。
\end{examplebox}

在严肃论证中,对手的背景并不是重要问题,要求注意它们可能在取赞扬或说服他人方面起心理作用。但是,无论多么有说服力,这种论证本质上都是谬误的。

有时,背景谬误被用来表明应当拒绝对手的结论,指责导致他们做出判断的是他们的特殊处境而不是推理或证据,所以他们的判断是\textbf{有偏见的}。但是,一个有利于某团体的论证,并非就没有讨论价值;若仅仅依据其被该团体成员提出从而为该团体服务为由而非难之,就是犯了背景谬误。例如,赞成保护关税的论证可能是糟糕的,但它们之所以糟糕,却并不是因为它们是由从关税保护中获得好处的制造商提出的。

\paragraph{污泉谬误}
\begin{examplebox}[title=污泉谬误的经典案例]
背景谬误论证之一,称做\logicwarn{"污泉"}(poisoning the well),尤为悖理。产生这个名字的事件典型地例示了这种论证。英国小说家和教士查尔斯•金斯利(Charles Kingsley)攻击著名的天主教智者约翰•亨利•卡迪拉尔•纽曼(John Henry Cardinal Newman)说,卡迪拉尔•纽曼的主张是不能信任的,因为作为一名罗马天主教的牧师,他首先要忠诚的不是真理。纽曼反驳道,这种人身攻击使他并且也使全体天主教徒的进一步论辩成为不可能,因为他们为自己辩护所说的任何东西都可以因被他人指责为根本不关心真理而遭到拒斥。卡迪拉尔•纽曼说,金斯利"污染了对话之泉"。
\end{examplebox}

人身攻击论证的诽谤谬误和背景谬误之间,存在一种清晰的联系:背景谬误可以被看做诽谤谬误的一种特殊情况。譬如,当使用背景谬误明显或暗含地指责对手缺乏一贯性(在他们的信念中,或者在他们的言行之间),它很明显就是一种诽谤;而用背景谬误指责对手由于其属于某集团或具有集团信仰而缺乏信任价值,显然也是指责对手具有自利偏见的诽谤手段。无论何种形式,人身攻击论证都是对论辩对手的谬误性诋毁。