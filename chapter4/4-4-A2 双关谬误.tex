\subsection{A2.双关}

\begin{theorembox}[title=双关谬误的定义与特征]
\logicwarn{产生原因}:由于前提的语法结构原因,会导致前提的表达歧义。当人们从这样的前提出发来论证时,就会出现\logicterm{双关}(amphiboly)谬误。

\logicemph{词汇来源}:"双关"这个词来源于希腊语,它的意思实质是"一团两个",或一团的"两倍"。

\logicterm{双关陈述的特征}:一个陈述是双关的,是指由于它的词汇组合松散或笨拙导致它的意义不确定。

\logicwarn{真假的相对性}:一个双关陈述可能在一种解释下可能是真的,而在另一种解释下却是假的。

\logicterm{双关谬误的机制}:当以使其为真的解释来表述论证前提,而以使其为假的解释得出结论时,那么就犯了双关谬误。

\logicemph{与歧义谬误的区别}:
\begin{itemize}
  \item \logicterm{歧义谬误}:由于词汇的多重含义造成
  \item \logicterm{双关谬误}:由于语法结构的模糊性造成
  \item 双关谬误更多涉及句法层面的歧义
  \item 歧义谬误更多涉及词汇层面的歧义
\end{itemize}
\end{theorembox>

\paragraph{政治中的双关现象}
\begin{examplebox}[title=政治中的双关现象]
\logicwarn{政治应用}:在指导选举策略时,双关既可以迷惑人也可以误导人。

\logicemph{历史案例}:20世纪90年代,当众议员托尼·科埃略(Tony Coelho)作为来自加利福尼亚州的一位民主党员而进入美国白宫代表中时,据报道,他说:

\begin{quote}
"Women prefer Democrats to men."$^{(1)}$
\end{quote}

\logicwarn{双重解释}:这句话可以有两种解释:
\begin{itemize}
  \item \logicemph{解释1}:女人比男人更喜欢民主党
  \item \logicemph{解释2}:女人更喜欢民主党而不是男人
\end{itemize}

\logicemph{危险性评估}:双关陈述构成危险前提,但是,在严肃的话题中人们很少遭遇它。

\logicwarn{政治策略}:在政治语境中,这种语言模糊性可能被故意利用来获得不同群体的支持。
\end{examplebox}

\paragraph{垂悬分词与短语}
\begin{examplebox}[title=垂悬分词与短语]
\logicwarn{语法现象}:文法家所谓的"垂悬"分词和短语经常有娱乐类型的双关出现。

\logicemph{媒体案例}:《纽约客》(The New Yorker)中的小栏新闻就曾给粗心忽视了双关的作者和编辑开了一个讽刺玩笑:

\begin{quote}
"Leaking badly, manned by a skeleton crew, one infirmity after another overtakes the little ship."$^{(2)}$ (The Herald Tribune, book section)
\end{quote>

\logicwarn{双重解释}:这句话可有两种解释:
\begin{itemize}
  \item \logicemph{解释1}:小船突然出现严重泄漏、配备人员最少等一个接一个的缺陷
  \item \logicemph{解释2}:严重泄漏、配备人员最少等一个接一个的(游戏)缺陷,突然降临小船
\end{itemize}

\logicemph{语法分析}:前者是指小船的缺陷,而后者意指游戏自身的缺陷。

\logicwarn{讽刺评论}:这些游戏几乎没有缺陷!$^{[30]}$

\logicemph{教训}:这个例子说明了精确的语法结构在避免误解方面的重要性。
\end{examplebox}

\footnotetext{(1)这句话可以有两种解释:女人比男人更喜欢民主党;女人更喜欢民主党而不是男人。\\
(2)这句话可有两种解释:小船突然出现严重泄漏、配备人员最少等一个接一个的缺陷;严重泄漏、配备人员最少等一个接一个的(游戏)缺陷,突然降临小船。前者是指小船的缺陷,而后者意指游戏自身的缺陷。
}