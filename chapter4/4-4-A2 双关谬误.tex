\subsection{A2.双关}

\begin{theorembox}[title=双关谬误的定义]
由于前提的语法结构原因,会导致前提的表达\logicterm{歧义}。当人们从这样的前提出发来论证时,就会出现\logicterm{双关}(amphiboly)\logicwarn{谬误}。"双关"这个词来源于希腊语,它的意思实质是"一团两个",或一团的"两倍"。一个陈述是\logicterm{双关的},是指由于它的词汇组合松散或笨拙导致它的意义不确定。一个\logicterm{双关}陈述可能在一种解释下可能是\logicemph{真的},而在另一种解释下却是\logicwarn{假的}。当以使其为\logicemph{真}的解释来表述论证前提,而以使其为\logicwarn{假}的解释得出结论时,那么就犯了\logicwarn{双关谬误}。
\end{theorembox}

\paragraph{政治中的双关现象}
\begin{examplebox}[title=政治中的双关现象]
在指导选举策略时,\logicterm{双关}既可以迷惑人也可以误导人。20世纪90年代,当众议员托尼•科埃略(Tony Coelho)作为来自加利福尼亚州的一位民主党员而进入美国白宫代表中时,据报道,他说:"Women prefer Democrats to men."${ }^{(1)}$ \logicterm{双关}陈述构成危险前提,但是,在严肃的话题中人们很少遭遇它。
\end{examplebox}

\paragraph{垂悬分词与短语}
\begin{examplebox}[title=垂悬分词与短语]
文法家所谓的"垂悬"分词和短语经常有娱乐类型的\logicterm{双关}出现。《纽约客》(The New Yorker)中的小栏新闻就曾给粗心忽视了\logicterm{双关}的作者和编辑开了一个讽刺玩笑:

"Leaking badly,manned by a skeleton crew,one infirmity after another overtakes the little ship."${ }^{(2)}$(The Herald Tribune, book section)

这些游戏几乎没有缺陷![30]
\end{examplebox}

\footnotetext{(1)这句话可以有两种解释:女人比男人更喜欢民主党;女人更喜欢民主党而不是男人。\\
(2)这句话可有两种解释:小船突然出现严重泄漏、配备人员最少等一个接一个的缺陷;严重泄漏、配备人员最少等一个接一个的(游戏)缺陷,突然降临小船。前者是指小船的缺陷,而后者意指游戏自身的缺陷。
}