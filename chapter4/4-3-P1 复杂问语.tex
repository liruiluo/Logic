\subsection{P1.复杂问语(Complex Question)}

\begin{theorembox}[title=复杂问语谬误的定义与机制]
\logicwarn{基本定义}:\logicterm{复杂问语谬误}发生在提问的方式预设了某些未被证实的假设为真的情况。

\logicwarn{谬误目的}:这种谬误的目的是使听者或读者接受事先被预设的假设,而无需对这些假设进行质疑或分析。

\logicemph{发生条件}:当一个问题含有一个或多个预设,而这些预设尚未被证明或接受时,就犯了复杂问语谬误。

\logicwarn{逻辑错误}:
\begin{itemize}
  \item 将有争议的假设作为既定事实
  \item 强迫回答者在错误前提下进行选择
  \item 绕过了对关键假设的论证过程
  \item 利用问题的形式来隐藏未经证实的断言
\end{itemize}
\end{theorembox}

\paragraph{经典的复杂问语例子}
\begin{examplebox}[title=经典的复杂问语例子]
\logicwarn{最著名的例子}:"你什么时候停止殴打你的妻子?"

\logicemph{陷阱分析}:
\begin{itemize}
  \item 无论对方回答"昨天"、"上个月"或者"去年",他都会承认自己曾经殴打妻子
  \item 即使他回答"我从未殴打我的妻子",这种回答也会显得不自然和可疑
  \item 问题预设了"他曾经殴打妻子"这一命题,而这一命题正是需要被证实的
\end{itemize}

\logicwarn{隐含假设}:
\begin{itemize}
  \item 他有妻子
  \item 他曾经殴打过妻子
  \item 他现在已经停止了这种行为
\end{itemize}

\logicemph{心理效果}:这种问题将回答者置于极其不利的位置,无论如何回答都会陷入困境。
\end{examplebox}

\paragraph{司法询问中的复杂问语}
\begin{examplebox}[title=司法询问中的复杂问语]
\logicwarn{法庭应用}:在法庭审讯中,律师常常试图诱使证人回答复杂问题。

\logicemph{酒驾案例}:
\begin{itemize}
  \item \logicwarn{问题}:"你在事故发生前喝了多少酒?"
  \item \logicwarn{预设}:被告事故前喝了酒
  \item \logicemph{律师的反击}:如果证人回答说他没喝酒,律师可以说:"我没问你是否喝了酒,我问你喝了多少。"
\end{itemize}

\logicemph{家暴案例}:
\begin{itemize}
  \item \logicwarn{问题}:"你多久打你的妻子一次?"
  \item \logicwarn{正确问法}:应该先问:"你是否曾经打过你的妻子?"
  \item \logicwarn{预设错误}:这种询问方式预设了被告确实曾经殴打过自己的妻子
\end{itemize}

\logicemph{策略分析}:律师通过复杂问语试图让对方在不知不觉中承认未经证实的事实。
\end{examplebox}

\paragraph{政治辩论中的预设问题}
\begin{examplebox}[title=政治辩论中的预设问题]
\logicwarn{政治应用}:政治辩论中的问题经常预设争议性的假设为真。

\logicemph{国家衰退案例}:
\begin{itemize}
  \item \logicwarn{问题}:"你认为哪些人应该承担我们国家衰退的责任?"
  \item \logicwarn{预设}:"我们的国家正在衰退"
  \item \logicemph{问题分析}:这个假设可能本身就是争议的焦点
\end{itemize}

\logicemph{官员错误案例}:
\begin{itemize}
  \item \logicwarn{问题}:"你准备采取什么措施来避免再次犯重大错误?"
  \item \logicwarn{预设}:官员已经犯了重大错误
  \item \logicemph{策略效果}:迫使官员在承认错误的前提下进行辩护
\end{itemize}

\logicemph{政治目的}:这类问题常用于政治攻击,通过预设对手的负面行为来获得辩论优势。
\end{examplebox}

\begin{theorembox}[title=识别和避免复杂问语的方法]
\logicemph{基本原则}:要避免被复杂问语所误导,我们应当始终审视问题中的隐含假设,而不是立即着手回答问题。

\logicwarn{问题分解法}:复杂问题往往可以分解为更基本的问题,例如:
\begin{itemize}
  \item 原问题:"你什么时候停止殴打你的妻子?"
  \item 分解为:
  \begin{enumerate}
    \item "你有妻子吗?"
    \item "如果有,你是否曾经殴打过她?"
    \item "如果是,你什么时候停止这样做的?"
  \end{enumerate}
\end{itemize}

\logicemph{分解的价值}:这种分解可以使人避免无意中认可未经证实的预设。

\logicwarn{心理机制}:复杂问语的效力来自于它的心理压力——它迫使回答者处于不利的防御位置。

\logicemph{逻辑分类}:在逻辑学中,复杂问语谬误属于\logicterm{预设谬误},因为它的错误在于预设了未经证实的命题为真。

\logicwarn{识别关键}:识别这类谬误的关键是识别问题中所隐含的所有假设,并对其真实性提出质疑。

\logicemph{应对策略}:
\begin{itemize}
  \item 拒绝回答包含未经证实预设的问题
  \item 要求提问者首先证明其假设的正确性
  \item 将复杂问题分解为简单的、可验证的子问题
  \item 保持理性分析,不被情感压力所左右
\end{itemize}
\end{theorembox>