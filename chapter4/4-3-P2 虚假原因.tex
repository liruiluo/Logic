\subsection{P2.虚假原因(False Cause)}

\begin{theorembox}[title=虚假原因谬误的定义与特征]
\logicwarn{基本错误}:显然,把实际上不是某情形或事件的原因当做原因,任何依赖于此的推理就必定是严重错误的。

\logicemph{认知倾向}:但是,我们常常倾向于假设或者被引导假设,我们理解了事实上我们并不理解的某些特定的原因—结果关系。

\logicwarn{谬误本质}:设定一个并不真实存在的因果联系,是一种常见的错误。

\logicemph{术语来源}:在拉丁语中,这种错误被称为"\logicterm{无因之因}"(non causa pro causa)的谬误,我们简单地称之为\logicterm{虚假原因谬误}。

\logicwarn{核心问题}:
\begin{itemize}
  \item 错误地建立因果关系
  \item 将相关性误认为因果性
  \item 忽略真正的原因
  \item 基于不充分的证据得出因果结论
\end{itemize}
\end{theorembox}

\begin{logicbox}[title=因果关系研究的重要性]
原因与结果之间的联系本性,以及我们怎样确定这样的联系是否存在或缺乏,都是归纳逻辑和科学方法论的中心问题。这些问题在本书的第三部分将进行详细讨论。
\end{logicbox}

\begin{examplebox}[title=因果关系与争议:评分膨胀案例]
\logicwarn{争议性问题}:所断定的因果联系是否的确错误,有时可能是有争议的问题。

\logicemph{理论假设}:有人辩说,有些大学教员评分宽松,是因为他们担心严格评分会导致学生降低对他们的评价,因而不利于他们的职业。逐渐的"评分膨胀"据说就是这种担心的结果。

\logicemph{教授的证词}:一位大学教授写道:

\begin{quote}
"现在,很多学校都要求由学生来完成课程评价表,并且薪水受这些结果的影响。30年前,我来密歇根大学时,我的薪水比人类学系任何人的都高,他们今天都还很活跃。我的评分标准没有追随膨胀潮流。学生对评分的抱怨增加了,而现在我的薪水就处在教授工资单的底层。"$^{[24]}$
\end{quote}

\logicwarn{思考问题}:你认为这段话犯有虚假原因谬误吗?

\logicemph{分析要点}:
\begin{itemize}
  \item 是否存在其他可能的原因解释薪水差异?
  \item 时间相关性是否等同于因果关系?
  \item 是否有足够的证据支持因果联系?
  \item 个人经历是否能代表普遍规律?
\end{itemize}
\end{examplebox>

\begin{theorembox}[title=时间连续性与因果错误]
\logicwarn{常见现象}:有时会发生这种现象:人们假定一事件是另一事件的原因,只因为另一事件在时间上紧随着前者。

\logicemph{理性认知}:我们当然知道,纯粹的时间连续并不能确证一种因果联系,但是很容易被欺骗。

\logicwarn{政策领域的例子}:在对外政策中,如果一项挑战性动议之后跟随出现了一件与其并不相关的国际事件,那么有人就可能错误地得出结论:挑战性动议就是那个事件的原因。

\logicemph{历史科学的错误}:在原始科学中,这样的错误是常见的;现在,我们把这种事件作为荒谬的说法来拒斥:

\logicwarn{荒谬例子}:敲锣打鼓是日食之后太阳重又出现的原因,因为不可否认在日食时每次敲锣打鼓之后太阳的确又会复现。

\logicemph{逻辑错误}:
\begin{itemize}
  \item 将时间先后关系误认为因果关系
  \item 忽略了真正的科学解释
  \item 基于表面现象得出错误结论
  \item 缺乏对复杂现象的深入理解
\end{itemize}
\end{theorembox}

\begin{examplebox}[title=缘出前物谬误的现代例子]
\logicwarn{广泛存在}:这类推理错误仍然广泛存在:

\logicemph{日常生活中的例子}:
\begin{itemize}
  \item 反常的天气状况被归咎于某些发生在前的不相关天象
  \item 实际上由病毒引起的感染,却被认为是伤风或湿脚所使然
  \item 等等
\end{itemize}

\logicterm{术语定义}:这种虚假原因被称为"\logicterm{缘出前物}"(post hoc ergo propter hoc)。

\logicwarn{现代案例}:近来,一名通讯员在给《纽约时报》的一封信中就出现了一个这样的例子。他写道:

\begin{quote}
"在工业世界中,美国的死刑带给我们的是,每100000人中最高的犯罪率和数量最多的囚犯。"$^{[25]}$
\end{quote}

\logicemph{分析}:这个例子暗示死刑的存在导致了高犯罪率,但这种因果关系是有问题的。

\logicwarn{普遍性问题}:当"缘出前物"非常明显时,它是一种容易发现的谬误;但是,甚至最伟大的科学家和政治家偶尔也会被它误导。

\logicemph{识别要点}:
\begin{itemize}
  \item 区分时间先后与因果关系
  \item 寻找其他可能的解释
  \item 考虑反向因果关系的可能性
  \item 注意第三因素的影响
\end{itemize}
\end{examplebox>