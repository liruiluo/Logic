\subsection{R4.诉诸情感(The Appeal to Emotion:Argument Ad Populum)}

\begin{logicbox}[title=引言]
\textit{这种常见谬误和接下来的两种谬误都非常明显,只需稍加说明。在三种情形中,前提都明显地与结论不相干,都是故意用来操纵听者或读者信念的工具。}
\end{logicbox}

\begin{theorembox}[title=诉诸情感谬误的定义与特征]
\logicemph{词汇来源}:\logicterm{诉诸情感}(Argument Ad Populum)的字面意义是"诉诸人群",意蕴诉诸情感容易激动的无序民众。

\logicwarn{谬误定义}:当论证试图通过激发听众的情感而非提供理性证据来说服时,就犯了\logicterm{诉诸情感谬误}。

\logicwarn{谬误机制}:这类论证之所以是谬误,是因为它用表达性语言和其他有计划的手段以博取情感,激起兴奋、愤怒或憎恨,而不是致力于提出证据和合理论证。

\logicemph{历史例证}:阿道夫·希特勒的讲演,激发德国听众达到一种狂热爱国状态,可以作为一种经典范例。

\logicwarn{情感操控的问题}:
\begin{itemize}
  \item 爱国是一种可敬的高尚情感
  \item 通过不适宜地诉诸它来操控听众,在智力上是低劣的
  \item 萨缪尔·约翰森挖苦地说:"爱国主义是恶棍的最后避难所。"
\end{itemize}
\end{theorembox}

\paragraph{广告中的诉诸情感}
\begin{examplebox}[title=商业广告中的诉诸情感]
\logicwarn{广告中的极端表现}:最为严重的诉诸情感可以在商业广告中找到,那里的运用几乎达到出神入化的境地。

\logicemph{情感联系策略}:广告的产品都或明或暗地与我们渴望的或惹人好感的事物相联系:
\begin{itemize}
  \item 早餐的麦片粥与健美年轻、体魄健壮和精力充沛相联系
  \item 威士忌与豪华和成就相联系
  \item 啤酒与崇高冒险相联系
  \item 汽车与浪漫、富有和性感相联系
\end{itemize}

\logicwarn{形象塑造}:广告产品描绘出的男人一般都是英俊而杰出,女人精明而迷人——或者干脆一丝不挂。

\logicemph{影响力评估}:
\begin{itemize}
  \item 我们这个时代广告艺术家的聪明和持之以恒足以使我们全部都在某种程度上受了影响,尽管我们决心抵制
  \item 几乎各种想象不到的手段都可以用来支配我们的注意力,甚至渗透到我们的潜意识之中
  \item 我们不断地被各种\logicwarn{诉诸情感谬误}所操纵
\end{itemize}
\end{examplebox}

\begin{theorembox}[title=产品与情感的隐性联系]
\logicemph{联系与论证的区别}:就其本身来说,产品与情感的纯粹联系并不是论证,但是,诉诸情感论证通常就存在于那种表面现象之下的不远处。

\logicwarn{隐含论证的谬误}:当广告者声称他们的产品设计是为赢得我们的情感赞赏时,当它表明我们应该购买因为这些产品性感或畅销或者与财富或能力相联系时,它就隐含地断言了该结论来自这种前提,这种断言就显然是谬误的。

\logicwarn{厚颜无耻的例子}:关于诉诸情感论证,有些例子是厚颜无耻的。下面是最近ABC-TV广告的原有语句:

\begin{quote}
"为什么庞蒂克汽车大奖吸引如此多的人们?是因为庞蒂克大奖吸引了如此多的人们,是因为一大奖吸引了如此多的人们!"
\end{quote}
\end{theorembox}

\begin{examplebox}[title=民意调查中的情感影响]
\logicwarn{民意调查中的危害}:在民意调查中,诉诸大众热情尤其有害;在那里,众所周知的某些特定词汇$^{[12]}$的情感影响(消极的或积极的)可以使所设计的问题本身就产生出要寻找的答案。

\logicemph{实际案例}:美国公众对减税和联邦政府如何支配即将出现的预算盈余持什么态度呢?这取决于你怎么询问。这个问题在2000年1月呈给了随机抽样的民众——两种不同的措辞$^{[13]}$:

\logicwarn{措辞1}:
\begin{quote}
"(在行将盈余的资金中)其大部分应当用于减税,还是应当用做新的政府计划的资金呢?"
\end{quote}

\logicemph{结果1}:对这个问题的这种措辞,60\%的被抽查者回答说"减税",而25\%回答"新计划"。毫不惊奇:"新的政府计划"明显不受很多人欢迎。

\logicwarn{措辞2}:
\begin{quote}
"(在行将盈余的资金中)其大部分应当用于减税,还是应当花费在教育、环境、保健、打击犯罪和军事防御等新计划上?"
\end{quote}

\logicemph{结果2}:对该问题的这种措辞,22\%的被抽查者回答"减税",69\%回答"新计划"。再次毫无惊奇:"教育、保健和打击犯罪等等"都是人们熟知的深受广泛欢迎的词汇。
\end{examplebox}

\begin{theorembox}[title=大众接受不等于真理]
\logicwarn{基本原则}:当然,一项政策的广为接受并不代表它明智,很多人都持某种观点这种事实也不能证明它就是真的。

\logicemph{罗素的批评}:罗素曾以有点过头的语言抨击了这种论证:

\begin{quote}
"一个为人们广泛持有的观点这种事实,无论如何都并非它不是完全荒谬的证据;实际上,在多数人的愚蠢观念中的广为接受的信念,更可能是愚笨多于明智。"$^{[14]}$
\end{quote}

\logicwarn{逻辑要点}:
\begin{itemize}
  \item 大众的支持并不能证明观点的正确性
  \item 流行程度与真理性之间没有必然联系
  \item 理性论证才是判断观点正误的标准
\end{itemize}
\end{theorembox}