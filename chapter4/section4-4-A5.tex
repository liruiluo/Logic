\subsection{A5.分解}

\textbf{分解谬误}是合成谬误的简单颠倒;在分解谬误中,存在相同的混淆,但推论是以相反方向进行的。与合成的情形相应,我们也可以区分出两种分解谬误。第一种分解谬误断言对一个整体为真的东西一定对它的部分也真。因为某公司非常重要,并且某先生是那个公司的官员,因此某先生就是非常重要的,这个论证就犯了分解谬误。同样,从某机器沉重、复杂或者贵重这个前提而得出该机器的任何部分都一定沉重、复杂或者贵重,这个结论也属于分解谬误。一个学生一定住着一个大房间,因为该房间位于一座大楼中,这也是这种分解谬误的实例。

\paragraph{从整体到部分的错误推理}
第二种分解谬误是从元素的汇集性质而得出元素自身的性质。因为大学生学习医学、法律、工程、牙科和建筑学,所以任何大学生都学习医学、法律、工程、牙科和建筑学,这个论证就犯了这种分解谬误。汇集地看,大学生学习所有这些科目是真的,但分布地看,大学生学习所有这些科目却是假的。这种分解谬误的例子常常看起来好像是有效论证,因为对一个类分布地为真的东西,肯定对其每一成员也是真的。例如如下论证:

\paragraph{有效推理与分解谬误的区别}
狗是肉食的。\\
阿富汗猎犬都是狗。\\
因此,阿富汗猎犬都是肉食的。 