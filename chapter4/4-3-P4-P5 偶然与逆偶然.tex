\subsection{P4.和 P5.偶然和逆偶然(Accident and Converse Accident)}

\textbf{偶然}和\textbf{逆偶然谬误}${ }^{(1)}$,可能出于无心之失,也可能是故意而做出误导他人的概括。在许多重要情形中,尤其在政治或伦理论证中,我们都要依赖关于事物的概括陈述,关于人的概括陈述,等等。但是,即使概括断言完全可行的地方,我们也必须小心不要把它们机械地或僵硬地运用于特殊事例。环境改变事例情况。一个总体上真的概括,由于给定事例的特殊或偶然环境,可能不能运用于该事例。当我们把一个概括运用于个别事例中而该事例并不适于这种运用时,我们就犯了\textbf{偶然谬误}。反之,当我们无心或故意地把对一个特殊事例为真的东西直接看做对大量事例为真,我们就犯了\textbf{逆偶然谬误}。

\paragraph{概括的例外性}
经验教导我们,概括,即使是那些广泛合适和有用的概括,往往也有例外,我们必须对之保持警惕。在法律中,总体上良好的规则有时却发现有非常特别的例外。例如,按照法理,传闻信息不能接受为庭审证据,但如果说这番话的一方已死,或报告传闻的一方是在与自身利益有很大冲突的情况下做此报告的,则这条规则就是不适用的。几乎所有良好的法律规则都有其相应的例外;当我们假设某些规则具有普遍的适用性并以该假设进行推理时,我们就可能犯谬误论证。\\

\paragraph{苏格拉底与欧西德姆斯的对话}
欧西德姆斯(Euthydemus)想成为一名政治家。在与其对话中,苏格拉底从欧西德姆斯那里获得了后者对很多习俗上接受的伦理真理的承诺,如欺骗是错误的,偷盗是不正义的,等等。接着,正如色诺芬(Xenophon)在其记载中所详述的那样,苏格拉底提出一系列假设事例,使欧西德姆斯不得不同意欺骗有时(为拯救我们的同胞)是正当的,以及偷盗有时(为挽救一个朋友的生命)是正义的,等等。对那些通过机械地诉诸概括规则来试图决定特定和复杂问题的人来说,偶然谬误是一种真实而严重的威胁。逻辑学家约瑟夫(H.W.B.Joseph)观察到"如果对待一个在很多方面都不令人误解的陈述,就好像它总是正确的没有限制条件的一样,那么没有比这种谬误更暗中为害的啦"。

\paragraph{由特殊到一般的错误推理}
偶然谬误是当我们轻率地从一个概括转移到(特殊问题)时所犯的谬误,而逆偶然谬误是当我们轻率地(从特殊问题)转移到概括时所犯的谬误。我们都熟悉这样一些人,由于某情形对一给定类型的一个或几个人是真的,他们就对那种类型的所有人做出结论。我们知道,并且需要记住,虽然在某些情况下一定的药物或食物可以是无害的,但是,它并不因而在所有的情况下就都是无害的。例如,食用油炸食物总体来说对一个人的胆固醇水平具有不利影响,但是,那种坏结果在某些人身上可能不会出现。近来,英国的一位"炸鱼片和炸土豆条"店主用如下论证为其油炸烹调方法的正当做了辩护:

\begin{displayquote}
以我的儿子马丁为例。他一生一直吃炸鱼片和炸土豆条,他刚进行了胆固醇测试,他的胆固醇水平低于国家平均水平。还有什么比一个油炸食品店主的儿子是更好的证据呢?${ }^{[27]}$
\end{displayquote}

逆偶然谬误作为推理谬误的一种,其错误一旦揭露出来,对每个人来说都是明显易懂的,然而,它却可以用做一种方便的欺骗方法。当人们漫不经心地或充满感情地进行论证时,就很可能落人这种谬误的陷阱。

\footnotetext{(1)在中文文献中,这两种谬误亦称为"以全概偏"和"以偏概全"。
} 