\subsection{P4.和 P5.偶然和逆偶然(Accident and Converse Accident)}

\begin{theorembox}[title=偶然和逆偶然谬误的定义与特征]
\logicemph{基本性质}:\logicterm{偶然}和\logicterm{逆偶然谬误}$^{(1)}$,可能出于无心之失,也可能是故意而做出误导他人的概括。

\logicwarn{应用背景}:在许多重要情形中,尤其在政治或伦理论证中,我们都要依赖关于事物的概括陈述,关于人的概括陈述,等等。

\logicemph{谨慎原则}:但是,即使概括断言完全可行的地方,我们也必须小心不要把它们机械地或僵硬地运用于特殊事例。

\logicwarn{环境因素}:环境改变事例情况。一个总体上真的概括,由于给定事例的特殊或偶然环境,可能不能运用于该事例。

\logicterm{偶然谬误}:当我们把一个概括运用于个别事例中而该事例并不适于这种运用时,我们就犯了偶然谬误。

\logicterm{逆偶然谬误}:反之,当我们无心或故意地把对一个特殊事例为真的东西直接看做对大量事例为真,我们就犯了逆偶然谬误。

\logicemph{中文术语}:在中文文献中,这两种谬误亦称为"以全概偏"和"以偏概全"。

\logicwarn{核心区别}:
\begin{itemize}
  \item \logicterm{偶然谬误}:从一般规则错误地推向特殊情况
  \item \logicterm{逆偶然谬误}:从特殊情况错误地推向一般规则
\end{itemize}
\end{theorembox}

\paragraph{概括的例外性}
\begin{examplebox}[title=概括的例外性]
\logicwarn{经验教训}:经验教导我们,概括,即使是那些广泛合适和有用的概括,往往也有例外,我们必须对之保持警惕。

\logicemph{法律中的例外}:在法律中,总体上良好的规则有时却发现有非常特别的例外。

\logicwarn{传闻证据规则的例外}:例如,按照法理,传闻信息不能接受为庭审证据,但在以下情况下这条规则就是不适用的:
\begin{itemize}
  \item 如果说这番话的一方已死
  \item 或报告传闻的一方是在与自身利益有很大冲突的情况下做此报告的
\end{itemize}

\logicemph{普遍原则}:几乎所有良好的法律规则都有其相应的例外。

\logicwarn{谬误的产生}:当我们假设某些规则具有普遍的适用性并以该假设进行推理时,我们就可能犯谬误论证。

\logicemph{实践意义}:
\begin{itemize}
  \item 需要考虑具体情境的特殊性
  \item 避免机械地应用一般规则
  \item 保持对例外情况的敏感性
  \item 在特殊情况下灵活调整判断
\end{itemize}
\end{examplebox}

\paragraph{苏格拉底与欧西德姆斯的对话}
\begin{examplebox}[title=苏格拉底与欧西德姆斯的对话]
\logicemph{历史背景}:欧西德姆斯(Euthydemus)想成为一名政治家。

\logicwarn{苏格拉底的策略}:在与其对话中,苏格拉底从欧西德姆斯那里获得了后者对很多习俗上接受的伦理真理的承诺,如:
\begin{itemize}
  \item 欺骗是错误的
  \item 偷盗是不正义的
  \item 等等
\end{itemize}

\logicemph{反例的提出}:接着,正如色诺芬(Xenophon)在其记载中所详述的那样,苏格拉底提出一系列假设事例,使欧西德姆斯不得不同意:
\begin{itemize}
  \item 欺骗有时(为拯救我们的同胞)是正当的
  \item 偷盗有时(为挽救一个朋友的生命)是正义的
  \item 等等
\end{itemize}

\logicwarn{偶然谬误的威胁}:对那些通过机械地诉诸概括规则来试图决定特定和复杂问题的人来说,偶然谬误是一种真实而严重的威胁。

\logicemph{约瑟夫的观察}:逻辑学家约瑟夫(H.W.B. Joseph)观察到:

\begin{quote}
"如果对待一个在很多方面都不令人误解的陈述,就好像它总是正确的没有限制条件的一样,那么没有比这种谬误更暗中为害的啦"。
\end{quote}

\logicwarn{哲学意义}:这个对话揭示了道德规则的复杂性和情境依赖性,说明了绝对化道德原则的危险。
\end{examplebox}

\paragraph{由特殊到一般的错误推理}
\begin{theorembox}[title=由特殊到一般的错误推理]
\logicwarn{谬误对比}:
\begin{itemize}
  \item \logicterm{偶然谬误}:当我们轻率地从一个概括转移到特殊问题时所犯的谬误
  \item \logicterm{逆偶然谬误}:当我们轻率地从特殊问题转移到概括时所犯的谬误
\end{itemize}

\logicemph{常见现象}:我们都熟悉这样一些人,由于某情形对一给定类型的一个或几个人是真的,他们就对那种类型的所有人做出结论。

\logicwarn{重要提醒}:我们知道,并且需要记住,虽然在某些情况下一定的药物或食物可以是无害的,但是,它并不因而在所有的情况下就都是无害的。

\logicemph{具体例子}:例如,食用油炸食物总体来说对一个人的胆固醇水平具有不利影响,但是,那种坏结果在某些人身上可能不会出现。
\end{theorembox}

\begin{examplebox}[title=逆偶然谬误的典型例子]
\logicwarn{现实案例}:近来,英国的一位"炸鱼片和炸土豆条"店主用如下论证为其油炸烹调方法的正当做了辩护:

\begin{quote}
"以我的儿子马丁为例。他一生一直吃炸鱼片和炸土豆条,他刚进行了胆固醇测试,他的胆固醇水平低于国家平均水平。还有什么比一个油炸食品店主的儿子是更好的证据呢?"$^{[27]}$
\end{quote}

\logicwarn{谬误分析}:
\begin{itemize}
  \item 用单一个案代替统计证据
  \item 忽略了个体差异和其他影响因素
  \item 将特殊情况错误地推广为一般规律
  \item 可能存在利益相关的偏见
\end{itemize}
\end{examplebox>

\begin{theorembox}[title=逆偶然谬误的特征与危害]
\logicwarn{识别特征}:逆偶然谬误作为推理谬误的一种,其错误一旦揭露出来,对每个人来说都是明显易懂的。

\logicemph{欺骗性}:然而,它却可以用做一种方便的欺骗方法。

\logicwarn{易犯情境}:当人们漫不经心地或充满感情地进行论证时,就很可能落入这种谬误的陷阱。

\logicemph{防范方法}:
\begin{itemize}
  \item 要求更大的样本量
  \item 考虑反例和例外情况
  \item 区分个案与统计趋势
  \item 保持理性和客观的态度
\end{itemize}
\end{theorembox}

\footnotetext{(1)在中文文献中,这两种谬误亦称为"以全概偏"和"以偏概全"。
}