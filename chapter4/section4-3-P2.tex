\section*{P2.虚假原因(False Cause)}
显然,把实际上不是某情形或事件的原因当做原因,任何依赖于此的推理就必定是严重错误的。但是,我们常常倾向于假设或者被引导假设,我们理解了事实上我们并不理解的某些特定的原因—结果关系。原因与结果之间的联系本性,以及我们怎样确定这样的联系是否存在或缺乏,都是归纳逻辑和科学方法论的中心问题。这些问题在本书的第三部分将进行详细讨论。设定一个并不真实存在的因果联系,是一种常见的错误。在拉丁语中,这种错误被称为"无因之因"(non causa pro causa)的谬误,我们简单地称之为虚假原因谬误。

所断定的因果联系是否的确错误,有时可能是有争议的问题。有人辩说,有些大学教员评分宽松,是因为他们担心严格评分会导致学生降低对他们的评价,因而不利于他们的职业。逐渐的"评分膨胀"据说就是这种担心的结果。一位大学教授写道:

现在,很多学校都要求由学生来完成课程评价表,并且薪水受这些结果的影响。30年前,我来密歇根大学时,我的薪水比人类学系任何人的都高,他们今天都还很活跃。我的评分标准没有追随膨胀潮流。学生对评分的抱怨增加了,而现在我的薪水就处在教授工资单的底层。 ${ }^{[24]}$

你认为这段话犯有虚假原因谬误吗?\\
有时会发生这种现象:人们假定一事件是另一事件的原因,只因为另一事件在时间上紧随着前者。我们当然知道,纯粒的时间连续并不能确证一种因果联系,但是很容易被欺骗。在对外政策中,如果一项挑战性动议之后跟随出现了一件与其并不相关的国际事件,那么有人就可能错误地得出结论:挑战性动议就是那个事件的原因。在原始科学中,这样的错误是常见的;现在,我们把这种事件作为荒谬的说法来拒斥:敲锣打鼓是日食之后太阳重又出现的原因,因为不可否认在日食时每次敲锣打鼓之后太阳的确又会复现。

这类推理错误仍然广泛存在:反常的天气状况被归咎于某些发生在前的不相关天象;实际上由病毒引起的感染,却被认为是伤风或湿脚所使然,等等。这种虚假原因被称为"缘出前物"(post hoc ergo propter $h o c)$ ;近来,一名通讯员在给《纽约时报》的一封信中就出现了一个这样的例子。他写道:

\section*{在工业世界中,美国的死刑带给我们的是,每 100000 人中最高的犯罪率和数量最多的囚犯。 ${ }^{[25]}$}
当"缘出前物"非常明显时,它是一种容易发现的谬误;但是,甚至最伟大的科学家和政治家偶尔也会被它误导。 