\subsection{A1.歧义}

\begin{theorembox}[title=歧义谬误的定义与机制]
\logicemph{语言的多义性}:大多数词汇都有多于一个的字面意义,但在多数情况下,通过注意语境和利用我们良好的感觉,我们在阅读和听讲时不难将这些意义分辨开来。

\logicwarn{歧义的产生}:但是,当人们有意无意地混淆一个词或短语的几个意义时,就是在\logicterm{歧义}地使用这个词或短语。

\logicterm{歧义谬误}:如果在论证中这样做,就犯了\logicterm{歧义谬误}。

\logicwarn{谬误机制}:
\begin{itemize}
  \item 在论证的不同部分使用同一词汇的不同含义
  \item 利用词汇的多义性来混淆推理
  \item 前提中的词汇含义与结论中的不同
  \item 破坏了论证的逻辑一致性
\end{itemize}

\logicemph{识别关键}:识别歧义谬误的关键在于检查关键词汇在论证中是否保持了一致的含义。
\end{theorembox>

\paragraph{文学中的歧义}
\begin{examplebox}[title=文学中的歧义]
\logicwarn{文学中的巧妙运用}:有时,这种歧义谬误非常明显,在某些玩笑的字里行间使用。

\logicemph{卡罗尔的例子}:刘易斯·卡罗尔(Lewis Carroll)在《爱丽丝镜中奇遇记》(\textit{Through the Looking Glass})中对爱丽丝的奇遇的讲述,就包含着机智和逗乐的歧义。其中一个如下:

\logicwarn{对话内容}:
\begin{quote}
"你们谁走过这条路?"国王继续走着,并向送信人伸出手要些千草。

"没有人(nobody)。"送信人说。

"很对,"国王说,"这位年轻的女子也看到过他(him)。所以,当然 Nobody 比你们走得更慢。"
\end{quote>

\logicemph{歧义分析}:在这段话中,歧义谬误其实用得是相当巧妙的:

\logicwarn{词汇转换过程}:
\begin{itemize}
  \item \logicemph{第一次使用}:"nobody"这个词仅仅是指"没有人"(no person)的意思
  \item \logicemph{代词指称}:接着用代词"他(him)"来指称,就好像"nobody"这个词命名了一个人
  \item \logicemph{名词化}:当相同的词被大写并明显地用做一个名字"Nobody"时,它就显然命名了一个人
  \item \logicemph{特性转移}:这个人具有没有走过这条路的特性,而这个特性又是从该词的第一次运用中得来的
\end{itemize>

\logicemph{文学价值}:有时,歧义是机智的工具,刘易斯·卡罗尔就是一位非常机智的逻辑学家。$^{[28]}$
\end{examplebox}

\paragraph{相对词的歧义问题}
\begin{logicbox}[title=歧义论证的严肃性]
歧义论证总是谬误的,但它们却不总是愚蠢和滑稽的,这一点将在下面节录的例子中看出来。
\end{logicbox}

\begin{examplebox}[title=相对词的歧义问题]
\logicwarn{特殊类型}:有一种歧义谬误特别值得一提。这是一种由错误使用\logicterm{相对性}(relative)词项而来的错误。

\logicemph{相对词的特征}:在不同语境中,相对词具有不同意义。

\logicwarn{"高"的例子}:例如,"高"就是一个相对词:
\begin{itemize}
  \item 高个子人与高建筑物就处于非常不同的类别
  \item 一个高的人是一个比大部分人都高的人
  \item 而一座高的建筑物是一座比大部分建筑物都高的建筑物
\end{itemize}

\logicemph{论证形式的对比}:
\begin{itemize}
  \item \logicterm{有效论证}:"象是动物,因此灰色的象是灰色的动物"这个论证是完全有效的。"灰色"这个词不是相对的。
  \item \logicwarn{无效论证}:"象是动物,因此小象是小动物"这个论证却是荒唐的。
\end{itemize}

\logicwarn{关键分析}:这里的关键之点是,"小"是个相对词:小象是非常大的动物。这个谬误就是一个关于相对词"小"的一种歧义谬误。

\logicemph{更隐蔽的例子}:然而,并非所有的有关相对词的歧义谬误都是这样显然。

\logicwarn{"好"的歧义}:"好"这个词是个相对词,关于它,经常出现歧义谬误:
\begin{itemize}
  \item 有人论证说某某是一位好将军,因此也会成为一位好总统
  \item 或者是一位好学者,从而也一定是一位好教师
\end{itemize}

\logicemph{问题分析}:这些论证忽略了"好"在不同领域中的不同标准和要求。
\end{examplebox>