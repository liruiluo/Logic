\section{相干谬误}

\begin{logicbox}[title=引言]
\textit{相干谬误是逻辑推理中最常见的一类错误,它表现为论证前提与结论之间缺乏必要的逻辑联系。识别这类谬误不仅有助于避免错误推理,也能帮助我们构建更有说服力的论证。}
\end{logicbox}

\subsection{相干谬误的本质}

当一个论证所依据的前提与其结论不相干因而不可能确立结论之真时,其所犯的就是\textbf{相干谬误}。或许,称之为不相干谬误更贴切,但是,(在实际论证中)这种论证的前提常常在心理上与结论是相干的,而正是这种相干性使得它们似乎正确和有说服力。\textbf{心理的相干}怎么会与\textbf{逻辑的相干}相混淆,可以用我们在第2章讨论的语言的不同用法进行部分阐释;这些混淆的机制在随后的分析中将变得更加清晰。

很多谬误传统上都有个拉丁名称;有些拉丁名称,像\textit{ad hominem}(人身攻击),已经进入普通英语语言之中。我们在这里将既使用拉丁名称又使用英语名称。 