\section{相干谬误}

\begin{logicbox}[title=引言]
\textit{相干谬误是逻辑推理中最常见的一类错误,它表现为论证前提与结论之间缺乏必要的逻辑联系。识别这类谬误不仅有助于避免错误推理,也能帮助我们构建更有说服力的论证。}
\end{logicbox}

\subsection{相干谬误的本质}

\begin{theorembox}[title=相干谬误的定义]
\logicemph{基本定义}:当一个论证所依据的前提与其结论不相干因而不可能确立结论之真时,其所犯的就是\logicterm{相干谬误}。

\logicwarn{命名问题}:或许,称之为不相干谬误更贴切,但是,(在实际论证中)这种论证的前提常常在心理上与结论是相干的,而正是这种相干性使得它们似乎正确和有说服力。
\end{theorembox}

\begin{theorembox}[title=心理相干与逻辑相干的混淆]
\logicemph{混淆机制}:\logicterm{心理的相干}怎么会与\logicterm{逻辑的相干}相混淆,可以用我们在第2章讨论的语言的不同用法进行部分阐释。

\logicemph{分析价值}:这些混淆的机制在随后的分析中将变得更加清晰。

\logicwarn{欺骗性根源}:正是心理相干与逻辑相干的混淆,使得相干谬误具有强烈的欺骗性和说服力。
\end{theorembox}

\begin{theorembox}[title=谬误的命名传统]
\logicemph{拉丁传统}:很多谬误传统上都有个拉丁名称,这些名称承载着丰富的逻辑学历史。

\logicemph{语言融合}:有些拉丁名称,像\textit{ad hominem}(人身攻击),已经进入普通英语语言之中。

\logicemph{双重命名}:我们在这里将既使用拉丁名称又使用英语名称,以便读者更好地理解和记忆。
\end{theorembox}

\begin{theorembox}[title=相干谬误的主要类型]
\logicemph{分类体系}:相干谬误包括以下主要类型:

\begin{enumerate}
  \item \logicterm{诉诸无知论证}(Argumentum ad Ignorantiam)
  \item \logicterm{诉诸不当权威}(Argumentum ad Verecundiam)
  \item \logicterm{人身攻击论证}(Argumentum ad Hominem)
  \item \logicterm{诉诸情感}(Argumentum ad Populum)
  \item \logicterm{诉诸同情}(Argumentum ad Misericordiam)
  \item \logicterm{诉诸暴力}(Argumentum ad Baculum)
  \item \logicterm{不相干结论}(Ignoratio Elenchi)
\end{enumerate}

\logicemph{学习重点}:每种谬误都有其独特的表现形式和识别方法,需要通过具体例子来深入理解。
\end{theorembox}