\section*{R7.不相干结论(Irrelevant Conclusion:Ignoratio Elenchi)}
当一个论证声称要确证一个特定的结论,但却去证明另一个与之不同的结论时,就犯有不相干结论谬误(Ignoratio elenchi 的字面意义是"错误证明")。它的前提"不得要领";它的推理本身可能并非不合理,但它在争论树结论的辩护却没有效力。社会法律领域中的论证经常犯有这种谬误。一个特殊方案的确是为某种被广泛支持的更大目标服务的,但为该方案进行论证的前提所提供的理由却只能支持那个大目标,而没有告诉我们关于那个特定方案的任何东西。有时这种方式是故意为之的,有时则是由于过于热情关心那种更大目标,而认识不到现有前提与特定方案的结论并不相干。 