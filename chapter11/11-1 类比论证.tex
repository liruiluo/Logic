\section{类比论证}

\begin{logicbox}[title=引言]
本节介绍\logicterm{类比论证}的概念及其在日常推理中的重要性。我们将分析\logicterm{类比论证}的基本结构,区分它与\logicterm{演绎论证}的根本差异,探讨类比在论证和非论证语境中的多种用途,并学习如何识别和表达\logicterm{类比论证}的基本形式,从而为理解这种\logicterm{归纳推理}方式奠定基础。
\end{logicbox}

\subsection{从演绎确定性到归纳或然性}

前几章讨论的是\logicterm{演绎论证}。\logicterm{演绎论证}是否\logicemph{有效},取决于其前提是否能够证明地(demonstratively)得到结论。

\begin{theorembox}[title=演绎确定性的局限性]
然而,还有许多良好的和重要的论证,这些论证的结论不能得到确定性的证明。这种局限性体现在以下几个方面:

\textbf{1. 经验知识的或然性}:我们充分相信许多\logicterm{因果连接}(causal connections),只是基于\logicterm{盖然性}(probability)——尽管\logicterm{盖然性}程度可能非常高。

\textbf{2. 科学认识的特征}:我们能够不加迟疑地说,吸烟是癌症的一个原因,但我们不能够赋予我们的这种知识与从前提中推得一个演绎\logicemph{有效的}论证结论这种知识以相同的确定性。

\textbf{3. 专业认知的限制}:一个著名的医科专家根据演绎标准声称:"没有人将能够证明(prove)吸烟导致癌症,或者说任何事情导致任何事情。从理论上讲,你不能够证明任何事情。"\cite{surgeon1964}

\textbf{4. 认识论的现实}:的确,当我们评价我们关于世界的事实的知识时,演绎确定性的标准太高了。
\end{theorembox}

\begin{examplebox}[title=归纳推理的历史必然性]
这种从演绎确定性向归纳或然性的转变,在认识论史上具有重要意义:

\textbf{古代哲学}:亚里士多德已经认识到,除了演绎推理之外,还存在从特殊到一般的归纳推理。

\textbf{近代科学革命}:培根强调经验归纳法在科学发现中的重要性,与笛卡尔的演绎理性主义形成对比。

\textbf{现代科学方法}:波普尔的证伪主义和库恩的范式理论都承认,科学知识本质上是可错的和或然的。

\textbf{当代认知科学}:人工智能和机器学习的发展进一步证实了归纳推理在人类认知中的核心地位。
\end{examplebox}

\begin{theorembox}[title=归纳论证与演绎论证的区别]
本章及以后的各章将转向分析这样的论证:人们在这些论证中并不声称结论的真理性是从前提必然地得到,而仅仅表明,前提对结论的支持是\logicterm{或然的}(probable),或者说结论\logicterm{盖然为真}。这种论证被称为\logicterm{归纳论证},其与\logicterm{演绎论证}具有根本性差异。我们已经在第1章中讨论了演绎和归纳之间的基本区别。本书第二部分已经对演绎进行了讨论,第三部分则用来讨论归纳。
\end{theorembox}

\subsection{类比论证的本质与特征}

在\logicterm{归纳论证}中有一种被普遍使用的论证类型:\logicterm{类比}(analogy)论证。

\begin{theorembox}[title=类比论证的认知基础]
类比论证在人类认知中占据特殊地位,其重要性体现在以下几个方面:

\textbf{1. 认知经济性}:类比允许我们利用已有知识来理解新情况,避免从零开始的认知负担。

\textbf{2. 模式识别}:人类大脑天生具有识别相似模式的能力,类比论证正是这种能力的逻辑表达。

\textbf{3. 创新思维}:许多科学发现和技术创新都源于类比思维,如开普勒的行星运动类比、达尔文的自然选择类比。

\textbf{4. 社会交往}:类比论证是说服和解释的重要工具,在法律、政治、教育等领域广泛应用。

\textbf{5. 文化传承}:通过类比,复杂的文化概念和价值观念得以在代际间传递。
\end{theorembox}

下面是两个\logicterm{类比论证}的例子:

\begin{examplebox}[title=类比论证实例一:教师资格测验]
一些人认为教师资格测验是不公正的双重测试。"教师已经是大学毕业生,"他们说,"他们为什么还要被测试?"其实这很简单。律师是大学毕业生,而且还是职业学院的毕业生,但他们不得不参加律师资格考试。还有其他大量的行业,如会计、精算师、医生、建筑师等,这些行业对想成为其成员的人都要求参加并通过资格考试,以证明他们的专业素质。没有理由说明教师不应当被要求做同样的事情。\cite{davis1986}
\end{examplebox}

\begin{examplebox}[title=类比论证实例二:行星生命假说]
在我们居住的地球和其他行星(土星、木星、火星、金星和水星)之间,我们可以观察到许多类似之处。它们均如地球一样围绕太阳运行,尽管它们绕太阳的半径不同、周期也不同。它们均从太阳那里获得光,地球也是如此。我们已经知道,其中一些行星,如地球一样,围绕它们的轴自转,因而它们必定有类似白天和黑夜的更替。一些行星有卫星,当太阳不再照射时,这些卫星给行星以光亮,就如我们的月亮给我们以光一样。这些行星的运动均与地球一样受制于万有引力定律。根据所有这些类似,认为这些行星可能与我们地球一样,有不同等级的生命存在,这不是不合理的。通过\logicterm{类比}得到的这个结论具有一定程度的可能性。\cite{reid1785}
\end{examplebox}

\subsection{类比推理在日常生活中的普遍性}

我们的许多日常推论是通过\logicterm{类比}进行的。

\begin{examplebox}[title=日常类比推理的典型实例]
\textbf{消费决策}:我推论我将从一台新的计算机那里得到好的服务,根据是,我从同样的生产厂家购买的一台计算机曾给了我很好的服务。

\textbf{文化选择}:我看到某个作者的新作,根据我读过该作者的其他著作并且喜欢这些著作而推断,我将喜欢读这本新作。

\textbf{经验应用}:我们过去的经验在未来同样成立的大多数日常推论,其基础就是\logicterm{类比}。

\textbf{本能反应}:当然,我们无法给出一个清楚的公式化的论证,我们只能说,曾被烧伤的孩童躲避火的行为即涉及\logicterm{类比推论}。
\end{examplebox}

\begin{theorembox}[title=类比推理的心理学机制]
类比推理的普遍性源于其深层的心理学机制:

\textbf{1. 记忆结构}:人类记忆以相似性为基础组织,相似的经验会被归类存储,便于类比提取。

\textbf{2. 模式匹配}:大脑会自动将新情况与已存储的模式进行匹配,寻找最相似的先例。

\textbf{3. 预测机制}:基于相似性的预测是生存的基本需要,类比推理是这种预测能力的体现。

\textbf{4. 学习转移}:类比允许我们将在一个领域学到的知识转移到另一个领域,提高学习效率。

\textbf{5. 概念形成}:许多抽象概念都是通过类比具体事物而形成的,如"时间流逝"类比"河水流动"。
\end{theorembox}

\logicwarn{这些论证中没有一个是确定的或者说是证明性地\logicemph{有效的}。这些论证中的结论,没有一个能够从前提中获得逻辑必然性。}这是逻辑可能的:用来判断律师和医生资格的方法,并不适合于判断教师的资格;这也是逻辑可能的:地球可能是唯一可以居住的行星,新的计算机可能运转不灵,我喜欢的作者的新书可能无趣而难以卒读;甚至这也是逻辑可能的:一团火能够烧伤人,另外一团火则不会。没有一个\logicterm{类比论证}可以指望具有数学的那种确定性。\logicterm{类比论证}不是按\logicemph{有效}和\logicwarn{无效}来区分的,我们只能用\logicterm{概率}来刻画它们。

\subsection{类比的多元功能:论证之外的应用}

除了在论证中频繁使用类比外,人们为了描述生动,经常将类比用于非论证的活动中。

\begin{theorembox}[title=类比的非论证功能分类]
类比在非论证语境中具有多种重要功能:

\textbf{1. 文学修辞功能}:
明喻和暗喻为类比在文学中的用法,它们为作家给读者的心中创造鲜活的画面提供了莫大的帮助。例如:砧骨上做马蹄铁而产生的副产品火花一样。火花比马蹄铁更为灿烂,但它们在本质上是无意义的。\cite{chesterton1910}

\textbf{2. 教学说明功能}:
类比也用于说明,将读者不熟悉的某种东西,与读者比较熟悉的另一种东西进行对照,比较它们的类似之处,而使读者得以理解。

\textbf{3. 科学解释功能}:
在科学传播中,类比是连接专业知识与公众理解的重要桥梁。

\textbf{4. 概念建构功能}:
许多抽象概念的形成都依赖于与具体事物的类比。

\textbf{5. 情感共鸣功能}:
类比能够激发情感反应,增强表达的感染力。
\end{theorembox}

\begin{examplebox}[title=科学类比的经典案例]
麻省理工学院基因组研究中心主任埃瑞克-兰德试图说明人类基因组计划的巨大影响。为了加强那些对基因研究不熟悉的人的理解,类比是他所用的一个工具:

"基因组计划完全类似于化学中创立周期表。正如门捷列夫在周期表中安排化学元素,使得以前不相关的大量数据变得连贯,同样,当前有机体中上万的基因,将能够从较少数量的简单基因模块或单元即所谓原始基因的组合中得到。"\cite{lander1995}

\textbf{这个类比的成功之处}:
\begin{itemize}
\item 利用了听众对周期表的熟悉程度
\item 突出了组织性和系统性的共同特征
\item 暗示了基因组研究的革命性意义
\item 将复杂的生物学概念转化为可理解的化学类比
\end{itemize}
\end{examplebox}

\subsection{类比论证的结构分析}

类比在描述和说明中的使用不同于在论证中的使用,尽管在某些实例下,不容易区分是哪一种用法。但是,无论是论证地使用类比还是其他的使用法,类比都是不难定义的。在两个或更多的实体之间进行一个类比,就是表明它们在一个或多个方面(respect)是类似的(similar)。

\begin{theorembox}[title=类比与类比论证的区别]
这说明了什么是类比,但是仍然没有刻画什么是类比论证。两者的关键区别在于:

\textbf{类比}:仅仅指出相似性,不涉及推理过程。
\textbf{类比论证}:基于相似性进行推理,从已知相似性推出未知相似性。

类比论证的核心是\logicterm{推理转移}:从一个或多个已知的相似方面,推断出另一个未知方面的相似性。
\end{theorembox}

让我们考察一个类比论证事例并分析它的结构。我们选择上面引用的例子中最简单的例子:我新买的计算机将给我好的服务,因为我的一台旧计算机是从同样厂家购买的,它给了我好的服务。

\begin{examplebox}[title=类比论证的结构解析]
具有类似方面的两个事物是两台计算机。这里存在三点类比,两个事物被认为在三个方面相似:
\begin{enumerate}
\item 均为计算机
\item 均在同样厂家购买
\item 给我好的服务
\end{enumerate}

\textbf{关键观察}:类比的这三点在论证中并不起相同的作用。前两点出现在前提中,而第三点既出现在前提中又出现在结论中。

\textbf{论证结构}:该论证具有这样的前提:首先断定两个事物在两点类似,其次断定了其中的一个事物还具有另外一个特点,从而推论得出另一个事物也具有这个特点的结论。

\textbf{逻辑形式}:
\begin{itemize}
\item \textbf{前提1}:A和B在性质P和Q方面相似
\item \textbf{前提2}:A具有性质R
\item \textbf{结论}:因此,B可能也具有性质R
\end{itemize}
\end{examplebox}

\begin{theorembox}[title=法庭类比论证的特殊地位]
类比论证是法庭最基本的工具之一。法官不是事先摆出严格的法规或原理,他们往往这样推理,因为两个案件——早先的已经被判决的案件和手头上待判决的案件——有相同的特点,它们应当具有相同的判决结果。

\textbf{判例法的逻辑基础}:
\begin{itemize}
\item \textbf{先例约束原则}:相似案件应得到相似处理
\item \textbf{法律连续性}:保持法律适用的一致性
\item \textbf{可预测性}:当事人能够预期法律后果
\item \textbf{公平性}:避免任意性和歧视性判决
\end{itemize}

\textbf{实例分析}:例如,一旦做出了不能禁止3K党发表言论的判决,那么法庭可能通过类比论证而得出不能禁止纳粹党游行的结论。\cite{collin1978} 通过判例的论证一旦做出,人们将确定和强调以前的案子和手头案子之间类似的那些特点。

\textbf{法庭类比的复杂性}:
\begin{itemize}
\item 需要识别法律上相关的相似性
\item 必须区分重要特征与偶然特征
\item 涉及价值判断和政策考量
\item 可能面临多个竞争性先例
\end{itemize}
\end{theorembox}

\subsection{类比论证的一般形式}

当然,不是每个类比论证都必须精确地涉及两个事物或者精确涉及三个不同的特点。托马斯-雷德(在上面我们已经提到)认为其他行星可能有人居住,他的论证是对六个事物(当时知道的行星)的八个方面进行类比的。

\begin{theorembox}[title=类比论证的一般结构]
然而,除了这些数量存在差别外,所有的类比论证均具有相同的一般结构或模式。每个类比推理都是这样进行的:从在一个或多个方面两个或更多的事物之间的类似性,到这些事物在某个其他方面具有类似性。

\textbf{形式化表示}:设$a 、 b 、 c 、 d$ 是实体,$P 、 Q 、 R$ 是属性或"相似方面",一个类比论证可以表示成下列形式:

$$
\begin{aligned}
& a 、 b 、 c 、 d \text { 均具有属性 } P \text { 和 } Q, \\
& a 、 b 、 c \text { 均具有属性 } R,
\end{aligned}
$$

$$
\text { 因而 } d \text { 可能具有属性 } R \text { 。 }
$$

\textbf{结构要素分析}:
\begin{itemize}
\item \textbf{类比对象}:参与比较的实体(a, b, c, d)
\item \textbf{共同属性}:所有对象都具有的属性(P, Q)
\item \textbf{已知属性}:部分对象已知具有的属性(R)
\item \textbf{推断属性}:待推断对象可能具有的属性(R)
\item \textbf{或然性}:结论的不确定性("可能")
\end{itemize}

在识别并且特别是评价类比论证时,将之表示成这种形式是有帮助的。
\end{theorembox}

\begin{center}
\fbox{\parbox{0.95\textwidth}{
\textbf{本节要点}
\begin{itemize}
\item \textbf{从演绎确定性到归纳或然性}:
  \begin{itemize}
  \item 演绎确定性标准在评价经验知识时过于严格
  \item 科学认识本质上具有或然性特征,如因果关联的确立
  \item 归纳推理的历史必然性:从古代哲学到现代认知科学的发展脉络
  \item 归纳论证与演绎论证的根本差异:或然性vs必然性
  \end{itemize}
\item \textbf{类比论证的认知基础}:
  \begin{itemize}
  \item \textbf{五大重要性}:认知经济性、模式识别、创新思维、社会交往、文化传承
  \item 类比推理是人类大脑天生的认知能力
  \item 许多科学发现和技术创新都源于类比思维
  \item 在法律、政治、教育等领域广泛应用
  \end{itemize}
\item \textbf{类比推理的心理学机制}:
  \begin{itemize}
  \item \textbf{五大机制}:记忆结构、模式匹配、预测机制、学习转移、概念形成
  \item 人类记忆以相似性为基础组织,便于类比提取
  \item 基于相似性的预测是生存的基本需要
  \item 许多抽象概念都是通过类比具体事物而形成
  \end{itemize}
\item \textbf{类比的多元功能}:
  \begin{itemize}
  \item \textbf{论证功能}:基于相似性进行推理,从已知推出未知
  \item \textbf{非论证功能}:文学修辞、教学说明、科学解释、概念建构、情感共鸣
  \item 类比与类比论证的区别:指出相似性vs基于相似性推理
  \item 科学传播中类比是连接专业知识与公众理解的重要桥梁
  \end{itemize}
\item \textbf{类比论证的结构分析}:
  \begin{itemize}
  \item \textbf{核心机制}:推理转移——从已知相似性推出未知相似性
  \item \textbf{逻辑形式}:前提1(A和B在P、Q方面相似)+前提2(A具有R)→结论(B可能具有R)
  \item 不同相似方面在论证中起不同作用:基础相似性vs推断相似性
  \item 结论的或然性是类比论证的本质特征
  \end{itemize}
\item \textbf{法庭类比论证的特殊地位}:
  \begin{itemize}
  \item \textbf{判例法的逻辑基础}:先例约束原则、法律连续性、可预测性、公平性
  \item 需要识别法律上相关的相似性,区分重要特征与偶然特征
  \item 涉及价值判断和政策考量,可能面临多个竞争性先例
  \item 体现了法律推理的类比本质
  \end{itemize}
\item \textbf{类比论证的一般形式}:
  \begin{itemize}
  \item 所有类比论证都具有相同的一般结构或模式
  \item \textbf{五大结构要素}:类比对象、共同属性、已知属性、推断属性、或然性
  \item 形式化表示有助于识别和评价类比论证
  \item 数量可变但结构不变:可涉及多个对象和多个属性
  \end{itemize}
\end{itemize}
}}
\end{center}