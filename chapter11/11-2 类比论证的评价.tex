\section{类比论证的评价}

\begin{logicbox}[title=引言]
本节讨论如何评价\logicterm{类比论证}的强度。我们将分析影响\logicterm{类比论证}可靠性的六个关键标准:实体数量、前提中实例的多样性、相似方面的数量、相关性、差异性以及结论断言的强度。通过理解这些标准及其相互关系,我们将能够更准确地判断\logicterm{类比论证}的说服力,认识到强类比与弱类比之间的区别。
\end{logicbox}

\subsection{类比论证评价的理论基础}

\logicwarn{没有一个\logicterm{类比论证}是演绎\logicemph{有效的}。这里演绎\logicemph{有效的}含义是指,从前提得出结论具有逻辑必然性。}

\begin{theorembox}[title=类比论证的评价原理]
但是,有些\logicterm{类比论证}比其他\logicterm{类比论证}的论证力要强。\logicterm{类比论证}被评价为比较好的还是比较差的,依赖于结论得以断定的\logicterm{或然性}程度。

\textbf{评价的基本原理}:
\begin{itemize}
\item \textbf{强度梯度}:类比论证存在强弱程度的连续谱,而非简单的有效/无效二分
\item \textbf{或然性标准}:评价标准是结论的或然性程度,而非逻辑必然性
\item \textbf{多因素综合}:论证强度由多个因素共同决定,需要综合考量
\item \textbf{语境依赖}:评价结果可能因具体语境和目的而有所不同
\end{itemize}

\textbf{评价的认识论意义}:
类比论证的评价体现了人类理性认识的重要特征——在不确定性中寻求最佳判断,这是科学方法和日常决策的核心能力。
\end{theorembox}

\begin{examplebox}[title=日常类比推理实例]
两个日常例子可以帮助我们弄清楚,什么因素影响\logicterm{类比论证}的有力性。你决定去购买特定的一双鞋,因为以前与之类似的其他鞋子使你感觉很舒服;你挑选了一只某一品种的狗,因为该品种的其他狗所表现出来的特征是你所喜欢的。在这两个例子中使用了\logicterm{类比推理}。为了评价这两个例子的论证强度——实际上也就是所有\logicterm{类比论证}的强度,我们可以确定 6 个显著的标准:
\end{examplebox}

\subsection{类比论证评价的六个核心标准}

\begin{theorembox}[title=标准一:实体数量的深入分析]

\textbf{1. 实体数量(Number of Entities)}

\textbf{基本原理}:如果我过去对特定种类的鞋子的经历仅限于我穿过的并喜欢的一双,对一双明显类似的鞋,我穿后发现具有意想不到的缺陷,这将使我很失望。但是如果我多次购买了那类鞋子,我可以有理由地认为,下一次购买的鞋子会与我以前穿的一样好。

\textbf{统计学基础}:在同样对象上的多次的同种经验将支撑结论(购买的鞋子将是合脚的),比单个经验支撑结论有力得多。每一个经验可看成是一个附加实体,在评价\logicterm{类比论证}中实体数量是第一个标准。

\textbf{数量与强度的关系}:
\begin{itemize}
\item \textbf{正相关性}:一般地讲,实体数量——过去所经历的场合数量——越大,论证越强
\item \textbf{非线性关系}:实体数量和结论成\logicemph{真}的\logicterm{概率}之间没有简单的比例关系
\item \textbf{边际递减效应}:增加实体数量的效果会逐渐递减
\end{itemize}

\textbf{实例分析}:与机敏、温顺的金色猎犬愉快相处的6次经历,使人们相信下一个金色猎犬同样是机敏和温顺的。但是,前提中具有6个经历的\logicterm{类比论证}其结论在可靠性上并不是前提中有2个经历的一个类似论证的3倍。

\textbf{理论意义}:增加实体数量是重要的,但其他因素也要增加。这体现了类比论证评价的多维性和复杂性。
\end{theorembox}

\begin{theorembox}[title=标准二:前提中实例多样性的深入分析]

\textbf{2. 前提中实例的多样性(Diversity of Instances)}

\textbf{基本原理}:如果我先前购买的那些合脚的鞋子,既有购买于大商店的,又有购买于专卖店的,既有在纽约制造的又有在加利福尼亚制造的,既有通过邮寄销售的,又有通过商店直接销售的,那么,我可以有信心地认为,鞋子合脚的原因在于鞋子本身,而不是售货员的服务。

\textbf{因果分析功能}:如果我先前的金色猎犬,既有公的也有母的,既有从小就领养的小犬,也有从仁慈的社会中得来的成年犬,我可以相信,正是犬的品种,而不是它们的性别、年龄或其来源,是它们先前与我愉快相处的原因。

\textbf{逻辑机制}:
\begin{itemize}
\item \textbf{排除法原理}:多样性有助于排除偶然因素和无关变量
\item \textbf{因果隔离}:通过变化无关因素来隔离真正的因果关系
\item \textbf{稳健性检验}:在不同条件下的一致性表现增强了结论的可靠性
\item \textbf{泛化能力}:多样的实例增强了结论的普遍适用性
\end{itemize}

\textbf{直觉理解}:我们可直觉地理解这个标准:\logicterm{类比论证}的前提中所涉及的实例越不相似,论证越强。

\textbf{科学方法论意义}:这一标准体现了科学实验中控制变量的重要性——通过改变无关变量而保持关键变量不变,来确定真正的因果关系。
\end{theorembox}

\begin{theorembox}[title=标准三:相似方面数量的深入分析]

\textbf{3. 相似方面的数量(Number of Similarities)}

\textbf{基本原理}:在作为前提的实例中可能出现了大量的相似性:也许鞋子属于同一类型,具有同样的价格,由同样种类的皮革制成;也许猎犬是同样品种,在同样的年龄由同一个饲养人饲养,等等。前提中的这些实例在所有这些方面类似,并与结论中的实例在这些方面类似,增强了结论中的实例具有的新属性的可能性——新鞋子将合脚,一只新的狗会具有温顺的品行。

\textbf{累积效应}:
\begin{itemize}
\item \textbf{相似性叠加}:多个相似方面的累积效应增强论证强度
\item \textbf{模式强化}:相似方面越多,相似模式越清晰
\item \textbf{偶然性降低}:多重相似性降低了偶然巧合的可能性
\end{itemize}

\textbf{直观理解}:我们同样可以直观地理解这个标准:结论中的实体与前提中的实体之间类似的方面越多,结论越可靠。

\textbf{重要限制}:但是,在结论和所识别出的类似方面的数量之间,也不存在简单的数字比例关系。这提醒我们质量比数量更重要。
\end{theorembox}

\begin{theorembox}[title=标准四:相关性的核心地位]

\textbf{4. 相关性(Relevance)}

\textbf{基本原理}:与共有的相似方面的数量同样重要的是,前提中的实例与结论中的实例在共有的相似方面的种类的类似。如果新鞋子与以前的鞋子一样,是在星期二购买的,这是一个与合脚没有关系的类似;但是,如果新的鞋子与先前购买的鞋子一样,由同样的厂商生产,这自然相当重要。

\textbf{相关性的判断标准}:
\begin{itemize}
\item \textbf{因果关联}:相似方面与推断属性之间是否存在因果关系
\item \textbf{逻辑关联}:相似方面与推断属性之间是否存在逻辑联系
\item \textbf{经验关联}:基于经验观察的关联性
\item \textbf{理论关联}:基于科学理论的关联性
\end{itemize}

\textbf{质量优于数量}:当相似方面是相关的时候(如鞋子的样式、价格以及材料),相似方面便增加论证的力度,并且,单个的具有高相关因素比一批不相关的类似对论证的贡献更大。

\textbf{相关性的核心地位}:相关性是所有其他标准的基础,没有相关性,其他标准都失去意义。
\end{theorembox}

至于哪些属性确实与论证结论的可靠性相关,人们有时意见不一致。但相关性本身的意义则不存在争论。当一个属性连接另外一个的时候,即当它们之间存在某种因果联系的时候,它们之间存在相关,那就是为什么确定因果联系在类比论证中是关键的原因,以及为什么在法庭上在确定证据是否有效(即相关还是不相关)的过程中,建立这样的连接往往是至关重要的原因所在。

类比论证之进行,无论是从原因到结果,还是从结果到原因,都是可能的。甚至,前提中的属性既不是结论巾的属性的原因,也不是其结果,而两者是同一原因的结果的话,此时,类比论证也是可能的。医生注意到她的病人出现了某种症状,她能够精确地预测另外的症状。这不是因为其中一个症状是另外一个的原因,而是因为身体的某种紊乱造成了它们的相继出现。一个产品的颜色往往与功能无关。但是,当那种颜色与众不同并且共同出现在前提和结论之中的时候,它可以作为论证的相关相似方面来使用。颜色本身可能与产品的功能无关,但是,如果我们知道该颜色是制

造商生产过程的一个相似方面,它可以用来进行一个论证。\\
因果连接是评价类比论证的关键,我们只能够通过观察和实验经验地发现它们。关于经验研究的一般理论是归纳逻辑关心的中心问题,在下面的几章中我们将以一定的篇幅来对之进行讨论。

5.差异性。一个差异(disanalogy)就是一个不同点,它是这样一个方面,在这个方面我们进行推理得出结论的实例有别于论证所基于的实例。回到鞋子的例子上来,如果我们想购买的这双鞋子看上去好像我们以前所穿的鞋子,但事实上这双鞋子更便宜,并且由不同的厂家生产,那么,这些差异使我们有理由对它能否使我们穿起来舒服产生怀疑。

上面论述的相关性在这里同样是重要的。当被确定了的差别具有相关性,与我们正在寻找的东西有因果连接的时候,差异使类比论证失效。投资者购买共有基金的股票,他们往往根据股票成功的"走势记录"。他们这样推理:先前的购买使资本得益,下一回的购买将同样使资本增益。当我们获悉在基金赢利期间操纵该基金的人刚刚被替换,我们面临着一个实质上的差异,它降低了类比论证的强度。

差异使类比论证减弱。因而,它们往往被用来攻击一个类比论证。正如批评者所认为的,结论中的情形在关键方面不同于早先发生的情形,因而在先前情形中正确的东西不可能在后面的情形中也正确。在司法中这是普遍使用类比的地方,某个(或某些)早先的案子通常作为对手头案件的判例被提供给法庭。论证是类比的。对方辩护律师努力将本案与以前的案子区别开来;即辩护律师努力表明,由于在本案中的事实与以前案子中的事实之间存在某个关键差别,以前的案件不是本案的恰当判例。如果差异较大,并且差异的确是关键性的,它能够成功地推翻所提出的类比论证。

因为差异性是反对类比论证的主要武器,因而,能够使潜在的差异得以消解的做法将加强该论证。这说明为什么在前提中实例的多样性会增强一个论证的力度,正如我们已经在前面第二个标准中所述。前提中的实例之间变化越大,批评者越不可能在前提中的实例与结论之间找到使论证减弱的差异。举例来说:吉姆-库玛尔进入一所大学,成为大一学生;来自于吉姆所在高中的另外十个学生已经在该大学里成功完成了学业。我们可以类比地说,她成功地完成学业是很可能的。在与大学的学习有关的某个方面,如果所有这些学生之间都类似,但他们在该方面与吉姆不同,该差异削弱吉姆将成功的论证。但是,如果我们了解到十个成功的师兄师姐在

许多方面——"在经济背景、家庭关系、宗教背景等方面——相互不同,他们间的这些不同使潜在的差异得以消解。如果以来自于同一高中的其他学生作为论证的前提,这些学生并不紧密地相似,而呈现出变化,那么,正如我们前面看到的,吉姆将成功的论证得到加强。

必须避免的一个混淆是:差异是类比论证得以弱化的原理,与前提中的差别(dissimilarity)使同样的论证加强的原理不同。对于前者,差异发生在前提中的实例与结论中的实例之间;对于后者,差别仅仅发生在前提的实例之间。一个差异(不相似)指的是,我们已经经历的实例和要得出的结论的实例之间的区别。由于结论中的实例与早先的实例不一样,结论得不到保证。(我们可以通过提出差异来进行反驳。)该类比被认为是 "牵强的"(strained)或者"行不通的"。但是当我们指出前提间的差别的时候,我们则强化了论证:我们说,该类比有广泛的效力,它在这些实例和那些实例中都行得通,因而前提中实例多种多样的相似方面与结论涉及的相似方面不相关。

总之,差异削弱一个类比论证;而前提中的差别使类比论证加强。这两方面都与恰当性问题相关联:差异表明了前提中的实例和结论中的实例在某些相关方面存在不同;而前提中的差别所表明的是,我们原以为与某个属性存在因果相关的事情事实上毫无关联。

需要注意的是,被认为是第一重要的标准,即被认为具有相似性的实体的数量,也与相关性有关。实例数越多,它们之间的差别也就可能越多。因而增加实体数量是人们所希望的——但是随着实体数量的增加,增加的实例其作用在降低。因为它所提供的差别更可能由先前的实例所提供———这样的话,增加的实例对于保护结论免遭差异性产生的破坏,起不到或几乎起不到作用。

\begin{theorembox}[title=标准六:结论断言强度的深入分析]

\textbf{6. 结论所做的断言(Strength of Conclusion)}

\textbf{基本原理}:每个论证均断言其前提给出了接受结论的理由。容易看到,论证断言得越多,保持该断言的负担也就越大。这对每个类比论证均是正确的。结论相对于前提而言是否适度在推理的评价中起关键作用。

\textbf{适度性原则}:如果我的朋友的新车每加仑汽油能行驶30英里,我会得出如果我购买同样厂家和同样型号的车,我至少能够使该车每加仑汽油行驶20英里。该结论是适度的,因而可靠性十分大。如果我的结论十分大胆,如,我将至少使每加仑汽油行驶29英里,该结论受我拥有的证据的支持程度就很低。

\textbf{一般原理}:一般而言,断言越适度,加于前提的负担越轻,论证越强;断言越大胆,前提的负担越大,论证也就越弱。

\textbf{论证强化策略}:
\begin{itemize}
\item \textbf{降低断言强度}:通过减少在确定的前提下的断言的内容
\item \textbf{增强前提支持}:使断言维持不变但用附加的或更强大的前提给予它以支持
\end{itemize}

\textbf{论证弱化因素}:
\begin{itemize}
\item \textbf{提高断言强度}:如果一个类比论证的结论变得更大胆,而前提保持不变
\item \textbf{削弱前提支持}:断言维持不变,但我们发现支持它的证据存在较大的缺陷
\end{itemize}
\end{theorembox}

\subsection{六个标准的综合应用}

\begin{theorembox}[title=标准间的相互关系与综合评价]
这六个标准不是孤立运作的,而是相互关联、相互影响的:

\textbf{1. 相关性的核心地位}:相关性是所有其他标准的基础,决定了其他标准的重要性。

\textbf{2. 数量与质量的平衡}:实体数量、相似方面数量等数量标准必须与相关性、多样性等质量标准相平衡。

\textbf{3. 正负因素的权衡}:需要综合考虑增强论证的因素(相似性、多样性)和削弱论证的因素(差异性、过强断言)。

\textbf{4. 语境依赖性}:不同领域和不同目的下,各标准的重要性可能有所不同。

\textbf{5. 动态调整}:可以通过调整结论强度或增加前提支持来优化论证效果。
\end{theorembox}

\begin{center}
\fbox{\parbox{0.95\textwidth}{
\textbf{本节要点}
\begin{itemize}
\item \textbf{类比论证评价的理论基础}:
  \begin{itemize}
  \item 类比论证存在强弱程度的连续谱,而非简单的有效/无效二分
  \item 评价标准是结论的或然性程度,而非逻辑必然性
  \item 论证强度由多个因素共同决定,需要综合考量
  \item 体现了人类在不确定性中寻求最佳判断的理性认识特征
  \end{itemize}
\item \textbf{六个核心评价标准的深入分析}:
  \begin{itemize}
  \item \textbf{实体数量}:正相关性、非线性关系、边际递减效应、统计学基础
  \item \textbf{实例多样性}:排除法原理、因果隔离、稳健性检验、泛化能力
  \item \textbf{相似方面数量}:相似性叠加、模式强化、偶然性降低、质量比数量重要
  \item \textbf{相关性}:因果关联、逻辑关联、经验关联、理论关联四大判断标准
  \item \textbf{差异性}:相关差异vs无关差异、累积效应、阈值效应、平衡考量
  \item \textbf{结论强度}:适度性原则、强度梯度、认识论谦逊性、实用价值
  \end{itemize}
\item \textbf{相关性的核心地位}:
  \begin{itemize}
  \item 相关性是所有其他标准的基础,没有相关性其他标准都失去意义
  \item 单个高度相关的相似点比多个无关相似点更有说服力
  \item 相关性决定了差异是否真正影响论证强度
  \item 体现了科学实验中控制变量的重要性
  \end{itemize}
\item \textbf{差异性与差别的重要区分}:
  \begin{itemize}
  \item \textbf{差异性}:前提与结论之间的不同点,削弱论证
  \item \textbf{差别}:前提内部实例之间的不同点,增强论证
  \item 多样的前提实例可消解潜在差异的影响
  \item 这种区分对于正确评价类比论证至关重要
  \end{itemize}
\item \textbf{标准间的相互关系与综合评价}:
  \begin{itemize}
  \item 相关性的核心地位决定了其他标准的重要性
  \item 数量与质量的平衡:数量标准必须与质量标准相平衡
  \item 正负因素的权衡:增强因素vs削弱因素的综合考量
  \item 语境依赖性:不同领域和目的下各标准重要性不同
  \item 动态调整:可通过调整结论强度或增加前提支持来优化论证
  \end{itemize}
\item \textbf{论证强化与弱化的策略}:
  \begin{itemize}
  \item \textbf{强化策略}:降低断言强度、增强前提支持、增加实例多样性
  \item \textbf{弱化因素}:提高断言强度、削弱前提支持、增加相关差异
  \item 标准之间不存在简单的数量比例关系
  \item 根据这些标准可以系统地改进和加强类比论证
  \end{itemize}
\end{itemize}
}}
\end{center}