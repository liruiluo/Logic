\section*{11.3 通过逻辑类推进行的反驳}
"你应当说出你的意思。"[野兔尖刻地责备爱丽丝。]\\
"我说了,"爱丽丝慌忙回应;"至少——至少我说的意思——那是同样的事情,你是知道的。"\\
"一点也不同!"哈特尔说。"哎呀,你不如说'我看到我吃的东西' 与 '我吃我看到的东西' 是同一回事情!"\\
"你不如说"野兔附和道,"'我喜欢我得到的东西'与'我

野兔、哈特尔和冬眠鼠都使用逻辑类推(logical analogy)试图反驳爱丽丝的看法,即你所说的与你的意图是一回事。从逻辑的观点来看,论证的形式与论证的内容不同,形式是论证的最重要的方面。因而,我们往往通过表明另外一个被认为是错误的论证,与给定的论证有相同的逻辑形式,而证明该论证是不可靠的。

在演绎情况下,对一给定论证进行反驳性的类推是这样:其形式与给定论证一样,但反驳用的类推的前提为真而结论为假。由于用来反驳的类推是无效的,因而遭攻击的论证也是无效的一一因为它具有相同的形式。这里的原理与在 6.2 节中阐述的作为检验直言三段论的基础的原理是一样的,该原理同样是我们在 8.4 节中反复强调的逻辑形式的基础。

在归纳论证情况下,我们目前所考虑的逻辑类推的反驳技术同样可以是有效力的。许多在科学中、政治中或经济中的论证并不宣称是演绎的,它们会受到这样的反驳:它们与其他的论证具有十分类似的结构,而这些其他的论证的结论是错误的,或者被普遍地认为是不可能的。归纳论证本质上不同于演绎论证,差别在于前提给结论所提供的支持程度不同。但是所有的论证,无论是归纳的还是演绎的,具有同样基本的形式或模式。当我们要攻击一个归纳论证时,我们也可以提出另外一个具有同样形式的归纳论证,但是该论证明显有缺陷,因而结论十分可疑,这样的话,我们同样怀疑待考察论证的结论。

考虑下面的例子。反对安乐死合法化的一个通常论证被称为"滑坡"论证("slippery slope"argument)。该论证的要点是这样的,一旦正式授权医生进行某种道德上可疑的行为,那将使该类型的不道德行为加剧和增多。该论证认为,这个初衷是仁慈的,但我们应当避免这个仁慈,因为它必定将我们置于陡滑的斜坡之上,一旦迈出第一步,将无法停止。针对这个论证,一个当代的批评家是这样回应的:

滑坡论证尽管有影响,但它是难以站住脚的。它提出,一旦我们允许医生在病人的请求下结束病人的生命,医生可能并且也会肆意杀害那些并不想死的累赘病人。这个想法得不到辩护……内科医师如果开出的药物剂量大于规定剂量,这会将病人杀害。但在实际中没有人害怕医生会使用大剂量药物而置人于死地。没有人因恐惧"滑坡"而反对医生开处方。授权医生帮助那些求援的病人以结束他们的生命,会产生医生结束那些不想死的病人的生命的结果,这无异于说,授权进行手术以切除肿瘤会导致切除病人的心脏的结果。 ${ }^{[7]}$

这是一个在归纳论证中用逻辑类推来进行反驳的极好例子。先摆出要驳斥的论证:如果我们给了医生帮助病人结束他们的生命的权力,一些医生将肆意滥用这种权力。因而,这个论证得出,我们在这条路上一步都不应当迈出,我们应当拒绝给任何医生帮助病人结束他们生命的权力。

针对这个论证,一个形式一样的类比性的反驳得以提出,它建立在由归纳得来的对医生行为的公共认识之上:我们确实给了医生可能被滥用的特权。我们给了医生开出危险剂量药物的特权,而使用低剂量药物能够是有益的,当然我们知道他们能够开出伤害病人的大剂量药物。但是开出这样剂量药物的特权的滥用有这样的后果,这个事实丝毫不会使我们后悔我们授予医生这样的特权。因此,可以看到的是,从授予医生以安乐死的特权的可能滥用,到授予医生以处方特权的可能滥用(反驳提出的)的这个论证表明,滑坡论证就它对于医生而言不是十分有说服力的。

上面引用的文章中也提供了另外一个类推反驳,它在形式上十分相似:假如说,给予医生特权,协助那些企求帮助的病人结束他们的生命 (根据待反驳的论证)将会导致结束那些确实不想死的病人的特权的产生,那么,在这种情况之中,不单单医生而且立法机构都在陡坡上滑动。

这里再次给出的反驳类推是:现在普遍地给予医生切除某个身体器官的特权,当然是在病人的同意之下。得出这样的结论是荒唐的:这种特权会导致人们(立法者或医生)认为,该特权包含了在没有准许的情况下切除某个功能正常的其他器官的权力。

在这种驳斥之中,焦点在于论证形式。滑坡论的辩护者可能这样回击

我们上面引述的抨击:这里的类推反驳是不成功的。理由是,反驳的形式没有正确地反映原来论证的形式。无疑,这个争论将会继续下去。但是,这里使用的逻辑技术具有重大的意义:如果一个论证确实与被反驳的另外一个论证具有相同的形式,并且,用来作为类推的论证明显是糟糕的,那么,可以确定的是,被反驳的论证遭到了破坏。

在归纳情景下与在演绎情景下一样,一个由逻辑类推进行的反驳常常包含这样的句子"你不如说……",或其他与之有相同意思的语言。在上述引用的段落中,其破坏性类推提示性的语言是"这无异于说"。一个学者攻击如下论证:因为伊斯兰文化是从外面被带到乍得,在乍得它只不过是一个装饰。他说,"[你说]乍得仅有一个'伊斯兰装饰'。人们也会敏锐地说法国仅有一个'基督教装饰'"[8]。在这个类推反驳中所用的提示性语言稍有不同。

在类推反驳很明显的地方,不需要提示性的语言。前密西西比州州长柯尔克-福迪斯争辩道:"这是一个简单的事实,美国是一个基督教国家。"因为"在美国基督教是主要的宗教"。与他进行电视辩论的记者迈克尔-金塞以生动的类推进行了回击:"本国妇女占大多数,这能够使我们得出我国是女性国家吗?再者,我们能够因我国的大多数人是白人而得出,我国是一个白人国家吗?"[9] 