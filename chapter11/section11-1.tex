\section{类比论证}

\begin{quotation}
本节介绍类比论证的概念及其在日常推理中的重要性。我们将分析类比论证的基本结构,区分它与演绎论证的根本差异,探讨类比在论证和非论证语境中的多种用途,并学习如何识别和表达类比论证的基本形式,从而为理解这种归纳推理方式奠定基础。
\end{quotation}

前几章讨论的是演绎论证。演绎论证是否有效,取决于其前提是否能够证明地(demonstratively)得到结论。然而,还有许多良好的和重要的论证,这些论证的结论不能得到确定性的证明。我们充分相信许多因果连接(causal connections),只是基于盖然性(probability)一一尽管盖然性程度可能非常高。我们能够不加迟疑地说,吸烟是癌症的一个原因,但我们不能够赋予我们的这种知识与从前提中推得一个演绎有效的论证结论这种知识以相同的确定性。一个著名的医科专家根据演绎标准声称:"没有人将能够证明(prove)吸烟导致癌症,或者说任何事情导致任何事情。从理论上讲,你不能够证明任何事情。"\cite{surgeon1964} 的确,当我们评价我们关于世界的事实的知识时,演绎确定性的标准太高了。

本章及以后的各章将转向分析这样的论证:人们在这些论证中并不声称结论的真理性是从前提必然地得到,而仅仅表明,前提对结论的支持是或然的(probable),或者说结论盖然为真。这种论证被称为归纳论证,其与演绎论证具有根本性差异。我们已经在第1章中讨论了演绎和归纳之间的基本区别。本书第二部分已经对演绎进行了讨论,第三部分则用来讨论归纳。

在归纳论证中有一种被普遍使用的论证类型:类比(analogy)论证。下面是两个类比论证的例子:

一些人认为教师资格测验是不公正的双重测试。"教师已经是大学毕业生,"他们说,"他们为什么还要被测试?"其实这很简单。律师是大学毕业生,而且还是职业学院的毕业生,但他们不得不参加律师资格考试。还有其他大量的行业,如会计、精算师、医生、建筑师等,这些行业对想成为其成员的人都要求参加并通过资格考试,以证明他们的专业素质。没有理由说明教师不应当被要求做同样的事情。\cite{davis1986}

在我们居住的地球和其他行星(土星、木星、火星、金星和水星)之间,我们可以观察到许多类似之处。它们均如地球一样

围绕太阳运行,尽管它们绕太阳的半径不同、周期也不同。它们均从太阳那里获得光,地球也是如此。我们已经知道,其中一些行星,如地球一样,围绕它们的辑自转,因而它们必定有类似白天和黑夜的更替。一些行星有卫星,当太阳不再照射时,这些卫星给行星以光亮,就如我们的月亮给我们以光一样。这些行星的运动均与地球一样受制于万有引力定律。根据所有这些类似,认为这些行星可能与我们地球一样,有不同等级的生命存在,这不是不合理的。通过类比得到的这个结论具有一定程度的可能性。\cite{reid1785}

我们的许多日常推论是通过类比进行的。我推论我将从一台新的计算机那里得到好的服务,根据是,我从同样的生产厂家购买的一台计算机曾给了我很好的服务。我看到某个作者的新作,根据我读过该作者的其他著作并且喜欢这些著作而推断,我将喜欢读这本新作。我们过去的经验在未来同样成立的大多数日常推论,其基础就是类比。当然,我们无法给出一个清楚的公式化的论证,我们只能说,曾被烧伤的孩童躲避火的行为即涉及类比推论。

这些论证中没有一个是确定的或者说是证明性地有效的。这些论证中的结论,没有一个能够从前提中获得逻辑必然性。这是逻辑可能的:用来判断律师和医生资格的方法,并不适合于判断教师的资格;这也是逻辑可能的:地球可能是唯一可以居住的行星,新的计算机可能运转不灵,我喜欢的作者的新书可能无趣而难以卒读;甚至这也是逻辑可能的:一团火能够烧伤人,另外一团火则不会。没有一个类比论证可以指望具有数学的那种确定性。类比论证不是按有效和无效来区分的,我们只能用概率来刻画它们。

除了在论证中频繁使用类比外,人们为了描述生动,经常将类比用于非论证的活动中。明喻和暗喻为类比在文学中的用法,它们为作家给读者的心中创造鲜活的画面提供了莫大的帮助。例如:砧骨上做马蹄铁而产生的副产品火花一样。火花比马蹄铁更为灿

烂,但它们在本质上是无意义的。\cite{chesterton1910}

类比也用于说明,将读者不熟悉的某种东西,与读者比较熟悉的另一种东西进行对照,比较它们的类似之处,而使读者得以理解。麻省理工学院基因组研究中心主任埃瑞克-兰德试图说明人类基因组计划的巨大影晌。为了加强那些对基因研究不熟悉的人的理解,类比是他所用的一个工具:

基因组计划完全类似于化学中创立周期表。正如门捷列夫在周期表中安排化学元素,使得以前不相关的大量数据变得连贯,同样,当前有机体中上万的基因,将能够从较少数量的简单基因模块或单元即所谓原始基因的组合中得到。\cite{lander1995}

类比在描述和说明中的使用不同于在论证中的使用,尽管在某些实例下,不容易区分是哪一种用法。但是,无论是论证地使用类比还是其他的使用法,类比都是不难定义的。在两个或更多的实体之间进行一个类比,就是表明它们在一个或多个方面(respect)是类似的(similar)。

这说明了什么是类比,但是仍然没有刻画什么是类比论证。让我们考察一个类比论证事例并分析它的结构。我们选择上面引用的例子中最简单的例子:我新买的计算机将给我好的服务,因为我的一台旧计算机是从同样厂家购买的,它给了我好的服务。具有类似方面的两个事物是两台计算机。这里存在三点类比,两个事物被认为在三个方面相似:第一,均为计算机;第二,均在同样厂家购买;第三,给我好的服务。

然而,类比的这三点在论证中并不起相同的作用。前两点出现在前提中,而第三点既出现在前提中又出现在结论中。容易见得,该论证具有这样的前提:首先断定两个事物在两点类似,其次断定了其中的一个事物还具有另外一个特点,从而推论得出另一个事物也具有这个特点的结论。

类比论证是法庭最基本的工具之一。法官不是事先摆出严格的法规或原理,他们往往这样推理,因为两个案件——早先的已经被判决的案件和手头上待判决的案件——有相同的特点,它们应当具有相同的判决结果。例如,一旦做出了不能禁止 3 K 党发表言论的判决,那么法庭可能通过类比论证而得出不能禁止纳粹党游行的结论。\cite{collin1978} 通过判例的论证一旦做出,

人们将确定和强调以前的案子和手头案子之间类似的那些特点。\\
当然,不是每个类比论证都必须精确地涉及两个事物或者精确涉及三个不同的特点。托马斯-雷德(在上面我们已经提到)认为其他行星可能有人居住,他的论证是对六个事物(当时知道的行星)的八个方面进行类比的。然而,除了这些数量存在差别外,所有的类比论证均具有相同的一般结构或模式。每个类比推理都是这样进行的:从在一个或多个方面两个或更多的事物之间的类似性,到这些事物在某个其他方面具有类似性。我们可以将之公式化:$a 、 b 、 c 、 d$ 是实体,$P 、 Q 、 R$ 是属性或"相似方面",一个类比论证可以表示成下列形式:

$$
\begin{aligned}
& a 、 b 、 c 、 d \text { 均具有属性 } P \text { 和 } Q, \\
& a 、 b 、 c \text { 均具有属性 } R,
\end{aligned}
$$

$$
\text { 因而 } d \text { 可能具有属性 } R \text { 。 }
$$

在识别并且特别是评价类比论证时,将之表示成这种形式是有帮助的。

\begin{center}
\fbox{\parbox{0.95\textwidth}{
\textbf{本节要点}
\begin{itemize}
\item \textbf{类比论证的定义与性质}:
  \begin{itemize}
  \item 类比论证是归纳推理的一种重要形式
  \item 与演绎论证不同,类比论证的结论不是必然真的,而是或然的
  \item 类比论证不按有效/无效区分,而是用概率强度来评价
  \end{itemize}
\item \textbf{类比论证的标准形式}:
  \begin{itemize}
  \item 从两个或多个事物在某些方面的相似性
  \item 推断它们在其他方面也可能相似
  \item 可用符号公式表示:若a,b,c,d具有P和Q属性,且a,b,c具有R属性,则d可能具有R属性
  \end{itemize}
\item \textbf{类比的多种用途}:
  \begin{itemize}
  \item 论证用途:如法庭判例论证、科学假设论证等
  \item 非论证用途:文学中的明喻暗喻、教学说明工具等
  \item 日常推理中的广泛应用:预测新产品质量、新书价值等
  \end{itemize}
\item \textbf{类比论证与其他论证的区别}:
  \begin{itemize}
  \item 结论具有可能性而非确定性
  \item 强度取决于相似性的性质和程度
  \item 是我们日常经验应用的基础形式
  \end{itemize}
\end{itemize}
}}
\end{center}