\section*{11.2 类比论证的评价}
没有一个类比论证是演绎有效的。这里演绎有效的含义是指,从前提得出结论具有逻辑必然性。但是,有些类比论证比其他类比论证的论证力要强。类比论证被评价为比较好的还是比较差的,依赖于结论得以断定的或然性程度。

两个日常例子可以帮助我们弄清楚,什么因素影响类比论证的有力性。你决定去购买特定的一双鞋,因为以前与之类似的其他鞋子使你感觉很舒服;你挑选了一只某一品种的狗,因为该品种的其他狗所表现出来的特征是你所喜欢的。在这两个例子中使用了类比推理。为了评价这两个例子的论证强度——"实际上也就是所有类比论证的强度,我们可以确定 6 个显著的标准:

1.实体数量。如果我过去对特定种类的鞋子的经历仅限于我穿过的并喜欢的一双,对一双明显类似的鞋,我穿后发现具有意想不到的缺陷,这将使我很失望。但是如果我多次购买了那类鞋子,我可以有理由地认为,下一次购买的鞋子会与我以前穿的一样好。在同样对象上的多次的同种经验将支撑结论(购买的鞋子将是合脚的),比单个经验支撑结论有力得多。每一个经验可看成是一个附加实体,在评价类比论证中实体数量是第一个标准。

一般地讲,实体数量——过去所经历的场合数量——越大,论证越强。但是实体数量和结论成真的概率之间没有简单的比例关系。与机敏、温顺的金色猎犬愉快相处的 6 次经历,使人们相信下一个金色猎犬同样是机敏和温顺的。但是,前提中具有 6 个经历的类比论证其结论在可靠性上并不是前提中有 2 个经历的一个类似论证的 3 倍。增加实体数量是重要的,但其他因素也要增加。

2.前提中实例的多样性。如果我先前购买的那些合脚的鞋子,既有购买于大商店的,又有购买于专卖店的,既有在纽约制造的又有在加利福尼亚制造的,既有通过邮寄销售的,又有通过商店直接销售的,那么,我可以有信心地认为,鞋子合脚的原因在于鞋子本身,而不是售货员的服务。如果我先前的金色猎犬,既有公的也有母的,既有从小就领养的小犬,也有从仁慈的社会中得来的成年犬,我可以相信,正是犬的品种,而不是它们的性别、年龄或其来源,是它们先前与我愉快相处的原因。

我们可直觉地理解这个标准:类比论证的前提中所涉及的实例越不相似,论证越强。

3.相似方面的数量。在作为前提的实例中可能出现了大量的相似性:也许鞋子属于同一类型,具有同样的价格,由同样种类的皮革制成;也许猎犬是同样品种,在同样的年龄由同一个饲养人饲养,等等。前提中的这些实例在所有这些方面类似,并与结论中的实例在这些方面类似,增强了结论中的实例具有的新属性的可能性——新鞋子将合脚,一只新的狗会具有温顺的品行。这是该论证所要达到的目的。

我们同样可以直观地理解这个标准:结论中的实体与前提中的实体之间类似的方面越多,结论越可靠。但是,在结论和所识别出的类似方面的数量之间,也不存在简单的数字比例关系。

4.相关性。与共有的相似方面的数量同样重要的是,前提中的实例与结论中的实例在共有的相似方面的种类的类似。如果新鞋子与以前的鞋子一样,是在星期二购买的,这是一个与合脚没有关系的类似;但是,如果新的鞋子与先前购买的鞋子一样,由同样的厂商生产,这自然相当重要。当相似方面是相关的时候(如鞋子的样式、价格以及材料),相似方面便增加论证的力度,并且,单个的具有高相关因素比一批不相关的类似对论证的贡献更大。

至于哪些属性确实与论证结论的可靠性相关,人们有时意见不一致。但相关性本身的意义则不存在争论。当一个属性连接另外一个的时候,即当它们之间存在某种因果联系的时候,它们之间存在相关,那就是为什么确定因果联系在类比论证中是关键的原因,以及为什么在法庭上在确定证据是否有效(即相关还是不相关)的过程中,建立这样的连接往往是至关重要的原因所在。

类比论证之进行,无论是从原因到结果,还是从结果到原因,都是可能的。甚至,前提中的属性既不是结论巾的属性的原因,也不是其结果,而两者是同一原因的结果的话,此时,类比论证也是可能的。医生注意到她的病人出现了某种症状,她能够精确地预测另外的症状。这不是因为其中一个症状是另外一个的原因,而是因为身体的某种紊乱造成了它们的相继出现。一个产品的颜色往往与功能无关。但是,当那种颜色与众不同并且共同出现在前提和结论之中的时候,它可以作为论证的相关相似方面来使用。颜色本身可能与产品的功能无关,但是,如果我们知道该颜色是制

造商生产过程的一个相似方面,它可以用来进行一个论证。\\
因果连接是评价类比论证的关键,我们只能够通过观察和实验经验地发现它们。关于经验研究的一般理论是归纳逻辑关心的中心问题,在下面的几章中我们将以一定的篇幅来对之进行讨论。

5.差异性。一个差异(disanalogy)就是一个不同点,它是这样一个方面,在这个方面我们进行推理得出结论的实例有别于论证所基于的实例。回到鞋子的例子上来,如果我们想购买的这双鞋子看上去好像我们以前所穿的鞋子,但事实上这双鞋子更便宜,并且由不同的厂家生产,那么,这些差异使我们有理由对它能否使我们穿起来舒服产生怀疑。

上面论述的相关性在这里同样是重要的。当被确定了的差别具有相关性,与我们正在寻找的东西有因果连接的时候,差异使类比论证失效。投资者购买共有基金的股票,他们往往根据股票成功的"走势记录"。他们这样推理:先前的购买使资本得益,下一回的购买将同样使资本增益。当我们获悉在基金赢利期间操纵该基金的人刚刚被替换,我们面临着一个实质上的差异,它降低了类比论证的强度。

差异使类比论证减弱。因而,它们往往被用来攻击一个类比论证。正如批评者所认为的,结论中的情形在关键方面不同于早先发生的情形,因而在先前情形中正确的东西不可能在后面的情形中也正确。在司法中这是普遍使用类比的地方,某个(或某些)早先的案子通常作为对手头案件的判例被提供给法庭。论证是类比的。对方辩护律师努力将本案与以前的案子区别开来;即辩护律师努力表明,由于在本案中的事实与以前案子中的事实之间存在某个关键差别,以前的案件不是本案的恰当判例。如果差异较大,并且差异的确是关键性的,它能够成功地推翻所提出的类比论证。

因为差异性是反对类比论证的主要武器,因而,能够使潜在的差异得以消解的做法将加强该论证。这说明为什么在前提中实例的多样性会增强一个论证的力度,正如我们已经在前面第二个标准中所述。前提中的实例之间变化越大,批评者越不可能在前提中的实例与结论之间找到使论证减弱的差异。举例来说:吉姆-库玛尔进入一所大学,成为大一学生;来自于吉姆所在高中的另外十个学生已经在该大学里成功完成了学业。我们可以类比地说,她成功地完成学业是很可能的。在与大学的学习有关的某个方面,如果所有这些学生之间都类似,但他们在该方面与吉姆不同,该差异削弱吉姆将成功的论证。但是,如果我们了解到十个成功的师兄师姐在

许多方面——"在经济背景、家庭关系、宗教背景等方面——相互不同,他们间的这些不同使潜在的差异得以消解。如果以来自于同一高中的其他学生作为论证的前提,这些学生并不紧密地相似,而呈现出变化,那么,正如我们前面看到的,吉姆将成功的论证得到加强。

必须避免的一个混淆是:差异是类比论证得以弱化的原理,与前提中的差别(dissimilarity)使同样的论证加强的原理不同。对于前者,差异发生在前提中的实例与结论中的实例之间;对于后者,差别仅仅发生在前提的实例之间。一个差异(不相似)指的是,我们已经经历的实例和要得出的结论的实例之间的区别。由于结论中的实例与早先的实例不一样,结论得不到保证。(我们可以通过提出差异来进行反驳。)该类比被认为是 "牵强的"(strained)或者"行不通的"。但是当我们指出前提间的差别的时候,我们则强化了论证:我们说,该类比有广泛的效力,它在这些实例和那些实例中都行得通,因而前提中实例多种多样的相似方面与结论涉及的相似方面不相关。

总之,差异削弱一个类比论证;而前提中的差别使类比论证加强。这两方面都与恰当性问题相关联:差异表明了前提中的实例和结论中的实例在某些相关方面存在不同;而前提中的差别所表明的是,我们原以为与某个属性存在因果相关的事情事实上毫无关联。

需要注意的是,被认为是第一重要的标准,即被认为具有相似性的实体的数量,也与相关性有关。实例数越多,它们之间的差别也就可能越多。因而增加实体数量是人们所希望的——但是随着实体数量的增加,增加的实例其作用在降低。因为它所提供的差别更可能由先前的实例所提供———这样的话,增加的实例对于保护结论免遭差异性产生的破坏,起不到或几乎起不到作用。

6.结论所做的断言。每个论证均断言其前提给出了接受结论的理由。容易看到,论证断言得越多,保持该断言的负担也就越大。这对每个类比论证均是正确的。结论相对于前提而言是否适度在推理的评价中起关键作用。

如果我的朋友的新车每加仑汽油能行驶 30 英里,我会得出如果我购买同样厂家和同样型号的车,我至少能够使该车每加仑汽油行驶 20 英里。该结论是适度的,因而可靠性十分大。如果我的结论十分大胆,如,我将至少使每加仑汽油行驶 29 英里,该结论受我拥有的证据的支持程度就很

低。一般而言,断言越适度,加于前提的负担越轻,论证越强;断言越大胆,前提的负担越大,论证也就越弱。

通过减少在确定的前提下的断言的内容,或者使断言维持不变但用附加的或更强大的前提给予它以支持,一个类比论证就得以加强。类似的,如果一个类比论证的结论变得更大胆,而前提保持不变,或者断言维持不变,但我们发现支持它的证据存在较大的缺陷,一个类比论证就被减弱。 