\chaptersummary{
本章深入探讨了\logicterm{命题逻辑}的核心内容,介绍了如何使用形式化方法分析复合命题和论证的\logicemph{有效性}。

9.1节讨论了\logicterm{真值函项逻辑}系统中的等价关系。我们学习了如何使用\logicterm{真值表}判断两个陈述形式是否等价,以及几种重要的等价关系,包括\logicterm{双重否定律}、\logicterm{德·摩根定律}、\logicterm{交换律}、\logicterm{结合律}、\logicterm{分配律}等。

9.2节介绍了\logicterm{推理规则}和\logicterm{形式证明}的方法。我们了解了19个基本\logicterm{推理规则},包括\logicterm{分离规则}、\logicterm{附加规则}、\logicterm{简化规则}等,这些规则构成了\logicterm{真值函项逻辑}的一个完全系统。\logicterm{形式证明}是验证论证\logicemph{有效性}的严格方法,通过有限步骤的逻辑推导展示结论如何从前提得出。

9.3节探讨了\logicterm{间接证明法}和\logicterm{归谬法}。这些方法通过假设结论的否定,然后导出矛盾来证明原论证的\logicemph{有效性}。\logicterm{间接证明}是处理复杂论证的有力工具,尤其在直接证明难以构造的情况下。

9.4节分析了\logicterm{不相容性}的逻辑后果。不相容的一组陈述可以推出任何结论,这种现象称为"\logicterm{爆炸原理}"。我们讨论了检测\logicterm{不相容性}的方法,以及\logicterm{不相容性}在实际推理中的重要意义。

通过本章的学习,我们掌握了\logicterm{命题逻辑}的形式化证明方法,这些方法对于理解和分析复杂的论证结构至关重要。
}

\printbibliography[heading=subbibliography,title={第9章参考文献}]