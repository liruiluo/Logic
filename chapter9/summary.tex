\section{第9章概要}
本章深入探讨了命题逻辑的核心内容,介绍了如何使用形式化方法分析复合命题和论证的有效性。

9.1节讨论了真值函项逻辑系统中的等价关系。我们学习了如何使用真值表判断两个陈述形式是否等价,以及几种重要的等价关系,包括双重否定律、德·摩根定律、交换律、结合律、分配律等。

9.2节介绍了推理规则和形式证明的方法。我们了解了19个基本推理规则,包括分离规则、附加规则、简化规则等,这些规则构成了真值函项逻辑的一个完全系统。形式证明是验证论证有效性的严格方法,通过有限步骤的逻辑推导展示结论如何从前提得出。

9.3节探讨了间接证明法和归谬法。这些方法通过假设结论的否定,然后导出矛盾来证明原论证的有效性。间接证明是处理复杂论证的有力工具,尤其在直接证明难以构造的情况下。

9.4节分析了不相容性的逻辑后果。不相容的一组陈述可以推出任何结论,这种现象称为"爆炸原理"。我们讨论了检测不相容性的方法,以及不相容性在实际推理中的重要意义。

通过本章的学习,我们掌握了命题逻辑的形式化证明方法,这些方法对于理解和分析复杂的论证结构至关重要。

\printbibliography[heading=subbibliography,title={第9章参考文献}] 