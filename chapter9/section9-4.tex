\section{不相容性}

\begin{quotation}
本节探讨前提互不相容的情况及其对演绎论证有效性的影响。我们将分析为什么不相容的前提集可以推出任何结论,即使这些结论看似无关或荒谬。通过理解这一"严格蕴涵怪论",我们能更全面地把握逻辑有效性的本质以及相容性在理性思维中的重要地位。
\end{quotation}

如果一种真值指派能使得一个论证的所有前提为真而结论为假,那么,这表明该论证是无效的。而如果一个演绎论证不是无效的,那么,它必定是有效的。因此,如果不可能对一个论证的简单分支陈述进行这种真值指派,即不可能使得它的前提为真而结论为假,那么该论证必定有效。尽管从"有效性"的定义可以推出这一点,但它也有一个怪异的推论。考虑下面的论证,它的前提看上去与结论完全不相干:

\begin{quote}
如果飞机的引擎出了故障,它就降落在本德了。\\
如果飞机的引擎没有出故障,它就降落在克利夫兰了。\\
飞机没有降落在本德或克利夫兰。\\
因此,飞机必定降落在丹佛了。
\end{quote}

把它翻译成符号就是:

$$
\begin{aligned}
& A \supset B \\
& \sim A \supset C \\
& \sim(B \vee C) \\
& \therefore D
\end{aligned}
$$

对其简单分支陈述进行使其所有前提为真而结论为假的真值指派的努力,都注定会失败。如果我们忽略结论,把注意力放在对其简单分支陈述进行使其所有前提都为真的真值指派上,我们也一定会失败——尽管这个计划初看上去并不难以实现。

\subsection{前提不相容与有效性}

这里之所以不能获得前提都真而结论为假的真值指派,乃因为在任何情形下使用任何真值指派都不能使前提都真。由于前提是\textbf{互不相容}的,故没有真值指派能使它们都真。前提的合取作为一个矛盾的陈述形式的代人例,乃是自相矛盾的。如果我们构造该论证的真值表,就会发现在每一行中至少有一个前提是假的。因为没有所有前提都为真这样一行,也就没有所有前提为真而结论为假这样一行。因此,该论证的真值表确立了它的有效性。下面的形式证明也可以确立它的有效性:

1.$A \supset B$\\
2.$\sim A \supset C$\\
3.$\sim(B \vee C)$\\
$\therefore D$\\
4.$\sim B \cdot \sim C$\\
5.$\sim B$\\
6.$\sim A$\\
7.$C$\\
8.$\sim C \cdot \sim B$\\
9.$\sim C$

3,De M.\\
4,Simp.\\
$1,5, \mathrm{M} . \mathrm{T}$ .\\
2,6,M.P.\\
4,Com.\\
8,Simp.

\begin{center}
\begin{tabular}{ll}
10.$C \vee D$ & 7, Add. \\
11.$D$ & 10,9, D.S. \\
\end{tabular}
\end{center}

在这个证明中, 1 至 9 行表明了前提中隐含的不相容性。这种不相容性呈现在第 7 行和第 9 行,它们分别断言了 $C$ 和 $\sim C$ 。一旦这种明显的矛盾被表示出来,根据附加律和析取三段论原理,很快就可以推出结论。

由此可见,如果一组前提不相容,这些前提就会有效地产生任何结论,而不论它们如何不相干。下面的论证更简单地表明了这一问题的精髓,其公然不相容的前提使得我们可以有效地推出一个不相干且荒谬的结论:

今天是星期天。\\
今天不是星期天。\\
因此,月亮是鲜奶酪做的。

用符号表示就是:\\
1.$S$\\
2.$\sim S$\\
$\therefore M$\\
它的有效性的形式证明十分显然:

\begin{center}
\begin{tabular}{ll}
3.$S \vee W$ & 1, Add. \\
4.$M$ & 3,2, D.S. \\
\end{tabular}
\end{center}

\subsection{不相容前提的问题}

问题出在哪里呢?如此贫乏甚至不相容的前提怎能使得它们在其中出现的论证有效?首先要注意到,如果一个论证因其前提的不相容性而有效,那么它不可能是一个合理的论证。如果前提互不相容,它们不可能都是真的。一个前提不相容的论证不能确立任何结论的真,因为它的前提本身不可能都是真的。

目前情形与所谓\textbf{实质蕴涵怪论}密切相关。在讨论后者时,我们注意到 (在 8.7 节),陈述形式 $\sim p \supset ~(p \supset q) ~$ 是一个重言式,其所有代人例都为真。它的自然语言表述断言的是:"如果一个陈述为假,那么它实质蕴涵任何陈述。"用真值表很容易证明这一点。当下讨论所确立的是下述论证形式有效:\\
$\sim p$\\
$\therefore q$\\
我们已经证明:不管其结论是什么,任何前提不相容的论证都是有效的。它的有效性可以用真值表,或者用形式证明判定。

一个有效论证的前提蕴涵它的结论,不仅仅是"实质"蕴涵意义上的,还有逻辑的或"严格"意义上的蕴涵。在一个有效论证中,当结论为假时,其前提为真是逻辑不可能的。只要前提为真是逻辑不可能的,即使忽略结论的真假问题,这种情形也照样成立。它和实质蕴涵相应性质的相似性,使某些逻辑学者称之为"\textbf{严格蕴涵怪论}"。然而,根据逻辑学家对 "有效性"的技术性定义.它似乎并不是特别怪异的。所宣称的这个怪论之所以产生,主要是由于把一个技术性术语当成日常语言中的普通术语。

\subsection{相容性的重要性}

前面的讨论有助于解释为什么对相容性评价如此之高。其基本原因当然是,两个不相容的陈述不能都是真的。这一事实乃是交互询问策略的基石。在交互询问中,律师会设法使对方证人陷人自相矛盾。如果证词肯定了不能自圆其说或不相容的断言,那么证词不能都是真的,证人的可信性就被破坏…或至少被动摇。 ${ }^{[4]}$ 不相容性令人如此反感的另一个原因是,任何结论都可从一些被当做前提的不相容陈述逻辑地推出。不相容陈述并不是"没有意义的",它们的麻烦正好相反:其意谓太多。在蕴涵任何东西这个意义卜说,它们意谓着所有东西。如果所有东西都被断言,那么被断言的有一半肯定是假的,因为每个陈述都有一个否定。

上面的讨论附带地为我们解答了一个古老难题:一个不可抗拒的力量遇到一个不可移动的物体,会发生什么事?这个描述含有一个矛盾。要一个不可抗拒的力量遇到一个不可移动的物体,这两者都必须存在。必定存在一个不可抗拒的力量,并且也必定存在一个不可移动的物体。但如果存在不可抗拒的力量,就不会存在不可移动的物体。在此,矛盾被表述得很清楚:存在一个不可移动的物体,并且不存在一个不可移动的物体。给定这种不相容的前提,任何结论都可有效地推出。因此,对"一个不可抗拒的力量遇到一个不可移动的物体,会发生什么事?"这一问题的正确回答是"任何事"!

\subsection{不相容性与幽默}

尽管在一个论证中发现不相容性是灾难性的,但正如伟大的棒球运动员扬基队的贝拉经常被引用的评论那样,不相容性是非常有趣的。据说,

贝拉曾宣称"那个餐馆如此拥挤以致不再有人去那儿了"。在谈到他的那段长而幸福的婚姻中的伴侣时,他说:"我们长时间待在一起,即使我们不在一起时也是如此。"

这些话语很有趣,因为它们所包含的矛盾(若照字面意义理解,这些评论都是胡说),似乎没被它们的作者意识到。因此,当我们听到学生说,澳大利亚内地的气候如此不好,以致居民不再住在那儿了,我们会暗自发笑。这种漫不经心且未意识到的不相容话语,有时被称为"Irish Bull" (爱尔兰牛皮)。

从逻辑上看,不相容的命题集不可能同时为真。但人们并非总是合乎逻辑的,有时确实会说出甚至会相信两个互相矛盾的命题。这一点似乎难以置信,但逻辑领域一个非常值得信赖的权威刘易斯•卡罗尔告诉我们,《爱丽丝漫游奇境记》中的白衣女王形成了这样一个习惯,即在早餐之前相信六件不可能的事。 

\begin{center}
\fbox{\parbox{0.95\textwidth}{
\textbf{本节要点}
\begin{itemize}
\item \textbf{前提不相容与论证有效性}:
  \begin{itemize}
  \item 互不相容的前提集意味着不可能所有前提同时为真
  \item 因此无法找到"前提全真而结论为假"的真值指派
  \item 这导致不相容前提的论证形式必定是有效的
  \end{itemize}
\item \textbf{严格蕴涵怪论}:
  \begin{itemize}
  \item 不相容的前提可以有效推出任何结论
  \item 形式上类似于实质蕴涵中"假陈述蕴涵任何陈述"
  \item 这种怪论源于逻辑术语的技术性定义与日常用法的差异
  \end{itemize}
\item \textbf{相容性的价值}:
  \begin{itemize}
  \item 相容性是理性思维的基础
  \item 不相容陈述的问题不是"无意义"而是"意义过多"
  \item 不相容陈述集逻辑上蕴涵一切可能的结论
  \end{itemize}
\item \textbf{不相容性在日常语言中}:
  \begin{itemize}
  \item 有时产生幽默效果(如"爱尔兰牛皮")
  \item 在法律中作为质疑证词可信度的基础
  \item 说明人类思维并非总是严格遵循逻辑规则
  \end{itemize}
\end{itemize}
}}
\end{center} 