\section{有效性的形式证明}

\begin{logicbox}[title=引言]
本节介绍如何通过一系列已知\logicemph{有效的}基本论证形式,来证明复杂论证的\logicemph{有效性}。我们将学习\logicterm{形式证明}的方法,以及九种常用的\logicterm{推论规则},这些规则可以帮助我们构造\logicemph{有效性}的形式证明。
\end{logicbox}

\subsection{形式证明方法的理论基础}

从理论上说,\logicterm{真值表}足以检验这里所探讨的任何一般类型论证的\logicemph{有效性}。但从实际操作上考虑,随着分支陈述数量的增加,真值表判定就变得愈益笨拙。

\begin{theorembox}[title=真值表方法的局限性]
真值表方法虽然理论上完备,但在实际应用中存在显著局限:

\textbf{1. 指数增长问题}:对于包含$n$个不同简单陈述的论证,需要构造$2^n$行的真值表。当$n=10$时,需要1024行;当$n=20$时,需要超过100万行。

\textbf{2. 机械性缺陷}:真值表方法是纯机械的,不能揭示论证的逻辑结构和推理步骤,缺乏洞察力。

\textbf{3. 可读性问题}:大型真值表难以阅读和验证,容易出错,不利于理解论证的逻辑关系。

\textbf{4. 教学局限}:真值表不能展示人类实际的推理过程,不利于培养逻辑思维能力。
\end{theorembox}

判定更复杂论证之\logicemph{有效性}的更有力的方法,是运用一系列已知\logicemph{有效的}基本论证,将其结论从前提演绎出来。这一方法也与日常论证方法相当吻合。

\begin{examplebox}[title=形式证明的优势]
\textbf{1. 效率优势}:对于复杂论证,形式证明通常只需要几个步骤,而真值表可能需要数千行。

\textbf{2. 结构清晰}:形式证明清楚地展示了从前提到结论的推理路径,揭示了论证的逻辑结构。

\textbf{3. 自然性}:形式证明模拟了人类的自然推理过程,符合我们的思维习惯。

\textbf{4. 可扩展性}:形式证明方法可以处理任意复杂的论证,不受陈述数量限制。
\end{examplebox}

\begin{examplebox}[title=形式证明的应用实例]
例如,考虑下述论证:

如果安德逊被提名,那么她会去波士顿。

如果她去波士顿,那么她会在那儿竞选。

如果她在那儿竞选,她会遇到道格拉斯。

安德逊没有遇到道格拉斯。

或者安德逊被提名,或者某个更合适的人被选中。

因此,某个更合适的人会被选中。
\end{examplebox}

它的\logicemph{有效性}可能在直觉上也很显然,但我们来考虑一下证明问题。为讨论方便起见,先把该论证翻译成下列符号表达式:
$A \supset B$
$B \supset C$
$C \supset D$
$\sim D$
$A \vee E$
$\therefore E$

若用\logicterm{真值表}判定这个论证的\logicemph{有效性},要求一个有 32 行的表,因为它涉及五个不同的简单陈述。但用一个只含有四个基本\logicemph{有效}论证的序列,将其结论从前提演绎出来,就能证明该论证\logicemph{有效}。从前两个前提,$A \supset B$ 和 $B \supset C$ ,根据\logicterm{假言三段论},可\logicemph{有效地}推出 $A \supset C$ 。根据另一个\logicterm{假言三段论},从 $A \supset C$ 和第三个前提 $C \supset D$ ,可\logicemph{有效地}推出 $A \supset D$ 。根据\logicterm{否定后件式},从 $A \supset D$ 和第四个前提 $\sim D$ ,可\logicemph{有效地}推出 $\sim A$ 。从 $\sim A$ 和第五个前提 $A \vee E$ ,据\logicterm{析取三段论},可\logicemph{有效地}推出该论证的结论 $E$ 。用这四个基本的\logicemph{有效}论证,结论可以从原论证的五个前提中演绎出来,这就证明了原论证是\logicemph{有效的}。在这里,基本的\logicemph{有效}论证形式,如\logicterm{假言三段论}(H.S.)、\logicterm{否定后件式}(M.T.)和\logicterm{析取三段论}(D.S.),是作为\logicterm{推理规则}来使用的。根据它们,结论从前提中\logicemph{有效地}演绎或推论出来。

把前提及从它们推出的那些陈述写在一栏,并在每个这样的陈述右边另立一栏,写上"理由",即我们所能给出的在证明中得到它的原因,我们就给出了\logicemph{有效性}的一个更形式化的证明。可以先直接列出所有前提,然后另立一行写下结论,稍微靠右侧把结论和前提分开。若所有陈述都编了号,那么,每个陈述的"理由"都要包括该陈述从之推出的那些在先的陈述的编号,以及它所依据的\logicterm{推理规则}的缩写。上述论证的\logicterm{形式证明}可以写成:

\begin{examplebox}[title=形式证明的格式]
1.$A \supset B$

2.$B \supset C$

3.$C \supset D$

4.$\sim D$

5.$A \vee E$

$\therefore E$

6.$A \supset C \quad 1,2, H . S$ .

7.$A \supset D \quad 6,3, \mathrm{H} . \mathrm{S}$ .

8.$\sim A \quad 7,4, \mathrm{M} . \mathrm{T}$ .

9.$E$ 5,8,D.S.
\end{examplebox}

\begin{theorembox}[title=形式证明的定义]
我们把给定论证的\logicemph{有效性}的一个\logicterm{形式证明}定义为一个陈述序列,该序列中的每个陈述或者是该论证的一个前提,或者是根据一个基本\logicemph{有效}论证从该序列中在先的陈述推论出来的,而该序列的最后一个陈述,就是所欲证明其\logicemph{有效性}的那个论证的结论。
\end{theorembox}

我们把一个\logicterm{基本的有效论证}定义为:该论证是一个基本的\logicemph{有效}论证形式的代入例。要强调的一点是,一个基本\logicemph{有效}论证形式的任何代人例都是一个基本\logicemph{有效}论证。例如,如下论证:

$$
\begin{aligned}
& (A \cdot B) \supset[C \equiv(D \vee E)] \\
& A \cdot B \\
& \therefore C \equiv(D \vee E)
\end{aligned}
$$

是一个基本的有效论证,因为它是基本的有效论证形式肯定前件式\\
(M.P.)的代人例。用 $A \cdot B$ 代人 $p, C \equiv(D \vee E)$ 代人 $q$ ,它可以从下述形式产生:

$$
\begin{aligned}
& p \supset q \\
& p \\
& \therefore q
\end{aligned}
$$

因此,尽管肯定前件式不是该论证的特征形式,它仍是具有肯定前件式的有效形式。

肯定前件式无疑是一个非常基本的有效论证形式,推理规则中还包括其他哪些有效论证形式呢?

\subsection{推论规则的理论基础}

\begin{theorembox}[title=推论规则的选择原则]
构造有效性的形式证明时常用的九个推论规则的选择基于以下原则:

\textbf{1. 基础性}:这些规则都是最基本、最直观的有效论证形式,符合人类的自然推理习惯。

\textbf{2. 完备性}:这九个规则加上后续的替换规则,构成了命题逻辑的一个完全系统,能够证明所有有效的真值函项论证。

\textbf{3. 独立性}:虽然某些规则在理论上可以相互推导,但保留它们是为了提高证明的效率和直观性。

\textbf{4. 实用性}:这些规则在实际的逻辑分析和数学证明中使用频率最高,具有重要的实用价值。
\end{theorembox}

\subsection{九种基本推论规则的系统分析}

如下是构造有效性的形式证明时常用的九个推论规则:
1.\textbf{肯定前件式}(M.P.)\\
$p \supset q$\\
$p$\\
$\therefore q$\\
3.\textbf{假言三段论}(H.S.)\\
$p \supset q$\\
$q \supset r$\\
$\therefore p \supset r$\\
5.\textbf{构造式二难}(C.D.)\\
$(p \supset q) \cdot(r \supset s)$\\
$p \vee r$\\
$\therefore q \vee s$\\
7.\textbf{简化律}(Simp.)\\
$p \cdot q$\\
$\therefore p$

2.\textbf{否定后件式}(M.T.)\\
$p \supset q$\\
$\sim q$\\
$\therefore \sim p$\\
4.\textbf{析取三段论}(D.S.)\\
$p \vee q$\\
$\sim p$\\
$\therefore q$\\
6.\textbf{吸收律}(Abs.)\\
$p \supset q$\\
$\therefore p \supset(p \cdot q)$

8.\textbf{合取律}(Conj.)\\
$p$\\
$q$\\
$\therefore p \cdot q$

9.\textbf{附加律}(Add.)\\
$p$\\
$\therefore p \vee q$

\subsection{推论规则的深入分析}

这九个推论规则的有效性很容易用真值表判定。在它们的帮助下,可以为大量更复杂的论证构造有效性的形式证明。

\begin{theorembox}[title=推论规则的分类与特征]
这九个推论规则可以按其逻辑功能进行分类:

\textbf{1. 条件推理规则}:
\begin{itemize}
\item \textbf{肯定前件式}(M.P.):最基本的条件推理,体现了"如果-那么"的核心逻辑
\item \textbf{否定后件式}(M.T.):反向条件推理,是反证法的基础
\item \textbf{假言三段论}(H.S.):条件链式推理,体现了传递性
\end{itemize}

\textbf{2. 析取推理规则}:
\begin{itemize}
\item \textbf{析取三段论}(D.S.):排除法推理,体现了"非此即彼"的逻辑
\item \textbf{构造式二难}(C.D.):复合析取推理,处理多重选择情况
\end{itemize}

\textbf{3. 合取操作规则}:
\begin{itemize}
\item \textbf{简化律}(Simp.):从合取中提取单个合取支
\item \textbf{合取律}(Conj.):将多个陈述合并为合取
\end{itemize}

\textbf{4. 扩展规则}:
\begin{itemize}
\item \textbf{附加律}(Add.):向陈述添加析取支
\item \textbf{吸收律}(Abs.):条件陈述的内部扩展
\end{itemize}
\end{theorembox}

\begin{examplebox}[title=推论规则的历史发展]
\textbf{古代起源}:肯定前件式和否定后件式可以追溯到古希腊的斯多葛学派,是最早被系统化的推理形式。

\textbf{中世纪发展}:假言三段论在中世纪逻辑学中得到了详细的分析和应用。

\textbf{现代形式化}:19世纪末20世纪初,这些规则被纳入现代符号逻辑体系,获得了精确的形式化表述。

\textbf{计算机应用}:在现代计算机科学中,这些规则成为了自动定理证明和专家系统的基础。
\end{examplebox}

所列的这些名称都是标准名称,而使用它们的缩写使得用最少量的书写就可以把形式证明记录下来。这种标准化的符号系统不仅提高了效率,也促进了国际学术交流。

\begin{center}
\fbox{\parbox{0.95\textwidth}{
\textbf{本节要点}
\begin{itemize}
\item \textbf{形式证明方法的理论基础}:
  \begin{itemize}
  \item 真值表方法的四大局限性:指数增长、机械性缺陷、可读性问题、教学局限
  \item 形式证明的四大优势:效率优势、结构清晰、自然性、可扩展性
  \item 形式证明模拟人类自然推理过程,与日常论证方法相当吻合
  \end{itemize}
\item \textbf{形式证明的定义与结构}:
  \begin{itemize}
  \item 陈述序列中每个陈述要么是原论证的前提,要么是根据推理规则得出
  \item 最后一个陈述是原论证的结论
  \item 每个推导步骤都必须明确标注所使用的推理规则和前提行号
  \end{itemize}
\item \textbf{推论规则的选择原则}:
  \begin{itemize}
  \item \textbf{基础性}:最基本、最直观的有效论证形式
  \item \textbf{完备性}:与替换规则一起构成命题逻辑的完全系统
  \item \textbf{独立性}:保留冗余规则以提高证明效率和直观性
  \item \textbf{实用性}:在逻辑分析和数学证明中使用频率最高
  \end{itemize}
\item \textbf{九种推论规则的系统分类}:
  \begin{itemize}
  \item \textbf{条件推理规则}:肯定前件式(M.P.)、否定后件式(M.T.)、假言三段论(H.S.)
  \item \textbf{析取推理规则}:析取三段论(D.S.)、构造式二难(C.D.)
  \item \textbf{合取操作规则}:简化律(Simp.)、合取律(Conj.)
  \item \textbf{扩展规则}:附加律(Add.)、吸收律(Abs.)
  \end{itemize}
\item \textbf{推论规则的历史发展}:
  \begin{itemize}
  \item 古代起源:斯多葛学派的肯定前件式和否定后件式
  \item 中世纪发展:假言三段论的详细分析
  \item 现代形式化:19-20世纪的符号逻辑体系
  \item 计算机应用:自动定理证明和专家系统的基础
  \end{itemize}
\item \textbf{基本有效论证与代入例}:
  \begin{itemize}
  \item 基本有效论证形式的任何代入例都是基本有效论证
  \item 复杂陈述可以作为简单变元的代入,扩展了规则的适用范围
  \item 标准化符号系统提高效率并促进国际学术交流
  \end{itemize}
\end{itemize}
}}
\end{center}