\section{有效性的形式证明}

\begin{logicbox}[title=引言]
本节介绍如何通过一系列已知\logicemph{有效的}基本论证形式,来证明复杂论证的\logicemph{有效性}。我们将学习\logicterm{形式证明}的方法,以及九种常用的\logicterm{推论规则},这些规则可以帮助我们构造\logicemph{有效性}的形式证明。
\end{logicbox}

从理论上说,\logicterm{真值表}足以检验这里所探讨的任何一般类型论证的\logicemph{有效性}。但从实际操作上考虑,随着分支陈述数量的增加,真值表判定就变得愈益笨拙。判定更复杂论证之\logicemph{有效性}的更有力的方法,是运用一系列已知\logicemph{有效的}基本论证,将其结论从前提演绎出来。这一方法也与日常论证方法相当吻合。

\begin{examplebox}[title=形式证明的应用实例]
例如,考虑下述论证:

如果安德逊被提名,那么她会去波士顿。

如果她去波士顿,那么她会在那儿竞选。

如果她在那儿竞选,她会遇到道格拉斯。

安德逊没有遇到道格拉斯。

或者安德逊被提名,或者某个更合适的人被选中。

因此,某个更合适的人会被选中。
\end{examplebox}

它的\logicemph{有效性}可能在直觉上也很显然,但我们来考虑一下证明问题。为讨论方便起见,先把该论证翻译成下列符号表达式:
$A \supset B$
$B \supset C$
$C \supset D$
$\sim D$
$A \vee E$
$\therefore E$

若用\logicterm{真值表}判定这个论证的\logicemph{有效性},要求一个有 32 行的表,因为它涉及五个不同的简单陈述。但用一个只含有四个基本\logicemph{有效}论证的序列,将其结论从前提演绎出来,就能证明该论证\logicemph{有效}。从前两个前提,$A \supset B$ 和 $B \supset C$ ,根据\logicterm{假言三段论},可\logicemph{有效地}推出 $A \supset C$ 。根据另一个\logicterm{假言三段论},从 $A \supset C$ 和第三个前提 $C \supset D$ ,可\logicemph{有效地}推出 $A \supset D$ 。根据\logicterm{否定后件式},从 $A \supset D$ 和第四个前提 $\sim D$ ,可\logicemph{有效地}推出 $\sim A$ 。从 $\sim A$ 和第五个前提 $A \vee E$ ,据\logicterm{析取三段论},可\logicemph{有效地}推出该论证的结论 $E$ 。用这四个基本的\logicemph{有效}论证,结论可以从原论证的五个前提中演绎出来,这就证明了原论证是\logicemph{有效的}。在这里,基本的\logicemph{有效}论证形式,如\logicterm{假言三段论}(H.S.)、\logicterm{否定后件式}(M.T.)和\logicterm{析取三段论}(D.S.),是作为\logicterm{推理规则}来使用的。根据它们,结论从前提中\logicemph{有效地}演绎或推论出来。

把前提及从它们推出的那些陈述写在一栏,并在每个这样的陈述右边另立一栏,写上"理由",即我们所能给出的在证明中得到它的原因,我们就给出了\logicemph{有效性}的一个更形式化的证明。可以先直接列出所有前提,然后另立一行写下结论,稍微靠右侧把结论和前提分开。若所有陈述都编了号,那么,每个陈述的"理由"都要包括该陈述从之推出的那些在先的陈述的编号,以及它所依据的\logicterm{推理规则}的缩写。上述论证的\logicterm{形式证明}可以写成:

\begin{examplebox}[title=形式证明的格式]
1.$A \supset B$

2.$B \supset C$

3.$C \supset D$

4.$\sim D$

5.$A \vee E$

$\therefore E$

6.$A \supset C \quad 1,2, H . S$ .

7.$A \supset D \quad 6,3, \mathrm{H} . \mathrm{S}$ .

8.$\sim A \quad 7,4, \mathrm{M} . \mathrm{T}$ .

9.$E$ 5,8,D.S.
\end{examplebox}

\begin{theorembox}[title=形式证明的定义]
我们把给定论证的\logicemph{有效性}的一个\logicterm{形式证明}定义为一个陈述序列,该序列中的每个陈述或者是该论证的一个前提,或者是根据一个基本\logicemph{有效}论证从该序列中在先的陈述推论出来的,而该序列的最后一个陈述,就是所欲证明其\logicemph{有效性}的那个论证的结论。
\end{theorembox}

我们把一个\logicterm{基本的有效论证}定义为:该论证是一个基本的\logicemph{有效}论证形式的代入例。要强调的一点是,一个基本\logicemph{有效}论证形式的任何代人例都是一个基本\logicemph{有效}论证。例如,如下论证:

$$
\begin{aligned}
& (A \cdot B) \supset[C \equiv(D \vee E)] \\
& A \cdot B \\
& \therefore C \equiv(D \vee E)
\end{aligned}
$$

是一个基本的有效论证,因为它是基本的有效论证形式肯定前件式\\
(M.P.)的代人例。用 $A \cdot B$ 代人 $p, C \equiv(D \vee E)$ 代人 $q$ ,它可以从下述形式产生:

$$
\begin{aligned}
& p \supset q \\
& p \\
& \therefore q
\end{aligned}
$$

因此,尽管肯定前件式不是该论证的特征形式,它仍是具有肯定前件式的有效形式。

肯定前件式无疑是一个非常基本的有效论证形式,推理规则中还包括其他哪些有效论证形式呢?如下是构造有效性的形式证明时常用的九个推论规则:

\subsection{推论规则}
1.\textbf{肯定前件式}(M.P.)\\
$p \supset q$\\
$p$\\
$\therefore q$\\
3.\textbf{假言三段论}(H.S.)\\
$p \supset q$\\
$q \supset r$\\
$\therefore p \supset r$\\
5.\textbf{构造式二难}(C.D.)\\
$(p \supset q) \cdot(r \supset s)$\\
$p \vee r$\\
$\therefore q \vee s$\\
7.\textbf{简化律}(Simp.)\\
$p \cdot q$\\
$\therefore p$

2.\textbf{否定后件式}(M.T.)\\
$p \supset q$\\
$\sim q$\\
$\therefore \sim p$\\
4.\textbf{析取三段论}(D.S.)\\
$p \vee q$\\
$\sim p$\\
$\therefore q$\\
6.\textbf{吸收律}(Abs.)\\
$p \supset q$\\
$\therefore p \supset(p \cdot q)$

8.\textbf{合取律}(Conj.)\\
$p$\\
$q$\\
$\therefore p \cdot q$

9.\textbf{附加律}(Add.)\\
$p$\\
$\therefore p \vee q$

之有效性很容易用真值表判定。在它们的帮助下,可以为大量更复杂的论证构造有效性的形式证明。所列的这些名称都是标准名称,而使用它们的缩写使得用最少量的书写就可以把形式证明记录下来。

\begin{center}
\fbox{\parbox{0.95\textwidth}{
\textbf{本节要点}
\begin{itemize}
\item \textbf{形式证明}是证明论证有效性的有力方法,尤其适用于复杂论证
\item 形式证明依赖于一系列已知有效的基本论证形式,将结论从前提演绎出来
\item 形式证明的结构:
  \begin{itemize}
  \item 陈述序列中每个陈述要么是原论证的前提,要么是根据推理规则得出
  \item 最后一个陈述是原论证的结论
  \end{itemize}
\item 九种常用推论规则:
  \begin{itemize}
  \item \textbf{肯定前件式}(M.P.)、\textbf{否定后件式}(M.T.)
  \item \textbf{假言三段论}(H.S.)、\textbf{析取三段论}(D.S.)
  \item \textbf{构造式二难}(C.D.)、\textbf{吸收律}(Abs.)
  \item \textbf{简化律}(Simp.)、\textbf{合取律}(Conj.)、\textbf{附加律}(Add.)
  \end{itemize}
\end{itemize}
}}
\end{center}