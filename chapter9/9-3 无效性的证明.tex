\section{无效性的证明}

\begin{logicbox}[title=引言]
本节讨论如何证明论证的无效性,介绍一种比真值表更简洁高效的方法。通过为论证中的简单陈述指派适当的真值,使前提为真而结论为假,我们可以直接证明一个论证形式的无效性,而无需构造完整的真值表。
\end{logicbox}

\subsection{形式证明方法的局限性}

对一个无效论证来说,当然没有其有效性的形式证明。但如果我们找不到一个给定论证的有效性的形式证明,这种失败并不就能证明该论证无效,也并不证明不可能构造出形式证明。

\begin{theorembox}[title=形式证明方法的不对称性]
形式证明方法存在以下根本性的不对称性:

\textbf{1. 正向完备性}:如果一个论证是有效的,那么存在该论证的有效性形式证明。这保证了所有有效论证原则上都是可证明的。

\textbf{2. 负向不完备性}:如果我们不能构造某个论证的有效性形式证明,这并不能证明该论证无效。失败可能有两种原因:
\begin{itemize}
\item 论证确实无效(客观原因)
\item 我们缺乏足够的聪明才智或坚持不懈的精神(主观原因)
\end{itemize}

\textbf{3. 非能行性后果}:这种不对称性是证明构造过程的非能行性的直接后果。我们没有机械的程序来构造形式证明,因此失败并不意味着不可能性。

\textbf{4. 方法论需求}:这表明我们需要一种积极的方法来证明论证的无效性,而不能仅仅依赖于构造证明的失败。
\end{theorembox}

那么,怎样构成一个给定论证\logicterm{无效性}的一个证明呢?

\subsection{无效性证明的理论基础}

如下描述的方法与真值表方法密切相关,尽管这个方法比后者简略得多。

\begin{theorembox}[title=真值指派方法的逻辑原理]
真值指派方法建立在以下逻辑原理之上:

\textbf{1. 反例原理}:要证明一个论证无效,只需要找到一个反例——即一种情况,在这种情况下前提都为真而结论为假。

\textbf{2. 存在性证明}:如果能在真值表中发现这样一个情形(或一行),即对一个论证形式中的陈述变元进行这样一种\logicterm{真值指派},使得其前提为真而结论为假,那么该论证形式就是无效的。

\textbf{3. 充分性原理}:如果我们能对一个论证的简单分支陈述进行这样的真值指派,即使得它的前提为真且结论为假,那么,这种指派就足以证明该论证无效。

\textbf{4. 效率原理}:实际上,进行这种指派正是真值表所做的。但如果我们不实际构造完整的真值表就能进行这种真值指派,那么可以省去很多工作。
\end{theorembox}

\begin{examplebox}[title=真值指派方法的优势]
相比于完整的真值表方法,真值指派方法具有以下优势:

\textbf{1. 效率优势}:对于包含$n$个简单陈述的论证,真值表需要$2^n$行,而真值指派方法只需要找到一行反例。

\textbf{2. 目标导向}:我们可以有策略地寻找使前提为真、结论为假的指派,而不是盲目地检查所有可能性。

\textbf{3. 认知友好}:这种方法更符合人类的推理习惯,我们通常通过寻找反例来质疑论证。

\textbf{4. 可扩展性}:对于非常复杂的论证,这种方法仍然可行,而完整真值表可能变得不可处理。
\end{examplebox}

考查这样一个论证:

如果地方官赞同政府为低收入者修建住房,那么他会赞成限制私有企业的规模。

如果地方官是一个社会主义者,那么他会赞成限制私有企业的规模。

因此,如果地方官赞同政府为低收入者修建住房,那么他是一个社会主义者。

它可以符号化为:

$$
\begin{aligned}
& F \supset R \\
& S \supset R \\
& \therefore F \supset S
\end{aligned}
$$

我们不必构造完整的真值表就可以证明它无效。首先可以提问:"使结论为假要求何种真值指派?"显然,一个条件陈述为假,仅当它的前件为真而后件为假。因此,给 $F$ 指派真值"真",且给 $S$ 指派真值"假",会使得结论 $F \supset S$ 为假。现在,如果把真值"真"指派给 $R$ ,那么两个前提都是真的,因为只要它的后件为真,该条件陈述就为真。于是我们可以说,如果把真值"真"指派给 $F$ 和 $R$ ,把真值"假"指派给 $S$ ,该论证就有真前提和假结论,据此,它就被证明为无效。

\subsection{与真值表方法的关系}

这种证明无效性的方法是真值表证明方法的一个变种,因而应注意这二者之间的本质联系。实际上,在我们进行如上所示的那种真值指派时,我们所做的就是构造给定论证的真值表中的一行。若我们把这种真值指派水平地写成下述形式:

\begin{center}
\begin{tabular}{|cccccc|}
\hline
$F$ & $R$ & $S$ & $F \supset R$ & $S \supset R$ & $F \supset S$ \\
\hline
真 & 真 & 假 & 真 & 真 & 假 \\
\hline
\end{tabular}
\end{center}

这种关系会看得更清楚。其中,上述真值指派构成了论证真值表中的一行 (第二行)。通过显示其真值表中至少有这样一行,即其前提都为真而结论为假,一个论证就被证明为无效。因此,要发现一个论证的无效性,我们不必检验它的真值表的每一行:只要发现有一行,它的前提都为真而结论为假,这就足够了。当前证明无效性的方法,就是一种构造这样一行而不必构造整个真值表的方法。\cite{jeffrey1967}

\subsection{真值指派的系统策略}

目前这种方法比写出整个真值表要简略,一个论证所涉及的简单分支陈述越多,这种方法相应地节省的时间和空间也越多。对一个前提相当多,或相当复杂的论证来说,进行所需的真值指派可能不太容易。虽然没有机械的处理办法,但某些系统性的策略是有帮助的。

\begin{theorembox}[title=真值指派的基本策略]
\textbf{1. 基本陈述优先原则}:如果要证明无效性,给那些立即就能看出是基本的陈述指派真值是最有效的做法。

\textbf{2. 前提约束原则}:任何仅断言某陈述 $S$ 为真的前提,立刻提示我们对 $S$指派 $T$(或 $F$ ,如果作为前提的 $S$ 已被断言为假),因为我们知道所有前提必须被处理为真。

\textbf{3. 结论否定原则}:同一原则适用于结论中的陈述,只是那里的真值指派必须使结论为假。例如:
\begin{itemize}
\item 形如 $A \supset B$ 的结论:对 $A$ 指派 $T$ ,对 $B$ 指派 $F$
\item 形如 $A \vee B$ 的结论:对 $A$ 指派 $F$ ,对 $B$ 指派 $F$
\item 形如 $A \cdot B$ 的结论:对 $A$ 指派 $F$ 或对 $B$ 指派 $F$
\end{itemize}

\textbf{4. 策略选择原则}:我们应该从寻求使前提为真出发,还是从寻求使结论为假出发,取决于这些命题的结构。一般来说,我们最好从最有把握的指派开始。
\end{theorembox}

\begin{examplebox}[title=真值指派的高级策略]
\textbf{1. 约束传播}:从一个确定的指派开始,通过逻辑约束传播到其他陈述。例如,如果$A \supset B$必须为真且$A$为真,那么$B$也必须为真。

\textbf{2. 冲突检测}:在指派过程中检测是否出现冲突。如果某个陈述被要求同时为真和假,则当前指派路径失败,需要回溯。

\textbf{3. 试探性指派}:当没有明显的强制指派时,可以进行试探性指派,然后检查是否能够完成一致的指派。

\textbf{4. 结构分析}:分析论证的逻辑结构,识别关键的"瓶颈"陈述,优先处理这些陈述的真值指派。
\end{examplebox}

当然,会有许多这样的情形,其第一次指派不得不是任意的和试探性的。一定数量的试错是必要的。但即使这样,这种证明无效性的方法,也几乎总比写出完整的真值表简略和容易。

\subsection{方法的局限性与适用范围}

\begin{theorembox}[title=真值指派方法的局限性]
虽然真值指派方法通常很有效,但它也有一些局限性:

\textbf{1. 非机械性}:与真值表方法不同,真值指派方法不是完全机械的,需要一定的策略和技巧。

\textbf{2. 试错需求}:对于复杂的论证,可能需要多次尝试不同的指派组合才能找到反例。

\textbf{3. 不完全性}:如果论证实际上是有效的,这种方法会失败,但失败本身不能证明有效性。

\textbf{4. 复杂度依赖}:对于某些特殊结构的论证,寻找反例可能仍然很困难。
\end{theorembox}

\begin{center}
\fbox{\parbox{0.95\textwidth}{
\textbf{本节要点}
\begin{itemize}
\item \textbf{形式证明方法的不对称性}:
  \begin{itemize}
  \item \textbf{正向完备性}:有效论证总是可以被证明的
  \item \textbf{负向不完备性}:不能构造证明不等于证明了无效性
  \item \textbf{非能行性后果}:构造证明的失败可能是主观或客观原因
  \item \textbf{方法论需求}:需要积极的方法来证明无效性
  \end{itemize}
\item \textbf{真值指派方法的逻辑原理}:
  \begin{itemize}
  \item \textbf{反例原理}:只需找到一个前提真结论假的情况
  \item \textbf{存在性证明}:在真值表中找到一行反例即可
  \item \textbf{充分性原理}:一个反例足以证明无效性
  \item \textbf{效率原理}:避免构造完整真值表的工作量
  \end{itemize}
\item \textbf{真值指派方法的优势}:
  \begin{itemize}
  \item \textbf{效率优势}:从$2^n$行减少到寻找1行反例
  \item \textbf{目标导向}:有策略地寻找反例而非盲目检查
  \item \textbf{认知友好}:符合人类通过反例质疑论证的习惯
  \item \textbf{可扩展性}:对复杂论证仍然可行
  \end{itemize}
\item \textbf{真值指派的系统策略}:
  \begin{itemize}
  \item \textbf{基本策略}:基本陈述优先、前提约束、结论否定、策略选择原则
  \item \textbf{高级策略}:约束传播、冲突检测、试探性指派、结构分析
  \item 从最有把握的指派开始,必要时进行试错
  \end{itemize}
\item \textbf{与真值表方法的关系}:
  \begin{itemize}
  \item 真值指派构造真值表中的一行而非整个表
  \item 只需发现一行前提真结论假的情况即可证明无效
  \item 对于含多个简单陈述的论证,效率提升尤为明显
  \item 本质上是真值表方法的优化版本
  \end{itemize}
\item \textbf{方法的局限性}:
  \begin{itemize}
  \item \textbf{非机械性}:需要策略和技巧,不是完全机械的过程
  \item \textbf{试错需求}:复杂论证可能需要多次尝试
  \item \textbf{不完全性}:失败不能证明有效性
  \item \textbf{复杂度依赖}:某些结构的论证仍然困难
  \end{itemize}
\end{itemize}
}}
\end{center}