\section{无效性的证明}

\begin{quotation}
本节讨论如何证明论证的无效性,介绍一种比真值表更简洁高效的方法。通过为论证中的简单陈述指派适当的真值,使前提为真而结论为假,我们可以直接证明一个论证形式的无效性,而无需构造完整的真值表。
\end{quotation}

对一个无效论证来说,当然没有其有效性的形式证明。但如果我们找不到一个给定论证的有效性的形式证明,这种失败并不就能证明该论证无效,也并不证明不可能构造出形式证明。这可能只意味我们的努力还不够。我们未能发现一个论证有效性的形式证明,可能是由该论证无效这一事实造成的,但也可能是由于我们缺乏聪明才智造成的,这是证明的构造过程的非能行性的后果。未能构造某个论证的有效性的形式证明并不能证明该论证无效。那么,怎样构成一个给定论证\textbf{无效性}的一个证明呢?

如下描述的方法与真值表方法密切相关,尽管这个方法比后者简略得多。回想我们如何用真值表证明一个无效的论证形式无效,有助于如下讨论。如果能在真值表中发现这样一个情形(或一行),即对一个论证形式

中的陈述变元进行这样一种\textbf{真值指派},使得其前提为真而结论为假,那么该论证形式就是无效的。如果我们能对一个论证的简单分支陈述进行这样的真值指派,即使得它的前提为真且结论为假,那么,这种指派就足以证明该论证无效。实际上,进行这种指派正是真值表所做的。但如果我们不实际构造完整的真值表就能进行这种真值指派,那么可以省去很多工作。考查这样一个论证:

如果地方官赞同政府为低收入者修建住房,那么他会赞成限制私有企业的规模。

如果地方官是一个社会主义者,那么他会赞成限制私有企业的规模。

因此,如果地方官赞同政府为低收入者修建住房,那么他是一个社会主义者。

它可以符号化为:

$$
\begin{aligned}
& F \supset R \\
& S \supset R \\
& \therefore F \supset S
\end{aligned}
$$

我们不必构造完整的真值表就可以证明它无效。首先可以提问:"使结论为假要求何种真值指派?"显然,一个条件陈述为假,仅当它的前件为真而后件为假。因此,给 $F$ 指派真值"真",且给 $S$ 指派真值"假",会使得结论 $F \supset S$ 为假。现在,如果把真值"真"指派给 $R$ ,那么两个前提都是真的,因为只要它的后件为真,该条件陈述就为真。于是我们可以说,如果把真值"真"指派给 $F$ 和 $R$ ,把真值"假"指派给 $S$ ,该论证就有真前提和假结论,据此,它就被证明为无效。

\subsection{与真值表方法的关系}

这种证明无效性的方法是真值表证明方法的一个变种,因而应注意这二者之间的本质联系。实际上,在我们进行如上所示的那种真值指派时,我们所做的就是构造给定论证的真值表中的一行。若我们把这种真值指派水平地写成下述形式:

\begin{center}
\begin{tabular}{|cccccc|}
\hline
$F$ & $R$ & $S$ & $F \supset R$ & $S \supset R$ & $F \supset S$ \\
\hline
真 & 真 & 假 & 真 & 真 & 假 \\
\hline
\end{tabular}
\end{center}

这种关系会看得更清楚。其中,上述真值指派构成了论证真值表中的一行 (第二行)。通过显示其真值表中至少有这样一行,即其前提都为真而结论为假,一个论证就被证明为无效。因此,要发现一个论证的无效性,我们不必检验它的真值表的每一行:只要发现有一行,它的前提都为真而结论为假,这就足够了。当前证明无效性的方法,就是一种构造这样一行而不必构造整个真值表的方法。\cite{jeffrey1967}

\subsection{真值指派的策略}

目前这种方法比写出整个真值表要简略,一个论证所涉及的简单分支陈述越多,这种方法相应地节省的时间和空间也越多。对一个前提相当多,或相当复杂的论证来说,进行所需的真值指派可能不太容易。虽然没有机械的处理办法,但亦可证明某些提示是有帮助的。

如果要证明无效性,给那些立即就能看出是基本的陈述指派真值是最有效的做法。例如,任何仅断言某陈述 $S$ 为真的前提,立刻提示我们对 $S$指派 $T$(或 $F$ ,如果作为前提的 $S$ 已被断言为假),因为我们知道所有前提必须被处理为真。同一原则适用于结论中的陈述,只是那里的真值指派必须使结论为假。因此,一个形如 $A \supset B$ 的结论,会立时提示对 $A$ 指派 $T$ ,对 $B$ 指派 $F$ ;一个形如 $A \vee B$ 的结论,会立时提示对 $A$ 指派 $F$ ,对 $B$也指派 $F$ ,因为只有这种指派才能产生无效性的证明。

我们应该从寻求使前提为真出发,还是从寻求使结论为假出发,取决于这些命题的结构。一般来说,我们最好从最有把握的指派开始。当然,会有许多这样的情形,其第一次指派不得不是任意的和试探性的。一定数量的试错是必要的。但即使这样,这种证明无效性的方法,也几乎总比写出完整的真值表简略和容易。 

\begin{center}
\fbox{\parbox{0.95\textwidth}{
\textbf{本节要点}
\begin{itemize}
\item 证明论证无效性的方法:
  \begin{itemize}
  \item 找到一种真值指派,使所有前提为真而结论为假
  \item 这种方法可视为构造真值表中的一行,而非整个表
  \item 对于复杂论证尤其高效,可大幅节省时间和工作量
  \end{itemize}
\item 真值指派策略:
  \begin{itemize}
  \item 从结论开始:确定使结论为假的真值指派方式
  \item 对前提进行真值指派,确保所有前提为真
  \item 基本陈述优先指派,然后处理复合陈述
  \end{itemize}
\item 与真值表的关系:
  \begin{itemize}
  \item 仅需发现真值表中一行满足条件,即可证明无效
  \item 对于含多个简单陈述的论证,简化工作尤为明显
  \item 可能需要试错法尝试不同真值指派组合
  \end{itemize}
\end{itemize}
}}
\end{center} 