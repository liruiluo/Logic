\section*{13.6 判决性实验和特设性假说}
\section*{A.判决性实验}
科学中的进步很少是直接的和容易的。认为通过对某个问题简单地使用几步假说一演绎法就能够达到答案,这种看法是愚奪的。答案一一正确的说明性假说一一往往是模糊的,需要非常精心制作的理论武器。建立最后的正确假说会极其困难。这个过程完全不是机械的,除了需要艰辛的观察和实验外,还需要深刻的洞察力和很大的创造性。

新的假说得以形成之后,如果它与某个先前已经接受的理论相矛盾,很难确定哪个正确。在某些场合下,两个竞争性假说用被称为一个"判决性实验"的东西进行检验。判决性实验是指这样一个实验,它被精心构造出来以表明所提出的说明中的一个而非另一个实际上是正确的。这样的判决性实验一旦被建立起来,可能是激动人心的和极其有成果的。

例如,美国物理学家阿尔伯特•迈克尔逊和化学家爱德华•莫雷在 1887年精心构造了一个测量光速的实验。通过这个实验使一个被广泛接受的理论(他们原来相信它是正确的)置于一个判决性实验之中。人们长期相信空间中充满着一个被称做"以太"(ether)的假设物质,(人们假定)该物质使光波运行,如同空气使声波运行一样。或者以太存在,或者它不存在。如果它存在,那么测量出的沿着地球运动方向的光速,应当与地球运动成一个直角方向上的光速不同。该实验产生了一个"否定的"结果。因该实验是一个对当时被广泛接受的理论的判决性检验,它成为物理

学史上最著名的实验之一。没有发现这两个不同方向上运动的光速存在差别。这个结果有力地破除了人们长期相信的以太概念。 ${ }^{[23]}$

但是遗憾的是,威力如此强大的判决实验不总是可行的。不同的可观察结果可能不会从不同假说中推演出来;或者,它们能够被推演出来,但是我们没有能力创造条件,以检验哪一个假说的结果将出现。

物理学在 21 世纪初面临的一个主要问题也正属此类。在两个最强有力的理论之间,存在一个明显的目前不能解决的冲突。广义相对论已经得到很好的证实,其定律(描述引力以及引力如何形成空间和时间)的一个必然推论是:某些塌陷的大质量的恒星将形成"黑洞",从该黑洞中逃脱是不可能的,因为它要求比光要快的速度。量子力学定律同样得到很好证实,但它们明确推论得,信息不能永久消失,即使掉到黑洞里也是如此。要么存在某个目前还不知道的时空性质,它能够用来对该信息的保持进行说明,要么在物理学中存在错误,指出它可以对该信息的永久消失给出解释。最终两个理论中的一个必定得到修正,但我们现在仍然不知道哪个要修正,我们也无法构造所需要的判决性实验。 ${ }^{[24]}$

判决性实验是科学探究一个重要方面,然而与构造判决性实验相关的另外一个困难是,人们提出某个说明性假说,我们希望通过进行某个判决性实验来检验它的推论,但它的推论不可能仅从该假说自身演绎出来。我们是使用该假说与其他理论一起而推论得到要检验的结论的。为此,我们假定那些其他理论完全可靠。它们确实可能是完全可靠的,当然它们也可能不完全可靠。如果它们不可靠,即使判决性实验似乎否证了待考察的假说,也有可能待考察的假说恰恰是正确的。科学中的进展依赖于假说集合,其中的任何一个都可能是有缺陷的。

当涉及相当高抽象程度的假说时,仅仅单个假说不可能直接演绎出可直接检验的预测。用做演绎前提的必定是一个统一的假说群体,如果观察到的事实不是预测的事实,那么我们可以得出结论,该假说群中至少一个是错的。但是,这个结论不能表明哪一个假说是错的。例如,在前面的对 DNA 结构发现的解释中,华生和克瑞克在检验核酸丝的形式是双螺旋、它的基指向内部的假说时,他们发现这样的安排不能与所有已知的事实和已接受的理论相一致。"已知的事实和已接受的理论"——水含量、双螺旋斜度、基(腺嘌呤、鸟嘌呤、氧氨嘧啶和胸腺嘧啶)的连接方式——在假说的检验中被假定是正确的。如果所有这些假定的确正确,长丝的结构

不可能是双螺旋。然而,在实际中,华生和克瑞克对他们的假说有足够的自信,他们开始怀疑描述基(A、G、C 和 T )相互结合的理论不完全正确。该理论被他们放弃,他们提出一个不同的理论,即假定结合物是氢的理论,此时,双螺旋的新假说(以及与之相连的理论)得到证实。

因此,在揭示一群假说有缺陷的过程中,一个实验能够是"判决的"。这样一群假说通常包含许多独立的假说。其中任何一个假说,无论实验结果对它多么不利,我们可以拒绝该群体中的某个其他假说,而坚持它的真理性。这就使得某些人得出结论说,从来不会有单个假说遭受判决性的实验。

\section*{B.特设性假说}
针对上面的批评,有人认为,一个实验在否证单个新假说中的确能够是判决性的,因为通过拒绝假说群体中某个其他假说(上面已经表明这是可能的)而"拯救"该假说的努力是完全特设的(Ad Hoc)。Ad Hoc 为拉丁术语,字面意义是"为此[特定目的]"。Ad Hoc 包含这样一个意思,所有的假说都是特设的。因为,一个假说之发明如果不是为了解释某个先前得到确立的事实或者其他事实,那是没有意义的。但是当我们以这样的意义滥用它的时候,特设性意味着,对假说集合进行调整仅仅是为了拯救被检验的假说这样一个目的,它没有其他的说明力或者可检验的结果。

没有科学假说是这第二个意义上特设的。如果"鬼是机器故障的原因"被用来解释一个复杂的机器发生故障,那么它明显不是科学的解释;我们嘲笑这样的假说,它是否定意义上特设性的。但是,在任何实际科学研究中,当一个新的假说之提出以调整一个旧的理论的时候,该调整是否是该否定意义上特设性的,这需要进一步确定。

科学史中的另外一个例子可以帮助我们弄清这个问题。在19世纪,天体力学理论被人们很好地理解。对于天文学家来说,天王星和水星这两个行星的轨道与当时所接受的理论对它们所预测的轨道不一致。行星运动的理论在当时应当被修改,但是事实上它被保留了下来。为了解决该理论的协调问题,有人提出,存在某个未发现的行星,其引力造成观察到的反常现象。引起天王星轨道偏差的新行星的轨道,由勒维烈在 1845 年预测出来,预测结果不久被海王星的发现而证实,其位置精确地解释了那些偏 513 差。 ${ }^{[25]}$ 这个假说——存在这样一颗行星——当然不是否定意义上特设性的假说。原因是,从该假说中能够演绎出许多结论,该假说是独立可检

验的。\\
但是在水星的案例中,存在另外一颗行星[该行星过早地被命名为 "火神星"(Vulcan)]干扰水星轨道的假说,不能得到证实。如果一个理论假想有"水星力",用它们来解释水星轨道的异常,而这些力不能解释其他任何事情,并且绝不能被找到,那么这样的一个理论发明自然是特设性的。实际情况是,该疑难长期得不到解决;直到1915年广义相对论提出后,观察到的水星轨道的不规则,才完全与不同的但完美的天文学理论相吻合。水星轨道异常,能够使用广义相对论来预测,这个事实构成该理论最引人注目的证实之一。爱因斯坦称它为"我生命中最辉煌的工作"${ }^{[26]}$ 。只有在那个时候,我们才给出了关于该现象的合适的(即真正理论化的)说明。

天文学史中的这个疑难给出了人们在使用特设性这个术语的第三个意义,它也是否定性的:表示一个单纯的描述性概括。一个描述,它是第三个意义上特设性的,它仅断定一个特定种类的所有事实只在某些特定种类的条件下发生;但是该假说和前面的那些特设假说一样没有任何解释力或解释范围。这样的假说的一个古典例子是,"菲兹吉拉德收缩效应"被提出来对迈克尔逊-莫雷在光速实验结果的解释。菲兹吉拉德断定,物体以极其高的速度运动会发生收缩,他确实对给定现象给出解释,并且他的假说能够为重复进行的同样实验所检验。但是他的"收缩效应"不能解释其他的任何东西。在当时它被普遍认为是特设性的而不是说明性的。(正如与水星行为中出现明显差别的情况一样)直到相对论的提出(爱因斯坦的狭义相对论),人们才得到迈克尔逊一莫雷实验结果的一个合适的理论说明。

我们可这样总结,实验对单个假说绝不能是判决性的,这不仅因为假说经常是在否定意义上特设性的;进一步地说,在本节前面已经表明,因为假说只是在群体中才是可检验的,实验绝不能成为判决单个假说的东西。 ${ }^{[27]}$ 这个限度阐明了科学的系统化特点。科学进步就是建立永远更加恰当(ever-more-adequate)的理论,以便解释不断增加的观察结果和实验事实。某些分离事实能够具有较大的价值,因为科学的最终基础是事实。但是科学结构主要不是通过点滴累积而得以发展的,其发展是在一个已得到普遍认同的理论体系的框架内整体地进行的。认为科学假说或者定律是分散的和独立的观点是朴素的,也是过时的。

在这样一个理论框架中工作,此时我们不对该理论框架进行质疑;进行一个"判决性实验"以证实或否证某个假说的观点仍然能够有意义。如果得到一个否定性结果,即,根据某个有疑问的假说与已经接受的科学理论一道进行预测的某个现象,它没有发生,那么该实验是判决性的,这个有疑问的假说可以被拒绝。但是正如我们已经看到的,在这个过程中不存在任何绝对的事情,因为,那些即使被人们普遍接纳的科学理论面临新的和矛盾的证据时,也要发生调整。科学无论在实践中还是在目标上,都不是一成不变的。

从前面的讨论中得到的启示是,将"隐藏的假设"揭示出来,以便能够对那些默认的假定进行重新审视,这在科学进步中是重要的。当一个关键假设是潜藏的时候,没有明显的必要,因而没有好的机会对之进行考察并确定它到底是真还是假;通过将以前潜藏的假设揭示出来,对之进行分析并(也许)否定它,科学往往获得进步。

例如,在日常生活中谈论两个事件"在同一个时刻"发生,这似乎完全没有问题。我们普遍假定事件常常同时发生。但是科学中一个重要的和巨大的进步开始于爱肉斯坦将这个假定揭露出来。他问,一个观察者如何能够确定两个距离遥远的事件是否真的在同一时刻发生。最终他得出这个结论:两个事件对于某些观察者来说能够是同时的,但对其他的观察者来说则不是;这依赖于观察者相对于待研究事件的相对位置和速度。正是对同时性假定的拒绝,使爱因斯坦发明了狭义相对论,从而为解释迈克尔逊——莫雷实验所揭示的现象跨出了重大的一步。当然,一个假定在它被挑战之前必须被人们所认识,因而,在科学中具有重大意义的是,将理论中起作用的所有有关的假定揭示出来,而不让任何一个隐藏起来。

通过描述和讨论科学史中最辉煌的篇章之一一伽利略对太阳系的哥白尼理论进行的观察证实,是对科学方法的进行总结并阐释科学整体的进步的意义的极好的途径。 