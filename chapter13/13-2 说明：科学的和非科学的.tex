\section{说明:科学的和非科学的——认识论的根本分野}

\begin{logicbox}[title=引言]
本节深入探讨\logicterm{科学说明}与\logicterm{非科学说明}的本质区别,这一区别构成了现代认识论的根本分野。我们将从逻辑学、认识论、科学哲学等多个角度分析什么构成一个\logicemph{有效的}说明,深入辨别\logicterm{科学态度}与\logicterm{非科学态度}之间的关键差异,并系统理解\logicterm{科学说明}的\logicterm{可检验性}、\logicterm{可证伪性}等核心特征。通过理解\logicterm{科学说明}必须是可证实的、探索性的、开放的,而非教条主义的、封闭的,我们将能够准确区分\logicemph{真正的}科学解释与仅凭权威、传统或习俗的\logicwarn{非科学说法},从而建立科学理性的认识基础。
\end{logicbox}

\subsection{说明的逻辑结构:从推理到解释的认识转换}

\begin{theorembox}[title=说明的逻辑本质]
当要对某个事情进行说明(explanation)时,我们需要什么?

\textbf{说明的定义}:
一个被寻求的解释(account)就是对世界的某个陈述集合,或某个叙说(story),从该解释中能够逻辑地推导出需要解释的事情。该解释能够对需要解释的有疑问的问题进行消解或者简约。

\textbf{说明与推理的逻辑关系}:
说明和推论可以被看成是同一个过程,只是方向相反:
\begin{itemize}
\item \textbf{演绎推理}:从前提到结论($P \rightarrow Q$)
\item \textbf{科学说明}:从结论到前提(已知$Q$,寻找$P$使得$P \rightarrow Q$)
\end{itemize}

\textbf{逻辑形式的双重功能}:
在本书第1章(1.6节)中我们阐述了,当我们要推得$Q$时,"由$P$得$Q$"如何表达一个论证;如果我们所进行的是从一个已经建立的$Q$到能够对之说明的前提的推理,它也可以表达一个说明。

\textbf{说明的认识论意义}:
说明不仅是逻辑操作,更是人类理解世界的基本认识活动。通过说明,我们将未知归约为已知,将复杂归约为简单,将特殊归约为普遍。
\end{theorembox}

\begin{examplebox}[title=说明逻辑结构的具体实例]
\textbf{物理学中的说明}:
\begin{itemize}
\item \textbf{待解释现象}:苹果从树上掉下来
\item \textbf{说明前提}:万有引力定律、地球质量、苹果质量
\item \textbf{逻辑推导}:从引力定律推导出苹果必然下落
\item \textbf{说明效果}:特殊现象被普遍定律所解释
\end{itemize}

\textbf{生物学中的说明}:
\begin{itemize}
\item \textbf{待解释现象}:长颈鹿的长脖子
\item \textbf{说明前提}:自然选择理论、环境压力、遗传变异
\item \textbf{逻辑推导}:从进化理论推导出长脖子的适应优势
\item \textbf{说明效果}:生物特征被进化机制所解释
\end{itemize}

\textbf{心理学中的说明}:
\begin{itemize}
\item \textbf{待解释现象}:学习效果的差异
\item \textbf{说明前提}:认知负荷理论、工作记忆容量、注意资源分配
\item \textbf{逻辑推导}:从认知理论推导出学习效果的差异
\item \textbf{说明效果}:心理现象被认知机制所解释
\end{itemize}
\end{examplebox}

\subsection{有效说明的标准:相关性、真实性与普遍性}

\begin{theorembox}[title=有效说明的基本要求]
自然,每一个好的说明必须满足几个基本标准。

\textbf{1. 相关性(Relevance)}:
如果我解释说,我上班迟到是因为在中部非洲发生持续的政治混乱,那么这会被认为什么都没有说明;它是不相关的——因为需要说明的我迟到的事实,不能从中被推论出来。

\textbf{2. 真实性(Truth)}:
当然每个真正的说明不仅是相关的而且是真实的。虚假的前提无法提供有效的说明。

\textbf{3. 普遍性(Universality)}:
无论我迟到的正确的说明是什么,之所以需要这个说明,是因为在我迟到的这个事件上存在疑问。然而,科学的说明除了相关和真实外,必须超越特定事件,而能够对给定种类的所有事件提供解释。

\textbf{科学说明的普遍性特征}:
科学说明的力量在于其普遍适用性。它不仅解释特定现象,更重要的是能够解释同类的所有现象。
\end{theorembox}

\begin{examplebox}[title=牛顿万有引力定律的说明力量]
牛顿力学的伟大在于万有引力定律。牛顿写道:

"宇宙中每个质点以一个力吸引另外一个质点。该力正比于质点质量的乘积,反比于它们间距离的平方。"

\textbf{万有引力定律的说明范围}:
\begin{itemize}
\item \textbf{地面现象}:物体下落、抛物运动
\item \textbf{天体运动}:行星轨道、月球运动、彗星轨道
\item \textbf{潮汐现象}:海洋潮汐的周期性变化
\item \textbf{双星系统}:恒星的相互绕转
\end{itemize}

\textbf{普遍性的认识价值}:
一个定律能够统一解释如此广泛的现象,这体现了科学理论的巨大解释力和预测力。这种普遍性使得科学说明超越了特定事件的描述,达到了对自然规律的深层理解。

\textbf{数学表达的精确性}:
$$F = G\frac{m_1 m_2}{r^2}$$
这个简洁的数学公式包含了对宇宙中所有引力现象的完整描述,体现了科学说明追求精确性和普遍性的特征。
\end{examplebox}

\subsection{科学说明与非科学说明的根本区别}

\begin{theorembox}[title=非科学说明的特征分析]
非科学的说明也可以是相关的和普遍的,但它们缺乏科学说明的核心特征。

\textbf{非科学说明的典型例子}:
\begin{itemize}
\item \textbf{机械故障的神秘解释}:用神秘的小鬼动了手脚,来解释引擎不能启动
\item \textbf{疾病的超自然解释}:疾病可以解释成邪恶的精灵侵入人体所引起的
\item \textbf{天体运动的拟人化解释}:在长达数个世纪的时间里,人们一直用在行星上生活并控制它们运动的"智慧生物"来解释行星的规则运动
\end{itemize}

\textbf{非科学说明的共同特征}:
\begin{itemize}
\item \textbf{不可检验性}:无法通过经验观察或实验来验证或证伪
\item \textbf{特设性}:为了解释特定现象而临时构造,缺乏独立的证据支持
\item \textbf{封闭性}:不接受批评和修正,拒绝与其他理论对话
\item \textbf{模糊性}:概念定义不清,缺乏精确的预测能力
\end{itemize}

\textbf{历史上的非科学说明}:
这些非科学说明在历史上曾经广泛存在,反映了人类在科学方法建立之前对世界的理解方式。它们虽然满足了人类解释世界的心理需求,但无法提供可靠的知识基础。
\end{theorembox}

但是我们对真正科学的说明感兴趣。科学的说明与非科学的说明在两个相互关联的方面相区别:

\begin{theorembox}[title=科学态度与非科学态度的根本区别]
\textbf{第一个区别:认识态度的差异}

\textbf{非科学态度的特征}:
接受非科学说明的人是教条的,解释被认为是绝对真的,是不能改进的。亚里士多德的观点在几个世纪里被非科学地接受成对事实的最终权威。尽管亚里士多德本人是谦虚的,但是一些中世纪的学者却以僵化的、非科学的态度对待他的观点。\cite{crombie1960}

\textbf{科学态度的特征}:
相反,真正科学的态度则与之十分不同。每个提出的说明都是探索性的或暂时的。科学说明被认为是假说——它们在现有证据下具有不同的可靠程度。

\textbf{态度差异的深层含义}:
\begin{itemize}
\item \textbf{开放性vs封闭性}:科学态度对新证据开放,非科学态度拒绝质疑
\item \textbf{可错性vs绝对性}:科学承认理论的可错性,非科学声称绝对真理
\item \textbf{进步性vs静态性}:科学追求不断改进,非科学维持现状
\item \textbf{批判性vs权威性}:科学鼓励批判思维,非科学依赖权威
\end{itemize}
\end{theorembox}

\begin{theorembox}[title=证据基础:科学与非科学的根本分野]
\textbf{第二个区别:证据基础的差异}

科学说明和非科学说明之间的第二个同时也是最根本的区别是,接受或拒绝某个观点所基于的基础。

\textbf{非科学说明的证据基础}:
一个非科学的说明被简单地认为是真的,因为"每个人知道"它如此。一个非科学信念之被坚持,不依赖于有利于它的证据之上。

\textbf{科学说明的证据基础}:
但是在科学中,一个假说仅仅在存在支持它的证据的条件下才值得接受,人们总是对它的真或假保持怀疑,寻找证据的过程永不停止。

\textbf{科学的经验性质}:
科学是经验的——真理的检验在于经验之中,因而科学说明的本质是,它是可检验的。

\textbf{证据标准的对比}:
\begin{itemize}
\item \textbf{科学标准}:经验证据、逻辑一致性、可重复性、可预测性
\item \textbf{非科学标准}:权威、传统、直觉、信仰、"常识"
\item \textbf{检验方式}:科学通过实验和观察检验,非科学拒绝检验
\item \textbf{修正机制}:科学有自我修正机制,非科学缺乏修正能力
\end{itemize}
\end{theorembox}

\subsection{可检验性:科学说明的核心特征}

\begin{theorembox}[title=直接检验与间接检验]
真理的检验可以是直接的也可以是间接的。

\textbf{直接检验}:
为了弄清外面是否下雨,我只要看一下外面。这种检验直接、简单、确定。

\textbf{间接检验的必要性}:
但是用做说明的假说是普遍性命题,它们不能是直接可检验的。科学理论通常涉及不可直接观察的实体或过程,因此需要间接检验。

\textbf{间接检验的逻辑结构}:
如果我对我上班迟到的解释是交通事故,我的老板如果对之怀疑,他能够借助于警察的事故报告而间接地检验我的解释。

一个间接的检验从待检验的命题(如我遭遇到一次交通事故)演绎出其他某个能够被直接检验的命题(如我提交了一个事故报告)。

\textbf{间接检验的逻辑后果}:
\begin{itemize}
\item 如果那个演绎出来的命题是错的,包含这个命题的说明必定是错的
\item 如果演绎出来的命题是真的,它提供了某个证据证明这个说明是真的、已经被间接证实
\item 但是这个证据不是结论性的
\end{itemize}
\end{theorembox}

\begin{theorembox}[title=间接检验的局限性与复杂性]
间接检验永远不会是确定的。它总是依赖于某些辅助的前提。

\textbf{辅助前提的依赖性}:
比如这样的前提:我对我的老板描述的该起事故与警察记载的一样。但是警察部门应当对我所涉案的事故的记录进行备案,但可能还没有备案;缺乏该记录并不证明我的说明是假的。

\textbf{证据的非结论性}:
并且,某个附加前提即使是真的,它并不给说明赋予确定性——尽管演绎出的结论(本例中事故报告的真实性)得到成功检验确实加固了它的前提。

\textbf{杜恒-奎因论题}:
这种复杂性反映了科学哲学中著名的杜恒-奎因论题:我们无法孤立地检验单个假说,而总是在检验一个假说网络。这意味着:
\begin{itemize}
\item 检验结果的解释具有多义性
\item 理论的修正可能涉及多个层面
\item 科学知识具有整体性特征
\item 确定性在科学中是相对的而非绝对的
\end{itemize}
\end{theorembox}

\begin{theorembox}[title=可证实性:科学说明的本质特征]
即使一个非科学的说明也有某个它喜爱的证据,即用它来解释的那个事实。

\textbf{非科学说明的证据局限}:
行星上居住着"智慧生物",他们使行星沿着我们观察到的轨道运动,这个非科学理论能够称这个事实——行星确实在它们的轨道上运动——为证据。

\textbf{科学与非科学说明的根本差别}:
但是,在该假说和关于行星运动的可靠的天文学说明之间存在巨大的差别:
\begin{itemize}
\item \textbf{非科学假说}:不能够从中演绎出其他的可直接检验的命题
\item \textbf{科学说明}:能够演绎出可直接检验的命题,而不是陈述待解释事实的命题
\end{itemize}

\textbf{经验可证实性的含义}:
这就是当我们说一个说明是经验可证实的时所要表达的意思。这样的可证实性是科学说明最本质的特征。\cite{popper1935}

\textbf{可证实性的深层意义}:
\begin{itemize}
\item \textbf{预测能力}:科学理论能够预测新的、未观察到的现象
\item \textbf{风险承担}:科学理论敢于做出可能被证伪的预测
\item \textbf{内容丰富}:可检验的预测越多,理论的经验内容越丰富
\item \textbf{进步机制}:通过检验和修正实现科学进步
\end{itemize}
\end{theorembox}

\begin{examplebox}[title=科学说明可证实性的历史实例]
\textbf{爱因斯坦相对论的预测与验证}:
\begin{itemize}
\item \textbf{理论预测}:光线在强引力场中会发生弯曲
\item \textbf{检验方法}:1919年日食观测
\item \textbf{验证结果}:观测结果与理论预测精确吻合
\item \textbf{科学意义}:确立了广义相对论的科学地位
\end{itemize}

\textbf{门捷列夫元素周期律的预测}:
\begin{itemize}
\item \textbf{理论预测}:预言了镓、钪、锗等未知元素的存在和性质
\item \textbf{检验方法}:化学实验和元素发现
\item \textbf{验证结果}:预言的元素相继被发现,性质与预测高度一致
\item \textbf{科学意义}:证明了周期律的科学价值
\end{itemize}

\textbf{达尔文进化论的预测}:
\begin{itemize}
\item \textbf{理论预测}:应该存在连接不同物种的过渡化石
\item \textbf{检验方法}:古生物学研究和化石发现
\item \textbf{验证结果}:大量过渡化石的发现支持了进化理论
\item \textbf{科学意义}:为进化论提供了强有力的证据支持
\end{itemize}
\end{examplebox}

\begin{center}
\fbox{\parbox{0.95\textwidth}{
\textbf{本节要点}
\begin{itemize}
\item \textbf{说明的逻辑结构:从推理到解释的认识转换}:
  \begin{itemize}
  \item \textbf{说明的本质}:从中能逻辑推导出待解释现象的陈述集合
  \item \textbf{逻辑关系}:说明与推论是相反方向的逻辑过程(演绎推理:$P \rightarrow Q$;科学说明:已知$Q$,寻找$P$)
  \item \textbf{认识意义}:通过说明将未知归约为已知,将复杂归约为简单,将特殊归约为普遍
  \item \textbf{跨学科应用}:物理学、生物学、心理学等各领域的说明实例
  \end{itemize}
\item \textbf{有效说明的标准:相关性、真实性与普遍性}:
  \begin{itemize}
  \item \textbf{相关性要求}:说明前提必须能够逻辑地推导出待解释现象
  \item \textbf{真实性要求}:说明的前提必须是真实的,虚假前提无法提供有效说明
  \item \textbf{普遍性特征}:科学说明超越特定事件,能对同类事件提供普遍解释
  \item \textbf{牛顿万有引力定律}:统一解释地面现象、天体运动、潮汐等广泛现象的典型例子
  \end{itemize}
\item \textbf{科学说明与非科学说明的根本区别}:
  \begin{itemize}
  \item \textbf{非科学说明特征}:不可检验性、特设性、封闭性、模糊性
  \item \textbf{历史实例}:神秘小鬼、邪恶精灵、行星智慧生物等非科学解释
  \item \textbf{认识态度差异}:科学态度的开放性vs非科学态度的教条性
  \item \textbf{证据基础差异}:科学依赖经验证据,非科学依赖权威传统
  \end{itemize}
\item \textbf{科学态度与非科学态度的深层对比}:
  \begin{itemize}
  \item \textbf{开放性vs封闭性}:科学对新证据开放,非科学拒绝质疑
  \item \textbf{可错性vs绝对性}:科学承认理论可错性,非科学声称绝对真理
  \item \textbf{进步性vs静态性}:科学追求不断改进,非科学维持现状
  \item \textbf{批判性vs权威性}:科学鼓励批判思维,非科学依赖权威
  \end{itemize}
\item \textbf{可检验性:科学说明的核心特征}:
  \begin{itemize}
  \item \textbf{直接vs间接检验}:科学理论通常需要间接检验,涉及辅助前提和假说网络
  \item \textbf{杜恒-奎因论题}:无法孤立检验单个假说,科学知识具有整体性特征
  \item \textbf{可证实性的深层意义}:预测能力、风险承担、内容丰富、进步机制
  \item \textbf{历史验证实例}:相对论光线弯曲预测、元素周期律预言、进化论化石预测
  \end{itemize}
\item \textbf{认识论的根本分野}:
  \begin{itemize}
  \item 科学说明与非科学说明的区别构成现代认识论的根本分野
  \item 可检验性是科学说明最本质的特征,决定了科学知识的可靠性
  \item 科学方法为人类认识世界提供了最可靠的途径
  \item 理解这一区别对建立科学理性的认识基础具有重要意义
  \end{itemize}
\end{itemize}
}}
\end{center}