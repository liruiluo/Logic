\section{科学研究的七个阶段}

\begin{quotation}
本节描述科学研究过程的七个关键阶段。我们将分析从确定问题到应用理论的完整科学研究流程,理解科学家如何通过系统化方法从初始观察发展到成熟理论。通过了解这一过程,我们将认识到科学不仅是一种知识体系,更是一种严谨的探索方法。
\end{quotation}

\subsection{确定问题}
科学研究开始于某个问题。一个问题可以表示成一个或一组没有可接受的说明的事实。例如,侦探面临一个案子,他的问题是如何将之侦破,即确定犯罪人并给予证明。在某些情况下,如在柯南•道尔关于伟大的夏洛克-福尔摩斯的故事中,问题产生于尚未发生犯罪的特定事件或环境之中。科学家的研究可能开始于十分明确的问题;然而,更为普遍的是,他们是渐渐地发现了不相容或奇怪之处,这些不相容或奇怪之处演化成一个特定问题。

如果不存在可思考的事情,甚至夏洛克•福尔摩斯或者爱因斯坦也不能从事深刻的思考。一个天才必须面对一个问题。正如约翰•杜威和许多其他现代哲学家正确认为的那样,反思性的思考——从犯罪侦查到物理学、数学的抽象思考这个范围广泛的活动——是解决问题的(problem- solving)活动。科学家开始工作之前,问题必须被确定,或者至少以模糊的形式被确定。

\subsection{2.形成一个假说}
对手边的问题的哪怕最初始的思考都要求某个初步理论。最初的尝试不可能产生最后的答案,但是需要某个理论以便能够知道,需要收集何种类的证据,到哪里寻找它们,以及如何寻找它们。侦探考察犯罪现场,询问嫌疑人,并寻找线索,但是赤裸裸的事实不是线索。只有当线索能够被安排进某个融贯的模式,哪怕是粗糙的和临时的模式之中时,它们才有意义。

科学家与此相同,他们用某个初始假说开始收集证据。这个假说是关于待寻求的说明的本质的。科学家必须依赖于某些以前的知识,科学不会

从绝对无知中开始。事实上,如果被说明的事实出现真正的问题,必定存在某些先验的信念。

对于任何一个严肃的问题,世界上存在太多的相关的事实、太多的数据、以至于人们不能将它们全部收集起来。一些事实将被注意并被观察,另外的事实则没有。最耐心和最全面的研究者必须选择:被发现的事实中哪些要研究,哪些要放弃。这需要某个假说来工作:为了这个假说或者根据这个假说而收集相关的证据。该假说不必是完善的理论一一但是在它那里至少显示出理论的轮廊。否则的话,研究者不能确定从整个事实全体中挑选出何种事实来。一个临时的初始假说不管是如何的不完善,任何严格的探究开始的时候它都是必需的。

\subsection{收集额外事实}
一般来说,最初令人迷惑不解的事实似乎太多了,以至于不能提出一个对它们非常满意的说明;如果情况不是这样的话,这些事实不可能表现出问题来。但是,特别的,对那些熟悉普遍种类(如天体现象、社会现象或历史现象)的事实或事件的科学家而言,原初的问题会激发出一个初始假说,该假说引导他们寻找额外的相关事实。这个额外证据可以起到引导作用.引导我们得出较完全和较接近的合适答案。收集证据的任务,既艰辛又耗时,经常是失望和沮丧。好的科学意味着艰巨的工作。这个费力的收集过程是许多科学工作的主要内容。

在实际的科学活动中,步骤 2 和步骤 3 自然不是完全分离的。它们紧密相连、相互依赖。在开始收集证据时,某个初始假说是必需的;使用该假说来收集证据的过程,也是调整和精练假说本身的过程,这又引导我们进一步地寻找……也许导致新的发现……它又使我们更加精练假说,等等。

\subsection{4.进行预测}
在任何成功的研究中迟早会达到这样一点,研究者(科学家、侦探,甚至普通人)将最终相信,解决原初问题所需要的所有事实都已经获得。一个难题,更可能的是一组难题,摆在他或她的面前,其任务是将它们组装成一个可以理解的整体。这样思考的终结产品——如果成功的话,是这样的假说:它将解释所有数据、产生问题的原有事实集,以及初始假说所涉及的额外事实。

不存在实现某个完善理论的机械方法。对真正说明性的假说的实际发

现或发明是一个创造性的过程,在这个过程中需要想象,也需要知识。某些研究者,如夏洛克•福尔摩斯和阿尔伯特•爱因斯坦,在对存在的现象进行说明的"逆向推理"的过程中展示了其才能。但是每一个成功的科学家必须完成智力整合这个挑战性的任务:对激发研究兴趣的成问题事实进行解释,以构造和形成最终假说。\cite{peirce1958}

\subsection{5.进行实验}
我们已经看到预测力是评价说明的标准之一。一个真正富于成果的假说将不仅说明激发假说形成的原初事实,而且解释许多其他的事实。好的假说超越初始的事实,它涉及新的和不同的事实——这些事实较早没有被怀疑。假说所引出的这些事实被证实,使得假说得以确证——当然不能给予确定性的证明。

被称做"大爆炸的"宇宙学理论可以看做对这样的预测进行阐述的例子。这个理论认为,如果目前的宇宙开始于一个大爆炸事件,最初的火球应是平稳的和均匀的,而没有任何结构。与此对照的是,目前的宇宙具有大量的结构,是多块状的;可见物质组成星系、星系群,等等。这样的结构对生命的起源和演化是基本的。但是该结构是何时产生并如何产生的?通过观察膨胀的宇宙中那些遥远的天体,天文学家能够"回顾过去"。通过这样的观察,他们最终必定能够找到目前结构的原初证据。如果如此早的证据不能通过最敏感的仪器来探测到,大爆炸理论将是不可辩护的。如果这样的结果被探测到,大爆炸理论得到确证,尽管不是被证明。

\subsection{对结果进行检验}
在生物学领域里我们可以提出这个假说:在哺乳动物中蛋白质之产生是为了对抗特定的酶,而这种酶是在一个特定基因引导下产生的。从该假说中我们可以推论出进一步的结论:缺少该基因的地方,该蛋白质将不出现或者蛋白质数量不足。

为了检验该生物学假说是否正确,我们构造某个特定基因的作用能够被测定的实验。经常的做法是,将去除特定基因的老鼠进行繁殖——被称为"基因剔除老鼠"。如果在这样的老鼠中被研究的酶以及与之有关的蛋白质发生缺失,我们的假说将得到证实。\cite{capecchi1994} 在医学中许多有价值的信息正是以这种方法获得的。这种实验是广泛的生物学研究中典型的实验。我们设计实验,以弄清我们认为是对的东西,在如此这般的条件得到满足的情况下,是否确实是真的。为此,我们必须构造这样或那样的特定的条件。\\
"一个实验",正如伟大的物理学家马克斯•普朗克说,"是科学给自然提出的一个问题,而测量是对自然回答的记录。"

对某些预测的结果,如同夏洛克-福尔摩斯的许多预测的结果,其检验可能是直接的。银行窃贼将打破拱顶而人?福尔摩斯和华生等待他们,并且他们确实来了。\cite{doyle1927} 医生将避开从假的通风口进来的毒蛇吗?福尔摩斯和华生从躲藏的地方观看,发现医生避开了。\cite{doyle1927b} 那些说明性的理论直接地被检验,并牢固地得到了证实。

当然,许多科学理论不能被简单的观察所检验。早期宇宙的结构不可能被直接观察到。但是,如果存在某个早期的结构,如大爆炸理论预测的那样,那么,由于目前的背景辐射根源于早期,在它之中将必定存在不规则、不均匀。在原则上测量背景中的微波辐射是可能的,因而我们能够以这种方式间接地确定在大爆炸之后十分短暂的时刻是否存在这样的不规则性。几年后,一个卫星被设计来探测这些不规则性——如果它们存在的话。该卫星(宇宙背景探测者 COBE)的观察对大爆炸理论是至关重要的。如果最终没有探测到长期寻找的宇宙中早期结构的证据,膨胀宇宙的大爆炸解释将遭受严重的质疑。然而,1992年春天,预测到的不规则性被 COBE 探测到并被测量出来。这些不规则性来自于最遥远的过去,一直存在到宇宙学家回顾它们的今天。这个成功检验尽管不能证明该理论是正确的,但的确给人印象深刻地确证了大爆炸理论。

\subsection{应用该理论}
通过科学,我们的目的是说明我们观察到的现象,但是我们另一个目的是控制这些现象,为我们所用。牛顿和爱因斯坦的抽象理论在太阳系的现代探索中发挥中心作用。举一个不同种类的例子,假定我们面临的问题是某个疾病,发明的说明性假说是某个特定细菌引起该疾病。假定通过给老鼠或啮齿动物注射该病菌,对该理论进行检验,并假定在进行实验的动物中产生了该种同样的疾病,这些检验给说明性假说以强的支持。我们当然试图在临床医学中使用该理论。做法是,通过消灭患有该病的病人身上的细菌而将病治愈——先在实验人群中进行,然后再按照常规来进行。正是按照这种方法,我们学会了如何与许多可怕的人类疾病进行战斗,在一些情况下甚至完全消灭这些疾病。通过科学,我们试图理解世界;同样通过科学,我们使用一些手段,对世界给予我们的危险进行控制。

\section{科学研究的七个阶段}
1.确定问题\\
2.选择初始假说\\
3.收集额外事实\\
4.形成说明性假说\\
5.推导进一步结果\\
6.检验结果\\
7.应用理论 

\begin{center}
\fbox{\parbox{0.95\textwidth}{
\textbf{本节要点}
\begin{itemize}
\item \textbf{科学研究的七个阶段}:
  \begin{itemize}
  \item 科学研究是系统化的问题解决过程,包含七个关键步骤
  \item 这些步骤在实际研究中常常交错进行,而非严格线性
  \item 每个阶段对科学发现的完整过程都至关重要
  \end{itemize}
\item \textbf{确定问题与形成假说}:
  \begin{itemize}
  \item 科学研究始于明确需要解释的现象或问题
  \item 初始假说是必要的,指导进一步的数据收集
  \item 没有假说就无法确定哪些数据相关,需要收集
  \end{itemize}
\item \textbf{收集数据与完善假说}:
  \begin{itemize}
  \item 收集额外事实帮助调整和完善初始假说
  \item 科学家需要相互循环地分析数据和修正假说
  \item 最终形成能解释所有观察事实的说明性假说
  \end{itemize}
\item \textbf{检验与应用}:
  \begin{itemize}
  \item 从假说推导出进一步可检验的预测
  \item 通过实验或观察对这些预测进行检验
  \item 成功的理论最终应用于解决问题和控制现象
  \end{itemize}
\end{itemize}
}}
\end{center} 