\section{作为假说的分类}

\begin{quotation}
本节探讨分类作为科学假说的特殊性质。我们将分析科学分类如何不仅仅是事物的简单归类,而是反映自然界内在联系的理论假设。通过理解分类系统如何帮助科学家发现新知识、预测未知性质,我们将认识到一个好的分类系统既有理论基础,也有实际应用价值。
\end{quotation}

分类与划分紧密相关。但是分类,即在哲学中所称的"等级"(genus,种的上义词)中的"种"(species)的秩序,可以看做是假说,而严格地说,划分并非如此。分类不仅是要求完备和互斥,而且要求物体、生物或者思想在"本质特征"的基础上得以分组,其目的是对所考察的现象进行解释,或者至少为之提供一个解释的框架。这些"本质特性"因而必须是解释性的。

存在这样一个观点,假说仅在比较发达的科学中而不是在相对不发达的科学中才发挥重要作用。这个观点应当被拒绝。有人主张,尽管说明性假说在物理学、化学这样的科学中起重要作用,它们在生物学和社会科学中则没有这样的作用,或者至少现在没有这样的作用;后者仍然处于描述性阶段,而假说方法对所谓描述性的科学如植物学、历史学是不合适的。我们很容易对这个观点给出反驳。对描述本性的考察将显示,描述本身是建立在假说之上的,或者说描述本身包含假说。假说在生物学的不同体系的分类法或分类学中是基本的,在历史学或其他社会科学中假说也是基本的。

在历史科学中,假说的重要性容易得到阐明,我们首先讨论它。一些历史学家相信,历史研究能够揭示存在着的单个宇宙目的或模式,该目的或模式或者是宗教的或者是自然的,它对有记载的历史的整个进程进行解释或说明。其他的历史学家则否认有任何这样的宇宙设计的存在,但他们坚持认为,历史研究将揭示某个历史规律,该规律解释过去事件的实际次序,并能够用来预测未来。无论哪种观点的历史学家,他们寻求的说明必须解释过去记载的事件,并被它们所证实。因而,无论是哪种观点,历史学是一个理论性的科学而不仅仅是描述性的科学,必须承认假说在历史学家事业中的中心作用。

然而,有第三种历史学家,他们更为谦虚地设定他们的目标。根据他

们的观点,历史学家的任务只是简单地将过去进行编年史记录,即以他们的编年史顺序将过去的事件进行简单的记录。似乎是,根据这种观点, "科学的"历史学家没有进行假说的必要,因为他们所关心的只是事实本身,而非与事实有关的理论。

但是过去的事件没有像该观点试图使我们相信的那样容易编年。过去本身不能提供这种记录。现在能够得到的是现在的记录和过去的痕迹。它们的范围是:从关于过去的官方档案,到对半传说式的英雄的征服行为的赞美的史诗;从以前的历史学家的作品,到考古学家挖掘出土的过去年代的物品。这些只是历史学家能够获得的事实,而从这些事实中他们必须推论得出过去事件的本质——这是他们描述的目的。不是所有的假说是全称的,有些是特称的。历史学家关于过去的描述是特称假说,使用该假说的意图是解释现有数据,而现有数据构成了它的证据。

从大范围来看,历史学家犹如侦探。\cite{collingwood1939} 他们的方法是共同的,遇到的困难也类似。其困难主要是证据不充足,并且,许多证据如果不是被笨手笨脚的地方警察所损坏,就是被相关的战争和自然灾害所破坏。正如罪犯可能留下了假的或误导的线索以甩掉追踪者,太多的现存"记录",据说是对过去的描述,而实际上是对过去的歪曲:或者是有意的,如"康斯坦丁的赠款"这样伪造的历史文档的案子,或者是无意的,如早期没有批判性的历史学家的著作。正如侦探建立和检验假说必须使用科学方法一样,历史学家也必须如此。即使将自己限于对过去的纯粹描述的那些历史学家,也必须使用假说来工作:他们是理论家,不管他们自己是如何认为的。

生物学家所处的位置稍微有利。他们处理的事实是现在的,易于检查。为了描述一个地区的动植物群落,他们不必精心构造历史学家那种遭到诟病的复杂推理。数据可以被直接地认识到。对这些项目的描述不是因果的,而只是系统的。他们被认为是对动物和植物进行分类,而不仅仅描述它们。但是分类和描述实际上是同一个过程。将一给定动物描述成食肉类,即是将它分类为一个食肉动物;将它归类为爬行类,即是将它描述成爬行动物。某个物体被描述具有一个给定属性,即是将之归类于具有该属性的对象类中的一个成员。

分类,正如通常理解的那样,不仅仅要将对象划分成不同的群体,而且要将每个群体进一步划分成次一级的群体或次一级的类,等等。这个模

式是我们大多数人所熟悉的:如果不是从学校学习中知道,那么可能是从 "动物,植物,还是矿物?"这个古老的游戏(或者更普遍地被叫做" 20个问题"${ }^{(1)}$ )中获得。分类是一个普遍的需要。原始人不得不将草根、浆果划分为可食用的还是有毒的,将动物分为危险的还是安全的,以及将部落分为友善的还是敌对的。所有人都根据自己的实际需要而进行区分,并忽视那些在他们的事务中不怎么重要的区别。农民会对谷物和蔬莱进行小心和仔细的分类,而将各种花只统称为"花";而卖花人却会细致地将他们的商品进行分类,而将农民的所有收成一起称为"农产品"。

我们对事物进行分类有两个基本的动机。一个是实践的,一个是理论的。某人仅有 3 或 4 本书,他对它们了如指掌,他一瞥就能够分辨它们,就此没有对它们进行分类的必要。但是在一个包含上万册书的图书馆里,情况便不同。如果不对图书进行分类,图书管理员就不能找到所需要的书,该图书馆的收藏将无实际用处。物体数量越大,越有必要对它们进行分类。分类的实际目的是使大量的采集成为可能。在图书馆、展览馆和各类公共记录展厅的情形中这特别明显。

当我们考虑分类的理论用途时,我们必须认识到,使用一种分类法或另一种分类法与真理和错误无关。可以用不同方式、以不同的观点来描述物体。使用的分类方案依赖于分类者的目的和兴趣。例如,图书管理员、图书装订商和藏书家对书的分类便有所不同。图书管理员根据书的内容或主题对书进行分类,图书装订商根据的是装订方式,图书收藏者根据的是印刷日期和相对稀有程度。当然我们不能穷尽各种可能性:图书包装者会根据书的形状和大小对书进行分类,而对书有其他兴趣的人根据他们不同的兴趣进行不同的分类。

那么,科学家的什么样的特别兴趣或目的,使他们偏爱一个分类方案而不是另外一个?科学家的目的是获得知识:不仅仅是关于这个或那个特别的事实的知识,尤其是关于用事实来确认的普遍定律的知识,以及事实之间因果连接的知识。从科学家的观点来看,一个分类方案比另外一个要好,一定程度上在于,在得出科学定律的过程中它是更富于成果的,并且在形成说明性假说过程中它是更有帮助的。

\footnotetext{(1)这是一个游戏。多人确定出一个东西,让另外的人通过询问问题来猜他们所确定的是什么东西。
}对物体进行分类,其理论的或科学的动机是增进这些物体知识的愿望。事物的知识的增加可以增进我们对事物的属性,它们的相似性及差别,以及它们的相互关系的进一步理解。一个分类方案如果只为狭窄的实际目的而制定,就可能抹杀了重要的相似性和差异性。把动物划分成"危险的"和"没有危险的",如把野猪和响尾蛇归为一类,把家猪和草蛇归为一类,这种划分为了强调表面的相似性,而忽视了更本质的相似性。对物体的任何科学的、富有成果的分类需要具有关于它们的大量的知识。对比较明显的特征的粗浅了解,会使人们将蝙蝠作为会飞的生物归为鸟类,把鲸作为生活在大海中的生物归为鱼类。如果我们具有更广阔的知识,我们便将蝙蝠和鲸两者均归为哺乳动物,因为它们都属于温血、胎生并哺育幼崽的动物——这些是分类所根据的更为重要的特征。

如果一个特征能够作为线索,以发现其他特征,它便是重要的特征。从科学的视角来看,一个重要的特征是指这样一个特征:它与许多其他的特征有因果连接关系,因而它能够作为最大数量的因果律的框架,并且有助于形成最普遍的说明性假说。因此,这样的分类方案是最好的,如果它建立在所要分类的物体的最重要的特征之上的话。但是正如我们已经强调的那样,我们事先并不知道会得到哪些因果律,而且因果律本身也带有假设的性质,因而,在哪个分类方案是最好的问题上的决策本身就构成一个假说,一个后续研究可能将之否定的假说。如果后来的研究揭示了,其他特征更为重要(即它与大量的因果律和说明性假说相关),我们能够合理地预期,原来的分类方案应当被否决,我们会选取基于更重要的特征之上的新的分类方案。

分类方式是假说的观点被它在科学中实际所起的作用所证实。分类学是生物学中一个正统的、重要的并且欣欣向荣的分支学科。在生物学中,某些分类方式.如林奈的分类法,被采纳、使用,后来因有了更好的方案而被弃用;更好的方案本身在新的数据下也经受着修改。一般来说,在科学的早期阶段或不发达阶段,分类最为重要。然而,随着科学的发展其重要性不一定总是降低。例如,由门捷列夫表所表明的元素的标准分类法、仍然是化学家的一个重要工具。

前面对自然科学中使用分类的解释,可启发我们进一步认识在历史研究中使用分类的重要性。我们已经说明,历史学家对过去事件的描述本身即是基于冒前资料之上的假说。然而,假说在描述的历史学家的事业中发

挥着另外一个同等重要的作用。任何数量级的历史事件都不能被完完全全地描述。即使人们能够知道它的所有细节,历史学家也不可能将之全部写进著作之中。生命过于短暂,它不允许人们对事物进行详尽无遗的描述。因此,历史学家必须对过去进行有选择地记载,记录下的仅仅是过去的一些特征。历史学家进行选择的基础是什么?显然,历史学家要叙述的是有意义的或重要的,而忽略无意义的或琐碎的。这个或那个历史学家的主观偏见会使他或她过分强调历史进程中宗教、经济、人物或其他某个方面的作用。但如果历史学家们考虑到,要做出客观的或科学的评价,他们便会重视那些能够形成因果律和普遍的说明性假说的因素。自然,这样的评价会随着进一步研究而经受着改变。

西方第一个历史学家希罗多德(Herodotus)细致描写了他编入编年史的事件,有人物的、文化的,以及政治的、军事的。所谓第一位科学的历史学家修昔底德,将自己的写作更多地限于政治和军事方面。在很长一段时间里,大多数历史学家跟随修昔底德,但是现在钟摆正摆向另外一个方向:历史学家十分重视过去的经济和文化方面。正如生物学家的分类方案包含了他们的假说——通过这个假说生物的特征与最大数量的因果律相关联,历史学家选择用一个典型事件集合而非另外一个集合来描述过去事件,这种选择包含了他们的假说:什么样的典型事件与最大数量的其他典型事件因果地连接在一起。这样的假说是必需的,哪怕历史学家对过去进行系统的描述的工作只是刚刚开始。正是分类和描述——无论是生物学的还是历史学的一一所具有的假说性的特点,使我们将假说看成是科学探究的通用方法(the all-method)。 

\begin{center}
\fbox{\parbox{0.95\textwidth}{
\textbf{本节要点}
\begin{itemize}
\item \textbf{分类作为科学假说的本质}:
  \begin{itemize}
  \item 分类不仅是将事物归类,而是基于"本质特征"的理论假设
  \item 科学分类系统是对自然界内在联系的假设性解释
  \item 分类方案本身即是假说,可被后续研究证实或否定
  \end{itemize}
\item \textbf{分类在科学中的广泛应用}:
  \begin{itemize}
  \item 不仅在自然科学中,在历史等描述性学科中同样重要
  \item 分类在科学发展早期尤为重要,但在成熟科学中仍有价值
  \item 历史学家对事件的分类和描述同样包含假说性质
  \end{itemize}
\item \textbf{科学分类的评价标准}:
  \begin{itemize}
  \item 好的分类基于与多种特征有因果联系的重要特征
  \item 科学价值取决于能否揭示事物间的内在联系和规律
  \item 能够促进形成普遍定律和说明性假说的分类更有价值
  \end{itemize}
\item \textbf{分类的实践与理论意义}:
  \begin{itemize}
  \item 实践意义:使对大量对象的管理和利用成为可能
  \item 理论意义:增进对事物属性、相似性和差异的理解
  \item 反映科学家对事物本质特征的理解和认识目标
  \end{itemize}
\end{itemize}
}}
\end{center}

\printbibliography[heading=subbibliography,title={第13章参考文献}]