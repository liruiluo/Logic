\section{科学的价值}

\begin{quotation}
本节探讨科学在人类社会中的重要价值。我们将分析科学的实践价值与认识价值,理解科学如何通过技术应用改善人类生活,以及如何通过揭示自然规律满足人类认识世界的需求。通过理解科学不仅是事实的收集,而是对现象的系统解释,我们将认识到科学理论和自然定律在解释世界中的核心地位。
\end{quotation}

现代科学几乎改变了我们生活的每个方面。它的实践价值在于,它使得更方便、更健康和更丰富的生活成为可能。尽管人们对于它的一些成果颇感忧虑,然而,大多数人同意,科学进步及其在通信、运输、制造、种植、娱乐和公共卫生等等方面的技术应用,总的来说大大地有益于人类。

科学在实现认识愿望方面也实现了内部价值。很久以前,亚里士多德写道:"认识某个事情(不仅哲学方面的,而且人类的其余方面的)是最大的乐趣。"\cite{aristotle1950} 爱因斯坦则代表所有时代的科学家写道:

\begin{displayquote}
什么因素推动我们发明一个又一个理论?我们为什么要发明理论?答案很简单:因为我们乐于"理解"(comprehending),即,通过逻辑过程将现象归约为已经知道或有(明显)证据的事物。\cite{einstein1935}
\end{displayquote}

科学的目标就是发现普遍真理。当然单个事实是关键的;用事实建造科学大廈,如同用石头建造房屋。采集了石头不等于建成房屋,仅仅事实的收集更不能成为科学。科学家寻求理解现象,为此,他们努力揭示现象发生的方式以及它们之间的系统的关系。

仅仅知道事实是不够的,对它们进行说明是科学的任务。因而,这需要理论(如爱因斯坦所说),以及与理论相连的支配事实的自然定律和基础性原理。 

\begin{center}
\fbox{\parbox{0.95\textwidth}{
\textbf{本节要点}
\begin{itemize}
\item \textbf{科学的实践价值}:
  \begin{itemize}
  \item 科学通过技术应用改变了人类生活的各个方面
  \item 使人类享有更便捷、健康和丰富的生活
  \item 在通信、运输、医疗、农业等领域提供了重大进步
  \end{itemize}
\item \textbf{科学的认识价值}:
  \begin{itemize}
  \item 满足人类天然的求知欲和理解世界的渴望
  \item 如亚里士多德所言,认识事物本身就是最大的乐趣
  \item 爱因斯坦强调,科学家乐于通过逻辑过程理解现象
  \end{itemize}
\item \textbf{科学与事实的关系}:
  \begin{itemize}
  \item 科学不仅仅是事实的收集,而是对事实的系统解释
  \item 单个事实是建造科学大厦的基石,但不等同于科学本身
  \item 科学家寻求揭示现象间的系统关系和规律
  \end{itemize}
\item \textbf{科学的目标}:
  \begin{itemize}
  \item 发现普遍真理和自然定律
  \item 通过理论建构对现象进行解释
  \item 揭示现象发生的方式和现象之间的系统关系
  \end{itemize}
\end{itemize}
}}
\end{center} 