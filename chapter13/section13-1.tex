\section*{篘 13 第}
\section*{科学和假说}

\section*{13.1 科学的价值}
现代科学几乎改变了我们生活的每个方面。它的实践价值在于,它使得更方便、更健康和更丰富的生活成为可能。尽管人们对于它的一些成果颇感忧虑,然而,大多数人同意,科学进步及其在通信、运输、制造、种植、娱乐和公共卫生等等方面的技术应用,总的来说大大地有益于人类。

科学在实现认识愿望方面也实现了内部价值。很久以前,亚里士多德写道:"认识某个事情(不仅哲学方面的,而且人类的其余方面的)是最大的乐趣。"${ }^{[1]}$ 爱因斯坦则代表所有时代的科学家写道:

\begin{displayquote}
什么因素推动我们发明一个又一个理论?我们为什么要发明理论?答案很简单:因为我们乐于"理解"(comprehending),即,通过逻辑过程将现象归约为已经知道或有(明显)证据的事物。 ${ }^{[2]}$
\end{displayquote}

科学的目标就是发现普遍真理。当然单个事实是关键的;用事实建造科学大廈,如同用石头建造房屋。采集了石头不等于建成房屋,仅仅事实的收集更不能成为科学。科学家寻求理解现象,为此,他们努力揭示现象发生的方式以及它们之间的系统的关系。

仅仅知道事实是不够的,对它们进行说明是科学的任务。因而,这需要理论(如爱因斯坦所说),以及与理论相连的支配事实的自然定律和基础性原理。 