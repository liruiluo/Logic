\section{说明:科学的和非科学的}

\begin{quotation}
本节探讨科学说明与非科学说明的本质区别。我们将分析什么构成一个有效的说明,辨别科学态度与非科学态度之间的关键差异,并理解科学说明的可检验性特征。通过理解科学说明必须是可证实的、探索性的,而非教条主义的,我们将能够区分真正的科学解释与仅凭权威或习俗的非科学说法。
\end{quotation}

当要对某个事情进行说明(explanation)时,我们需要什么?一个被寻求的解释(account)就是对世界的某个陈述集合,或某个叙说(sto- ry),从该解释中能够選辑地推导出需要解释的事情。该解释能够对需要解释的有疑问的问题进行消解或者简约。说明和推论可以被看成是同一个过程,只是方向相反。一个逻辑推论之进行是从前提到结论,而对任何给定事实的说明是确定从中事实能够被逻辑地推论出来的前提。在本书第1

章(1.6 节)中我们阐述了,当我们要推得 $Q$ 时,"由 $P$ 得 $Q$"如何表达一个论证;如果我们所进行的是从一个已经建立的 $Q$ 到能够对之说明的前提的推理,它也可以表达一个说明。

自然,每一个好的说明必须是相关的(relevant)。如果我解释说,我上班迟到是因为在中部非洲发生持续的政治混乱,那么这会被认为什么都没有说明;它是不相关的一一因为需要说明的我迟到的事实,不能从中被推论出来。当然每个真正的说明不仅是相关的而且是真实的。

无论我迟到的正确的说明是什么,之所以需要这个说明,是因为在我迟到的这个事件上存在疑问。然而,科学的说明除了相关和真实外,必须超越特定事件,而能够对给定种类的所有事件提供解释。牛顿力学的伟大在于万有引力定律。牛顿写道:

\begin{displayquote}
宇宙中每个质点以一个力吸引另外一个质点。该力正比于质\\
点质量的乘积,反比于它们间距离的平方。
\end{displayquote}

非科学的说明也可以是相关的和普遍的。可以用神秘的小鬼动了手脚,来解释引擎不能启动,这是非科学的;疾病可以解释成邪恶的精灵侵人人体所引起的。在长达数个世纪的时间里,人们一直用在行星上生活并控制它们运动的"智慧生物"来解释行星的规则运动。

但是我们对真正科学的说明感兴趣。科学的说明与非科学的说明在两个相互关联的方面相区别:

第一个区别是态度上的区别。接受非科学说明的人是教条的,解释被认为是绝对真的,是不能改进的。亚里士多德的观点在几个世纪里被非科学地接受成对事实的最终权威。尽管亚里士多德本人是谦虚的,但是一些中世纪的学者却以僵化的、非科学的态度对待他的观点。 ${ }^{13}$ 相反.真正科学的态度则与之十分不同。每个提出的说明都是探索性的或暂时的。科学说明被认为是假说一一它们在现有证据下具有不同的可靠程度。

科学说明和非科学说明之间的第二个同时也是最根本的区别是,接受或拒绝某个观点所基于的基础。一个非科学的说明被简单地认为是真的,因为"每个人知道"它如此。一个非科学信念之被坚持,不依赖于有利于它的证据之上。但是在科学中,一个假说仅仅在存在支持它的证据的条件下才值得接受,人们总是对它的真或假保持怀疑,寻找证据的过程永不停

止。科学是经验的一一真理的检验在于经验之中,因而科学说明的本质是,它是可检验的。

真理的检验可以是直接的也可以是间接的。为了弄清外面是否下雨,我只要看一下外面。但是用做说明的假说是普遍性命题,它们不能是直接可检验的。如果我对我上班迟到的解释是交通事故,我的老板如果对之怀疑,他能够借助于警察的事故报告而间接地检验我的解释。一个间接的检验从待检验的命题(如我遭遇到一次交通事故)演绎出其他某个能够被直接检验的命题(如我提交了一个事故报告)。如果那个演绎出来的命题是错的,包含这个命题的说明必定是错的。如果演绎出来的命题是真的,它提供了某个证据证明这个说明是真的、已经被间接证实。但是这个证据不是结论性的。

间接检验永远不会是确定的。它总是依赖于某些辅助的前提,比如这样的前提:我对我的老板描述的该起事故与警察记载的一样。但是警察部门应当对我所涉案的事故的记录进行备案,但可能还没有备案;缺乏该记录并不证明我的说明是假的。并且,某个附加前提即使是真的,它并不给说明赋予确定性——尽管演绎出的结论(本例中事故报告的真实性)得到成功检验确实加固了它的前提。

即使一个非科学的说明也有某个它喜爱的证据,即用它来解释的那个事实。行星上居住着"智慧生物",他们使行星沿着我们观察到的轨道运动,这个非科学理论能够称这个事实一一行星确实在它们的轨道上运动——为证据。但是,在该假说和关于行星运动的可靠的天文学说明之间存在巨大的差别:对于非科学假说,不能够从中演绎出其他的可直接检验的命题。另外一方面,一个给定现象的任何一个科学说明能够演绎出可直接检验的命题,而不是陈述待解释事实的命题。这就是当我们说一个说明是经验可证实的时所要表达的意思。这样的可证实性是科学说明最本质的特征。 ${ }^{[5]}$ 

\begin{center}
\fbox{\parbox{0.95\textwidth}{
\textbf{本节要点}
\begin{itemize}
\item \textbf{说明的基本性质}:
  \begin{itemize}
  \item 说明是从中能推导出待解释现象的陈述集合
  \item 说明与推论是相反方向的逻辑过程
  \item 有效的说明必须是相关的和真实的
  \end{itemize}
\item \textbf{科学说明的特点}:
  \begin{itemize}
  \item 超越特定事件,能对同类事件提供普遍解释
  \item 具有可检验性,能从中演绎出可直接验证的命题
  \item 是探索性的、暂时的,而非绝对或最终的
  \end{itemize}
\item \textbf{科学与非科学说明的态度区别}:
  \begin{itemize}
  \item 科学态度:假说被视为暂时的,总是存在修正可能
  \item 非科学态度:教条式接受,解释被视为绝对真理
  \item 科学家对所有理论保持适当怀疑,不断寻求新证据
  \end{itemize}
\item \textbf{科学与非科学说明的本质区别}:
  \begin{itemize}
  \item 科学说明基于证据,依赖经验检验
  \item 非科学说明基于权威、习俗或"常识"
  \item 科学说明能产生可检验的预测,而非仅解释已知事实
  \item 经验可证实性是科学说明最关键的特征
  \end{itemize}
\end{itemize}
}}
\end{center} 