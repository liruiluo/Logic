\section{对科学说明的评价}

\begin{quotation}
本节探讨如何评价相互竞争的科学说明。我们将分析科学家用于判断假说优劣的三个关键标准:与已有理论的协调性、预测力或说明力,以及简单性。通过理解这些评价标准,我们将能够认识科学知识是如何在理论竞争中发展的,以及如何在相互矛盾的科学解释之间做出合理选择。
\end{quotation}

对同样的现象,人们往往会提出不同的、相互不协调的科学说明以对之进行解释。我的同事动作生硬,可以解释成她生气了,也可以解释成她害羞。在刑事调查中,对犯罪的认定有两个相互不协调的假说,都对犯罪事实有很好的解释。但是当两个假说不能都真的时候,我们将如何在其中

做出选择?\\
这里,我们在做的是评价相互竞争的科学说明。我们假定两个(或所有的)假说都是相关的并且是可检验的。我们应当采用什么标准以便从手边的理论中选择出最好的理论?我们不能指望存在这样的规则,它们引导我们发现假说;发现假说是科学事业的创造方面,它体现了天才和想象力,在某些方面类似于艺术工作。尽管不存在发现新假说的公式,但存在比相关性和可检验性更进一步的标准,我们可以用这些标准对可接受的假说进行确证 (conform)。

人们在评判竞争的科学假说的优缺点时普遍地使用三个标准:\\
1.与原有已确立假说的协调性\\
科学的目标是获得一个说明性的假说系统。当然,这样的系统必须是自我相容的,因为没有一个自相矛盾的命题集合能够是真的。进步之得出是通过渐渐发展假说以理解越来越多的事实,这样的进步要求每个新假说应与已经得到证实的那些假说相一致。例如,在天王星轨道外面存在另外一个未知的行星的假说,与天文学理论的主要部分吻合完美,它导致海王星的发现(1846 年)。 ${ }^{[6]}$ 科学中的进步是有序的,这要求任何新的理论与以前的理论相一致。

科学的理想是通过一个又一个新理论的增加而使知识发生渐渐地增长,但是科学进步的实际历史不总是遵循这种有序的方式。有时重要的新假说与已有理论不相容,它直接替代了已有理论,而不是努力与旧理论相一致。爱因斯坦的相对论就是这种假说,它突破了旧的牛顿理论中的许多原有概念。19世纪后期放射性物质的发现推翻了物质守恒原则。物质守恒原则断定,物质既不能被创造也不能被消灭。镭原子发生自发衰变的假说直接与这旧的、已被接受的原则不相容,最终这个旧的原则不得不被抛弃。

科学中旧理论被抛弃,较新的和较好的理论被接受,这个过程不是很快或者无抵抗的。事实上,旧理论不是被认为一无是处地被抛弃。爱因斯坦自己总是坚持,他自己的工作是对牛顿工作的一个修正,而非抛弃。物质守恒原则被修正成更为广泛的质能守恒原则。一个理论之被建立,因它显示出能够解释大量的数据或已知事实的能力。它不能被某些新假说所废弃,除非新假说对同样的事实能够进行解释甚至更好。

因此,科学通过采用更为广泛因而更为相关的说明而得以进步。通过

说明,世界将它展现在我们的经验面前。这种进步不会是反复无常的。当不相容产生的时候,一个假说的年岁较长不能自动证明它是正确的,但是这个假定(年岁较长)有利于旧假说——如果旧假说已经得到广泛的确证。如果与它发生冲突的新假说同样获得广泛的确证,考虑假说的年岁或提出的先后是不合适的。当两个假说发生冲突的时候,为了在它们间做出选择,我们必须求助于可观察的事实。上诉的最终法庭是经验。

这个标准一一与先前良好建立的假说相协调——所要达到的最终结果是,任何时候所接受的假说全体必须是相互融贯的。 ${ }^{[7]}$ 在其他情况一样的条件下,与已接受的科学理论吻合得较好的假说应当被偏爱。与"其他情况一样"有关联的问题将我们带到第二个标准。

\subsection{预测力或说明力}
正如我们已经看到的,每个科学假说必须是可检验的;如果某个可观察的事实能够从中演绎出来,它就是可检验的。当我面临两个可检验的假说,其中一个比另外一个演绎出更大范围的事实,我们说该假说具有较大的预测力或解释力。

举例来说明。伽利略(1564-1642)建立了落体定律公式,该定律对靠近地球表面的物体的行为给出了一个十分普遍的解释。差不多同时,德国天文学家乔哈恩斯•开普勒,用丹麦的第谷•布拉赫收集的天文数据建立了行星运动定律,该定律描述了行星绕日运行的椭圆轨道。这些科学家将各自研究领域里(伽利略——陆地上的力学,开普勒——天体力学)的不同现象统一起来。这些发现自然是辉煌的成就,但是它们是各自分离的。艾萨克-牛顿提出了三大运动定律和万有引力理论,将这些分离理论统一了起来并给予了解释。牛顿万有引力解释了所有伽利略和开普勒解释的结果,以及除此之外更多的事实。从一个给定假说中演绎出一个可观察的事实,我们说该事实被该假说所说明,并且我们也能够说该事实被该假说所预测。牛顿理论具有巨大的预测力。一个假说预测力越大,它解释得越多,并且它对我们理解它所涉及的现象的贡献越大。 ${ }^{[8]}$

这第二个标准具有否证作用。如果一个假说与某个得到证实的观察不一致,该假说便是错的,必须被拒绝。当两个不同假说都能完全解释某个事实集合,都是可预测的,并且都与已经构建的整个科学理论相协调,此时,在它们之间做出选择是可能的:从它们推出可检验的但相互不协调的命题。为了在冲突的理论中做出选择,可以建立一个判决性实验。根据第

一个假说,在确定条件下一个给定结果将发生;而根据第二个假说,在那些同样的条件下给定结果将不发生,我们可以通过观察该结果发生还是不发生,而在两个假说中做出选择:它的发生否证了第二个假说,它的不发生则否证了第一个假说。

对两种相竞争的假说进行判决的这种判决性实验也许不容易实现。原因在于,制造那些关键性事件是困难的或者不可能的。牛顿理论和爱因斯坦广义相对论之间的决策不得不等到日全食的发生——这是一个明显超出我们自己能够创立的事件。 ${ }^{[9]}$ 在其他情况,判决性实验可能要等到新工具的发明:这些新工具或者是为了创造所需的条件,或者是为了对已经做出预测的现象进行观察或测量。因此,竞争的天文假说的支持者有时必须等待时机,等待建造出新的、威力更强大的望远镜。对于判决性实验,在 13.6 节中我们将进一步讨论。

\subsection{简单性}
两个竞争性假说可能是相关的和可检验的,可能与已有理论吻合得同样好,甚至可能具有大致相当的预测力。在这样的条件下,我们可能支持两个中比较简单的那个。关于天体运动的托勒密理论(地球中心)和哥白尼理论(太阳中心)之间的冲突就是如此。两者都与早先的理论吻合良好,它们都同等程度地预测天体运动。两个假说都依赖于一个笨拙的(自然是错误的)工具——假想的本轮(较小的圆在较大圆上运动),以解释已做出的天文观察。但是哥白尼系统依赖这样的本轮更少,因而它更简单,这个较大的简单性是后来的天文学家接受该理论的主要原因。 ${ }^{[10]}$

简单性似乎是一个可以求助的"自然"标准。在日常生活中我们同样趋向于接受符合所有事实的最简单的理论。在刑事法庭上对一犯罪行为会提出两种观点,最终在该案子上更简单、更自然的观点可能被支持(或应当被支持)。

但是"简单性"是一个不好捉摸的观念;只有在非常少量的情况下,如在托勒密和哥白尼的冲突中,我们根据较少的实体数的要求选择比较简单的理论。在两个竞争的理论中可能是,在不同的方面一个比另外一个简单。一些人可能依赖于比较少的实体数量,而其他人可能基于较简单的数学方程。甚至"自然"(naturalness)可能是欺骗人的。许多人会更"自然"地相信,明显不运动的地球事实上是不动的,而明显运动着的太阳确实环绕我们在运行。简单性是一个重要的标准,有时甚至是决定性的。但

是它是难以公式化,并且不总是易于应用的。 

\begin{center}
\fbox{\parbox{0.95\textwidth}{
\textbf{本节要点}
\begin{itemize}
\item \textbf{科学说明评价的必要性}:
  \begin{itemize}
  \item 同一现象常有多种相互竞争的科学解释
  \item 需要可靠标准来判断哪些假说更优越
  \item 假说的创造依赖创造力和想象力,但评价需要严格标准
  \end{itemize}
\item \textbf{与已有理论的协调性}:
  \begin{itemize}
  \item 科学追求自洽的假说系统,新理论应与已建立理论相容
  \item 科学进步通常是渐进的,新假说要兼容已证实的理论
  \item 重大科学革命可能打破旧理论框架,但必须解释旧理论能解释的现象
  \item 任何时候所接受的假说体系必须是相互融贯的
  \end{itemize}
\item \textbf{预测力或说明力}:
  \begin{itemize}
  \item 能从假说演绎出更广泛事实的理论具有更大预测力
  \item 如牛顿理论统一并解释了伽利略和开普勒的分离理论
  \item 具有否证作用:若理论与观察不符,必须被拒绝
  \item 判决性实验可以在竞争理论间做出选择
  \end{itemize}
\item \textbf{简单性原则}:
  \begin{itemize}
  \item 在其他条件相同时,应选择更简单的理论
  \item 如哥白尼日心说比托勒密地心说更简单
  \item 简单性标准难以精确定义,不同理论可能在不同方面简单
  \item 简单性是重要标准,但难以公式化且应用存在挑战
  \end{itemize}
\end{itemize}
}}
\end{center} 