\section*{藛8密}
\section*{符号逻辑}
\section*{8.1 现代逻辑的符号语言}
我们一直在寻求对演绎论证进行分析和评估的技术。演绎理论旨在提供这样的技术,它已经发展出两个不同的分支来做这件工作:此前三章所考察的是经典逻辑或亚里士多德型逻辑,本章和下两章的主题则是现代符号逻辑。

然而,论证的分析和评估经常因其表述语言的特性(如英语或任何其他自然语言的特性)而非常困难。自然语言使用的语词可能是模糊的或歧义的,论证的结构可能是含混的,比喻和习语可能会引起混淆或误导,诉诸情感可能会引起混乱等,这些问题在第一部分已经探讨过了。要避免这些困难就要直接进人论证的逻辑核心,为此逻辑学家们构造了一种能避免自然语言缺陷的人工符号语言。使用这种符号语言能精确地表述论证。

符号也能便利我们对论证的思考。"由于符号系统之助,"一位杰出的现代逻辑学家写道,"我们几乎用眼睛就可以机械地进行推理转换,否则,这种转换本来要求大脑有很高的智能。"${ }^{[1]}$ 这似乎有点悖谬,但符号语言确实可以帮助我们不需大伤脑筋就能完成某些智力活动。

古代的和古典的逻辑学家们也承认某种特殊逻辑记号的价值。亚里士多德在自己的分析中就使用了变项,而如前面几章所表明,改进了的亚里士多德型逻辑也以很复杂的方式使用了符号。 ${ }^{[2]} 20$ 世纪又有很大的改进。

在现代逻辑中,处于核心地位的不是三段论(如亚里士多德传统上的),而是逻辑联结词,它们是每个演绎论证,不管是不是三段论,在其构成要素之间的关系中所必须运用的。命题和论证的内在结构是现代逻辑关注的焦点。要理解这种结构,我们必须首先掌握现代逻辑分析中所使用的一些特殊符号。

现代符号逻辑不受演绎论证要转换成三段论形式的制约(亚里士多德型逻辑受这种制约)。正如我们在第 7 章所见,那种工作是很费力的。不必进行这种转换使得我们可以更直接地追求演绎分析的目标。下面给出的现代逻辑的符号记法是分析论证的特别有力的工具。使用这种记法我们可以更全面地达到演绎逻辑的核心目标:区分有效论证和无效论证。 