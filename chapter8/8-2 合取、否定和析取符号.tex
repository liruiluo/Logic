\section{合取、否定和析取符号}

\begin{quotation}
本节介绍符号逻辑中最基本的三种逻辑联结词:合取、否定和析取,以及它们的符号表示和真值定义。通过掌握这些基本符号和标点符号的使用,我们能够将复杂的自然语言论证转换为明确的符号形式,消除歧义,并为后续的形式化分析做准备。
\end{quotation}

在本章,我们将关注一些如下述例子般简单的论证:

那个盲囚戴红帽子或者那个盲囚戴白帽子。\\
那个盲囚没戴红帽子。\\
因此,那个盲囚戴白帽子。

以及

如果鲁宾逊先生是那个司闸员的邻居,那么鲁宾逊先生住在底特律和芝加哥之间。

鲁宾逊先生不住在底特律和芝加哥之间。\\
因此,鲁宾逊先生不是那个司闸员的邻居。

这种类型的论证都至少包含一个复合陈述。研究这样的论证时,我们把所有陈述分为两个大类,即\textbf{简单的}和\textbf{复合的}。一个简单陈述就是一个不包含任何其他陈述作为其分支的陈述。臂如,"查理是整洁的"就是一个简单陈述。一个复合陈述就是包含另外一个陈述作为其分支的陈述。譬如,"查理是整洁的并且查理是可爱的"就是一个复合陈述,因为它包含两个简单陈述作为其分支。当然,一个复合陈述的分支陈述自身也可以是复合的。\cite{tarski1946}

\subsection{合取}
复合陈述有几种不同类型,每种都需要有其逻辑记法。第一种复合陈述是\textbf{合取}。通过在两个陈述之间使用语词 and("和"、"并且"),可以形成它们的合取;被如此联结的两个陈述叫\textbf{合取支}。因此,复合陈述"查理是整洁的并且查理是可爱的"就是一个合取,它的第一个合取支是"查理是整洁的",第二个合取支是"查理是可爱的"。

语词"和"是个简短且便利的词,但除了联结陈述外,它还有其他一些用法。臂如,陈述"林肯和格兰特是同时代人"不是一个合取,而是一

个表达关系的简单陈述。为了有一个其唯一功能是合取地联结陈述的独特符号,我们引人圆点"•"作为合取符号。于是,前述合取可以写成"查理是整洁的-查理是可爱的"。更一般的,如果 $p$ 和 $q$ 代表任意两个陈述,它们的合取就写为 $p \cdot q$ 。

我们知道每个陈述是或真或假的。因此我们说,每个陈述都有一个\textbf{真值},一个真陈述的真值是真,一个假陈述的真值是假。用这种"真值"概念,按照一个复合陈述的真值是完全由它的分支陈述的真值确定,还是由它的分支陈述的真值以外的任何其他东西确定,可以把复合陈述分成两个不同的种类。

我们把这种区分运用到合取上。两个陈述的合取的真值完全地由它的两个合取支的真值确定。如果它的两个合取支都是真的,该合取就是真的;否则它就是假的。基于这个原因,我们说合取是\textbf{真值函项复合陈述},其合取支是它的\textbf{真值函项分支}。

然而,并非所有复合陈述都是真值函项的。例如,复合陈述"奥赛罗相信苔丝德蒙娜爱卡西奥"的真值,无论如何都不是由作为它的分支的简单陈述"苔丝德蒙娜爱卡西奥"的真值确定的,因为不管苔丝德蒙娜是否爱卡西奥,奥赛罗相信苔丝德蒙娜爱卡西奥仍然可以是真的。因此,"苔丝德蒙娜爱卡西奧"不是陈述"奥赛罗相信苔丝德蒙娜爱卡西奧"的真值函项分支,该陈述自身也不是一个真值函项复合陈述。

为当前目的起见,如果一个复合陈述中的某个分支被任何有相同真值但互相区别的陈述替换,其所得不同复合陈述相互之间有相同的真值,那么,我们就把这个复合陈述的分支定义为它的一个真值函项分支。这样,如果一个复合陈述的所有分支都是它的真值函项分支,我们就可以把该复合陈述定义为一个真值函项复合陈述。\cite{church1956}

我们将只关注真值函项复合陈述。因此,在本书的余下部分,我们将用术语简单陈述指称不是真值函项复合陈述的任何陈述。

一个合取就是一个真值函项复合陈述,因此,我们的圆点符号就是一个真值联结词。已知任何两个陈述 $p$ 和 $q$ ,它们只有四种可能的真值组合。这四种可能情形及每种情形下该合取的真值可以排列如下:

如果 $p$ 为真且 $q$ 为真,那么 $p \cdot q$ 为真。如果 $p$ 为真且 $q$ 为假,那么 $p \cdot q$ 为假。

如果 $p$ 为假且 $q$ 为真,那么 $p \cdot q$ 为假。\\
如果 $p$ 为假且 $q$ 为假,那么 $p \cdot q$ 为假。

如果我们分别用大写字母 $\mathbf{T}$ 和 $\mathbf{F}$ 代表真值"真"和"假",那么,一个合取的真值由其合取支的真值确定的情形,可以用"真值表"的方式更简明地刻画如下:

\begin{center}
\begin{tabular}{|ccc|}
\hline
$p$ & $q$ & $p \cdot q$ \\
\hline
T & T & T \\
T & F & F \\
F & T & F \\
F & F & F \\
\hline
\end{tabular}
\end{center}

该真值表可看做是圆点符号的定义,因为它表明了在每种可能情形下, 303 $p \cdot q$ 所拥有的真值。

我们将发现用大写字母缩写简单陈述很方便。为此,我们一般用一个有助于我们记住它所缩写的那个陈述的字母。于是,我们把"Charlie's neat and Charlie's sweet"(查理是整洁的并且查理是可爱的)缩写为 $N \cdot$ $S$ 。(1)在自然语言中,通过在两个谓项之间加"和"而不重复主项,可以使得合取支有相同主项的那些合取更简明甚或更自然。譬如,"拜伦是一个伟大的诗人并且拜伦是一个伟大的冒险家"就可以写成"拜伦是一个伟大的诗人和伟大的冒险家"。我们把后者看做和前者一样表示了同样的陈述,并且把它们无差别地符号化为 $P \cdot A$ 。同样,在自然语言中,如果一个合取的所有合取支都有相同的谓项,该合取通常被写成在两个主项之间加"和"而不重复谓项。例如,"刘易斯是一个著名的探险家并且克拉克是一个著名的探险家"可以写成"刘易斯和克拉克是著名的探险家"。这两种表述中的任何一个都可以符号化为 $L \cdot C$ 。

正如圆点号的真值表定义所表明的,一个合取是真的,当且仅当,它的合取支都是真的。但语词 and("和"、"并且")还有另外一种用法,其指谓的不只是(真值函项)陈述,还有"随之而来"的意味,即时序关联。例如,陈述"琼斯从纽约进入该国并且直接赶往芝加哥"是有意义的且可能是真的,而陈述"琼斯直接赶往芝加哥且从纽约进入该国"则几乎

\footnotetext{(1)在汉语中可采用汉语拼音首位字母的方式。
}不可理解。"他脱了鞋并且上了床"和"他上了床并且脱了鞋"之间也有很大的区别。\cite{grice1975} 对这样例子的更深人的把握,就需要一个不同于真值函项联结词用法的特殊符号。

请注意,自然语言语词"但是"、"还"、"也"、"仍然"、"尽管"、"然而"、"此外"、"虽然如此"等,甚至逗号和分号都可以用来把两个陈述联结成一个复合陈述,在合取的意义上来说,它们都可以用圆点符号表示。

\subsection{否定}
在自然语言中,一个陈述的否定(或拒斥、否认)的形成通常是在原陈述前加一个"并非"。或者可以通过给一个陈述加一个前(后)缀"这是假的"或"事情并非如此",来表达该陈述的否定。通常用符号"~" (叫做"波浪号"或"波形号")来表示一个陈述的否定。例如,若用符号 M 表示陈述"所有人都是有死的",则陈述"并非所有人都是有死的"、 "有的人不是有死的"、"所有人都是有死的是假的",以及"情况并非是所 804 有人都是有死的"等都可以无差别地符号化为 $\sim M$ 。更一般的,如果 $p$ 是一任意陈述,则它的否定可写为 $\sim p$ 。显然,波浪号是一个真值函项算子。任何真陈述的否定都是假的,任何假陈述的否定都是真的。这一事实可以用真值表简明地刻画如下:

\begin{center}
\begin{tabular}{|cc|}
\hline
$P$ & $\sim P$ \\
\hline
T & F \\
F & T \\
\hline
\end{tabular}
\end{center}

这个真值表可以看做是否定符号"~"的定义。

\subsection{析取}
在自然语言中,两个陈述的析取(或选言)是通过在它们中间插入语词"或"形成的。如此结合的两个分支陈述叫"析取支"(或"选声支")。

自然语言语词"或"很模糊,它有两个相关但可区分的含义。其中一个含义可以用陈述"保险金会因生病或失业而被取消"为例来说明。这里的含义显然是,不仅生病的人和失业的人没有保险金,而且那些既生病又失业的人也没有保险金。"或"的这种含义叫做\textbf{弱的}或\textbf{相容的}含义。当某一个析取支为其或者两个析取支都为真时,该相容析取式是真的;仅当两个析取支均为假时,这两个析取支构成的相容析取式是假的。相容意义上的"或"有"两者之一,可能两者都"之意。保险单里的这种精确含义与

合同和其他法律文本中的一样,可以用词组"和/或"给予明晰表达。\\
语词"或"也可以用做强的或不相容的含义,此时其含义不是"至少一个",而是"至少一个且至多一个"。如果餐馆的菜单上列有"沙拉或甜点",很清楚,它的意思是说,根据所标的就餐价格,就餐者可以点一种或另外一种,但不能两者都点。在保险单里要表达"或"的不相容的精确含义,通常要加上词组"二者不可得兼"。

我们把两个陈述的相容析取解释为断言至少其中有一个是真的,把它们的不相容析取解释为,断言至少其中有一个为真,但并非两者都为真。注意,这两种析取的含义有一部分是共同的。这部分共同含义——至少有一个析取支为真——是相容的"或"的全部含义,是不相容的"或"的含义的一部分。

尽管在现代自然语言中析取的表述很模糊,但在拉丁文中并不模糊。对应于上述"或"的两种不同含义,拉丁文有两个不同的语词。拉丁语词 vel 指谓弱的或相容的析取,aut 对应强的或不相容意义上的语词"或"。习惯上用 vel 的第一个字母来代表弱的、相容意义上的"或"。如果 $p$ 和 $q$是任意两个陈述,它们的弱的或相容的析取写为 $p \vee q$ 。相容析取符号 (叫"楔劈号",有时也叫做"$\vee$ 形号")也是一个真值函项联结词。一个弱析取为假,仅当它的两个析取支均为假。我们可以用真值表把楔劈号定义如下:

\begin{center}
\begin{tabular}{|ccc|}
\hline
$p$ & $q$ & $p \vee q$ \\
\hline
T & T & T \\
T & F & T \\
F & T & T \\
F & F & F \\
\hline
\end{tabular}
\end{center}

本节所举的第一个样本论证就是一个析取三段论\cite{boole1854}:

那个盲囚戴红帽子或者那个盲囚戴白帽子。\\
那个盲囚没戴红帽子。\\
因此,那个杗囚戴白帽子。

其形式特征可以描述为:第一个前提是一个析取;第二个前提是第一个前提的第一个析取支的否定;结论与第一个前提的第二个析取支一样。很显

然,无论对语词"或"作何种解释,即不管是相容析取还是不相容析取,如此定义的析取三段论都是有效的。\cite{russell1903} 既然像析取三段论这样的以析取为前提的典型有效论证,无论对语词"或"作何种解释都是有效的,那么,我们可以简单地把语词"或"翻译为逻辑符号"$V$",而不管语词"或"采取何种含义。一般的,只有通过对上下文进行严格考察,或明确追问说话者或写作者,才能发现其采取的是何种含义。如果我们约定把语词 "或"的任意一次出现都当做相容的,那么,这个通常难以解决的问题就可以避免。另一方面,如果通过附加词组"二者不可得兼"的方式,明确地表达了是不相容析取,那么,正如即将见到的,我们有符号方法来描述这种附加意义。

在自然语言中,当两个析取支有同样的主项或谓项时,用"或"来压缩它们的析取表述,而不必重复这两个析取支的公共部分,这是很自然的。例如,"或者史密斯是所有者或者史密斯是管理者"可以同等好地表述为"史密斯或是所有者或是管理者",并且两者中的任何一个都可以合适地符号化为 $O \vee M$ 。"或者瑞德有罪或者巴奇有罪"通常被陈述为"瑞德或者巴奇有罪",它们都可以符号化为 $R \vee B$ 。

语词"除非"(unless)通常用来形成两个陈述的析取。例如,"除非你努力学习,否则你考不好"可正确地符号化为 $P \vee S$ 。原因在于我们用 "除非"意指,如果一个命题不是真的,则另一个会是真的。上面的例子可以理解为"如果你不努力学习,你就会考不好"——这正是析取的要义,因为它断言其中一个析取支是真的,由此,如果其中一个是假的,则另外一个必定是真的。当然,你也可能努力学习了但考得不好。

然而语词"除非"有时也被用来传达比这更多的信息。它的意思可以是:一个或另一个命题是真的但并非两者都是真的。也就是说,"除非"意指不相容析取。例如,杰里米•边沁(Jeremy Bentham)写道:"政治上好的东西不可能在道德上是坏的,除非对大数目来说是好的算术规则,对小数目来说不好。"\cite{hume1748} 在此,作者的意思确实是说,两个析取支中至少有一个是真的,但他显然也暗示它们不能两者都真。"除非"的其他用法有点含混。当我们说,"野餐将举行,除非下雨"(或者,"除非下雨,野餐将举行"),我们的意思当然是,如果不下雨,将举行野餐。但我们是否有如果下雨就不举行野餐这样的意思呢?这是不清楚的。在这里和其他地方一样,把每个析取当成弱的或相容的是明智的做法;"除非"最好简单

地用楔劈号(V)来符号化。

\subsection{标点符号}
在自然语言中,要使复杂陈述意义明确,标点符号是必需的。若没有大量不同的标点符号的使用,许多句子就会非常含混。譬如,给"The teacher says John is a fool"加不同的标点符号,它就会有很不相同的含义。 ${ }^{(1)}$ 有些语句加上标点才可以理解,如"Jill where Jack had had had had had had had had had had had the teacher's approval"。在数学中,标点符号也同样必要。在没有特别约定的情况下, $2 \times 3+5$ 不能确定指称某个特定的数,而在使用标点清楚地表明其成分如何组合的情形下,$(2 \times 3)+5$ 指称 $11,2 \times(3+5)$ 指称 16 。为了避免歧义和使意义明确,数学中的分组符号以圆括号()、方括号[]和大括号 \textbackslash {\} 等形式出现。 () 用来组合基本符号,[]用来组合包含圆括号的表达式,\textbackslash {\} 用来组合包含方括号的表达式。

在符号逻辑语言中,分组标点符号——圆括号、方括号、大括号——也是同样基本的。因为在逻辑中,复合陈述自身通常复合成一些更复杂的陈述。例如,$p \cdot q \vee r$ 是含混的:它可能意指 $p$ 与 $q$ 和 $r$ 的析取的合取,或者意指这样一个析取,其第一个析取支是 $p$ 和 $q$ 的合取,第二个析取支是 $r$ 。通过把公式加标点为 $p \cdot(q \vee r)$ 或 $(p \cdot q) \vee r$ ,我们可以区分这两种不同含义。不同标点方式所产生的差别,可以通过考察 $p$ 为假,$q$ 和 $r$都为真的情形看出。在这种情形中,第二个加标点的公式是真的(因为它的第二个析取支是真的),而第一个公式是假的(因为它的第一个合取支是假的)。在此,标点的不同导致了真和假的区别,因为不同的加标点方式会对含混的 $p \cdot q \vee r$ 赋不同的真值。

语词"either"(或者)在英语中有很多不同的意义和用法。在语句 "There is danger on either side"(两边都有危险)中,它有合取的力量。但它更常用来引入析取式的第一个析取支,如"Either the blind prisoner has a red hat or the blind prisoner has a white hat"(或者那个盲囚戴红帽子或者那个盲囚戴白帽子)。在此,它有助于语句修辞上的平衡,但并不影响语句的意义。"either"最重要的用法其实是给复合陈述加标点。例如,语句"The organization will meet on Thursday and Anand will be

\footnotetext{(1)此句可分别标点为:"The teacher says,John is a fool"(那个教师说约翰是優瓜)和 "The teacher,says John,is a fool"(约輸说那个教师是便瓜)。
}
elected or the election will be postponed"(那个组织星期四将开会并且安纳德会当选或者选举被推迟)是有歧义的,可以通过把"either"放在该语句的开头,或者把它插在名字"Anand"之前以消除歧义。在符号语言中,这种加标点的作用是通过加括号的方式实现的。前一段所讨论的那个含混公式 $p \cdot q \vee r$ 恰与刚才所考察的这个含混语句相对应。该公式的两种不同的加标点方式可与这个语句的两种不同的加标点方式相对应,而该语句的两种加标点方式是通过"either"的两种不同插人实现的。

析取的否定通常是用词组"不一也不"形成的。因此,陈述"或者费尔莫尔或者哈定是最伟大的美国总统"与陈述"费尔莫尔不是最伟大的美国总统,哈定也不是"矛盾。这个析取陈述可以符号化为 $F \vee H$ ,其否定或者是 $\sim(F \vee H)$ ,或者是 $(\sim F) \cdot(\sim H)$ 。(这两个符号公式的逻辑等价将在 8.5 节讨论。)应该清楚的是,否定断言两个陈述至少一真的析取式,要求把两个析取支都断言为假。

语词"两者都"在逻辑标点上扮演着重要角色,值得给予仔细的关注。正如上面所提到的,当我们说"杰玛和德勒克两者都不……"时,我们是说"杰玛不……德勒克也不……";我们是对他们每一个都进行否定。但当我们说"杰玛和德勒克并非两者都……"时,说的却是某件非常不同的事,我们是在对他们共同组成的对子进行否定,说的是"他们两者都……情况并非如此"。这种差别是非常根本的。在日常句子中,当"两者都"放在不同的地方时,会产生完全不同的意义。考虑下面语句意义的重要差别:

杰玛和德勒克不会两者都当选。\\
杰玛和德勒克两者都不会当选。

第一个语句否定的是合取 $J \cdot D$ ,可以符号化为 $\sim(J \cdot D)$ 。第二个语句是说他们中的每一个都不会当选,可以符号化为 $\sim(J) \cdot \sim(D)$ 。只需改变两个语词"两者都"和"不"的位置就改变了所断言的东西的逻辑力量。

当然,"两者都"并不总是扮演这种角色;有时只用它来增强语气。我们说"刘易斯和克拉克两者都是伟大的探险家",只是以之更强调地陈述"刘易斯和克拉克是伟大的探险家"所言说的东西。但在进行逻辑分析时,必须非常小心地确定"两者都"的标点符号作用。

为简化起见,即为了减少所需的括号数量,作如下约定是很便利的:在任意公式中,否定符号将被理解为施加于标点符号所管辖的最小陈述。没有这种约定,公式 $\sim p \vee q$ 是含混的,它意谓 $(\sim p) \vee q$ ,或者 $\sim(p \vee$ $q)$ 。但采用上述约定,其意指的就是备选者中的第一个,波浪号只能(根据约定)施加于第一个分支 $p$ ,而不是更大的公式 $p \vee q$ 。

为符号语言建立一套标点符号,不仅可以用来表述合取、否定和弱析取,而且也能够表述不相容析取。 $p$ 和 $q$ 的不相容析取式,断言它们当中至少有一个是真的,但并非两者都为真,可以简单地刻画为 $(p \vee q) \cdot \sim(p \cdot q)$ 。

任何仅用真值函项联结词——如圆点、波浪号和楔劈号——从简单陈述构造而成的复合陈述的真值,都完全由组成它的简单陈述的真或假确定。只要知道简单陈述的真值,它们的任何真值函项复合体的真值就很容易计算。在处理这样的复合陈述时,我们总是从它们最内部的组成分支开始,然后逐步外推。例如,设 $A$ 和 $B$ 都是真陈述且 $X$ 和 $Y$ 都是假陈述,即可计算复合陈述 $\sim[\sim(A \cdot X) \cdot(Y \vee \sim B)]$ 的真值如下:因为 $X$ 为假,故 $A \cdot X$ 为假,从而否定式 $\sim(A \cdot X)$ 为真;因为 $B$ 为真,故它的否定~ $B$ 为假,又因为 $Y$ 也为假,故 $Y$ 和 $\sim B$ 的析取 $Y \vee \sim B$ 亦为假;加方括号的公式 $[\sim(A \cdot X) \cdot(Y \vee \sim B)]$ 是一个真陈述和一个假陈述的合取,因此是假的;由此,它的否定即原整个陈述是真的。这样一种逐步程序,使得我们总能根据一个复合陈述的分支的真值来确定它的真值。

在某些情形下,即使我们不能确定一个或多个简单分支陈述的真或假,我们也能确定一个真值函项复合陈述的真值。首先通过假定某简单分支陈述为真,计算出该复合陈述的真值,然后假定该同一简单分支陈述为假,计算出该复合陈述的真值,对其真值未知的每个分支施行同样的步骤,我们就可以做到这一点。如果这些计算对被考察的复合陈述产生同样的真值,我们不必先确定它的分支的真值,就可以确定该复合陈述的真值,因为我们知道真值不是真就是假。

\begin{center}
\fbox{\parbox{0.95\textwidth}{
\textbf{本节要点}
\begin{itemize}
\item \textbf{简单陈述}与\textbf{复合陈述}的区别:
  \begin{itemize}
  \item 简单陈述不包含其他陈述作为分支
  \item 复合陈述包含其他陈述作为分支
  \end{itemize}
\item \textbf{合取}($\cdot$)的特点:
  \begin{itemize}
  \item 用圆点"$\cdot$"表示
  \item 当且仅当两个合取支都为真时,合取才为真
  \item 是真值函项复合陈述
  \end{itemize}
\item \textbf{否定}($\sim$)的特点:
  \begin{itemize}
  \item 用波浪号"$\sim$"表示
  \item 真陈述的否定为假,假陈述的否定为真
  \end{itemize}
\item \textbf{析取}($\vee$)的特点:
  \begin{itemize}
  \item 用楔劈号"$\vee$"表示
  \item 相容析取:当至少一个析取支为真时,析取为真
  \item 不相容析取:恰好一个析取支为真时,析取为真
  \end{itemize}
\item 标点符号在逻辑中的重要性:
  \begin{itemize}
  \item 消除歧义和不明确性
  \item 通过括号明确指明复合陈述的结构
  \item 允许表示复杂的逻辑关系
  \end{itemize}
\end{itemize}
}}
\end{center} 