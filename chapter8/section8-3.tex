\section*{8.3 条件陈述与实质蕴涵}
当把语词"如果"放在第一个陈述之前,把语词"那么"放在第一个和第二个陈述之间来结合两个陈述时,如此构成的复合陈述就是一个条件陈述(也叫"假言陈述"、"蕴涵"或"蕴涵陈述")。在一个条件陈述中,跟在"如果"后面的分支陈述叫前件(或"蕴涵者",偶尔也叫"前式"),跟在"那么"后面的分支陈述叫后件(或"被蕴涵者",偶尔也叫"后式")。例如,"如果琼斯先生是那个司闸员的邻居,那么琼斯先生挣的钱是那个司闸员的三倍"是一个条件陈述,其中,"琼斯先生是那个司闸员的邻居"是前件,"琼斯先生挣的钱是那个司闸员的三倍"是后件。

一个条件陈述断言在其前件为真的任何情形下,它的后件也是真的。它并不断言其前件为真,而只是断言如果其前件为真,其后件也为真。它也并不断言其后件为真,而仅仅断言它的后件会为真,如果前件为真的

话。一个条件陈述的基本含义,是断言其前后件之间的某种关系以特定次序成立。要理解一个条件陈述的含义,我们必须理解何为蕴涵关系。\\
"蕴涵"一词不止一个含义。我们已经看到,在引进一个特殊的逻辑符号来表示日常语词"或者"的某个单一含义之前,区分它的不同含义是有用的。要是我们不这样做,日常语言的含混性就会影响我们的逻辑符号系统,妨碍我们达到所欲获得的明晰性和精确性。在我们把一个特殊的逻辑符号引入这种联系中之前,区分"蕴涵"或"如果一那么"的不同含义亦同样有用。

考查下面的四个条件陈述,它们每个都断言一种不同类型的蕴涵,都对应于一种不同含义的"如果…那么":

A.如果所有人都有死且苏格拉底是人,那么苏格拉底有死。

B.如果莱士里是单身汉,那么莱士里是未婚的。\\
C.如果把这张蓝色的石蕊纸放在酸液中,那么这张蓝色的石蕊纸会变红。

D.如果斯塔德輸掉了这场比赛,那么我就吞下我的帽子。

即使随意地观察一下这四个条件陈述也会发现,它们具有非常不同的类型。 A 的后件乃由它的前件逻辑地推出,而 B 的后件是根据其前件中的术语"单身汉"的定义而得来,而"单身汉"的定义就是未婚男人。C 的后件不是仅根据逻辑或其词项的定义从其前件推出,这种联系必须经验地发现,因为这里所陈述的蕴涵是因果关系。最后,D的后件既不是根据逻辑或定义从前件推得,也没有涉及因果性定律——就这个词的通常意义来说。大多数因果性定律,臂如物理学和化学中发现的那些定律,描述的是世界发生了什么,而不管人的希望或欲求如何。当然,没有这样一种定律和陈述 D 相联系。这个陈述表述的是说话者在某种特定的情形下以特定的方式行事的决策。

可见,这四个条件陈述的不同之处,就在于每个断言了其前件和后件之间的一种不同类型的蕴涵关系。但它们并非完全不同,它们所断言的都是蕴涵的类型。那么,它们是否存在任何可识别的共同含义,即是否存在尽管可能不是其中任何一个的完整含义,但是这些公认的不同种类蕴涵所

共有的部分含义呢?\\
关于探求共同的部分含义的重要性,我们可以回想一下对日常语词 "或"进行符号刻画的过程。那时我们是如下进行的。首先,在对比相容和不相容析取时,我们强调"或"的两种含义之间的区别。我们注意到,两个陈述的相容析取的意思是说,它们当中至少一个为真。不相容析取的意思是说,它们当中至少一个为真,但不是两者都为真。其次,我们注意到这两种类型的析取有一个共同的部分含义。这个部分的共同含义,即至少有一个析取支为真,被看做是弱的、相容的"或"的整个含义,是强的、相容的"或"的含义的一部分。然后,我们引入特殊符号"V"来表达这个共同的部分含义(它是"或"的弱意义上的整个含义)。最后,我们注意到,表达共同的部分含义的符号刻画也是对语词"或"在下述意义上的合适翻译,即可以把析取三段论作为一个有效的论证形式保留下来。我们承认把不相容的"或"翻译成符号"V",忽略和丢掉了它的部分含义。但由这种翻译所保留的那个部分含义,是析取三段论继续成为一个有效论证必需的全部东西。既然析取三段论是我们这里所关注的涉及析取的典型论证,那么,语词"或"的这种部分翻译——在某些情形,可以从它的"完全的"或"全部的"含义中抽取出来——对我们目前的目的是完全合适的。

现在,我们希望以同样的方式抽取日常语言辞组"如果一那么"的含义。第一步已经完成:我们已经强调了短语"如果一那么"对应于四种不同蕴涵的四种意义之间的区别。现在准备做第二步,即发现一个至少是所有这四种不同类型的蕴涵的含义的一部分的那种意义。

要解决这个问题,可先看什么情形足以确立一个给定条件陈述的假。在什么情形下,我们会同意下面的条件陈述为假呢?

\begin{displayquote}
如果把这张蓝色的石蘂纸放进那种溶液中,那么这张蓝色的石蕊纸会变红。
\end{displayquote}

这个条件陈述并未断言任何一张蓝色的石蕊纸实际上被放进了这种溶液中,或任何一张蓝色的石蕊纸实际上变红了,认识到这一点是很重要的。它仅仅断言如果把这张蓝色的石総纸放进那种溶液中,那么这张蓝色的石蕊纸会变红。如果这张蓝色的石䓗纸实际上被放进这种溶液中,并且它没

变红,就证明该陈述是假的。可以说,当一个条件陈述的前件为真时,就获得一个关于该条件陈述的虚假性的严峻检验,因为如果它的后件为假且前件为真,该条件陈述本身就被证明为假。

对任一条件陈述"如果 $p$ 那么 $q$"来说,如果已知合取 $p \cdot \sim q$ 为真,也就是说,如果它的前件为真且后件为假,则可知该条件陈述为假。而若一个条件陈述为真,则上面所示合取式必定为假,也就是说,它的否定 $315 \sim(p \cdot \sim q)$ 必定为真。换句话说,对任何为真的条件陈述"如果 $p$ 那么 $q$"而言,它的前件和后件的否定的合取的否定,即 $\sim(p \cdot \sim q)$ ,必定也为真。据此,我们可把~$(p \cdot \sim q)$ 当做"如果 $p$ 那么 $q$"的含义的一部分。

每个条件陈述都意谓否定其前件为真且后件为假,但这不必是其整个含义。前面的 A 那样的条件陈述还断言了其前件和后件之间的一种逻辑联系,B 那样的条件陈述还断言了一种定义性联系,C 那样的条件陈述还断言了一种因果性联系,而 D 那样的条件陈述则还断言了一种决策性联系。但不管一个条件陈述断言的是何种蕴涵,它的一部分含义是对其前件和后件的否定的合取的否定。

现在,我们引进一个特殊的符号来表达短语"如果一那么"的这种共同的部分含义。通过以 $p \supset q$ 缩写 $\sim(p \cdot \sim q)$ ,我们来定义新符号"つ" (叫"马蹄号")。符号"つ"的确切含义可以用真值表方法揭示如下:

\begin{center}
\begin{tabular}{|l|l|l|l|l|l|}
\hline
$p$ & $q$ & $\sim q$ & $p^{\cdot \sim q}$ & $\sim(p \cdot \sim q)$ & $p$ つ $q$ \\
\hline
T & T & F & F & T & T \\
\hline
T & F & T & T & F & F \\
\hline
F & T & F & F & T & T \\
\hline
F & F & T & F & T & T \\
\hline
\end{tabular}
\end{center}

其中,前两列是导引列,它们只是列出 $p$ 和 $q$ 真值组合的所有可能情形。第三列据第二列得来,第四列据第一和第三列得来,第五列据第四列得来,根据定义,第六列与第五列真值相同。

符号"コ"不应被看成是指谓"如果一那么"的某种含义,或代表 (上列蕴涵类型中的)某种蕴涵关系。那是不可能的,因为没有单一的 "如果一那么"的含义,而是有几个含义。不存在该符号所刻画的单一蕴涵关系,而是有几种不同的蕴涵关系。故符号"つ"不应被看成是代表 "如果一那么"的所有含义。这些含义各不相同,用单个逻辑符号来缩写

所有这些含义的任何企图都会使符号变得含混,正如日常语言辞组"如果一那么"或"蕴涵"一样含混。符号"つ"是完全不含混的。 $p \supset q$ 缩写的就是 $\sim(p \cdot \sim q)$ ,它的含义包含在被探讨的各种蕴涵的含义之中,但它并不构成它们中任何一个的完整含义。

既然读 $p \supset q$ 的一种方便方式是"如果 $p$ 那么 $q$",我们也可以把符号 "D"看成表示了另一种蕴涵,而且这样做是很有好处的。但它不是与前面提到过的任何一种蕴涵相同的蕴涵,它被逻辑学家叫做实质蕴涵。给出这个特殊的名称,就是承认它是一个特殊概念,不应该把它和其他更常见类型的蕴涵相混淆。

日常语言中的所有条件陈述并非都必须断言前面所讨论的四种蕴涵之一。实质蕴涵实际上也是日常话语中所断言的第五种蕴涵。考虑这样一个评论:"如果希特勒是军事天才,那么我是猴子的叔叔"。很显然,它不是断言逻辑的、定义性的或因果性的蕴涵。它也不表达决策性蕴涵,因为说话者并没有能力使后件为真。这里的前后件之间没有"真正的联系",不管是逻辑的、定义性的还是因果性的。这种条件陈述经常被当做一种强调或幽默的方法来使用,它否定的是其前件,其后件通常是一个滑稽的、显然为假的陈述。既然没有任何为真的条件陈述有这样的真前件和假后件,那么,肯定这样一个条件陈述就意味着否定它的前件为真。上述条件陈述的完整含义就是,只要"我是猴子的叔叔"为假,即可否定"希特勒是军事天才"为真。既然前者明显为假,该条件陈述必被理解为否定后者。

这里的关键在于,实质蕴涵没有表明前后件之间的"实在关联",实际上,它所断言的仅仅是并非后件为假时前件为真。请注意:实质蕴涵符号像合取和析取符号一样,是真值函项联结词。它可用真值表定义如下:

\begin{center}
\begin{tabular}{|ccc|}
\hline
$p$ & $q$ & $p \supset q$ \\
\hline
T & T & T \\
T & F & F \\
F & T & T \\
F & F & T \\
\hline
\end{tabular}
\end{center}

正如这个真值表定义所表明,马蹄符"つ"有几个乍看起来很奇怪的特征:假前件实质蕴涵真后件的断言是真的;假前件实质蕴涵假后件的断言也是真的。这种表面的怪异可以由下面的探讨得到部分驱散。因为数 2 比数 4 小 (用符号表示为 $2<4$ ),可以推出任何小于 2 的数都小于 4 。条件公式:

如果 $x<2$ 那么 $x<4$\\
对任一 $x$ 都是真的。我们来看数 $1 、 3$ 和 4 ,依次以它们中的每一个代人前述条件公式的数字变项 $x$ ,可以观察到如下结果:

如果 $1<2$ 那么 $1<4$\\
在这种情形下,前后件都是真的,该条件陈述当然也是真的。\\
如果 $3<2$ 那么 $3<4$\\
在这种情形下,前件为假且后件为真,该条件陈述当然也是真的。\\
如果 $4<2$ 那么 $4<4$\\
在这种情形下,前件和后件都是假的,但该条件陈述仍然是真的。这三种情形分别对应于马蹄符"コ"的真值表定义中的第一、第三和第四行。可见,在一个条件陈述的前后件皆为真、前件为假且后件为真或前后件皆为假时,该条件陈述应该为真,这一点并不特别令人奇怪或惊讶。当然,没有小于 2 且不小于 4 的数,也就是说,没有其前件为真且后件为假的真条件陈述。这恰好是"つ"的真值表定义所表明的。

现在,我们打算把词组"如果——那么"的任何一次出现翻译成逻辑符号 "つ"。这种处理方式的意思是说,在把条件陈述翻译成符号时,我们把它们都只看做是实质蕴涵。当然,大多数条件陈述断言,在前后件之间不只实质蕴涵成立。因此,这种处理方式即意味着在把一个条件陈述翻译成符号语言时,应该忽略、撇开或"抽掉"它的部分含义。怎样辩护这种处理方式呢?

前面对用符号"$V$"来翻译相容和不相容析取这个处理方式的辩护是基于这样的理由:即使忽略附着在不相容析取"或"之上的附加含义,析取三段论的有效性也得到了保留。我们现在提议用符号"$\supset$"把所有的条件陈述仅翻译成实质蕴涵,可用完全同样的方式得到辩护。许多论证包含各种不同类型的条件陈述,但是,即便忽略这些论证的条件陈述的附加含义,我们所关注的一般类型的有效论证的有效性也都得到了保留。当然,这一点还需要证明,这是本章下一节的主题。

条件陈述可用多种不同方式表述。如下陈述:

如果他有一个好律师,那么他会被宣判无罪。

可以不用"那么"而被同样适当地表述为:

如果他有一个好律师,他会被宣判无罪。

前件和后件的表述次序可以颠倒,此时"如果"仍应在前件之前:

他会被宣判无罪,如果他有一个好律师的话。

显然,在上面所给的任何一个例子中,语词"如果"可被诸如"一旦"、318 "假如"、"倘若"或"在……条件下"等短语代替,而含义没有任何改变。经措辞调整还可把上述条件陈述表述为:

他有一个好律师蕴涵他会被宣判无罪。

或

他有一个好律师涵衍(entail)他会被宣判无罪。

从主动语态到被动语态的转换伴随着前后件次序的颠倒,可得其逻辑等价表述:

他会被宣判无罪被他有一个好律师所蕴涵(或涵衍)。

上列表述均可符号化为 $L \supset A$ 。\\
必要条件和充分条件的观念提供了条件陈述的其他一些表述形式。对任何一个特定事件来说,它的出现需要有许多必要情境。例如,一辆正常的轿车要能行使,油箱里有油,火花塞被校准,油泵能运转等都是必要条件。因此,如果该事件出现,它的出现所必需的每个条件必定都已经得到满足。据此,下述陈述:

油箱里有油是轿车行驶的一个必要条件。

可以同样适当地表述为:

轿车行使仅当它的油箱里有油。

它是如下说法的另一方式:

如果轿车行使,那么它的油箱里有油。

这些表述形式中的任何一个都可以符号化为 $R \supset F$ 。一般地说,"$q$ 是 $p$ 的必要条件"和"$p$ 仅当 $q$"可以符号化为 $p \supset q$ 。

对某特定情形而言,会有许多备选条件,它们中的任何一个都足以产生该情形。例如,就一个钱包里不止一美元来说,它里面有 101 便士、21个五分镍币、 11 个一角的硬币、 5 个两角五分钱等都是充分条件。如果获得其中的任何一个条件,那个特定的情形就会实现。因此,说"那个钱包里有 5 个两角五分钱是它里面超过一美元的充分条件",与说"如果那个钱包里有 5 个两角五分钱,那么它里面超过一美元"是一样的。一般的, "$p$ 是 $q$ 的充分条件"被符号化为 $p \supset q$ 。

如果 $p$ 是 $q$ 的一个充分条件,我们就有 $p \supset q$ ,并且 $q$ 必定是 $p$ 的一个必要条件。如果 $p$ 是 $q$ 的一个必要条件,我们就有 $q \supset p$ ,并且 $q$ 必定是 $p$ 的一个充分条件。因此,如果 $p$ 是 $q$ 的必要且充分条件,那么,$q$ 是 $p$的充分且必要条件。

并非每个含有"如果"(或类似语词)的陈述都是条件陈述。下列陈述中没有一个是条件陈述:"冰箱里有食品,如果你想吃","您的桌子准备好了,如果您乐意的话","假如感兴趣,有个消息给你","即便没得到允许,会议也会举行"。特定语词的出现与否决不是决定性的。在每种情形下,必须先理解给定语句的含义,然后用符号公式重新表述这种含义。

语词"如果"和"不确定的"之间没有必然的或逻辑的联系,尽管经常有这样一种说法:跟在语词"如果"后面的东西有点不确定。这一点可由下面的逸事所例示:

有一次,乔治•伯纳德•肖给温斯顿•丘吉尔送了两张他的新剧的首演式的票,附言"带一个朋友——如果你有的话";对

对此,丘吉尔回复说他忙于出席别的首演式,但他会很感激第二场演出的票,"如果有这样一张票的话"${ }^{[9]}$ 。 