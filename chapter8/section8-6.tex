\section{逻辑等价}

\begin{quotation}
本节引入逻辑等价的概念,探讨其与实质等值的区别以及在逻辑推理中的重要应用。通过德摩根定理和实质蕴涵的重要等价关系,我们能够在复杂的逻辑分析中进行命题转换,简化论证过程,并更深入地理解不同逻辑联结词之间的内在联系。
\end{quotation}

本节不是要引进一种新的联结词,而是要引进一种非常重要且十分有用的新关系,这种关系比刚才讨论过的任何一个真值函项联结词都要复杂些。

当陈述有相同的真值时,它们是实质等值的。因为两个实质等值的陈述或者都是真的,或者都是假的,既然假前件(实质)蕴涵任何陈述,真后件被任何陈述所(实质)蕴涵,我们就能看出它们必定彼此(实质)蕴涵。因此,我们可以把三杜号"三"读做"当且仅当"。

但确定为实质等值的陈述并不能互相替换。知道它们实质等值,我们只是知道它们的真值相同。陈述"木星比地球大"和陈述"东京是日本的首都"是实质等值的,因为它们都是真的,但我们显然不能用一个替换另一个。同样,陈述"所有蜘蛛都有毒"和陈述"所有蜘蛛都无毒"都是假的,所以它们实质等值,它们当然也不能彼此替换!

但在很多情况下,我们必须表示那种允许相互替换的关系。两个陈述可以在比实质等值强得多的意义上等值;它们可以在真值相同的同时,意义(meaning)也相等。如果它们有同样的意义,那么,与它们中的某个相结合的任何命题,也可以和另一个结合。没有一一也不可能有一一这样一种情形,即这些陈述中的一个是真的,而另一个是假的。这种非常强的意义上等值的陈述,我们称之为\textbf{逻辑等价}。

当然,任何两个逻辑等价的陈述也是实质等值的,因为它们显然必须有相同的真值。无疑,如果两个陈述逻辑等价,那么,它们在所有情形下都实质等值一一由此可获得逻辑等价的简短而有力的定义:若两个陈述的实质等值陈述是一个重言式,则两个陈述逻辑等价。这就是说,它们有同样的真值这样一个陈述自身必然是真的。这就是我们为什么要用三杜号的上方加一个 T,即"坖",来表示这种很强的逻辑关系。"$\xlongequal{T}$"表示的是这种逻辑关系的本质:两个逻辑等价陈述的实质等值式是一个重言式。因为实质等值式是一个"双条件陈述"(两个陈述互相蕴涵),所以,我们可以把这个逻辑等价符号"乍"视为表示一个重言的双条件陈述。

某些运用得非常普遍的简单逻辑等价式可以使这种关系及其威力得以彰显。 $p$ 和 $\sim \sim p$ 意谓同样的东西,这是一个常识;"他意识到那个困难"和"他不是没有意识到那个困难"是两个有同样意义的陈述。要言之,这两个表述中的任何一个都可以由另一个替换,因为它们说的是同一件事。这个双重否定原则的真对所有人来说都是显然的,这一点可以用真值表来展示。在此,两个陈述形式的实质等值式被表明是一个重言式:

\begin{center}
\begin{tabular}{|cccc|}
\hline
$p$ & $\sim p$ & $\sim \sim p$ & $p \stackrel{T}{\equiv} \sim \sim p$ \\
\hline
T & F & T & T \\
F & T & F & T \\
\hline
\end{tabular}
\end{center}

这个真值表证明 $p$ 和 $\sim \sim p$ 是逻辑等价的。因此,这个非常有用的逻辑等价式,即\textbf{双重否定式},可以符号化为:

$$
p \stackrel{\mathrm{~T}}{=} \sim \sim p
$$

实质等值和逻辑等价这两者之间的差别很大,并且也很重要。前者是一个真值函项联结词,即"三",它为真或为假仅取决于它所联结的分支的真或假;但后者,即逻辑等价"坖",不只是一个联结词,它还表达两个陈述之间某种非真值函项的关系。两个陈述逻辑等价,仅当它们绝对不可能有不同的真值。但如果它们总是有相同的真值,那么,逻辑等价陈述必定有同样的意义。在此种情形下,它们可以在任何真值函项语境中互相替换而不改变在该语境中的真值。反之,如果两个陈述仅仅碰巧有相同的真值,甚至它们之间没有实际的联系,那么,它们就只是实质等值的。只是实质等值的陈述当然不能互相替换!

有两个著名的逻辑等价式(即逻辑地真的双条件陈述)非常重要,因为它们表示了合取、析取及它们的否定之间的相互关系。下面即严格地考察这两个逻辑等价式。

首先,我们用什么来否定一个析取为真呢?任何析取式 $p \vee q$ 只是断言它的两个析取支中至少有一个是真的。若断言其析取支至少有一个为假,并不能与之相矛盾;(要否定它)我们必须断言两个析取支都为假。因此,断言析取 $\boldsymbol{p} \vee \boldsymbol{q}$ 的否定,逻辑地等价于断言 $\boldsymbol{p}$ 的否定和 $\boldsymbol{q}$ 的否定的合取。要在真值表中表明这一点,可以构造双条件陈述 $\sim(p \vee q) \equiv$ $(\sim p \cdot \sim q)$ ,把它放在它自己那一栏的顶端,然后检査它在所有情形下即每一行中的真值。

\begin{center}
\begin{tabular}{|l|l|l|l|l|l|l|l|}
\hline
$p$ & $q$ & $p \vee q$ & $\sim(p \vee q)$ & $\sim p$ & $\sim q$ & $\sim p \cdot \sim q$ & \begin{tabular}{l}
$\sim(p \vee q) \equiv$ \\
$(\sim p \cdot \sim q)$ \\
\end{tabular} \\
\hline
T & T & T & F & F & F & F & T \\
\hline
T & F & T & F & F & T & F & T \\
\hline
F & T & T & F & T & F & F & T \\
\hline
F & F & F & T & T & T & T & T \\
\hline
\end{tabular}
\end{center}

我们看到,无论 $p$ 和 $q$ 的真值如何,这个双条件陈述必定总是真的,因而是一个重言式。因为这个实质等值的陈述是一个重言式,所以可得出这两个陈述逻辑等价的结论。故我们已经证明:

$$
\sim(p \vee q) \stackrel{\mathrm{T}}{=} \sim p \cdot \sim q
$$

同样,由于断言 $p$ 和 $q$ 的合取,就是断言这两者都为真,要与该断言相矛盾,我们只需断定其中至少有一个为假。因此,断定合取 $(p \cdot q)$ 的否定,逻辑地等价于断定 $p$ 的否定和 $q$ 的否定的析取。在真值表中可以用符号表明,双条件陈述 $\sim(p \cdot q) \equiv(\sim p \vee \sim q)$ 是一个重言式。这样一个真值表就证明了:

$$
\sim(p \cdot q) \stackrel{\mathrm{T}}{=} \sim p \vee \sim q
$$

这两个重言的双条件陈述或逻辑等价式,被叫做\textbf{德摩根定理},因为它们是由数学家兼逻辑学家奥古斯塔•德摩根(1806-1871)正式表述出来的。德摩根定理用自然语言可以表述为:

(a) 两个陈述的析取的否定逻辑等价于这两个陈述的否定的合取;\\
(b) 两个陈述的合取的否定逻辑等价于这两个陈述的否定的析取。

这两个徳摩根定理被证明是特别有用的。\\
当我们试图系统处理真值函项联结词时,另一个重要的逻辑等价式非常有帮助。在本章的早些地方(8.3节),我们把实质蕴涵"コ"定义为说 $\sim(p \cdot \sim q)$ 的一种简略方式。也就是说,根据定义,"$p$ 实质蕴涵 $q$"的意思就是,并非 $p$ 为真而 $q$ 为假。在这个定义中,我们可以看到,定义项 $\sim(p \cdot \sim q)$ ,是一个合取的否定。根据德摩根定理,我们知道,任何这种否定逻辑等价于这些合取支的否定的析取;也就是说,我们知道, $\sim(p \cdot \sim q)$ 逻辑等价于 $(\sim p \vee \sim \sim q)$ ;再运用双重否定原则,这个表达式又逻辑等价于~pVq。逻辑等价的表达式意谓同样的事情,因此,马蹄号原来的定义项 $\sim(p \cdot \sim q)$ ,可以用一个更简单的表达式 $\sim p \vee q$ 来替换而不改变其含义。这就给了我们一个非常有用的实质蕴涵定义:$p \supset q$ 逻辑地等价于 $\sim p \vee q$ 。 可用符号写为:

$$
p \supset q \xlongequal{\mathrm{~T}} \sim p \vee q
$$

在表述逻辑陈述和分析论证时,需广泛地依赖实质蕴涵的这一定义。当我们所处理的陈述有相同的核心联结词时,操作起来通常会很简便也更有效力。运用我们刚才所建构的马蹄号的简单定义,即 $p \supset q \xlongequal{\mathrm{~T}} \sim p \vee q$ ,那些以马蹄号为联结词的陈述,可以方便地用那些以楔劈号为联结词的陈述来替换;同样,析取形式的陈述可以用蕴涵形式的陈述替换。在给出演绎论证有效性的形式证明时,这种替换被证明确实非常有用。

\begin{center}
\fbox{\parbox{0.95\textwidth}{
\textbf{本节要点}
\begin{itemize}
\item \textbf{逻辑等价}的特点:
  \begin{itemize}
  \item 表示两个陈述具有相同的意义
  \item A和B逻辑等价当且仅当"A当且仅当B"是重言式
  \item 逻辑等价的陈述可在任何真值函项语境中互相替换
  \end{itemize}
\item \textbf{逻辑等价}与\textbf{实质等值}的区别:
  \begin{itemize}
  \item 实质等值仅表示真值相同(偶然相同也可)
  \item 逻辑等价表示必然具有相同真值(基于意义相同)
  \item 仅实质等值的陈述不能互相替换
  \end{itemize}
\item \textbf{常见逻辑等价式}:
  \begin{itemize}
  \item 双重否定式:$p \xlongequal{\mathrm{~T}} \sim \sim p$
  \item 德摩根第一定理:$\sim(p \vee q) \xlongequal{\mathrm{~T}} \sim p \cdot \sim q$
  \item 德摩根第二定理:$\sim(p \cdot q) \xlongequal{\mathrm{~T}} \sim p \vee \sim q$
  \item 实质蕴涵的等价式:$p \supset q \xlongequal{\mathrm{~T}} \sim p \vee q$
  \end{itemize}
\item 逻辑等价在演绎推理中的应用:
  \begin{itemize}
  \item 允许在论证分析中进行有效的命题转换
  \item 简化复杂论证形式的分析过程
  \item 将一种逻辑联结词转换为另一种联结词
  \end{itemize}
\end{itemize}
}}
\end{center} 