\section{陈述形式与实质等值}

\begin{logicbox}[title=引言]
本节探讨陈述形式、重言式、矛盾式与实质等值的概念及其相互关系。通过理解这些概念,我们能够辨别不同类型的逻辑真理,将复杂的逻辑关系转化为更易于分析的形式,并利用真值函项联结词准确表达命题间的各种关系。
\end{logicbox}

\subsection{陈述形式的理论基础}

现在,我们来明确一下前一节所假定的一个概念,即\logicterm{陈述形式}。以论证和论证形式之间的关系为一方,陈述和陈述形式之间的关系为另一方,这两者是完全平行的。

\begin{theorembox}[title=陈述形式的数学定义]
一个\logicterm{陈述形式}是任何一个含有陈述变元但不含陈述的符号序列,若用陈述代入这些陈述变元——用同一个陈述始终一致地代入同一个陈述变元——其结果是一个陈述。

这个定义具有以下重要特征:
\begin{itemize}
\item \textbf{形式性}:陈述形式抽象掉了具体的语义内容,只保留逻辑结构
\item \textbf{生成性}:每个陈述形式可以生成无穷多个具体陈述
\item \textbf{分类性}:陈述形式为陈述提供了逻辑分类的标准
\item \textbf{可操作性}:陈述形式支持机械化的逻辑操作和分析
\end{itemize}
\end{theorembox}

例如,$p \vee q$ 是陈述形式,因为若用陈述代入变元 $p$和 $q$ ,就会产生一个陈述。由于所产生的陈述是一个析取句,$p \vee q$ 就叫做\logicterm{析取陈述形式}。同样,$p \cdot q$ 和 $p \supset q$ 分别叫做\logicterm{合取陈述形式}和\logicterm{条件陈述形式},$\sim p$ 叫做\logicterm{否定形式}或者\logicterm{否认形式}。

\begin{examplebox}[title=陈述形式的分类体系]
\textbf{基本陈述形式}:
\begin{itemize}
\item 原子形式:$p, q, r, \ldots$
\item 否定形式:$\sim p, \sim q, \ldots$
\end{itemize}

\textbf{复合陈述形式}:
\begin{itemize}
\item 合取形式:$p \cdot q, (p \cdot q) \cdot r, \ldots$
\item 析取形式:$p \vee q, (p \vee q) \vee r, \ldots$
\item 条件形式:$p \supset q, (p \supset q) \supset r, \ldots$
\item 双条件形式:$p \equiv q, (p \equiv q) \equiv r, \ldots$
\end{itemize}

\textbf{混合形式}:$p \cdot (q \vee r), (p \supset q) \vee (r \cdot s), \ldots$
\end{examplebox}

正像某种形式的论证称为该论证形式的代入例一样,具有某种形式的任一陈述称为该陈述形式的\logicterm{代入例}。正像我们判别一个给定论证的特征形式一样,我们把一个给定陈述的\logicterm{特征形式}判别为这样一种陈述形式:通过一致地用不同的简单陈述代入每个不同的陈述变元,就可以从其产生该给定陈述。例如,$p \vee q$就是陈述"那个盲囚戴红帽子或者那个盲囚戴白帽子"的特征形式。

\subsection{逻辑真理的三重分类:重言式、矛盾式与偶真陈述形式}

尽管陈述"林肯是被暗杀的"(记为 $L$ )和"林肯或者是被暗杀的,或者不是"(记为 $L \vee \sim L$ )都是真的,但我们会非常自然地感觉到,它们是"在不同方面"为真,或有"不同种类"的真。同样,尽管陈述"华盛顿是被暗杀的"(记为 $W$ )和"华盛顿既是被暗杀的又不是被暗杀的"(记为 $W \cdot \sim W)$ 这两者都为假,但我们也会非常自然地感觉到,它们也是 "在不同方面"为假,或有"不同种类"的假。

\begin{theorembox}[title=逻辑真理与经验真理的根本区别]
这种直觉上的区别反映了逻辑学中的一个根本性分类:

\textbf{经验真理}:其真值依赖于世界的实际状况,需要通过经验观察来确定。例如,"林肯是被暗杀的"是一个历史事实,其真值取决于历史事件的实际发生。

\textbf{逻辑真理}:其真值完全由逻辑形式决定,独立于世界的实际状况。例如,"林肯或者是被暗杀的,或者不是"无论在什么可能世界中都必然为真。

这种区别在哲学上具有深远意义:它揭示了人类知识的两种不同来源——经验观察和逻辑推理。
\end{theorembox}

陈述 $L$ 为真和陈述 $W$ 为假乃属于历史事实,它们没有逻辑必然性。所有事件都有以不同方式出现的可能,因而像 $L$ 和 $W$ 这样的陈述的真值,必须通过对历史的经验研究才能被发现。而陈述 $L \vee \sim L$ 尽管是真的,但它不是历史地真,而具有\logicemph{逻辑的必然性}:事件不可能如此这般以致使它为假,它的真可以独立于任何经验研究而被知晓。

\subsection{重言式的深入分析}

陈述 $L \vee \sim L$ 是一个\logicterm{逻辑真理},或曰\logicterm{形式真理},其真仅因其形式,它是一个其所有代入例都是真陈述的陈述形式的代入例。

\begin{theorembox}[title=重言式的本质特征]
一个只有真代入例的陈述形式叫\logicterm{重言的陈述形式},或\logicterm{重言式}(Tautology)。重言式具有以下本质特征:

\textbf{1. 逻辑必然性}:重言式在所有可能的情况下都为真,体现了逻辑的必然性。

\textbf{2. 形式性}:重言式的真值完全由其逻辑形式决定,与具体内容无关。

\textbf{3. 先验性}:重言式的真值可以通过纯粹的逻辑分析确定,无需经验观察。

\textbf{4. 信息空虚性}:重言式虽然必然为真,但不提供关于世界的实质信息。
\end{theorembox}

要表明陈述形式 $p \vee \sim p$ 是一个重言式,可构造如下真值表:

\begin{center}
\begin{tabular}{|ccc|}
\hline
$p$ & $\sim p$ & $p \vee \sim p$ \\
\hline
T & F & T \\
F & T & T \\
\hline
\end{tabular}
\end{center}

这个真值表只有一个初始栏或导引栏,因为被探究的形式只含有一个陈述变元。它只有两行,代表了所有可能的代入例。被检验陈述形式下面的那一栏里只有 T ,这表明,它的所有代入例都是真的。任何一个作为重言的陈述形式的代入例的陈述,依据其形式就是真的,其本身被称为\logicterm{重言陈述},亦称为一个\logicterm{重言式}。

\subsection{矛盾式的深入分析}

\begin{theorembox}[title=矛盾式的本质特征]
一个只有假代入例的陈述形式称为\logicterm{自相矛盾的陈述形式},或\logicterm{矛盾式}(Contradiction),它是逻辑地为假的。矛盾式具有以下本质特征:

\textbf{1. 逻辑不可能性}:矛盾式在所有可能的情况下都为假,体现了逻辑的不可能性。

\textbf{2. 形式矛盾性}:矛盾式的假值完全由其逻辑形式决定,与具体内容无关。

\textbf{3. 先验可知性}:矛盾式的假值可以通过纯粹的逻辑分析确定,无需经验检验。

\textbf{4. 爆炸原理}:从矛盾式可以推出任何陈述,这在逻辑学中被称为"爆炸原理"(Principle of Explosion)。
\end{theorembox}

陈述形式 $p \cdot \sim p$ 是自相矛盾的,因为在它的真值表中只有 F 在它下面出现,这表明它的所有代入例都是假的。任何一个作为自相矛盾的陈述形式的代入例的陈述,如 $W \cdot \sim W$ ,依据其形式就是假的,其本身称为\logicterm{自相矛盾的陈述},亦称为一个\logicterm{矛盾式}。

\subsection{偶真陈述形式的深入分析}

\begin{theorembox}[title=偶真陈述形式的特征]
其代入例既有真陈述又有假陈述的陈述形式,叫做\logicterm{偶真陈述形式}(Contingent Statement Form)。偶真陈述形式具有以下特征:

\textbf{1. 真值依赖性}:其真值依赖于具体的代入内容,而不仅仅是逻辑形式。

\textbf{2. 经验可检验性}:需要通过经验观察或事实调查来确定其真值。

\textbf{3. 信息承载性}:偶真陈述承载关于世界的实质信息,具有认知价值。

\textbf{4. 可能性空间}:偶真陈述在逻辑上既可能为真也可能为假,体现了现实世界的复杂性。
\end{theorembox}

其特征形式是偶真的陈述称为\logicterm{偶真陈述}。\cite{whitehead1911}

\begin{examplebox}[title=三类陈述形式的对比分析]
\textbf{重言式例子}:
\begin{itemize}
\item $p \vee \sim p$(排中律)
\item $p \supset p$(同一律)
\item $(p \cdot q) \supset p$(简化律)
\end{itemize}

\textbf{矛盾式例子}:
\begin{itemize}
\item $p \cdot \sim p$(矛盾律)
\item $(p \supset q) \cdot (p \cdot \sim q)$
\item $p \cdot (p \supset \sim p)$
\end{itemize}

\textbf{偶真陈述形式例子}:
\begin{itemize}
\item $p, \sim p, p \cdot q, p \vee q, p \supset q$
\item 对应的具体陈述:$L, \sim L, L \cdot W, L \vee W, L \supset W$
\end{itemize}
\end{examplebox}

例如,$p, \sim p, p \cdot q, p \vee q$ 和 $p \supset q$ 都是偶真陈述形式,$L, \sim L, L \cdot W, L \vee W, L \supset W$ 这样的陈述都是偶真陈述,因为它们的真值取决于它们的内容,而不只是它们的形式。

并非所有陈述形式都如上面所引的简单例子那样,明显是重言的、自相矛盾的或者偶真的。例如,陈述形式 $[(p \supset q) \supset p] \supset p$ 就一点也不明显,虽然真值表将表明它是一个重言式。它甚至还有一个特殊的名称—— "皮尔士法则"。

\subsection{实质等值的深入分析}

正如析取和实质蕴涵一样,实质等值也是一个真值函项联结词。如前所释,任何真值函项的真值,都取决于其所联结的陈述的真或假(是它们的一个函项)。

\begin{theorembox}[title=实质等值的概念分析]
\logicterm{实质等值}是这样一种真值函项联结词:它断言它所联结的陈述有同样的真值。这个概念具有以下重要特征:

\textbf{1. 真值同步性}:两个陈述实质等值,当且仅当它们具有相同的真值。

\textbf{2. 对称性}:如果A实质等值于B,那么B也实质等值于A。

\textbf{3. 传递性}:如果A实质等值于B,B实质等值于C,那么A实质等值于C。

\textbf{4. 双向蕴涵性}:实质等值等价于双向的实质蕴涵。
\end{theorembox}

因此,两个在真值上相同的陈述,就是实质上等值的。可将之径直定义为:当两个陈述都为真或都为假时,它们就是\logicterm{实质等值的}。

\subsection{实质等值的符号表示与真值条件}

正像析取的符号是楔劈号、实质蕴涵的符号是马蹄号一样,实质等值也有一个特殊的符号,即\logicterm{三杠号}"$\equiv$"。三杠号同样也可以用真值表定义如下:

\begin{center}
\begin{tabular}{|ccc|}
\hline
$p$ & $q$ & $p \equiv q$ \\
\hline
T & T & T \\
T & F & F \\
F & T & F \\
F & F & T \\
\hline
\end{tabular}
\end{center}

任何两个真陈述彼此实质地蕴涵,这是实质蕴涵含义的一个推论;同样,任何两个假陈述也彼此实质地蕴涵。因此,任何两个实质等值的陈述必定彼此蕴涵,因为它们或者都是真的,或者都是假的。

由于任何两个实质等值的陈述 A 和 B 彼此蕴涵,故而从它们的实质等值,我们可以推断出 B 是真的,当 A 是真的;也可以推断出 B 是真的,仅当 A 是真的。由于这两种关系都被实质等值所蕴涵,我们可以把三杠号"三"读做"当且仅当"。

在日常话语中,我们只偶尔使用这种逻辑关系词。有人会说,我去看冠军赛,当且仅当,我获得人场券。当我确实获得了人场券,我会去;但仅当我获得人场券,我才能去。这就是说,我去看比赛和我获得人场券,是实质上等值的。

如前所见,每个蕴涵式都是一个条件陈述。若已知 A 和 B 两个陈述实质上等值,既可推出条件陈述 $\mathrm{A} \supset \mathrm{B}$ 的真,也可推出条件陈述 $\mathrm{B} \supset \mathrm{A}$ 的真。由于在实质等值成立时,蕴涵是双向的,故而一个形如 $\mathrm{A} \equiv \mathrm{B}$ 的陈述通常称为\textbf{双条件陈述}。

合取、析取、实质蕴涵和实质等值,就是演绎论证通常所依赖的四个真值函项联结词。我们现在已经完成了对它们的讨论。

\subsection{真值函项联结词}
真值函项联结词,就是真值函项复合命题中的逻辑联结词。真值函项复合命题,就是其真(或假)完全取决于其组成分支的真或假的复合命题。具有核心重要意义的真值函项联结词有四个:\\
-圆点号.表示合取。读做:"P且Q"。\\
$P \cdot Q$ 为真,当且仅当,$P$ 为真且 $Q$ 为真。\\
V 楔劈号.表示析取。读做:" P 或 Q "。\\
$P \vee Q$ 为真,当且仅当,$P$ 为真,或 $Q$ 为真,或 $P$ 和 $Q$ 两者都为真。\\
$\supset$ 马蹄号.表示实质蕴涵。读做:" P 蕴涵 Q "。\\
$P \supset Q$ 为真,当且仅当,并非 $P$ 为真且 $Q$ 为假,也就是,当且仅当, P 为假或 Q 为真。

三三杠号.表示实质等值。读做:" P 当且仅当 Q "。\\
$\mathrm{P} \equiv \mathrm{Q}$ 为真,当且仅当, P 和 Q 有同样的真值,也就是,当且仅当, P 为真且 Q 为真,或 P 为假且 Q 为假。

\subsection{论证、条件陈述与重言式}
每个论证都对应着这样一个条件陈述:它的前件是该论证的前提的合取,它的后件是该论证的结论。例如,一个论证若具有肯定前件式的形式:

$$
\begin{aligned}
& p \supset q \\
& p \\
& \therefore q
\end{aligned}
$$

则可以被表达成一个具有形式 $[(p \supset q) \cdot p] \supset q$ 的条件陈述。如果原论证具有有效的论证形式,即在每种情形下其结论必定可以从其前提推出,那么,可以在真值表中表明,转化后的条件陈述是一个重言式。这就是说,一个论证的前提的合取蕴涵它的结论这样一个陈述有且只有真代人例(如果该论证有效的话)。

真值表是评价论证的有力工具。一个论证形式有效,当且仅当,在真值表中,其所有前提下面都是 $\mathbf{T}$ 的每一行上,其结论栏的下面也是 $\mathbf{T}$ 。这一点可以从"有效性"的精确含义中得出。如果表达该论证形式的条件陈述成了真值表中某一栏的题头,那么,$F$ 只能出现在该栏中其所有前提下面都是 $T$ ,且结论下面是 $F$ 的那一行。但如果该论证是有效的,就不会有这样一行。因此,只有 T 会出现在与一个有效论证相对应的条件陈述的下面,从而该条件陈述必定是一个重言式。所以,我们可以断言:一个论证

形式有效,当且仅当,其条件陈述表达形式(其前件是该论证形式的前提 339 的合取,其后件是该论证形式的结论)是一个重言式。

显然,对关于真值函项的任一无效论证来说,相应的条件陈述必定不是重言式。由一个无效论证的前提的合取蕴涵其结论构成的条件陈述,或者是偶真陈述,或者是矛盾陈述。

\begin{center}
\fbox{\parbox{0.95\textwidth}{
\textbf{本节要点}
\begin{itemize}
\item \textbf{陈述形式的理论基础}:
  \begin{itemize}
  \item 含有陈述变元但不含陈述的符号序列
  \item \textbf{四大特征}:形式性、生成性、分类性、可操作性
  \item 与论证形式的完全平行关系
  \item 系统的分类体系:基本形式、复合形式、混合形式
  \end{itemize}
\item \textbf{逻辑真理的三重分类}:
  \begin{itemize}
  \item \textbf{重言式}(Tautology):所有代入例都为真
    \begin{itemize}
    \item 逻辑必然性、形式性、先验性、信息空虚性
    \item 例子:$p \vee \sim p$(排中律)、$p \supset p$(同一律)
    \end{itemize}
  \item \textbf{矛盾式}(Contradiction):所有代入例都为假
    \begin{itemize}
    \item 逻辑不可能性、形式矛盾性、先验可知性、爆炸原理
    \item 例子:$p \cdot \sim p$(矛盾律)
    \end{itemize}
  \item \textbf{偶真陈述形式}(Contingent):代入例有真有假
    \begin{itemize}
    \item 真值依赖性、经验可检验性、信息承载性、可能性空间
    \item 例子:$p, p \cdot q, p \vee q, p \supset q$
    \end{itemize}
  \end{itemize}
\item \textbf{逻辑真理与经验真理的根本区别}:
  \begin{itemize}
  \item \textbf{经验真理}:依赖世界实际状况,需经验观察确定
  \item \textbf{逻辑真理}:由逻辑形式决定,独立于世界状况
  \item 哲学意义:揭示人类知识的两种来源(经验观察与逻辑推理)
  \end{itemize}
\item \textbf{实质等值的深入分析}:
  \begin{itemize}
  \item 断言两个陈述有相同真值的真值函项联结词
  \item \textbf{四大特征}:真值同步性、对称性、传递性、双向蕴涵性
  \item 符号:三杠号"$\equiv$",读作"当且仅当"
  \item 形成双条件陈述,蕴涵关系双向成立
  \item 日常应用:条件性承诺和决策表达
  \end{itemize}
\item \textbf{四种基本真值函项联结词的系统总结}:
  \begin{itemize}
  \item 合取($\cdot$):P且Q,当且仅当P为真且Q为真
  \item 析取($\vee$):P或Q,当且仅当P为真或Q为真或两者都真
  \item 实质蕴涵($\supset$):P蕴涵Q,当且仅当P为假或Q为真
  \item 实质等值($\equiv$):P当且仅当Q,当且仅当P和Q有相同真值
  \end{itemize}
\item \textbf{论证与重言式的深层关系}:
  \begin{itemize}
  \item 每个论证对应一个条件陈述(前提合取蕴涵结论)
  \item 有效论证的对应条件陈述必定是重言式
  \item 无效论证的对应条件陈述是偶真陈述或矛盾陈述
  \item 这种对应关系为论证有效性提供了机械化检验方法
  \end{itemize}
\item \textbf{复杂陈述形式}:
  \begin{itemize}
  \item 皮尔士法则:$[(p \supset q) \supset p] \supset p$
  \item 并非所有陈述形式的逻辑性质都显而易见
  \item 真值表方法的普遍适用性
  \end{itemize}
\end{itemize}
}}
\end{center}