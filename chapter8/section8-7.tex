\section{实质蕴涵怪论}

\begin{quotation}
本节讨论实质蕴涵在逻辑上正确但在直觉上似乎怪异的特性,如"真陈述被任何陈述所蕴涵"和"假陈述蕴涵任何陈述"。通过分析这些看似悖论的情况,我们能够澄清实质蕴涵的真正含义以及其在形式逻辑系统中的合理性。
\end{quotation}

有两个很容易证明是重言式的陈述形式,$p \supset(q \supset p)$ 和 $\sim p \supset(p \supset$ $q)$ 。在它们的符号表述中,这些陈述形式可能是无关紧要的,但若用日常语言表述出来,它们看起来令人惊奇,甚至怪异。第一个可以表述为: "如果一个陈述是真的,那么它被任何一个陈述所蕴涵。"由于"地球是圆的"是真的,可以推出"月亮是新鲜奶酪做的蕴涵地球是圆的",这确实

十分怪异,特别是因为它也可以得出:"月亮不是新鲜奶酪做的蕴涵地球是圆的。"第二个重言式可以表述为:"如果一个陈述是假的,那么它蕴涵任何陈述。"由于"月亮是新鲜奶酪做的"是假的,可以推出"月亮是新鲜奶酪做的蕴涵地球是圆的";当我们意识到由之也可以得出"月亮是新鲜奶酪做的蕴涵地球不是圆的"时,这就更怪异了。

这些陈述之所以看起来怪异,是因为我们相信,地球的形状和月亮的质料彼此之间是完全不相干的;我们还相信,没有任何真的或假的陈述能真正地蕴涵任何一个与之完全不相干的假的或真的陈述。可是真值表表明:一个假陈述蕴涵任何一个陈述,一个真陈述被任何陈述所蕴涵。然而,若我们认识到语词"蕴涵"的歧义性,该怪论很容易解决。根据语词 "蕴涵"的某几种含义,没有一个偶真陈述能蕴涵与其主题毫不相干的任何其他偶真陈述,这一点是非常正确的。诸如在逻辑蕴涵、定义性蕴涵和因果性蕴涵场合,这都是正确的。甚至在决策性蕴涵场合,这也是正确的,尽管相干概念在此必须作更宽泛的解释。

但严格说来,主题或意义与\textbf{实质蕴涵}不相干,实质蕴涵是一个真值函项。这里只有真和假是相干的。说任何一个至少含有一个真析取支的析取是真的,并没有任何怪异之处。这一事实是具有 $p \supset(\sim q \vee p)$ 和 $\sim p \supset$ $(\sim p \vee q)$ 形式的陈述所断言的所有东西,这两种形式的陈述逻辑等价于那两个"怪论"陈述。我们已经为把实质蕴涵当做"如果一那么"的一种含义提供了辩护,并且为把"如果一那么"的每次出现都翻译成符号 "つ"这种逻辑的权宜之计提供了辩护。这种辩护基于这样一个事实:把 "如果一那么"翻译成"コ",保留了在我们的逻辑研究所关注的那种论证中的所有有效论证的有效性。有人还提出了另一些符号体系,它们适合于其他类型的蕴涵,但它们超出了本书的范围,属于逻辑的更高级部分。 

\begin{center}
\fbox{\parbox{0.95\textwidth}{
\textbf{本节要点}
\begin{itemize}
\item \textbf{实质蕴涵怪论}的两个重言式:
  \begin{itemize}
  \item $p \supset(q \supset p)$:"如果一个陈述是真的,那么它被任何一个陈述所蕴涵"
  \item $\sim p \supset(p \supset q)$:"如果一个陈述是假的,那么它蕴涵任何陈述"
  \end{itemize}
\item 怪论产生的原因:
  \begin{itemize}
  \item 日常直觉认为命题间应该有相关性
  \item 实质蕴涵仅仅考虑真值而非内容关联
  \item "蕴涵"一词在自然语言中有多种含义
  \end{itemize}
\item 怪论的解决方式:
  \begin{itemize}
  \item 认识到实质蕴涵是一个纯粹的真值函项
  \item 区分实质蕴涵与其他类型的蕴涵(逻辑、定义性、因果性蕴涵等)
  \item 理解实质蕴涵的真值表定义
  \end{itemize}
\item 实质蕴涵的等价形式:
  \begin{itemize}
  \item $p \supset (q \supset p)$ 等价于 $p \supset (\sim q \vee p)$
  \item $\sim p \supset (p \supset q)$ 等价于 $\sim p \supset (\sim p \vee q)$
  \end{itemize}
\end{itemize}
}}
\end{center} 