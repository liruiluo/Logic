\section{三大"思想法则"}

\begin{quotation}
本节探讨传统上被称为"思想法则"的三个基本逻辑原理:同一原理、不矛盾原理和排中原理。通过分析这些原理的本质及其在逻辑推理中的应用和局限,我们能够更全面地理解形式逻辑的基础以及对这些原理的误解与批评。
\end{quotation}

一些早期思想家把逻辑定义为"关于思想法则的科学",并进一步断言:刚好有三个基本思想法则,它们如此基本以至遵从它们既是正确思维的必要条件又是其充分条件。传统上,这三大法则叫做:

1. \textbf{同一原理}。\\
这个原理断言:如果一个陈述是真的,那么它就是真的。我们可以用符号这样重述它:同一原理断言的是每个具有 $p \supset p$ 形式的陈述必定是真的,每个这样的陈述都是重言式。\\

2. \textbf{不矛盾原理}。\\
这个原理断言:没有陈述是既真又假的。我们可以用符号这样重述它:不矛盾原理断言的是每个具有 $p \cdot \sim p$ 形式的陈述必定是假的,每个这样的陈述是自相矛盾的。\\

3. \textbf{排中原理}。\\
这个原理断言:每个陈述或者是真的或者是假的。我们可以用符号这样重述它:排中原理断言的是每个具有 $p \vee \sim p$ 形式的陈述必定是真的,每个这样的陈述都是重言式。

显然,这三大原理确实是真的,是逻辑地为真的一一但说它们具有最基本的思想法则这一特权地位,是值得怀疑的。第一个(同一原理)和第三个(排中原理)是重言式,但还有许多其他的重言形式,它们的真是同等确定的。第二个(不矛盾原理)(所排除的 $p \cdot \sim p$ )也绝不是唯一的自相矛盾的陈述形式。

在构造真值表时,我们确实使用了这几个原理。受排中原理指导,我们在真值表每一行的初始栏下填人一个 T 或 F。受不矛盾原理指导,我们不在任何地方同时既填 T 又填 F 。一旦在某个指定行中把 T 填在某个符号下面,那么(受同一原理指导),当我们在那一行的其他栏下遇到该符号时,我们把它看做仍然被赋予T。因此,我们可以把这三大思想法则看做是支配真值表构造的原理。

不过,在考虑整个演绎逻辑体系时,这三大原理并不比其他许多原理更重要或更富有成效。确实,为演绎起见,有一些比它们更有成效的重言式。在这个意义上说,它们比这三大原理更重要。更深人地讨论这一点超出了本书的范围。\cite{hamilton1833}

由于相信自己设计出了某种新的不同逻辑,一些思想家声称这三大原理实际上是不正确的,遵循它们是不必要的限制。但这些批评都是建立在误解的基础上的。

基于事物都是变化的而且一直在变化这一理由,同一原理遭到了攻击。例如,对原来的由 13 个州所组成的美国来说为真的某些陈述,对今

天有 50 个州的美国来说就不再是真的。然而,这并不能伤害同一原理。语句"美国只有 13 个州"是不完整的表述,它是陈述"1790 年的美国只有 13 个州"的一种省略表述,和它在1790年时一样,这个陈述在今天也是真的。若我们把注意力限制到命题的完整的、非省略的表述,我们就会看到,它们的真(或假)并不随时间而改变。同一原理之为真,并不妨碍我们对连续性变化的认识。

不矛盾原理受到了黑格尔主义者和马克思主义者的非难,其理由是:实际矛盾是普遍存在的,世界充满着不可避免的矛盾力量的冲突。说实在世界中存在着相冲突的力量,这当然是对的,但把这些冲突力量称为"矛盾",则是对该术语的一种不精确且令人误解的使用。劳工联盟和工厂私有者发现他们确实处于冲突之中——但私有者和劳工联盟都不是对方的 "否定"、"否认"或"矛盾"。若径直按照逻辑学家所意谓的那种意义理解,不矛盾原理是不可反驳、完全准确的。

基于其导致"二值化"这一理由,排中原理成了许多批评的靶子。 "二值化"意味着断言世界上的事物必定是"或白或黑"的,由此,它妨碍了妥协的实现,导致绝对化分层。这种反对意见也来自误解。陈述 "这是黑的"当然不能与陈述"这是白的"同时为真——假如"这"指的恰是同一事物的话。尽管这两个陈述不能同时为真,但它们却能同时为假。"这"可以既不是白的又不是黑的;这两个陈述是反对关系,而不是矛盾关系。与陈述"这是白的"有矛盾关系的陈述是"并非这是白的",并且(如果在这两个陈述中,"白的"都是在完全同样的意义上使用的话),它们当中必定有一个为真而另一个为假。排中原理是不可摆脱的。

总之,所有这三大"思想法则"都是不可驳倒的——只要它们被运用于那些使用非歧义、非省略且精确的词项的陈述。它们可能不具有某些哲学家所赋予它们的那种尊贵地位\cite{kant1781},但它们无疑都是正确的。

\begin{center}
\fbox{\parbox{0.95\textwidth}{
\textbf{本节要点}
\begin{itemize}
\item \textbf{三大思想法则}及其形式表达:
  \begin{itemize}
  \item 同一原理:$p \supset p$(陈述为真则它为真)
  \item 不矛盾原理:$\sim(p \cdot \sim p)$(陈述不能既真又假)
  \item 排中原理:$p \vee \sim p$(陈述或真或假)
  \end{itemize}
\item 这些原理在真值表构造中的应用:
  \begin{itemize}
  \item 排中原理决定每个命题必赋予T或F
  \item 不矛盾原理确保不会同时赋予T和F
  \item 同一原理保证同一符号在同一行中具有相同真值
  \end{itemize}
\item 对这些原理的误解和批评:
  \begin{itemize}
  \item 基于变化而对同一原理的批评(忽略了时间因素)
  \item 基于现实冲突而对不矛盾原理的批评(误用"矛盾"概念)
  \item 基于"二值化"对排中原理的批评(混淆反对和矛盾关系)
  \end{itemize}
\item 这些原理的地位:
  \begin{itemize}
  \item 它们是真的,但非唯一的基本逻辑原理
  \item 在整个演绎体系中,其他重言式可能更为重要
  \item 它们的正确性取决于明确、精确的陈述表达
  \end{itemize}
\end{itemize}
}}
\end{center} 