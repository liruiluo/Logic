\section*{8.5 陈述形式与实质等值}
\section*{A.陈述形式与陈述}
现在,我们来明确一下前一节所假定的一个概念,即"陈述形式"。以论证和论证形式之间的关系为一方,陈述和陈述形式之间的关系为另一方,这两者是完全平行的。"陈述形式"的如下定义可使这一点很明显:一个陈述形式是任何一个含有陈述变元但不含陈述的符号序列,若用陈述代入这些陈述变元——用同一个陈述始终一致地代入同一个陈述变元——其结果是一个陈述。例如,$p \vee q$ 是陈述形式,因为若用陈述代人变元 $p$和 $q$ ,就会产生一个陈述。由于所产生的陈述是一个析取句,$p \vee q$ 就叫做 "析取陈述形式"。同样,$p \cdot q$ 和 $p \supset q$ 分别叫做"合取陈述形式"和"条件陈述形式",$\sim p$ 叫做"否定形式"或者"否认形式"。正像某种形式的论证称为该论证形式的代人例一样,具有某种形式的任一陈述称为该陈述形式的代入例。正像我们判别一个给定论证的特征形式一样,我们把一个给定陈述的特征形式判别为这样一种陈述形式:通过一致地用不同的简单陈述代入每个不同的陈述变元,就可以从其产生该给定陈述。例如,$p \vee q$就是陈述"那个盲囚戴红帽子或者那个盲囚戴白帽子"的特征形式。

\section*{B.重言的、矛盾的和偶真的陈述形式}
尽管陈述"林肯是被暗杀的"(记为 $L$ )和"林肯或者是被暗杀的,或者不是"(记为 $L \vee \sim L$ )都是真的,但我们会非常自然地感觉到,它们是"在不同方面"为真,或有"不同种类"的真。同样,尽管陈述"华盛顿是被暗杀的"(记为 $W$ )和"华盛顿既是被暗杀的又不是被暗杀的"(记为 $W \cdot \sim W)$ 这两者都为假,但我们也会非常自然地感觉到,它们也是 "在不同方面"为假,或有"不同种类"的假。尽管我们不能给这些"感觉"以心理学解释,但我们仍然可以指出它们相应的逻辑区别。

陈述 $L$ 为真和陈述 $W$ 为假乃属于历史事实,它们没有逻辑必然性。所有事件都有以不同方式出现的可能,因而像 $L$ 和 $W$ 这样的陈述的真值,必须通过对历史的经验研究才能被发现。而陈述 $L \vee \sim L$ 尽管是真的,但它不是历史地真,而具有逻辑的必然性:事件不可能如此这般以致使它为假,它的真可以独立于任何经验研究而被知晓。陈述 $L \vee \sim L$ 是一个逻辑真理,或曰形式真理,其真仅因其形式,它是一个其所有代人例都是真陈

述的陈述形式的代人例。\\
一个只有真代入例的陈述形式叫重言的陈述形式,或重言式。要表明陈述形式 $p \vee \sim p$ 是一个重言式,可构造如下真值表:

\begin{center}
\begin{tabular}{|ccc|}
\hline
$p$ & $\sim p$ & $p \vee \sim p$ \\
\hline
T & F & T \\
F & T & T \\
\hline
\end{tabular}
\end{center}

这个真值表只有一个初始栏或导引栏,因为被探究的形式只含有一个陈述变元。它只有两行,代表了所有可能的代人例。被检验陈述形式下面的那一栏里只有 T ,这表明,它的所有代人例都是真的。任何一个作为重言的陈述形式的代人例的陈述,依据其形式就是真的,其本身被称为重言陈述,亦称为一个重言式。

一个只有假代入例的陈述形式称为自相矛盾的陈述形式,或矛盾式,它是逻辑地为假的。陈述形式 $p \cdot \sim p$ 是自相矛盾的,因为在它的真值表中只有 F 在它下面出现,这表明它的所有代人例都是假的。任何一个作为自相矛盾的陈述形式的代人例的陈述,如 $W \cdot \sim W$ ,依据其形式就是假的,其本身称为自相矛盾的陈述,亦称为一个矛盾式。

其代入例既有真陈述又有假陈述的陈述形式,叫做偶真陈述形式。其特征形式是偶真的陈述称为偶真陈述。 ${ }^{[15]}$ 例如,$p, \sim p, p \cdot q, p \vee q$ 和 $p \supset q$ 都是偶真陈述形式,$L, \sim L, L \cdot W, L \vee W, L \supset W$ 这样的陈述都是偶真陈述,因为它们的真值取决于它们的内容,而不只是它们的形式。

并非所有陈述形式都如上面所引的简单例子那样,明显是重言的、自相矛盾的或者偶真的。例如,陈述形式 $[(p \supset q) \supset p] \supset p$ 就一点也不明显,虽然真值表将表明它是一个重言式。它甚至还有一个特殊的名称—— "皮尔士法则"。

\section*{C.实质等值}
正如析取和实质蕴涵一样,实质等值也是一个真值函项联结词。如前所释,任何真值函项的真值,都取决于其所联结的陈述的真或假(是它们的一个函项)。例如,如果 A 或 B 是真的,或者 A 和 B 都是真的,那么, $A$ 和 $B$ 的析取就是真的。实质等值则是这样一种真值函项联结词:它断言它所联结的陈述有同样的真值。因此,两个在真值上相同的陈述,就是实质上等值的。可将之径直定义为:当两个陈述都为真或都为假时,它们就

是"实质等值的"。\\
正像析取的符号是楔劈号、实质蕴涵的符号是马蹄号一样,实质等值也有一个特殊的符号,即三杜号"三"。三杜号同样也可以用真值表定义如下:

\begin{center}
\begin{tabular}{|ccc|}
\hline
$p$ & $q$ & $p \equiv q$ \\
\hline
T & T & T \\
T & F & F \\
F & T & F \\
F & F & T \\
\hline
\end{tabular}
\end{center}

任何两个真陈述彼此实质地蕴涵,这是实质蕴涵含义的一个推论;同样,任何两个假陈述也彼此实质地蕴涵。因此,任何两个实质等值的陈述必定彼此蕴涵,因为它们或者都是真的,或者都是假的。

由于任何两个实质等值的陈述 A 和 B 彼此蕴涵,故而从它们的实质等值,我们可以推断出 B 是真的,当 A 是真的;也可以推断出 B 是真的,仅当 A 是真的。由于这两种关系都被实质等值所蕴涵,我们可以把三杠号"三"读做"当且仅当"。

在日常话语中,我们只偶尔使用这种逻辑关系词。有人会说,我去看冠军赛,当且仅当,我获得人场券。当我确实获得了人场券,我会去;但仅当我获得人场券,我才能去。这就是说,我去看比赛和我获得人场券,是实质上等值的。

如前所见,每个蕴涵式都是一个条件陈述。若已知 A 和 B 两个陈述实质上等值,既可推出条件陈述 $\mathrm{A} \supset \mathrm{B}$ 的真,也可推出条件陈述 $\mathrm{B} \supset \mathrm{A}$ 的真。由于在实质等值成立时,蕴涵是双向的,故而一个形如 $\mathrm{A} \equiv \mathrm{B}$ 的陈述通常称为双条件陈述。

合取、析取、实质蕴涵和实质等值,就是演绎论证通常所依赖的四个真值函项联结词。我们现在已经完成了对它们的讨论。

\section*{真值函项联结词}
真值函项联结词,就是真值函项复合命题中的逻辑联结词。真值函项复合命题,就是其真(或假)完全取决于其组成分支的真或假的复合命题。具有核心重要意义的真值函项联结词有四个:\\
-圆点号.表示合取。读做:"P且Q"。\\
$P \cdot Q$ 为真,当且仅当,$P$ 为真且 $Q$ 为真。\\
V 楔劈号.表示析取。读做:" P 或 Q "。\\
$P \vee Q$ 为真,当且仅当,$P$ 为真,或 $Q$ 为真,或 $P$ 和 $Q$ 两者都为真。\\
$\supset$ 马蹄号.表示实质蕴涵。读做:" P 蕴涵 Q "。\\
$P \supset Q$ 为真,当且仅当,并非 $P$ 为真且 $Q$ 为假,也就是,当且仅当, P 为假或 Q 为真。

三三杠号.表示实质等值。读做:" P 当且仅当 Q "。\\
$\mathrm{P} \equiv \mathrm{Q}$ 为真,当且仅当, P 和 Q 有同样的真值,也就是,当且仅当, P 为真且 Q 为真,或 P 为假且 Q 为假。

\section*{D.论证、条件陈述与重言式}
每个论证都对应着这样一个条件陈述:它的前件是该论证的前提的合取,它的后件是该论证的结论。例如,一个论证若具有肯定前件式的形式:

$$
\begin{aligned}
& p \supset q \\
& p \\
& \therefore q
\end{aligned}
$$

则可以被表达成一个具有形式 $[(p \supset q) \cdot p] \supset q$ 的条件陈述。如果原论证具有有效的论证形式,即在每种情形下其结论必定可以从其前提推出,那么,可以在真值表中表明,转化后的条件陈述是一个重言式。这就是说,一个论证的前提的合取蕴涵它的结论这样一个陈述有且只有真代人例(如果该论证有效的话)。

真值表是评价论证的有力工具。一个论证形式有效,当且仅当,在真值表中,其所有前提下面都是 $\mathbf{T}$ 的每一行上,其结论栏的下面也是 $\mathbf{T}$ 。这一点可以从"有效性"的精确含义中得出。如果表达该论证形式的条件陈述成了真值表中某一栏的题头,那么,$F$ 只能出现在该栏中其所有前提下面都是 $T$ ,且结论下面是 $F$ 的那一行。但如果该论证是有效的,就不会有这样一行。因此,只有 T 会出现在与一个有效论证相对应的条件陈述的下面,从而该条件陈述必定是一个重言式。所以,我们可以断言:一个论证

形式有效,当且仅当,其条件陈述表达形式(其前件是该论证形式的前提 339 的合取,其后件是该论证形式的结论)是一个重言式。

显然,对关于真值函项的任一无效论证来说,相应的条件陈述必定不是重言式。由一个无效论证的前提的合取蕴涵其结论构成的条件陈述,或者是偶真陈述,或者是矛盾陈述。 