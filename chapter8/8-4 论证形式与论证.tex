\section{论证形式与论证}

\begin{logicbox}[title=引言]
本节讨论如何评估论证的有效性,介绍论证形式的概念及真值表检验方法。通过学习逻辑类推反驳法和几种常见的有效论证形式,我们能够系统地判断一个论证是否在逻辑上成立,从而更准确地分析推理的正确性。
\end{logicbox}

\subsection{逻辑类推反驳法的理论基础}

本节我们将更精确地阐明术语\logicterm{有效}的含义。通过探讨运用逻辑类推进行反驳的方法,我们把形式定义与某些更熟悉的、直觉的观念联系起来。\cite{copi1980}

\begin{theorembox}[title=逻辑类推反驳法的哲学基础]
逻辑类推反驳法建立在现代逻辑的一个根本原理之上:\logicemph{论证的有效性完全取决于其逻辑形式,而与具体内容无关}。这一原理体现了现代逻辑的形式主义特征,它使我们能够通过分析抽象的逻辑结构来评估具体论证的合理性。

这种方法的认识论意义在于:它提供了一种客观的、不依赖于主观判断的论证评估标准。无论论证涉及的是政治、科学还是日常生活,只要它们具有相同的逻辑形式,它们的有效性就是相同的。
\end{theorembox}

兹以如下论证为例:

培根是一位伟大的作家。\\
因此,培根写了那些通常归功于莎士比亚的剧本。

我们可能同意其前提但不同意其结论,从而断定该论证无效。证明无效性的方式之一就是运用逻辑类推的方法。我们可以反驳说,"你是否也可以这样来论证",

如果华盛顿是被暗杀的,那么华盛顿死了。\\
华盛顿死了。\\
因此,华盛顿是被暗杀的。\\
"但你不可能对该论证进行严格的辩护",我们可以继续说,"因为在这里,已知前提为真且结论为假,这个论证显然是无效的;而你前面的论证有同样的形式,因此,你的论证也是无效的。"这种类型的反驳是非常有效力的。

这种用逻辑类推进行反驳的方法,为获得一种检查论证的极好的一般方法指示了方向。要证明一个论证的无效性,构造另外一个这样的论证就足够了:(1)它与第一个论证有完全一样的形式;(2)它有真的前提和假的结论。这种方法建立在这样一个事实之上,即\logicterm{有效性}和\textbf{无效性}是论证的纯粹形式的特征,这就是说,不管它们所探讨的题材有何差别,任何两个有完全相同形式的论证或者都是有效的或者都是无效的。\cite{jevons1886}

当用大写字母缩写给定论证中的简单陈述时,该论证就很清楚地展示了它的形式。例如,分别用 $B 、 G 、 A 、 D$ 来缩写"培根写了那些通常归功于莎士比亚的剧本"、"培根是一位伟大的作家"、"华盛顿是被暗杀的"、 "华盛顿死了",用熟悉的三点符"$\therefore$"代替"因此",我们可以把前面的两个论证分别符号化为

\begin{center}
\begin{tabular}{ll}
$B \supset G$ & $A \supset D$ \\
$G$ & $D$ \\
$\therefore B$ & $\therefore A$ \\
\end{tabular}
\end{center}

经过如此改写,它们的共同形式就很容易看清楚。\\
要讨论论证的形式而不是具有这些形式的特定论证,我们需要某种把这些论证形式本身符号化的方法。为了获得这种方法,我们引人变元的概念。在前面几节中,我们是用大写字母来符号化特定的简单陈述的。为避免混淆,我们从字母表的中间部分选取小写字母 $p, q, r, s \cdots \cdots$ 作为\textbf{陈述变元}。我们将如此使用这个术语:一个陈述变元就是这样一个字母,一个陈述可以被代入它或它所在的位置。复合陈述和简单陈述一样,也可以被代人到陈述变元中。

\subsection{论证形式的深入分析}

我们把一个\logicterm{论证形式}定义为,任何这样一列包含陈述变元而不包含陈述的符号序列,当用陈述代入陈述变元时——同一陈述始终代入同一陈述变元——其结果就是一个论证。

\begin{theorembox}[title=论证形式的数学性质]
论证形式的概念具有重要的数学性质:

\textbf{1. 抽象性}:论证形式抽象掉了具体的内容,只保留了逻辑结构。这种抽象使我们能够研究推理的纯粹形式特征。

\textbf{2. 生成性}:每个论证形式可以生成无穷多个具体论证(代入例)。这种生成性体现了形式逻辑的强大表达能力。

\textbf{3. 分类性}:论证形式为论证提供了分类标准。具有相同形式的论证属于同一逻辑类别,具有相同的有效性特征。

\textbf{4. 可判定性}:对于命题逻辑中的论证形式,存在机械化的决定程序(如真值表方法)来判定其有效性。
\end{theorembox}

为确定性起见,我们作这样一个约定:在任何论证形式中,$p$ 是在其中出现的第一个陈述变元,$q$ 是第二个,$r$ 是第三个,等等。例如,表达式

$$
\begin{aligned}
& p \supset q \\
& q \\
& \therefore p
\end{aligned}
$$

是一个论证形式。因为当分别用陈述 $B 、 G$ 代人陈述变元 $p 、 q$ 时,其结果就是本节中的第一个论证。如果用陈述 $A$ 和 $D$ 代入陈述变元 $p 、 q$ ,其结果就是第二个论证。以陈述代入一个论证形式中的陈述变元而产生的任何论证,就叫该论证形式的一个\textbf{代入例}。显然,一个论证形式的任何代入例都可以说成具有该形式的论证,具有某种形式的任何论证都是该形式的一个代入例。

对任何论证来说,通常都有多个论证形式,它们以该给定论证作为代人例之一。例如,本节的第一个论证:

$$
\begin{aligned}
& B \supset G \\
& G \\
& \therefore B
\end{aligned}
$$

就是下列四个论证形式巾每一个的代入例:

\begin{center}
\begin{tabular}{llll}
$p \supset q$ & $p \supset q$ & $p \supset q$ & $p$ \\
$q$ & $r$ & $r$ & $q$ \\
$\therefore p$ & $\therefore p$ & $\therefore s$ & $\therefore r$ \\
\end{tabular}
\end{center}

如此,在第一个论证形式中以 $B$ 代人 $p$ ,以 $G$ 代人 $q$ ,在第二个形式中以 $B$ 代人 $p$ ,以 $G$ 代人 $q$ 和 $r$ ,在第三个论证中以 $B$ 代人 $p$ 和 $s$ ,以 $G$ 代人 $q$和 $r$ ,在第四个论证中以 $B \supset G$ 代人 $p$ ,以 $G$ 代人 $q$ ,以 $B$ 代人 $r$ ,我们都得到了上述论证。在这四个论证形式中,第一个比其他几个更紧密地对应

于给定论证的结构。这是因为,该论证是通过以不同的简单陈述,代人其中的每个不同陈述变元而获得的。我们把第一种论证形式称为该给定论证的特征形式。我们将一个论证的特征形式定义为:只要一个论证是通过一致地以不同的简单陈述代入一个论证形式中每个不同的陈述变元而产生的,该论证形式就是这个论证的特征形式。对任何给定论证来说,都有一个独特的论证形式作为该论证的特征形式。

现在,我们可以对用逻辑类推进行反驳的方法做更精确的描述。如果一个给定论证的特征形式有任意一个其前提为真且结论为假的代人例,那么,该论证就是无效的。我们可以把运用在论证形式上的术语"无效的"定义如下:一个论证形式是无效的,当且仅当,它至少有一个前提为真且结论为假的代入例。运用逻辑类推进行反驳建立在这样一个事实之上:其特征形式是一个无效的论证形式的任何论证都是一个无效论证。任何一个不是无效的论证形式必定是有效的。因此,一个论证形式是有效的,当且仅当,它没有前提为真且结论为假的代入例。由于有效性是一个形式概念,所以,一个论证有效,当且仅当,该论证的特征形式是一个有效论证形式。

如果能找到与所论证的一个反驳性类推,那么,该论证就被证明为无效,但"想出"这样的反驳性类推并非总是很容易。幸而这并不是必需的,因为对这种类型的论证来说,有一种建立在同样原则基础上的、更简单的、纯机械性的检验方法。给定任何论证,我们只需检验它的特征形式,因为特征形式的有效和无效决定了该论证的有效和无效。

\subsection{真值表检验方法的理论基础}

要检验一个论证形式,我们可以考察它的所有可能的代入例,看它们当中是否有一个前提为真而结论为假。当然,任何一个论证形式都有无穷多个代入例,但不必担心,我们用不着逐一去考察它们。

\begin{theorembox}[title=真值表方法的数学原理]
真值表方法的有效性建立在以下数学原理之上:

\textbf{1. 有穷性原理}:虽然论证形式有无穷多个代入例,但由于我们只关心真值,而每个陈述变元只有真、假两种可能的真值,所以$n$个不同陈述变元只有$2^n$种不同的真值组合。

\textbf{2. 完备性原理}:真值表穷尽了所有可能的真值组合,因此能够完全确定论证形式的有效性。

\textbf{3. 机械性原理}:真值表的构造和解读是完全机械化的过程,不依赖于直觉或主观判断。

\textbf{4. 可靠性原理}:如果真值表显示论证形式有效,那么该形式的所有代入例都必然有效;如果显示无效,则存在反例。
\end{theorembox}

因为我们感兴趣的只是它们的前提和结论的真或假,在此只需考虑真值问题。我们这里所探讨的论证只含有简单陈述和用真值联结词联结简单陈述而构成的复合陈述,这些真值联结词可用上述圆点、波浪号、楔劈号和马蹄号符号化。

\subsection{真值表方法的历史发展}

\begin{examplebox}[title=真值表方法的历史里程碑]
\textbf{路德维希·维特根斯坦(1921)}:在《逻辑哲学论》中首次系统地使用真值表来分析命题的逻辑结构,奠定了现代真值表方法的基础。

\textbf{埃米尔·波斯特(1921)}:独立发展了真值表方法,并证明了命题逻辑的完备性和一致性。

\textbf{现代发展}:真值表方法成为计算机科学中布尔代数和数字电路设计的理论基础,体现了逻辑学与技术应用的深度结合。
\end{examplebox}

通过考察某些陈述真值的所有可能的不同排列组合(这些陈述被用来代入到被检验论证形式的不同陈述变元中),我们就获得了其前提和结论有不同真值的所有可能代入例。

如果一个论证形式只包含两个不同的陈述变元 $p$ 和 $q$ ,它们的所有代入例就是:或者 $p$ 和 $q$ 都代人真陈述,或者 $p$ 代人真陈述而 $q$ 代人假陈

述,或者 $p$ 代人假陈述而 $q$ 代人真陈述,或者 $p$ 和 $q$ 都代人假陈述。用真值表形式可以最方便地把这些不同的情形集结在一起。为判定下列论证形式的有效性:

$$
\begin{aligned}
& p \supset q \\
& q \\
& \therefore p
\end{aligned}
$$

可构造下列真值表:

\begin{center}
\begin{tabular}{|ccc|}
\hline
$p$ & $q$ & $p \supset q$ \\
\hline
T & T & T \\
T & F & F \\
F & T & T \\
F & F & T \\
\hline
\end{tabular}
\end{center}

这个表的每一行代表一整类代人例。两个初始栏或导引栏中的 T 和 F ,表示该论证形式中的变元 $p$ 和 $q$ 的代人陈述的真值。回过来依据初始栏或导引栏及马蹄号的定义,即可填写第三栏。第三栏的题头是该论证形式的第一个"前提",第二栏的题头是第二个"前提",第一栏的题头是"结论"。考察这个真值表,我们发现在第三行中,两个前提下都是 T ,但结论下面是 F 。这表明,上列论证形式至少有一个前提为真结论为假的代人例,这一行足以表明该论证形式是无效的。具有这种特征形式的论证(也就是说,任何以上列论证形式为特征形式的论证),被看做犯了肯定后件的谬误,因为其第二个前提肯定的是条件前提的后件。

尽管概念上很简单,真值表却是非常有力的工具。以其来判定一个论证形式的有效性和无效性时,首先且至关重要的是正确地构造真值表。要正确地构造真值表,就要为论证形式中的每个陈述变元( $p 、 q$ 和 $r$ 等)都列出一个导引栏,其排列必须展示所有这些变元的真假值的全部组合。因此,真值表的横行必须满足:如果有两个变元,就要有四行,如果有三个变元,就要有八行,如此等等。每个前提和结论都必须附有一个竖栏,而构成前提和结论的每个符号表达式也都有一竖栏。用这种方式来构造真值表,本质上是一件机械性工作;它只要求仔细地计数,小心地把 T 和 F填人合适的栏中。所有这些都受我们对几个真值函项联结词——圆点、楔劈号和马蹄号——的理解的支配,还受真值函项复合陈述为真和为假的条

件的支配。\\
一旦真值表构造完毕,完整地排列呈现在我们面前,正确地解读它,即正确地用它来评价被检验的论证形式也是很重要的。我们必须仔细观察哪些栏表达被检验论证的前提,哪一栏表达该论证的结论。例如,在检验前述无效论证时,我们观察到真值表的第二栏和第三栏是表示前提的,而结论则由第一(最左边的)栏表示。但是,根据我们检验的论证形式的不同,以及构造真值表时我们放置栏的次序,前提和结论有可能以任何次序出现在真值表顶端。它们的位置在左或在右并不重要,重要的是使用真值表时必须理解各栏表示的是什么,以及我们追寻的是什么。我们要问的是,是否存在这样一种情形,即某行中的所有前提为真而结论为假?如果有这样一行,该论证形式就是无效的;如果没有这样一行,则该论证形式必定是有效的。在完整的排列被整齐而精确地陈列出来以后,准确而细心地解读真值表是极端重要的。

\subsection{常见有效论证形式的系统分析}

现代逻辑识别出了一系列基本的有效论证形式,这些形式构成了复杂推理的基础构件。理解这些基本形式对于掌握逻辑推理至关重要。

\subsubsection{析取三段论的深入分析}

析取三段论是最简单的有效论证形式之一,其依赖这样一个事实:在每个为真的析取式中,至少有一个析取支必定是真的。因此,如果其中一个析取支为假,则另一个必定为真。

\begin{theorembox}[title=析取三段论的逻辑原理]
析取三段论的有效性基于以下逻辑原理:

\textbf{1. 排斥原理}:如果我们知道"A或B"为真,并且知道A为假,那么B必须为真。这体现了逻辑推理中的排除法思维。

\textbf{2. 完备性}:析取的真值条件保证了当一个析取支被排除时,另一个析取支必须承担使整个析取为真的责任。

\textbf{3. 实用性}:这种推理形式在日常生活、科学研究和法律推理中都有广泛应用,是最直观的推理方式之一。
\end{theorembox}

析取三段论可用符号表示如下:

$$
\begin{aligned}
& p \vee q \\
& \sim p \\
& \therefore q
\end{aligned}
$$

为表明它的有效性,可构造如下真值表:

\begin{center}
\begin{tabular}{|cccc|}
\hline
$p$ & $q$ & $p \vee q$ & $\sim p$ \\
\hline
T & T & T & F \\
T & F & T & F \\
F & T & T & T \\
F & F & F & T \\
\hline
\end{tabular}
\end{center}

这里,初始栏或导引栏亦展示了用来代人变元 $p$ 和 $q$ 的那些陈述的所有可能的不同真值。依据前两栏可以填上第三栏,依据第一栏可以填上第四栏。现在第三行是 T 出现在两个前提栏(第三和第四栏)的唯一一行,而在此行上 T 也出现在结论栏(第二栏)。于是该真值表表明,这个论证形

式没有前提为真而结论为假的代人例,从而证明了该被检验论证形式的有效性。\cite{jevons1879}

此处真值表的准确解读同样关键:应该细致地识别出表示结论的栏 (左起第二栏)和表示前提的栏(左起第三和第四栏)。只有正确地使用这三栏,我们才能可靠地确立被检验论证形式的有效性或无效性。请注意,相同的真值表可以用来检验一个非常不同的论证形式的有效性,该论证形式的前提由第二和第三栏表示,而结论则由第四栏表示。从该真值表的第一行可看到,这样的论证形式是无效的。

真值表技术为检验这里所讨论的任何一个一般类型的论证的有效性,提供了一种完全机械的方法。我们现在即可以之为把短语"如果一那么"的任何一次出现翻译成实质蕴涵符"コ"进行辩护。在前一节中,我们作了这样一个断言:当我们这里所涉及的"如果一那么"陈述都被解释为只断定实质蕴涵时,这种一般类型的所有有效论证仍然是有效的。可以用真值表来证实这个断言,并为我们把"如果一那么"翻译成马蹄号提供辩护。

\subsubsection{肯定前件式的深入分析}

最简单的一种涉及条件陈述的、直觉上有效的论证可以用下列论证来例示:

如果第二个土著人说真话,那么只有一个土著人是政客。\\
第二个土著人说真话。\\
因此,只有一个土著人是政客。

\begin{theorembox}[title=肯定前件式的逻辑基础]
肯定前件式(Modus Ponens)是逻辑学中最基本、最重要的推理规则之一:

\textbf{1. 历史地位}:这一推理形式可以追溯到古希腊的斯多葛学派,是形式逻辑的基石之一。

\textbf{2. 直觉合理性}:这种推理形式完全符合人类的直觉思维,体现了"如果条件满足,结果必然发生"的因果逻辑。

\textbf{3. 普遍适用性}:无论在数学证明、科学推理还是日常决策中,肯定前件式都是最常用的推理形式。

\textbf{4. 构造性特征}:与某些其他推理形式不同,肯定前件式是构造性的——它不仅告诉我们某些东西不成立,还积极地确立了结论的真实性。
\end{theorembox}

这个论证的特征形式被称为\logicterm{肯定前件式}(Modus Ponens):\\
$p \supset q$\\
$p$\\
$\therefore q$

如下真值表可以证明它是有效的:

\begin{center}
\begin{tabular}{|ccc|}
\hline
$p$ & $q$ & $p \supset q$ \\
\hline
T & T & T \\
T & F & F \\
F & T & T \\
F & F & T \\
\hline
\end{tabular}
\end{center}

在此,两个前提由第三栏和第一栏表示,结论由第二栏表示。只有第一行表示两个前提都真的代人例,而第二栏该行上的 T 表明,在这样的论证中结论也为真。这个真值表确立了具有肯定前件式的任何论证的有效性。

\subsubsection{否定后件式的深入分析}

如前所见,如果一个条件陈述是真的,那么,如果其后件为假,其前件必假。这种论证形式很普遍地用来确定被探究命题的假。

\begin{theorembox}[title=否定后件式的逻辑特征]
否定后件式(Modus Tollens)具有以下重要特征:

\textbf{1. 反证性质}:这是一种间接推理方法,通过否定结果来否定原因,体现了逻辑学中的反证法思维。

\textbf{2. 科学方法论意义}:在科学研究中,否定后件式是假设检验的逻辑基础——如果理论预测的结果不成立,那么理论本身就受到质疑。

\textbf{3. 诊断功能}:在医学诊断、故障排除等领域,这种推理形式被广泛用于排除可能的原因。

\textbf{4. 逻辑强度}:否定后件式提供了一种强有力的反驳方法,是批判性思维的重要工具。
\end{theorembox}

举例来说:在最近一次世界拼字游戏冠军赛中,马特•格雷厄姆和约耳•谢尔曼迎战叫马文的计算机程序。在比赛的某一局,他们发现他们的牌可以组成语词 "triduum";但对宾果游戏来说——用所有八张牌组成一个词——他们还要使用" s "。约耳对他的合作者说,"triduum"确实是一个词,但 "triduums"是否是正确的复数形式呢?马特用一种非常普遍的有效论证形式进行回答:"它必定是。如果它的复数是'tridua',我们就应该知道那个词,但我们不知道。"\cite{carroll1896}

\begin{examplebox}[title=否定后件式在不同领域的应用]
\textbf{科学研究}:如果这个理论是正确的,那么实验应该产生X结果。实验没有产生X结果,因此这个理论是错误的。

\textbf{医学诊断}:如果患者得了A病,那么应该有Y症状。患者没有Y症状,因此患者没有得A病。

\textbf{日常推理}:如果今天下雨了,那么地面应该是湿的。地面不是湿的,因此今天没有下雨。
\end{examplebox}

该论证可以符号化为:

$$
\begin{aligned}
& p \supset q \\
& \sim q \\
& \therefore \sim p
\end{aligned}
$$

这个叫\textbf{否定后件式}的论证形式的有效性可用如下真值表表明:

\begin{center}
\begin{tabular}{|ccccc|}
\hline
$p$ & $q$ & $p \supset q$ & $\sim q$ & $\sim p$ \\
\hline
T & T & T & F & F \\
T & F & F & T & F \\
F & T & T & F & T \\
F & F & T & T & T \\
\hline
\end{tabular}
\end{center}

这里同样没有这样的代人例:在其中有这样一行,其前提 $p \supset q$ 和 $\sim q$ 都为真,而结论 $\sim p$ 为假。

\section*{D.一些常见的无效论证形式}
有两个无效的论证形式值得特别注意,因为它们与有效形式具有表面的相似性,因而经常迷惑粗心的作者或读者。在7.7节中讨论过的肯定后件谬误可以符号化为:

$$
\begin{aligned}
& p \supset q \\
& q \\
& \therefore p
\end{aligned}
$$

尽管这个论证形式在形式上有点类似肯定前件式,但这两个论证形式实际上很不相同。当然,该形式是无效的。例如,假如我们论证说,既然只要是美国公民自由联盟的成员就强烈支持言论自由,可知一个保护言论自由的人必定是美国公民自由联盟的支持者,这就犯了肯定后件谬误。

另一个无效形式叫否定前件谬误,它和否定后件式在形式上有点相像,可以符号化为:

$$
\begin{aligned}
& p \supset q \\
& \sim p \\
& \therefore \sim q
\end{aligned}
$$

这种谬误的一个例子是几年前一个纽约市市长候选人所用的一条竞选标

语:"如果不懂得赚钱,就不懂得这项工作——但亚伯懂得赚钱。"投票者被有意引导而未被陈述出来的结论是"亚伯懂得这项工作"一一个不能从所陈述的前提推出来的命题。

这两种常见的谬误都可以用真值表方法表明是无效的。在每种情形下,真值表中都有这样一行:这些谬误论证的前提都是真的,但结论是假的。

\section*{E.代入例与特征形式}
如我们早先在定义"论证形式"时所注意到的那样,一个给定论证可以是几个不同论证形式的代人例。本章开头所考察的那个有效析取三段论可以符号化为:

$$
\begin{aligned}
& R \vee W \\
& \sim R \\
& \therefore W
\end{aligned}
$$

它是下列有效论证形式的一个代人例:

$$
\begin{aligned}
& p \vee q \\
& \sim p \\
& \therefore q
\end{aligned}
$$

而且,它也是下列无效论证形式的一个代人例:

$$
\begin{aligned}
& p \\
& q \\
& \therefore r
\end{aligned}
$$

显然在最后一个形式中,从两个前提 $p$ 和 $q$ ,我们不能有效地推出 $r$ 。因此很清楚,一个无效的论证形式能够以一个有效的或一个无效的论证作为其代人例。所以,在确定某给定论证是否有效时,我们必须注意被探究论证的特征形式。只有论证的特征形式才准确地揭示了它的完整逻辑结构,正因为如此,我们才能够知道,如果一个论证的特征形式有效,那么该论证本身必定有效。

在上面所举例子中,我们看到了一个论证( $R \vee W, \sim R$ ,因此,$W$ )和以该论证为代人例的两个论证形式。这两个论证形式中的第一个( $p \mathrm{~V}$ $q, \sim p$ ,因此,$q$ )是有效的,因为该形式是给定论证的特征形式,它的

有效性确立了给定论证的有效性。第二个论证形式无效,但因为它不是给定论证的特征形式,所以,它不能被用来表明该给定论证无效。

应该强调指出:一个有效的论证形式只能以有效论证作为代人例。这就是说,一个有效形式的所有代人例必定有效。有效论证形式的有效性的真值表证明可以确证这一点,这表明,一个有效形式有前提为真而结论为假的代人例是不可能的。

\begin{center}
\fbox{\parbox{0.95\textwidth}{
\textbf{本节要点}
\begin{itemize}
\item \textbf{逻辑类推反驳法的理论基础}:
  \begin{itemize}
  \item 建立在现代逻辑的形式主义原理之上
  \item 论证的有效性完全取决于逻辑形式,与具体内容无关
  \item 提供客观的、不依赖主观判断的论证评估标准
  \item 体现了现代逻辑的认识论意义
  \end{itemize}
\item \textbf{论证形式的数学性质}:
  \begin{itemize}
  \item \textbf{抽象性}:抽象掉具体内容,只保留逻辑结构
  \item \textbf{生成性}:每个形式可生成无穷多个具体论证
  \item \textbf{分类性}:为论证提供分类标准和有效性特征
  \item \textbf{可判定性}:存在机械化决定程序判定有效性
  \item 有效性和无效性是论证的纯粹形式特征
  \end{itemize}
\item \textbf{真值表方法的数学原理}:
  \begin{itemize}
  \item \textbf{有穷性原理}:$n$个变元只有$2^n$种真值组合
  \item \textbf{完备性原理}:穷尽所有可能的真值组合
  \item \textbf{机械性原理}:完全机械化的构造和解读过程
  \item \textbf{可靠性原理}:准确反映论证形式的有效性
  \item 历史发展:维特根斯坦、波斯特的贡献及现代应用
  \end{itemize}
\item \textbf{基本有效论证形式的深入分析}:
  \begin{itemize}
  \item \textbf{析取三段论}:$p \vee q, \sim p, \therefore q$
    \begin{itemize}
    \item 基于排斥原理和完备性
    \item 体现排除法思维,实用性强
    \end{itemize}
  \item \textbf{肯定前件式}(Modus Ponens):$p \supset q, p, \therefore q$
    \begin{itemize}
    \item 形式逻辑的基石,追溯到斯多葛学派
    \item 直觉合理性强,普遍适用,构造性特征
    \end{itemize}
  \item \textbf{否定后件式}(Modus Tollens):$p \supset q, \sim q, \therefore \sim p$
    \begin{itemize}
    \item 反证性质,科学方法论基础
    \item 诊断功能强,批判性思维工具
    \item 在科学、医学、日常推理中广泛应用
    \end{itemize}
  \end{itemize}
\item \textbf{常见无效论证形式}:
  \begin{itemize}
  \item 肯定后件谬误:$p \supset q, q, \therefore p$
  \item 否定前件谬误:$p \supset q, \sim p, \therefore \sim q$
  \item 与有效形式的表面相似性及其迷惑性
  \end{itemize}
\item \textbf{代入例与特征形式}:
  \begin{itemize}
  \item 一个论证可以是多个论证形式的代入例
  \item 只有特征形式才准确揭示论证的完整逻辑结构
  \item 有效论证形式的所有代入例必定有效
  \item 特征形式的有效性决定具体论证的有效性
  \end{itemize}
\end{itemize}
}}
\end{center}