\section{第8章概要}

\chaptersummary{
本章介绍了现代符号逻辑的基础知识,讨论了如何使用符号语言来分析和评估论证。

\logicemph{8.1节}讨论了现代逻辑使用符号语言的原因和优势。符号语言避免了自然语言的缺陷,能够更精确地表述论证,并简化复杂的推理过程。

\logicemph{8.2节}介绍了命题逻辑中的五个基本\logicterm{真值函项联结词}:否定、合取、析取、条件和双条件。这些逻辑联结词是分析复合命题结构的基础工具。

\logicemph{8.3节}解释了\logicterm{真值表}的构造和应用,展示了如何通过真值表分析复合命题的真值条件和逻辑关系。

\logicemph{8.4节}阐明了\logicterm{形式有效性}的概念,并探讨了使用逻辑类推进行反驳的方法。

\logicemph{8.5节}讨论了三种特殊类型的真值函项陈述:\logicterm{重言式}、\logicterm{矛盾式}和\logicterm{偶真陈述}。

\logicemph{8.6节}介绍了真值函项陈述和论证的翻译技巧,展示了如何将自然语言表述转换为精确的符号形式。

\logicemph{8.7节}探讨了真值函项逻辑系统的局限性,特别是它无法分析命题内部结构的问题。

\logicemph{8.8节}讨论了三大"思想法则":\logicterm{矛盾律}、\logicterm{排中律}和\logicterm{同一律},以及它们在逻辑学中的地位和意义。

通过本章的学习,我们掌握了使用符号逻辑分析论证的基本技能,为后续章节中更深入的逻辑分析奠定了基础。
}

\printbibliography[heading=subbibliography,title={第8章参考文献}]