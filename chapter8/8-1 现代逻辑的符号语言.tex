\section{现代逻辑的符号语言}

\begin{logicbox}[title=引言]
本节介绍\logicterm{现代符号逻辑}的基本概念和优势。与古典逻辑不同,现代逻辑使用专门的符号语言来避免自然语言的缺陷,能够更精确地表述论证,并简化复杂的推理过程。我们将学习为什么\logicterm{符号逻辑}是分析演绎论证的强大工具。
\end{logicbox}

我们一直在寻求对演绎论证进行分析和评估的技术。\logicterm{演绎理论}旨在提供这样的技术,它已经发展出两个不同的分支来做这件工作:此前三章所考察的是\logicterm{经典逻辑}或\logicterm{亚里士多德型逻辑},本章和下两章的主题则是\logicterm{现代符号逻辑}。

然而,论证的分析和评估经常因其表述语言的特性(如英语或任何其他自然语言的特性)而非常困难。自然语言使用的语词可能是模糊的或歧义的,论证的结构可能是含混的,比喻和习语可能会引起混淆或误导,诉诸情感可能会引起混乱等,这些问题在第一部分已经探讨过了。要避免这些困难就要直接进人论证的逻辑核心,为此逻辑学家们构造了一种能避免自然语言缺陷的\logicterm{人工符号语言}。使用这种符号语言能\logicemph{精确地}表述论证。

\begin{examplebox}[title=符号语言的优势]
符号也能便利我们对论证的思考。"由于符号系统之助,"一位杰出的现代逻辑学家写道,"我们几乎用眼睛就可以机械地进行推理转换,否则,这种转换本来要求大脑有很高的智能。"\cite{quine1940} 这似乎有点悖谬,但符号语言确实可以帮助我们不需大伤脑筋就能完成某些智力活动。
\end{examplebox}

古代的和古典的逻辑学家们也承认某种特殊逻辑记号的价值。亚里士多德在自己的分析中就使用了变项,而如前面几章所表明,改进了的\logicterm{亚里士多德型逻辑}也以很复杂的方式使用了符号。\cite{aristotle-logic} 20 世纪又有很大的改进。

\begin{theorembox}[title=现代逻辑的核心特征]
在\logicterm{现代逻辑}中,处于核心地位的不是\logicterm{三段论}(如亚里士多德传统上的),而是\logicterm{逻辑联结词},它们是每个演绎论证,不管是不是三段论,在其构成要素之间的关系中所必须运用的。命题和论证的内在结构是现代逻辑关注的焦点。要理解这种结构,我们必须首先掌握现代逻辑分析中所使用的一些特殊符号。
\end{theorembox}

\logicterm{现代符号逻辑}不受演绎论证要转换成\logicterm{三段论}形式的制约(\logicterm{亚里士多德型逻辑}受这种制约)。正如我们在第 7 章所见,那种工作是很费力的。不必进行这种转换使得我们可以更直接地追求演绎分析的目标。下面给出的现代逻辑的符号记法是分析论证的特别有力的工具。使用这种记法我们可以更全面地达到演绎逻辑的核心目标:区分\logicemph{有效}论证和\logicwarn{无效}论证。

\begin{center}
\fbox{\parbox{0.95\textwidth}{
\textbf{本节要点}
\begin{itemize}
\item \logicterm{现代符号逻辑}是\logicterm{演绎理论}的重要分支,与古典逻辑并列
\item 自然语言存在的问题:
  \begin{itemize}
  \item 语词模糊或歧义
  \item 论证结构含混
  \item 比喻和习语引起混淆
  \item 情感因素造成干扰
  \end{itemize}
\item \logicterm{人工符号语言}的优势:
  \begin{itemize}
  \item 避免自然语言的缺陷
  \item \logicemph{精确}表述论证
  \item 简化复杂的推理过程
  \end{itemize}
\item 现代逻辑与古典逻辑的区别:
  \begin{itemize}
  \item 核心是\logicterm{逻辑联结词}而非\logicterm{三段论}
  \item 关注命题和论证的内在结构
  \item 不需要将论证转换为\logicterm{三段论}形式
  \end{itemize}
\item 符号记法是更全面地实现演绎逻辑目标的强大工具
\end{itemize}
}}
\end{center}