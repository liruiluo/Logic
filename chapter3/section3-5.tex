\section{内涵定义}

\begin{quotation}
\textit{内涵定义通过词项所表示的属性或特征来解释其意义,是定义概念的核心方法。理解各种内涵定义的方式有助于我们更准确地表达概念,并有效地解决因概念混淆而产生的争议。}
\end{quotation}

如前所述,词项的内涵,由词项指谓的所有对象共有且仅为这些对象特有的属性构成。例如,如果"椅子"的内涵由属性"单个的座位并且有一个靠背"构成,那么就意味着每一张椅子都是具有靠背的单个座位,并且只有椅子才是具有靠背的单个座位。

\subsection{内涵的三种含义}

但是,内涵定义这个概念被三种不同含义的内涵区别弄得复杂了,这三种不同含义的内涵是\textbf{主观内涵}、\textbf{客观内涵}和\textbf{规约内涵}。

对说话者来说,词的主观内涵就是他认为该词指谓对象所具有的属性集。这种集合显然是因人而异的,甚至对同一个人也因时而异。毕竟,我们所感兴趣的是词语的公共意义,而不是它们的私人解释。

客观内涵是词项外延的所有对象共同拥有的属性全集。例如,"圆"这个词的客观内涵可以拥有圆的各种普遍特性(例如,圆包围的面积比其他任何封闭的与其具有相等周长的平面图包围的面积都大),而我们很多人在运用这个词时完全没有注意到这些普遍属性。要知道大多数词项的指谓对象所共同拥有的全部属性,就要求完完全全的全知,而由于没有人能够具有这样的全知,所以客观内涵就不是我们所追求的公共意义的解释。

然而,对大多数普遍词项来说,很显然,必定存在可为公众使用并广泛理解的内涵,即既不是主观的也不是客观的内涵。因为我们确实可以相互交流,而且我们的确常常是在一般的用法上来理解词项的。词项之所以具有稳定的意义,乃是因为对任何对象来说,在决定其是否某词项外延的一部分时,我们都同意使用同样的标准。比如,从规约的角度看,按照 "圆"这个词的通常用法,圆之所以为圆,就在于它是这样一种封闭的平面曲线,其线上所有的点到一个叫做圆心的点的距离都相等。通过非正式的承诺,我们建立了普遍词项的规约内涵。就定义之目的而言,这是内涵的最为重要的含义,因为它既是公共的,也不为使用它而要求全知。实际上,"内涵"这个词通常就是用来指"规约内涵"的——这也将是我们的用法,除非另有说明。

\subsection{识别规约内涵的方法}

人们实际上是怎样定义一个词的呢?识别词的规约内涵,即这个词的指谓对象为人们认同的共同与特有属性,要运用哪些方法呢?常用的方法有如下几种。

\subsection{同义定义}

最简单且最常用的方法(但功能有限)就是提供另一个意义已经被理解的词,而且它与被定义的词具有相同的意义。两个具有相同意义的词被称做\textbf{"同义词"},因此这种定义就被称做\textbf{同义定义}。词典,尤其是较小的词典,就主要依靠这种方法来定义词项。例如,袖珍词典可以将"谚"(adage)定义为"谚语"(proverb),"腼腆"(bashful)定义为"害羞"(shy),等等。当需要解释另一种语言的词义时,同义定义特别有用,往往是不可或缺的。在法语中,"chat"意指"猫";在西班牙语中,"amigo"意指"朋友",等等。人们学习外语词汇要依赖于同义定义。

同义定义是一种定义词项的好方法,它容易、方便而实用,但它也有很大局限性。很多词汇并没有真正的同义词,因而同义定义就常常不够完全精确并引人误解。有一句意大利谚语就是基于这种认识而来的:翻译者就是窜改者。

同义定义的一个更严重的局限是:如果我们寻求定义的词所表示的概念对我们来说完全是外来的和令人费解的,那么,其任何简单的同义词都将像被定义项本身一样令人费解。例如,需要"tylotoxea"的定义而又完全不熟悉这个名词指什么的人,当给出的定义是一个简单的同义词时,就不会有多大帮助。你问道,什么是"tylotoxea"?哦,它不是别的,就是 tylostyle。显然,在这个事例中,还需要提供比单独一个词所能提供的更多解释。当要给一个为人所知的但又模糊不清的词项提供一个比较精确或充分的定义时,也会出现同样的困难。因此,当寻求的是一个理论定义或精确定义(3.2节已解释过)时,同义词是不可能满足要求的。

\subsection{操作定义}

\textbf{"操作定义"}(operational definition),是诺贝尔奖得主物理学家 P.W.布里奇曼在他那本有影响的《现代物理学的逻辑》(1927)一书中首次使用的一个术语;为了把被定义项与一组可描述的动作或操作联系在一起,一些科学家就引进了它。例如,在爱因斯坦的相对论获得成功并被广泛接受之后,"空间"和"时间"就不能再按照牛顿所用的那种抽象方式来定义了。于是,有人提出操作地定义它们,即以在测量距离和时间中所使用的操作方法来定义之。词项的操作定义就是指这个词项被正确地运用到某个给定场合,当且仅当在那个场合中,特有的操作行为会产生特有结果。于是,给定的长度数值就可以通过参考特有测量程序的结果而操作地定义出来,如此等等。在操作定义中,仅仅涉及公共的可重复的操作。

有些社会科学家也试图把这种定义方法结合到他们的研究领域中去以避免混淆和分歧,而这些混淆和分歧已经使一些关键术语的传统定义备受质疑。例如,有些心理学家已经寻求用仅仅涉及行为或者心理学的观察的操作定义来替代"感觉"和"心灵"的抽象定义;在心理学和其他社会科学中,依靠操作定义已经有了与行为主义相联系的倾向。有的极端经验主义者坚持认为,一个词项有意义仅当它能够操作定义;但评价这种断言已超出了本书的范围。

\subsection{属加种差定义}

在不可用同义定义也不适合用操作定义的地方,我们通常可以使用\textbf{"属加种差定义"}来解释一个词项的规约内涵。这种方法也被称做"划分定义"、"分析定义"、"属种定义",或者直接称做"内涵定义"。有人错误地认为这是唯一的一种"真实"定义,但这种方法确实比任何其他方法都有更广泛的可应用性。

通过属加种差来定义词项的可能性取决于有很多属性的复杂事实,即它们可以分析为两个或更多的其他属性。这种复杂性和可分析性可以依据类这个概念而得到最好的说明。

\subsection{属和种的关系}

具有多个元素的类可以把它们的元素分为子类。例如,所有三角形这个类可以分为三个非空的子类:等边三角形、二等边的三角形和不等边三角形。\textbf{"属"}和\textbf{"种"}这两个词常常用于这种关系:被分为子类的类是属,而各种各样的子类都是种。就我们这里的用法而言,"属"与"种"这两个词是相对的,就像"父母"与"子女"一样。在关系上,相同的人是他们孩子的父母亲,但又是他们自己父母亲的子女;同样,同一个类在关系上可以是它的子类的属,也可以是它所从属的更大类的一个种。这样,所有三角形的类,相对于不等边三角形这个种就是一个属,而相对于多边形这个属它则是一个种。逻辑学家对"属"和"种"这两个词作为相对术语的用法,与生物学家把它们作为严格术语的用法是不同的,我们不应当混淆二者。

由于一个类就是具有某些共同特征事物的一个汇集,所以给定的属的所有元素都具有某些共同特征。例如,多边形这个属的所有元素都具有这样的特征,即由直线线段连接而成的封闭平面图形。这个属可以分成不同的种或子类,因此每个子类的所有元素都具有更进一步的共同属性,而这些共同属性却不为任何其他子类的元素所共享。多边形这个属分为三角形、四边形、五边形和六边形等等。多边形这个属的每个种都与其他所有的种不同;六边形这个子类的元素与任何其他子类的元素之间的特有差异是,仅有六边形这个子类的元素恰好具有六条边。一般的,一个给定的属的所有种的元素共享某些属性,这些共享属性使它们成为该属的元素;但是,任何一个种的元素都共享某些更进一步的属性,而这些属性将它们与该属的任何其他种的元素区分开来。那种用来区分它们的性质叫\textbf{"种差"}。如,具有六条边就是六边形这个种与多边形这个属的所有其他种之间的种差。

\subsection{属加种差定义的应用}

在这个含义上,六边形的属性可以分析为多边形的属性和六条边的属性。对于不知道"六边形"这个词或任何它的同义词的意义,但又的确知道"多边形"、"边"和"六"等词的意义的人来说,"六边形"这个词的意义就可以用"属加种差"的定义而得到解释:

\begin{displayquote}
"六边形"这个词意思是"具有六条边的多边形"。
\end{displayquote}

另一个例子是"质数"的定义:

\begin{displayquote}
质数就是任何大于 1 而且又仅能为它自己或 1 整除的自然数。
\end{displayquote}

可见,通过属加种差来定义一个词项要经过两步:首先,必须找出一个属,即包括被定义的那个种的较大的类;接着必须找出种差,即将被定义的那个种的元素与那个属的其他所有种的元素区分开来的性质。在上例中,属就是一个比 1 大的自然数的类:$2,3,4$,等等;其种差是仅能为它自己或 1 整除的性质。属加种差定义可以非常简明并且常常是极其有用的。

属加种差定义的另一个例子是古代人将"人"的意义定义为"有理性的动物"。这里,属是"动物","人"是其下的种,通过理性把人与其他所有的种区别开来。在这种情况下,人们也可以把"所有的有理性的生物"类看做属,而把"动物"类看做种差。对于把某个类而不是把另一个类看做属来说,虽然可能存在超逻辑的原因,但从逻辑的观点看,这种顺序并不是绝对的。

\subsection{属加种差定义的局限性}

属加种差来定义的方法也有其局限性。首先,这种方法仅能运用于那些暗含有复杂属性的词汇。如果是简单得不可再分析的属性,那么暗含这些属性的词汇就不能由属加种差来定义。有人提出,人们所感知的具体光谱段的颜色性质就是这种简单属性的范例。是否存在这样不可再分析的属性仍然是一个未解决的问题,但是如果这种属性存在,那就限制了属加种差定义的运用。第二种局限性与表达"大全"(universal)性质的词汇有关,如"存在"、"本体"、"存在物"和"客体"等。这些词都不能通过属加种差的方法来定义;例如,所有本体的类就不是某个更大的属的一个种;大全类(universal class)是最高的类,或者有人所谓的最高的属。这同样适用于那些指称形而上学的最终范畴的词汇,诸如"物质"或"性质",等等。然而从这种定义方法的实际运用角度看,这些局限性不是很重要的。

内涵定义,尤其属加种差定义,可以满足构造定义的各种目的:它们可以帮助人们消除歧义、减少模糊、阐释理论,甚至影响态度。它们也可用于增加和丰富人们的词汇。在 3.2 节,我们注意到,在达到这些不同目标的过程中,要区分五种不同的定义:词典定义、规定定义、精确定义、理论定义和说服定义。对这些种类的每一个来说,都可以运用内涵定义的方法。

\begin{center}
\begin{tabular}{|l|l|}
\hline
\multicolumn{2}{|c|}{五种定义} \\
\hline
\multicolumn{2}{|l|}{\begin{tabular}{l}
1.规定定义 \\
2.词项定义 \\
3.精确定义 \\
4.理论定义 \\
5.说服定义 \\
\end{tabular}} \\
\hline
\multicolumn{2}{|c|}{定义词项的六种方法} \\
\hline
\begin{tabular}{l}
A.外延方法 \\
1.示例定义 \\
2.实指定义 \\
3.准实指定义 \\
\end{tabular} & \begin{tabular}{l}
B.内涵方法 \\
4.同义定义 \\
5.操作定义 \\
6.属加种差定义 \\
\end{tabular} \\
\hline
\end{tabular}
\end{center}

\begin{center}
\fbox{\parbox{0.9\textwidth}{
  \centering
  \textbf{内涵定义的类型与特点}\\
  内涵的三种含义:主观内涵(个人理解)、客观内涵(全部共有属性)、规约内涵(公共约定);\\
  同义定义:通过具有相同意义的词来定义,简单但局限性大;\\
  操作定义:通过可观察、可重复的操作来定义,在科学中广泛应用;\\
  属加种差定义:通过属(上位类)和种差(区别性特征)来定义,应用最广;\\
  局限:不能定义简单不可分析的属性和表示大全性质的词汇。
}}
\end{center} 