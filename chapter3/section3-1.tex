\section*{3.1 论争、言辞之争与定义}
语言是一种非常复杂的设施,是人类最重要的交流工具。但是,当语词被漫不经心或错误地使用时,这种设施就会变成我们的负担。在 2.5 节中,我们阐释了冲突双方的歧见既可能是信念的也可能是态度的;而我们看到,两者之中任何一种对立都可能是实质歧见。但是,也存在这样的情况:表面上的歧见实际上却不是真正的歧见,而仅仅是误解或词汇误用的结果。在前一章中,我们考察了不同类型的实质歧见;在这里,我们转而讨论不同类型的论争,分辨其是否具有真正的分歧。

必须区分出三种不同的论争。第一种是明显的实质论争,在这种论争中,各方或者在信念上或者在态度上,明确地毫不含糊地对立。例如,如果美国佬(Yankees)赢得了世界联赛,虽然在胜利者本身的认同上没有争论,但如果 A 为此高兴而 B 为此恼怒,那么他们的态度歧见(不可能解决的问题)就是显然的,或许甚至是激烈的。在另一种语境中,如果 A坚持认为巴拿马运河的太平洋人口比其大西洋人口更靠东,而 B 则否定如此,那么他们的争论就不是在态度上而是在事实上;而一张好地图就可以平息这个论争。 ${ }^{[1]}$ 无论是态度上的还是信念上的,这种论争总是包含某种实质歧见。将论争双方区别开来的不仅仅是语言,在他们对事实的断定上或对事实的评价上亦存在实质差别。因此,这样的实质论争不能通过定义或任何简单的语言调整来解决。

当然,可以存在关于词汇本身的实质论争,例如,某个单词怎样拼写或者怎么使用;也可以存在关于态度本身的实质论争,例如,是否某第三方不友好或仅是差怯。事实可以是物理的或地理的,也可以是语言的或心理的,而且各方可以在任何种类事实上产生歧见。但是,如果论争确实是关于某个事实的,那么它就是实质的,并可以通过确认某些事实而得到解决。

然而,也存在第二种类型的论争,即纯粹的言辞之争;在这种论争中,双方之间根本没有实质歧见,然而却好像是具有歧见。语言的误解或误用可能是这里的症结所在。当论争者的信念表达中某个关键语词有歧义,而这种歧义又遮蔽了双方并没有实质对立的情况时,言辞之争就会产生。若争论某方误用一个重要词语,或者争论的某个核心语词或短语具有

不同含义,而这些含义可能同等合法但产生了不应有的混淆,或者由于各方对语词或短语的用法都对但含义不同,而这一点又没有被明确认知,就可能产生这种表面的言辞之争。

言辞之争并非总是容易发现,但一旦识别了它,通过具体化歧义语词或短语的不同含义,就可以相当容易地获得解决。在这种语境中,良好的定义对相互理解是非常关键的。

威廉•詹姆士给出了这种言辞之争的一个经典例子:

几年前,我随野营队一起在山上露营。当我独自散步返回时,发现每个人都参加了一场激烈的形而上学论战。争论主题是一只松鼠。设想一只松鼠抓附在树干一侧,而一个人站在树的另一侧;那人绕树迅速转动以试图看到松鼠,但无论他转多么快,松鼠在相对的方向都以同样快的速度转动,在它自己和那人之间总隔着那棵树,因此使他看不到松鼠。作为结果的形而上学问题是:这个人是否绕着松鼠走了一圈?确确实实,他绕树走了一圈,而且松鼠就在树上;但是,他绕松鼠走了一圈吗?原野中的讨论持续良久,直到变得乏味。每个人都赞同一种观点并固执己见,并且双方的人数势均力敌。因此,当我出现时,每方都希望我加入以便成为多数派。 ${ }^{[2]}$

显然不难看出,这场论争的双方之间不存在实质歧见,这正是詹姆士讲这个故事所要说明的。所有论争者对松鼠和树的态度都是中立的,都完全理解和赞同给定事例的所有事实。因此,在这个事例中(许多其他事例中也同样),争论不过是言辞之争。詹姆士继续写道:有绕它走一圈,因为松鼠做了相对运动,它始终保持着将其腹部

对着那个人,而将背部朝着外面。做出这种区分,就没有什么可争论的了。你们都又对又不对,就看你们对"绕走一圈' 这个动词实际上是怎么理解的。"[3]

解决这个论争不要求新的事实,并且那样做也不可能有帮助;它需要的仅仅是詹姆士所提供的东西:对争论中一个关键语词的不同意义做出区分。使用"绕走…圈"这个词语的不同定义,这个争论就消失了;故这种歧见根本不是实质的。无论何处,如果论争纯粹是言辞之争,我们就可以通过提供能够消除关键歧义的定义来解决。在这种情况下,我们可表明论争各方并不是真正的相互对立;它们可能仅仅是运用相同的词汇的不同含义或意义来维护不同主张或者运用不同语词来维护相同主张罢了。一旦确定了不同意义以及源于对不同意义的使用而涉及意义的不同主张,那么双方之间就不会再有什么论争。 ${ }^{[4]}$

第二种论争,指那些表面上是言辞的但实际上是实质的论争。当双方互相误解了对㡰词语的用法时会出现混淆,而这种混淆可以得到识别。但是,有时也会出现这样的争执远超出语词不同用法的范围。在这种情况下,仅仅解决歧义问题不会平息论争,因为争论双方之间还存在某种实质歧见:可能在信念上,更可能在态度上。

举例来说:对给定的有露骨性活动镜头的影片是否应该作为"色情作品"来处理,双方可能产生争执。一方坚持认为,它的露骨使它成了邪恶的色情作品;另一方则坚持,考虑到其细腻的情感和美学价值,它是真正的艺术,根本不是什么色情作品。显然,双方的歧见在于"色情作品"一词的意义;但是,即使言语的不同得到了充分理解并清除了所有歧义,双方很可能对影片仍然存在实质歧见;论争实际上不是真正关于"色情作㗊"这个词的适用性的,他们的歧见更深人地涉及影片的性感露骨性质是否造成影片的好与坏。

第三种论争有时被称为"标准"之争或"概念"之争。论争双方对某个关键词语的运用有着不同标准,也就是说,该词语指谓的是不同概念。因而,在不同标准的明智或正确性之下,各方就处于尖锐冲突之中。比如在上例中,即使双方都认识到他们有歧义地使用了"色情作品"这个词,甚至词语的歧义已经得到阐明和区分,但各方都仍可能声称其对手误用了他们的标准来确定什么是色情作品。一方可能坚决主张,如果影片包含露

骨的性活动场景,便可以将它划归为色情作品;而对手则可能回应,那种划归是一种概念错误。这种论争表面上仅仅是言辞之争,但在表面之下,却是非常实质的论争。

为帮助人辨识和理解论争的这些不同种类,我们可以做一个有用的 "流程"。一旦我们确定存在某种论争,我们就可以问:"出现歧义了吗?"如果对该问题回答"没有",那么我们得到的论争就是类型—1(显然是实质的)。如果回答"出现了",那么我们就问第二个问题:"清除歧义可以消除对立吗?"如果对此问题回答"可以",那么我们得到的论争就是类型——2(纯粹言辞之争)。如果对第二个问题回答"没有",那么我们得到的论争就是类型——3(表面上是言辞的但实际上是实质的)。

这三种论争可以概述如下:\\
1.在明显的实质论争中,不存在言辞歧义,争论双方的确有歧见,或是在态度上或是在信念上的歧见。

2.在纯粹言辞之争中,存在言辞歧义但根本没有实质歧见。\\
3.在表面上是言辞的但实际上是实质的论争中,既存在言辞歧义又有歧见论争,或在态度上或在信念上,或者是关于事实的或者是关于某些语词的运用标准的。 