\section{第3章概要}

\begin{center}
\fbox{\parbox{0.95\textwidth}{
解释词项的意义就是给出它的定义。在本章中,我们讨论了几种定义及其用法,以及构建定义的方法和运用这些方法的规则。

\textbf{3.1 论争、言辞之争与定义}解释了三种论争:
1.明显的实质争论,其中没有语词歧义,而且论争双方的确在态度上或信念上对立。
2.纯粹言辞之争,其中出现语词歧义,但根本没有实质歧见。
3.表面上是言辞的但实际上是实质的论争,其中既存在语词歧义,也存在论争双方在态度上或在信念上的歧见。

\textbf{3.2 定义的类型和论争的解决}首先解释了定义总是符号的定义,并且引进了术语被定义项(被定义的符号)和定义项(用来解释被定义项意义的符号)。还在五种定义及其基本用法中进行了区分:

1.规定定义,把一个意义指派给某个符号。规定定义不是报道,因而既不真也不假;它是运用被定义项来意指定义项指谓事物的建议、解决、请求或工具。

2.词典定义,它报道被定义项已经具有的意义,因而它可以或对或错。

3.精确定义,它超出了平常用法,用于消除与临界状况有关的麻烦的不确定性。其被定义项有一个现存的意义,但这个意义是模糊的;增添什么可以达至精确性,部分上是个规定问题。

4.理论定义,它寻求对它的适用对象精确表述一个理论上足够或科学上有用的描述。

5.说服定义,它运用表达性语言而不是信息性语言来寻求影响态度或激发情感。

在这五种定义中,前两种(规定定义和词典定义)主要用于消除歧义;第三种(精确定义)主要用于降低模糊性;第四种(理论定义)用于促进理论理解;而第五种(说服定义)用于影响行为。

\textbf{3.3 外延和内涵}解释了普遍词项指谓其可以正确适用的多个对象。这些对象的汇集构成该词项的外延。说明了为词项外延中的所有对象并且仅为那些对象所共有的属性集就是该词项的内涵。词项的内涵决定其外延,但外延却不能决定内涵;因此,几个词项可以具有不同内涵而外延却相同;但外延不同的词项却不可能具有相同内涵。

\textbf{3.4 外延定义}解释了怎样利用普遍词项的外延来构造外延定义;外延定义有几种类型,其局限性也被揭示出来:

1.列举定义,即在定义中列出或给出词项指谓对象的范例。
2.实指定义,在定义时,我们用手指出或以姿势标明被定义项的外延。
3.准实指定义,在定义中,姿势或手指的指示伴有一些其意义被认为是已为人所知的描述短语。

\textbf{3.5 内涵定义}解释了怎样利用普遍词项的内涵来构建内涵定义;内涵定义也有几种类型,其局限性也被揭示出来:

1.同义定义,在定义中提供另一个其意义已为人所知的词,这个词与被定义的词具有相同意义。
2.操作定义,它表明词项正确运用于一个给定场合,当且仅当,在该场合下特有的操作行为产生特有结果。
3.属加种差定义,首先要找出一个属,被定义项所指代的种是该属的一个子类;然后找出属性(或种差),即把该种的分子与属的所有其他种的分子区分开来的那种属性。

内涵定义的方法可以用于构建 3.2 节中五种定义的任何一种:规定定义、词典定义、精确定义、理论定义和说服定义。

\textbf{3.6 属加种差定义的五条规则}明确表述和解释了传统的属加种差定义的五条规则:
1.定义应当揭示种的本质属性。
2.定义不能循环。
3.定义既不能过宽又不能过窄。
4.定义不能用歧义的、䀲涩的或比喻的语言来表述。
5.定义在可以用肯定的地方就不应当用否定定义。
}}
\end{center}

\section{注释}

$[1]$ A 是正确的,巴拿马运河的太平洋入口确实是在其大西洋人口的东面。

$[2]$ William James,Pragmatism(1907).

$[3]$ lbid.

$[4]$ 为掩盖实质论争,语词有时也被故意地用做两种含义以避免争执。拉比- A•J•鲁丁(Rabbi A.J.Rudin)把"有趣的"(interesting)释义为"英语中有争议的最该县咒的词",经常被大批的宗教会众使用以掩盖说话者的真正意见。鲁丁写道: "当用于布道时,'有趣的'的通常意思是'我患有失眠症',或者'我认为你说得精彩极了'。在其最为隐䀲的意义上,'有趣的'意指职员鲁葬……表达说话者不赞同的观点。"Religious News Service,January 1992。

$[5]$ 1991年,度量衡总委员会(General Committee on Weights and Measures)对它们进行了规定定义。该委员会是一家国际机构,其管理领域是科学的单位。另外,一千亿亿也叫一"zepto",一万亿亿也叫一"yocto"。

$[6]$ 这个新术语是在纽约城由普林斯顿大学的约翰•阿奇贝尔德•威勒(John Archibald Wheeler)博士在1967年的空间研究组织的一次会议上引进的。

$[7]$ "夸克"出现在詹姆斯-乔伊斯(James Joyce)的小说(Finnegan's Wake)的 "Three Quark for Muster Mark"一行文字中;但是,盖尔曼博士报告说,他在看到那个名字之前就已经选择了它,他仅仅是根据乔伊斯拼写了它。

$[8]$ See The Chronicle of Higher Education, 30 May 1993.

$[9]$ 一匹 600 千克(1 323 磅)重的真马的功率要比这个数大得多,估计大约为 18000 瓦。因此,一辆 200 马力的汽车大约相当于 8 匹真马的功率。

$[10]$ 与 1 升水的质量相同的单位长期被接受为 1 "千克"的定义。但是 1 千克现在已经更精确地定义为"与巴黎附近的保险库内的金属块具有相同质量的单位"。然而,人们仍在为"千克"寻找更加精确的定义,一种以一定数量的某种原子质量为依据的精确定义。

$[11]$ Cali fornia v.Hodari D., 499 U.S.621, 1991.

$[12]$ American Civil Liberties Union v.Reno, 929 Fed.Supp.824, 11 June 1996.

$[13]$ "Defining Abortion a Tricky Business,"Honolulu Advertise, 14 February 1970.

$[14]$ 逻辑学家有时用"含义"(connotation)这个词来取代内涵,并较为普遍地使用"指称"(denotation)这个词来取代外延。但是,"指称"在日常话语中有其他更加普通的用法;大多数时候,它意指一个词项的情感意义,因此,此处引进它并无帮助。正如我们这里所做,通过运用词项内涵和外延来处理这种关键的区分,什么也不会丢失,并且可以避免一些混淆。

$[15]$ 内涵与外延之间非常有用的区分是由坎特伯雷的圣安瑟伦(St.Anselm of Canterbury,1033~1109)引进并强调的,他以他的"本体论论证"而著称,上述那个谬误的论证与他的论证并不相同。请参见 Wolfgang L.Gombocz,"Logik and Existenz in Mittelater",Philosophische Rundschau(1997)。

$[16]$ John P.Sisk,"Art,Kitsch and Politics",Commentary,May 1988.

$[17]$ Jay Livingston,Compulsive Gamblers(New York:Harper \& Row,1974), p. 2.

$[18]$ W.H.Voge,"Strees-The Neglected Variable in Experimental Pharmacology and Toxicology,"Trends in Pharmacological Science,January 1987.

$[19]$ Herbert Spencer,Principles of Biology, 1864.

$[20]$ Samuel Johnson,Dictionary of the English Language, 1755.

$[21]$ Ambrose Bierce,The Devil's Dictionary, 1911. 