\section{第3章概要}

\chaptersummary{
解释词项的意义就是给出它的定义。在本章中,我们讨论了几种定义及其用法,以及构建定义的方法和运用这些方法的规则。

\logicemph{3.1 论争、言辞之争与定义}解释了三种论争:
1.\logicterm{明显的实质争论},其中没有语词歧义,而且论争双方的确在态度上或信念上对立。
2.\logicterm{纯粹言辞之争},其中出现语词歧义,但根本没有实质歧见。
3.\logicterm{表面上是言辞的但实际上是实质的论争},其中既存在语词歧义,也存在论争双方在态度上或在信念上的歧见。

\logicemph{3.2 定义的类型和论争的解决}首先解释了定义总是符号的定义,并且引进了术语\logicterm{被定义项}(被定义的符号)和\logicterm{定义项}(用来解释被定义项意义的符号)。还在五种定义及其基本用法中进行了区分:

1.\logicterm{规定定义},把一个意义指派给某个符号。规定定义不是报道,因而既不真也不假;它是运用被定义项来意指定义项指谓事物的建议、解决、请求或工具。

2.\logicterm{词典定义},它报道被定义项已经具有的意义,因而它可以或对或错。

3.\logicterm{精确定义},它超出了平常用法,用于消除与临界状况有关的麻烦的不确定性。其被定义项有一个现存的意义,但这个意义是模糊的;增添什么可以达至精确性,部分上是个规定问题。

4.\logicterm{理论定义},它寻求对它的适用对象精确表述一个理论上足够或科学上有用的描述。

5.\logicterm{说服定义},它运用表达性语言而不是信息性语言来寻求影响态度或激发情感。

在这五种定义中,前两种(规定定义和词典定义)主要用于消除歧义;第三种(精确定义)主要用于降低模糊性;第四种(理论定义)用于促进理论理解;而第五种(说服定义)用于影响行为。

\textbf{3.3 外延和内涵}解释了普遍词项指谓其可以正确适用的多个对象。这些对象的汇集构成该词项的外延。说明了为词项外延中的所有对象并且仅为那些对象所共有的属性集就是该词项的内涵。词项的内涵决定其外延,但外延却不能决定内涵;因此,几个词项可以具有不同内涵而外延却相同;但外延不同的词项却不可能具有相同内涵。

\textbf{3.4 外延定义}解释了怎样利用普遍词项的外延来构造外延定义;外延定义有几种类型,其局限性也被揭示出来:

1.列举定义,即在定义中列出或给出词项指谓对象的范例。
2.实指定义,在定义时,我们用手指出或以姿势标明被定义项的外延。
3.准实指定义,在定义中,姿势或手指的指示伴有一些其意义被认为是已为人所知的描述短语。

\textbf{3.5 内涵定义}解释了怎样利用普遍词项的内涵来构建内涵定义;内涵定义也有几种类型,其局限性也被揭示出来:

1.同义定义,在定义中提供另一个其意义已为人所知的词,这个词与被定义的词具有相同意义。
2.操作定义,它表明词项正确运用于一个给定场合,当且仅当,在该场合下特有的操作行为产生特有结果。
3.属加种差定义,首先要找出一个属,被定义项所指代的种是该属的一个子类;然后找出属性(或种差),即把该种的分子与属的所有其他种的分子区分开来的那种属性。

内涵定义的方法可以用于构建 3.2 节中五种定义的任何一种:规定定义、词典定义、精确定义、理论定义和说服定义。

\textbf{3.6 属加种差定义的五条规则}明确表述和解释了传统的属加种差定义的五条规则:
1.定义应当揭示种的本质属性。
2.定义不能循环。
3.定义既不能过宽又不能过窄。
4.定义不能用歧义的、䀲涩的或比喻的语言来表述。
5.定义在可以用肯定的地方就不应当用否定定义。
}

\printbibliography[heading=subbibliography,title={第3章参考文献}]