\section{第3章概要}

\chaptersummary{
本章系统探讨了论争的性质、定义的类型和方法,为理解语言的精确性和有效交流奠定了理论基础。

\logicemph{3.1 论争的分类与识别}:
\begin{itemize}
  \item \logicterm{明显的实质论争}:不存在言辞歧义,存在实质性歧见,需通过解决信念或态度分歧来解决
  \item \logicterm{纯粹言辞之争}:存在言辞歧义但无实质歧见,可通过明确定义和消除歧义来解决
  \item \logicterm{表面言辞实质论争}:既有言辞歧义又有实质歧见,需同时解决语言问题和实质分歧
  \item \logicterm{识别方法}:使用两步诊断流程,先判断是否存在歧义,再判断清除歧义是否能消除对立
\end{itemize}

\logicemph{3.2 定义的类型体系}:
\begin{itemize}
  \item \logicterm{报告性定义}(词典定义):阐释既有词项的既有意义,可评价为真或假
  \item \logicterm{规定性定义}:赋予新词项以意义或赋予既有词项以新意义,不能评价为真或假
  \item \logicterm{精确定义}:消除既有词项的模糊性,需兼顾与现有用法的融贯性
  \item \logicterm{理论定义}:为阐明理论概念而提出的规定定义,促进理论理解
  \item \logicterm{说服定义}:为影响态度或行为而提出的带有情感色彩的定义
\end{itemize}

\logicemph{3.3 外延与内涵理论}:
\begin{itemize}
  \item \logicterm{外延}:词项可以正确适用的所有对象的集合,体现词项的指谓范围
  \item \logicterm{内涵}:词项所表示的属性或特征的集合,提供判断标准
  \item \logicterm{决定关系}:内涵决定外延,而非相反;词项可有相同外延但内涵不同
  \item \logicterm{反变规律}:当词项内涵增加时,其外延处于非递增状态
\end{itemize}

\logicemph{3.4 外延定义方法}:
\begin{itemize}
  \item \logicterm{列举定义}:列出或给出词项指谓对象的范例,存在完全列举困难和部分列举歧义
  \item \logicterm{实指定义}:用手指出或以姿势标明被定义项的外延,有地域限制和指示歧义
  \item \logicterm{准实指定义}:姿势指示伴有描述短语,但仍存在根本局限
  \item \logicterm{空外延问题}:某些有意义的词没有所指对象,无法通过外延定义
\end{itemize}

\logicemph{3.5 内涵定义方法}:
\begin{itemize}
  \item \logicterm{同义定义}:提供意义相同的词,简单有效但存在真正同义词稀缺的问题
  \item \logicterm{操作定义}:通过特定操作行为的结果来定义,适用于科学概念
  \item \logicterm{属加种差定义}:最重要的内涵定义方法,通过属和种差来精确定义
\end{itemize}

\logicemph{3.6 属加种差定义的五条规则}:
\begin{itemize}
  \item \logicterm{本质性}:定义应当揭示种的本质属性,而非偶然属性
  \item \logicterm{非循环性}:定义不能循环,避免用被定义项来定义自身
  \item \logicterm{适度性}:定义既不能过宽又不能过窄,外延必须完全一致
  \item \logicterm{清晰性}:定义不能用歧义的、晦涩的或比喻的语言表述
  \item \logicterm{肯定性}:在可以用肯定的地方就不应当用否定定义
\end{itemize}

\logicwarn{理论意义与实践价值}:
\begin{itemize}
  \item 准确识别论争类型有助于避免在虚假争议上浪费时间,提高交流效率
  \item 理解不同定义类型的特点和适用条件,有助于选择合适的定义策略
  \item 掌握外延与内涵的关系,为逻辑推理和概念分析提供理论工具
  \item 熟练运用各种定义方法,是学术写作和科学研究的基本技能
\end{itemize}
}

% 参考文献将在主文档末尾统一显示