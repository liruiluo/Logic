\section*{3.4 外延定义}
外延定义就是指被定义的普遍词项所适用对象的汇集。告诉某人词项外延的最方便有效的方法就是列举出其指谓的那些对象。不过,我们须认清这种方法的局限性。

上一节我们曾指出(以"等边三角形"和"等角三角形"为例),具有不同意义即不同内涵的两个词项可以具有恰好相同的外延。因此,即使我们能够完全列举出其中一个词项指谓的对象,由此而得出的外延定义也不能把它与另一个指谓同样对象的词项区分开来。这样的两个词项不是同义词,但是,外延定义却不能在它们之间做出区分。

然而,这还不是问题的棘手之处,因为外延可以完全列举出来的词项是极少的。要列举出"数"这个词所指谓的所有数字是完全不可能的。要列举出"恒星"这个词所指谓的天文数字的对象,实际上也是绝对不可能的。就其他大多数普遍词项来说,完全列举其外延都是不可能的。

这样,外延定义通常就要限定于所指谓对象的部分列举,这是个引起严重困难的局限。任何给定对象[比如,约翰•多伊(John Doe)这个人]都具有许许多多性质,因而被包括在许许多多不同的普遍词项的外延之中。当约翰-多伊作为一个词项的外延定义实例而被给出时,在许多其他词项的外延定义中,他也可以作为实例而被同样恰当地提及。约翰•多伊是"人"、"动物"、"哺乳动物"的实例,或许也是"丈夫"、"父亲"和 "学生"等等的实例。因此提到他,并不能帮助我们在这些词项的意义之间做出区分。即使我们给出两个、三个或更多实例,也会遇到同样的困

难。再臂如,在定义"摩天大厦"这个词时,我们可以使用帝国大厦 (Empire State)、克莱斯勒大厦(Chrysler building)和沃尔华斯大屏 (Woolworth building)等明显的范例,但是,这三座大厦作为实例同样也完全可以作为词项"二十世纪的伟大建筑"、"曼哈顿的昂贵房地产"或"纽约城的地面标志物"的外延。然而,每个这些普遍词项都指谓其他词项不指谓的对象,因此通过使用部分列举,我们甚至不能在具有不同外延的词项之间做出区分。引进"反面事例",例如"不是泰姬陵" (Taj Mahal)或"不是五角大楼"(the Pentagon),可以帮助说明被定义项的意义;但是,否定事例也必定是不完善的,这个基本局限仍然存在。

我们可以设法换一种列举方式,即不是每次列举一例,而是列举一整组例子。使用这种方法,也就是通过子类来定义,有时可能做到完全的列举。例如,我们把"脊椎动物"定义为两栖动物、鸟类、鱼类、哺乳动物和爬行动物。通过列举定义,无论完全列举还是部分列举,无论是列举类的个体元素还是列举子类,虽然都具有某些心理上的用处,但是要完全确定被定义项的意义,在逻辑上都是不充分的。

用列举法下定义的一个特殊类型被称做"实指定义"或"示范定义"。与一般外延定义不同,实指定义是通过用手或其他姿势指着对象来定义,不是通过命名或描述被定义词项指谓的对象来定义。例如,"'桌子' 这个词意指这",伴随着一个姿势如用手指指着桌子的方向,就是一个实指定义。

实指定义既有其自身的某些特殊局限性,也有前面所提到的各种局限性。首先,它有显而易见的地域局限:一个人只能指着看得见的东西,例如,他不能在乡村来实指地定义"摩天大厦",也不能在内陆山谷中去实指地定义"海洋"。其次,更为严重的是,姿势也有着不可避免的歧义。指着一张桌子也是指着它的一部分,以及它的颜色、大小、形状、质料等等,事实上,也就是指着位于桌子所在的方向上的所有东西,包括它后面的墙壁或者更远处的花园。

这种歧义有时可以通过给定义项增加一些描绘的短语而得到解决,其结果被称做准实指定义。例如,"桌子"这个词意指"这件"家具(伴随以相应的姿势)。但是,因为这种附加假设了对"家具"这个短语的事先理解,就使实指定义的宗旨难以达到。

实指定义历来被某些人视为"基本"或"原初"定义,其意思是说:我们最初都是凭借这种方式来理解词项意义的,而其他定义都依赖于这种理解。但是,这种原初性断言是错误的,因为人们必须理解姿勢本身的意义。当我们用手指指向婴儿床沿时,如果那个婴儿的注意力被吸引,其注意力可能放到被指向的东西上,也可能放到我们的手指上。如果我们想用其他姿势来定义一个姿势,也会出现同样的困难。如果你要理解任何符号 (sign)的定义,那么某些符号就必须已经得到理解。我们学习使用语言的根本途径是经过观察和模仿,而不是经过定义。

一如有些逻辑学家所做的那样,人们可以很宽泛地解说"实指定义"这个术语,甚至包括"当这个词指谓的对象出现时,不断地听到这个词"的过程。但是,这种过程根本不是定义,与我们这里对词项"定义"的使用不一样,它是学习使用语言的基本的和前定义的(predefinitional)方式。

最后,需要指出的是,虽然有些词意义丰富,但是它们完全不指谓任何事物,因此不能从外延上定义它们。例如,当我们说不存在独角兽时,我们在断定"独角兽"这个词没有所指,具有一个"空"外延。这类词项不仅仅展示了外延定义的局限性,也显示出"意义"的确更适用于内涵而不是外延。因为虽然"独角兽"这个词的外延是空的,但由此肯定不是说它就是无意义的。的确,它不指谓任何事物,因为根本就没有独角兽;但是,如果"独角兽"这个词项毫无意义,那么说"不存在独角兽"也就是无意义的。然而,这个陈述并不是没有意义的,我们完全理解它的意义,而且它是真的。显然,内涵对定义来说是真正的关键所在,我们下一节即转向讨论内涵。 