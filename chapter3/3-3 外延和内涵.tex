\section{外延和内涵}

\begin{logicbox}[title=引言]
\textit{在理解词项意义时,外延与内涵是两个不可或缺的概念。正确把握二者的关系有助于我们更精确地定义概念、避免语义混淆,从而提高思维和论证的准确性。}
\end{logicbox}

\begin{theorembox}[title=意义概念的复杂性]
\logicemph{基本问题}:定义旨在表明一个词项的\logicterm{意义}(meaning),但是"意义"这个词却有不同\logicterm{含义}(sense)。

\logicemph{理论发展}:
\begin{itemize}
  \item 前面已区分了词项的描述(字面)意义与表达性意义
  \item 现在需要更仔细地考察字面意义,尤其是普遍词项的字面意义
\end{itemize}

\logicemph{普遍词项的重要性}:
\begin{itemize}
  \item \logicterm{定义}:可以运用于多于一个对象的类(class)的词项
  \item \logicterm{重要性}:在推理中,普遍词项的定义是特别重要的
\end{itemize}
\end{theorembox}

\subsection{外延意义与内涵意义}

\begin{examplebox}[title="行星"的外延意义分析]
\logicemph{适用对象}:普遍词项"行星"对水星、金星、地球、火星、土星等都是在同等含义上适用的。

\logicemph{外延构成}:
\begin{itemize}
  \item 在一种含义上,词项"行星"意谓所有这些不同对象
  \item 所有行星的汇集(collection)就构成"行星"的意义
\end{itemize}

\logicemph{逻辑分析}:
\begin{itemize}
  \item 如果我说"所有行星都有椭圆轨道"
  \item 那么我所断定的包括:火星有椭圆轨道、金星有椭圆轨道,等等
  \item 词项"行星"的意义便是由它适用的那些对象构成的
\end{itemize}
\end{examplebox}

\begin{theorembox}[title=外延的基本概念]
\logicemph{外延意义}:词项意义的这种含义被称做\logicterm{外延意义}。

\logicemph{指谓关系}:
\begin{itemize}
  \item 普遍词项或类词项\logicterm{指谓}(denote)其可以正确适用的那些对象
  \item 一个普遍词项可以正确适用的对象的汇集构成那个词项的\logicterm{外延}
\end{itemize}
\end{theorembox}

\begin{theorembox}[title=内涵的基本概念]
\logicemph{理解与使用的区别}:
\begin{itemize}
  \item 理解普遍词项的意义就是知道怎样正确使用它
  \item 但这\logicwarn{并不一定要知道}它可以正确适用的所有对象
\end{itemize}

\logicemph{共同属性的作用}:
\begin{itemize}
  \item 对一个给定词项,其外延内的所有对象具有某些共同的性质或属性
  \item 这些性质或属性引导我们使用同一词项来指谓它们
  \item 因此,我们可以知道一个词项的意义而无须知道其外延
\end{itemize}

\logicemph{内涵意义}:
\begin{itemize}
  \item 在第二种含义上,"意义"设定了决定任一对象是否属于那个词项外延的某种标准
  \item 这种含义被称做词项的\logicterm{内涵意义}
  \item \logicterm{内涵}(intension):普遍词项指谓的所有对象并且仅仅那些对象共同拥有的属性集
\end{itemize}
\end{theorembox}

\subsection{外延与内涵的关系}

\begin{theorembox}[title=普遍词项的双重意义]
\logicemph{基本原理}:每个普遍或类词项都既有一个内涵意义又有一个外延意义。
\end{theorembox}

\begin{examplebox}[title="摩天大厦"的内涵与外延分析]
\logicemph{内涵}:
\begin{itemize}
  \item 包括所有超过一定高度的建筑物的共同和特有性质
  \item 提供了判断标准:什么样的建筑物可以被称为摩天大厦
\end{itemize}

\logicemph{外延}:
\begin{itemize}
  \item 是一个类,包括该词项适用对象的汇集
  \item 具体例子:纽约的世贸中心(World Trade Center)、芝加哥的希尔斯塔(Sears Tower)、上海世界金融中心(Shanghai World Financial Center)、吉隆坡的国油双峰塔(Petronas Twin Towers)等等
\end{itemize}
\end{examplebox}

\begin{theorembox}[title=外延变化的误解与澄清]
\logicwarn{常见误解}:有时人们断言一个词项的外延不时发生变化,尽管它的内涵没有变化。

\logicemph{典型例子}:
\begin{itemize}
  \item \logicterm{错误观点}:词项"人"的外延,正如人的死亡和婴儿的降生一样,持续变化
  \item \logicterm{混淆根源}:这个说法源于一种概念混淆
\end{itemize}

\logicemph{正确分析}:
\begin{itemize}
  \item 词项"人"用来指谓所有的人,包括死去的以及尚未出生的
  \item 它并没有一个不确定的外延
  \item \logicwarn{真正变化的}是词项"活着的人"的外延
\end{itemize}

\logicemph{深层原理}:
\begin{itemize}
  \item "活着的人"这个词项的外延具有"现在活着的人"这种含义
  \item 其中"现在"这个词是指不断变化的现时
  \item 因此,词项"活着的人"的内涵在不同的时候也是不同的
\end{itemize}

\logicwarn{重要结论}:任何具有变化外延的词项必定也有一个变化的内涵,二者是同等恒定的。
\end{theorembox}

\subsection{内涵决定外延而非相反}

\begin{theorembox}[title=内涵与外延的决定关系]
\logicemph{基本原理}:当一个词项的内涵固定下来时,它的外延也就固定了。

\logicwarn{重要原则}:词项的外延由它的内涵决定,但是反过来说却不对。
\end{theorembox}

\begin{examplebox}[title=等边三角形与等角三角形的对比分析]
\logicemph{"等边三角形"的分析}:
\begin{itemize}
  \item \logicterm{内涵}:由三条等长的直线所围成的平面图形的性质
  \item \logicterm{外延}:所有那些并且仅仅那些具有这种性质的对象的类
\end{itemize}

\logicemph{"等角三角形"的分析}:
\begin{itemize}
  \item \logicterm{内涵}:由三条相互相交而形成等角的直线所围成的平面图形的性质
  \item \logicterm{外延}:与"等边三角形"的外延完全相同
\end{itemize}

\logicwarn{关键发现}:两个词项具有不同的内涵,但外延却相同。
\end{examplebox}

\begin{theorembox}[title=决定关系的逻辑分析]
\logicemph{逻辑推论}:
\begin{itemize}
  \item 确认了这些词项其中一个词项的外延,而它的内涵却处于不确定状态
  \item \logicwarn{外延不决定内涵},但是,\logicterm{内涵却必定决定外延}
\end{itemize}

\logicemph{重要结论}:
\begin{itemize}
  \item 词项可以具有\logicterm{不同的内涵但外延却相同}
  \item 而具有\logicterm{不同外延的词项却不可能有同样的内涵}
\end{itemize}
\end{theorembox}

\subsection{内涵与外延的反变关系}

\begin{theorembox}[title=内涵增加的定义]
\logicemph{基本概念}:当给一个词项的内涵添加性质时,我们就说该内涵增加了。
\end{theorembox}

\begin{examplebox}[title=内涵与外延反变关系的典型例子]
\logicemph{词项序列}(按内涵增加次序排列):
\begin{enumerate}
  \item "人"
  \item "活着的人"
  \item "活着的二十岁以上的人"
  \item "活着的二十岁以上有红发的人"
\end{enumerate}

\logicemph{内涵分析}:
\begin{itemize}
  \item 每个词项的内涵都包含其后相随的词项的内涵
  \item 每个词项的内涵都比其前的那些词项的内涵多
  \item 这些词项是按照内涵增加的次序来排列的
\end{itemize}

\logicemph{外延分析}:
\begin{itemize}
  \item 如果我们倒过来看这些词项的外延,就会发现情况相反
  \item "人"的外延比"活着的人"的外延大,等等
  \item 这些词项是按照外延减少的次序排列的
\end{itemize}
\end{examplebox}

\begin{theorembox}[title=反变规律的修正]
\logicwarn{传统观点}:有些逻辑学家得出一条公式化的\logicterm{"反变规律"},断言外延与内涵总是反向变化。

\logicemph{评价}:这种断言具有启发性,但并不完全正确。
\end{theorembox}

\begin{examplebox}[title=反变规律的反例]
\logicemph{反例序列}(按内涵增加次序排列):
\begin{enumerate}
  \item "活着的人"
  \item "活着的有脊骨的人"
  \item "活着的有脊骨的不超过一千岁的人"
  \item "活着的有脊骨的不超过一千岁的没有读完国会图书馆里所有书的人"
\end{enumerate}

\logicemph{分析结果}:
\begin{itemize}
  \item 这些词项的次序是增加内涵
  \item 但是它们每个的外延都是相同的,完全没有减少
\end{itemize}
\end{examplebox}

\begin{theorembox}[title=修正的反变规律]
\logicemph{正确表述}:如果词项按照内涵增加的次序排列,那么它们的外延将处于\logicterm{非递增的次序}。

\logicemph{精确含义}:也就是说,如果外延变化,那么它们将是沿着内涵的反向变化。

\logicwarn{重要说明}:外延可能减少,也可能保持不变,但不会增加。
\end{theorembox}

\subsection{外延为空的词项与意义歧义}

\begin{theorembox}[title=空外延词项的存在]
\logicemph{重要事实}:有些词项的外延,例如"独角兽"的外延,可能是空的。

\logicemph{理论价值}:认识到这一点,并运用我们对内涵与外延的区分,就可以把玩弄"意义"歧义的谬误论证揭露出来。
\end{theorembox}

\begin{examplebox}[title=上帝存在论证的谬误分析]
\logicemph{谬误论证}:
\begin{displayquote}
"上帝"这个词不是无意义的,因此它有意义。但是按照定义,"上帝"这个词的意思是全能的至善的存在(being)。因此,全能的至善的存在,即上帝,必然存在(exist)。
\end{displayquote}

\logicwarn{歧义分析}:
\begin{itemize}
  \item 歧义在于"意义"和"无意义"这两个词
  \item "意义"在一种含义上指的是内涵,在另一种含义上指的却是外延
\end{itemize}

\logicemph{正确分析}:
\begin{itemize}
  \item "上帝"这个词不是无意义的,因此可以肯定,存在一个内涵是它的意义
  \item 但是,由此并不能得出:一个具有内涵的词项,其内涵一定指谓一个存在物\cite{gombocz1997}
\end{itemize}
\end{examplebox}

我们在下面这个语段中也发现了一个类似的谬误:

\begin{examplebox}[title=乌托邦论证的谬误分析]
\logicemph{另一个类似谬误}:
\begin{displayquote}
kitsch (低劣作品) 以展示粗鄙、卑劣、下贱、脆弱和邪恶信仰来表现并败坏人类境况。这就是乌托邦之所以能被定义为kitsch 这一词项已消失的状况的原因, 因为在乌托邦中该词项已没有所指了。\cite{sisk1988}
\end{displayquote}

\logicwarn{谬误根源}:作者没能在意义与所指(referent)之间做出区分。

\logicemph{正确观点}:
\begin{itemize}
  \item 许多有价值的词项(例如,那些命名希腊神话中的动物的词项)都不存在所指
  \item 但是,我们并不要求或期望这样的词项消失
  \item 具有内涵但没有外延的词项是非常有用的
\end{itemize}

\logicemph{实用价值}:如果有一天乌托邦变成了现实,我们也许想要表达对减少或消除"低劣作品"或"粗鄙"等的庆幸。而要这样做,我们就需要能够有意义地使用这些词项。
\end{examplebox}

\begin{theorembox}[title=章节过渡]
\logicemph{前面内容回顾}:在前面的几节中,我们考察了定义的种类和它们的用途:
\begin{itemize}
  \item 词典定义和规定定义可消除或避免歧义
  \item 精确定义可以减少模糊性
\end{itemize}

\logicemph{后续内容预告}:在随后的几节中,我们将考察构建定义的方法:
\begin{itemize}
  \item 有些定义通过外延或所指来处理普遍词项
  \item 其他定义则通过内涵来处理
  \item 我们将会看到,每种处理方法都既有优点又有缺点
\end{itemize}
\end{theorembox}

\chaptersummary{
外延与内涵是理解词项意义的两个基本维度,它们之间的关系揭示了语言和思维的深层结构。

\logicemph{基本概念}:
\begin{itemize}
  \item \logicterm{外延}:指词项可以正确适用的所有对象的集合,体现了词项的指谓范围
  \item \logicterm{内涵}:指词项所表示的属性或特征的集合,提供了判断标准
  \item 每个普遍词项都既有外延意义又有内涵意义
\end{itemize}

\logicemph{决定关系}:
\begin{itemize}
  \item \logicwarn{内涵决定外延},而非相反:当内涵固定时,外延也就固定了
  \item 词项可以有相同外延但内涵不同(如"等边三角形"与"等角三角形")
  \item 具有不同外延的词项不可能有同样的内涵
\end{itemize}

\logicemph{反变规律}:
\begin{itemize}
  \item 当词项内涵增加时,其外延处于非递增状态(可能减少或保持不变,但不会增加)
  \item 传统的"反变规律"过于绝对,修正版本更为准确
\end{itemize}

\logicemph{空外延现象}:
\begin{itemize}
  \item 词项可以有内涵而无外延(如"独角兽"),这不影响词项的有意义性
  \item 理解这一点有助于识别和避免关于"意义"的歧义谬误
  \item 具有内涵但没有外延的词项在语言中是非常有用的
\end{itemize}

\logicwarn{理论意义}:外延与内涵的区分为构建不同类型的定义提供了理论基础,是逻辑学和语言哲学的重要工具。
}