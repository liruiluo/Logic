\section{论争、言辞之争与定义}

\begin{logicbox}[title=引言]
\textit{在交流过程中,人们的分歧有时是实质性的,有时却只是语言上的误解。理解不同类型的论争有助于我们更有效地解决问题,避免在无实质内容的争论上浪费时间和精力。}
\end{logicbox}

\begin{theorembox}[title=语言的双重性质]
语言是一种极其复杂的设施,是人类最重要的交流工具。然而,这种强大的工具也可能成为我们的负担:

\logicemph{积极作用}:
\begin{itemize}
  \item 促进思想交流和理解
  \item 传递信息和知识
  \item 表达情感和态度
\end{itemize}

\logicwarn{消极影响}:当语词被漫不经心或错误地使用时,语言会阻碍而非促进交流。
\end{theorembox}

\begin{theorembox}[title=从歧见到论争的理论发展]
\logicemph{理论回顾}:在2.5节中,我们阐释了冲突双方的歧见既可能是\logicterm{信念的}也可能是\logicterm{态度的},两者都可能构成实质歧见。

\logicemph{新的发现}:除了实质歧见外,还存在另一种情况:
\begin{itemize}
  \item \logicwarn{表面歧见}:看似存在分歧,实际上并无真正对立
  \item \logicterm{根本原因}:误解或词汇误用的结果
  \item \logicterm{识别意义}:避免在虚假争议上浪费时间和精力
\end{itemize}

\logicemph{研究转向}:从考察实质歧见转向讨论不同类型的\logicterm{论争},重点在于分辨其是否具有真正的分歧。
\end{theorembox}

\subsection{三种不同类型的论争}

\begin{theorembox}[title=三种论争类型]
必须区分出三种不同的论争:

\textbf{第一种:}\logicterm{明显的实质论争},在这种论争中,各方或者在信念上或者在态度上,明确地毫不含糊地对立。

\textbf{第二种:}\logicterm{纯粹的言辞之争},表面上的分歧实际上只是语言使用的差异。

\textbf{第三种:}\logicterm{表面上的言辞之争但实际上的实质论争},既有语言问题又有实质分歧。
\end{theorembox}

\begin{examplebox}[title=实质论争的典型例子]
\logicemph{态度歧见的例子}:
\begin{itemize}
  \item \logicterm{情境}:美国佬(Yankees)赢得了世界联赛
  \item \logicterm{共同认知}:双方对胜利者本身的认同没有争论
  \item \logicterm{态度分歧}:A为此高兴,B为此恼怒
  \item \logicterm{特征}:态度歧见显然且可能激烈,属于难以解决的问题
\end{itemize}

\logicemph{信念歧见的例子}:
\begin{itemize}
  \item \logicterm{争议内容}:巴拿马运河的太平洋入口是否比大西洋入口更靠东
  \item \logicterm{分歧性质}:不是态度上的,而是事实认知上的
  \item \logicterm{解决方法}:一张好地图就可以平息这个论争
\end{itemize}
\end{examplebox}

\begin{theorembox}[title=实质论争的基本特征]
\logicemph{核心特征}:无论是态度上的还是信念上的,实质论争总是包含某种\logicterm{实质歧见}。

\logicemph{区别标准}:
\begin{itemize}
  \item 将论争双方区别开来的\logicwarn{不仅仅是语言}
  \item 在对事实的断定上或对事实的评价上存在\logicterm{实质差别}
  \item 不能通过定义或简单的语言调整来解决
\end{itemize}
\end{theorembox}

\begin{theorembox}[title=事实的多样性与论争的实质性]
\logicemph{事实的广泛性}:
\begin{itemize}
  \item \logicterm{物理事实}:如地理位置、物理现象
  \item \logicterm{语言事实}:如单词的拼写或使用方法
  \item \logicterm{心理事实}:如某人是不友好还是仅仅害羞
\end{itemize}

\logicemph{论争的实质性判断}:
\begin{itemize}
  \item 各方可以在任何种类的事实上产生歧见
  \item \logicwarn{关键原则}:如果论争确实是关于某个事实的,那么它就是实质的
  \item \logicterm{解决途径}:可以通过确认相关事实而得到解决
\end{itemize}
\end{theorembox}

\subsection{纯粹的言辞之争}

\begin{theorembox}[title=纯粹言辞之争的定义与特征]
\logicemph{基本定义}:\logicterm{纯粹的言辞之争}是指双方之间根本没有实质歧见,但却表面上好像具有歧见的论争。

\logicemph{根本原因}:语言的误解或误用是症结所在。

\logicemph{产生机制}:
\begin{itemize}
  \item 论争者信念表达中的关键语词有歧义
  \item 这种歧义遮蔽了双方实际上并没有实质对立的事实
\end{itemize}
\end{theorembox}

\begin{theorembox}[title=言辞之争的具体成因]
\logicemph{三种典型情况}:
\begin{itemize}
  \item \logicterm{误用情况}:争论某方误用了一个重要词语
  \item \logicterm{歧义情况}:核心语词或短语具有不同含义,这些含义可能同等合法但产生了不应有的混淆
  \item \logicterm{认知盲区}:各方对语词或短语的用法都正确但含义不同,而这一点没有被明确认知
\end{itemize}

\logicwarn{共同特点}:所有这些情况都可能产生表面的言辞之争。
\end{theorembox}

\begin{theorembox}[title=言辞之争的识别与解决]
\logicemph{识别难度}:言辞之争并非总是容易发现,需要仔细分析。

\logicemph{解决方法}:
\begin{itemize}
  \item 一旦识别了言辞之争,可以相当容易地获得解决
  \item \logicterm{核心策略}:具体化歧义语词或短语的不同含义
  \item \logicterm{关键工具}:良好的定义对相互理解非常关键
\end{itemize}
\end{theorembox}

威廉•詹姆士给出了这种言辞之争的一个经典例子:

\begin{displayquote}
几年前,我随野营队一起在山上露营。当我独自散步返回时,发现每个人都参加了一场激烈的形而上学论战。争论主题是一只松鼠。设想一只松鼠抓附在树干一侧,而一个人站在树的另一侧;那人绕树迅速转动以试图看到松鼠,但无论他转多么快,松鼠在相对的方向都以同样快的速度转动,在它自己和那人之间总隔着那棵树,因此使他看不到松鼠。作为结果的形而上学问题是:这个人是否绕着松鼠走了一圈?确确实实,他绕树走了一圈,而且松鼠就在树上;但是,他绕松鼠走了一圈吗?原野中的讨论持续良久,直到变得乏味。每个人都赞同一种观点并固执己见,并且双方的人数势均力敌。因此,当我出现时,每方都希望我加入以便成为多数派。\cite{james1907}
\end{displayquote}

\begin{theorembox}[title=松鼠例子的深层分析]
\logicemph{实质歧见的缺失}:
\begin{itemize}
  \item 论争双方之间不存在实质歧见
  \item 所有论争者对松鼠和树的态度都是中立的
  \item 都完全理解和赞同给定事例的所有事实
\end{itemize}

\logicwarn{争论的本质}:这个事例(以及许多其他类似事例)中,争论不过是言辞之争。
\end{theorembox}

\begin{examplebox}[title=詹姆士的解决方案]
詹姆士继续写道:

\begin{displayquote}
有绕它走一圈,因为松鼠做了相对运动,它始终保持着将其腹部对着那个人,而将背部朝着外面。做出这种区分,就没有什么可争论的了。你们都又对又不对,就看你们对"绕走一圈"这个动词实际上是怎么理解的。\cite{james1907}
\end{displayquote}

\logicemph{解决策略}:
\begin{itemize}
  \item \logicterm{概念区分}:明确"绕走一圈"的不同含义
  \item \logicterm{相对性认识}:双方都既对又不对,取决于对概念的理解
  \item \logicterm{争论消解}:一旦做出区分,就没有什么可争论的了
\end{itemize}
\end{examplebox}

\begin{theorembox}[title=纯粹言辞之争的解决原理]
\logicemph{解决要求}:
\begin{itemize}
  \item \logicwarn{不需要新事实}:解决这个论争不要求新的事实,那样做也不可能有帮助
  \item \logicterm{核心需求}:仅需要对争论中关键语词的不同意义做出区分
  \item \logicterm{效果}:使用不同定义,争论就消失了
\end{itemize}

\logicemph{一般原理}:
\begin{itemize}
  \item 如果论争纯粹是言辞之争,可以通过提供能够消除关键歧义的定义来解决
  \item 可以表明论争各方并不是真正的相互对立
  \item 他们可能仅仅是运用相同词汇的不同含义来维护不同主张,或运用不同语词来维护相同主张
\end{itemize}

\logicwarn{最终结果}:一旦确定了不同意义以及源于对不同意义使用的不同主张,双方之间就不会再有什么论争。\cite{rudin1992}
\end{theorembox}

\subsection{表面上的言辞之争与实质论争}

\begin{theorembox}[title=第三种论争的定义与特征]
\logicemph{基本定义}:\logicterm{表面上是言辞的但实际上是实质的论争}

\logicemph{表面特征}:
\begin{itemize}
  \item 当双方互相误解了对方词语的用法时会出现混淆
  \item 这种混淆可以得到识别
  \item 看似只是语言问题
\end{itemize}

\logicemph{深层实质}:
\begin{itemize}
  \item 争执远超出语词不同用法的范围
  \item 仅仅解决歧义问题\logicwarn{不会平息论争}
  \item 争论双方之间还存在某种实质歧见:可能在信念上,更可能在态度上
\end{itemize}
\end{theorembox}

\begin{examplebox}[title=色情作品争议的案例分析]
\logicemph{争议情境}:对有露骨性活动镜头的影片是否应该作为"色情作品"来处理

\logicemph{双方立场}:
\begin{itemize}
  \item \logicterm{甲方观点}:露骨性使它成了邪恶的色情作品
  \item \logicterm{乙方观点}:考虑到细腻的情感和美学价值,它是真正的艺术,根本不是色情作品
\end{itemize}

\logicemph{分析层次}:
\begin{itemize}
  \item \logicterm{表面层次}:双方的歧见在于"色情作品"一词的意义
  \item \logicwarn{深层实质}:即使言语的不同得到充分理解并清除所有歧义,双方很可能对影片仍然存在实质歧见
  \item \logicterm{真正争议}:不是关于"色情作品"这个词的适用性,而是影片的性感露骨性质是否造成影片的好与坏
\end{itemize}
\end{examplebox}

\subsection{标准之争与概念之争}

\begin{theorembox}[title=标准之争与概念之争]
\logicemph{别名}:第三种论争有时被称为\logicterm{标准之争}或\logicterm{概念之争}。

\logicemph{核心特征}:
\begin{itemize}
  \item 论争双方对某个关键词语的运用有着\logicterm{不同标准}
  \item 该词语指谓的是\logicterm{不同概念}
  \item 各方在不同标准的明智或正确性问题上处于尖锐冲突
\end{itemize}

\logicemph{深层冲突}:即使歧义得到阐明和区分,各方仍可能声称对手误用了标准。
\end{theorembox}

\begin{examplebox}[title=色情作品争议的深层分析]
即使双方都认识到他们有歧义地使用了"色情作品"这个词,甚至词语的歧义已经得到阐明和区分,但冲突仍然存在:

\logicemph{标准冲突}:
\begin{itemize}
  \item \logicterm{甲方标准}:如果影片包含露骨的性活动场景,便可以将它划归为色情作品
  \item \logicterm{乙方回应}:那种划归是一种概念错误
\end{itemize}

\logicwarn{论争本质}:这种论争表面上仅仅是言辞之争,但在表面之下,却是非常实质的论争。
\end{examplebox}

\subsection{识别不同类型的论争}

\begin{theorembox}[title=论争类型识别流程]
为帮助人们辨识和理解论争的不同种类,我们可以使用一个有用的\logicterm{诊断流程}:

\logicemph{第一步}:确定存在某种论争后,问:\logicwarn{"出现歧义了吗?"}
\begin{itemize}
  \item 如果回答"没有" → \logicterm{类型一}(明显的实质论争)
  \item 如果回答"出现了" → 进入第二步
\end{itemize}

\logicemph{第二步}:问:\logicwarn{"清除歧义可以消除对立吗?"}
\begin{itemize}
  \item 如果回答"可以" → \logicterm{类型二}(纯粹言辞之争)
  \item 如果回答"不可以" → \logicterm{类型三}(表面言辞实质论争)
\end{itemize}
\end{theorembox}

\begin{theorembox}[title=三种论争类型的系统总结]
\logicemph{类型一:明显的实质论争}
\begin{itemize}
  \item \logicterm{歧义状况}:不存在言辞歧义
  \item \logicterm{歧见性质}:争论双方确实有歧见(态度上或信念上)
  \item \logicterm{解决方法}:通过解决实质性的信念或态度歧见
\end{itemize}

\logicemph{类型二:纯粹言辞之争}
\begin{itemize}
  \item \logicterm{歧义状况}:存在言辞歧义
  \item \logicterm{歧见性质}:根本没有实质歧见
  \item \logicterm{解决方法}:通过明确定义和消除歧义
\end{itemize}

\logicemph{类型三:表面言辞实质论争}
\begin{itemize}
  \item \logicterm{歧义状况}:既存在言辞歧义
  \item \logicterm{歧见性质}:又有实质歧见(态度上或信念上,关于事实或语词运用标准)
  \item \logicterm{解决方法}:需同时解决语言问题和实质分歧
\end{itemize}
\end{theorembox}

\chaptersummary{
论争的类型分析是有效解决分歧的前提,不同类型的论争需要采用不同的解决策略。

\logicemph{三种基本类型}:
\begin{itemize}
  \item \logicterm{明显的实质论争}:不存在言辞歧义,存在实质性歧见,需通过解决信念或态度歧见来解决
  \item \logicterm{纯粹的言辞之争}:存在言辞歧义但无实质歧见,可通过明确定义和消除歧义来解决
  \item \logicterm{表面言辞实质论争}:既有言辞歧义又有实质歧见,需同时解决语言问题和实质分歧
\end{itemize}

\logicemph{识别方法}:
\begin{itemize}
  \item 使用两步诊断流程:先判断是否存在歧义,再判断清除歧义是否能消除对立
  \item 特别注意标准之争和概念之争的复杂性
  \item 威廉·詹姆士的松鼠例子展示了纯粹言辞之争的典型特征
\end{itemize}

\logicwarn{实践意义}:准确识别论争类型有助于避免在虚假争议上浪费时间,将精力集中在真正的分歧上,提高交流和解决问题的效率。
}