\section{属加种差定义的五条规则}

\begin{quotation}
\textit{属加种差定义作为最常用的内涵定义方法,遵循一套传统规则。这些规则不仅有助于构建清晰、准确的定义,也是评价已有定义质量的重要标准。理解这些规则能够帮助我们避免定义中的常见错误。}
\end{quotation}

前一节我们详细讨论了属加种差定义的本质和应用。为了确保这种定义方法发挥最大效用,逻辑学家们总结了五条传统规则,用以指导属加种差定义的构建和评价。这五条规则既是构建定义的指南,也是判断定义优劣的标准。

\subsection{规则一:定义应当揭示种的本质属性}

第一条规则要求,定义揭示的应当是被定义事物的\textbf{本质属性},而非偶然属性。本质属性是事物必然具有的、缺少它事物就不成其为此种事物的属性。例如,人的理性能力是人的本质属性,而身高、体重则是偶然属性。因此,将"人"定义为"有理性的动物"比定义为"会使用工具的生物"更符合这一规则。

本质属性的选择不仅取决于事物本身,也取决于定义的目的。在科学定义中,本质属性往往是事物最深层次的特征,而在日常生活中,则可能是事物最易识别的特征。无论哪种情况,定义都应当努力揭示事物区别于其他事物的最重要、最根本的特征。

偶然地使用一种不是本质的属性可能会导致定义失去普遍性和稳定性。例如,如果我们将"大学生"定义为"穿校服的年轻人",就没有捕捉到大学生的本质特征,因为不是所有大学生都穿校服,而且有些非大学生也可能穿校服。

\subsection{规则二:定义不能循环}

第二条规则禁止在定义中直接或间接地使用被定义项本身。这种错误被称为\textbf{循环定义}。例如,"教育是教育人的活动"就是一个明显的循环定义,因为它在定义"教育"时又使用了"教育"这个词。

间接的循环定义更加隐蔽,例如:"老师是教育者,教育者是从事教学的人,教学的人是老师。"在这个定义链条中,"老师"最终通过一系列中间环节被定义为它自己。

循环定义之所以是错误的,是因为它没有提供任何新的信息,无法帮助人们理解被定义项的意义。良好的定义应当使用已知的、更基本的概念来解释未知的概念,而不是用未知解释未知。

\subsection{规则三:定义既不能过宽又不能过窄}

第三条规则要求定义的外延必须与被定义词项的外延完全一致,即定义不能太宽也不能太窄。

当定义\textbf{过宽}时,定义包含了不属于被定义项的对象。例如,将"鸟"定义为"会飞的动物"就过于宽泛,因为蝙蝠、昆虫等非鸟类动物也会飞。这种定义违反了第三条规则,因为它使用的种差(会飞)不足以将鸟与其他动物区分开来。

当定义\textbf{过窄}时,定义排除了属于被定义项的一些对象。例如,将"鸟"定义为"会飞的有羽毛的脊椎动物"就过于狭窄,因为企鹅、鸵鸟等不会飞的鸟类被排除在外。这种定义同样违反了第三条规则。

良好的定义应当使用足够特定的种差,使得定义项的外延与被定义项的外延完全一致。例如,"鸟是有羽毛、下颌骨形成喙的脊椎动物"就更为准确,因为它既包含了所有鸟类,又排除了所有非鸟类动物。

\subsection{规则四:定义不能用歧义的、晦涩的或比喻的语言}

第四条规则要求定义使用清晰、精确、直接的语言,避免使用歧义的、晦涩的或比喻性的表达。

\textbf{歧义}指的是一个词或短语有多种可能的解释。例如,将"银行"定义为"由河边延伸的倾斜地形或处理金钱的机构"就是歧义的,因为它没有明确指出是哪一种"银行"。良好的定义应当消除歧义,而不是引入新的歧义。

\textbf{晦涩}指的是使用难以理解的术语或表达。例如,将"水"定义为"氢氧化二氢"对于不懂化学的人来说就是晦涩的。虽然这个定义在技术上是准确的,但对于一般受众来说并不起到澄清意义的作用。

\textbf{比喻性语言}使用一种事物来象征或代表另一种事物。例如,将"时间"定义为"生命的河流"就是使用比喻。虽然比喻有时可以增加表达的生动性,但在定义中使用比喻往往会导致不精确,因为比喻依赖于个人解释。

良好的定义应当使用直接、确切的表达,以最大程度地减少误解的可能性。

\subsection{规则五:定义在可以用肯定的地方就不应当用否定定义}

第五条规则建议,只要有可能,定义就应当采用肯定的形式,避免使用否定形式。这是因为否定定义往往不能充分说明被定义项是什么,而只说明它不是什么。

例如,将"和平"定义为"没有战争的状态"就是一个否定定义。虽然这个定义在某种程度上是正确的,但它并没有积极地说明和平的特征是什么。相比之下,将"和平"定义为"人们和谐相处、互相尊重的社会状态"就是一个更好的肯定定义。

有些情况下,当被定义项本身是否定性质时,使用否定定义是适当的。例如,"无神论"可以恰当地定义为"不相信神存在的哲学立场"。但即使在这些情况下,如果可能的话,也应该尝试提供一些肯定的特征。

否定定义的问题在于,它们经常过于宽泛。例如,"非鱼"包括了除鱼以外的所有事物,这样的定义几乎没有信息价值。良好的定义应该提供足够的积极信息,使人们能够识别和理解被定义的概念。

\begin{center}
\fbox{\parbox{0.9\textwidth}{
  \centering
  \textbf{属加种差定义的五条规则}\\
  规则一:定义应当揭示种的本质属性,而非偶然属性;\\
  规则二:定义不能循环,即不能直接或间接地用被定义项来定义自身;\\
  规则三:定义既不能过宽也不能过窄,其外延必须与被定义项完全一致;\\
  规则四:定义不能用歧义的、晦涩的或比喻的语言,应当清晰精确;\\
  规则五:定义在可以用肯定的地方就不应当用否定定义,应积极揭示事物特征。
}}
\end{center} 