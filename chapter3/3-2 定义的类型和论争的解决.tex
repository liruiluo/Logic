\section{定义的类型和论争的解决}

\begin{logicbox}[title=引言]
\textit{定义是解决语言争端的重要工具,理解不同类型的定义及其应用场景,有助于我们更准确地表达思想,避免不必要的争论。}
\end{logicbox}

\begin{theorembox}[title=定义的基本概念与功能]
\logicemph{基本定义}:定义是对词项意义的解说。

\logicemph{核心功能}:
\begin{itemize}
  \item 减少或消除由词项意义的不确定性引起的困难
  \item 减少或消除由词项意义的模糊性引起的困难
  \item 促进清晰的思考和有效的交流
\end{itemize}

\logicemph{两种基本方向}:
\begin{itemize}
  \item \logicterm{阐释既有意义}:解释一个既有词项的既有意义
  \item \logicterm{赋予新意义}:给一个新词项赋予新意义
\end{itemize}
\end{theorembox}

\begin{theorembox}[title=报告与规定的根本区别]
\logicemph{报告性定义}(词典定义):
\begin{itemize}
  \item \logicterm{目的}:报告一种既定的用法
  \item \logicterm{评价标准}:可以是真的或假的
  \item \logicterm{准确性}:报告的意义可以是准确的也可以是不准确的
\end{itemize}

\logicemph{规定性定义}:
\begin{itemize}
  \item \logicterm{目的}:规定一种用法
  \item \logicterm{评价标准}:\logicwarn{不能是真的或假的}
  \item \logicterm{实用性}:规定的意义可以是有用的也可以是无用的,可以是方便的也可以是不方便的
\end{itemize}

\logicwarn{根本区别}:在真假与有用无用之间存在根本区别。
\end{theorembox}

\subsection{报告性定义与规定性定义}

\begin{examplebox}[title=报告性定义的典型例子]
\logicemph{"奇数"的定义}:
\begin{itemize}
  \item \logicterm{定义内容}:"任何不能被2整除的整数"
  \item \logicterm{定义性质}:报告性定义
  \item \logicterm{理由}:报道了"奇数"这个词的既定用法
  \item \logicterm{真假评价}:这个定义是真的
\end{itemize}
\end{examplebox}

\begin{examplebox}[title=规定性定义的典型例子]
\logicemph{"圆方形"的定义}:
\begin{itemize}
  \item \logicterm{定义内容}:"既是圆又是正方形的图形"
  \item \logicterm{定义性质}:规定性定义
  \item \logicterm{理由}:规定这个新词的意义,而不是报道既有意义
  \item \logicterm{实用性评价}:在应用上无用(因为不存在圆方形)
  \item \logicterm{逻辑性评价}:逻辑上不可能(结合了两个不相容的属性)
  \item \logicwarn{真假评价}:既非真也非假
\end{itemize}
\end{examplebox}

\subsection{精确定义的作用}

\begin{theorembox}[title=精确定义的必要性]
\logicwarn{问题识别}:当我们面对歧义时,词项的既定意义并不总是很清楚的。

\logicemph{解决需求}:在这种情况下,仅仅报道词项的既有意义是不够的,需要更精确的定义。
\end{theorembox}

\begin{examplebox}[title="安乐死"的定义演进]
\logicemph{报告性定义}(常见用法):
\begin{itemize}
  \item \logicterm{定义内容}:"有意致病人于死地,而又没有该病人同意的医疗行为"
  \item \logicterm{问题}:这个定义存在模糊性,需要澄清
\end{itemize}

\logicemph{精确定义}:
\begin{itemize}
  \item \logicterm{定义内容}:"经过病人的自由和知情的同意,由医生有意地导致的一个不能忍受持久痛苦的病人无痛苦的死亡"
  \item \logicterm{作用}:澄清词项的模糊性,使标准更为精确
\end{itemize}

\logicemph{功能关系}:
\begin{itemize}
  \item 报告性定义的作用是建立一种标准
  \item 精确定义则使这种标准更为精确
\end{itemize}
\end{examplebox}

\begin{theorembox}[title=精确定义的特殊性质]
\logicemph{与规定定义的区别}:
\begin{itemize}
  \item 精确定义:针对模糊的既有词项
  \item 规定定义:给新词项赋予意义
\end{itemize}

\logicemph{精确定义的约束}:
\begin{itemize}
  \item \logicwarn{非武断性}:精确定义并不是武断的
  \item \logicterm{既有用法}:所定义的是已经具有固定用法的词项
  \item \logicterm{保持原则}:必须尽可能地保持这种固定用法
\end{itemize}

\logicemph{双重考虑}:
\begin{itemize}
  \item 使词项摆脱模糊性
  \item 与现有用法的融贯性
\end{itemize}

\logicemph{评价标准}:精确定义仍然可以被评价为真的或假的,但这要根据它是否符合现有用法而定。
\end{theorembox}

\subsection{规定定义的应用场景}

\begin{theorembox}[title=规定定义的基本特征]
\logicwarn{评价标准}:真或假的问题对于规定定义来说是完全无关的。

\logicemph{应用场景}:
\begin{itemize}
  \item 当一个新词项引进时
  \item 当一个既有词项要在新语境中使用时
\end{itemize}

\logicwarn{根本原因}:规定定义不能是真的或假的,因为它们定义的是一些此前没有定义的词项。
\end{theorembox}

\begin{examplebox}[title=规定定义的科学应用]
\logicemph{物理学例子}:
\begin{itemize}
  \item \logicterm{新词项}:"焦耳"
  \item \logicterm{定义内容}:"1牛顿的力使物体在力的方向上移动1米所做的功"
  \item \logicterm{目的}:表示功或能的单位
\end{itemize}

\logicemph{计算机科学例子}:
\begin{itemize}
  \item \logicterm{词项}:"硬盘"
  \item \logicterm{用法}:用于新型存储装置
  \item \logicterm{性质}:规定性使用
\end{itemize}
\end{examplebox}

\begin{theorembox}[title=规定定义的动机]
\logicemph{常见动机}:
\begin{itemize}
  \item \logicterm{便利性}:为了方便交流和使用
  \item \logicterm{命名需求}:因为有些新事物需要命名
  \item \logicterm{概念简化}:用简短的词来指称复杂的概念
\end{itemize}
\end{theorembox}

\begin{examplebox}[title="因特网"的规定定义]
\logicemph{词项特征}:
\begin{itemize}
  \item \logicterm{新词性质}:相对较新的词
  \item \logicterm{定义内容}:"一个全球性的计算机网络系统,它使用TCP/IP协议族来连接全世界数以百万计的计算机"
  \item \logicterm{定义目的}:用一个简短的词来指称一个复杂的概念
\end{itemize}
\end{examplebox}

\subsection{理论定义与说服定义}

\begin{theorembox}[title=理论定义]
\logicemph{基本特征}:
\begin{itemize}
  \item \logicterm{目的}:出于理论的目的
  \item \logicterm{功能}:旨在阐明一个理论概念
  \item \logicterm{性质}:也是规定性的,赋予词项新的、更专门的意义
\end{itemize}
\end{theorembox}

\begin{examplebox}[title=经济学中的理论定义]
\logicemph{"效用"的理论定义}:
\begin{itemize}
  \item \logicterm{定义内容}:"一种衡量消费者从消费一种商品或服务中获得的满足程度的指标"
  \item \logicterm{理论框架}:经济学理论
  \item \logicterm{目的}:在经济理论框架内阐明"效用"这个概念
\end{itemize}
\end{examplebox}

\begin{theorembox}[title=说服定义]
\logicwarn{基本特征}:
\begin{itemize}
  \item \logicterm{目的}:出于说服的目的
  \item \logicterm{功能}:影响人们的态度或行为
  \item \logicterm{性质}:也是规定性的,赋予词项带有情感色彩的意义
\end{itemize}

\logicwarn{警惕性}:说服定义可能包含偏见和操纵意图。
\end{theorembox}

\begin{examplebox}[title=说服定义的典型例子]
\logicemph{政治领域的例子}:
\begin{itemize}
  \item \logicterm{词项}:"爱国主义"
  \item \logicterm{说服定义}:"对自己国家盲目的、不加批判的忠诚"
  \item \logicterm{目的}:贬低那些持有不同政见的人
\end{itemize}

\logicemph{商业领域的例子}:
\begin{itemize}
  \item \logicterm{词项}:某种产品
  \item \logicterm{说服定义}:"成功的象征"
  \item \logicterm{目的}:吸引消费者购买
\end{itemize}
\end{examplebox}

\subsection{定义在解决论争中的应用}

\begin{theorembox}[title=定义在解决论争中的策略应用]
\logicwarn{基本原则}:在解决论争时,识别和理解不同类型的定义是至关重要的。

\logicemph{针对性解决策略}:
\begin{itemize}
  \item \logicterm{意义分歧}:如果论争是由于对词项意义存在分歧而引起的,通过提供清晰的、双方都能接受的定义来解决
  \item \logicterm{模糊性问题}:如果论争涉及词项的模糊性,精确定义可以帮助澄清问题
  \item \logicterm{新概念理论}:如果论争涉及新的概念或理论,规定定义或理论定义可以帮助阐明相关思想
  \item \logicwarn{说服定义}:如果论争涉及说服定义,需要警惕定义中可能包含的情感偏见
\end{itemize}
\end{theorembox}

\begin{theorembox}[title=定义的工具价值]
\logicemph{核心价值}:定义是解决论争和促进清晰思考的重要工具。

\logicemph{实践意义}:通过仔细地考察和运用不同类型的定义,我们可以更有效地进行交流和推理。
\end{theorembox}

\chaptersummary{
定义作为语言分析和论争解决的核心工具,具有多种类型和功能,理解其特征和应用条件对于有效交流至关重要。

\logicemph{五种主要定义类型}:
\begin{itemize}
  \item \logicterm{报告性定义}:阐释既有词项的既有意义,可评价为真或假,如词典定义
  \item \logicterm{精确定义}:消除既有词项的模糊性,需兼顾与现有用法的融贯性,仍可评价真假
  \item \logicterm{规定性定义}:赋予新词项以意义或赋予既有词项以新意义,不能评价为真或假
  \item \logicterm{理论定义}:为阐明理论概念而提出的规定定义,具有专门的学术用途
  \item \logicterm{说服定义}:为影响态度或行为而提出的带有情感色彩的定义,需要警惕其偏见
\end{itemize}

\logicemph{核心区别}:
\begin{itemize}
  \item 报告性定义与规定性定义的根本区别在于真假评价的适用性
  \item 精确定义介于两者之间,既要保持既有用法又要消除模糊性
  \item 理论定义和说服定义是规定性定义的特殊应用
\end{itemize}

\logicwarn{实践指导}:在论争解决中,必须根据争议的性质选择合适的定义类型,特别要警惕说服定义中的情感操纵。
}