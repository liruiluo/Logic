\section{外延和内涵}

\begin{quotation}
\textit{在理解词项意义时,外延与内涵是两个不可或缺的概念。正确把握二者的关系有助于我们更精确地定义概念、避免语义混淆,从而提高思维和论证的准确性。}
\end{quotation}

定义旨在表明一个词项的意义(meaning),但是意义这个词却有不同含义(sense)。我们前面已区分了词项的描述或字面意义与表达性意义,现在,我们要更仔细地考察字面意义,尤其是普遍词项的字面意义。普遍词项就是可以运用于多于一个对象的类(class)的词项。在推理中,普遍词项的定义是特别重要的。

\subsection{外延意义与内涵意义}

普遍词项"行星"对水星、金星、地球、火星和土星等都是在同等含义上适用的。在一种含义上,词项"行星"意谓所有这些不同对象,而所有行星的汇集(collection)就构成"行星"的意义。如果我说所有行星都有椭圆轨道,那么我所断定的部分东西是火星有椭圆轨道,另一部分是金星有椭圆轨道,等等。在这种重要含义下,词项"行星"的意义便是由它适用的那些对象而构成的。"意义"的这种含义被称做词项的\textbf{外延意义}。通常,人们认为,普遍词项或曰类词项指谓(denote)其可以正确适用的那些对象。一个普遍词项可以正确适用的对象的汇集构成那个词项的\textbf{外延}。

理解普遍词项的意义就是知道怎样正确使用它;但是,这样做并不是一定要知道它可以正确适用的所有对象。对一个给定词项,其外延内的所有对象具有某些共同的性质或属性,这些性质或属性可以引导我们使用同一词项来指谓它们。因此,我们可以知道一个词项的意义而无须知道其外延。在第二种含义上,"意义"设定了决定任一对象是否属于那个词项外延的某种标准。"意义"的这种含义被称做词项的\textbf{内涵意义}。普遍词项指谓的所有对象并且仅仅那些对象共同拥有的属性集,称做那个词项的\textbf{内涵}(intension)。

\subsection{外延与内涵的关系}

这样,我们看到,每个普遍或类词项都既有一个内涵意义又有一个外延意义。普遍词项"摩天大厦"的内涵包括所有超过一定高度的建筑物的共同和特有性质。"摩天大厦"的外延是一个类,这个类包括纽约的世贸中心(World Trade Center)、芝加哥(Chicago)的希尔斯塔(Sear Tow- er)、上海世界金融中心(Shanghai World Financial Center)、吉隆坡 (Kuala I-umpur)的国油双峰塔(Petronas Twin Towers)等等,也即该词项适用对象的汇集。

有时,人们断言一个词项的外延不时发生变化,尽管它的内涵没有变化。例如,有人认为,词项"人"的外延,正如人的死亡和婴儿的降生一样,持续变化。这个说法源于一种混淆。词项"人"用来指谓所有的人,包括死去的以及尚未出生的,它并没有一个不确定的外延。变化的外延是词项"活着的人"的外延。但是,"活着的人"这个词项的外延具有"现在活着的人"这种含义,其中"现在"这个词是指不断变化的现时。因此,词项"活着的人"的内涵在不同的时候也是不同的。这样就清楚了,任何具有变化外延的词项必定也有一个变化的内涵,二者是同等恒定的。

\subsection{内涵决定外延而非相反}

当一个词项的内涵固定下来时,它的外延也就固定了。注意,词项的外延由它的内涵决定,但是反过来说却不对。词项"等边三角形"的内涵是由三条等长的直线所围成的平面图形的性质。它的外延是所有那些并且仅仅那些具有这种性质的对象的类。而"等角三角形"这个词项具有的内涵却不同,它是指由三条相互相交而形成等角的直线所围成的平面图形的性质。当然,"等角三角形"这个词项的外延与"等边三角形"这个词项的外延是完全相同的。因此,确认了这些词项其中一个词项的外延,而它的内涵却处于不确定状态;外延不决定内涵,但是,内涵却必定决定外延。因此,词项可以具有不同的内涵但外延却相同;而具有不同外延的词项却不可能有同样的内涵。

\subsection{内涵与外延的反变关系}

当给一个词项的内涵添加性质时,我们就说该内涵增加了。在下面一串词项中,每个词项的内涵都包含其后相随的词项的内涵:"人"、"活着的人"、"活着的二十岁以上的人"、"活着的二十岁以上有红发的人"。在这个序列中,每个词项的内涵都比其前的那些词项的内涵多;这些词项是按照内涵增加的次序来排列的。但是,如果我们倒过来看这些词项的外延,就会发现情况相反。"人"的外延比"活着的人"的外延大,等等,并且这些词项是按照外延减少的次序排列的。

有些逻辑学家得出一条公式化的\textbf{"反变规律"},断言外延与内涵总是反向变化。这种断言具有启发性,但并不完全正确。我们可以按照增加内涵的次序构建一系列词项,但外延却不减少而保持原样。考虑这样的序列:"活着的人"、"活着的有脊骨的人"、"活着的有脊骨的不超过一千岁的人"、"活着的有脊骨的不超过一千岁的没有读完国会图书馆(Library of Congress)里所有书的人"等。显然,这些词项的次序是增加内涵,但是它们每个的外延都是相同的,完全没有减少。正确的修订"规律"是,如果词项按照内涵增加的次序排列,那么它们的外延将处于非递增的次序;也就是说,如果外延变化,那么它们将是沿着内涵的反向变化。

\subsection{外延为空的词项与意义歧义}

当然,有些词项的外延,例如"独角兽"的外延,可能是空的。认识到这一点,并运用我们对内涵与外延的区分,就可以把玩弄"意义"歧义的谬误论证揭露出来。例如,下述论证的提出旨在证明上帝的存在:

\begin{displayquote}
"上帝"这个词不是无意义的,因此它有意义。但是按照定义,"上帝"这个词的意思是全能的至善的存在(being)。因此,全能的至善的存在,即上帝,必然存在(exist)。
\end{displayquote}

这里的歧义在于"意义"和"无意义"这两个词,其中的"意义"在一种含义上指的是内涵,而在另一种含义上指的却是外延。"上帝"这个词不是无意义的,因此可以肯定,存在一个内涵是它的意义。但是,由此并不能得出:一个具有内涵的词项,其内涵一定指谓一个存在物。\cite{gombocz1997}我们在下面这个语段中也发现了一个类似的谬误:

\begin{displayquote}
kitsch (低劣作品) 以展示粗鄙、卑劣、下贱、脆弱和邪恶信仰来表现并败坏人类境况。这就是乌托邦之所以能被定义为kitsch 这一词项已消失的状况的原因, 因为在乌托邦中该词项已没有所指了。\cite{sisk1988}
\end{displayquote}

这里列举的这个歧义谬误,就是因为作者没能在意义与所指(referent)之间做出区分。许多有价值的词项(例如,那些命名希腊神话中的动物的词项)都不存在所指,但是,我们并不要求或期望这样的词项消失。实际上,具有内涵但没有外延的词项是非常有用的;如果有一天乌托邦变成了现实,那么,我们也许想要表达对减少或消除"低劣作品"或 "粗鄙"等的庆幸。而要这样做,我们就需要能够有意义地使用这些词项。

在前面的几节中,我们考察了定义的种类和它们的用途:词典定义和规定定义可消除或避免歧义,精确定义可以减少模糊性,等等。在随后的几节中,我们将考察构建定义的方法。有些定义通过外延或所指来处理普遍词项,而其他定义则通过内涵来处理之。我们将会看到,每种处理方法都既有优点又有缺点。

\begin{center}
\fbox{\parbox{0.9\textwidth}{
  \centering
  \textbf{外延与内涵的重要特征}\\
  外延:指词项可以正确适用的所有对象的集合;\\
  内涵:指词项所表示的属性或特征的集合;\\
  关系:内涵决定外延,而非相反;词项可有相同外延但内涵不同;\\
  反变规律:当词项内涵增加时,其外延不会增加,通常会减少;\\
  空外延:词项可以有内涵而无外延,这不影响词项的有意义性。
}}
\end{center} 