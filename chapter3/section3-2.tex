\section*{3.2 定义的类型和论争的解决}
在对论争的各种类型和起因进行讨论之后,我们可以考虑定义的类型和它们在解决论争中的作用。定义是对词项意义的解说;因此,定义就旨在减少或消除由词项意义的不确定性或模糊性引起的困难。定义既可以阐释一个既有词项的既有意义,也可以赋予一个新词项以新意义。当目的在于前者时,定义就报告一种既定的用法;当目的在于后者时,定义就规定一种用法。当然,报告的意义可以是准确的也可以是不准确的;规定的意义可以是有用的也可以是无用的,可以是方便的也可以是不方便的。然而,在真或假与有用或无用之间存在根本区别。任何报告性定义(即词典定义)都可以是真的或假的,而任何规定性定义却都不能是真的或假的。

例如,"奇数"(odd number)可以定义为"任何不能被 2 整除的整数"。这是一个报告性定义,因为它报道了"奇数"这个词的既定用法,它是真的。或者,人们也可以提出"圆方形"(circle-square)这个新词,并把它定义为"既是圆又是正方形的图形",这是规定性定义,由于不存在圆方形,它在应用上就是无用的;由于它结合了两个不相容的属性,它就是逻辑上不可能的;但是,它既非真也非假。这个定义是"规定"这个新词的意义,而不是"报道"这个新词的既有意义。

当我们面对歧义时,词项的既定意义并不总是很清楚的。在这种情况下,仅仅报道词项的既有意义是不够的。例如,考虑短语"安乐死",它的定义可以为"有意致病人于死地,而又没有该病人同意的医疗行为"。这是对这个词的一种常见用法的报道性定义。但是,我们可能想要一个更精确的定义,一个可以澄清这个词项的模糊性的定义。为了达到这个目的,我们可以提出这样一个精确定义:"安乐死就是,经过病人的自由和知情的同意,由医生有意地导致的一个不能忍受持久痛苦的病人无痛苦的死亡"。在这种情况下,报告性定义的作用就是建立一种标准,精确定义则使这种标准更为精确。

对一个模糊词项给出一个精确定义,与给一个新词项赋予一个意义的规定定义,这两者之间是有区别的。精确定义并不是武断的,因为精确定义所定义的是一个已经具有固定用法的词项,而精确定义必须尽可能地保持这种固定用法。因此,在提出一个精确定义时,我们不仅要考虑使词项摆脱模糊性,也要考虑与现有用法的融贯性。一个词项的精确定义仍然可以被评价为真的或假的,但这要根据它是否符合现有用法而定。

然而,这种真或假的问题,对于一个规定定义来说是完全无关的。规定定义通常是当一个新词项引进时,或者当一个既有词项要在一个新语境中使用时所必需的。例如,物理学家引进了"焦耳"这个新词来表示一个功或能的单位,并规定性地将它定义为" 1 牛顿的力使物体在力的方向上移动 1 米所做的功"。或者,当一个计算机科学家把"硬盘"这个短语用于一种新型存储装置时,她对这个短语的使用就是规定性的。规定定义不能是真的或假的,因为它们定义的是一些此前没有定义的词项。

在规定性地定义一个词项时,我们通常会受到某些动机的引导。通常,规定定义是为了方便,或者是因为有些新事物需要命名。例如,"因特网"这个词就是一个相对较新的词,它被规定性地定义为"一个全球性的计算机网络系统,它使用 TCP/IP 协议族来连接全世界数以百万计的计算机"。在这种情况下,规定定义是为了用一个简短的词来指称一个复杂的概念。

然而,有时规定定义是出于理论的目的。在这种情况下,定义就旨在阐明一个理论概念。例如,在经济学中,"效用"这个词可以规定性地定义为"一种衡量消费者从消费一种商品或服务中获得的满足程度的指标"。这种定义是为了在一个经济理论的框架内阐明"效用"这个概念。这种理论定义也是规定性的,因为它们赋予一个词项一种新的、更专门的意义。

最后,有时定义是出于说服的目的。在这种情况下,定义的目的是影响人们的态度或行为。例如,政治家可能会将"爱国主义"定义为"对自己国家盲目的、不加批判的忠诚",以此来贬低那些持有不同政见的人。或者,广告商可能会将某种产品定义为"成功的象征",以此来吸引消费者购买。这种说服定义也是规定性的,因为它们赋予一个词项一种带有情感色彩的意义。

在解决论争时,识别和理解不同类型的定义是至关重要的。如果一个论争是由于对一个词项的意义存在分歧而引起的,那么通过提供一个清晰的、双方都能接受的定义,就可以解决这个论争。如果论争涉及到词项的模糊性,那么一个精确定义可以帮助澄清问题。如果论争涉及到新的概念或理论,那么规定定义或理论定义可以帮助阐明相关的思想。然而,如果论争涉及到说服定义,那么就需要警惕定义中可能包含的情感偏见。

总之,定义是解决论争和促进清晰思考的重要工具。通过仔细地考察和运用不同类型的定义,我们可以更有效地进行交流和推理。 