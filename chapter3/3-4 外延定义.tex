\section{外延定义}

\begin{logicbox}[title=引言]
\textit{外延定义是通过列举词项所适用的对象来界定其意义的方法。尽管这种定义方式直观明了,但它存在一定的局限性,理解这些局限性有助于我们更合理地运用外延定义。}
\end{logicbox}

\begin{theorembox}[title=外延定义的基本概念]
\logicemph{基本定义}:\logicterm{外延定义}就是指被定义的普遍词项所适用对象的汇集。

\logicemph{操作方法}:告诉某人词项外延的最方便有效的方法就是列举出其指谓的那些对象。

\logicwarn{重要提醒}:我们须认清这种方法的局限性。
\end{theorembox}

\subsection{外延定义的基本局限}

\begin{theorembox}[title=局限一:相同外延的不同内涵]
\logicemph{理论基础}:上一节我们曾指出(以"等边三角形"和"等角三角形"为例),具有不同意义即不同内涵的两个词项可以具有恰好相同的外延。

\logicwarn{问题后果}:
\begin{itemize}
  \item 即使我们能够完全列举出其中一个词项指谓的对象
  \item 由此而得出的外延定义也不能把它与另一个指谓同样对象的词项区分开来
  \item 这样的两个词项不是同义词,但外延定义却不能在它们之间做出区分
\end{itemize}
\end{theorembox}

\begin{theorembox}[title=局限二:完全列举的不可能性]
\logicwarn{更严重的问题}:这还不是问题的棘手之处,因为外延可以完全列举出来的词项是极少的。

\logicemph{典型例子}:
\begin{itemize}
  \item \logicterm{"数"}:要列举出这个词所指谓的所有数字是完全不可能的
  \item \logicterm{"恒星"}:要列举出这个词所指谓的天文数字的对象,实际上也是绝对不可能的
  \item \logicterm{其他大多数普遍词项}:完全列举其外延都是不可能的
\end{itemize}
\end{theorembox}

\subsection{部分列举的困难}

\begin{theorembox}[title=局限三:部分列举的严重困难]
\logicwarn{现实约束}:外延定义通常就要限定于所指谓对象的\logicterm{部分列举},这是个引起严重困难的局限。

\logicemph{根本问题}:任何给定对象都具有许许多多性质,因而被包括在许许多多不同的普遍词项的外延之中。
\end{theorembox}

\begin{examplebox}[title=约翰·多伊的多重身份问题]
\logicemph{问题分析}:
\begin{itemize}
  \item 当约翰·多伊(John Doe)作为一个词项的外延定义实例而被给出时
  \item 在许多其他词项的外延定义中,他也可以作为实例而被同样恰当地提及
\end{itemize}

\logicemph{多重归属}:约翰·多伊是以下词项的实例:
\begin{itemize}
  \item "人"、"动物"、"哺乳动物"
  \item 或许也是"丈夫"、"父亲"和"学生"等等
\end{itemize}

\logicwarn{区分失效}:因此提到他,并不能帮助我们在这些词项的意义之间做出区分。即使我们给出两个、三个或更多实例,也会遇到同样的困难。
\end{examplebox}

\begin{examplebox}[title=摩天大厦定义的歧义问题]
\logicemph{定义尝试}:在定义"摩天大厦"这个词时,我们可以使用明显的范例:
\begin{itemize}
  \item 帝国大厦(Empire State Building)
  \item 克莱斯勒大厦(Chrysler Building)
  \item 沃尔华斯大厦(Woolworth Building)
\end{itemize}

\logicwarn{歧义问题}:这三座大厦作为实例同样也完全可以作为以下词项的外延:
\begin{itemize}
  \item "二十世纪的伟大建筑"
  \item "曼哈顿的昂贵房地产"
  \item "纽约城的地面标志物"
\end{itemize}

\logicemph{更严重的问题}:每个这些普遍词项都指谓其他词项不指谓的对象,因此通过使用部分列举,我们甚至不能在具有不同外延的词项之间做出区分。
\end{examplebox}

\begin{theorembox}[title=反面事例的局限性]
\logicemph{改进尝试}:引进\logicterm{"反面事例"},例如"不是泰姬陵"(Taj Mahal)或"不是五角大楼"(the Pentagon),可以帮助说明被定义项的意义。

\logicwarn{根本局限}:但是,否定事例也必定是不完善的,这个基本局限仍然存在。
\end{theorembox}

\subsection{通过子类列举的方法}

\begin{theorembox}[title=子类列举方法]
\logicemph{方法改进}:我们可以设法换一种列举方式,即不是每次列举一例,而是列举一整组例子。

\logicemph{定义方式}:使用这种方法,也就是通过\logicterm{子类来定义},有时可能做到完全的列举。
\end{theorembox}

\begin{examplebox}[title=脊椎动物的子类定义]
\logicemph{定义实例}:我们把"脊椎动物"定义为:
\begin{itemize}
  \item 两栖动物
  \item 鸟类
  \item 鱼类
  \item 哺乳动物
  \item 爬行动物
\end{itemize}

\logicemph{优势}:这种方法实现了完全列举,避免了遗漏。
\end{examplebox}

\begin{theorembox}[title=列举定义的总体评价]
\logicwarn{逻辑不充分性}:通过列举定义,无论完全列举还是部分列举,无论是列举类的个体元素还是列举子类,虽然都具有某些心理上的用处,但是要完全确定被定义项的意义,在逻辑上都是不充分的。

\logicemph{心理价值}:尽管逻辑上不充分,但在帮助理解方面仍有价值。
\end{theorembox}

\subsection{实指定义及其局限}

\begin{theorembox}[title=实指定义的基本概念]
\logicemph{基本定义}:用列举法下定义的一个特殊类型被称做\logicterm{"实指定义"}或\logicterm{"示范定义"}。

\logicemph{与一般外延定义的区别}:
\begin{itemize}
  \item \logicterm{实指定义}:通过用手或其他姿势指着对象来定义
  \item \logicterm{一般外延定义}:通过命名或描述被定义词项指谓的对象来定义
\end{itemize}
\end{theorembox}

\begin{examplebox}[title=实指定义的典型例子]
\logicemph{定义表述}:"'桌子'这个词意指这",伴随着一个姿势如用手指指着桌子的方向。

\logicemph{特征}:直接的物理指示行为构成了定义的核心部分。
\end{examplebox}

\begin{theorembox}[title=实指定义的双重局限性]
\logicwarn{局限性来源}:实指定义既有其自身的某些特殊局限性,也有前面所提到的各种局限性。
\end{theorembox}

\begin{theorembox}[title=局限一:地域限制]
\logicwarn{显而易见的地域局限}:一个人只能指着看得见的东西。

\logicemph{典型例子}:
\begin{itemize}
  \item 他不能在乡村来实指地定义"摩天大厦"
  \item 也不能在内陆山谷中去实指地定义"海洋"
\end{itemize}

\logicemph{根本原因}:物理指示需要对象的在场和可见性。
\end{theorembox}

\begin{theorembox}[title=局限二:指示歧义]
\logicwarn{更为严重的问题}:姿势也有着不可避免的歧义。

\logicemph{歧义的复杂性}:指着一张桌子也是指着:
\begin{itemize}
  \item 它的一部分
  \item 它的颜色、大小、形状、质料等等
  \item 位于桌子所在的方向上的所有东西
  \item 包括它后面的墙壁或者更远处的花园
\end{itemize}

\logicwarn{根本问题}:指示行为本身具有内在的多义性。
\end{theorembox}

\begin{theorembox}[title=准实指定义的改进尝试]
\logicemph{解决方案}:这种歧义有时可以通过给定义项增加一些描绘的短语而得到解决,其结果被称做\logicterm{准实指定义}。

\logicemph{改进例子}:"桌子"这个词意指"这件"家具(伴随以相应的姿势)。

\logicwarn{新问题}:因为这种附加假设了对"家具"这个短语的事先理解,就使实指定义的宗旨难以达到。
\end{theorembox}

\subsection{实指定义的非原初性}

\begin{theorembox}[title=关于基本性的误解]
\logicwarn{常见误解}:实指定义历来被某些人视为"基本"或"原初"定义,其意思是说:我们最初都是凭借这种方式来理解词项意义的,而其他定义都依赖于这种理解。

\logicemph{错误性质}:但是,这种原初性断言是错误的。
\end{theorembox}

\begin{theorembox}[title=姿势理解的前提性]
\logicwarn{根本依赖}:人们必须理解姿势本身的意义。

\logicemph{婴儿实验的启示}:
\begin{itemize}
  \item 当我们用手指指向婴儿床沿时,如果那个婴儿的注意力被吸引
  \item 其注意力可能放到被指向的东西上,也可能放到我们的手指上
  \item 如果我们想用其他姿势来定义一个姿势,也会出现同样的困难
\end{itemize}

\logicwarn{符号理解的循环性}:如果你要理解任何符号(sign)的定义,那么某些符号就必须已经得到理解。

\logicemph{语言学习的真实途径}:我们学习使用语言的根本途径是经过观察和模仿,而不是经过定义。
\end{theorembox}

\begin{theorembox}[title=术语使用的混淆]
\logicwarn{宽泛解释的问题}:一如有些逻辑学家所做的那样,人们可以很宽泛地解说"实指定义"这个术语,甚至包括"当这个词指谓的对象出现时,不断地听到这个词"的过程。

\logicemph{概念澄清}:
\begin{itemize}
  \item 这种过程根本不是定义
  \item 与我们这里对词项"定义"的使用不一样
  \item 它是学习使用语言的基本的和前定义的(predefinitional)方式
\end{itemize}
\end{theorembox}

\subsection{空外延词项的问题}

\begin{theorembox}[title=空外延词项的根本挑战]
\logicwarn{最终局限}:虽然有些词意义丰富,但是它们完全不指谓任何事物,因此不能从外延上定义它们。

\logicemph{典型例子}:当我们说不存在独角兽时,我们在断定"独角兽"这个词没有所指,具有一个\logicterm{"空"外延}。
\end{theorembox}

\begin{theorembox}[title=空外延词项的理论意义]
\logicemph{双重启示}:这类词项不仅仅展示了外延定义的局限性,也显示出"意义"的确更适用于内涵而不是外延。

\logicemph{逻辑分析}:
\begin{itemize}
  \item 虽然"独角兽"这个词的外延是空的,但由此肯定不是说它就是无意义的
  \item 的确,它不指谓任何事物,因为根本就没有独角兽
  \item 但是,如果"独角兽"这个词项毫无意义,那么说"不存在独角兽"也就是无意义的
  \item 然而,这个陈述并不是没有意义的,我们完全理解它的意义,而且它是真的
\end{itemize}

\logicwarn{理论结论}:显然,内涵对定义来说是真正的关键所在,我们下一节即转向讨论内涵。
\end{theorembox}

\chaptersummary{
外延定义作为通过列举对象来说明词项意义的方法,虽然直观易懂,但存在多重根本性局限。

\logicemph{外延定义的基本方法}:
\begin{itemize}
  \item \logicterm{完全列举}:列举词项所适用的所有对象
  \item \logicterm{部分列举}:列举词项的部分适用对象
  \item \logicterm{子类列举}:通过列举子类而非个体对象
  \item \logicterm{实指定义}:通过指示动作来定义词项
\end{itemize}

\logicemph{系统性局限分析}:
\begin{itemize}
  \item \logicwarn{完全列举的不可能性}:大多数词项无法完全列举其所有适用对象,如"数"、"恒星"等
  \item \logicwarn{部分列举的歧义性}:难以准确区分不同词项,同一对象可能是多个不同词项的实例
  \item \logicwarn{相同外延的不同内涵}:具有相同外延的词项可能有不同内涵,外延定义无法区分
  \item \logicwarn{实指定义的特殊问题}:地域限制、指示歧义、非原初性等问题
  \item \logicwarn{空外延词项}:某些有意义的词没有所指对象,根本无法通过外延定义
\end{itemize}

\logicemph{理论启示}:
\begin{itemize}
  \item 外延定义虽然具有心理上的用处,但在逻辑上是不充分的
  \item 空外延词项的存在表明"意义"更适用于内涵而不是外延
  \item 内涵对定义来说是真正的关键所在
\end{itemize}

\logicwarn{过渡意义}:这些局限性的分析为转向内涵定义提供了充分的理论依据。
}