\section{无效性证明}

\begin{logicbox}[title=引言]
本节讨论如何证明涉及量词的论证的无效性。通过引入特定的模型和真值指派方法,我们可以系统性地处理含有量化命题的论证,并确定它们是否无效。相比于简单的逻辑类比,这种方法提供了更加严格和普遍的无效性证明技术。
\end{logicbox}

\subsection{逻辑类比的局限性}

要证明一个涉及量词的论证无效,我们可以用逻辑类推进行反驳的方法。例如:"所有保守派都是行政机关的反对者;有些代表是行政机关的反对者;因此,有些代表是保守派。"这个论证可以通过这样一个逻辑类推被证明为无效,即"所有猫都是动物;有些狗是动物;因此,有些狗是猫"。这个论证显然无效,因为已知它的前提为真而结论为假。但这种类比并非总是很容易构造。因此,需要某种更有力的证明无效性的方法。

\subsection{真值指派方法的扩展}

在前一章中,我们详述了一种证明涉及真值函项复合陈述的论证之无效性的方法。这种方法是通过对论证中的简单分支陈述进行真值指派,使得论证的前提为真而结论为假。我们可以设法使这种方法适用于使用量词的论证。这涉及这样一个一般假定,即至少存在一个个体。若一个涉及量词的论证有效,那么,只要至少有一个个体存在,这个论证的前提为真而结论为假就必定是不可能的。

\subsection{有限域模型中的量化}

如果恰好存在一个个体,两个个体,三个个体……那么,至少存在一个个体这个一般假定就得到了满足。如果作了任何这样一个关于个体的确切数量的假定,就有一个关于普遍命题与单称命题的真值函项复合式的等价式。如果刚好存在一个个体,譬如说 $a$ ,那么:

$$
(x)\left(\phi_{\mathrm{x}}\right) \stackrel{\mathrm{T}}{=} \phi_{a} \stackrel{\mathrm{T}}{=}(\exists x)\left(\phi_{x}\right)
$$

如果刚好存在两个个体,臂如说 $a$ 和 $b$ ,那么:

$$
(x)\left(\phi_{x}\right) \stackrel{\mathrm{T}}{\equiv}\left[\phi_{a} \cdot \phi_{b}\right] \text {, 而 }(\exists x)\left(\phi_{x}\right) \stackrel{\mathrm{T}}{\equiv}\left[\phi_{a} \vee \phi_{b}\right]
$$

如果刚好存在三个个体,譬如说 $a 、 b$ 和 $c$ ,那么:

$$
(x)\left(\phi_{x}\right) \stackrel{\mathrm{T}}{=}\left[\phi_{a} \cdot \phi_{b} \cdot \phi_{c}\right] \text {, 而 }(\exists x)\left(\phi_{x}\right) \stackrel{\mathrm{T}}{=}\left[\phi_{a} \vee \phi_{b} \vee \phi_{c}\right]
$$

一般的,如果刚好存在 $n$ 个个体,譬如说 $a 、 b 、 c \cdots \cdots n$ ,那么:

$$
\begin{aligned}
& (x)\left(\phi_{x}\right) \stackrel{\mathrm{T}}{=}\left[\phi_{a} \cdot \phi_{b} \cdot \phi_{c} \cdots \cdot \phi_{n}\right] \\
& \text { 而 }(\exists x)\left(\phi_{x}\right) \cong\left[\phi_{a} \vee \phi_{b} \vee \phi_{c} \vee \cdots \vee \phi_{n}\right]
\end{aligned}
$$

由于它们是我们关于全称和存在量词定义的推论,所以这些双条件陈述为真。这里并没有用到前一节所阐释的四个量化规则。

\subsection{使用模型证明无效性}

一个涉及量词的论证有效,当且仅当,不管存在多少个体它都是有效的,一一假定至少存在一个个体的话。因此,如果存在一个至少含有一个个体的可能域或\textbf{模型},它使得某论证相对该模型来说,其前提为真而结论为假,那么,这样一个涉及量词的论证就被证明为无效。考察论证:"所有雇佣兵都是不可靠的。没有游击队员是雇佣兵。因此没有游击队员是不可靠的。"它可以符号化为:

$$
\begin{aligned}
& (x)(M x \supset U x) \\
& (x)(G x \supset \sim M x) \\
& \therefore(x)(G x \supset \sim U x)
\end{aligned}
$$

如果刚好存在一个个体,譬如说 $a$ ,这个论证逻辑地等价于:

$$
\begin{aligned}
& M a \supset U a \\
& G a \supset \sim M a \\
& \therefore G a \supset \sim U a
\end{aligned}
$$

给 $G a$ 和 $U a$ 指派真值真,给 Ma 指派真值假,即可以证明上式是无效的。 (这种真值指派是一种简略的描述方式,它把所讨论的模型描述成只含有一个个体 $a$ ,这个个体是游击队员且不可靠,但不是雇佣兵。)于是,原来的论证对于一个只含有一个个体的模型来说不是有效的,因此它是无效的。类似的,通过描述只含有一个个体 $a$ 的模型,使得 $A a$ 和 $D a$ 被赋值为真,且 $C a$ 被赋值为假,我们就可以证明本节提到的第一个论证的无效性。\cite{tarski1954}

\subsection{逐渐扩大模型的方法}

有些论证对于刚好只有一个个体的模型来说是有效的,但对于有两个或更多个体的模型来说则不然。譬如:

$$
\begin{aligned}
& (\exists x) F x \\
& \therefore(x) F x
\end{aligned}
$$

这样的论证必须被当做是无效的,因为只要至少存在一个个体,那么,一个有效的论证就必定有效而不管存在多少个体。这种论证的另一个例子是:"所有牧羊犬都是可爱的。有些牧羊犬是看门狗。因此,所有看门狗都是可爱的。"它的符号翻译是:

$$
\begin{aligned}
& (x)(C x \supset A x) \\
& (\exists x)(C x \cdot W x) \\
& \therefore(x)(W x \supset A x)
\end{aligned}
$$

对一个刚好只有一个个体 $a$ 的模型来说,该论证逻辑地等价于:

$$
\begin{aligned}
& C a \supset A a \\
& C a \cdot W a \\
& \therefore W a \supset A a
\end{aligned}
$$

这个论证是有效的。但对一个有两个个体譬如 $a$ 和 $b$ 的模型来说,它逻辑地等价于:

$$
\begin{aligned}
& (C a \supset A a) \cdot(C b \supset A b) \\
& (C a \cdot W a) \vee(C b \cdot W b) \\
& \therefore(W a \supset A a) \cdot(W b \supset A b)
\end{aligned}
$$

通过对 $C a 、 A a 、 W a 、 W b$ 指派真,对 $C b 、 A b$ 指派假,可以证明该论证无效。于是,原论证对一个刚好有两个个体的模型来说不是有效的,因此它是无效的。对任何这种一般类型的无效论证来说,有可能描述一个含有有限数量个体的模型,用真值指派的方法可以证明,与这个论证逻辑等价的真值函项论证相对于该模型是无效的。

需要再次强调:在从一个涉及普遍命题的论证转化为一个真值函项论

证(相对于某特定模型,它逻辑等价于给定论证)的过程中,并没有用到我们的那四个量化规则。相反,真值函项论证的每个陈述,逻辑地等价于给定论证中与之对应的普遍命题。这种逻辑等价可以由本节中早些时候所阐述的那些双条件陈述来解释。相对于所讨论的那个模型,它们的逻辑真可以从全称量词和存在量词的定义推出。

\subsection{无效性证明的一般步骤}

证明一个含有普遍命题的论证无效的程序如下。首先,考察一个只含有一个个体 $a$ 的\textbf{一元模型}。然后,写出该论证相对于此模型的逻辑等价真值函项论证。通过把原论证的每个普遍命题(量化的命题函项)转化为该命题函项关于 $a$ 的代人例,就可以做到这一点。如果对它的简单分支陈述进行真值指派可以证明该真值函项论证无效,那么这就足以证明原论证无效。如果不能做到这一点,就接着考察一个含有两个体 $a$ 和 $b$ 的\textbf{二元模型}。为了得到相对于这个更大模型来说逻辑等价的真值函项论证,我们可以简单地把原来关于 $a$ 的每个代人例和一个关于 $b$ 的新代人例结合起来。这种"结合"必须依照前面所陈述的那些逻辑等价式。也就是说,在原论证含有一个全称量化的命题函项( $x$ )( $\phi_{x}$ )时,就用合取("•")把新的代人例 $\phi_{b}$ 和第一个代人例 $\phi_{a}$ 结合起来;在原论证含有一个存在量化的命题函项( $\exists x$ )( $\phi_{x}$ )时,就用析取(" V ")把新的代人例 $\phi_{b}$ 和第一个代入例 $\phi_{a}$ 结合起来。前述例子说明了这种程序。如果对它的简单分支陈述进行真值指派可以证明该真值函项论证无效,那么这就足以证明原论证无效。如果做不到这一点,就接着考察一个含有个体 $a 、 b$ 和 $c$ 的三元模型等等。本书中没有哪个习题要求一个含有超过三个元素的模型。

\begin{center}
\fbox{\parbox{0.95\textwidth}{
\textbf{本节要点}
\begin{itemize}
\item \textbf{无效性证明方法}:
  \begin{itemize}
  \item 逻辑类比方法:构造结构相似但明显无效的类比论证
  \item 真值指派方法:在特定模型中找到使前提为真而结论为假的情况
  \item 模型扩展法:从一元模型开始,逐渐扩大模型规模直到找到反例
  \end{itemize}
\item \textbf{有限域中的量化等价式}:
  \begin{itemize}
  \item 一元模型:(x)(φx) ≡ φa ≡ (∃x)(φx)
  \item 二元模型:(x)(φx) ≡ [φa·φb],(∃x)(φx) ≡ [φa∨φb]
  \item 三元模型:(x)(φx) ≡ [φa·φb·φc],(∃x)(φx) ≡ [φa∨φb∨φc]
  \item 这些等价式基于量词的定义,而非推论规则
  \end{itemize}
\item \textbf{无效性的定义}:
  \begin{itemize}
  \item 论证有效当且仅当在任何模型中前提为真时结论也为真
  \item 只要存在一个模型使前提为真而结论为假,论证就是无效的
  \item 有些论证在一元模型中有效,在更大模型中无效
  \end{itemize}
\item \textbf{实际证明步骤}:
  \begin{itemize}
  \item 从一元模型开始,将量化命题转换为真值函项命题
  \item 若找不到反例,扩展到二元模型,按相应规则合并新代入例
  \item 全称量化用合取连接各代入例,存在量化用析取连接各代入例
  \item 在扩展的模型中寻找使前提为真而结论为假的真值指派
  \end{itemize}
\end{itemize}
}}
\end{center}