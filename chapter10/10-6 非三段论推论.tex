\section{非三段论推论}

\begin{quotation}
本节将我们的分析扩展到更复杂的论证形式——非三段论推论。这类论证不限于传统直言三段论的结构,而是涉及更复杂的内部逻辑关系和命题形式。我们将学习如何正确地符号化这些复杂论证,避免常见的误解,并应用已建立的逻辑工具来判断它们的有效性。
\end{quotation}

\subsection{超越三段论的限制}

前两节讨论的所有论证都具有传统上叫做直言三段论的形式。它们由两个前提和一个结论组成,每个前提和结论都可以分析成一个单称命题或 $A 、 E 、 I 、 O$ 中的某一种。现在我们转向评价更复杂一些的论证。评估这些论证并不需要比此前已经给出的更多的逻辑工具。这些论证称为\textbf{非三段论论证},这就是说,它们不能划归为标准形式的直言三段论。因此,评价它们就需要一种比传统上检验直言三段论所使用的更有力的逻辑。

本节我们仍关注普遍命题,它们是通过量化只含有一个个体变元的命题函项而形成的。在直言三段论中,被量化的命题函项具有 $\phi_{x} \supset \Psi x$ , $\phi_{x} \supset \sim \Psi_{x}, \phi_{x} \cdot \Psi x, \phi_{x} \cdot \sim \Psi x$ 形式。但现在我们要量化一些具有更复杂内部结构的命题函项。下述例子有助于说明问题,请考虑论证:

\begin{quote}
旅馆都是既贵又令人压抑的。\\
有些旅馆简陋。\\
因此,有些贵的东西简陋。
\end{quote}

\subsection{适当符号化的重要性}

该论证显然是有效的,但它并不能用传统方法加以分析。若分别用符号 $H x, B x, S x$ 和 $E x$ 缩写命题函项"$x$ 是旅馆","$x$ 既贵又令人压抑", "$x$ 是简陋的"和"$x$ 是贵的",该论证的确可以用 A 和 I 命题来表达。\cite{lukasiewicz1951}用这些缩写形式可把该论证符号化为:

$$
\begin{aligned}
& (x)(H x \supset B x) \\
& (\exists x)(H x \cdot S x) \\
& \therefore(\exists x)(E x \cdot S x)
\end{aligned}
$$

但以这种方式强迫该论证受传统的 A 和 I 形式的束缚,就遮蔽了它的有效性。尽管原来的论证非常有效,但刚才用符号给出的论证却是无效的。这里对直言命题所施加的符号限制遮蔽了 $B x$ 和 $E x$ 之间的逻辑联系。用如上所解释的 $H x 、 S x$ 和 $E x$ ,加上 $D x$ ,我们可以获得一个更适当的分析。在此,$D x$ 是"$x$ 是令人压抑的"的缩写。原来的论证用这些符号可以翻译成:

1.$(x)[H x \supset(E x \cdot D x)]$\\
2.$(\exists x)(H x \cdot S x)$\\
$\therefore(\exists x)(E x \cdot S x)$\\
经过如此符号化,它的有效性证明很容易构造。这样的证明可以如下进行:

\begin{center}
\begin{tabular}{ll}
3.Hw•Sw & 2,EI \\
4.Hwつ(Ew•Dw) & 1,UI \\
5.Hw & 3,Simp. \\
6.Ew•Dw & 4,5,M.P. \\
7.Ew & 6,Simp. \\
8.Sw•Hw & 3,Com. \\
9.Sw & 8, Simp. \\
10.Ew•Sw & 7,9,Conj. \\
11.$(\exists x)(E x \cdot S x)$ & 10,EG \\
\end{tabular}
\end{center}

\subsection{自然语言符号化的潜在问题}

在对经量化更复杂的命题函项而产生的普遍命题进行符号化时,必须小心不要被日常语言的表述方式所误导。我们不能依照任何形式的或机械的规则来把自然语言翻译为逻辑符号。在每种情形下,必须理解自然语言

语句的意义,然后用命题函项和量词术语加以符号化。\\
日常语言中有时令人困扰的三种表达方式是这样的。第一,像"所有运动员力气大或跑得快"这样的陈述,尽管它含有联结词"或",但它不是一个析取式。它无疑和"或者所有运动员力气大或者所有运动员跑得快"不具有同样的含义。使用缩写形式,前者可以恰当地符号化为:

$$
(x)[A x \supset(S x \vee Q x)]
$$

而后者却可以符号化为:

$$
(x)(A x \supset S x) \vee(x)(A x \supset Q x)
$$

第二,我们注意到,"牡蛎和蚌好吃"这样的陈述,可以被表述为两个普遍命题的合取,即"牡蛎好吃并且蚌好吃";但它也可被表述为一个单一的非复合普遍命题。在这种情况下,语词"和"可以用"$V$"而不是 "•"来恰当地符号化。该命题可以符号化为:

$$
(x)[(O x \vee C x) \supset D x]
$$

而不是

$$
(x)[(O x \cdot C x) \supset D x]
$$

因为说牡蛎和蚌好吃,就是说任何一个或者是牡蛎或者是蚌的东西好吃,而不是说任何一个既是牡蛎又是蚌的东西好吃。

\subsection{除外命题的处理}

第三,对所谓的\textbf{除外命题}要格外小心。如"除以前的获胜者外,都符合条件"这样的命题,可以被处理成两个普遍命题的合取。利用刚给出的那个例子,我们可以合理地把此命题理解为断言:以前的获胜者不符合条件,并且那些不是以前的获胜者的人符合条件。因此,它可以符号化为:

$$
(x)(P x \supset \sim E x) \cdot(x)(\sim P x \supset E x)
$$

但这个同样的除外命题也可以翻译成一个非复合的普遍命题,这个命题是一个含有实质等值符"三"的命题函项的全称量化式,它是一个双条件陈述,可以符号化为:

$$
(x)(E x \equiv \sim P x)
$$

这个符号表达式也可以用日常语言翻译成"任何人要符合条件,当且仅当,这个人不是以前的获胜者"。一般来说,除外命题可以最方便地看做

是量化了的双条件陈述。\\
有时很难确定一个命题事实上是否是除外命题。近期一件要求联邦法庭全体陪审员解决的纠纷说明了这种情境上的困难。《人口调查法》制定了每十年进行一次的全国普查的一些规则,它有这样一段话:

195 节.除为了在几个州中分配国会代表的席位而确定人口数量以外,[商业]部长在执行这项权利的有关规定时,有权批准使用"抽样"统计方法,如果他认为这是可行的话。

在因分配国会代表席位要确定人口数量而进行的 2000 年的普查中,普查局想使用抽样技术,但被众议院控诉。众议院宣称上面的引文禁止在这样一次普查中进行抽样。普查局对此作了辩护,认为这段话批准在某些情境中使用抽样,但在席位分配情境中却悬而末决。对法规中除外规定的哪种解释是正确的呢?

法庭认为众议院的见解正确,它写道:

考察这样一个指令,"除我祖母的结婚礼服外,把我衣榭里的东西都送到洗衣店去"。……这似乎是说,如果该孙女的指令的接受者把结婚礼服送到洗衣店去,并且随后争辩说她把这留给他作决定,那么她会气恼。产生这一结果的原因……是因为我们关于结婚礼服的背景知识:我们知道它们特别易坏,并且对家庭成员具有极深的情感价值。因此,我们不希望决定把礼服送到洗衣店是完全任意的。

各州国会代表席位的分配就是衣瀜中的那件结婚礼服……分配函数是"十年一度的普查的单调构成性函数",其执行方式不仅影响各州代表席位的分配,而且影响众议院中政治力量的平衡……本法庭认为,《人口调查法》禁止为了在州中分配代表席位而去确定人口数量时使用统计抽样法……[10]

因此,这个法规中的除外命题被理解为断定这两个命题的合取:(1)在分配席位的情境中,使用抽样是不允许的,(2)在所有其他情境中,可以任意使用抽样。一个除外形式的争议性语句必须在其情境中来理解。

\subsection{非三段论论证的有效性判断}

在 10.4 节,我们的推论规则表增加了 4 个规则,并且表明,这个扩展表足以证明有效的直言三段论的有效性。刚才已经看到,同一扩展表足以确立所描述类型的非三段论论证的有效性。现在我们可以观察到,正如扩展表足以在非三段论论证中判定有效性一样,证明三段论无效的(在 10.5 节所解释的)方法,即通过描述非空的可能域或模型,也足以证明当前这种非三段论论证的无效性。考虑下面这个非三段论论证:

经理和主管或者是有能力的员工,或者是所有者的亲属。\\
敢抱怨的人必定或者是主管,或者是所有者的亲属。\\
唯有经理和工头是有能力的员工。\\
某人敢抱怨。\\
因此,某个主管是所有者的亲属。

可以符号化为:

$$
\begin{aligned}
& (x)[(M x \vee S x) \supset(C x \vee R x)] \\
& (x)[D x \supset(S x \vee R x)] \\
& (x)(M x \equiv C x) \\
& (\exists x) D x \\
& \therefore(\exists x)(S x \cdot R x)
\end{aligned}
$$

通过描述一个只含有个体 $a$ 的可能域或模型,并对 $C a 、 D a 、 F a$ 和 $R a$ 指派真值真,对 Sa 指派真值假,我们可以证明它无效。 

\begin{center}
\fbox{\parbox{0.95\textwidth}{
\textbf{本节要点}
\begin{itemize}
\item \textbf{非三段论推论的特点}:
  \begin{itemize}
  \item 超越了传统直言三段论的形式和限制
  \item 涉及更复杂内部结构的量化命题函项
  \item 可用相同的逻辑工具(四个量化规则)进行分析
  \end{itemize}
\item \textbf{符号化中的关键问题}:
  \begin{itemize}
  \item 准确符号化对判断论证有效性至关重要
  \item 过度简化可能掩盖论证的真实逻辑结构
  \item 正确分析复合概念(如"既贵又令人压抑")的内部结构
  \end{itemize}
\item \textbf{自然语言表达的陷阱}:
  \begin{itemize}
  \item "所有A都是B或C"≠"或者所有A都是B,或者所有A都是C"
  \item "A和B都是C"应译为"(x)[(Ax∨Bx)⊃Cx]",而非"(x)[(Ax·Bx)⊃Cx]"
  \item 除外命题需要根据上下文正确解释,可表示为双条件陈述或合取命题
  \end{itemize}
\item \textbf{有效性和无效性判断}:
  \begin{itemize}
  \item 证明有效性:使用相同的四个量化推论规则
  \item 证明无效性:构造具有特定真值指派的有限模型
  \item 同样的方法适用于三段论和非三段论推论
  \end{itemize}
\end{itemize}
}}
\end{center} 