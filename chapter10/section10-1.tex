\section{单称命题}

\begin{quotation}
本节引入了单称命题的概念及其符号化方法。通过分析命题的内在逻辑结构,我们将学习如何区分个体与属性,以及如何使用个体常元和谓词符号来表示各种类型的单称命题,从而为分析更复杂的论证奠定基础。
\end{quotation}

前两章的逻辑技术使得我们可以对某种类型的有效论证和无效论证进行区分。该类型的论证可以粗略地刻画为:其有效性仅取决于简单陈述通过真值函项结合成复合陈述的方式。然而,还有另外一些类型的论证,前两章的有效性标准对它们不够用。一个明显有效的不同类型论证的例子是:

所有人都是有死的。\\
苏格拉底是人。\\
因此,苏格拉底是有死的。

如果把前面介绍的评估方法运用到这个论证上,我们可以将之符号化为:\\
A\\
H\\
$\therefore M$\\
在这种符号式中,该论证显然无效。因此,到目前为止所介绍的符号逻辑技术不能直接运用到这种新型论证上。该论证的有效性并不取决于简单陈述的复合方式,因为在该论证中没有出现任何复合陈述。毋宁说它的有效性取决于所涉及非复合陈述的内在逻辑结构。要阐明检验这种新型论证有效性的方法,就必须依据它们的内在逻辑结构,设计出一些描述和符号化非复合陈述的技术。

\subsection{单称命题的结构}

一种最简单的非复合陈述的例示是前述论证中的第二个前提:"苏格拉底是人。"这种类型的陈述传统上叫做\textbf{单称命题}。一个(肯定的)单称命题断言的是,一个特定个体具有某种特定属性。在上述例子中,日常语法和传统逻辑都一致地把"苏格拉底"划为主项,把"人"划为谓项。主项指称某特定个体,谓项指谓该个体据称所具有的某种属性。

显然,同一主项可以在不同的单称命题中出现。因此,在下述每个命题中,我们都以词项"苏格拉底"做主项:"苏格拉底是有死的","苏格拉底是胖的","苏格拉底是聪明的"和"苏格拉底是漂亮的"。当然,有些是

真的(第一和第三个),有些是假的(第二和第四个)。 ${ }^{[2]}$ 同一谓项显然也可以出现在不同的单称命题中。因此在下述每个命题中,我们都以词项"人"做谓项:"亚里士多德是人","巴西是人","芝加哥是人"和"奥基夫是人"。当然,有些是真的(第一和第四个),有些是假的(第二和第三个)。

从前所述,我们应该清楚语词"个体"不仅用来指人,还可以指事物,譬如,国家、城市,实际上可以指谓像是人或有死的这样能被有意义地断言为其属性的任何事物。前面所举的例子中,有些谓项是形容词。从日常语法的观点看,形容词与名词的区分是相当重要的。但在本章中这种区别并不重要,我们并不区分"苏格拉底是有死的"和"苏格拉底是有死者",或"苏格拉底是聪明的"和"苏格拉底是一个聪明的人"。一个谓项可以是一个形容词或者是一个名词,甚或是一个动词。如在"亚里士多德写作"中,它有时可以被表述为"亚里士多德是一个写作者"。

\subsection{单称命题的符号化}

假定我们能区分开具有某属性的个体和它们所具有的属性,我们引进并使用两种不同的符号来指称它们。在随后的讨论中,我们将用从 $a$ 到 $w$的小写字母来指谓\textbf{个体}。这些符号是\textbf{个体常元}。在它们出现的任何特定上下文,每个在该整个上下文中都指称一个特定的个体。用它(他,或她)的名称的第一个字母指称一个个体,通常是很方便的。因此在当前的上下文中,我们应分别用字母 $s 、 a 、 b 、 c 、 o$ 指称苏格拉底、亚里士多德、巴西、芝加哥和奥基夫。我们将用大写字母来符号化\textbf{属性},在此使用同样的指导原则是很便利的。因此,我们用字母 $H 、 M 、 F 、 W 、 B$ 分别符号化属性是人、有死的、胖的、聪明的、漂亮的。

有两组符号,一组是个体的符号,另一组是个体属性的符号。我们采取这样一个约定:把属性符号直接写在个体符号的左边,表征被命名的个体具有规定的属性这样一个单称命题。于是,单称命题"苏格拉底是人"可以符号化为 $H s$ 。上面提到的涉及谓项"人"的其他一些单称命题,分别可以符号化为 $H a 、 H b 、 H c$ 和 $H o$ 。我们注意到它们都有某种共同的模式,即它们不是被符号化为 $H$ 自身,而是 $H$ —。在此,"一"表示在谓项符号的右边有另一个符号即个体符号出现。习惯上用字母 $x$(这是可以的,因为我们只用从 $a$ 到 $w$ 的字母做个体变元)而不是用破折号 ("——")作替代标示。我们用 $H x$[有时写成 $H(x)$ ]来符号化所有以 "是人"作为个体属性的单称命题的共同模式。被称做\textbf{个体变元}的字母 $\boldsymbol{x}$只是一个位置标示,用来指示从 $a$ 到 $w$ 的各个字母——个体常元——可以

填入以便产生单称命题的位置。

\subsection{命题函项}

各种单称命题 $H a 、 H b 、 H c$ 和 $H d$ 是或真或假的;但由于 $H x$ 根本不是陈述或命题,它既不真也不假。表达式 $H x$ 是一个\textbf{命题函项},它可以被定义成这样一个表达式:(1)含有个体变元;(2)当一个个体常元代入个体变元时,它就变成一个陈述。 ${ }^{[3]}$ 个体常元被认为是个体的专名。任何单称命题都是一个命题函项的代入例,是用个体常元代人该命题函项中的个体变元所产生的结果。一般说来,一个命题函项有真代人例和假代入例。到目前为止所讨论的命题函项——Hx、Mx、Fx、Bx 和 $W x$ ——都是这种类型。为了把它们与后面几节将介绍的更复杂的命题函项区分开,我们把这些命题函项叫\textbf{简单谓述}。因此,一个简单谓述是一个有一些真代人例和假代人例的命题函项,并且每个代人例都是一个单称肯定命题。 

\begin{center}
\fbox{\parbox{0.95\textwidth}{
\textbf{本节要点}
\begin{itemize}
\item \textbf{单称命题的本质}:
  \begin{itemize}
  \item 断言特定个体具有某种属性的命题
  \item 由主项(个体)和谓项(属性)构成
  \item 有效性取决于命题的内在逻辑结构
  \end{itemize}
\item \textbf{符号化系统}:
  \begin{itemize}
  \item 个体常元:小写字母a到w,指称特定个体
  \item 属性符号:大写字母,表示个体可能具有的属性
  \item 单称命题形式:属性符号写在个体符号左边(如Hs)
  \end{itemize}
\item \textbf{命题函项}:
  \begin{itemize}
  \item 含有个体变元的表达式(如Hx)
  \item 代入个体常元后变成真或假的陈述
  \item 简单谓述是最基本的命题函项形式
  \end{itemize}
\end{itemize}
}}
\end{center} 