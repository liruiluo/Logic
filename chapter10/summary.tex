\section{第10章概要}
本章介绍了谓词逻辑的基本概念和方法,扩展了命题逻辑的框架,使我们能够分析命题的内部结构。

10.1节讨论了命题函项的概念,说明了如何用符号表示单称命题以及简单谓述的特点。我们学习了个体常元和个体变元的区别,以及如何通过代入形成具体命题。

10.2节介绍了量词的概念和使用。全称量词(∀)和存在量词(∃)允许我们表达关于全部个体和部分个体的陈述,使逻辑分析更加精确和灵活。

10.3节探讨了直言命题的逻辑形式,解释了如何将传统A、E、I、O四种命题形式翻译成谓词逻辑的符号表达式。我们看到了谓词逻辑如何精确捕捉自然语言中各种表达方式的共同逻辑结构。

10.4节讲解了谓词逻辑的推理规则,包括全称实例化、全称推广、存在实例化和存在推广等规则。这些规则构成了谓词逻辑推理的基础。

10.5节讨论了谓词逻辑的语义学,介绍了模型论的基本概念,探讨了如何通过模型来判断公式的有效性和可满足性。

10.6节分析了谓词逻辑对传统三段论的重新解释,展示了如何用现代逻辑工具分析和验证传统逻辑中的推理形式。

通过本章的学习,我们掌握了一套更强大的逻辑工具,能够处理命题逻辑无法表达的复杂关系和推理,为理解和分析更广泛的论证提供了基础。

\printbibliography[heading=subbibliography,title={第10章参考文献}] 