\chaptersummary{
本章介绍了\logicterm{谓词逻辑}的基本概念和方法,扩展了\logicterm{命题逻辑}的框架,使我们能够分析命题的内部结构。

10.1节讨论了\logicterm{命题函项}的概念,说明了如何用符号表示\logicterm{单称命题}以及\logicterm{简单谓述}的特点。我们学习了\logicterm{个体常元}和\logicterm{个体变元}的区别,以及如何通过代入形成具体命题。

10.2节介绍了\logicterm{量词}的概念和使用。\logicterm{全称量词}(∀)和\logicterm{存在量词}(∃)允许我们表达关于全部\logicterm{个体}和部分\logicterm{个体}的陈述,使逻辑分析更加\logicemph{精确}和灵活。

10.3节探讨了\logicterm{直言命题}的逻辑形式,解释了如何将传统A、E、I、O四种命题形式\logicterm{翻译}成\logicterm{谓词逻辑}的符号表达式。我们看到了\logicterm{谓词逻辑}如何\logicemph{精确}捕捉自然语言中各种表达方式的共同逻辑结构。

10.4节讲解了\logicterm{谓词逻辑}的\logicterm{推理规则},包括\logicterm{全称实例化}、\logicterm{全称推广}、\logicterm{存在实例化}和\logicterm{存在推广}等规则。这些规则构成了\logicterm{谓词逻辑}推理的基础。

10.5节讨论了\logicterm{谓词逻辑}的语义学,介绍了\logicterm{模型论}的基本概念,探讨了如何通过模型来判断公式的\logicemph{有效性}和\logicterm{可满足性}。

10.6节分析了\logicterm{谓词逻辑}对传统\logicterm{三段论}的重新解释,展示了如何用现代逻辑工具分析和验证传统逻辑中的推理形式。

通过本章的学习,我们掌握了一套更强大的逻辑工具,能够处理\logicterm{命题逻辑}无法表达的复杂关系和推理,为理解和分析更广泛的论证提供了基础。
}

% 参考文献将在主文档末尾统一显示