\section{单称命题}

\begin{logicbox}[title=引言]
本节引入了\logicterm{单称命题}的概念及其符号化方法。通过分析命题的内在逻辑结构,我们将学习如何区分\logicterm{个体}与\logicterm{属性},以及如何使用\logicterm{个体常元}和\logicterm{谓词符号}来表示各种类型的单称命题,从而为分析更复杂的论证奠定基础。
\end{logicbox}

前两章的逻辑技术使得我们可以对某种类型的\logicemph{有效}论证和\logicwarn{无效}论证进行区分。该类型的论证可以粗略地刻画为:其\logicemph{有效性}仅取决于简单陈述通过\logicterm{真值函项}结合成复合陈述的方式。然而,还有另外一些类型的论证,前两章的\logicemph{有效性}标准对它们不够用。一个明显\logicemph{有效的}不同类型论证的例子是:

\begin{examplebox}[title=经典三段论实例]
所有人都是有死的。

苏格拉底是人。

因此,苏格拉底是有死的。
\end{examplebox}

如果把前面介绍的评估方法运用到这个论证上,我们可以将之符号化为:
A
H
$\therefore M$

在这种符号式中,该论证显然\logicwarn{无效}。因此,到目前为止所介绍的符号逻辑技术不能直接运用到这种新型论证上。该论证的\logicemph{有效性}并不取决于简单陈述的复合方式,因为在该论证中没有出现任何复合陈述。毋宁说它的\logicemph{有效性}取决于所涉及非复合陈述的内在逻辑结构。要阐明检验这种新型论证\logicemph{有效性}的方法,就必须依据它们的内在逻辑结构,设计出一些描述和符号化非复合陈述的技术。

\subsection{单称命题的结构}

\begin{theorembox}[title=单称命题的定义]
一种最简单的非复合陈述的例示是前述论证中的第二个前提:"苏格拉底是人。"这种类型的陈述传统上叫做\logicterm{单称命题}。一个(肯定的)\logicterm{单称命题}断言的是,一个特定\logicterm{个体}具有某种特定\logicterm{属性}。在上述例子中,日常语法和传统逻辑都一致地把"苏格拉底"划为主项,把"人"划为谓项。主项指称某特定\logicterm{个体},谓项指谓该\logicterm{个体}据称所具有的某种\logicterm{属性}。
\end{theorembox}

显然,同一主项可以在不同的\logicterm{单称命题}中出现。因此,在下述每个命题中,我们都以词项"苏格拉底"做主项:"苏格拉底是有死的","苏格拉底是胖的","苏格拉底是聪明的"和"苏格拉底是漂亮的"。当然,有些是\logicemph{真的}(第一和第三个),有些是\logicwarn{假的}(第二和第四个)。\cite{quine1953} 同一谓项显然也可以出现在不同的\logicterm{单称命题}中。因此在下述每个命题中,我们都以词项"人"做谓项:"亚里士多德是人","巴西是人","芝加哥是人"和"奥基夫是人"。当然,有些是\logicemph{真的}(第一和第四个),有些是\logicwarn{假的}(第二和第三个)。

从前所述,我们应该清楚语词"\logicterm{个体}"不仅用来指人,还可以指事物,譬如,国家、城市,实际上可以指谓像是人或有死的这样能被有意义地断言为其\logicterm{属性}的任何事物。前面所举的例子中,有些谓项是形容词。从日常语法的观点看,形容词与名词的区分是相当重要的。但在本章中这种区别并不重要,我们并不区分"苏格拉底是有死的"和"苏格拉底是有死者",或"苏格拉底是聪明的"和"苏格拉底是一个聪明的人"。一个谓项可以是一个形容词或者是一个名词,甚或是一个动词。如在"亚里士多德写作"中,它有时可以被表述为"亚里士多德是一个写作者"。

\subsection{单称命题的符号化}

\begin{theorembox}[title=符号化系统]
假定我们能区分开具有某\logicterm{属性}的\logicterm{个体}和它们所具有的\logicterm{属性},我们引进并使用两种不同的符号来指称它们。在随后的讨论中,我们将用从 $a$ 到 $w$的小写字母来指谓\logicterm{个体}。这些符号是\logicterm{个体常元}。在它们出现的任何特定上下文,每个在该整个上下文中都指称一个特定的\logicterm{个体}。用它(他,或她)的名称的第一个字母指称一个\logicterm{个体},通常是很方便的。因此在当前的上下文中,我们应分别用字母 $s 、 a 、 b 、 c 、 o$ 指称苏格拉底、亚里士多德、巴西、芝加哥和奥基夫。我们将用大写字母来符号化\logicterm{属性},在此使用同样的指导原则是很便利的。因此,我们用字母 $H 、 M 、 F 、 W 、 B$ 分别符号化\logicterm{属性}是人、有死的、胖的、聪明的、漂亮的。
\end{theorembox}

有两组符号,一组是\logicterm{个体}的符号,另一组是\logicterm{个体属性}的符号。我们采取这样一个约定:把\logicterm{属性}符号直接写在\logicterm{个体}符号的左边,表征被命名的\logicterm{个体}具有规定的\logicterm{属性}这样一个\logicterm{单称命题}。于是,\logicterm{单称命题}"苏格拉底是人"可以符号化为 $H s$ 。上面提到的涉及谓项"人"的其他一些\logicterm{单称命题},分别可以符号化为 $H a 、 H b 、 H c$ 和 $H o$ 。我们注意到它们都有某种共同的模式,即它们不是被符号化为 $H$ 自身,而是 $H$ —。在此,"一"表示在谓项符号的右边有另一个符号即\logicterm{个体}符号出现。习惯上用字母 $x$(这是可以的,因为我们只用从 $a$ 到 $w$ 的字母做\logicterm{个体变元})而不是用破折号 ("——")作替代标示。我们用 $H x$[有时写成 $H(x)$ ]来符号化所有以 "是人"作为\logicterm{个体属性}的\logicterm{单称命题}的共同模式。被称做\logicterm{个体变元}的字母 $\boldsymbol{x}$只是一个位置标示,用来指示从 $a$ 到 $w$ 的各个字母——\logicterm{个体常元}——可以填入以便产生\logicterm{单称命题}的位置。

\subsection{命题函项}

\begin{theorembox}[title=命题函项的定义]
各种\logicterm{单称命题} $H a 、 H b 、 H c$ 和 $H d$ 是或\logicemph{真}或\logicwarn{假}的;但由于 $H x$ 根本不是陈述或命题,它既不\logicemph{真}也不\logicwarn{假}。表达式 $H x$ 是一个\logicterm{命题函项},它可以被定义成这样一个表达式:(1)含有\logicterm{个体变元};(2)当一个\logicterm{个体常元}代入\logicterm{个体变元}时,它就变成一个陈述。\cite{reichenbach1947} \logicterm{个体常元}被认为是\logicterm{个体}的专名。任何\logicterm{单称命题}都是一个\logicterm{命题函项}的代入例,是用\logicterm{个体常元}代人该\logicterm{命题函项}中的\logicterm{个体变元}所产生的结果。一般说来,一个\logicterm{命题函项}有\logicemph{真}代人例和\logicwarn{假}代入例。到目前为止所讨论的\logicterm{命题函项}——Hx、Mx、Fx、Bx 和 $W x$ ——都是这种类型。为了把它们与后面几节将介绍的更复杂的\logicterm{命题函项}区分开,我们把这些\logicterm{命题函项}叫\logicterm{简单谓述}。因此,一个\logicterm{简单谓述}是一个有一些\logicemph{真}代人例和\logicwarn{假}代人例的\logicterm{命题函项},并且每个代人例都是一个\logicterm{单称肯定命题}。
\end{theorembox}

\begin{center}
\fbox{\parbox{0.95\textwidth}{
\textbf{本节要点}
\begin{itemize}
\item \logicterm{单称命题}的本质:
  \begin{itemize}
  \item 断言特定\logicterm{个体}具有某种\logicterm{属性}的命题
  \item 由主项(\logicterm{个体})和谓项(\logicterm{属性})构成
  \item \logicemph{有效性}取决于命题的内在逻辑结构
  \end{itemize}
\item 符号化系统:
  \begin{itemize}
  \item \logicterm{个体常元}:小写字母a到w,指称特定\logicterm{个体}
  \item \logicterm{属性}符号:大写字母,表示\logicterm{个体}可能具有的\logicterm{属性}
  \item \logicterm{单称命题}形式:\logicterm{属性}符号写在\logicterm{个体}符号左边(如Hs)
  \end{itemize}
\item \logicterm{命题函项}:
  \begin{itemize}
  \item 含有\logicterm{个体变元}的表达式(如Hx)
  \item 代入\logicterm{个体常元}后变成\logicemph{真}或\logicwarn{假}的陈述
  \item \logicterm{简单谓述}是最基本的\logicterm{命题函项}形式
  \end{itemize}
\end{itemize}
}}
\end{center}