\section{有效性证明}

\begin{quotation}
本节讨论如何证明那些有效性取决于非复合陈述内在结构的论证。我们将介绍四个新的推论规则:全称列举、全称概括、存在列举和存在概括,通过这些规则,我们能够构造出涉及量化命题的形式证明,从而扩展我们的逻辑分析工具,使之能够处理更复杂的论证形式。
\end{quotation}

\subsection{扩充推论规则}

某些论证的有效性取决于在其中出现的非复合陈述的内在结构,为了构造它们有效性的形式证明,我们必须进一步扩充推论规则表。只需增加四个规则,我们将在涉及必须使用它们的那些论证时逐次引人。

考虑本章所引的第一个论证:"所有人都是有死的。苏格拉底是人。因此,苏格拉底是有死的。"可以符号化为:

$$
(x)(H x \supset M x)
$$

Hs\\
$\therefore M s$\\
第一个前提断定了命题函项 $H x \supset M x$ 的全称量化式。由于一个命题函项的全称量化式为真,当且仅当它的所有代人例都为真,从第一个前提可以推出命题函项 $H x \supset M x$ 的任何一个我们需要的代人例。此处即可以推出代人例 $H s \supset M s$ 。从它和第二个前提 $H s$ ,根据肯定前件式,可以直接得出结论 $M s$ 。

如果给我们的推论规则表加上这样一个原则,即一个命题函项的任一代入例都可以有效地从其全称量化式推得,那么,依照扩充了的基本有效论证形式表,我们可以给出该论证之有效性的形式证明。这种新的推论规则就是\textbf{全称列举原则},简写为"UI"。用希腊字母 $n u(v)$ 表示任一个体符号,我们可以把该新规则表述为:

$$
\begin{aligned}
& \mathrm{UI}:(x)(\phi x) \\
& \quad \therefore \phi v \quad(v \text { 是任一个体符号 })
\end{aligned}
$$

其有效性的形式证明现在可以写成:

\begin{center}
\begin{tabular}{ll}
1.$(x)(H x \supset M x)$ &  \\
2.$H s$ &  \\
 & $\therefore M s$ \\
3.$H s \supset M s$ & $1, \mathrm{UI}$ \\
4.$M s$ & $3,2, \mathrm{M} . \mathrm{P}$. \\
\end{tabular}
\end{center}

\subsection{全称概括原则}

增加 UI 大大地强化了我们的证明工具,但我们还需要更多的规则。需要另一些支配量化的规则,这种需要是和这样的论证相联系的,如"所有人都是有死的。所有希腊人都是人。因此所有希腊人都是有死的。"这个论证的符号翻译是:

$$
\begin{aligned}
& (x)(H x \supset M x) \\
& (x)(G x \supset H x) \\
& \therefore(x)(G x \supset M x)
\end{aligned}
$$

其中,前提和结论都是普遍命题而不是单称命题,是命题函项的全称量化式而不是其代人例。根据 UI,我们可以有效地从这两个前提推出下述条件陈述对子:

$$
\begin{array}{lllll}
G a \supset H a & G b \supset H b & G c \supset H c & G d \supset H d & \\
H a \supset M a & H b \supset M b & H c \supset M c & H d \supset M d & \ldots \ldots
\end{array}
$$

通过连续使用假言三段论规则,我们可以有效地推出结论:

$$
G a \supset M a, \quad G b \supset M b, \quad G c \supset M c, \quad G d \supset M d \quad \ldots \ldots
$$

如果 $a, b, c, d \cdots \cdots$ 是所有存在的个体,那么,我们从前提的真就可以有效地推出命题函项 $G x \supset M x$ 的所有代人例的真。由于一个命题函项的全称量化式为真,当且仅当,它的所有代人例都为真,我们可以继续推出 ( $x$ )( $G x \supset M x$ )为真,而它就是该论证的结论。

前面一段可以看做构成了上述论证有效性的一个非形式的证明,在证明中运用了假言三段论规则和支配量化的两个规则。它描述了一个长度不确定的陈述序列:前提中两个被全称量化的命题函项的所有代人例的序列,以及其全称量化式是结论的那个命题函项的所有代人例的序列。一个形式证明不能包含这样的不确定的甚或无限长的陈述序列。因此,必须寻求某种方法,它能以某种有限的、确定的方式来表达这些长度不确定的序列。

基础数学的一个一般技巧为做到这一点提供了提示。一个试图证明所有三角形都具有某种属性的几何学者,可以从"令 $A B C$ 是一个任意选取的三角形"出发。然后对三角形 $A B C$ 进行推理,确立它具有被探究的那种属性,由此可得出结论,所有三角形具有该属性。是什么东西能为他的最后结论进行辩护呢?承认这个特定的三角形 $A B C$ 有该属性,为什么可以得出所有的三角形都有这种属性?答案很容易见得:如果除了假定它是三角形外,我们对三角形 $A B C$ 没作任何其他假定,那么,符号"$A B C$"可以被看做是指称你所挑选的任何三角形。几何学者的论证确立了任一三角形具有所探究的属性,而如果任一三角形都具有某属性,那么所有的三角形都具有该属性。我们现在也可引进一个符号,它类似于几何学者所谈论的"一任意选取的三角形 $A B C$"。这使我们可以谈论某命题函项的任一代人例,而不用去罗列其不确定的或无限数量的代人例。

我们将用(迄今还没用过)小写字母 $y$ 来指称一任意选取的个体,以一种类似于几何学者使用字母 $A B C$ 的方式来使用它。由于从一个命题函项的全称量化式可以推出它的任一代人例,故亦可推出以 $y$ 替换 $x$ 所得到的那个代人例。在此,$y$ 指称"一任意选取的个体"。这样,我们可以着手进行上述论证有效性的形式证明:

1.$(x)(H x \supset M x)$\\
2.$(x)(G x \supset H x)$

$$
\therefore(x)(G x \supset M x)
$$

\begin{center}
\begin{tabular}{ll}
3.$H y \supset M y$ & 1, UI \\
4.$G y \supset H y$ & 2, UI \\
5.$G y \supset M y$ & 4,3, H.S. \\
\end{tabular}
\end{center}

我们从前提演绎出了陈述 $G y \supset M y$ ,由于 $y$ 指称"一任意选取的个体",所以,该陈述实际上是断言命题函项 $G x \supset M x$ 的任一代人例为真。既然任一代人例为真,所有的代人例必定为真,因此,该命题函项的全称量化也必是真的。我们可以把这个原则加到推论规则表中,表述如下:从一个

命题函项关于任意选取的个体名称的代入例,我们可以有效地推出该命题函项的全称量化式。这个规则允许我们进行概括,也就是从一个特定的代人例进到一个概括的或全称量化的表述式,故称为\textbf{全称概括原则},并缩写为"UG"。它被表述成:

$$
\begin{aligned}
\mathrm{UG}: & \phi_{y} \\
\therefore(x)\left(\phi_{x}\right) & (y \text { 指称"一任意选取的个体") }
\end{aligned}
$$

前面的形式证明的第六行即最后一行,现在就可以写(并被证明)为:\\
6.$(x)(G x \supset M x)$\\
5,UG

我们来回顾一下前面的讨论。在几何学者的证明中,对 ABC 所作的唯一假定就是它是一个三角形,因此,被证明为对 ABC 为真的东西也就被证明为对任一三角形为真。在我们的证明中,对 $y$ 所作的唯一假定是它是一个个体词,因此,被证明为对 $y$ 为真的东西也就被证明为对任一个体为真。符号 $y$ 是一个个体符号,但它是一个很特殊的个体符号。特别是通过使用 UI,它被引人到证明中,并且只有当出现了 $y$ 时才允许使用UG。

以下是另一个有效论证,它的有效性的证明要求使用 UG 和 UI:"没有人是完美的。所有希腊人都是人。因此没有希腊人是完美的。"\cite{kneale1962} 它的有效性的形式证明是:

1.$(x)(H x \supset \sim P x)$\\
2.$(x)(G x \supset H x)$

$$
\therefore(x)(G x \supset \sim P x)
$$

3. $\mathrm{H} y \supset \sim \mathrm{P} y \quad 1$ ,UI\\
4.GyつHy 2,UI\\
5.Gyコ~Py 4,3,H.S.\\

6.$(x)(G x \supset \sim P x) \quad$ 5,UG\\
上面的证明看起来多少有点不自然,需要我们对 $(x)\left(\phi_{x}\right)$ 和 $\phi_{y}$ 做出仔细的区分。说它们尽管不同但根据 UG 和 UI 又必定可以相互推出,似乎二者之间没有实质差别。但它们之间确实有一种形式的差别。陈述 ( $x$ )( $H x \supset M x$ )是一个非复合陈述,而 $H y \supset M y$ 作为一个条件陈述,是一个复合陈述。依照原先含有 19 个规则的推论规则表,从两个非复合陈述 $(x)(G x \supset H x)$ 和 $(x)(H x \supset M x)$ 出发,我们不能作相关的推理。

但从复合陈述 $G y \supset H y$ 和 $H y \supset M y$ 出发,根据假言三段论,就可以得出所要的结论 $G y \supset M y$ 。规则 $U I$ 用来从非复合陈述得出复合陈述,我们先前的推论规则无法施于非复合陈述,但可以施于复合陈述以得出想要的结论。因此,量化规则增加了我们的逻辑工具,使得我们能够证明本质地涉及非复合(概括的)命题的论证的有效性,以及前一些章节所讨论的另一类(更简单的)论证的有效性。另一方面,尽管有这种形式的差别,( $x$ ) ( $\phi_{x}$ )和 $\phi_{y}$ 必定是逻辑等价的,否则,规则 UG 和 UI 就不是有效的。对依据推论规则表来证明论证的有效性来说,这种差别和逻辑等价都很重要。把 UG 和 UI 加到推论规则表中使之得到了很大强化。

\subsection{存在量化规则}

当我们转向涉及存在命题的论证时,推论规则表必须进一步扩充。我们可从这样一个很便捷的例子着手:"所有罪犯都是邪恶的。有些人是罪犯。因此有些人是邪恶的。"它可以符号化为:

$$
\begin{aligned}
& (x)(C x \supset V x) \\
& (\exists x)(H x \cdot C x) \\
& \therefore(\exists x)(H x \cdot V x)
\end{aligned}
$$

一个命题函项的存在量化式为真,当且仅当,它至少有一个真代人例。因此,无论 $\phi$ 指谓何种属性,( $\exists x$ )( $\phi_{x}$ )所说的就是,至少存在一个具有属性 $\phi$ 的个体。如果一个个体常元(除了特定的符号 $y$ )在早先的上下文中没有使用过,我们可以用它来指称具有属性 $\phi$ 的那个个体,或者,如果有几个具有属性 $\phi$ 的个体,用它指称其中的某一个。若知道存在这样一个个体,臂如 $a$ ,我们就知道 $\phi_{a}$ 是命题函项 $\phi_{x}$ 的一个真代人例。故我们给推论规则表加上这样一个规则:从一个命题函项的存在量化式,可以推得关于在其语境中早先没有出现过的任一个体常元(除 $y$ 之外)的代入例。这个新推论规则叫\textbf{存在列举原则},可缩写为"EI"。它可以表述成:

$$
\mathrm{EI}:(\exists x)\left(\phi_{x}\right)
$$

$\therefore \phi U \quad[U$ 是任一在语境中先前没有出现过的个体常元(除 $y$之外)〕

如果确认所添加的推理规则 EI,我们即可着手证明上述论证的有效性:

1.$(x)(C x \supset V x)$

2.$(\exists x)(H x \cdot C x)$

$$
\therefore(\exists x)(H x \cdot V x)
$$

3. $\mathrm{Ha} \cdot \mathrm{Ca}$ 2,EI\\
4. $\mathrm{Ca} \supset \mathrm{Va} \quad 1$ ,UI\\
5. $\mathrm{Ca} \cdot \mathrm{Ha}$ 3,Com.\\
6. Ca 5,Simp.\\
7.Va 4,6,M.P.\\
8. Ha 3 ,Simp.\\
9. $\mathrm{Ha \cdot Va} \quad 8,7$, Conj.\\
到目前为止,我们演绎出了 $\mathrm{Ha} \cdot \mathrm{Va}$ ,它是其存在量化式被结论所断定的那个命题函项的代人例。由于一个命题函项的存在量化式为真,当且仅当,它至少有一个为真的代人例,我们为推论规则表再增加这样一个规则:从一个命题函项的任一为真的代入例,我们可以有效地推出该命题函项的存在量化式。这第四个也是最后一个推论规则叫\textbf{存在概括原则},缩写为"EG",它可以表述为:

$$
\begin{gathered}
\text { EG: } \phi_{v} \quad(v \text { 是任一个体符号) } \\
\therefore(\exists x)\left(\phi_{x}\right)
\end{gathered}
$$

前面开始的那个证明的第十行也即最后一行,现在可以写(并且被证明)为:

$$
\text { 10. }(\exists x)(H x \cdot V x) \quad 9, \mathrm{EG}
$$

\subsection{使用限制}

对 EI 的使用必须施加必要的限制,这一点可以通过考察如下明显无效的论证看出来:"有些短伆鳄被关在笼子里。有些鸟被关在笼子里。因此有些短吻鳄是鸟。"如果我们不对 EI 施加这样一种限制——根据 EI 从一个命题函项的存在量化式推出的代人例,只能含有一个在语境中早先没出现过的个体符号(除 $y$ 之外),那么,我们就可以构造出这个无效论证的有效性"证明"。这样一个错误的"证明"可以如下进行:

1.$(\exists x)(A x \cdot C x)$\\
2.$(\exists x)(B x \cdot C x)$

$$
\therefore(\exists x)(A x \cdot B x)
$$

4. $\mathrm{Ba} \cdot \mathrm{Ca}$\\
5.$A a$\\
6.$B a$\\
7.$A a \cdot B a$\\
8.$(\exists x)(A x \cdot B x)$

2,EI(错!)\\
3,Simp.\\
4,Simp.\\
5,6,Conj.\\
7,EG

这个"证明"的错误出现在第 4 行。我们从第二个前提 $(\exists x)(B x \cdot C x)$可知,至少存在这样一个事物,它既是鸟又被关在笼子里。如果我们在第 4 行给它自由地指派一个名称 $a$ ,我们当然就可以断言 $B a \cdot C a$ 。但我们绝不能自由地指派这样一个"$a$",因为它作为一只关在笼子里的短吻鳄的名字,已经先在第 3 行中出现了。为避免这种错误,我们使用 EI 时必须服从这种必要的限制。由前面的讨论可明显见得:在任何要使用 EI 和 UI 的证明中,应该总是先使用 EI 。

对更复杂的论证模式来说,特别是那些涉及关系的论证,我们还必须对四个量化规则施加某些附加限制。但就目前这种类型的论证即传统上叫做直言三段论的论证来说,目前的限制已足以避免出错。

\subsection{四个附加推论规则总结}

下述四个规则使得我们可以把非复合的、概括的命题转化为与其等值的复合命题,第 9 章所列的那 19 个推论规则适用于这些复合命题。它们还使我们可以把复合命题转化为等值的非复合命题。因此,这四个附加规则使得构造某些论证的有效性的形式证明成为可能,这些论证的有效性取决于它们所包含的一些非复合陈述的内在结构。这四个附加规则如下:

\paragraph{1.全称列举}
UI:$(x)(\phi x), \therefore \phi \cup$(在此,$v$ 是任一个体符号)\\
这个规则大体上说的是:一个命题函项的任何代入例都可以从它的全称量化式推出。

\paragraph{2.全称概括}
UG:$\phi_{y}, \therefore(x)\left(\phi_{x}\right)$(在此,$y$ 指称"一任意选取的个体")\\
这个规则大体上说的是:从一个命题函项关于一任意选取的个体名称的代入例,我们可以有效地推出该命题函项的全称量化式。

\paragraph{3.存在列举}
EI:$(\exists \mathrm{x})\left(\phi_{\mathrm{x}}\right), \therefore \phi_{\mathrm{u}}$[在此,$v$ 是任一在上下文中先前没有出现

过的个体常元(除了 y)〕\\
这个规则大体上说的是:从一个命题函项的存在量化式,我们可以推出,它关于早先上下文的任何地方都没出现的任一个体常元(除了 $y) ~$ 的代入例为真。

\paragraph{4.存在概括}
EG:$\phi_{v}, \therefore(\exists x)\left(\phi_{x}\right)$(在此,$v$ 是任一个体符号)\\
这个规则大体上说的是:从一个命题函项的任一为真的代入例,我们可以有效地推出该命题函项的存在量化式。 

\begin{center}
\fbox{\parbox{0.95\textwidth}{
\textbf{本节要点}
\begin{itemize}
\item \textbf{四个量化推论规则}:
  \begin{itemize}
  \item 全称列举(UI):从(x)(φx)推出φv
  \item 全称概括(UG):从φy推出(x)(φx)
  \item 存在列举(EI):从(∃x)(φx)推出φv
  \item 存在概括(EG):从φv推出(∃x)(φx)
  \end{itemize}
\item \textbf{规则使用的重要限制}:
  \begin{itemize}
  \item UG规则中,y必须是"任意选取的个体"
  \item EI规则中,v必须是先前未使用过的个体符号
  \item 在证明中应先使用EI再使用UI
  \item 这些限制避免了错误的推论
  \end{itemize}
\item \textbf{规则的功能和意义}:
  \begin{itemize}
  \item 使我们能够在非复合陈述和复合陈述间转换
  \item 使已有的19个推论规则能够应用于量化表达式
  \item 扩展了我们处理更复杂论证形式的能力
  \item 使我们能够形式化地证明直言三段论的有效性
  \end{itemize}
\item \textbf{规则的应用}:
  \begin{itemize}
  \item 可以处理类似"所有S是M,所有P是S,因此所有P是M"的传统三段论
  \item 可以证明包含存在命题的论证的有效性
  \item 为分析非复合陈述的内在结构提供了必要工具
  \end{itemize}
\end{itemize}
}}
\end{center} 