\section{多功能话语}

\begin{logicbox}[title=引言]
\textit{在实际应用中,语言常常不仅仅具有单一功能,而是多种功能的混合体,理解这种多功能性对于准确分析语言的意图和效果至关重要。}
\end{logicbox}

\begin{theorembox}[title=理论与现实的差距]
前面所给出的\logicterm{信息性}、\logicterm{表达性}和\logicterm{指令性}话语的例子,如同\logicemph{纯正的化学标本}——在实验室中是理想的,但在现实中很少以纯粹形式存在。

\logicwarn{重要认识}:
\begin{itemize}
  \item 三重划分具有启发性和价值,但\logicwarn{不能机械运用}
  \item 几乎任何正常交流都可能表现出语言的多种用法
  \item 现实中的语言使用通常是\logicterm{混合性的}
\end{itemize}
\end{theorembox}

\begin{examplebox}[title=诗歌的多功能性]
华兹华斯(Wordsworth)的诗句展示了表达性话语的复杂性:

\begin{displayquote}
我们身边的世界丰富精彩:迟早,无论是得到和失去,我们都要浪费掉自己的精力:
\end{displayquote}

\logicemph{功能分析}:
\begin{itemize}
  \item \logicterm{主要功能}:表达性(抒发情感和感受)
  \item \logicterm{次要功能}:信息性(包含关于世界的观察)
  \item \logicterm{潜在功能}:指令性(可能引导读者反思生活方式)
\end{itemize}
\end{examplebox}

\subsection{语言功能的融合}

\begin{examplebox}[title=不同文体的功能融合]
\logicemph{宗教布道}:
\begin{itemize}
  \item \logicterm{主要功能}:指令性(希冀带来行动:抛弃罪恶,乐善好施)
  \item \logicterm{表达性功能}:表达并激发宗教情感
  \item \logicterm{信息性功能}:传递福音的好消息
\end{itemize}

\logicemph{科学论文}:
\begin{itemize}
  \item \logicterm{主要功能}:信息性(传递科学知识和发现)
  \item \logicterm{表达性功能}:表达作者的理性激情
  \item \logicterm{指令性功能}:含蓄地请读者独立验证结论
\end{itemize}

\logicwarn{普遍规律}:语言的大多数日常用法都是\logicterm{混合的}。
\end{examplebox}

\begin{theorembox}[title=多功能用法的必要性]
语言的\logicterm{多功能用法}并非说话者的混淆,而是\logicemph{成功交流的必要条件}。

\logicemph{权威关系的局限性}:
\begin{itemize}
  \item 只有在明确的权威关系中(父母-子女、雇主-雇员),才能直接发布命令
  \item \logicwarn{赤裸裸的命令}往往引起反感和敌对
  \item 单纯的命令经常\logicwarn{自生自灭}
\end{itemize}

\logicemph{间接策略的重要性}:
\begin{itemize}
  \item 必须使用间接方式来实现目标
  \item 需要采用委婉的方法而非直截了当的命令
  \item 功能的巧妙结合是有效交流的关键
\end{itemize}
\end{theorembox}

\subsection{行动的复杂动机}

\begin{theorembox}[title=行动的心理学基础]
\logicemph{研究领域的分工}:虽然心理学家比逻辑学家更适合研究动机,但理解行动的基本机制对语言分析至关重要。

\logicemph{行动的双重条件}:
\begin{itemize}
  \item \logicterm{欲望}(desire):行动者的意愿和动机
  \item \logicterm{信念}(belief):行动者对现实的认知
\end{itemize}

\logicwarn{缺一不可}:两个条件必须同时满足才能产生行动。
\end{theorembox}

\begin{examplebox}[title=欲望与信念的相互作用]
\logicemph{饮食行为的分析}:
\begin{itemize}
  \item \logicterm{情况1}:饥饿的人(有欲望)+ 不相信面前是食物(缺乏信念)= 不会进食
  \item \logicterm{情况2}:不想吃的人(缺乏欲望)+ 确信面前是食物(有信念)= 不会进食
  \item \logicterm{情况3}:饥饿的人(有欲望)+ 相信面前是食物(有信念)= 会进食
\end{itemize}
\end{examplebox}

\begin{theorembox}[title=语言功能与行动动机的对应]
\logicemph{影响行动的两种策略}:
\begin{itemize}
  \item \logicterm{态度策略}:通过激起适当态度来引起行动(对应表达性功能)
  \item \logicterm{信息策略}:通过提供信息来影响信念(对应信息性功能)
\end{itemize}

\logicemph{概念联系}:
\begin{itemize}
  \item \logicterm{欲望}是"态度"或"情感"的特殊类型
  \item \logicterm{信念}通常受所接收信息的影响
\end{itemize}
\end{theorembox}

\begin{examplebox}[title=慈善捐助的说服策略]
假设目标是促使听众向特定慈善组织捐助,根据听众的不同状况需要采用不同策略:

\logicemph{策略一:信息导向}
\begin{itemize}
  \item \logicterm{前提条件}:听众已有助人为乐的态度
  \item \logicterm{策略}:提供慈善机构良好工作的信息
  \item \logicterm{功能分析}:表面上是信息性的,实际目的是指令性的
  \item \logicterm{优势}:避免了毫无掩饰的命令或不客气的要求
\end{itemize}

\logicemph{策略二:情感导向}
\begin{itemize}
  \item \logicterm{前提条件}:听众已相信慈善机构信誉良好
  \item \logicterm{策略}:激发乐善好施的情感
  \item \logicterm{功能分析}:实现"动人诉求"(moving appeal)
  \item \logicterm{结果}:语言具有表达性和指令性的混合用法
\end{itemize}

\logicemph{策略三:综合导向}
\begin{itemize}
  \item \logicterm{前提条件}:听众既缺乏慈善态度,也不了解机构信誉
  \item \logicterm{策略}:同时使用三种语言功能
  \item \logicterm{必要性}:这不是偶然为之,而是成功交流的\logicwarn{基本手段}
\end{itemize}
\end{examplebox}

\subsection{语言的特殊用法}

\begin{theorembox}[title=三重划分的局限性]
虽然\logicterm{信息性}、\logicterm{表达性}和\logicterm{指令性}用法构成了语言的三种基本功能,但语言在某些特殊语境中还具有一些\logicwarn{不能完全归属于三重划分}的特殊用法。

\logicemph{理论的开放性}:语言的复杂性超越了任何单一的分类系统。
\end{theorembox}

\begin{theorembox}[title=礼仪性用法]
\logicterm{礼仪性用法}是一种普遍存在的特殊语言功能:

\logicemph{基本特征}:
\begin{itemize}
  \item 通常是表达性和指令性话语的混合
  \item 主要服务于使人际互动变得融洽
  \item 具有社会润滑剂的作用
\end{itemize}

\logicemph{常见例子}:
\begin{itemize}
  \item \logicterm{社交场合}:问候、赴宴邀请、雇用告知
  \item \logicterm{宗教场合}:庄重的仪式语言
\end{itemize}
\end{theorembox}

\begin{examplebox}[title=结婚仪式的语言分析]
结婚仪式的语言展现了礼仪性用法的复杂性:

\logicemph{多重功能}:
\begin{itemize}
  \item \logicterm{表达性功能}:突出场合的庄重性,营造神圣氛围
  \item \logicterm{指令性功能}:使新郎新娘理解严肃的结婚誓言,引起新的角色行为
  \item \logicterm{礼仪性功能}:标志人生重要转折,确认社会关系变化
\end{itemize}
\end{examplebox}

\begin{theorembox}[title=践行性用法]
与礼仪性用法相近,语言还有另一种特殊用法:\logicterm{践行性}(performative)用法。

\logicemph{核心特征}:在某些语境中,\logicwarn{说出某些话实际上就是实施一种行动}。

\logicemph{与其他功能的区别}:
\begin{itemize}
  \item 不仅仅是表明态度或预告行为
  \item 不仅仅是描述说话者在做什么
  \item 语言本身就构成了行动
\end{itemize}
\end{theorembox}

\begin{examplebox}[title=践行性用法的典型例子]
\logicemph{承诺行为}:
\begin{itemize}
  \item \logicterm{情境}:有人邀请你参加会议
  \item \logicterm{回应}:"好,我答应你"
  \item \logicterm{分析}:这句话本身就是许诺行为,而不仅仅是描述许诺
\end{itemize}

\logicemph{宣告行为}:
\begin{itemize}
  \item \logicterm{情境}:婚礼结束时
  \item \logicterm{宣告}:"我宣布你们是夫妻"
  \item \logicterm{分析}:这句话本身就是使两人成为夫妻的行为
\end{itemize}
\end{examplebox}

\begin{theorembox}[title=践行性动词的特点]
\logicemph{定义}:践行性动词是一种特殊类别,它们代表的行动通常通过第一人称使用该动词而完成。

\logicemph{更多例子}:
\begin{itemize}
  \item "我祝贺你……"(祝贺行为)
  \item "我向你道歉……"(道歉行为)
  \item "我建议……"(建议行为)
  \item "我将这艘船命名为……"(命名行为)
  \item "我接受你的建议"(接受行为)
\end{itemize}

\logicwarn{条件限制}:这些行为的成功实施需要适当的语境和条件。
\end{theorembox}

\begin{theorembox}[title=自然语言的复杂性]
语词和语句的这些特殊用法展示了\logicemph{自然语言的丰富性}:

\logicwarn{分类系统的局限}:语言的诸多复杂功能\logicwarn{难以归纳为任何单一的分类系统}。

\logicemph{理论意义}:这提醒我们在分析语言时需要保持开放和灵活的态度。
\end{theorembox}

\chaptersummary{
现实中的语言使用远比理论分类复杂,呈现出丰富的多功能特征。

\logicemph{多功能性的普遍性}:
\begin{itemize}
  \item 语言在实际使用中常常兼具信息性、表达性和指令性的多种功能
  \item 纯粹的单一功能语言如同"化学标本",在现实中很少存在
  \item 成功的交流通常要求这些功能的巧妙结合
\end{itemize}

\logicemph{行动的心理学基础}:
\begin{itemize}
  \item 行动需要欲望和信念的双重条件
  \item 可以通过激发态度或提供信息来影响他人行动
  \item 不同的说服策略对应不同的语言功能组合
\end{itemize}

\logicemph{特殊用法的存在}:
\begin{itemize}
  \item 礼仪性用法:服务于人际关系的润滑和社会秩序的维护
  \item 践行性用法:语言本身就构成行动,而非仅仅描述行动
  \item 这些特殊用法超越了三重划分的范围
\end{itemize}

\logicwarn{理论启示}:自然语言的复杂功能难以归纳为任何单一的分类系统,这要求我们在语言分析中保持开放和灵活的态度。
}