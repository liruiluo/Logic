\section{语言的三种基本功能}

\begin{logicbox}[title=引言]
\textit{语言作为人类交流的工具有着多种用法,理解语言的三种基本功能对于正确分析和解读语言表达至关重要。}
\end{logicbox}

语言是一种极其精细和复杂的工具,其\logicterm{用法}(uses)的多样性常常被我们忽视。在日常交流中,我们倾向于简化理解,\logicwarn{不注意语言使用的具体语境及其不同目的},从而可能被表面的语词或话语形式所误导。

\begin{examplebox}[title=语言功能的复杂性实例]
\logicemph{表面与实际的差异}:
\begin{itemize}
  \item \logicterm{表面形式}:"你好吗?"(疑问句形式)
  \item \logicterm{实际功能}:友好问候,而非健康状况询问
  \item \logicwarn{误解后果}:详细描述健康状况的人可能被认为不懂社交礼仪
\end{itemize}
\end{examplebox}

这个例子说明,\logicemph{语词并不总是服务于它们的表面诉求}。请求、报道、问候等只是语言众多功能中比较明显的几种。

\subsection{语言功能的多样性}

\begin{theorembox}[title=贝克莱的语言功能观]
哲学家\logicterm{乔治·贝克莱}(George Berkeley)在《人类知识原理》(1710)中提出了重要观点:

\begin{displayquote}
……思想交流……并非像通常想象的那样是语言的首要的和唯一的目的。使用语言还有许多其他宗旨,诸如引起某些情感、鼓动或者抑制行动、使人专心于某些特定的安排等等。前者(交流思想)在很多情况下都基本上是从属性的;当没有前者也能实现那些宗旨时,前者甚或完全被忽略,我认为,这种情况在人们熟悉的语言用法中经常发生。
\end{displayquote}

\logicemph{核心洞察}:语言的功能远超信息传递,包括情感激发、行为引导等多重目的。
\end{theorembox}

\begin{theorembox}[title=维特根斯坦的语言游戏理论]
20世纪哲学家\logicterm{路德维希·维特根斯坦}在《哲学研究》(1953)中进一步发展了这一思想:

\logicemph{核心观点}:"我们称之为'符号'、'语词'和'语句'的东西有无数不同种类的用法。"

\logicemph{语言用法的丰富性}:
\begin{itemize}
  \item \logicterm{认知功能}:描述、报道、推测、假说检验、实验结果展示
  \item \logicterm{行为功能}:命令、询问、翻译、解题
  \item \logicterm{社交功能}:问候、开玩笑、诅咒
  \item \logicterm{文化功能}:编故事、演戏、唱歌、祈祷
  \item \logicterm{娱乐功能}:猜谜、游戏
\end{itemize}
\end{theorembox}

\begin{theorembox}[title=语言功能的三重划分]
面对语言用法的惊人多样性,学者们提出了一个实用的分类框架:

\logicemph{三种基本功能}:
\begin{itemize}
  \item \logicterm{信息性用法}(the informative):传递信息,描述世界
  \item \logicterm{表达性用法}(the expressive):表达情感,激发共鸣
  \item \logicterm{指令性用法}(the directive):引导行为,产生行动
\end{itemize}

\logicwarn{重要说明}:这种三重划分是一种简化,甚至可能过于简化,但许多逻辑学家和语言学家发现它是非常有用的分析工具。
\end{theorembox}

\subsection{信息性用法}

\begin{theorembox}[title=信息性功能的定义与特征]
\logicemph{核心定义}:语言的\logicterm{信息性功能}是指通过明确表述并肯定(或否定)命题来进行信息交流的用法。

\logicemph{主要特征}:
\begin{itemize}
  \item \logicterm{命题性质}:能够被肯定或否定
  \item \logicterm{论证功能}:可以为命题提供论证支持
  \item \logicterm{描述功能}:用来描述世界和进行推理
  \item \logicwarn{包容性}:既包括真信息也包括假信息,既包括正确论证也包括错误论证
\end{itemize}

\logicemph{适用范围}:无论所报道的事实是重要还是琐碎、是普遍还是特殊,用来描述和报道的语言都属于信息性用法。
\end{theorembox}

下面是语言信息性用法的典型例子,来自佛罗里达高等法院的报道:

\begin{displayquote}
2000年11月7日,星期二,佛罗里达州,和美国其他州一道,进行美国总统普选。11月8日,星期三,该选举分区(佛罗里达州)报道说,共和党候选人乔治•布什获得 2909135 张选票,民主党候选人小艾伯特•戈尔获得 2907351 张选票。因为投给他们的全部票数的总差(1784张),低于该选区全部投票票数的百分之一的一半,所以根据佛罗里达州法律规定进行了自动重新计票。\cite{palmbeach2000}
\end{displayquote}

\subsection{表达性用法}

\begin{theorembox}[title=表达性功能的核心特征]
正如信息性话语的典型例子来自法院或实验室报道,\logicterm{表达性用法}的最佳例子来自抒情诗歌。

\logicemph{核心目的}:表达和激发情感、感受或态度,而非传递事实信息。
\end{theorembox}

\begin{examplebox}[title=诗歌中的表达性用法]
约翰·W·伯根(John W. Burgon)面对古城帕特拉(Petra)遗迹时写下的诗句:

\begin{displayquote}
如此奇迹今我惊叹,它保留在东部的风情中——玫瑰一样红的城市——"几乎和时间一样永恒"!
\end{displayquote}

\logicemph{分析}:
\begin{itemize}
  \item \logicwarn{非信息性}:并非意图告诉我们关于世界的事实和理论
  \item \logicterm{情感表达}:表达诗人的赞赏和敬畏之情
  \item \logicterm{情感激发}:目的在于在读者心中激起类似情感
\end{itemize}
\end{examplebox}

\begin{theorembox}[title="表达"概念的界定]
\logicwarn{术语澄清}:为避免混淆,本书对"表达"一词的使用范围比日常用法更为狭窄:

\logicemph{本书用法}:
\begin{itemize}
  \item \logicterm{表达}:专指揭示或交流情感、感受和态度
  \item \logicterm{陈述/表明}:用于见解、信念或信仰的表述
\end{itemize}

\logicemph{区分意义}:这种区分有助于清晰地分离语言的信息性功能和表达性功能。
\end{theorembox}

\begin{examplebox}[title=表达性语言的多样形式]
表达性语言远不限于诗歌,在日常生活中随处可见:

\logicemph{情感表达的常见形式}:
\begin{itemize}
  \item \logicterm{悲伤表达}:"真糟糕"、"真遗憾"
  \item \logicterm{兴奋表达}:"好极了"、"太妙了"
  \item \logicterm{爱情表达}:恋人间的喃喃私语
  \item \logicterm{宗教表达}:主祷文、大卫王赞美诗第23篇
\end{itemize}

\logicemph{共同特征}:这些用法都不在于交流信息,而在于表达情感、感受或态度。
\end{examplebox}

\begin{theorembox}[title=表达性话语的真假问题]
\logicwarn{重要原理}:表达性话语既不真也不假。

\logicemph{错误评价的危害}:
\begin{itemize}
  \item 用真假标准衡量表达性话语是\logicwarn{文不对题}的
  \item 这种做法会使表达性话语的价值\logicwarn{丧失殆尽}
\end{itemize}

\logicemph{经典例子}:因为知道是巴尔波(Balboa)而非柯尔特斯(Cortés)发现太平洋,就降低对济慈《初读查普曼译荷马》的欣赏,这是\logicwarn{蹩脚读者}的表现。该诗的宗旨不是教授历史知识。
\end{theorembox}

\begin{theorembox}[title=混合用途的概念]
\logicemph{复杂情况}:某些语言表达具有多重功能:
\begin{itemize}
  \item 有些诗歌含有重要的信息性内容
  \item 伟大诗人的作品常常是优秀的"生活评判"
  \item 这类作品具有\logicterm{混合用途}(mixed usage)
\end{itemize}

\logicwarn{概念预告}:混合用途的详细讨论将在后续章节展开。
\end{theorembox}

\begin{theorembox}[title=表达的两种基本情形]
表达性语言可以根据其交流对象分为两种类型:

\logicemph{自我表达型}:
\begin{itemize}
  \item \logicterm{特征}:独自发泄,不以他人为对象
  \item \logicterm{例子}:私人写诗、孤独祷告
  \item \logicterm{功能}:表达说话者或写作者的情感
  \item \logicwarn{限制}:不是为了在别人心中引起共鸣
\end{itemize}

\logicemph{感染他人型}:
\begin{itemize}
  \item \logicterm{特征}:寻求感染他人,以他人为对象
  \item \logicterm{例子}:演讲、求爱诗歌、运动队欢呼
  \item \logicterm{功能}:既表达说话者情感,又意图在听者心中引发共鸣
\end{itemize}

\logicemph{综合性质}:表达性话语可以同时具有这两方面的功用。
\end{theorembox}

\subsection{指令性用法}

\begin{theorembox}[title=指令性功能的定义与特征]
\logicemph{核心定义}:当语言意图引起或阻止明显的行动时,它就具有\logicterm{指令性功能}。

\logicemph{主要特征}:
\begin{itemize}
  \item \logicterm{行为导向}:目标是产生或阻止具体行动
  \item \logicwarn{非信息性}:不是为了交流信息
  \item \logicwarn{非表达性}:不是为了表达或激发情感
  \item \logicterm{结果导向}:为了获得指令结果
\end{itemize}
\end{theorembox}

\begin{examplebox}[title=指令性用法的典型例子]
\logicemph{命令与请求}:
\begin{itemize}
  \item \logicterm{家庭场景}:父母告诉孩子"洗手吃饭"
  \item \logicterm{商业场景}:对售票员说"请给两张票"
  \item \logicterm{微妙差别}:命令和请求可通过语调或"请"字转换
\end{itemize}

\logicemph{问题的指令性}:当提出问题以寻求回答时,该问题通常也属于指令性话语。
\end{examplebox}

\begin{theorembox}[title=指令性话语的真假问题]
\logicwarn{重要原理}:在单纯的祈使形式中,指令性话语既不真也不假。

\logicemph{真假不适用性}:
\begin{itemize}
  \item \logicterm{例子}:"关上窗户"这样的命令既不能是真的也不能是假的
  \item \logicterm{争议焦点}:人们可能对是否应当遵从命令有不同意见
  \item \logicwarn{共识}:但对命令是否有真假不会有分歧
\end{itemize}

\logicemph{替代评价标准}:
\begin{itemize}
  \item \logicterm{合理性}:合理或不合理
  \item \logicterm{适当性}:适当或不适当
  \item \logicemph{类比关系}:这些属性与信息性话语的真假有相似之处
\end{itemize}
\end{theorembox}

\begin{examplebox}[title=指令性话语中的论证]
当指令伴随理由时,整个过程可以视为论证:

\begin{displayquote}
小心驾驶!谨记墓地中满是守法的公民,他们有走路的权利。\cite{lander1988}
\end{displayquote}

\logicemph{论证分析}:
\begin{itemize}
  \item \logicterm{命令}:小心驾驶
  \item \logicterm{理由}:墓地中满是守法的公民,他们有走路的权利
  \item \logicterm{逻辑处理}:可以将命令视为命题"你应当小心驾驶"
\end{itemize}
\end{examplebox}

\begin{theorembox}[title=祈使逻辑]
\logicemph{学术发展}:一些学者专门研究指令性语言的逻辑问题,发展出了\logicterm{祈使逻辑}(logic of imperatives)。

\logicwarn{范围限制}:祈使逻辑的详细讨论超出了本书的范围。\cite{rescher1996}
\end{theorembox}

\chaptersummary{
语言作为人类交流的复杂工具,具有远超表面形式的多样功能。通过三重划分,我们可以系统地理解语言的基本用法:

\logicemph{三种基本功能}:
\begin{itemize}
  \item \logicterm{信息性用法}:通过肯定或否定命题来交流信息,描述和解释世界,包括真假信息和正误论证
  \item \logicterm{表达性用法}:表达和激发情感、感受或态度,既不真也不假,评价标准不是真假而是情感共鸣
  \item \logicterm{指令性用法}:引起或阻止行动,如命令、请求和问题,评价标准是合理性和适当性
\end{itemize}

\logicwarn{重要认识}:
\begin{itemize}
  \item 语言的表面形式可能掩盖其真实功能
  \item 不同功能需要不同的评价标准
  \item 某些语言表达具有混合用途,同时承担多种功能
  \item 理解语言功能对于准确分析和解读语言表达至关重要
\end{itemize}
}