\section*{2.5 一致与歧见的种类}
任何事物或行动都可以选择不同的短语来描述:传达赞许或反对意见,或者中立意见;不同类型的一致(agreement)和歧见(disagree- ment)可以就任何事情展开交流。

两个人可能会在某事情是否已经实际地发生上意见相左,这种情况可称为"信念歧见"。另一方面,他们也可能都同意已经事实上发生了一件事,因而是信念一致的;但对那件事,他们仍可以具有不同的或者甚至相反的态度。你可以用语言描述那件事来表达赞许,别人却可以用语言来表达反对。这里也存在歧见,但不是信念歧见。这是对这件事的感受不同,

是态度歧见。 ${ }^{[9]}$\\
明确了这两种歧见,我们就可以区分出两个人(让我们把他们称做 A和 B)之间的四种关系,来讨论某些给定事件或者其他事实情况。

第一,他们可能达到充分一致,即他们对于事件发生的信念和对事件的态度都是一致的。

第二,他们可能对于事件的信念是一致的,但在态度上却对立,一个人认为是好事,而另一个人却认为是坏事。设想我们讨论的事件是:对于某个有争议问题,一位政治候选人的立场改变。对于的确发生了这个改变,A 和 B 可能意见一致;但是,A 认为好极了,而 B 则发现它令人担忧。如果像 A 所理解的那样,这位候选人就会被称赞为"倾听了理性的呼声";如果像 B 所理解的那样,这位候选人就将被谴责为"机会主义的反复无常"。

第三,他们可能态度一致,但对于引起态度的事实,他们却可能有信念歧见。因此,A 和 B 都可能热情地赞扬所论及的那位候选人,而他们对该候选人的实际立场却有不同理解。A 可能认为,由于"倾听了理性的呼声",那位候选人确实改变了他的立场;而 B 则可能认为,由于"坚定不移地拒斥了为奉承所左右",他根本没有改变自己的立场。乍一看,这个第三种可能性好像不合理,但经过思考之后就会被认为是平常的;我们知道,在政治选举中,同一候选人常常可能有不同的支持原因,这些原因不但不同而且有时还不相容。

第四,这两个人可能会处于一种完全对立的状态,他们不但在事实上有歧见而且对事实的态度也对立。由于相信那位候选人改变了立场,A 可能会非常热烈地称赞这种改变是"明智的重新考虑"的结果;由于认为候选人的立场保持未变, B 可能会激烈地贬斥他"顽固地拒绝承认错误"。

当我们以解决歧见为目标时,我们就必须既要关心给定情况下的事实,又要关心争论者对这些事实的不同态度。不同类型的歧见需要不同的解决方法。因此,如果我们不清楚所存在的歧见是什么类型,那么我们就不清楚去使用什么方法。信念歧见可以通过确认事实而得到最好的解决。为了明确这些事实,如果它足够重要,可以询问证人、查阅文本和检查记录等等。当事实得到了确证、解决了事实问题时,歧见就会得到解决。科学的探究方法在这里都可以用到,这将足以指导他们直面有关信念歧见的事实问题。

另一方面,如果是态度歧见而不是信念歧见,那么适于解决它的方法就有相当大的差别,难以如此径情直遂。以确证事件发生与否为目的,召唤证人、查阅文本,诸如此类,对解决这样的争论不会有什么效果,因为争论的问题并不是事实,这种歧见不是关于事实是什么而是关于怎样评价它们的对立。解决这种态度歧见的努力可能会涉及有关事实问题,但不是那种存在态度冲突的事实。或许考虑那种引出愉快或不愉快的结论的事情若不发生将会怎样,可能是有益的。动机和目的也可能具有重要性。诚然,它们都是事实问题,但如果歧见在信念上而不是在态度上,它们就都不会成为争论的主题。其他一些方法有时也可以解决态度歧见,你可以大量地使用表达性语言来尝试说服方法;在凝结团体意志和取得统一态度中,修辞艺术也可能富有成效(当然,它在解决事实问题上完全没有价值)。\\
"好"和"坏","对"和"错",诸如此类的语词,在严格的伦理用法中,往往具有非常强烈的情感色彩。无疑,当我们把某行动描述为对或把某情形描述为好时,我们就对它表达了一种赞赏态度。有些伦理学者认为,这些语词"没有"词汇意义或认知意义,而只有适合它们的情感意义;而另一些伦理学者则坚持这些语词的确具有认知意义,它们指称所讨论事物的客观特征。在这种争论中,学习逻辑的学生不必偏向某一方。但显然,不是所有的赞同或不赞同态度都蕴涵道德判断,因为还有审美价值的考虑,还有个人偏好或口味的影响。对某些事物(例如有些食品或服装款式)的否定态度,不必牵涉伦理或审美判断,但它也可以由情感色彩强烈的语言来表达。

当歧见是在态度上而不是在信念上时,最强烈的(当然也是实质的)歧见可以用朴实无华的真实陈述来表达。当双方针锋相对并以逻辑上相容的陈述来明确表述他们的分歧时,若认为双方的歧见不是"实质的"或者是"纯粹言辞的"就是错误的。他们并不是仅仅"用不同的言辞说相同的事物"。当然,他们可以用不同言辞来断言词汇意义上相同的事实,但是,他们也可以用不同言辞来表达对相同事实的矛盾态度。在这样的情况下,虽然他们的歧见不是"词汇意义上的",然而却是实质的。这不是"纯粹言辞的"歧见,因为语词既具有表达性功能也具有信息性功能。如果我们有兴趣解决歧见,我们就必须弄清楚其本性,因为适于解决一种歧见的技巧,正如我们已看到的,可能对另一种毫无用处。

有时,我们难以确定一种歧见是信念的或是态度的,或者既是信念的又是态度的;这可能取决于争论者对词汇的某种解说。冲突意见的表达方式可能会把不同态度之间的区别以及不同信念之间的区别弄得模糊,因此争论的关键核心就会难以把握。当两个人对在一个事物是否比另一个"更好"或"更重要"上意见对立时,他们都可能认为,真实情况很可能是不同信念使他们产生了分歧。但是在某些情况下,一个其表面形式是关于所谓的事实问题的差别论争,实际上是一个关于态度的实质争论,当争论的东西是事物或行为的价值时尤其如此。

在关于赢球的重要性上,一位著名体育运动作家和一位著名足球教练产生了深刻的分歧。新闻记者格兰特兰德•赖斯(Grantland Rice)写道:

\begin{displayquote}
当球星走进球场,为其声名留下光芒,这光芒不论输与赢,只记录场上飞奔的身影。
\end{displayquote}

文斯•隆巴蒂(Vince Lombardi)教练却说:

\begin{displayquote}
赢球不是别的,它就是唯一(竞赛目标)。
\end{displayquote}

很明显,这两位对赢球的态度是冲突的。你相信这种态度上的歧见的根源是信念歧见吗?

尽管有这些不可避免的困难,态度歧见与信念歧见之间的区分还是非常有用的;留心语言的不同用法有助于理解我们可能遭遇的种种歧见。当然,找出区别本身并不能解决问题或消除歧见。但是,由此可以澄清讨论的问题,揭示出它们的类型并找出冲突的所在。我们越充分地理解歧见的本性,我们就越能够更好地去解决歧见。 