\section{一致与歧见的种类}

\begin{logicbox}[title=引言]
\textit{语言交流中的分歧不仅可能来自事实认知的差异,也可能源于对同一事实的态度不同,理解这些不同类型的分歧有助于我们采取正确的方法解决分歧。}
\end{logicbox}

任何事物或行动都可以选择不同的短语来描述,传达赞许、反对或中立的意见。这种描述方式的多样性导致了不同类型的\logicterm{一致}(agreement)和\logicterm{歧见}(disagreement)。

\begin{theorembox}[title=歧见的基本分类]
人际交流中的分歧可以分为两种根本不同的类型:

\logicemph{信念歧见}(Disagreement in Belief):
\begin{itemize}
  \item \logicterm{核心}:对事实本身的认知不同
  \item \logicterm{争议焦点}:某事是否实际发生
  \item \logicterm{性质}:关于客观现实的不同判断
\end{itemize}

\logicemph{态度歧见}(Disagreement in Attitude):
\begin{itemize}
  \item \logicterm{核心}:对同一事实的评价不同
  \item \logicterm{争议焦点}:如何看待已知事实
  \item \logicterm{性质}:关于价值判断的不同立场
\end{itemize}

\logicwarn{重要区别}:态度歧见中,双方对事实本身可能完全一致,分歧在于对事实的感受和评价。$^{[9]}$
\end{theorembox}

\subsection{一致与歧见的四种关系}

基于信念和态度这两个维度,我们可以构建一个分析框架,区分出两个人(设为A和B)之间可能存在的四种关系类型:

\begin{theorembox}[title=四种关系类型的系统分析]
\logicemph{类型一:充分一致}
\begin{itemize}
  \item \logicterm{信念维度}:对事件发生的认知一致
  \item \logicterm{态度维度}:对事件的评价一致
  \item \logicterm{特征}:完全和谐,无任何分歧
\end{itemize}

\logicemph{类型二:信念一致,态度对立}
\begin{itemize}
  \item \logicterm{信念维度}:对事实的认知相同
  \item \logicterm{态度维度}:对同一事实的评价截然不同
  \item \logicterm{特征}:事实无争议,价值观有冲突
\end{itemize}

\logicemph{类型三:态度一致,信念对立}
\begin{itemize}
  \item \logicterm{信念维度}:对事实的认知不同
  \item \logicterm{态度维度}:对各自认知的事实评价相同
  \item \logicterm{特征}:价值观相同,但基于不同的事实认知
\end{itemize}

\logicemph{类型四:完全对立}
\begin{itemize}
  \item \logicterm{信念维度}:对事实的认知不同
  \item \logicterm{态度维度}:对各自认知的事实评价也不同
  \item \logicterm{特征}:全面冲突,既有事实争议又有价值冲突
\end{itemize}
\end{theorembox}

\begin{examplebox}[title=政治候选人立场变化的案例分析]
设想讨论的事件是:某位政治候选人在争议问题上的立场改变。

\logicemph{类型二示例}(信念一致,态度对立):
\begin{itemize}
  \item \logicterm{共同认知}:A和B都认为候选人确实改变了立场
  \item \logicterm{A的态度}:认为这是好事,称赞为"倾听了理性的呼声"
  \item \logicterm{B的态度}:认为这是坏事,谴责为"机会主义的反复无常"
\end{itemize}

\logicemph{类型三示例}(态度一致,信念对立):
\begin{itemize}
  \item \logicterm{共同态度}:A和B都热情赞扬该候选人
  \item \logicterm{A的认知}:认为候选人"倾听了理性的呼声"而改变立场
  \item \logicterm{B的认知}:认为候选人"坚定不移地拒斥奉承"而未改变立场
\end{itemize}

\logicwarn{现实性说明}:类型三在政治选举中很常见,同一候选人常因不同甚至不相容的原因获得支持。

\logicemph{类型四示例}(完全对立):
\begin{itemize}
  \item \logicterm{A的立场}:认为候选人改变了立场,并称赞这是"明智的重新考虑"
  \item \logicterm{B的立场}:认为候选人未改变立场,并贬斥其"顽固地拒绝承认错误"
\end{itemize}
\end{examplebox}

\subsection{解决不同类型歧见的方法}

\begin{theorembox}[title=歧见解决的基本原则]
解决歧见的关键在于\logicwarn{准确识别歧见类型},因为不同类型的歧见需要完全不同的解决策略。

\logicemph{诊断的重要性}:
\begin{itemize}
  \item 必须既关心事实层面,又关心态度层面
  \item 如果不清楚歧见类型,就无法选择正确的解决方法
  \item 错误的方法不仅无效,还可能加剧冲突
\end{itemize}
\end{theorembox}

\begin{theorembox}[title=信念歧见的解决策略]
\logicemph{核心方法}:通过确认事实来解决分歧

\logicemph{具体步骤}:
\begin{itemize}
  \item \logicterm{证据收集}:询问证人、查阅文献、检查记录
  \item \logicterm{事实核实}:运用科学探究方法
  \item \logicterm{结果}:当事实得到确证,歧见自然解决
\end{itemize}

\logicemph{适用条件}:争议的核心是"事实是什么"
\end{theorembox}

\begin{theorembox}[title=态度歧见的解决策略]
\logicemph{基本特征}:解决方法与信念歧见\logicwarn{有相当大的差别},难以直接了当

\logicemph{无效方法}:
\begin{itemize}
  \item 召唤证人、查阅文本等事实确认方法
  \item 争议的问题不是事实本身,而是如何评价事实
\end{itemize}

\logicemph{有效策略}:
\begin{itemize}
  \item \logicterm{后果分析}:考虑相关事情不发生会怎样
  \item \logicterm{动机探讨}:分析行为的动机和目的
  \item \logicterm{表达性说服}:大量使用表达性语言
  \item \logicterm{修辞艺术}:在凝聚团体意志和统一态度方面富有成效
\end{itemize}

\logicwarn{重要说明}:这些方法涉及的虽然也是事实问题,但不是存在态度冲突的那些事实。修辞艺术在解决事实问题上完全没有价值。
\end{theorembox}

\subsection{情感语言与伦理判断}

\begin{theorembox}[title=伦理语词的双重性质]
\logicemph{伦理语词的特征}:
\begin{itemize}
  \item \logicterm{典型词汇}:"好"与"坏"、"对"与"错"等
  \item \logicterm{情感色彩}:在严格伦理用法中具有非常强烈的情感色彩
  \item \logicterm{态度表达}:描述某行动为"对"或某情形为"好"时,表达赞赏态度
\end{itemize}

\logicemph{哲学争议}:
\begin{itemize}
  \item \logicterm{情感主义观点}:这些词汇"没有"认知意义,只有情感意义
  \item \logicterm{认知主义观点}:这些词汇具有认知意义,指称客观特征
  \item \logicwarn{逻辑学立场}:学习逻辑的学生不必在此争论中偏向某一方
\end{itemize}
\end{theorembox}

\begin{theorembox}[title=态度表达的多样性]
\logicwarn{重要区别}:并非所有的赞同或不赞同态度都蕴含道德判断。

\logicemph{态度的不同来源}:
\begin{itemize}
  \item \logicterm{道德判断}:基于伦理标准的评价
  \item \logicterm{审美价值}:基于美学标准的评价
  \item \logicterm{个人偏好}:基于个人口味的评价
\end{itemize}

\logicemph{例证}:对某些食品或服装款式的否定态度,不必涉及伦理或审美判断,但同样可以用情感色彩强烈的语言表达。
\end{theorembox}

\subsection{实质歧见与言辞歧见}

\begin{theorembox}[title=态度歧见的实质性]
\logicwarn{常见误解}:认为态度歧见不是"实质的"或者是"纯粹言辞的"

\logicemph{态度歧见的特征}:
\begin{itemize}
  \item 最强烈的态度歧见可以用\logicterm{朴实无华的真实陈述}来表达
  \item 双方可能以\logicterm{逻辑上相容的陈述}来表述分歧
  \item 他们并非仅仅"用不同言辞说相同的事物"
\end{itemize}

\logicemph{实质性的体现}:
\begin{itemize}
  \item 可以用不同言辞断言词汇意义上相同的事实
  \item 也可以用不同言辞表达对相同事实的\logicterm{矛盾态度}
  \item 虽然歧见不是"词汇意义上的",但却是\logicemph{实质的}
\end{itemize}

\logicwarn{理论意义}:这不是"纯粹言辞的"歧见,因为语词既具有表达性功能也具有信息性功能。
\end{theorembox}

\begin{theorembox}[title=歧见类型识别的困难]
\logicemph{实践中的复杂性}:
\begin{itemize}
  \item 有时难以确定歧见是信念的、态度的,还是两者兼有
  \item 这可能取决于争论者对词汇的特定解释
  \item 冲突意见的表达方式可能模糊不同类型的区别
\end{itemize}

\logicemph{识别的挑战}:
\begin{itemize}
  \item 争论的关键核心变得难以把握
  \item 表面的事实争议可能实际上是态度争论
  \item 当争论涉及事物或行为的\logicterm{价值}时尤其如此
\end{itemize}

\logicwarn{实践启示}:准确识别歧见类型对于选择正确的解决策略至关重要。
\end{theorembox}

\begin{examplebox}[title=体育哲学中的态度歧见]
关于赢球重要性的经典争论展示了态度歧见的复杂性:

\logicemph{格兰特兰德·赖斯}(体育作家)的观点:
\begin{displayquote}
当球星走进球场,为其声名留下光芒,这光芒不论输与赢,只记录场上飞奔的身影。
\end{displayquote}

\logicemph{文斯·隆巴蒂}(足球教练)的观点:
\begin{displayquote}
赢球不是别的,它就是唯一(竞赛目标)。
\end{displayquote}

\logicemph{分析思考}:
\begin{itemize}
  \item 两位对赢球的态度明显冲突
  \item 这种态度歧见的根源是信念歧见吗?
  \item 还是纯粹的价值观差异?
\end{itemize}

\logicwarn{启发性问题}:这个例子帮助我们思考如何区分不同类型的歧见。
\end{examplebox}

\begin{theorembox}[title=歧见分析的价值与局限]
\logicemph{理论价值}:
\begin{itemize}
  \item 态度歧见与信念歧见的区分\logicterm{非常有用}
  \item 留心语言的不同用法有助于理解各种歧见
  \item 可以澄清讨论的问题,揭示冲突类型和所在
\end{itemize}

\logicwarn{实践局限}:
\begin{itemize}
  \item 找出区别本身\logicwarn{并不能解决问题或消除歧见}
  \item 识别过程中存在不可避免的困难
  \item 需要结合具体情况进行综合判断
\end{itemize}

\logicemph{核心原理}:我们越充分地理解歧见的本性,就越能够更好地解决歧见。
\end{theorembox}

\chaptersummary{
人际交流中的分歧具有复杂的结构,理解这种复杂性是有效解决冲突的前提。

\logicemph{歧见的基本分类}:
\begin{itemize}
  \item \logicterm{信念歧见}:对事实本身的认知不同,争议焦点是"事实是什么"
  \item \logicterm{态度歧见}:对同一事实的评价不同,争议焦点是"如何看待事实"
  \item 两种歧见在性质和解决方法上根本不同
\end{itemize}

\logicemph{四种关系类型}:
\begin{itemize}
  \item \logicterm{充分一致}:信念和态度都一致
  \item \logicterm{信念一致,态度对立}:事实无争议,价值观有冲突
  \item \logicterm{态度一致,信念对立}:价值观相同,但基于不同事实认知
  \item \logicterm{完全对立}:既有事实争议又有价值冲突
\end{itemize}

\logicemph{解决策略的差异}:
\begin{itemize}
  \item \logicterm{信念歧见}:通过确认事实、科学探究等方法解决
  \item \logicterm{态度歧见}:需要后果分析、动机探讨、表达性说服、修辞艺术等策略
  \item 准确识别歧见类型是选择正确解决方法的关键
\end{itemize}

\logicwarn{实践意义}:态度歧见同样是实质性的,不是"纯粹言辞的";理解歧见的本性越充分,就越能够更好地解决分歧。
}