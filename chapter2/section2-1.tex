\section{语言的三种基本功能}

\begin{quotation}
\textit{语言作为人类交流的工具有着多种用法,理解语言的三种基本功能对于正确分析和解读语言表达至关重要。}
\end{quotation}

语言是一种非常精细和复杂的工具,以至于我们可能会忽视它的用法 (uses)的多样性。我们会自然而然地去追求简单化,而不注意语言使用的语境及其不同使用目的,从而就可能被我们遇到的语词或者话语形式所误导。

语词并不总是服务于它们的表面诉求。在非正式谈话中,"你好吗?"这个问题并不是真正地在询问他人的健康状况。虽然这句话好像是在进行信息请求(a request for information),但是我们知道,它通常仅仅是一句友好的问候。那些通过描述其健康状况来回答该问题的人,就很可能被认为是愚義的。请求、报道和问候等仅是语言所具有的一些比较明显的功能。

\subsection{语言功能的多样性}

哲学家乔治•贝克莱(George Berkeley)在他的《人类知识原理》 (Treatise Concerning the Principle of Human Knowledge)(1710)中说:\\
……思想交流……并非像通常想象的那样是语言的首要的和唯一的目的。使用语言还有许多其他宗旨,诸如引起某些情感、鼓动或者抑制行动、使人专心于某些特定的安排等等。前者(交流思想)在很多情况下都基本上是从属性的;当没有前者也能实现那些宗旨时,前者甚或完全被忽略,我认为,这种情况在人们熟悉的语言用法中经常发生。

20世纪的哲学家们非常详尽地阐明了多种多样的语言用法。路德维希•维特根斯坦(Ludwig Wittgenstein)在其《哲学研究》(Philosophi- cal Investigations)(1953)中正确地主张,"我们称之为'符号'、'语词'和'语句'的东西有无数不同种类的用法。"在维特根斯坦所列举的例子中有:发出命令、描述物体的外表或者给出它的测量结果、报道事件、推测事件、提出和检验假说、结合图表提交实验的结果、编故事、演戏、唱歌、猜谜、开玩笑、解算术题、语言翻译、询问、思考、诅咒、问候和祈祷等等。

语言的用法多得令人吃惊,通过将它们分为三种非常笼统的种类,即\textbf{信息性用法}(the informative)、\textbf{表达性用法}(the expressive)和\textbf{指令性用法}(the directive),可以使之具有一定的条理。诚然,这种三重划分是一种简化,甚或过于简化。但是,许多逻辑和语言学者发现这是一种非常有用的划分。

\subsection{信息性用法}

1.语言的第一种用法是用于信息交流。通常,它是通过明确表述并肯定(或者否定)命题来完成的。能被用于肯定或否定命题,或者能为此提出论证,称为语言的\textbf{信息性功能}。这里使用的"信息"一词也包括错误信息,即既包括真命题也包括假命题,既包括错误的论证也包括正确的论证。信息性话语用来描述世界和进行有关世界的推理。无论其所报道的事实重要与否、是普遍还是特殊,用来描述和报道的语言都是用来提供信息的。下面就是一个语言的信息性用法的简单例子,它出自最近佛罗里达高等法院的一篇报道:

\begin{displayquote}
2000年11月7日,星期二,佛罗里达州,和美国其他州一道,进行美国总统普选。11月8日,星期三,该选举分区(佛罗里达州)报道说,共和党候选人乔治•布什获得 2909135 张选票,民主党候选人小艾伯特•戈尔获得 2907351 张选票。因为投给他们的全部票数的总差(1784张),低于该选区全部投票票数的百分之一的一半,所以根据佛罗里达州法律规定进行了自动重新计票。 ${ }^{[1]}$
\end{displayquote}

\subsection{表达性用法}

2.正如信息性话语的最为清晰的例示来自于法院或者实验室的报道一样,语言用做\textbf{表达性用法}的最好的例子来自抒情诗。面对令人惊奇的古城帕特拉(Petra)遗迹,约翰•W•伯根(John W.Burgon)的诗句:

\begin{displayquote}
如此奇迹今我惊叹,它保留在东部的风情中——玫瑰一样红的城市——"几乎和时间一样永恒"!
\end{displayquote}

并不是意欲告诉我们关于世界的任何事实和理论,而是要表达诗人的赞赏和敬畏之情。这些诗句告诉了我们他眼前的一些真实景色,但是它们的主要目的并不是为了报道信息。诗句表达了作者感受到的强烈情感,其目的在于在读者心灵中激起类似的情感。无论何时,如果语言被用来发泄和激发情感,那么它就具有表达性的功能。\\
"表达"这个词在这里的运用范围比通常要狭窄。很自然,我们可以说表达感情、情感或态度等。但是通常情况下,人们也可以说表达见解、信念或者信仰。为了避免混淆语言的信息性的和表达性的功能,我们将说陈述或表明见解或信念,而在本章中将"表达"这个词留用于揭示或交流情感、感受和态度。

并非所有的表达性语言都是诗歌。我们用"真糟糕"或"真遗憾"来表达悲伤,用"好极了"或"太妙了"来表达兴奋。强烈的情感可以通过恋人爱慕的喃喃私语来表达。崇信上帝者,通过诵读主祷文或大卫王的赞美诗第 23 篇,可以表达他对广袤无垠和神秘莫测的宇宙的敬畏和惊叹之情。语言的这种用法不在于交流信息,而在于表达情感、感受或态度。正因为表达性话语只是表达性的,故而其既不真也不假。若把真与假、正确与错误作为衡量抒情诗之类的表达性话语的标准,那就会文不对题,使其价值丧失殆尽。谁要是因为知道是巴尔波(Balboa)而不是柯尔特斯 (Cortés)发现了太平洋,而降低对济慈(Keats)的十四行诗《初读查普曼译荷马》(On First Looking into Chapman's Homer)的欣赏,那他就是一个蹩脚的读者。这首诗的宗旨并不是教给人们历史知识。当然,有些诗的确含有重要的信息性内容。在伟大的诗人那里,有些诗的确是很好的 "生活评判"。但是,这种诗就不仅仅是表达性的了。可以说,这种诗具有 "混合用途"(mixed usage)或曰包括多种功能。这个概念将在后面作进一步讨论。

表达可以分成两种情形。当一个人独自发泄,写诗却不视之以人,或者孤独地祷告时,他的语言就起到表达说话者或写作者情感的功能,但不是为了在别人心中引起共鸣。另一方面,当演讲者要寻求感染他人时,当恋人在求爱中使用诗歌语言时,以及当人群为运动队欢呼时,语言就不但被用来表示说话者的情感,而且还意欲在听者心中引发共鸣。总之,表达性话语或者用来表达说话者的情感,或者用来激发听讲者的情感共鸣。当然,它也可以同时具有这两方面的功用。

\subsection{指令性用法}

3.当语言意欲引起或阻止明显的行动时,它就具有了第三种功能,即\textbf{指令性功能}。其最显然的例子就是命令和请求。当父母告诉孩子洗手吃饭时,其意图不是为了交流任何信息或者表达或激发任何特殊情感。这种语言是为了获得其指令结果。当你去电影院对售票员说"请给两张"时,语言也是被指令性地使用以产生行动。命令和请求之间的差别是微妙的,因为通过语调的适当转换,或仅仅是加上"请"这个词,几乎所有的命令都可以变为请求。通常,当提出问题以寻求回答时,该问题也被归类于指令性话语。

在单纯的祈使形式中,指令性话语既不真也不假。一个命令,比如 "关上窗户",既不能是真的也不能是假的。对一个命令是否应当遵从,我们可能会有不同意见;但是,对命令是否有真假,我们不会有分歧,因为真假这样的词语不能直接运用于命令句。不过,命令和请求还具有其他属性——合理或不合理、适当或不适当——这些属性与信息性话语的真或假有相类似之处。在第1章中我们看到,我们可以为履行某个行动给出理由,而如果陈述这些理由时伴随着命令,那么我们就可以把整个过程视为论证。例如:

\begin{displayquote}
小心驾驶!谨记墓地中满是守法的公民,他们有走路的权利。 $^{[2]}$
\end{displayquote}

在把这种话语处理为论证时,我们可以把其中包含的命令视做命题。命令的接受者从中被告知他们应当或应该履行所命令的行动。有些学者研究了这类问题,已经发展出了"祈使逻辑"(logic of imperatives),但对它的讨论超出了本书的范围。 $^{[3]}$ 

\begin{center}
\fbox{\parbox{0.9\textwidth}{
  \centering
  \textbf{语言的三种基本功能}\\
  信息性用法:用于交流信息,肯定或否定命题,描述和解释世界;\\
  表达性用法:用于表达和激发情感、感受或态度,既不真也不假;\\
  指令性用法:用于引起或阻止行动,如命令、请求和问题。
}}
\end{center} 