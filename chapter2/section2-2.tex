\
\\section*{2.2 多功能话语}
前面所给出的信息性的、表达性的和指令性的话语的例子,打个比方说,都像纯正的化学标本。语言交流的这种三重划分是启发性的,有价值的,但是不能机械运用,因为几乎任何一种正常交流都可能会表现出语言的这三种用法。例如,一首诗可能主要是表达性话语,但也可能会有教育意义并因而也可以引导读者走向不同的生活方式。华兹华斯(Words- worth)写道:

我们身边的世界丰富精彩:迟早,无论是得到和失去,我们都要浪费掉自己的精力:

显然,诗歌也可以包含一定数量的信息。\\\\
再如,尽管布道主要是指令性的,希冀在会众中带来一定令人称赞的行动(抛弃罪恶,乐善好施),但是它也可以表达并激发情感,因而具有表达性功能,而且还可以包括一些信息,比如福音(the Gospels)的好消息。科学论文虽然本质上是信息性的,但是也可以表达作者的理性激情,而且还可以至少是含蓄地请读者去独立地证明作者的结论。语言的大多数平常的用法都是混合的。

语言的这种混合的或多功能的用法,并不是因为说话者或作者混淆了它们。相反,成功的交流都要求一定的功能结合。除清晰的语境和正式的关系——一父母与子女,雇主与雇员——之外,人们不能简单地发布命令并希望它得到执行。赤裸裸的命令会引起反感和敌对,并且经常是自生自灭。因此,必须使用一定的间接方式。通常,为了追求我们要引起的行动,我们并不直截了当地发布命令,而需要使用比较委婉的方法。

行动常常具有非常复杂的原因。与逻辑学家相比,心理学家更为适合研究动机,但是行动通常既涉及行动者的欲望又涉及他的信念,这是常识。除非饥饿的人相信他们面前的东西是食物,否则,他们就不会把它放进自己的嘴里;除非想吃,否则,即使人们毫不怀疑面前就是食物,他们也不会去碰它。

欲望是我们所谓的"态度"或"情感"的特殊类型,而信念通常会受到所接收到的信息的影响。因此,我们有时是通过激起他人的适当态度而成功地引起他们的行动,而有时则是通过提供信息以影响他们的相关信念而做到这一点。

假设你的目标是使你的听众向某个特定的慈善组织提供捐助,假定你的听众态度是助人为乐的,你就可以通过给他们提供该慈善机构的良好工作信息而促使他们行动。你的语言是指令性的,目的是引起行动,但你是通过提供信息,而不是通过发布一条毫无掩饰的命令或不客气的要求,而达到你的诉求的。再如,假定你的听众已经被深深地说服,相信我们谈论的那个慈善机构确实信誉良好,但对捐献的鲁莽要求仍然可能失败;但是,通过在一定程度上激发他们的乐善好施情感,你就可能会成功地使他们向该慈善机构提供捐助。在这种情况下,你就通过使用表达性的话语而

达到了自己的目的:你实现了一个"动人诉求"(moving appeal)。这样,你的语言自然而然地就具有了混合用法,既有表达性功能又有指令性功能。

最后,再假设你要向那些既缺乏乐善好施的态度,也缺乏对你推荐的慈善组织之信誉的认识的人进行捐献动员,那么你就必须一并使用语言的三种功能,既要有表达与信息功能,又旨在引发行动。这种一并使用并不是可偶尔为之的用法,而是必须为成功交流而精心准备的基本手段。

语言的三种基本用法是:信息性用法、表达性用法和指令性用法。但值得一提的是,语言在某些特殊语境中还具有一些特殊用法,而这些用法并不能完好地归属这三重划分。

语言的礼仪性用法是很普遍的,在有些场合下它是一种表达性的和指令性的话语的混合。社交中的问候、赴宴邀请、雇用告知等表述,都是体现礼仪功能的例子;语言还有很多其他相关的不确定用法,它们主要服务于使人们之间的互动变得融洽。礼仪功能还是宗教场合中一种庄重的语言用法。给人留下深刻印象的结婚仪式的语言,既要突出场合的庄重性(表达性的功能),又要使新郎和新娘提高对严肃的结婚誓言的正确理解而引起新的角色行为(指令性的功能)。

与礼仪性用法相近,语言还有其他一些用法,它们也不仅仅是那些主要用法的混合。假如有人邀请你在某一时间和地点去参加会议,你回答 "好,我答应你",那么你就不仅是以此表明了你的态度和预告了你的行为,你同时还用语言来许诺。相似的,在婚礼结束时,司仪或主持人说 "我宣布你们是夫妻",这也不仅是表明了说话者在干什么。在有些语境中,说出某些话实际上包含了一种重要行动。这些都是语言的践行性 (performative)用法的例子。

践行性话语实际上就是实施一种行动,即其所报告或描述的行动。践行性动词是一种特殊的种类,它们代表行动,这种行动通常是以第一人称使用动词而(在适当的情况下)完成的。这里可以再举出一些例子,如 "我祝贺你……"、"我向你道歉,我……"、"我建议……"、"我将这艘船命名为 $\\cdots \\cdots$"、"我接受你建议......"等等。

语词和语句的这些及其他的特殊用法显示了自然语言的丰富性,它们的诸多复杂功能难以归纳为任何一个单独的分类系统。 