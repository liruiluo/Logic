\section*{2.6 情感中性语言}
语言的表达性用法与信息性用法一样是正当的。情感语言本身没有什么不当,非情感语言或曰中性语言也没有什么不当。这就如同我们可以说枕头没什么错,锤子也没有什么错一样毋庸置疑。但是我们不能用枕头钉钉子,也不能枕在锤子上舒服地休息。同样,当我们用实话实说的话语来替换诗人的情感语言时,尽管可以保留罗曼蒂克诗句的字面意义,但它会失去很多兴味。在某些类型的诗歌中,情感色彩浓郁的语言比中性语言更受喜爱;而在另一些领域中,中性语言则比情感色彩浓郁的语言更为可取。

那是些什么领域呢?如果把探求现实真理作为我们的目标,那么中性语言就应更受重视。当我们试图了解事实的真相所在,或者试图加以论证时,心猿意马就会招致失败,而情感因素正是一种分散注意力的力量。激情倾向于掩盖理性,"冷静"(dispassionate)与"客观"(objective)两词接近同义的用法就反映了这个道理。因此,当我们试图以冷静和客观的方式推论事实时,使用强烈的情感语言便是有害而无益的。

当我们处理某些冲突的话题时,使用完全中性的、不受感情影响的语言或许并不可行。例如,在流产是否正当的论辩中,论辩对手(无论哪一方)所使用的关键词汇就非常可能为情感所扭曲,因为此时并不存在完全不带情感色彩的所有各方都接受的价值中性词汇。在这种情况下,如果真正的目标仍然是求真,那么就应当尽可能地减少所用词汇的情感负荷。情感中性的目标不可能完全达到,但是我们至少可以尽力使用这种语言:它仅预设论辩者都赞同的信念(无论何种)。情感色彩的语言肯定是分散注意力的,情感意义负荷过重的语言是不可能促进求真的。

在论文《决定论的两难》(The Dilemma of Determinism)中,威廉•詹姆士(William James)提出"希望消灭"自由' 这个词",其理由是"它的歌功颂德的联想……使它的所有其他意义黯然失色"。他更喜欢恰当地使用词汇"决定论"和"非决定论"来讨论"意志自由",他说,这是因为"它们冰冷的和数学的面孔没有情感牵连,而情感牵连可以预先以各种方式来贿赂我们"。詹姆士的做法为我们树立了榜样。

从事专业民意调查的采访者必须非常谨慎,以免对调查询问中使用情感措辞而收到的反馈产生偏见。如果忽视这种慎重,调查结果就可能是没有价值的。1993年,《时代》(Time)和有线电视新闻网(Cable News Network)在一项大型民意调查中询问道:"应当通过立法来禁止利益集团赞助竞选吗?或者,利益集团的确应当拥有赞助他们支持的候选人的权利吗?"在所有回复者中, $40 \%$ 的人说他们赞同禁止利益集团的赞助, $55 \%$ 的人回答利益集团有赞助的权利。同年,罗斯-佩罗(Ross Perot),一位非常富有的总统候选人,组织了其自己的民意测验,其中如此询问: "应该通过立法来消除所有特殊利益者给候选人大笔金钱的可能性吗?"冊庸惊奇,对这个问题 $80 \%$ 的回答是"是的",这样的赞助应当禁止。包含像"特殊利益"和"大笔金钱"这样的短语无疑妨害了了解人们对这种事情的真实看法。 ${ }^{[11]}$ 可以说,这两个民意测验问的不是同样的问题。但即使如此,逻辑要点仍然是:如果我们的目标是交流信息,如果我们希望避免误会,那么我们就应该尽可能少地使用情感色彩浓厚的语言。玩弄情感,而不是诉诸理性,是那些从歪曲真相中获得好处的人的常用手段。这种操纵的努力最公开的展示是广告世界,那里压倒一切的目的总是说服、销售并且常常是获利。我们必须经常警惕这些情感负荷的语言用法,也要警惕它们在政治活动中的使用,几乎每一种修辞手段都在政治活动中一再使

用。本杰明•迪斯雷利说:"我们能利用语词来支配人。"我们最好的防御就是对语言及其不同的用法要多思而敏感,并具备识别那些不讲道德原则的人强词夺理的伎俩的能力。 