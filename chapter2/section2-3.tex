\section*{2.3 话语形式}
语句常常被定义为表达一个完整思想的语言单位。在语法教科书中,语句一般被分为四种类型,即陈述句、疑问句、祈使句和感叹句。但是,这四种语法分类与陈述、询问、命令和惊叹并不完全对应。我们可能会尝试把形式等同于功能,即认为陈述句和信息性话语是对应的,感叹句仅仅适用于表达性话语,或者我们会认为指令性话语完全包括祈使句或疑问语气的句子(把询问总是当做寻求回答的请求)。假如这种整齐的等同是可能的,那么交流问题就会大大地简单化了,因为这样我们就可以仅仅通过一段话的形式而说出它的原有功能,而其形式是很容易直接检查的。但是,那些把形式与功能等同的人会错误理解别人的话,而且或许会漏掉别人要传达的很多要点。

设想任何具有陈述句形式的事物都是信息性话语,真的就予以好评, 77 假的就予以拒斥,这显然是不正确的。"我在你的聚会上度过了一段美好的时光"是个陈述句,但它的功能完全不必是信息性的;倒不如说,它是礼仪性的或者表达性的,表达了友好和欣慰的感情。尽管很多诗歌和祷文的功能都不是信息性的,但它们却具有陈述句的形式。简单地认为它们是信息性的并简单地将它们评价为真的或假的,将会把自己排除在富有价值的审美和宗教体验之外。同样,许多请求和命令都是间接地——或许更加委婉地———用陈述句来表述的。"我喜欢咖啡"这个判断句不应当被侍者仅仅当做是顾客的表白,而应当被视为针对行动的指令或请求。对于陈述句,诸如"对这些帮助,我将非常感谢"或"我希望你课后能够在图书馆见到我",假如我们僵硬地判别它们的真与假,只是将它们视为信息报道,那么我们很快就会没朋友了。这些例子向我们表明,陈述的形式并不一定就标朋信息性功能。陈述句在每种话语类型的表达方式中都能见到。

其他语句形式也是如此。"你意识到我们几乎要迟到了吗?"这个疑问句,不必是在询问你的大脑状态的有关信息,而可能是要求抓紧时间。疑问句"1939年,俄国和德国签署了一个条约,它导致了第二次世界大战,不是吗?"可能完全不是询问,而可能是交流信息的间接方式或者是企图表达并激发一种对俄国的敌对情感;在第一种情况下起的是信息性功能,

而在第二种情况下起的是表达性功能。甚至语法上的祈使句,如在公文中以"兹请周知 $\cdots \cdots$"开头,可能并不是命令,在其所断言的东西中是信息性话语;在其激发神圣和庄重的适当情感的语言用法中是表达性话语。就感叹句来说,尽管它与表达性话语的功能关系紧密,但也可以有相当不同的功能。感叹句"天啊,要迟到了!"在语境中可以表示抓紧时间的请求。而房地产经纪人对潜在的顾客说出"多么美好的景色啊!"这个感叹句,其祈使性功能比表达性功能更浓重。

许多话语都企图同时达到语言的两种或者可能是其全部的三种功能。在这种情况下,给定语段的每一方面或功能都从属于它自己的适当标准。一个具有信息性功能的语段可以具有能够评价为真或假的方面。同样的语段,也可以具有指令性功能,能够以恰当或不恰当、对或错等进行评价。而假如这个语段还具有表达性功能,那么其组成成分还可以评价为真诚或虚假、是否宝贵等。正确地评价某一给定语段,需要把握语言的不同功能以及该语段本身的宗旨。

对逻辑学家来说,真与假,以及与之相关的论证正确与错误的概念,是最重要的。因此,作为学习逻辑的学生,我们必须能够将信息性功能与非信息性功能的话语区分开来。进而,我们还必须能够理顺给定语段所具有的信息性功能与其可能具有的所有其他功能之间的关系。为了做到"理顺",我们必须知道语言可以具有哪些功能,以及必须能够将它们区分开来。语段的语法结构常常能够标示其功能,但是,功能与语法形式之间并没有必然联系。功能与语段表面上断言的内容之间也不存在严格的关系。关于这一点,一位大语言学家在他对"意义"(meaning)的讨论中给出了例证说明:

\begin{displayquote}
一个顽皮的孩子,该上床睡觉时却说"我饿了",他母亲知道他的把戏,就以打发他去睡觉来回答他。这是移位语(dis- placed speech)(1)的一个例子。 ${ }^{[4]}$
\end{displayquote}

在这里,这个孩子的话是指令性的一一虽然没有成功地实现其所希望的改变。关于语段的功能,我们一般是指它想要达到的那种功能。但不幸的

\footnotetext{(1)日常会话中的"托词"是"移位语"的一种典型,本处即为托词之例。
}是,那并不是总能够轻而易举地判定的。\\\\
当孤立地引用一个语段时,要判定该语段最初欲要达到的功能就特别困难。其原因在于,语境在判定功能的过程中极其重要。某些本身是祈使性或者普通信息性的语句,如果在实际语境中将它们安排到一个具有诗化效果的整体之中,就可以变为表达性语句。例如,孤立的:

到窗口来吧,

是一个起指令性功能的祈使句。而:

\section*{今夜海上风平浪静}
是陈述句,起着信息性功能。它们好像都没有较大的表达性威力,但是在马修•阿诺德(Matthew Arnod)的诗歌《多弗海滨》(Dover Beach)的语境中,二者都主要用为诗的表达性功能,而且效果显著。很多诗歌完全依赖于语段的表达性用法,而在其他语境中,这些语段却具有根本不同的功能。

下面的区分也是重要的,即句子表示的命题与关于说话者的事实(句子的说出就是证据)之间的区分。当有人说"天在下雨"时,这个被断言的命题是关于天气的,而不是关于说话者的。但做出断言却构成了说话者相信天在下雨的证据,这就是关于说话者的事实。也可能发生下述情况:人们做出的陈述只是表面上关乎他们的信念,实际上并不是为了给出关于他们自己的信息,而完全是一种述说其他事情的方式。如说"我认为黄金不应该用做货币的标准",通常并不能理解为关于说话者信念的心理自我表白,而只能理解为断言不应该使用黄金作为货币标准的一种方式。而当说话者发出命令时,由此推断说话者希望有人完成某事却是合理的;的确,在有些环境中,只要断言某人有某种特殊渴望,实际上就是给出了一个命令或者做出了一个请求。快乐地欢呼证明说话者非常愉快,即使他在整个过程中没有做出任何断言。但是,作为心理报道,断言说话者快乐就是肯定一个命题,这与只是快乐地欢呼相当不同。

在1.6节中我们注意到,论证与说明之间的差别常常取决于语段的说话者或作者的目的。现在,对语言的不同功能的探讨允许我们可以对该问

题进行更深人的考察。\\\\
当说话者论及某个有争议问题时,如果他说"我强烈反对什么什么",那么我们就理解,这样一句话的目的通常不是为了报道说话者的观点(除非这样的话是一位公职候选人或者其观点代表了公众利益的知名人士所讲的)。实际上,这种自我报道的表达形式是述说什么什么是个坏主意并且我们都应当反对它的常用方式。当说话者不断地证明我们所持的观点是正确的时候,他并不是在说明他的判断,而是有意的论证,以说服别人相信他的判断是正确的。对于有些争论性问题,通过陈述自己的观点而展开论证,并不是有意欺骗;在这种情况下,即使把判断和自我报道混合在一块儿,也不是有意欺骗。

在一个单独语段中,一种以上重要功能的组合可能会成为问题。思想表达,受我们宪法第一修正案的保护,可能会包含极具冒犯性的语言;在这种情况下,认识到冒犯言语中信息性与情感性功能的融合(integra- tion),对保护言论自由可能是极其关键的。在洛杉矶地区法院,一个年轻人身穿故意装饰有亵渎之物的夹克来抗议越战时期的军事法案;根据加利福尼亚刑法典,他以冒犯行为而被判处有罪。高等法院推翻了对他的指控,并雄辩地精确表述了这个问题:

我们不能忽视这个事实,这里牵涉的事件极好地表明,许多语言表达都具有双重交流功能:加上附带说明,它们不仅可以传达相当精确的思想,还可以传达其他不可表达的情感。实际上,与其情感力量一样,语词也常常因其认知力量而被选用。我们不能赞同这样的观点:宪法关注个人言辞的认知内容,而没有或不关心其情感功能。的确,情感功能可能常常是在寻求交流的信息的更重要因素……同样,我们也不能纵容这种轻率假定:可以禁止特殊言辞,但在这个过程中又不遭遇压制思想的真正危险。 ${ }^{[5]}$

区别语言的信息性以及论证性功能与语言的其他功能,并没有一个机械的方法。在后面几章中,我们发展的逻辑技术可以相当机械地运用于检验论证的有效性,但是没有一个机械技术可以识别论证的出现。识别在一给定语境中的话语的不同功能,要对语言的灵活性和其用法的复杂性熟虑

而敏感。\\\\
80

\begin{center}
\begin{tabular}{|l|l|}
\hline
语言的主要用法 & 语言的语法形式 \\
\hline
 & 陈述可 \\
\hline
信息性用法 &  \\
\hline
 & 疑问句 \\
\hline
表达性用法 &  \\
\hline
 & 祈使句 \\
\hline
指令性用法 &  \\
\hline
 & 感叹句 \\
\hline
 & 语法形式常常是功能的一个标志,但语法形式与其使用目的之间并没 \\
\hline
 & 定的联系。用做三种主要功能(左侧纵列)的任何一个功能的语言, \\
\hline
能会采用四种语法形式(右侧纵列)中的任何一种。 &  \\
\hline
\end{tabular}
\end{center} 