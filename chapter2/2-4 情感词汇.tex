\section{情感词汇}

\begin{logicbox}[title=引言]
\textit{词汇不仅具有字面意义,还蕴含着情感色彩。了解词汇的这种双重性对于正确解读语言表达和分析语言使用十分重要。}
\end{logicbox}

现在,我们从讨论语句和较复杂的语段转向探讨构成它们的词汇。正如 2.2 节所见,一个单独的语句,可以同时具有信息性的和表达性的用法。要具有前者,句子必须明确表述一个命题,而要做到这一点,它的词汇必须具有字面的或描述性的意义,以指示客体或事件以及它们的性质或关系。而当句子表达态度或感情时,其词汇就会具有情感的暗示或影响。一个语词或短语可以既具有字面意义又具有情感影响。后者通常被称为词汇的\textbf{情感意义}。

\subsection{字面意义与情感意义}

词汇的\textbf{字面意义}和\textbf{情感意义}在很大程度上是各自独立的。例如,"官僚"(bureaucrat)、"政府官员"(government official)与"公仆"(public servant)的字面意义几乎一样,但它们的情感意义却很有区别。"官僚"倾向于表达厌恶和反对,而作为敬语的"公仆"则倾向于表达尊重和赞赏。"政府官员"则更接近中性。

显然,我们用以指示事物的词汇会明显地影响我们对事物的态度。花的实际芳香不会因其名称而改变。正如莎士比亚所写的那样,一朵玫瑰不论使用其他任何名字,闻起来都是香甜的。然而,假如有人告诉我们有一种称做"臭菘"(skunkweed)的玫瑰,我们对它的反应就可能会受到影响。在华尔街(Wall Street),谨慎地选择语言可以促使人们在股票市场上采取行动。某几天是"回升",它意味着价格上涨;另几天是"取短期利息",这意味着价格在下降,因为很多人都在抛售股票,但这个词语仍然好听。如今大公司极少再进行"破产",但它们可以"重组",这听起来要好多了。

\subsection{委婉语的应用}

这种态度上的影响可以说明\textbf{委婉语}增多的现象,委婉语就是用温和的词汇表示严酷的现实。在战争中,己方军队的失败可能被称做容易接受的 "暂时撤退",而重大的撤退可能被报道为"兵力的有序集结"。在越南战争中,正在竞选总统提名的参议员尤金-麦卡锡(Eugene McCarthy)曾对美国的军事干预政策及公众不愿坦诚地面对它的状况进行了颇具讽刺性的批评:"我们不再宣称战争",他说,"我们宣称国家防御。"$^{[6]}$

我们在不断地创造新词汇以替代那些不再令人满意的旧词汇。"殡葬人员"(undertake)变成了"殡仪员"(mortician),"看门者"(janitors)变为"守护员"(maintenance men),"老人"(old people)变成了"年长公民"(senior citizens)。但是,与旧的实际情况相联系的新替换的词汇最终也会失去它们的吸引力;"守护员"结果被"保卫者"(custodian)所代替,"殡仪员"为"殡仪主管"(funeral director)所取代,等等。杰梅恩•格瑞尔(Germaine Greer)写道:

\begin{displayquote}
经过与其指代的实际相联系,委婉语便迅速地失去了它们的功能,这是它们的宿命。因此,它们必须经常被它们自己的委婉语所取代。 $^{[7]}$
\end{displayquote}

据说,杜鲁门总统的夫人贝丝(Bess)的朋友请求她阻止杜鲁门再说"大粪"(manure),她回答说,她花费了四十年时间才使他开始说"大粪"。

\subsection{情感意义的力量}

语言的确有它自己的生命,独立于它用以描绘的事实。有些包含生殖和排泄的生理活动可以用医学词汇不带情感地进行描述,而不至于引起神经质的不快;但是使用下流粗俗的词汇描述同样的活动却可以震惊除最麻木不仁者之外的所有听众。用我们的术语来说,可以说这两种词汇具有相同的字面或描述意义,但是在它们的情感意义上却有或缓和或激烈的对立。

在某个具体的人的思想中,词或短语有时可以产生情感意义,那不是产生于其字面上指代的东西,而是产生于第一次学习或遇到它的语境。一个作者曾描述说:

\begin{displayquote}
这是一个有启发意义的故事。一个小姑娘近来学会了阅读,正在拼读报纸上的一篇政治性文章。"爸爸,"她问,"什么是坦曼尼协会(Tammany Hall)?"她爸爸用通常社交禁语的口吻回答说:"亲爱的,你长大了就会明白的。"按照这种奇怪的成年人遁词,她终止了询问;但在她父亲的语气中,有某些东西使她相信坦曼尼协会必定与不正当的性事有关,以至于多年来她只要一听到这个政治机构$^{(1)}$就会体验到一种神秘的非政治性震颤。$^{[8]}$
\end{displayquote}

对于很多人来说,一定的词汇或短语,由于与我们的生活具有某些特殊联系,可以携带某种我们或许不愿公开承认的隐私情感。

\footnotetext{(1)坦曼尼协会最初是旨在通过捐赠与赞助进行控制的纽约市民主党执行委员会,成立于 1789 年。1805年转型为有明确政治宗旨的"慈着机构"。}

\subsection{情感意义的创造性应用}

字面意义和情感意义之间的对比,以及它们不同的可操作性用法,促使哲学家伯特兰•罗素设计了一种寓教于乐的游戏。他这样"调配"出一类"不规则动词"(irregular verb):

我坚定;你倔强;他头脑呆板。

伦敦的《新政治家和国家》(New Statesman and Nation)杂志随后举行了一次竞赛,以征求这样的不规则词汇表,获胜者如下:

我义愤;你生气;\\
他破口大骂。

我重新考虑过了;你改变了想法;\\
他食言了。

这种游戏证实了普通经验的教益:相同事物可以被情感色彩非常不同的词汇所指称。

\chaptersummary{
词汇的双重性质——字面意义与情感意义的并存,揭示了语言影响思维和态度的深层机制。

\logicemph{词汇的双重维度}:
\begin{itemize}
  \item \logicterm{字面意义}:词汇的描述性功能,指示客体、事件及其性质关系
  \item \logicterm{情感意义}:词汇的情感暗示和影响,表达态度和感情
  \item 两种意义在很大程度上\logicwarn{相互独立},可以分别变化
\end{itemize}

\logicemph{情感意义的社会功能}:
\begin{itemize}
  \item \logicterm{委婉语现象}:用温和词汇表示严酷现实,改变情感反应
  \item \logicterm{词汇演变}:新词汇不断替代旧词汇,但最终也会失去吸引力
  \item \logicterm{态度引导}:相同事物可被不同情感色彩的词汇表达,影响人们的态度
\end{itemize}

\logicemph{实践启示}:
\begin{itemize}
  \item 语言具有独立于事实的生命力,能够塑造我们对现实的感知
  \item 理解词汇的情感维度有助于识别语言的真实意图和效果
  \item 在逻辑分析中需要区分词汇的描述功能和情感功能
\end{itemize}
}