\section{第2章概要}

\chaptersummary{
本章深入探讨了语言的复杂性和多样性,揭示了语言使用中的微妙之处,为理性分析和有效交流奠定了基础。

\logicemph{核心主题}:语言的多种用法和形式,以及因未能认识这些复杂性而可能引起的误解和滥用。

\logicemph{2.1 语言的三种基本功能}:
\begin{itemize}
  \item 建立了语言功能的基本分类框架:\logicterm{信息性}、\logicterm{表达性}和\logicterm{指令性}功能
  \item 阐明了每种功能的特征、目的和评价标准
  \item 强调了不同功能需要不同的分析方法
\end{itemize}

\logicemph{2.2 多功能话语}:
\begin{itemize}
  \item 展示了现实语言使用的复杂性:一个语段可能同时承担多种功能
  \item 分析了行动的心理学基础:欲望与信念的双重条件
  \item 介绍了礼仪性和践行性等特殊用法,超越了三重划分的范围
\end{itemize}

\logicemph{2.3 话语形式}:
\begin{itemize}
  \item 揭示了语法形式与语言功能之间的\logicwarn{非对应关系}
  \item 证明了陈述句、疑问句、祈使句和感叹句可以承担任何语言功能
  \item 强调了语境在判定语言功能中的决定性作用
\end{itemize}

\logicemph{2.4 情感词汇}:
\begin{itemize}
  \item 分析了词汇的双重性质:字面意义与情感意义
  \item 探讨了委婉语现象和词汇演变规律
  \item 揭示了语言塑造态度和感知的能力
\end{itemize}

\logicemph{2.5 一致与歧见的种类}:
\begin{itemize}
  \item 区分了\logicterm{信念歧见}(对事实认知不同)与\logicterm{态度歧见}(对事实评价不同)
  \item 构建了四种关系类型的分析框架
  \item 提出了针对不同歧见类型的解决策略
  \item 强调了准确识别歧见本性的重要性
\end{itemize}

\logicemph{2.6 情感中性语言}:
\begin{itemize}
  \item 论证了在真理追求中使用中性语言的重要性
  \item 分析了情感语言在不同领域的适用性和局限性
  \item 警示了情感操纵的危险,提出了防范策略
  \item 强调了语言选择的伦理责任
\end{itemize}

\logicwarn{整体意义}:本章为后续的逻辑分析提供了语言学基础,帮助读者理解语言的复杂性,培养对语言使用的敏感性和批判能力。
}

\vspace{1em}
% 参考文献将在主文档末尾统一显示