\section{第2章概要}

\begin{center}
\fbox{\parbox{0.95\textwidth}{
本章阐释语言的多种用法和形式,以及可能由于没有认识到这些复杂性而引起的错解和滥用。

\textbf{2.1 语言的三种基本功能}区分了语言的三种基本功能:信息性功能、表达性功能和指令性功能。

\textbf{2.2 多功能话语}展示了一个给定语段可能行使的多种功能的方式:同时行使两种甚或所有三种功能。

\textbf{2.3 话语形式}表明了标准语法形式的句子,即陈述句、疑问句、祈使句和感叹句,并不总是行使与其名称相关的功能。陈述句可以用做指令性的或者表达性的功能;疑问句可以具有信息性的或指令性的功能,等等。语法形式不决定语言功能。

\textbf{2.4 情感词汇}考察了构成句子的词汇的用法,也讨论了情感语言和字面意义的关系。

\textbf{2.5 一致与歧见的种类}区分了信念歧见与态度歧见。冲突双方可以既在事实是什么上一致也在对事实的态度上一致,或者在两方面都对立。他们可能在事实上一致,而在对事实的态度上对立。他们还可能在事实是什么上对立,但在对他们所相信的事实的态度上却一致。要解决歧见问题,了解其真正本性是极其重要的。

\textbf{2.6 情感中性语言}讨论当辩论的目的是求真时,尽可能地将负载情感的语段归约为情感中性语言的重要性。
}}
\end{center}

\printbibliography[heading=subbibliography,title={第2章参考文献}] 