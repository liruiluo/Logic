\section*{第2章概要}
本章阐释语言的多种用法和形式,以及可能由于没有认识到这些复杂性而引起的错解和滥用。\\
2.1 节区分了语言的三种基本功能:信息性功能、表达性功能和指令性功能。\\
2.2 节展示了一个给定语段可能行使的多种功能的方式:同时行使两种甚或所有三种功能。\\
2.3 节表明了标准语法形式的句子,即陈述句、疑问句、祈使句和感叹句,并不总是行使与其名称相关的功能。陈述句可以用做指令性的或者表达性的功能;疑问句可以具有信息性的或指令性的功能,等等。语法形式不决定语言功能。\\
2.4 节考察了构成句子的词汇的用法,也讨论了情感语言。\\
2.5 节区分了信念歧见与态度歧见。冲突双方可以既在事实是什么上一致也在对事实的态度上一致,或者在两方面都对立。他们可能在事实上一致,而在对事实的态度上对立。他们还可能在事实是什么上对立,但在对他们所相信的事实的态度上却一致。要解决歧见问题,了解其真正本性是极其重要的。

2. 6 节讨论当辩论的目的是求真时,尽可能地将负载情感的语段归约为情感中性语言的重要性。

\section*{【注释】}
[1]Palm Beach County Canvassing Board v.Katherine Harris,decided 21 No-\\
vember 2000.\\
[2]Ann Lander,"You Could Be Dead Right!"syndicated column, 26 August 1988.\\
[3]关于对这个主题的介绍,有兴趣的读者可以参考 Nicholas Rescher 的 The Logic of Commands(London:Routledge \textbackslash &Kegan Paul,1996)。\\
[4]Leonard Bloomfield,Language(New York:Henry Holt,1933).\\
[5]Cohen v.California, 403 U.S.15,at p.26, 1971.\\
[6]In a speech at the National Convention of the Democratic Party,in Chicago,Ju- ly 1968.\\
[7]In The Female Eunuch(New York:McGraw-Hill,1971).\\
[8]Margaret Schlauch,The Gift of Tongues(New York:Viking Press,1942).\\
[9]关于短语"在信念上"和"在态度上"的一致和歧见,以及我们在第 4 章讨论的概念的"说服定义"(persuasive definition),我们感激我们的同事和朋友斯蒂文森(Charles L.Stevenson)教授。参见他的《伦理学和语言》(Ethics and Language) (New Haven,CT:Yale University Press,1944)。\\
[10]对于这种歧见,切斯特顿(G.K.Chesterton)评论道:""我们的国家,或对或错'就像说'我的母亲,或陶醉或哭泣'。"\\
[11]佩罗民意测验的 17 个问题,大多都包含情感负荷的词汇;它们以"公民选票"的形式登在1993年3月20日的TVGuide上;1993年3月21日 NBC TV 也在全国范围内进行征询。1993年3月,New York Times 报道了 Time/CNN 的民意测验。 