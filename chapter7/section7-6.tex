\section*{7.6 连锁三段论(Sorites)}
有时会出现一个三段论论证,其前提多于两个。如果其结论是由前提依次推得的,那么它就是有效的,否则就是无效的。例如下面这个论证,它的前提有四个:

所有外交官都是机敏的人,\\
有外交官是欠思考的人,\\
所有欠思考的人都是轻率的,\\
没有轻率的人是谨慎的,\\
所以,有谨慎的人不是机敏的。

这个论证可以通过一系列环环相扣的直言三段论来进行检验。如果能把这个链条上的所有三段论都写出来,那么,任何一个违反了三段论六条规则的三段论都会使整个推理无效。

上述论证中的结论("有谨慎的人不是机敏的")可以由前提"没有轻率的人是谨慎的"和一个未出现的命题共同推出,这个未出现的命题就是"有轻率的人是机敏的"。这个未出现的命题,正是前面三个前提的结论。这样,我们就可以从一个论证推出另一个来。第一个论证是:

所有外交官都是机敏的人,\\
有外交官是欠思考的人,\\
所有欠思考的人都是轻率的,\\
所以,有轻率的人是机敏的。

而第二个论证的两个前提是:第一个论证的结论,以及原论证的第四个前提("没有轻率的人是谨慎的")。第二个论证是:

有轻率的人是机敏的,\\
没有轻率的人是谨慎的,\\
所以,有谨慎的人不是机敏的。

这样,就可以分别检验这两个三段论了。如果两个都有效,原论证就有效。由于第二个论证(结论是"有谨慎的人不是机敏的")的前提中,"轻率的"是中项,但两个前提中都没有出现"机敏的人"(大项)和"谨慎的人"(小项),所以,它并不符合标准形式。第二个论证的大前提(其中有大项)是"有轻率的人是机敏的",小前提是"没有轻率的人是谨慎的"。

这样,第二个论证的形式就是 IEO-3。这个形式违反了规则 3,因为大项在结论中周延而在前提中不周延,因此犯了大项不当周延谬误。这样,原论证的第二个环节无效,就使得整个论证无效。

这种包含几个前提和若干结论的三段论,如果每一个结论都成为下一个三段论的前提,就称为连锁三段论(sorites)。如果这些前提都是以标准形式排列,也就是说,每个词项(除了第一个前提的主项和最后一个前提的谓项)都分别作为前提的主项和谓项出现,这样的连锁三段论就可以看做是标准式的。如下例所示:

所有 $A$ 是 $B$ ,\\
所有 $B$ 是 $C$ ,\\
所有 $C$ 是 $D$ ,\\
没有 $D$ 是 $E$ ,\\
所以,没有 $A$ 是 $E$ 。

任何一个标准形式的连锁三段论都可以通过依次进行的三段论推论而得到检验。例如上面的连锁三段论,就可以通过如下三个三段论进行检验:

(1)所有 $B$ 是 $C$ ,\\
所有 $A$ 是 $B$ ,\\
所以,所有 $A$ 是 $C$ 。

(2)所有 $C$ 是 $D$ ,\\
所有 $A$ 是 $C$ ,(前一个三段论的结论)\\
所以,所有 $A$ 是 $D$ 。

(3)没有 $D$ 是 $E$ ,\\
所有 $A$ 是 $D$ ,(前一个三段论的结论)\\
所以,没有 $A$ 是 $E$ 。

这里所有的三段论都是第一格的。第一个和第二个是 Barbara 式,第三个是 Celarent 式,它们都是有效的。因此,原连锁三段论是有效的。

一个连锁三段论的前提可以写成任何顺序,为了检验其有效性,需要先把它们整理为标准顺序。一个标准式连锁三段论的有效性(或无效性)取决于构成它的所有三段论的有效性(或无效性)。 