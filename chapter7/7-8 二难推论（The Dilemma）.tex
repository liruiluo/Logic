\section{二难推论(The Dilemma)}

\begin{logicbox}[title=引言]
本节讨论二难推论这一强有力的论证形式,它使对手面临两种选择,而无论选择哪一种,都会导致不利的结论。我们将分析二难推论的结构、类型以及应对二难推论的三种主要方法。
\end{logicbox}

没有什么特别重要的地方。但从修辞角度看,二难推论是一种最有力量的说服工具之一,可谓论战中的一种致命性武器。

不严格地说,如果一个人必须在两种选项中做出决断,但两个选项都很糟糕或令人不愉快,那么,我们就说这个人"陷入"了两难(或者说进退维谷)之中。\textbf{二难推论}就是一种旨在使对手陷入这样境地的论证方式。在争论过程中,二难推论使得对手必须做出选择,但无论选择什么,都会得出一个他不能接受的结论。

理查德·费曼(Richard Feynman)是一位著名的物理学家,他在回忆1986年"挑战者"号爆炸的调查时,猛烈地抨击了(美国)国家航空航天局(NASA)的管理失误,他用的就是下面的二难推论:

\begin{quote}
我们每次问起高层管理者,他们都会说关于手下发生的事,他们什么都不知道……或者最高领导团确实不知道,这样他们就不知道应该知道的事,或者他们知道,这样他们就在对我们说谎。\cite{feynman1988}
\end{quote}

如此的质问就将对手(此处指的是国家航空航天局的管理者们)推入两难境地,令他们无地自容。其中唯一明确表述的前提是一个析取命题,但析取支必定有一个为真,或者他们知道或者他们不知道手下发生的事。不管选择哪一方,结果对对手来说都是不利的。二难推论的结论本身也可以是一个析取命题(例如,"国家航空航天局的管理者或者不知道他们应该知道的事,或者他们说谎"),此时我们称之为\textbf{复杂式}(complex)二难推论。结论也可以是直言命题,这时就称之为\textbf{简单式}二难推论。

二难推论的结论并非总是令人不愉快的,如下简单式二难推论得出的就是个好结论:

\begin{quote}
如果天上的神明没有欲求,那么他们就会很满足,如果他们有欲求而能完全实现,那么他们也会很满足。他们或者没有欲求或者能完全实现欲求。总之,他们都会很满足。
\end{quote}

二难推论的前提并没有特殊的顺序要求,提供选项的析取前提可前可后。表述选择后果的两个条件命题可以联合表述,也可分开陈述。二难推论常用省略式表述,结论一般都是显而易见的,无须表述出来。有一个例子取自林肯总统的一封信,他为废止美国南部邦黑奴制度的宣言作了如下辩护:

\begin{quote}
此宣言如同法律一样,或者有效或者无效。如果无效,就没必要取消。如果有效,就不能取消。任何人都明白。\cite{lincoln1861}
\end{quote}

\subsection{避开二难推论的方法}

避开或驳斥二难推论的结论的方法有三种,它们也有各自的名称,都与二难的两个(或多个)"死角"有关。分别称为"绕过(或避开)死角法"、"直击(擒拿)一角法"、"构造反二难法"。它们并非证明二难推论形式无效,而是在不改变推论形式有效性的前提下,寻找避免结论的方法。

\subsubsection{绕过死角法}

\textbf{绕过死角法}是拒斥其析取前提。这是常用的最容易的避开二难的手段。除非析取前提的两个支命题是矛盾关系,否则它们很有可能是假的。常用来说明这个方法的例示是给学生分级打分的例子,有人认为好的分数能激励学生更努力地学习。但学生们想出这样一个二难推论用来驳斥上述理论:

\begin{quote}
如果学生喜爱学习,那么就不需要激励。如果学生厌烦学习,那么激励也没有用。学生或者是喜爱学习的或者是厌烦学习,所以,激励是不需要的或者没用的。
\end{quote}

该论证形式是有效的,但我们能用绕过死角法来反驳这个论证。其析取前提是假的,因为学生会有不同的学习态度:有的喜爱,有的厌烦,还有许多人不同于前两者。对于后面这些人来说,激励既是需要的也是可以发挥作用的。这种方法并不是证明结论为假,只是表明推论本身并没有给结论提供充足的理由。

\subsubsection{直击一角法}

如果析取前提穷尽了所有可能性,是不可驳倒的,就不能用上述方法了。必须有另外的方法来避开结论,其中之一就是\textbf{直击一角法},即拒斥两个假言前提中的一个。要否定两假言前提的组合,我们只需否定其中的一个即可。直击一角,就是要试图表明条件前提至少一假。刚才驳斥学校分级打分的例子,所依据的条件前提之一是"如果学生喜爱学习,就不需要激励",反驳者可以争辩说,即使一个学生喜爱学习,也需要激励,好分数会带来额外的奖励,甚至能激励最勤奋的学生更认真地学习。这样一来,就很可能得到好的回应——原来的二难的一角就被击破了。

\subsubsection{构造反二难法}

\textbf{构造反二难法}是最巧妙的方法,但并不总能令人信服,我们来看这是为什么。用这种方法驳斥给定的二难推论,需要构造另一个二难推论,它的结论与原来的结论相反。辩驳中可以使用任何一个二难推论,但最理想的反二难推论应当与原来的推论有相同的组成成分(直言命题)。

有个古老的例子能说明这种方法,相传雅典有一位母亲劝儿子不要从政时说道:

\begin{quote}
如果你主持公道,人们就会仇视你。如果你不主持公道,神灵们就会仇视你。你必定或者主持公道或者不主持公道,所以无论如何都会被仇视。
\end{quote}

他的儿子反驳说:

\begin{quote}
如果我主持公道,神灵们就会施爱于我。如果我不主持公道,人们就会施爱于我。我必定或者主持公道或者不主持公道,所以我都会被爱。
\end{quote}

在把二难推论作为强力工具的日常论辩中一般人的争论中,这种驳斥方法,从几乎相同的前提得到相反的结论,是种很不错的修辞手法。但如果更细致地研究,就会发现它们的结论并不像初看上去那样对立。

第一个二难推论的结论是儿子会被仇视(被人们或者被神灵们),而反二难的结论是儿子会被爱(被神灵们或被人们)。实际上两者完全是相容的。反驳用的反二难仅仅是建构了一个结论不同的论证而已。两个结论可能都是真的,因而这里并没有达成真正的反驳。但在唇枪舌剑的辩论中,并不需要细致分析,如果在公共争辩中出现这样的反驳,听众大多会把它当做对原论证的毁灭性攻击。

如此反驳并不能驳倒推理,而只是将注意力引向同一事情的不同方面,这从如下的二难推论可能看得更清楚。"乐观主义者"认为:

\begin{quote}
如果我工作,就能挣钱,如果赋闲在家,那么我乐得自在。我或者工作或者不工作,总之,我能挣钱或者乐得自在。
\end{quote}

而悲观主义者却会给出这样一个反二难:

\begin{quote}
如果我工作,就不能乐得自在,如果赋闲在家,就不能挣钱。或者工作或者不工作,总之,我或者不能乐得自在或者不能挣钱。
\end{quote}

这些结论只能说明看问题的视角不同,并非对事实状况的意见不一致。

\subsection{普罗塔哥拉与欧提勒士的二难困境}

通常讲二难推论,都要说到普罗塔哥拉(Protagoras)和欧提勒士(Euathlus)之间著名的讼案。普罗塔哥拉是生活在公元前5世纪的希腊的一名教师,他开设了很多课程,其中最著名的是法庭辩护术,欧提勒士想跟他学习当一名律师,但他负担不起学费。于是两人定了一个契约,普罗塔哥拉先不收学费,等欧提勒士学成并在第一场官司中获胜时,再交学费。可是,欧提勒士学成之后,迟迟没有在法庭上进行辩护,普罗塔哥拉等得不耐烦了,于是把他的学生告上法庭,要求收回学费。欧提勒士忘记了"律师为自己的案子辩护乃属愚行"的格言,决定为自己进行辩护。审理开始后,普罗塔哥拉就用一个压倒性二难推论陈述己方要求:

\begin{quote}
如果欧提勒士打输了官司,那么他必须还我学费(根据法庭的判决),如果欧提勒士打赢了官司,那么他也必须还我学费(根据我们之间的契约),或者他打输或者打赢官司,都必须还我学费。
\end{quote}

情况看来对欧提勒士而言十分不利,但他已把修辞术学得很好,于是他向法庭提出了如下相反的二难推论:

\begin{quote}
如果我打赢了官司,我不必交学费(根据法庭的判决),如果我打输了官司,我也不必交学费(根据我们之间的契约),或者我打赢或者打输,都不必交学费。
\end{quote}

如果你是法官,该如何判决呢?

注意欧提勒士的反二难的结论与普罗塔哥拉的结论的确不相容,一个确实是另一个的否定。这种相反二难推论与原来的二难推论的互相拒斥的情况并不多见。在这样的情况下,前提就是不相容的,两个二难推论可用于澄清其中蕴涵的矛盾。

\begin{center}
\fbox{\parbox{0.95\textwidth}{
\textbf{本节要点}
\begin{itemize}
\item \textbf{二难推论}的基本特征:
  \begin{itemize}
  \item 包含析取前提和两个条件前提
  \item 使对手面临两种选择,两者都导致不利的结论
  \item 可分为复杂式(结论是析取命题)和简单式(结论是直言命题)
  \end{itemize}
\item 避开或驳斥二难推论的三种方法:
  \begin{itemize}
  \item \textbf{绕过死角法}:拒斥析取前提,指出存在第三种可能性
  \item \textbf{直击一角法}:否定两个假言前提中的至少一个
  \item \textbf{构造反二难法}:构造另一个结论相反的二难推论
  \end{itemize}
\item 构造反二难法的局限性:
  \begin{itemize}
  \item 往往只是转移注意力而非真正反驳
  \item 原论证和反驳可能结论都为真,只是视角不同
  \end{itemize}
\item 普罗塔哥拉与欧提勒士案例展示了真正相互矛盾的二难推论
\end{itemize}
}}
\end{center}