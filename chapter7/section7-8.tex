\section*{7.8 二难推论(The Dilemma)}
没有什么特别重要的地方。但从修辞角度看,二难推论是一种最有力量的说服工具之一一可谓论战中的一种致命性武器。\\
不严格地说,如果一个人必须在两种选项中做出决断,但两个选项都很糟糕或令人不愉快,那么,我们就说这个人"陷人"了两难(或者说进退维谷)之中。二难推论就是一种旨在使对手陷人这样境地的论证方式。在争论过程中,二难推论使得对手必须做出选择,但无论选择什么,都会得出一个他不能接受的结论。

理查德•费曼(Richard Feynman)是一位著名的物理学家,他在回忆1986年"挑战者"号爆炸的调查时,猛烈地抨击了(美国)国家航空航天局(NASA)的管理失误,他用的就是下面的二难推论:

\begin{displayquote}
我们每次问起高层管理者,他们都会说关于手下发生的事,他们什么都不知道……或者最高领导团确实不知道,这样他们就不知道应该知道的事,或者他们知道,这样他们就在对我们说说。 ${ }^{[7]}$
\end{displayquote}

如此的质问就将对手(此处指的是国家航空航天局的管理者们)推人两难境地,令他们无地自容。其中唯一明确表述的前提是一个析取命题,但析取支必定有一个为真,或者他们知道或者他们不知道手下发生的事。不管选择哪一方,结果对对手来说都是不利的。二难推论的结论本身也可以是一个析取命题(例如,"国家航空航天局的管理者或者不知道他们应该知道的事,或者他们说谎"),此时我们称之为复杂式(complex)二难推论。结论也可以是直言命题,这时就称之为简单式二难推论。

二难推论的结论并非总是令人不愉快的,如下简单式二难推论得出的就是个好结论:

如果天上的神明没有欲求,那么他们就会很满足,如果他们有欲求而能完全实现,那么他们也会很满足。他们或者没有欲求或者能完全实现欲求。总之,他们都会很满足。

二难推论的前提并没有特殊的顺序要求,提供选项的析取前提可前可后。表述选择后果的两个条件命题可以联合表述,也可分开陈述。二难推论常用省略式表述,结论一般都是显而易见的,无须表述出来。有一个例子取

自林肯总统的一封信,他为废止美国南部邦黑奴制度的宣言作了如下辩护:

此宣言如同法律一样,或者有效或者无效。如果无效,就没必要取消。如果有效,就不能取消。任何人都明白。 ${ }^{[8]}$

避开或驳斥二难推论的结论的方法有三种,它们也有各自的名称,都与二难的两个(或多个)"死角"有关。分别称为"绕过(或避开)死角法"、"直击(擒拿)一角法"、"构造反二难法"。它们并非证明二难推论形式无效,而是在不改变推论形式有效性的前提下,寻找避免结论的方法。

绕过死角法是拒斥其析取前提。这是常用的最容易的避开二难的手段。除非析取前提的两个支命题是矛盾关系,否则它们很有可能是假的。常用来说明这个方法的例示是给学生分级打分的例子,有人认为好的分数能激励学生更努力地学习。但学生们想出这样一个二难推论用来驳斥上述理论:

如果学生喜爱学习,那么就不需要激励。如果学生厌烦学习,那么激励也没有用。学生或者是喜爱学习的或者是厌烦学习,所以,激励是不需要的或者没用的。

该论证形式是有效的,但我们能用绕过死角法来反驳这个论证。其析取前提是假的,因为学生会有不同的学习态度:有的喜爱,有的厌烦,还有许多人不同于前两者。对于后面这些人来说,激励既是需要的也是可以发挥作用的。这种方法并不是证明结论为假,只是表明推论本身并没有给结论提供充足的理由。

如果析取前提穷尽了所有可能性,是不可驳倒的,就不能用上述方法了。必须有另外的方法来避开结论,其中之一就是直击一角法,即拒斥两个假言前提中的一个。要否定两假言前提的组合,我们只需否定其中的一个即可。直击一角,就是要试图表明条件前提至少一假。刚才驳斥学校分级打分的例子,所依据的条件前提之一是"如果学生喜爱学习,就不需要激励",反驳者可以争辩说,即使一个学生喜爱学习,也需要激励,好分数会带来额外的奖励,甚至能激励最勤奋的学生更认真地学习。这样一

来,就很可能得到好的回应一一原来的二难的一角就被击破了。\\
构造反二难法是最巧妙的方法,但并不总能令人信服,我们来看这是为什么。用这种方法驳斥给定的二难推论,需要构造另一个二难推论,它的结论与原来的结论相反。辩驳中可以使用任何一个二难推论,但最理想的反二难推论应当与原来的推论有相同的组成成分(直言命题)。

有个古老的例子能说明这种方法,相传雅典有一位母亲劝儿子不要从政时说道:

如果你主持公道,人们就会仇视你。如果你不主持公道,神灵们就会仇视你。你必定或者主持公道或者不主持公道,所以无论如何都会被仇视。

他的儿子反驳说:

如果我主持公道,神灵们就会施爱于我。如果我不主持公道,人们就会施爱于我。我必定或者主持公道或者不主持公道,所以我都会被爱。

在把二难推论作为强力工具的日常论辩中一般人的争论中,这种驳斥方法,从几乎相同的前提得到相反的结论,是种很不错的修辞手法。但如果更细致地研究,就会发现它们的结论并不像初看上去那样对立。

第一个二难推论的结论是儿子会被仇视(被人们或者被神灵们),而反二难的结论是儿子会被爱(被神灵们或被人们)。实际上两者完全是相容的。反驳用的反二难仅仅是建构了一个结论不同的论证而已。两个结论可能都是真的,因而这里并没有达成真正的反驳。但在唇枪舌剑的辩论中,并不需要细致分析,如果在公共争辩中出现这样的反驳,听众大多会把它当做对原论证的毁灭性攻击。

如此反驳并不能驳倒推理,而只是将注意力引向同一事情的不同方面,这从如下的二难推论可能看得更清楚。"乐观主义者"认为:

如果我工作,就能挣钱,如果赋闲在家,那么我乐得自在。我或者工作或者不工作,总之,我能挣钱或者乐得自在。

\section*{而悲观主义者却会给出这样一个反二难:}
如果我工作,就不能乐得自在,如果赋闲在家,就不能挣钱。或者工作或者不工作,总之,我或者不能乐得自在或者不能挣钱。

这些结论只能说明看问题的视角不同,并非对事实状况的意见不一致。通常讲二难推论,都要说到普罗塔哥拉(Protagoras)和欧提勒士 (Euathlus)之间著名的讼案。普罗塔哥拉是生活在公元前 5 世纪的希腊的一名教师,他开设了很多课程,其中最著名的是法庭辩护术,欧提勒士想跟他学习当一名律师,但他负担不起学费。于是两人定了一个契约,普罗塔哥拉先不收学费,等欧提勒士学成并在第一场官司中获胜时,再交学费。可是,欧提勒士学成之后,迟迟没有在法庭上进行辩护,普罗塔哥拉等得不耐烦了,于是把他的学生告上法庭,要求收回学费。欧提勒士忘记了"律师为自己的案子辩护乃属愚行"的格言,决定为自己进行辩护。审理开始后,普罗塔哥拉就用一个压倒性二难推论陈述己方要求:

如果欧提勒士打输了官司,那么他必须还我学费(根据法庭的判决),如果欧提勒士打赢了官司,那么他也必须还我学费 (根据我们之间的契约),或者他打输或者打赢官司,都必须还我学费。

情况看来对欧提勒士而言十分不利,但他已把修辞术学得很好,于是他向法庭提出了如下相反的二难推论:

如果我打赢了官司,我不必交学费(根据法庭的判决),如果我打输了官司,我也不必交学费(根据我们之间的契约),或者我打赢或者打输,都不必交学费。

如果你是法官,该如何判决呢?\\
注意欧提勒士的反二难的结论与普罗塔哥拉的结论的确不相容,一个确实是另一个的否定。这种相反二难推论与原来的二难推论的互相拒斥的情况并不多见。在这样的情况下,前提就是不相容的,两个二难推论可用于澄清其中蕴涵的矛盾。 