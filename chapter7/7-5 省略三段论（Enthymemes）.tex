\section{省略三段论(Enthymemes)}

\begin{quotation}
本节讨论省略三段论,即那些前提和结论未完全陈述、部分需要听者或读者"领会"的不完整论证。我们将学习如何识别、补充和评估这类在日常语言和科学中广泛使用的论证形式。
\end{quotation}

三段论论证很常用,但其前提和结论并不总是都得到明确地陈述。常常只把论证的一部分表述出来,而其余部分就要靠"领会"了。比如,只提及"琼斯是个土生土长的美国人"这个前提,就可以得到结论:"琼斯是美国公民"。上述论证的表述并不完整,但很容易根据美国宪法把省略的前提补出来。加上被省略了的前提,完整的论证就是:

\begin{quote}
所有土生土长的美国人是公民,

琼斯是土生土长的美国人,

所以,琼斯是美国公民。
\end{quote}

完整表述后,这个论证就是一个直言三段论,其形式为 AAA-1,即 Barbara,它完全有效。如果一个推理是不完整的,其中有一部分需要"领会"或仅仅"在心中",我们就称之为\logicterm{省略三段论}。省略(enthymematic)是不完整三段论的特征。

\subsection{省略三段论的原因与特点}

在日常话语甚至科学中,许多推论都是省略式的。原因不难理解,因为有相当一部分命题是公共知识。对于那些广为人知或无关紧要的真命题,听众和读者很容易就能想到并且补充完整,说话者和写作者就不再重复以减少麻烦。另外,用省略式描述推理,能够增加修辞效果,比描述出所有细节更强、更有说服力。亚里士多德在其著作《修辞学》中写道:"基于省略三段论的……演讲更受人欢迎。"

由于省略式不完整,所以要检验其有效性必须找到被省略的部分。如果省去的是一个必不可少的前提,缺了它推论就是无效的。但是如果省略的前提很容易补出来,评估时应该把它包括在论证当中才是公平的。这时,要假定论证者心中所想比明确说出的信息更多些。大多数情况下,很容易将说话人(或写作者)想到而没有表达出来的前提补充完整。举例来说,要说明"银色马"的怪事,神探福尔摩斯构造了这样一个推论,其中就省略了关键性前提,但是很容易猜到:

\begin{quote}
马厩中有一条狗。然而,尽管有人进来,并且把马牵走,狗却没有出声……显然,来者是这只狗非常熟悉的人物。
\end{quote}

我们都能很好地理解其中暗含的意思:如果来者是陌生人,狗就会发出叫声。把这个前提看做福尔摩斯论证的一部分,对作者柯南·道尔来说才是公平的。

\subsection{补充隐藏前提的原则}

补充隐藏的前提时最重要的原则是:说话人确实认为听者可以接受这个命题为真。因此,要是把结论本身当做隐藏的前提就太愚蠢了。如果论证者希望听者把它当做前提而不加证明,那就无须再作为论证的结论来表述。

\subsection{省略三段论的类型}

任何论证都能以省略式表达,但得到最广泛研究的还是三段论的省略式,本节也只限于研究三段论的省略问题。根据未表述部分的不同,传统上把省略三段论分为几种不同的省略体。

\textbf{第一种省略体}指不出现大前提的情形。上面的例子就是这种省略体。

\textbf{第二种省略体}保留大前提和结论,而不出现小前提。例如"所有学生都是反对新规则的,所以,所有大二学生都是反对新规则的",这里的小前提是个明显为真很容易补充出的命题:"所有大二学生是学生"。

\textbf{第三种省略体}中两个前提都出现,但未表述结论。下面就是这个类型的例子:

\begin{quote}
我们的观念超不出我们的经验;我们没有关于神圣的属性与作为的经验;我们用不着为我这个三段论下结论:你自己能得出推论来。\cite{hume1748}
\end{quote}

\subsection{省略三段论的检验}

检验省略三段论的有效性共需两步:首先恢复省略的部分,然后再检验。公正地表示出省去的命题,需要语境敏感性以及对说话者意图的理解。请看这样一个论证:"没有真正的基督徒是精神空虚的,但有些常去教堂礼拜的人是精神空虚的",其中没给出结论,属第三种省略体,那么,原本要得出的结论是什么呢?如果说话者是要得出"有些常去教堂的人不是真正的基督徒",那么,推理就是有效的(EIO-2,Festino);但是如果说话者想说的是"有些真正的基督徒不是常去教堂的人",那么,这个省略式就是无效的(IEO-2),犯了大项不当周延谬误。

但一般说来,语境可以无歧义地确定未表述的命题。例如根据最高法院的意见,控制州内性暴力的联邦立法("针对妇女的暴力行为法案")是违反宪法的,大多数法官的关键性论证如下:

\begin{quote}
在任何意义上,性暴力犯罪都不是经济行为……迄今为止,在美国的历史上最高法院的判例中,只有经济行为才适用控制州内行为的条款。\cite{usc2000}
\end{quote}

可以领会但没有明确表述的结论是:根据最高法院的长期实行的规则,性暴力犯罪不归国会控制。

检验第三种省略体,先要把前提和(显而易见的)结论变形为标准形式。首先陈述大前提(含有结论之谓项的前提),然后确定其式与格,如上例:

大前提:根据最高法院的规则,国会控制的所有行为是经济行为。

小前提:没有州内的性暴力犯罪是经济行为。

结论(并未表述但结合语境却很清楚):没有州内的性暴力犯罪是最高法院规定为国会控制的。

这个三段论的式为AEE,中项在两个前提中都做谓项,因此是第二个格,其形式是 Camestres——有效的三段论。

第三种省略体,在某些情况下,即使不结合语境也能看出是无效的——例如,两个前提都是否定的,或者都是特称的,或者中项不周延。如果这样的话,不可能得出有效的结论。因此,这种省略式在任何语境中都是无效的。

也可能有这样的情况,在省略的是论证的一个前提的情况下,只有加上一个高度不合理的前提,才能把论证写成有效式。此时,指出这一点就构成对省略三段论的一种合理批判(legitimate criticism)。当然,更具毁灭性的批判是:有些三段论无论补上什么样的前提(即使是不合理的前提),也不能成为有效的三段论。

省略三段论与普通三段论的区别,从本质上说是修辞上的,而不是逻辑上的。不需添加什么新规则就能处理省略式,它们终究要接受与标准直言三段论同样的检验。

\begin{center}
\fbox{\parbox{0.95\textwidth}{
\textbf{本节要点}
\begin{itemize}
\item \logicterm{省略三段论}是前提或结论部分未明确表述的不完整论证
\item 省略三段论广泛存在的原因:
  \begin{itemize}
  \item 某些命题为公共知识,无需重复陈述
  \item 增强修辞效果和说服力
  \end{itemize}
\item 省略三段论的三种主要类型:
  \begin{itemize}
  \item \textbf{第一种省略体}:省略大前提
  \item \textbf{第二种省略体}:省略小前提
  \item \textbf{第三种省略体}:省略结论
  \end{itemize}
\item 补充隐藏前提的原则:所补前提应是说话人认为听者能接受为真的命题
\item 检验省略三段论的两个步骤:
  \begin{itemize}
  \item 恢复省略的部分(依据语境和说话者意图)
  \item 对完整三段论进行有效性检验
  \end{itemize}
\item 某些省略三段论无论如何补充都不可能有效,这构成对其最有力的批判
\end{itemize}
}}
\end{center} 