\section*{7.3 直言命题的标准化}
如 7.1 节所述,日常语言中三段论论证的形式可能偏离标准形式,不仅可能出现含有三个以上词项的情况(如 7.2 节讨论的那样),还可能有

这样的情况,即构成命题不都是标准的直言命题。显然,A、E、I、O命题有些生硬,而日常生活中许多三段论都是由非标准的命题组成的。要把这些论证化归为标准形式,就要把构成命题都翻译为标准形式。但日常语言内容丰富、形式多样,根本无法找出一套完善的翻译规则。在各种情形中,最关键的是理解已知的非标准命题的含义,这样才能在翻译时不丢失,也不改变原意。

尽管没有完善的规则,我们仍然可以介绍一些方便的方法,它们在处理某些特殊命题时常常十分有用。这些方法——本节介绍九种方法——只能被看做一种指针而不是规则,也就是说,它们是处理某些特定种类的非标准命题的技巧。

1.单称命题。有些命题肯定或否定的是一个特定的个体或对象属于某个类,例如"苏格拉底是哲学家"、"这张桌子不是古董"等。这样的命题叫做单称命题。它们肯定或否定的不是一个类与另一个类的包含关系 (像标准式直言命题那样),但我们可以把单称命题解释为处理类与类间关系的命题。可以按如下方式做到这一点:

每一个个体对象都对应着一个单元类(由一个元素组成的类),其中只有一个对象。这样,断定一个对象 $s$ 属于类 $P$ ,在逻辑上等价于断定了只含有一个元素的单元集 $S$ 完全包含于类 $P$ 之中。而断定一个对象 $s$ 不属于类 $P$ ,在逻辑上等价于断定只含有一个元素的单元类 $S$ 完全排斥在类 $P$之外。通常将这种解释看做自然而然的,无须调整记法。据此,我们就可以将任何一个单称肯定命题"$s$ 是 $P$"看做逻辑上等价的 A 命题"所有 $S$是 $P$"。同样,可以简单地将单称否定命题"$s$ 不是 $P$"看做逻辑上等价的 E 命题"没有 $S$ 是 $P$"——S 指称的都是只有一个对象 $s$ 的单元类。因此,不需要对单称命题进行明确的翻译,一般把它们分别归到 $\mathrm{A} 、 \mathrm{E}$ 命题当中。康德说过"在三段论中判断之使用,逻辑学者把单称判断类如全称判断处理,是很恰当的"${ }^{[1]}$ 。

然而,情况并不那么简单。特称命题有存在含义,而全称命题没有。在布尔解释下(如5.6节说明),如果机械地把单称命题当做三段论推理的 A、E 命题;再用文恩图或三段论规则来检验其有效性,就会出现严重的困难。

很明显,在某些情况中,可以把含单称命题的有效的两前提论证转化为有效的三段论。例如:

$$
\begin{array}{ll}
\text { 所有 } H \text { 是 } M, & \text { 可以变为三段论的 Barba- } \\
\frac{s \text { 是 } H,}{\therefore s \text { 是 } M_{0}} & \text { ra, 即 AAA-1 式, 很明显 } \\
\text { 是有效的 } &
\end{array}
$$

$$
\begin{aligned}
& \text { 所有 } H \text { 是 } M, \\
& \text { 所有 } S \text { 是 } H, \\
& \therefore \text { 所有 } S \text { 是 } M \text { 。 }
\end{aligned}
$$

但在另外的某些情形下,把含单称命题的有效的两前提论证转化为三段论,却是明显无效的。例如:

$$
\begin{array}{ll}
s \text { 是 } M, & \text { 得到的直言三段论是无效 } \\
\frac{s \text { 是 } H,}{\therefore \text { 有 } H \text { 是 } M_{0}} & \text { 的 AAI-3 式 }
\end{array}
$$

后者违反了规则 6 ,犯了存在谬误。\\
再者,如果把单称命题转化为特称命题,也会有同样的困难。有些情况下转化是有效的,例如:

\begin{center}
\begin{tabular}{lll}
所有 $H$ 是 $M$, & 可以变为三段论的 Darii, & 所有 $H$ 是 $M$, \\
$\frac{s \text { 是 } H,}{\therefore s \text { 是 } M \text { 。 }}$ & 即 AII-1 式,很明显是有效的 & 有 $S$ 是 $H$, \\
$\therefore$ 有 $S$ 是 $M$ 。 &  &  \\
\end{tabular}
\end{center}

但在另一些情况中,这种翻译却会得出明显无效的直言三段论。例如:

\begin{center}
\begin{tabular}{lll}
$s$ 是 $M$, & 得到的直言三段论是无效的 & 有 $S$ 是 $M$, \\
$\frac{s \text { 是 } H,}{\therefore}$ 有 $H$ 是 $M$ 。 & 有 $S$ 是 $H$, &  \\
$\therefore$ & 有 $H$ 是 $M$ 。 &  \\
\end{tabular}
\end{center}

后者违反了规则 2 ,犯了中项不周延谬误。\\
问题来自如下事实:单称命题要比任何一个标准式命题负载更多信息。如果把"$s$ 是 $P$"当做"所有 $S$ 是 $P$",那么,就丢掉了单称命题的存在含义,实际上这里 $S$ 非空。而如果把"$s$ 是 $P$"当做"有 $S$ 是 $P$",又漏掉了单称命题的全称性,即主项周延,它说的是全部 $S$ 是 $P$ 。

解决此问题的办法,就是把单称命题分析为两个直言命题的合取,即

一个单称肯定命题等价于相互关联着的 A、I 命题的合取。这样"$s$ 是 $P$"就等价于"所有 $S$ 是 $P$"合取"有 $S$ 是 $P$",单称否定命题则等价于"没有 $S$ 是 $P$"合取"有 $S$ 不是 $P$"。图7—1就是单称命题的肯定式和否定式的文恩图。在用三段论规则评估这种推理时,必须考虑它提供的所有信息,既考虑周延性也考虑存在含义。\\
\includegraphics[max width=\textwidth, center]{2025_05_15_6a28331d5e7c993ad07ag-310}

图7—1\\
对于含有单称命题的三段论,引用文恩图或规则检验其有效性时,只要我们记住其中有存在含义,就可以直接把它们看做全称(A 或 E )命题。

2.谓项为形容词或形容词短语,而非名词或类词项的直言命题。例如"有花是美的"、"没有战船是可调用的"都是直言命题,但"美的"和 "可调用的"表示的只是属性而不是类,所以它们的形式不标准,必须转化为标准形式。不过,每个属性都可以确定一个类,即具有这种属性的事物组成的类,所以对于每个这样的命题,都有一个相应的标准式直言命题。两个例句分别对应的是:I命题"有花是美的事物"和 E 命题"没有战船是可调用的事物"。如果一个直言命题的形式是标准的,只有谓项为形容词或形容词短语时,就把形容词或短语替换为这样一个词项,它指称由所有具有形容词表示之属性的事物所组成的类。

3.主要动词不是标准的联项"是"或"不是"的直言命题。常见的例子有"所有人都寻求赞誉"、"有人饮用希腊酒"。通常,转化的方法是把主项和量项之外的所有成分看做类的定义特征。先把能被替换的成分换成这样的词项,它们指称由类定义特征所确定的类,再改用标准的联项把它们同主项联结起来。这样上面两个例子就成了:"所有人是赞誉的寻求者"、"有人是希腊酒的饮用者"。

4.标准形式的各成分都出现,却没有按标准顺序排列的陈述句。"赛马全是良种马"和"结果好的事总是好事"就是这样的例子。在这种情形

下,首先要找出哪个是主项,然后再重新把各个成分排列一下,使之成为标准式直言命题。这种翻译通常都很直接。十分清楚,上述两个例句可翻译为"所有赛马是良种马"和"所有结果好的事是好事"。

5.量词不是"所有"、"没有"和"有"这些标准语词的直言命题。以 "每一"、"任何"等开头的陈述句很好转化。"每一只狗都有其得意之时"、 "任何贡献都会得到赞赏"可分别转化为"所有狗是有其得意之时的动物"和"所有贡献是会得到赞赏的事情"。"每一事物"、"任何东西"类似于 "每一"、"任一"。与此同一系列但限于人类的是"每人"、"任何人"、"无论谁"、"不管是谁"、"那些……的人"以及"每个......的人"等等。以上各表达式都不会带来什么麻烦。

语法冠词"a"和"an"("一个"等)也可用于指代量词,必须依据当时的语境,确定它们的意思是"所有"还是"有"。例如,"A bat is a mammal"(一只蝙蝠是一个哺乳动物)与"An elephant is a pachyderm" (一头大象是一个厚皮动物)可以合理地解释为"所有蝙蝠都是哺乳动物"与"所有大象都是厚皮动物"。但"A bat flew in the window"(一只蝙蝠飞进窗户)和"An elephant escaped"(一头大象逃跑了)显然指的是 "有蝙蝠是飞进窗户的动物"和"有大象是逃跑的动物"。

冠词"the"("这"、"这些"等)既可以用于指称一个特定的个体,也可以指称一个类的全部元素,有可能引起混淆。例如"The whale is a mammal"(鲸是哺乳动物)这句话,在一般情况下都会被理解为"所有鲸都是哺乳动物",而单称命题"The first president was a military hero" (第一任总统是军旅英雄)可以说是标准形式的 A 命题(一个有存在含义的单称命题),其道理本节前面已经讨论过了。 ${ }^{[2]}$

尽管以"每一"和"任一"开头的肯定句都可以译为"所有 $S$ 是 $P$",但对于以"not every"(并非每一个)和"not any"(并非任一)开头的否定句,却有很大区别。它们的译法不那么明确,需要更加小心。比如, "Not every $S$ is $P$"意思是有 $S$ 不是 $P$ ,而"Not any $S$ is $P$"意思是没有 $S$ 是 $P$ 。

6.排斥命题(exclusive propositions)。含有"只"(only)、"只有" (none but)的直言命题通常叫做排斥命题,因为一般说来,它们断言的是谓项排他性地适用于主项。例如"只有公民能成为选民"、"只有勇敢者是值得公平对待的",第一句转化为标准形式是"所有能成为选民的是公

民",第二句转化为"所有值得公平对待的人是勇敢者"。以"只"、"只有"开头的命题一般可以按以下途径转化为 A 命题:将主、谓项互换位置,把"只有"换为"所有"。因此"只有 $S$ 是 $P$"和"只有 $S$'s 是 $P$'$s$"通常被理解为"所有 $P$ 是 $S$"。

但是,在某些语境中,"只"、"只有"被用于表达某种更多的含义。 "只有 $S$ 是 $P$"和"只有 $S$'s 是 $P$'$s$"表明的可能是"所有 $S$ 是 $P$"或者 "有 $S$ 是 $P$"。但这种情况并不常见。这个时候就需要语境的辅助了。如果没有附加信息,前面的翻译就是适当的。

7.不含量词的直言命题。例如"狗是肉食动物"、"孩子在场"。欠缺量词,句子的含义就不十分明确。只有考察它们所处的语境才能确定其含义,一般来说,考察之后就能把疑义清理掉。第一个例句"狗是肉食动物"很可能述及了所有的狗,可以转化为"所有狗都是肉食动物"。而第二个例句一般只述及某些孩子,转化为标准形式为"有孩子是在场的人"。

8.完全不像标准式直言命题但也可以有标准式翻版的命题。例如 "不是所有孩子都相信圣诞老人"、"有白色的大象"、"没有粉色的大象"以及"没有既圆又方的东西"。反思这些命题就会发现,它们在逻辑上等价于(因而可翻译为)下面的标准式命题:"有孩子不是相信圣诞老人的人"、"有大象是白色的事物"、"没有大象是粉色的事物"和"没有圆的东西是方的"。

9.除外命题(exceptive propositions)。还有一些这样的例子:"除了雇员(all except)都是合格的"、"雇员之外的人(all but)都是合格的"与"只有(alone)雇员不是合格的"。要把这样的除外命题翻译为标准形式,情况就会复杂一些,因为这种命题(与单称命题很类似)做出了两个而不是一个方面的断定。所给例子断言的不仅是所有非雇员是合格的,还断定了(在通常的语境中)没有雇员是合格的。如果把"雇员"记为 $S$ 、 "合格的人"记为 $P$ ,那么,这两个命题可以写成"所有非 $S$ 是 $P$"和 "没有 $S$ 是 $P$"。这两个命题是独立的,但联合起来就断定了 $S$ 和 $P$ 互为补类。

每个除外命题都是复合句,因此,不能转化为单一的标准式直言命题。确切地说,每一个除外命题应当翻译为一个合取式,即两个标准式直言命题的合取式。所以,上面关于合格性的三个例句都可以翻译为"所有非雇员是合格者,并且没有雇员是合格者"。

应该注意到,有些论证的有效性离不开数字或类数字(quasi-numeri- cal),但数字无法译为标准形式。这些推理本身就是非三段论的(asyllo- gistic)。因此,对它们进行分析就需要一种比直言三段论复杂一些的理论。当然,有些含有类数字量词的推理也可以用三段论分析。"几乎所有"、"并非全部"、"除少数几个之外都"、"几乎每个人"等就是这样的词。如果一个命题含有看起来像量词的词项,那么就可以处理为刚刚讲过的除外命题。下面几个除外命题都含有类数字:"几乎所有学生都参加了舞会"、"并非所有学生都参加了舞会"、"除少数几个之外,学生们都参加了舞会"和"只有一些学生参加了舞会",它们都肯定了有些学生参加了舞会,同时又否定了所有学生都参加了舞会。从三段论推论的观点看,它们给出的类数字信息并不相干,转化之后都是"有学生是参加了舞会的人,并且有学生不是参加了舞会的人"。

由于除外命题不是直言命题,而是合取式,含有这些命题的论证并不是我们所说的三段论论证。但是,对它们进行三段论分析和评估也未尝不可。含有除外命题的论证,要依据该命题所处的位置来进行检验。如果它是前提,那么就要分两次进行检验。举例来说,看下面这个论证:

\begin{displayquote}
每个看过比赛的人都参加了舞会,\\
不是全体学生都参加了舞会,\\
所以,有学生没有看过比赛。
\end{displayquote}

其中,第一个前提以及结论都是直言命题,很容易译为标准形式。但第二个前提是一个除外命题,不是简单句而是复合句。要检查前提是否蕴涵结论,首先要检验由论证的第一个前提、第二个前提的前一半以及结论组成的三段论。我们有:

\begin{displayquote}
所有看过比赛的人都是参加了舞会的人,\\
有学生是参加了舞会的人,\\
所以,有学生不是看过比赛的人。
\end{displayquote}

这个标准式的直言三段论是 AIO-2,违反了规则 2,犯了中项不周延的谬

误。但不能由此就得出结论说原来的论证是无效的,因为受检验的三段论只包含它的一部分前提。现在再来检验由第一个前提、第二个前提的后一半以及结论组成的三段论。译为标准形式后,得到一个非常不同的三段论:

所有看过比赛的人都是参加了舞会的人,\\
有学生不是参加了舞会的人,\\
所以,有学生不是看过比赛的人。

这是一个标准的 Baroko,即三段论的 AOO-2。很容易看出它是有效的。原来的三段论与这个有效式的结论相同,并且前者的前提包含着后者的前提,所以原来的论证也是有效的。因此,如果一个论证中有一个前提是除外命题,那么,对其有效性的检验要分为两次,即分别对两个不同的标准式直言三段论进行检验。

如果前提都是直言命题,但结论是除外命题,那么我们就可断言它是无效的。尽管两个直言命题可以蕴涵其中一个,即蕴涵结论复合句的一半,但不可能同时蕴涵两个。最后,如果两个前提和结论都是除外命题的话,那么,由原来论证所能建构的任何一个可能的三段论都要接受检验,才能确定其有效性。以上解释已足够处理这种情况了。

学会将多种非标准命题翻译为标准形式的技巧是很重要的,因为我们已经掌握的检验方法——文恩图解和三段论规则——只能直接用于标准式直言三段论。 