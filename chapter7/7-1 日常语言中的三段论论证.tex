\section{日常语言中的三段论论证}

\begin{logicbox}[title=引言]
日常语言中的三段论论证往往不像\logicterm{标准式三段论}那样整齐规范。本节将介绍如何识别和转换日常语言中的三段论论证,使其符合标准形式,便于进行\logicemph{有效性}检验。
\end{logicbox}

前几章考察的\logicterm{标准式直言三段论}往往显得生硬、不自然。它们就像 "化学纯净物"一样,不含任何杂质和不相关的东西。但是,日常语言论证并不总是这么整齐划一地出现的。在此,我们更广义的使用\logicterm{三段论}这一术语,用来指谓符合如下条件的任一论证:或者本来就是\logicterm{标准式直言三段论},或者是可以变形为\logicterm{标准式直言三段论}而没有失掉或改变原意的论证。

\begin{theorembox}[title=日常语言三段论的检验方法]
三段论论证相当常见,所以我们要设法检验其\logicemph{有效性}。但由于日常论证通常比标准形式松散,前面提到的检验方法——\logicterm{文恩图}和直言三段论的规则一一不能直接适用于它们。日常的三段论论证形式变化多样,不可能为每一种形式都发明一个特殊的检验方法,除非有一种极度复杂的逻辑工具。要检验众多三段论论证的\logicemph{有效性},最明智的方法通常是:在不改变原意的前提下,把它们\logicterm{变形}(reformulate)为\logicterm{标准式三段论}。这个方法就是向标准形式的\logicterm{化归}(reduction)或\logicterm{翻译}(translation),最后得到的三段论叫做原给定三段论的\logicterm{标准式翻版}。
\end{theorembox}

评估日常语言三段论要满足两个条件。首先,要有一种便于应用的检验方法,将\logicterm{标准式三段论}的\logicemph{有效式}和\logicwarn{无效式}区分开来,这种方法我们已经有了(前面章节中讲到的图示和规则)。其次,要有一种\logicterm{翻译}方法,将任何形式的三段论推理转变为标准形式,一旦掌握了这种方法,再用先前介绍的判定\logicemph{有效}三段论的规则或\logicterm{文恩图解}方法进行检验,我们就能评估任何三段论。

\subsection{非标准形式的三种偏离情形}

\begin{examplebox}[title=非标准形式的三种偏离情形]
要说明将日常语言中的非标准三段论论证\logicterm{翻译}为标准形式的方法,首先要区分非标准形式偏离标准形式的不同情形。下面是三种基本的偏离情形:

1.前提和结论的顺序不标准。这是小问题,因为如果仅仅是叙述的顺序不标准,很容易调整过来。

2.日常语言论证的构成命题中表面上包含不止三个项,但可以证明事实上并非如此。

3.日常语言论证的构成命题不都是\logicterm{标准式直言命题}。

第二、三种偏离情形同样有可能\logicterm{翻译}为标准形式,下面即讨论翻译方法。
\end{examplebox}

\begin{center}
\fbox{\parbox{0.95\textwidth}{
\textbf{本节要点}
\begin{itemize}
\item \logicterm{日常语言三段论}是可变形为\logicterm{标准式直言三段论}而不失去或改变原意的论证
\item 检验日常三段论\logicemph{有效性}的两个条件:
  \begin{itemize}
  \item 有将\logicterm{标准式三段论}区分\logicemph{有效}与\logicwarn{无效}的检验方法
  \item 有将任何形式的三段论推理转变为标准形式的\logicterm{翻译}方法
  \end{itemize}
\item 日常语言三段论偏离标准形式的三种情形:
  \begin{itemize}
  \item 前提和结论顺序不标准
  \item 表面上包含超过三个词项
  \item 构成命题不都是\logicterm{标准式直言命题}
  \end{itemize}
\end{itemize}
}}
\end{center}