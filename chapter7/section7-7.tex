\section*{7.7 析取三段论和假言三段论}
一个三段论就是包含两个前提的演绎论证。可以有许多不同种类的三段论,最常见、最重要的就是直言三段论,前面几章已经对它进行了详细的考察。本节将简要讨论其他几种三段论。

\section*{A.析取三段论(Disjunctive Syllogisms)}
在一个析取(或"选言")三段论中,有一个前提是析取命题。例如:

或者傻瓜,或者无赖,\\
他不是傻瓜,\\
所以,他是个无赖。

这样的论证是有效的。 ${ }^{[1]}$ 传统上把这种论证形式称为"通过否定进行肯定"(modus tollendo ponens),即通过否定其中一个选言支来肯定另一个选言支。有效的析取三段论定义如下:

一个析取三段论是有效的,当且仅当其中一个前提是析取命题,另一个前提是对其中一个选言支的否定,而结论是未被否定的那个选言支。

这里讨论的析取命题指的是"弱的"或"可兼的"析取(见8.1节),意思是它断言的是至少有一个选言支为真,但可能两个都真。如果析取三段论的前提是一个强的(或"不可兼的")析取命题,即断言恰好只有一个选言支为真,另一个为假,那么,下面的论证形式也是有效的:

或者傻瓜,或者无赖,\\
他是个傻瓜,\\
所以,他不是无赖。

传统上把这种论证形式称为"通过肯定进行否定"(modus ponendo tollens),即通过肯定其中一个选言支来否定另一个选言支。

\footnotetext{(1)在第 8 章,我们会看到,这种形式以及其他形式的论证,都很容易用真值表加以验证。
}
大多数情况下,不通过语境就无法确定析取命题到底是强的还是弱的。但是,对于有效的析取三段论的定义,并不需要这种区分。这个定义对于两种析取命题都适用。

如果一个论证具有析取三段论的形式,但并不符合这里的定义,那么,这种论证就是无效的,它所犯的错误传统上称之为肯定选言支谬误。这种谬误往往出现在弱析取命题的论证中,比如"或者傻瓜,或者无赖。他是个傻瓜,所以他不是无赖"。如果"或者傻瓜,或者无赖"被解释为弱析取,那么,即使两个前提都真,结论也可能为假。因此,这样的论证是无效的。

\section*{B.假言三段论(Hypothetical Syllogisms)}
标准形式的假言三段论包含两个假言命题作为前提,外加一个假言命题作为结论。例如:

如果第一个土著人是政客,那么他会说谎,\\
如果他说谎,那么他会否认自己是政客,\\
所以,如果第一个土著人是政客,那么他会否认自己是政客。

这个论证叫做纯粹假言三段论,因为它的前提和结论都是假言命题。这种论证是有效的。 ${ }^{[1]}$

还有一种常见形式的假言三段论,它有一个假言前提,一个直言前提,以及一个直言结论。有两种这样的有效形式,在此要分别进行讨论。

第一种有效的形式,传统上称为肯定前件(modus ponens),例如:

如果第二个土著人说真话,那么只有他是政客,\\
第二个土著人说真话,\\
所以,只有他是政客。

这里,第一个前提是假言命题,第二个前提肯定了第一个前提的前件,而结论则肯定了第一个前提的后件。这种形式的任何论证都是有效的。

但是,如果一个论证具有与肯定前件相似的形式,但却不是有效的,这种错误就称为肯定后件谬误。例如:

\footnotetext{(1)同样,在第 8 章会看到,用真值表可以很容易证明这种形式的有效性。
}
如果培根写了《哈姆雷特》,那么培根是个大作家,\\
培根是个大作家,\\
所以,培根写了《哈姆雷特》。

显然,这个论证是无效的。第二个前提肯定了第一个前提的后件,而结论肯定了第一个前提的前件。

第二种有效的假言三段论,传统上称为否定后件(modus tollens),例如:

如果这位客人是陌生人,那么狗会叫,\\
狗没有叫,\\
所以,这位客人不是陌生人。

这里,第一个前提是假言命题,第二个前提否定了第一个前提的后件,而结论否定了第一个前提的前件。这种形式的任何论证都是有效的。

但是,如果一个论证具有与否定后件相似的形式,但却不是有效的,这种错误就称为否定前件谬误。例如:

如果洛克菲勒拥有福特汽车公司的全部黄金,那么洛克菲勒很富有,\\
洛克菲勒并不拥有福特汽车公司的全部黄金,\\
所以,洛克菲勒并不富有。

显然,这个论证是无效的。第二个前提否定了第一个前提的前件,而结论否定了第一个前提的后件。 