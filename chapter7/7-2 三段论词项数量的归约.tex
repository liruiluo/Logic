\section{三段论词项数量的归约}

\begin{logicbox}[title=引言]
日常语言中的三段论论证常常看似包含超过三个词项,但通过适当的转换可以归约为标准的三词项形式。本节介绍两种主要的归约方法:去除同义词和处理补类,这些方法帮助我们将复杂的日常语言论证化简为可用标准规则检验的形式。
\end{logicbox}

\logicwarn{如果日常语言中的一个论证看起来有三段论的形式,但包含着三个以上的词项,那么不应该即刻把它看成犯了\logicwarn{四项谬误},从而认为它是\logicwarn{无效的}。}这样的论证往往能被\logicterm{翻译}为与之逻辑上等价的只有三个词项且完全\logicemph{有效的}标准形式三段论。完成这样的\logicterm{翻译}要掌握两种方法:

\subsection{去除同义词}
\begin{examplebox}[title=去除同义词的方法]
在应用\logicterm{文恩图}或三段论规则之前,应当去除日常语言论证中的同义词。举例来说,这样一个论证:

没有富人(wealthy)是游民(vagrant),

所有律师(lawyer)都是有钱人(rich people),

所以,没有法律代理人(attorney)是流浪者(tramps)。

其中包含着"富人"、"律师"和"游民"的同义词。去除同义词之后,该论证可\logicterm{翻译}为:

没有富人是游民,

所有律师都是富人,

所以,没有律师是游民。

这个三段论是标准的 EAE-1(Celarent),很明显是\logicemph{有效的}。
\end{examplebox}

\subsection{去除补类}
\begin{examplebox}[title=去除补类的方法]
有时仅仅去除同义词是不够的。来看下面这个论证,其中所有命题都是\logicterm{标准式直言命题}:

所有哺乳动物是温血动物,

没有蜥蜴是温血动物,

所以,所有蜥蜴都是非哺乳动物。

如果直接用第6章给出的三段论规则来检验,这个三段论违反了不止一个规则。一方面,它包含着四个词项:"哺乳动物"、"温血动物"、"蜥蜴"和"非哺乳动物"。另一方面,它从否定前提得到了一个肯定结论。但实际上这个推理是\logicemph{有效的}。因为其中虽含有四个词项,但不是标准形式,不能直接用三段论规则检验。要想用第6章给出的几个规则来检验,必须首先把它\logicterm{翻译}为标准形式。这是很容易的,因为四个词项中有两个("哺乳动物"和"非哺乳动物")互为\logicterm{补类}。如果将结论进行\logicterm{换质},就可以减少词项的数量——翻译的结果是原论证的一个标准式翻版:

所有哺乳动物是温血动物,

没有蜥蜴是温血动物,

所以,没有蜥蜴是哺乳动物。

它与原来论证的前提相同而结论等价,所以两者在逻辑上是等价的。这个标准式翻版遵守所有规则因而是\logicemph{有效的}。其形式为 AEE-2(Camestres)。
\end{examplebox}

尽管后者是最容易得到的,但它并不是唯一的标准式翻版。还可以对第一个前提进行\logicterm{换位}、对第二个前提进行换质,而不改变结论,就可以得到另一个不同(但逻辑上等价的)标准式翻版:

\begin{quote}
所有非温血动物是非哺乳动物,

所有蜥蜴是非温血动物,

所以,所有蜥蜴都是非哺乳动物。
\end{quote}

这是一个 AAA-1(Barbara),也是遵守规则的有效式。对给定的三段论论证进行翻译,并没有唯一固定的标准形式,但如果其中一个是有效的,那么其他所有翻版都应该是有效的。

如果四个词项中有两个互为补类,那么任何含有这样的四个词项的三段论都可以化归为标准形式(或逻辑上等价的标准直言三段论);如果其中两个(或三个)与另外两个(或三个)互为补类,那么任何含有五个(或六个)词项的三段论也都可以化归为标准形式。这种化归都是通过\textbf{换位法}、\textbf{换质法}、\textbf{换质位法}等\textbf{直接推论}而实现的,这些方法在5.5节都讲过。

\subsection{多重词项的归约}
一个三段论论证,其构成命题如果都是标准式直言命题,它有可能含有半打不同的词项,要把它化归为标准形式,进行一次直接推论是不够的。下面的例子就是一个六词项的三段论,但它的确是有效的:

\begin{quote}
没有非居民是公民,

所有非公民是非选举人,

所以,所有选举人都是居民。
\end{quote}

可以用两种方法进行化归,第一种方法需要用到直接推论的三种方法,或许这是最自然也最明显的方法。首先把第一个前提换位再换质,把第二个前提换质位,于是得到如下一个标准式直言三段论:

\begin{quote}
所有公民都是居民,

所有选举人都是公民,

所以,所有选举人都是居民。
\end{quote}

这也是一个 Barbara 式,用第6章阐明的任何一种方法都很容易证明它是有效的。

\begin{center}
\fbox{\parbox{0.95\textwidth}{
\textbf{本节要点}
\begin{itemize}
\item 日常语言论证看似包含超过三个词项时,可通过两种方法归约:
  \begin{itemize}
  \item \textbf{去除同义词}:识别并合并表达相同概念的不同词项
  \item \textbf{处理补类}:通过换质、换位等方法消除互为补类的词项
  \end{itemize}
\item 归约方法基于\textbf{直接推论}技术:
  \begin{itemize}
  \item 换位法:调换主谓项位置
  \item 换质法:改变命题的质(肯定变否定或否定变肯定)
  \item 换质位法:同时改变质和主谓项位置
  \end{itemize}
\item 一个论证可能有多个不同但逻辑等价的标准式翻版
\item 甚至包含五个或六个词项的论证,通过适当的归约也能检验其有效性
\end{itemize}
}}
\end{center}