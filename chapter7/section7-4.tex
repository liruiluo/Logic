\section*{7.4 协同翻译}
要对三段论论证进行有效性检验,其中总共只能包含三个项。有时做到这一点很难,需要比前面所述方法更细致的处理。请考虑命题"你总是与穷人为伍",显然,它既不是断言所有穷人总是在你身边,也不是说有些(特称的)穷人总是(always)在你身边。把该命题化归为标准形式的一种方法,也是一种最自然的方法,就是从其中的关键词"总是"着手分析。这个词意味着"在所有时间"(at all time),它表明原命题的一种标准式翻版为"所有时间都是你与穷人为伍的时间"。主、谓项中都出现的 "时间" 这个词可视为一个参项(parameter)。所谓参项,就是一个有助于以标准形式表达原来断言的辅助词项。

当然,绝不能机械地、不加思考地引入和使用参项,必须始终以所要翻译的那个命题为依据。命题"史密斯总是在台球比赛中获胜",显然断定的并不是史密斯从不间断地、始终在获胜!较合理的解释是,这句话是说:每当史密斯玩台球时,他就会获胜。如果这么理解,就可以直接把原句转化为"所有史密斯玩台球的时间是他获胜的时间"。并非所有参项都是时间性的。在对另一些命题进行翻译时,"地点"(place)、"情形"(case)也能被用做参项。例如"没有幻想的地方人类就会毁灭"(Where there is no vi- sion the people perish)和"每当琼斯迟到就丢失一次推销机会"(Jones lo- ses the sale whenever he is late)可分别译为:"所有没有幻想的地方都是人类毁灭的地方"和"所有琼斯迟到的情形都是他丢失推销机会的情形"。

在对三段论的三个构成命题进行协同翻译的过程中,参项的引人是必不可少的。一个直言三段论恰好包含三个项,要检验三段论就必须把它的构成命题都转化为标准式直言命题,其中只出现三个项。去除同义词以及换位法、换质法、换质位法的运用已经在 7.2 节讨论过。即使这样,还有很多三段论论证的项数仍然不能被缩减到三。此时,协同翻译就需要把同一个参项引到三个构成命题中去。请看下面这个论证:

\begin{displayquote}
哪里脏纸盒散落哪里就曾有不自爱者在此野餐,这里散落着脏纸盒,
\end{displayquote}

所以,一定有不自爱者在这里野餐过。

这个推理是完全有效的,但只有把前提和结论都翻译为标准式直言命题,并且其中只能有三个项,才能用文恩图或三段论规则来证明其有效性。第二个前提和结论能很自然地译为"有脏纸盒是散落在这里的东西"和"有不自爱者是在这里野餐过的人"。但这两个陈述句当中有四个不同的项。要把给定的论证化归为标准形式,就需要在三个命题中使用同一个参项。我们从第一个前提着手寻找这个参项,然后再用同样的参项去翻译第二个前提以及结论。"哪里"一词表明可以用"地方"作为参项。如果翻译三个命题时都用这个参项,则论证可变为:

所有散落脏纸盒的地方是不自爱者野餐过的地方,这个地方是散落脏纸盒的地方,所以,这个地方是不自爱者野餐过的地方。

这个标准式直言三段论的形式为 AAA-1,即 Barbara,是一个已经被证明有效的形式。

利用参项使表达式标准化的方法不是很容易掌握的,但有些三段论论证的确无法用其他方法进行翻译。再看一个例子有助于弄清其中的技巧:

每当狐狸经过那里,猎犬一定会发出叫声,所以,狐狸走的一定是别的路,因为猎犬都很安静。

首先,我们必须明白上述论证说的是什么。要把"猎犬很安静"这句话理解为"猎犬此时此地没有发出叫声"。这一步是去除同义词的必需步骤,因为第一个命题说的是"猎犬发出叫声"。同样,"狐狸走的一定是别的路"的结论,应理解为断言"狐狸没有经过那里"。第一个前提中的"那里"一词表明翻译时也可用"地方"做参项。于是,可得到这样一个标准式翻版:

$$
\begin{aligned}
& \text { 所有狐狸经过的地方是猎犬发出叫声的地方, } \\
& \text { 这个地方不是猎犬发出叫声的地方, } \\
& \text { 所以, 这个地方不是狐狸经过的地方。 }
\end{aligned}
$$

这个标准式直言三段论的形式为 AEE-2,即 Camestres,其有效性很容易确定。 