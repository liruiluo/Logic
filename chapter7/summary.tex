\section{第7章概要}
本章考察日常语言中的三段论论证。我们看到标准式三段论的理论如何应用于这些论证。

日常语言中的三段论论证是逻辑学重要的研究对象,因为它们广泛存在于我们的日常交流、论辩和科学研究中。然而,这些论证往往不像教科书中的标准式三段论那样整齐规范。本章介绍了一系列方法,帮助我们识别、翻译和评估这些非标准的三段论论证。

7.1 节指出,日常语言论证很少以标准形式出现。要把它们翻译为标准形式,需要理解它们的含义,而不能仅仅依赖表面的语法形式。这种翻译过程需要对原始论证的真实意图有准确理解。

7.2 节解释如何把一个表面上有三至六个词项的论证归约为只有三个词项的标准式三段论。这需要(1)去除同义词,以及(2)对某些词项换质以处理其补类。这种词项归约对于识别和评估看似复杂的论证非常关键。

7.3 节提出九种有用的方法,用以处理那些构成命题不是标准式的三段论论证。这些方法包括:

1.单称命题,如"苏格拉底是哲学家",可当做全称命题(A 或 E)对待,但需注意其存在含义。

2.如果命题的谓项是形容词或形容词短语,可把它们替换为指称相应类的词项,使命题符合标准形式。

3.如果命题的主要动词不是标准联项"是"或"不是",可把动词及其他词语(主项与量项之外)看做类的定义特征,从而把原命题改写为标准式。

4.如果命题的各成分虽已出现但顺序不标准,找出主项,重新调整各成分的顺序,使之符合标准形式。

5.处理非标准量词时,通常要把它们替换为"所有"、"没有"或"有"。要把"并非每个……"翻译为"有……不是……"。

6.排斥命题,如"只有公民是选民",一般要通过颠倒主、谓项位置翻译为 A 命题。结论通常是"所有选民是公民"。

7.不含量词的命题,要依据语境把量词"所有"或"有"补充完整,明确其逻辑含义。

8.有些命题的表达形式完全不像标准式直言命题,但其逻辑上等价的直言命题可以明确地表述出来。

9.除外命题,如"除了雇员之外所有人都合格",不是简单的直言命题,而要翻译为两个标准式直言命题的合取。

7.4 节说明并举例解释了协同翻译的方法,即有时需要把同一个辅助词项(参项)引入三段论的三个构成命题中,以便把整个论证翻译为标准形式。这种方法对于处理包含时间、地点、情境等关系的日常语言论证特别有用。

7.5 节考察省略三段论,即前提或结论未明确表述出来的三段论。我们看到,在对省略三段论进行检验之前,如何发现并明确表述出未出现的命题。这种省略形式在日常论证中极为常见,理解它们需要敏感于语境和说话者意图。

7.6 节解释并举例说明连锁三段论。连锁三段论包含多个前提,形成一系列环环相扣的推理链条。检验这类论证的有效性,需要将其拆解为一系列标准三段论,并分别验证每个环节。

7.7 节解释了析取三段论和假言三段论,指出了它们的有效形式和可能产生的谬误。这两种三段论形式在日常推理中广泛应用,了解它们的有效形式和常见谬误对于批判性思维至关重要。

7.8 节讨论二难推论的各种形式,并说明了反驳二难推论的三种方法:绕过死角法(拒斥析取前提)、直击一角法(否定条件前提)以及构造反二难法(建立相反结论的二难推论)。二难推论是强有力的修辞工具,了解其结构和应对方法有助于我们在论辩中保持清晰思考。

总之,本章提供了一套全面的工具和方法,帮助我们将复杂多样的日常语言论证翻译为可以用标准规则检验的形式。这些技能不仅对于学术研究有价值,也对提高我们在日常生活中分析和评估论证的能力大有裨益。 

\printbibliography[heading=subbibliography,title={第7章参考文献}] 