\section{第7章概要}
本章考察日常语言中的三段论论证。我们看到标准式三段论的理论如何应用于这些论证。

7.1 节指出,日常语言论证很少以标准形式出现。要把它们翻译为标准形式,需要理解它们的含义。

7.2 节解释如何把一个表面上有三至六个词项的论证归约为只有三个词项的标准式三段论。这需要(1)去除同义词,以及(2)对某些词项换质以处理其补类。

7.3 节提出九种有用的方法,用以处理那些构成命题不是标准式的三段论论证。

1.单称命题,如"苏格拉底是哲学家",可当做全称命题(A 或 E)对待。

2.如果命题的谓项是形容词或形容词短语,可把它们替换为指称相应类的词项。

3.如果命题的主要动词不是标准联项"是"或"不是",可把动词及其他词语(主项与量项之外)看做类的定义特征,从而把原命题改写为标准式。

4.如果命题的各成分虽已出现但顺序不标准,找出主项,重新调整各成分的顺序。

5.处理非标准量词时,通常要把它们替换为"所有"、"没有"或"有"。要把"并非每个……"翻译为"有……不是……"。

6.排斥命题,如"只有公民是选民",一般要通过颠倒主、谓项位置翻译为 A 命题。结论通常是"所有选民是公民"。

7.不含量词的命题,要依据语境把量词"所有"或"有"补充完整。

8.有些命题的表达形式完全不像标准式直言命题,但其逻辑上等价的直言命题可以明确地表述出来。

9.除外命题,如"除了雇员之外所有人都合格",不是简单的直言命题,而要翻译为两个标准式直言命题的合取。

7.4 节说明并举例解释了协同翻译的方法,即有时需要把同一个辅助词项(参项)引入三段论的三个构成命题中,以便把整个论证翻译为标准形式。

7.5 节考察省略三段论,即前提或结论未明确表述出来的三段论。我们看到,在对省略三段论进行检验之前,如何发现并明确表述出未出现的命题。

7.6 节解释并举例说明连锁三段论。其中有些包含三个以上前提,有些则包含一系列相互关联的三段论。

7.7 节解释了析取三段论和假言三段论,指出了它们的有效形式和可能产生的谬误。

7.8 节讨论二难推论的各种形式,并说明了反驳二难推论的三种方法:抓住联言前提的虚假性、抓住析取前提的虚假性以及构造反二难推论。 