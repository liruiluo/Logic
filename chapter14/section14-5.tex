\section*{14.5 期望值}
我们经常必须在几个可能的行为之间做出选择。当在这些行为的结果中包含有不确定性时,概率计算可以帮助我们做出最好的选择。我们如何才能在这些不确定的结果中做出选择呢?

一个被广泛接受的规则是:我们应该以这样一种方式行动,以使我们的期望值(expected value)最大。期望值是指,在一个赌博或商业冒险中,一个人平均期望获得的价值。在一个游戏的特定场合下,一个人赢或者输,他不可能刚好赢得其期望值。但是,如果他多次参加这个游戏,他可以期望获得所有赢的次数和所有输的次数的平均值,这个平均值等于他的期望值。许多人认为,当在不确定的选项之间进行选择时,一个理性的人将选择期望值最高的那个选项。

我们可以通过一个简单的例子来说明期望值是什么。假定你持有 1000张已售出的彩票中的一张,头奖为 500 美元。这张彩票的期望值是多少?如果我们多次参加这个游戏,我们将在 1000 次中赢一次。这意味着,在 1000 次中,我们所获得的钱数为 500 美元;平均下来,每次获得的钱数为 $500 / 1000$ 美元,即 0.5 美元。因而,这张彩票的期望值为 50美分。

一般地,一个特定彩票的期望值等于任何奖金的概率乘以该奖金的价值。如果用符号 $E$ 代表期望值, $P$ 代表获得奖金的概率, $V$ 代表奖金的价值,我们可以将规则表示为:

$$
E=P \times V
$$

这可以推广到多个奖金的情况。假定我持有上述彩票中的一张,头奖为 500 美元,二等奖为 100 美元,三等奖为 20 美元。假定在 1000 张已售出的彩票中,一张彩票将获得头奖,一张彩票将获得二等奖,三张彩票将获得三等奖。现在我的彩票的期望值是什么?获得头奖的概率是 $1 / 1000$ ,头奖的价值是 500 美元,头奖的期望值为 $1 / 1000 \times 500$ 美元,即 0.50 美元。获得二等奖的概率是 $1 / 1000$ ,二等奖的价值为 100 美元,二等奖的期望值为 $1 / 1000 \times 100$ 美元,即 0.10 美元。获得三等奖的概率是 $3 / 1000$ ,三等奖的价值为 20 美元,三等奖的期望值为 $3 / 1000 \times 20$ 美元,即 0.06 美元。我这张彩票的总的期望值是这三个期望值之和: $0.50+0.10+0.06=0.66$ 美元。这比只有头奖时的期望值要高一些。因而,在这样的多奖项的彩票中,值得多花几分钱购买。

期望值的概念在赌博中非常重要。如果一个赌博的期望值为 0 ,它便是公平的赌博。这意味着,赌博者平均下来既不赢也不输。一个赌徒必须支付的费用应等于他获胜的概率乘以他获胜的价值。这就是为什么当在赌场掷骰子时,如果掷出 7 点或 11 点,庄家支付一赔一的赌注。如果他支付更多,他将会输钱;如果他支付更少,赌徒们将发现这个赌博不公平而拒绝参加。如果一场赌博的期望值为正,它对赌徒有利;如果期望值为负,则对庄家有利。

如果人们普遍遵循最大化期望值的规则,那么在日常事务中运用概率将是非常有用的。例如,假定你面对在两个工作中选择一个。一个工作稳定,每年薪水 3 万美元。另外一个工作风险较大,每年薪水 5 万美元,但如果你所在的公司破产,你将失业,每年薪水只有 1.5 万美元(失业救济金)。根据你对该公司前景的评估,你估计公司成功的概率为 $80\%$ ,破产的概率为 $20\%$ 。你应该选择哪个工作?为了遵循最大化期望值的规则,我们应该计算每个选项的期望值,然后选择期望值高的那个。第一个工作的期望值很容易计算,因为它是确定的: $1.00 \times 30000=30000$ 美元。第二个工作的期望值是: $(0.80 \times 50000)+(0.20 \times 15000)=40000+3000=43000$ 美元。由于第二个工作的期望值更高,根据最大化期望值的规则,你应该选择风险较大的那个工作。

当然,在现实生活中,还有其他因素需要考虑,例如一个人对风险的承受能力,以及对稳定性的偏好。然而,期望值的计算提供了一个有用的工具,可以帮助我们在不确定的情况下做出更明智的决策。

期望值这个概念也可以用来解释为什么保险是合理的。例如,假定一所价值 10 万美元的房子每年被烧毁的概率为 $1 / 500$ 。房主每年支付 250 美元的保险费。这是否是一个好的交易?我们来计算一下不买保险和买保险的期望值。

如果不买保险,期望损失是 $1 / 500 \times 100000 = 200$ 美元。这意味着,平均而言,房主每年会因为火灾损失 200 美元。

如果购买保险,房主每年固定支出 250 美元。如果发生火灾,保险公司将赔偿损失,所以房主的损失为 0。因此,购买保险的期望"损失"(支出)是 $1.00 \times 250 = 250$ 美元。

乍一看,不买保险的期望损失 (200美元) 低于购买保险的固定支出 (250美元)。那么为什么还要买保险呢?这里的关键在于"风险规避"。对于大多数人来说,一次性损失 10 万美元的灾难性后果,远比每年多支付 50 美元的成本更难以承受。保险通过将个体的巨大风险分摊给大量的投保人,从而降低了个体面临的风险。虽然从纯粹的期望值来看,保险公司会盈利 (因为保费总额大于预期的赔付总额),但对于个体投保人而言,购买保险是为了规避那种虽然概率低但一旦发生就无法承受的巨大损失。 