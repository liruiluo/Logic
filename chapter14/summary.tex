\section*{第14章概要}
在所有归纳论证中,前提只是以某个概率度对结论进行支持,在科学假说中我们只是简单地把这个度描述成"更"可能或"不太"可能。本章

说明了如何能够将一个定量的概率(表示为 0 与 1 之间的小数)分派给归纳结论。

14. 1 节给出两种概率概念,它们都可以给予定量配置:(1)相对频率理论,根据这个理论,概率被定义成一个类的成员出现一个特定属性的相对频率。(2)先验理论,根据这个理论,一个事件发生的概率,由事件能够发生的途径数除以等可能的后果数来确定。

这两个理论均与 14.2 节介绍的概率计算相协调。如果复杂事件的各单元事件的概率能够确定,复杂事件的概率就能够计算出。在概率计算中使用两个基本的定理:乘法定理和加法定理。

如果复杂事件是一个共同发生的事件,两个或更多的单元事件均发生的概率可用乘法定理得到, 14.3 节给出说明。乘法定理断定,如果单元事件是独立的,它们共同发生的概率等于它们各自的概率的积。但如果单元事件是不独立的,可以运用通用乘法定理:( $a$ 且 $b$ )的概率等于 $a$ 的概率乘以在 $a$ 发生的条件下 $b$ 的概率。

如果复杂事件是替代性发生的(两个或更多事件中至少一个发生的概率),可应用加法定理,在 14.4 节得到说明。加法定理断定,如果单元事件是相互排斥的,它们的概率之和给出了替代性发生的概率。但如果单元事件不是相互排斥的,它们替代性发生的概率可以这样计算:(1)通过将所需要的场合分解成相互独立的事件,然后将他们的概率相加;或者 (2)确定至少替代性发生事件将不发生的概率,然后用 1 减去这个数。

为了计算一项投资或赌博的预期值(14.5节的内容),我们既要考虑可能后果的概率,又要考虑每个可能事件下获得的收益。先将每个后果预期回报与该回报发生的概率相乘,然后将这些乘积相加便得到投资的预期值。 