\
\\section*{14.2 概率计算}
我们来确定一个复合事件的概率。复合事件可以被看做由多个事件构成的整体。例如,我们问:从一副牌中连续抽出两张黑桃的概率是多少?连续抽两张牌这样的复合事件是一个由两个部分组成的整体。这两个部分是,第一次抽出黑桃的事件,和第二次抽出黑桃的事件。再举一个例子,新娘和新郎活到庆祝金婚纪念日的复合事件,是由新娘再活 50 年的事件和新郎再活 50 年的事件,以及不发生离婚的事件组成的。当人们知道各个组成事件是如何相互关联的时候,人们能够根据单个事件的概率而求得该复合事件的概率。因而,我们把"概率计算"——用单元事件的概率计算出复合事件的概率——规定为纯数学的一个分支。

概率计算在日常生活中是极其有用的。知道某个结果的可能性可以帮助我们进行决策,而使我们做事谨慎。因而,其基本定理的掌握和运用是逻辑研究最有用的结果之一。

概率计算最容易用机会游戏(games of chance)——掷骰子、玩扑克等等——的术语来解释。原因是,这些游戏所限定的人工世界使概率定理的直接使用成为可能。因此,尽管概率计算有广泛的应用范围,在这一章中,我们通过赌博中引申出来的问题,初步地阐明概率计算。在阐释过程中我们使用了概率的先验理论,当然,所有结果经过少量的重新解释后也能够用相对频率理论来表述和分析。 