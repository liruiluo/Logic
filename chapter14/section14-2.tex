\section{概率计算}

\begin{quotation}
\textit{概率计算帮助我们确定复合事件的可能性,通过理解单元事件如何组合,我们能够处理日常生活和科学研究中的不确定性。}
\end{quotation}

我们来确定一个\textbf{复合事件}的概率。复合事件可以被看做由多个事件构成的整体。例如,我们问:从一副牌中连续抽出两张黑桃的概率是多少?连续抽两张牌这样的复合事件是一个由两个部分组成的整体。这两个部分是,第一次抽出黑桃的事件,和第二次抽出黑桃的事件。再举一个例子,新娘和新郎活到庆祝金婚纪念日的复合事件,是由新娘再活 50 年的事件和新郎再活 50 年的事件,以及不发生离婚的事件组成的。当人们知道各个组成事件是如何相互关联的时候,人们能够根据单个事件的概率而求得该复合事件的概率。因而,我们把"概率计算"——用单元事件的概率计算出复合事件的概率——规定为纯数学的一个分支。

\subsection{概率计算的实用性}

概率计算在日常生活中是极其有用的。知道某个结果的可能性可以帮助我们进行决策,而使我们做事谨慎。因而,其基本定理的掌握和运用是逻辑研究最有用的结果之一。

概率计算最容易用\textbf{机会游戏}(games of chance)——掷骰子、玩扑克等等——的术语来解释。原因是,这些游戏所限定的人工世界使概率定理的直接使用成为可能。因此,尽管概率计算有广泛的应用范围,在这一章中,我们通过赌博中引申出来的问题,初步地阐明概率计算。在阐释过程中我们使用了概率的先验理论,当然,所有结果经过少量的重新解释后也能够用相对频率理论来表述和分析。

\subsection{复合事件的类型}

在接下来的小节中,我们将讨论两种主要的复合事件:
\begin{itemize}
\item \textbf{事件的共同发生}是指所有被考虑的单元事件均发生,例如,连续掷三次硬币得到三次正面的概率为多少?
\item \textbf{事件的替代性发生}是指至少有一个被考虑的单元事件发生,例如,掷两次骰子,至少得到一次6点的概率是多少?
\end{itemize}

我们要先讨论事件的共同发生的概率,然后讨论事件的替代性发生的概率。

\subsection{事件的关系}

为了计算复合事件的概率,我们需要知道单元事件之间的关系:它们是独立的还是非独立的。

\begin{itemize}
\item \textbf{独立事件}是指其中一个事件是否发生,不影响另一个事件发生的概率。例如,连续掷两次硬币,第一次是否出现正面,不影响第二次出现正面的概率。
\item \textbf{非独立事件}是指一个事件的发生会影响另一个事件发生的概率。例如,从一副牌中抽两张牌(不放回),第一次抽出黑桃的事件会影响第二次抽出黑桃的概率。
\end{itemize}

这种独立性或非独立性的区分对于正确计算复合事件的概率至关重要,我们将在后续章节中详细探讨。

\begin{center}
\fbox{\parbox{0.95\textwidth}{
\textbf{本节要点}
\begin{itemize}
\item \textbf{概率计算的基本概念}:
  \begin{itemize}
  \item 概率计算是用单元事件的概率计算复合事件概率的数学分支
  \item 复合事件由多个单元事件组成,可能是共同发生或替代性发生
  \item 概率值总在0到1之间,不可能事件为0,必然事件为1
  \end{itemize}
\item \textbf{事件的关系类型}:
  \begin{itemize}
  \item 独立事件:一个事件的发生不影响另一个事件的概率
  \item 非独立事件:一个事件的发生改变另一个事件的概率
  \item 互斥事件:两个事件不能同时发生
  \end{itemize}
\item \textbf{实际应用价值}:
  \begin{itemize}
  \item 帮助评估科学假说的可靠性
  \item 为日常决策提供理性基础
  \item 在不确定性中识别最优选择
  \end{itemize}
\end{itemize}
}}
\end{center} 