\section{替代性发生的概率}

\begin{quotation}
\textit{计算多个事件中至少一个发生的概率是概率论中的重要问题,它不仅帮助我们评估赌博风险,还在医学诊断、工程安全性和科学推理中发挥着关键作用。}
\end{quotation}

我们有时对一系列事件中的一个或多个发生的概率感兴趣。例如,当我们掷两枚硬币时,我们想知道一枚或另外一枚着地时正面向上的可能性是多少。在抽两张牌的扑克牌游戏中,我们想知道抽到或者一张黑桃或者一张梅花的概率为多少。替代性发生的概率总是大于每个事件发生的概率。如同在共同发生的情况下,两个事件共同发生的概率将小于其中一个单独事件发生的概率。

\subsection{计算替代性发生概率的基本方法}

人们如何计算替代性发生的概率?在共同发生的情况下我们将两个分数相乘,得到了一个低的概率值。不同的是,当我们求替代性发生的概率时,我们将分数相加,概率值增加。然而,我们同样碰到了复杂的情况,需要我们将之分为两类进行考虑。

替代性发生的事件可能是相互排斥的,也可能不是相互排斥的。两个事件如果不能同时发生,它们便是相互排斥的。如果我掷两枚硬币并得到两个正面,我不能在这两次投掷中得到两次反面。两个正面和两个反面明显是相互排斥的。但是如果我从一副牌中抽取两张牌,两张牌中一张是黑桃或一张是梅花,是可以出现的不同情形。在一副牌中抽取两张牌,"抽到一张黑桃"和"抽到一张梅花"不是相互排斥的事件。计算替代性发生概率的方法将因事件是否为相互排斥而大大不同。我们依次来分析。

\subsection{相互排斥事件的替代性发生}

如果事件是相互排斥的,计算直接而且容易:将两个事件的概率进行简单相加即可。将一枚硬币掷两次,出现两次正面或者两次反面的概率是多少?自然的是,一个概率与另外一个概率相加。两次正面的概率为 $1 /$ 4 ,两次反面的概率为 $1 / 4$ ,或者两次正面或者两次反面的概率为 $1 / 4+1 /$ $4=1 / 2$ 。

计算两个相互排斥事件构成的复合事件的概率公式为:

$$
P(a \text { 或 } b)=P(a)+P(b)
$$

这是加法定理,它可以推广到适合任意多的事件( $a, b, c \cdots \cdots$ )。如果所有的事件是相互排斥的,它们中至少一个发生的概率为它们的概率和。

通过对扑克牌游戏中能够被分发到同花色牌(5 张牌为同一种花色)的概率的计算,我们来说明上面的公式。这里,有四个相互排斥的可能性:拿到 5 张黑桃的事件,拿到 5 张红桃的事件,拿到 5 张梅花的事件,拿到 5 张方块的事件。让我们先来计算拿到 5 张黑桃的概率。这是一个由 5 个明显非独立的子事件构成,因为分发到黑桃将减低下面得到黑桃的概率。利用非独立的乘法定理,我们有 $13 / 52 \times 12 / 51 \times 11 / 50 \times 10 / 49 \times$ $9 / 48=33 / 66640$ 。其他每一个可能性(5 张红桃、5 张梅花、5 张方块)均有与此相同的概率。这 4 种同花色是相互排斥的事件,因此,利用加法定理,得到任何一种同花色的概率为 $33 / 66640+33 / 66640+33 / 66640+$ $33 / 66640=33 / 16660$ 。

\subsection{多球问题示例}

再举一个例子。从两个袋子中各摸一个球,一个袋子中有两个白球和 4 个黑球,另一个袋子中有 3 个白球和 9 个黑球,摸到两个同颜色的球的概率是多少?我们感兴趣的概率的事件是两个互斥事件的替代性发生:一个是摸到两个白球的事件,另外一个是摸到两个黑球的事件。分别计算这两个事件的概率,然后相加。摸到两个白球的概率为 $2 / 6 \times 3 / 12=1 / 12$ ;摸到两个黑球的概率为 $4 / 6 \times 9 / 12=1 / 2$ 。因此摸到两个同样颜色的概率为 $1 / 12+1 / 2=7 / 12$ 。

\subsection{非互斥事件的替代性发生}

到目前为止,我们对替代性发生的讨论都是针对互斥事件。但是我们必须计算由非互斥的两个或更多的事件中至少一个发生的复合事件的概率。例如,将一枚硬币掷两次,至少得到一次正面的概率是多少?事件不是互斥的,因为可以肯定的是,能够两次投掷都得到正面。我们清楚,第一次投掷得到正面的概率为 $1 / 2$ ,第二次投掷得到正面的概率也是 $1 / 2$ ,但这两个概率之和为 1 ,即事件为确定的,然而至少一次投掷为正面是不确定的!这个例子说明,当我们计算非互斥事件替代性发生的概率时,加法定理不能直接应用。我们可以用两个间接的方法来计算这种类型的概率。

\subsection{计算非互斥事件替代性发生的方法}

计算两个非互斥事件中至少一个发生的概率的第一个方法,要求我们将事件分解成互斥事件。在求解将一枚硬币投掷两次得到至少一面为正面的概率的问题中,等可能的状态是 $\mathrm{H}-\mathrm{H}, \mathrm{H}-\mathrm{T}, \mathrm{T}-\mathrm{H}, \mathrm{T}-\mathrm{T}$ 。它们是相互排斥的,每一个状态的概率为 $1 / 4$ 。前三个状态为我们要求的;即在前三个状态中的任何一个状态发生的条件下,两次投掷中至少一次为正面就是真的事实。于是,投掷出至少一面为正面的概率,等于所有符合要求的互斥状态的单独概率之和,即为 $1 / 4+1 / 4+1 / 4=3 / 4$ 。

计算两个非互斥事件中至少一个发生的概率的另外一种方法,建立在这样的事实上,没有状态既是满足条件的又是不满足条件的。我们用 $a$ 表示将一枚硬币投掷两次得到至少一次正面的事件,那么,我们用符号 $\bar{a}$ 表示与 $a$ 不同的事件,即两次投掷没有一次正面的事件。因为没有状态既是我们需要的又是我们不需要的,$a$ 与 $\bar{a}$ 是相互排斥的,$a$ 与 $\bar{a}$ 不能都发生。由于每个状态必定是,或者是这个事件或者不是这个事件,可以肯定的是,或者 $a$ 或者 $\bar{a}$ 必定发生。我们给不能发生的事件指派概率值 0 ,给必定发生的事件指派概率值 1。下面两个等式是成立的:

$$
P(a \text { 且 } \bar{a})=0
$$

$$
P(a \text { 或 } \bar{a})=1
$$

这里,$P(a$ 且 $\bar{a})$ 为 $a$ 和 $\bar{a}$ 均发生的概率,$P(a$ 或 $\bar{a})$ 为 $a$ 或者 $\bar{a}$ 发生的概率。由于 $a$ 和 $\bar{a}$ 是互斥的,可以应用加法定理。我们得到:

$$
P(a \text { 或 } \bar{a})=P(a)+P(\bar{a})
$$

$$
P(a)+P(\bar{a})=1
$$

由上式得到非常有用的等式:

$$
P(a)=1-P(\bar{a})
$$

于是,我们可以通过计算一个事件不发生的概率,再用 1 减去这个数,就得到一个事件发生的概率。我们应用这个方法来求投掷两次硬币得到至少一次正面事件的概率。我们容易看到,该事件不发生的唯一情况为,两次投掷均为反面——这是不满足条件的状态,由乘法定理,概率值为 $1 / 2 \times$ $1 / 2=1 / 4$ ,据此,投掷两次硬币中得到至少一次正面事件确实发生的概率为 $1-1 / 4=3 / 4$ 。

\subsection{实际应用:概率的违反直觉性质}

有时,应用概率计算得到一个尽管正确的结果,但是与我们对已知事实进行因果分析后所期望的结论不同。这样的结果被认为是违反直觉的。当一个问题的解违反直觉的时候,人们可能在概率判断上发生错误。这样 "自然"的错误驱使人们在狂欢场所及其他地方进行如下的赌博。摇三个骰子,赌场庄家与你打一赔一的赌(如果打一元的赌,如果你赢了,你取回你押的一元,庄家再给你一元),庄家赌三个骰子中均不出现么点(一点)。骰子有六面,每个面上有不同的数字。你有三个机会得到么点,表面上看,这似乎是一个公平的赌博。

事实上,这不是一个公平的赌博。利用这个与直觉相反的事实的骗子能够获得丰厚的利润。这个赌博仅当在这样的条件下才是公平的:三个骰子中的一个骰子出现某一特定点数后,而使另外两个骰子中的任一个骰子不出现该点数。这显然不正确。粗心的下注者错误地(和下意识地)认为它们具有互斥性。然而它们不是相互排斥的,一些投掷中两个或者三个骰子会出现相同点数。试图通过确定并计算所有可能结果,以计算至少一个幺点出现的结果数,很快就会发现,这样的努力是难以进行的。但是,因为任何给定点数的出现并不能排除其他骰子也出现同样点数,这样的赌博确实是一个欺骗。我们先确定输的概率然后从 1 中减去这个概率值,从而计算出胜出的概率,此时,这个欺骗便显出来。单个骰子非-幺点(出现 2 点,或 3 点,或 4 点,或 5 点,或 6 点)向上的概率为 $5 / 6$ 。输的概率为 3 个非 - 幺点出现向上的概率,其概率(由于骰子之间是不相互影响的)为 $5 / 6 \times 5 / 6 \times 5 / 6=125 / 216$ ,即 0.579 。下注者摇到至少一个幺点的概率为 $1-125 / 216=91 / 216$ ,即为 0.421 。这就是赌博的原理!

\subsection{双骰赌博问题分析}

让我们用概率求解一个中等难度的问题。双骰赌博(craps)是用两个骰子进行。下注者如果在第一次投掷中得到(总和为) 7 点或者 11 点,那么他赢了;如果在第一次投掷中得到 2 点或 3 点或 12 点,那么他就输了。如果第一次摇出的骰子出现其他的点数( $4$、$5$、$6$、$8$、$9$、$10$ ),摇骰子者继续摇盅。在以后的摇骰子中,如果出现与上次同样的点数,那么下注者赢了;如果出现 7 点,那么下注者输了。双骰赌博被普遍认为是公平的赌博——下注者有一半的获胜机会。真是这样的吗?让我们计算在双骰赌博中下注者获胜的概率。

为此,我们首先得有不同点数出现的概率。两个骰子落下后,有 36 个等可能的情况。2 点只有一种出现方式,即 1-1。它出现的概率为 $1 / 36$ 。只有一种状态出现 12 点,其概率为 $1 / 36$ 。有两种状态得到 3 点: $1-2,2-1$ ,点数 3 的概率为 $2 / 36$ 。类似的,得到 11 点数的概率为 $2 / 36$ 。 3 种状态可以得到 4 点: $1-3,2-2$ 及 $3-1$ ,因此点数为 4 的概率为 $3 / 36$ 。类似的,点数为 10 的概率值为 $3 / 36$ 。由于有 4 种状态得到 5 点 $(1-4,2-3,3-2,4-1)$ ,其概率为 $4 / 36$ ,这同样是点数 9 的概率。得到点数 6 的状态有 5 种 $(1-5,2-4,3-3,4-2,5-1)$ ,点数 6的概率为 $5 / 36$ ,点数 8 的概率值与此相同。有 6 种可能状态产生点数 $7(1-6,2-5,3-4,4-3,5-2$ 及 6-1),摇出点数 7 的概率值为 $6 / 36$。

下注者在第一次摇骰子中获胜的概率为出现点数 7 的概率和出现点数 11 的概率之和,其值为 $6 / 36+2 / 36=8 / 36$ 。第一次摇骰子中他输的概率为出现点数 $2$、$3$、$12$ 的概率和,值为 $1 / 36+2 / 36+1 / 36=4 / 36$ ,即 $1 / 9$ 。在第一次摇骰子中下注者赢的可能性为输的可能性的两倍。然而在第一次摇骰子中下注者很有可能既不赢又不输,即摇到点数 $4$、$5$、$6$、$8$、$9$、$10$ 。如果掷出这 6 个数中的一个,下注者得再次摇盅,直到该点数重新出现——下注者赢了,或者点数 7 出现——下注者输了。第一次摇骰子中没有出现的点数和点数 7 的状态可以忽略,因为它们不起决定作用。假定下注者在第一次摇骰中得到点数 4 ,下一次摇骰子中起决定作用的是出现点数 4 或者 7 。在决定作用的摇骰子中,等可能的状态是使点数出现 4 的 3种组合 $(1-3$、$2-2$、$3-1)$ ,和使点数 7 出现 6 种组合;因而第二次投掷得到点数 4 的概率为 $3 / 9$ 。第一次摇骰子中得到 4 点的概率为 $3 / 36$ ,因此,第一次摇得点数 4 、第二次又摇得点数 4 而未出现点数 7 的概率为 $3 /$ $36 \times 3 / 9=1 / 36$ 。类似的,下注者第一次摇得点数 10 、第二次又摇得点数 10 而未出现点数 7 的概率也是 $3 / 36 \times 3 / 9=1 / 36$ 。

下注者赢的方式有 8 个不同种类:第一次出现点数 7 或点数 11 ;或者第一次得到 $4$、$5$、$6$、$8$、$9$、$10$ 中的一个点数,并且第二次得到同样的点数。这些方式都是相互排斥的,所以下注者总的赢的概率为能够获胜的各个可能性的概率之和。这个概率为 $6 / 36+2 / 36+1 / 36+2 / 45+25 / 396+$ $25 / 396+2 / 45+1 / 36=244 / 495$ 。如果表示成分数,概率值为 0.493 。这表明在双骰赌博中,下注者赢的机会小于输的机会——尽管略小,但仍小于 0.5 。

\begin{center}
\begin{tabular}{|p{0.95\textwidth}|}
\hline
\textbf{加法定理} \\
\hline
\textbf{计算两个或更多的替代性的事件发生的概率的方法:} \\[6pt]
\textbf{A.如果事件(如 $a$、$b$ )是相互排斥的:} \\
至少一个事件发生的概率为它们概率的简单相加:
$P(a \text { 或 } b) = P(a)+P(b)$ \\[6pt]
\textbf{B.如果事件(如 $a$、$b$、$c$ )不是相互排斥的:} \\
它们中至少一个发生的概率由下面的方法确定: \\
(1)将满足条件的状态区分为互相排斥的事件,然后将这些事件的概率相加; \\
(2)计算这些可能事件不发生的概率,然后用1减去这个概率。 \\
\hline
\end{tabular}
\end{center}

\begin{center}
\fbox{\parbox{0.95\textwidth}{
\textbf{本节要点}
\begin{itemize}
\item \textbf{替代性发生的基本概念}:
  \begin{itemize}
  \item 替代性发生指多个事件中至少一个发生的情况
  \item 替代性发生的概率总大于单个事件的概率
  \item 计算方法取决于事件是否互斥
  \end{itemize}
\item \textbf{互斥事件的替代性发生}:
  \begin{itemize}
  \item 互斥事件指不能同时发生的事件
  \item 概率通过简单加法计算:$P(a \text{ 或 } b) = P(a) + P(b)$
  \item 适用于掷骰子点数、抽牌花色等互相排斥的情形
  \end{itemize}
\item \textbf{非互斥事件的替代性发生}:
  \begin{itemize}
  \item 事件可以同时发生,需要特殊处理
  \item 方法一:将事件分解为互斥状态后相加
  \item 方法二:用1减去所有事件都不发生的概率
  \end{itemize}
\item \textbf{与直觉相悖的概率}:
  \begin{itemize}
  \item 直觉判断在概率计算中常常出错
  \item 非互斥事件的替代性发生尤其容易误判
  \item 精确计算能避免在赌博和决策中的错误判断
  \end{itemize}
\end{itemize}
}}
\end{center}
% The rest of the file, including exercises, is removed. 