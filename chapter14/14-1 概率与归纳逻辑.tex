\section{概率与归纳逻辑:不确定性的数学化}

\begin{logicbox}[title=引言]
概率是归纳逻辑的核心工具,它为我们提供了量化不确定性、评估科学假说可靠性的数学框架。本章将从哲学、数学、科学方法论等多个角度深入探讨概率的不同解释及其在推理中的应用,分析概率论如何成为现代科学和理性决策的基础工具,以及概率思维如何改变了人类认识世界的方式。通过理解概率的本质和应用,我们将能够更好地处理不确定性,做出更加明智的判断和决策。
\end{logicbox}

\subsection{概率在认识论中的地位:从确定性到或然性的转变}

\begin{theorembox}[title=概率概念的认识论革命]
概率概念的引入标志着人类认识论的一次重大转变——从追求绝对确定性到接受和量化不确定性。

\textbf{古典认识论的确定性追求}:
传统的认识论,从古希腊哲学到笛卡尔的理性主义,都追求绝对确定的知识。这种观点认为,真正的知识必须是确定无疑的,不容任何怀疑。

\textbf{现代科学的或然性认识}:
然而,现代科学的发展表明,绝大多数科学知识都具有或然性特征。科学理论不是绝对真理,而是在当前证据基础上最可能的解释。

\textbf{概率论的认识论意义}:
\begin{itemize}
\item \textbf{量化不确定性}:为不确定性提供精确的数学表达
\item \textbf{理性决策基础}:在不完全信息下做出最优选择
\item \textbf{科学方法工具}:评估假说、设计实验、分析数据
\item \textbf{风险评估框架}:在各个领域进行风险管理
\end{itemize}
\end{theorembox}

\subsection{概率概念的基本特征与应用领域}

\begin{theorembox}[title=概率在归纳逻辑中的核心地位]
在归纳逻辑中,\textbf{概率}(probability)是一个核心概念,在前面关于科学方法的讨论中已多次阐述。

\textbf{科学假说的或然性特征}:
一个假说即使符合所有接触到的事实,它也不是决定性地得以确立;它只具有或然性。我们看到,即使我们慎之又慎地使用实验探究的密尔方法,也不能决定性地证明我们所得到的因果律的真理性,而只是以高的或然性(概率)确证它们。

\textbf{归纳推理的本质限制}:
即使最好的归纳论证也不具有有效演绎论证所拥有的那种确定性。这种限制不是方法的缺陷,而是归纳推理的本质特征。

\textbf{概率分析的必要性}:
因而,恰当地说,我们对归纳逻辑的考察离不开对概率这个关键概念的分析。概率为我们提供了处理不确定性的精确工具。
\end{theorembox}

\begin{examplebox}[title=概率概念的典型应用类型]
我们必须区分"盖然的"(probable)和"概率"的不同用法。下面三个命题显示了概率的最典型的用法:

\textbf{类型一:数学概率}
1. 一个投出去的硬币出现正面的概率是 $1/2$。

\textbf{类型二:统计概率}
2. 一个25岁的妇女过26岁生日的概率为0.971。

\textbf{类型三:认识概率}
3. 基于现有的证据,爱因斯坦相对论的正确性是高度盖然的。

\textbf{三种类型的特征对比}:
\begin{itemize}
\item \textbf{数学概率}:基于逻辑分析和对称性原理
\item \textbf{统计概率}:基于经验数据和频率分析
\item \textbf{认识概率}:基于证据强度和理论支持度
\end{itemize}
\end{examplebox}

存在使用"盖然的"和"概率"的其他情境,如我们说测量中的"可能错误"(probable errors),但这三种是最重要的。在前两个命题中,一个数字——被称为概率数值——被赋予一个特定事件;第三个命题则不同,它没有被赋予这样的数字。当我们谈论一个可疑的科学假说时,我们通常赋予一个盖然程度。比如,人们说达尔文理论比《创世记》中对生命起源的解释更可靠(probable),再比如,原子理论具有比其他的关于原子核内部结构的假说有较高的盖然度。

前两个例子中所给予的数字是十分有用的,并且似乎十分合理。但它们从何处得来?

硬币有两面:正面和反面。当硬币落地时,必定有一面朝上。两个机会中的一个机会是正面向上,因此以概率 $1 / 2$ 赋予正面。为了得到第二个例子中的概率值,我们必须进行死亡率统计并进行比较。在 1000 个庆祝

正如前两个例子所表明的,概率研究与赌博和死亡统计有关;事实上,现代概率研究开始于这两个领域。熟知的是,概率论起步于帕斯卡 (Blaise Pascal,1623-1662)和费玛(Pierre Fermat,1608-1665)关于机会赌博中合理赌注的通信,另外一种说法是,概率起源于帕斯卡给切瓦里•德•梅尔(Chevalier De Mere)一一个著名的赌徒——如何在掷骰子时下赌注的建议。与死亡率相关的是,自从1592年伦敦开始保存死亡记录;1662年,约翰•格朗特上尉发表了对这些记录的一个研究,探讨了从这些记录中用概率能够推得什么。可能是因为这复合的血缘,概率有如下两种解释。

\subsection{概率的先验解释}

经过拉普拉斯、德摩根、凯恩斯等人的发展,关于概率本质的古典理论认为,概率是合理信念(rational belief)度的测定。当我们完全相信某个事情,我们信念度的测定被赋予数字1。当我们绝对相信一个特定事件不可能发生,该事件将发生的信念度被赋予数字 0 。因而,一个理性人在一个掷出去的硬市或者出现正面或者不出现正面上的信念度是 1 ;既出现正面又不出现正面的信念度是 0 。当他不能肯定的时候,他的合理信念度将为 0 和 1 之间的某个数。概率是关于事件的一个属性,它是人们合理地相信一个事件将发生的程度。或者说,概率是一个陈述或命题的谓词,一个完全理性的人总是依据这个值相信该陈述或命题。

在古典理论看来,概率总是部分有知和部分无知的结果。如果我们能够知道掷硬市的手指的精确运动,加上硬币的初始位置、大小、重量分布,人们确信能够预测硬市的轨道以及最后的不动的位置。但是,这些完全的信息不可能得到。我们只能知道某些信息:硬币有两面;它将下落;等等。因此,\textbf{硬币正面向上}的信念度由几种可能性所决定——这里可能性为 2 个、出现正面的可能性为 1 个。因而, $1 / 2$ 的概率值被赋予硬市出现正面的事件。类似的,人们要分发一副纸牌时,纸牌以一确定的顺序被分发。如果发牌诚实,牌中的黑桃、红桃、方块、梅花,以及 A、K、Q、 $J$ ,均以洗牌时确定的次序而得以分发。但我们不知道这个次序。我们只知道总共的 52 张牌中有 13 张黑桃,因此,所发的第一张牌为黑桃的概率精确地为 $13 / 52$ ,或者 $1 / 4$ 。

这些观点为概率论的先验论观点。之所以如此称呼,是因为无须做实验,也无须选择样本来考察,就可以得到概率。只需要知道先行条件:纸牌中只有 13 张黑桃;总共有 52 张牌;发牌是诚实的,任何一张牌与其他牌有同样的机会被第一次分发。以先验的观点,为了计算在某些特定情形下一个事件发生的概率,我们把该事件能够发生的途径数,除以该情形下可能的结果总数——如果我们没有任何理由相信任何一个可能的结果比其他的更有可能的话。于是,一个事件的概率以一个分数来表示,其中,除数是等可能的结果总数,被除数是使待考察事件发生的结果数。一种诚实地出售 1000 张彩票的彩票发行,有 1000 个等可能的结果。因而,其中任何一张彩票能够中彩的概率是 1 除以 1000 ,即 $1 / 1000$ 。

\subsection{概率的相对频率解释}

与先验论不同的一个理论认为概率是相对频率的一个度量。相对频率理论特别适合解释统计研究的概率判断。例如,保险公司精算师希望确定 25 岁妇女的死亡率。这里,我们有一个对象总体和一个属性:这个总体是 25 岁的妇女;属性是活到 26 岁生日。该理论中,赋予的概率是这样的相对频率测度:该人群以这个频率体现了这个被研究的属性。这里同样的是,概率也表示成分数。不过,在这里,分母是对象总体数量,分子是具有该属性的对象的数量。如果考察了 1000 个 25 岁的妇女的记录,发现其中有 971 个活到 26 岁生日,那么 0.971 就是该对象总体出现该属性的概率系数。这里没有出现合理信念。在概率的相对频率理论中,概率被定义成总体成员体现某一特定属性的相对频率。

必须说明的是,在这两个理论中,被赋予的概率是相对于采集的证据而言的。在相对频率理论中这是明显的。因为一个给定属性的概率,必定随着选择用来计算的特定对象总体的变化而变化。在上面用到的例子中,构成研究总体的 1000 名妇女是随机地从埃及人中选取,人们会发现,活到 26 岁的这个频率,将与随机地从法国人中选取的 1000 名妇女活到 26岁的概率不同。 25 岁的妇女再活 1 年的概率在埃及和在法国是不同的。类似的,在斯堪的纳维亚地区的人口总体中金发的概率高于在世界总人口中金发的概率。因此,在使用概率的相对频率理论时,一个关键的步骤是选择最合适的研究总体。

但是,在先验理论那里概率也是相对的。根据该理论的古典解释,任何事件均不具有内在的概率。一个事件的概率值之获得只能建立在做出其概率值指派的人所获得证据之上。这样的概率被解释成这样的一个观点,概率为合理信念的测度,因为一个理性人的信念随着他的知识的变化而变化。

譬如,假设两个人观看洗牌。当洗牌完成时,洗牌者因某个偶然的因素意外地使最上面的一张牌"露"了一下。第一个观察者看到了那张牌是黑的,但他没有看到是黑桃还是梅花。第二个观察者没有任何察觉。如果让这两个观察者估计第一张牌是黑桃的概率,第一个观察者将指派概率值 $1 / 2$ ,因为他知道有 26 张牌(黑色的牌),其中一半是黑桃。但第二个观察者将指派概率值 $1 / 4$ ,因为他知道的仅仅是 52 张牌中黑桃为 13 张。两个观察者对同一个事件指派了不同的概率。其中一个观察者犯了错误?当然没有:每个人相对于可用证据赋予了正确的概率。即使这张牌被翻开后为梅花,两个人的估计均是正确的。任何事件自身不具有内在的或关于它的概率。这里的意思是,任何预测所具有的不同概率是相对于不同背景而言的,即相对于不同证据集而言的。然而,人们在做出概率断定之前,应当尽可能地寻求收集最大量的证据集。

\subsection{概率解释的对比与融合}

概率的这两种解释——相对频率解释和先验解释——在认为概率是相对于证据的这一点上是一致的。因而,这两个理论的信奉者在接受和使用概率计算上也是一致的。下一节将介绍概率计算的初步知识。

\begin{center}
\fbox{\parbox{0.95\textwidth}{
\textbf{本节要点}
\begin{itemize}
\item \textbf{概率在认识论中的地位:从确定性到或然性的转变}:
  \begin{itemize}
  \item \textbf{认识论革命}:概率概念标志着从追求绝对确定性到接受量化不确定性的转变
  \item \textbf{现代科学特征}:科学知识具有或然性特征,理论是当前证据基础上最可能的解释
  \item \textbf{多重功能}:量化不确定性、理性决策基础、科学方法工具、风险评估框架
  \item \textbf{认识论意义}:改变了人类认识世界的方式,为处理不确定性提供数学框架
  \end{itemize}
\item \textbf{概率概念的基本特征与应用领域}:
  \begin{itemize}
  \item \textbf{归纳逻辑核心}:概率是归纳推理的核心概念,科学假说只具有或然性
  \item \textbf{本质限制}:归纳推理不具有演绎推理的确定性,这是本质特征而非缺陷
  \item \textbf{三种应用类型}:数学概率(逻辑分析)、统计概率(频率分析)、认识概率(证据强度)
  \item \textbf{量化工具}:为不确定性提供精确的数学表达和分析方法
  \end{itemize}
\item \textbf{概率的先验解释(古典理论)}:
  \begin{itemize}
  \item \textbf{理论基础}:概率被视为合理信念度的测定,由拉普拉斯、德摩根、凯恩斯等发展
  \item \textbf{计算方法}:通过等可能结果的数学比例计算,无需实验即可推导
  \item \textbf{认识特征}:概率是部分有知和部分无知的结果,反映信息的不完全性
  \item \textbf{相对性质}:概率相对于观察者的知识和证据,不同证据导致不同概率值
  \end{itemize}
\item \textbf{概率的相对频率解释(统计理论)}:
  \begin{itemize}
  \item \textbf{理论本质}:概率表示属性在总体中出现的相对频率
  \item \textbf{应用领域}:特别适合统计研究,如保险精算、医学统计、质量控制
  \item \textbf{计算基础}:基于观察数据和统计分析,分子为具有属性的对象数,分母为总体数
  \item \textbf{总体选择}:关键步骤是选择最合适的研究总体,不同总体产生不同概率
  \end{itemize}
\item \textbf{两种解释的统一性与互补性}:
  \begin{itemize}
  \item \textbf{共同特征}:两种解释都认为概率相对于证据,都接受概率的相对性
  \item \textbf{证据依赖}:概率值随可用证据的变化而变化,需要尽可能收集最大量证据
  \item \textbf{计算兼容}:两种理论在概率计算的接受和使用上是一致的
  \item \textbf{互补应用}:在不同情境下各有优势,共同构成完整的概率理论体系
  \end{itemize}
\end{itemize}
}}
\end{center}