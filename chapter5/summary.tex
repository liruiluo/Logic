\section{第5章概要}
本章介绍并讨论的是\textbf{古典逻辑}即\textbf{亚里士多德型逻辑}的基本构件,也是\textbf{演绎逻辑}的基本构件。

\subsection{类与标准式直言命题}
5.2节介绍了\textbf{类}的概念。传统逻辑正是以类为基础建立起来的。我们阐明了四种基本的标准式直言命题:

\begin{itemize}
  \item \textbf{A命题}:全称肯定命题
  \item \textbf{E命题}:全称否定命题
  \item \textbf{I命题}:特称肯定命题
  \item \textbf{O命题}:特称否定命题
\end{itemize}

\subsection{命题的质与量}
5.3节更加详细地考察这四种命题。探讨了命题的\textbf{质},即肯定和否定,以及命题的\textbf{量},即全称和特称。说明了\textbf{周延的项}与\textbf{不周延的项}。

\subsection{对当关系}
5.4节探讨这几种直言命题之间的\textbf{对当关系}的种类:命题之间的\textbf{矛盾关系}、\textbf{反对关系}、\textbf{下反对关系}以及上位式与下位式之间的\textbf{差等关系}。并用一个\textbf{对当方阵}图示了这几种关系,进而说明了一些基于传统方阵的直接推理。

\subsection{直接推论方法}
5.5节阐明其他三种直接推论:\textbf{换位法}、\textbf{换质法}和\textbf{换质位法}。

\subsection{存在含义问题}
5.6节探讨\textbf{存在含义}问题。要保留传统对当方阵,只有做出一种假定,即全盘假定命题主项所指称的类总是有元素的——这是现代逻辑极不赞同的。然后,我们又对本书通篇采用的\textbf{布尔解释}作了说明。布尔解释能保留传统逻辑对当方阵中的大部分内容,同时又避免了非空类的假定。在布尔解释中,特称命题,即称为$\mathbf{I}$和$\mathbf{O}$的命题之中有存在含义,但全称命题,即$\mathbf{A}$和$\mathbf{E}$则没有存在含义。我们也很细致地说明了采用这种解释的结果。

\subsection{命题的符号化与图示化}
5.7节介绍将直言命题符号化与图示化的方法,包括使用\textbf{文恩图},用交叉的圆加以恰当的标记或阴影来刻画类与类之间的关系。

\begin{center}
\fbox{\parbox{0.9\textwidth}{
  \centering
  \textbf{第5章要点总结}\\
  \textbf{演绎逻辑基础}:古典逻辑的基本构件与概念\\
  \textbf{命题分类}:A全称肯定、E全称否定、I特称肯定、O特称否定\\
  \textbf{推理方法}:直接推论、对当关系、布尔解释、文恩图\\
}}
\end{center}

有了这些必要的工具,我们就可以考察——在接下来的两章中——建基于标准式直言命题之上的\textbf{三段论},以及传统演绎逻辑在日常语言中的其他主要用途。 

\printbibliography[heading=subbibliography,title={第5章参考文献}] 