\section*{第5章概要}
本章介绍并讨论的是古典逻辑即亚里士多德型逻辑的基本构件,也是演绎逻辑的基本构件。\\
5.2 节介绍类的概念。传统逻辑正是以类为基础建立起来的。我们阐明了四种基本的标准式直言命题。

\begin{itemize}
  \item A 命题:全称肯定命题
  \item E 命题:全称否定命题
  \item I 命题:特称肯定命题
  \item O 命题:特称否定命题
\end{itemize}

5. 3 节更加详细地考察这四种命题。探讨了命题的质,即肯定和否定,以及命题的量,即全称和特称。说明了周延的项与不周延的项。\\
5.4 节探讨这几种直言命题之间的对当关系的种类:命题之间的矛盾关系、反对关系、下反对关系以及上位式与下位式之间的差等关系。并用一个对当方阵图示了这几种关系,进而说明了一些基于传统方阵的直接推理。\\
5.5 节阐明其他三种直接推论:换位法、换质法和换质位法。\\
5.6 节探讨存在含义问题。要保留传统对当方阵,只有做出一种假定,即全盘假定命题主项所指称的类总是有元素的一一这是现代逻辑极不赞同的。然后,我们又对本书通篇采用的布尔解释作了说明。布尔解释能保留传统逻辑对当方阵中的大部分内容,同时又避免了非空类的假定。在布尔解释中,特称命题,即称为 $\mathbf{I}$ 和 $\mathbf{O}$ 的命题之中有存在含义,但全称命题,即 $\mathbf{A}$ 和 $\mathbf{E}$ 则没有存在含义。我们也很细致地说明了采用这种解释的结果。\\
5.7 节介绍将直言命题符号化与图示化的方法,包括使用文恩图,用交叉的圆加以恰当的标记或阴影来刻画类与类之间的关系。

有了这些必要的工具,我们就可以考察——在接下来的两章中——建基于标准式直言命题之上的三段论,以及传统演绎逻辑在日常语言中的其他主要用途。 