\section{直言命题及其类别}

\begin{logicbox}[title=引言]
\textit{直言命题是演绎理论的基石,它们关注类与类之间的关系。本节将介绍直言命题的基本类型及其特征,为理解演绎论证奠定基础。}
\end{logicbox}

\begin{theorembox}[title=直言命题的基本概念]
\logicemph{研究焦点}:亚里士多德对演绎的研究主要集中在由一种特殊命题组成的论证上,这种命题是关于范畴(categories)和类(classes)的,被称为\logicterm{直言命题}(categorical proposition)。

\logicwarn{重要地位}:直言命题是演绎理论的基石。要了解这种关于类的演绎理论,必须首先对直言命题进行非常精细的分析。

\logicemph{典型例子}:请考虑如下论证:

\begin{quote}
没有运动员是素食主义者,所有足球队员都是运动员,

所以,没有足球队员是素食主义者。
\end{quote}

\logicwarn{结构分析}:
\begin{itemize}
  \item 这个论证中的三个命题都是直言命题,包括两个前提、一个结论
  \item 这些命题肯定或否定某个类$\boldsymbol{S}$全部或部分地包含于另一个类$\boldsymbol{P}$之中
  \item 三个命题涉及的是运动员的类、素食者的类和足球队员的类
\end{itemize}

\logicemph{核心特征}:直言命题表达的是类与类之间的包含、排斥或部分重叠关系。
\end{theorembox}

\begin{theorembox}[title=类之间的关系]
\logicemph{类的定义}:有关类的知识在第3章讨论定义时已经简要地说明,一个\logicterm{类}就是共有某种特定属性的所有对象(objects)的汇集。

\logicwarn{关系类型}:两个类之间有着多种不同的关系:

\logicemph{三种基本关系}:
\begin{enumerate}
  \item \logicterm{完全包含关系}:如果一个类的所有元素(member)都是另一个类的元素,例如狗的类与哺乳动物的类,则称第一个类包含于(be included)或包括在(be contained)第二个类之中

  \item \logicterm{部分包含关系}:如果一个类中有元素是另一个类的元素,但并非其所有元素都是另一个类的元素,例如女人的类和运动员的类,则称第一个类部分地包含于第二个类之中

  \item \logicterm{相互排斥关系}:如果两个类没有共同的元素,例如三角形的类和圆形的类,则称这两个类之间是相互排斥(exclude)的
\end{enumerate}

\logicwarn{重要性}:这些关系是直言命题表达的核心内容,理解它们对于掌握演绎推理至关重要。

\logicemph{图示理解}:
\begin{itemize}
  \item 完全包含:一个圆完全在另一个圆内部
  \item 部分包含:两个圆部分重叠
  \item 相互排斥:两个圆完全分离
\end{itemize}
\end{theorembox}

\begin{theorembox}[title=直言命题的四种标准形式]
\logicemph{形成原理}:类与类之间的这些关系被直言命题所肯定或否定,其结果是恰好能形成直言命题的四种标准形式。

\logicwarn{标准例示}:可分别由如下标准命题例示:

\begin{enumerate}
  \item 所有政客是说谎者。
  \item 没有政客是说谎者。
  \item 有政客是说谎者。
  \item 有政客不是说谎者。
\end{enumerate}

\logicemph{分析维度}:这四种形式可以从两个维度来理解:
\begin{itemize}
  \item \logicterm{量的维度}:全称(所有/没有)vs. 特称(有些)
  \item \logicterm{质的维度}:肯定(是)vs. 否定(不是)
\end{itemize}

\logicwarn{系统性}:下面我们就细致地考察直言命题这四种标准形式。
\end{theorembox}

\begin{examplebox}[title=全称肯定命题(A命题)]
\logicemph{典型例子}:第一个例子——所有政客是说谎者——是一个\logicterm{全称肯定命题}。

\logicwarn{结构分析}:
\begin{itemize}
  \item 其中涉及两个类,即政客的类和说谎者的类
  \item 它说的是第一个类包含于或包括在第二个类中
  \item 全称肯定命题断言第一个类中所有元素都是第二个类的元素
\end{itemize}

\logicemph{术语说明}:
\begin{itemize}
  \item 主项"政客"指称(designate)政客的类
  \item 谓项"说谎者"指称说谎者的类
\end{itemize}

\logicterm{标准形式}:所有全称肯定命题都可以写成如下形式:

\begin{center}
所有$S$是$P$。
\end{center}

其中字母$S$和$P$分别代表主项和谓项。

\logicwarn{名称合理性}:"全称肯定命题"这一名称是恰当的,因为:
\begin{itemize}
  \item 这个命题肯定了两个类之间的包含于关系
  \item 并且是完全或者说全部包含于关系
  \item 断言$S$的所有元素同时都是$P$的元素
\end{itemize}
\end{examplebox}

\begin{examplebox}[title=全称否定命题(E命题)]
\logicemph{典型例子}:第二个例子——没有政客是说谎者——是一个\logicterm{全称否定命题}。

\logicwarn{逻辑含义}:
\begin{itemize}
  \item 它是对全部政客而言,否定他们是说谎者
  \item 就这样两个类来说,全称否定命题断言第一个类与第二个类是完全排斥的
  \item 也就是说第一个类中没有元素是第二个类的元素
\end{itemize}

\logicterm{标准形式}:所有全称否定命题都可以写成如下形式:

\begin{center}
没有$S$是$P$。
\end{center}

其中$S$和$P$也分别代表主项和谓项。

\logicwarn{名称合理性}:"全称否定命题"这一名称是恰当的,因为:
\begin{itemize}
  \item 这个命题否定了这两个类之间的包含于关系
  \item 并且是全部否定
  \item 断言在$S$的所有元素中,没有一个是$P$的元素
\end{itemize}

\logicemph{注释}:我国逻辑教材中一般把全称否定命题的形式写为:"所有$S$不是$P$",其与"没有$S$是$P$"同义。但根据英语语法,"All $S$ are not $P$"并不与"No $S$ are $P$"同义,而等义于"Not all $S$ are $P$"。故英文著作一般将"No $S$ is(are)$P$"作为全称否定命题的形式。
\end{examplebox}

\paragraph{特称肯定命题}
第三个例子—有政客是说谎者——是一个\textbf{特称肯定命题}。显然,这个例子肯定的是政客类中有元素(也)是说谎者类的元素。但并没有对政客类作全部断言:它说的并不是所有政客,而是某个或某些政客是说谎者。此命题既没有肯定也没有否定所有政客是说谎者,对此并没有给出主张。从字面含义看,它并没有断言有政客不是说谎者,尽管在某些语境中它可能暗含这样的意思。这个命题的字面含义或者说最小的(minimal)解释,即政客的类和说谎者的类之间有某个或某些元素是共同的。为确定性起见,我们这里采取最小解释。

"有"(some)这个词的含义是不确定的。它指的是"至少有一个"、"至少有两个",还是"至少有一百个"呢?到底有多少个?尽管与某些场合中的通常用法不太一致,但为了保持确定性,我们一般把"有"看做"至少有一个"的意思。这样,特称肯定命题可以写成如下形式:

有$S$是$P$。

它断言的是,主项$\boldsymbol{S}$指称的类中至少有一个元素是谓项$\boldsymbol{P}$指称的类的元素。"特称肯定命题"这个名称是恰当的,因为这种命题肯定了类之间具有某种包含于关系,但不是全部而只是部分地(partially)肯定第一个类${ }^{(1)}$中的某个或某些元素包含于第二个类。

\paragraph{特称否定命题}
第四个例子——有政客不是说谎者———是一个\textbf{特称否定命题}。这个例子,正如上面的例子一样,谈论的并不是全部政客,而只是政客类中某个或某些元素,因而是特称的。不同于第三个例子的是,它并非肯定第一类中的某部分包含于第二个类中,相反,它是否定的。所有特称否定命题可以写成如下形式:

有$S$不是$P$。

它断言的是,主项$\boldsymbol{S}$指称的类中至少有一个元素被谓项$\boldsymbol{P}$指称的类的全体所排斥。

\subsection{标准式直言命题的多样性}

并非所有标准式直言命题都像以上四个例子那样简单明了。标准式命题的主项、谓项指称的都是类,但这些词项可能是复杂的表达式而非一个单词。举例来说,在命题"所有这个职位的候选人都是诚实而正直的人"中,主项是"这个职位的候选人",谓项是"诚实而正直的人"。

\begin{theorembox}[title=直言命题的传统符号系统]
\logicemph{历史观点}:曾经有一种传统观点,认为所有演绎论证都可以用类或范畴以及它们之间的关系加以分析。

\logicwarn{基石地位}:这样,如上说明的直言命题的四种标准形式,就被认为是所有演绎论证的基石:

\logicterm{四种标准符号}:
\begin{itemize}
  \item \logicterm{A命题}:全称肯定命题(所有$S$是$P$)
  \item \logicterm{E命题}:全称否定命题(没有$S$是$P$)
  \item \logicterm{I命题}:特称肯定命题(有$S$是$P$)
  \item \logicterm{O命题}:特称否定命题(有$S$不是$P$)
\end{itemize}

\logicemph{符号来源}:这些字母来自拉丁词:
\begin{itemize}
  \item A和I来自"Affirmo"(我肯定)
  \item E和O来自"Nego"(我否定)
\end{itemize}

\logicwarn{理论价值}:尽管这种传统观点是不正确的,但的确有许多逻辑理论——正如我们将要看到的——就是以这四种命题为基础建立起来的。

\logicemph{现代意义}:这个符号系统至今仍在逻辑学教学和研究中广泛使用。
\end{theorembox}

\footnotetext{(1)我国逻辑教材中一般把全称否定命题的形式写为:"所有$S$不是$P$",其与"没有$S$是$P$"同义。但根据英语语法,"All $S$ are not $P$"并不与"No $S$ are $P$"同义,而等义于"Not all $S$ are $P$"。故英文著作一般将"No $S$ is(are)$P$"作为全称否定命题的形式。}

\footnotetext{(1)括号内的话为译者所加。}

\chaptersummary{
\logicterm{直言命题}是关于范畴和类的特殊命题,是演绎理论的基石。类之间存在\logicemph{完全包含}、\logicemph{部分包含}和\logicemph{互相排斥}三种基本关系。直言命题有四种标准形式:\logicterm{全称肯定命题}(A命题)、\logicterm{全称否定命题}(E命题)、\logicterm{特称肯定命题}(I命题)和\logicterm{特称否定命题}(O命题)。
}