\section{其他直接推论}

\begin{quotation}
除了基于对当方阵的直接推论外,逻辑学还发展了其他几种重要的直接推论形式。本节将介绍三种主要的直接推论方法:换位法、换质法和换质位法,这些方法允许我们从一个标准式直言命题有效地推出另一个相关命题。
\end{quotation}

除了基于对当关系的直接推论,还有另外一些直接推论,本节我们来讨论其中的三种。

\subsection{换位法(Conversion)}
第一种叫做\textbf{换位法},它是一种仅仅通过交换命题中主、谓项的位置而进行的推论。对于 E 命题和 I 命题来说,换位法肯定是有效的。很显然,若断言"没有人是天使",也就可以断言"没有天使是人"。这两个命题可以通过换位法进行有效的互推。同样显然的是,"有作家是妇女"与"有妇女是作家"在逻辑上也是等价的,可以通过换位从其中一个有效地推出另一个。一个标准式直言命题叫做另一个的\textbf{换位命题},如果它是通过交换另一个命题的主、谓项的位置而得到的。例如,"没有理想主义者是政治家"是"没有政治家是理想主义者"的换位命题,它们可以通过换位有效地互推。换位法直接推论中的前提叫做\textbf{被换位命题}(convertend),结论叫做\textbf{换位命题}(converse)。

请注意,从被换位 A 命题不能普通有效地推出换位 A 命题。比如,已知"所有狗是动物",当然不能有效地推出它的换位命题"所有动物是狗",因为被换位命题为真,而换位命题为假。传统逻辑自然也认识到这一点,但它认为对于 A 命题进行某种类似换位的推论可以是有效的。如 5.4 节表明,依据对当方阵,从 A 命题(所有 $S$ 是 $P$ )可以有效地推出其相应的下位 I命题(有 $S$ 是 $P$ )。A 命题说的是 $S$ 类中全部元素的情况,而 I 命题则限制为只述及 $S$ 类中的部分元素的情况。我们已经知道 I 命题是可以有效换位的。

这样,给定一个 A 命题(所有 $S$ 是 $P$ ),就可以根据差等关系,有效地得到相应的下位命题(有 $S$ 是 $P$ ),而下位命题(有 $S$ 是 $P$ )又可以进行有效换位,因此,通过差等关系和换位法的结合,从所有 $\boldsymbol{S}$ 都是 $\boldsymbol{P}$ 就可有效地推出有 $\boldsymbol{P}$ 是 $\boldsymbol{S}$ 。这种推论称为\textbf{限制换位}(或"偶然换位"[conversion per accidens]),即交换主谓项的位置,同时将命题的量由全称改为特称。因此,按照传统逻辑的认识,"所有狗都是动物"可以有效地推出"有动物是狗",这个推论就是"限制换位"。下一节将进一步探讨这个问题。

请注意,作为限制换位结论的换位命题与原来的 A 命题并不等价,原因在于限制换位需要改变命题的量,把全称改为特称。因此,限制换位的结果不是一个 A 命题而是 I 命题,它不可能与被换位命题有同样的意义,从而不可能在逻辑上等价。但 E 命题的换位命题仍是一个 E 命题, I命题的换位命题仍是一个 I 命题,在这样的情况下,被换位命题与换位命题有同样的量,并且在逻辑上是等价的。

最后需要注意的是, O 命题的换位一般是无效的。 O 命题"有动物不是狗"很明显是真的,但它的换位命题"有狗不是动物"显然是假的。O命题与其换位命题并不等价。

一命题的换位命题总与原命题词项相同(只是位置互换),并且质也相同。下表是对传统换位推论的完整描述:

\begin{center}
\begin{tabular}{|l|l|}
\hline
\multicolumn{2}{|c|}{换位法} \\
\hline
被换位命题 & 换位命题 \\
\hline
A:所有 $S$ 是 $P$ & I:有 $P$ 是 $S$ \\
\hline
E :没有 S 是 $P$ & E :没有 $P$ 是 $S$ \\
\hline
I:有 $S$ 是 $P$ & I:有 $P$ 是 $S$ \\
\hline
O :有 $S$ 不是 $P$ & (换位无效) \\
\hline
\end{tabular}
\end{center}

\subsection{换质法(Obversion)}
接下来讨论的直接推论类型叫做\textbf{换质法}。在解说之前,我们先简要回顾一下"类"这个概念,并由此引人一些新的概念,以便更容易讨论换质法。一个类就是具有某种共同属性的所有对象的汇集。这种共同属性叫做 \textbf{类的定义特征}(class-defining characteristic)。举例来说,所有人的类就是所有具有"是人"这个特征的事物的集合,属性"是人"就是这个类的定义特征。类的定义特征不一定是"简单"的属性,任何一个属性都可以确定一个类。比如"左撒子、有红头发并且是学生"这个复杂属性就确定了一个类——所有是左撇子、有红头发的学生的类。

所有的类都有一个相应的\textbf{补类}(complementary class),或简称\textbf{补}(complement),即不属于原来的类的所有东西的汇集。比如,所有人的类的补就是所有不是人的东西组成的类。该类的定义特征是不是人这样一个(否定的)属性。所有人的类的补包括除人之外的所有东西:鞋子、轮船、封蜡和大白菜等等——但不包括国王,因为国王是人。把所有人的类的补称为"非人的类"更简洁一些,词项 $S$ 所指称的类的补则由词项非 $S$ 指称,因而可以说词项非 $S$ 就是词项 $S$ 的补。$^{[3]}$

我们在两种意义上使用"补"这个词:一是类意义上的补,二是词项的补。尽管两者有所不同,但却是密切联系着的。一个词项是另一词项的词项补,仅当第一个词项指称第二个词项所指称的类的补。应当说明的是,正如一个类是其(类)补的补一样,一个词项也是其(词项)补的补。其中用到了"双重否定"法则,这样就可以省去许多用做前缀的 "非"字。例如,如果把词项"选举人"的补写做"非选举人",而"非选举人"的补就简记为"选举人",而不是"非非选举人"。

必须注意不要把\textbf{反对词项}当做\textbf{互补词项},比如将"懦夫"等同于"非英雄"。没有既是懦夫又是英雄的人,但并非每个人--当然更不是任何东西--都必须或者是懦夫或者是英雄,所以词项"懦夫"与"英雄"之间是反对关系。再比如 "胜者"的补不是"败者"而是"非胜者",因为并非所有东西——或者说所有人——必须或是胜者或是败者。但每个东西必定或是胜者或是非胜者。

了解了词项补的含义,换质法就比较容易描述了。在换质法中,主项保持不变,被换质命题的量也不需改变。对一个命题进行换质,就是改变其质,并用谓项的补替换原来的谓项。例如下面这个 A 命题:

\begin{quote}
所有居民都是选举人。
\end{quote}

换质后成为一个 E 命题:

\begin{quote}
没有居民是非选举人。
\end{quote}

显然,这样两个命题在逻辑上是等价的,因此从一个可以有效地推出另一个。换质法应用到任何标准式直言命题,都是有效的直接推论。例如,下面的 E 命题:

\begin{quote}
没有仲裁人是偏心的。
\end{quote}

换质后得到一个等值的 A 命题:

\begin{quote}
所有仲裁人都是不偏心的。
\end{quote}

同样的,I 命题:

\begin{quote}
有金属是导体。
\end{quote}

换质后得到一个 O 命题:

\begin{quote}
有金属不是非导体。
\end{quote}

最后,O 命题:

\begin{quote}
有国家不是好战的。
\end{quote}

换质后得到一个 I 命题:

\begin{quote}
有国家是不好战的。
\end{quote}

换质法直接推论中的前提叫做\textbf{被换质命题}(obvertend),结论叫做\textbf{换质命题}(obverse)。所有标准式直言命题与其换质命题在逻辑上都是等价的,所以,对任何一个标准式直言命题而言,换质法都是有效的。要得到一个命题的换质命题,不需改变原命题的量和主项,而是要改变它的质,并用谓项的补替换原来的谓项。下表对传统上的换质推理进行了全面的刻画:

\begin{center}
\begin{tabular}{|l|l|}
\hline
\multicolumn{2}{|c|}{换质表} \\
\hline
被换质命题 & 换质命题 \\
\hline
A:所有 $S$ 是 $P$ & E :没有 $S$ 是非 $P$ \\
\hline
E :没有 S 是 $P$ & A:所有 $S$ 是非 $P$ \\
\hline
I:有 $S$ 是 $P$ & O :有 $S$ 不是非 $P$ \\
\hline
O :有 $S$ 不是 $P$ & I :有 $S$ 是非 $P$ \\
\hline
\end{tabular}
\end{center}

\subsection{换质位法(Contraposition)}
讨论第三种直接推论并不需要引人新的原理,从一定意义上讲,这种方法可以还原为前面两种推论。对给定的命题进行\textbf{换质位},就是将主项换为原命题谓项的补,并将其谓项换为原命题主项的补。例如,A命题:

\begin{quote}
所有会员都是选举人。
\end{quote}

换质位后是 A 命题:

\begin{quote}
所有非选举人都是非会员。
\end{quote}

容易见得,以上两个命题在逻辑上是等价的。对 A 命题进行换质位是有效的直接推论形式。对 A 命题首先换质,再换位,然后再换质,于是就从最初的"所有 $S$ 是 $P$"转化为"所有非 $P$ 是非 $S$"。因此,对任何一个 A 命题进行换质位,都是将原命题先换质,再换位,然后再换质。

换质位法用于 A 命题是最有用的,用于 O 命题也是有效的直接推论形式。例如对于 $O$ 命题:

\begin{quote}
有学生不是理想主义者。
\end{quote}

换质位后得到一个有点绕口的 O 命题:

\begin{quote}
有非理想主义者不是非学生。
\end{quote}

它与第一个命题在逻辑上是等价的。如果每次只转化一步,即先换质,再换位,再换质,那么就可以显示出其逻辑等价性。可把其中的推论用公式表示为:从"有 $S$ 不是 $P$"换质得"有 $S$ 是非 $P$",再换位得"有非 $P$ 是 $S$",继续换质得"有非 $P$ 不是非 $S$"(换质位命题)。

一般说来,换质位法对于 I 命题无效。用下面这个真的 I 命题可以证明这一点:

\begin{quote}
有公民是非议员。
\end{quote}

换质位后得到一个假命题:

\begin{quote}
有议员是非公民。
\end{quote}

其无效的原因在于对 I 命题进行换质位,就要对 I 命题先换质,再换位,然后再换质。I命题"有 $S$ 是 $P$"换质后得 O 命题"有 $S$ 不是非 $P$",而后者一般不能有效换位。

E 命题"没有 $S$ 是 $P$"的换质位命题是"没有非 $P$ 是非 $S$",这也不是从原命题有效地得出的,下面的例子可以说明这一点, E 命题:

\begin{quote}
没有摔跤运动员是体弱的人。
\end{quote}

其为真,但完全换质位命题却是假的:

\begin{quote}
没有非体弱的人是非挥跤运动员。
\end{quote}

为了得到其换质位命题,我们对 E 命题先进行换质,再换位,然后再换质,就可以找到无效的原因。 E 命题"没有 $S$ 是 $P$"换质后得 A 命题 "所有 $S$ 是非 $P$"。一般说来,A命题不能有效地换位,除非进行限制换位。于是,通过限制换位得"有非 $P$ 是 $S$",再换质得"有非 $P$ 不是非 $S$",我们称之为\textbf{限制换质位}。下一节我们将进一步讨论这个问题。

请注意,通过限制性换质位法,我们可从一个 E 命题推得一个 O 命题——即从"没有 $S$ 是 $P$"推出"有非 $P$ 不是非 $S$"——与限制换位有同样的特点。由于从全称命题只能推特称命题,结果得到的换质位命题与原命题意义不同,与作为原命题的 E 命题逻辑上不等价。而 A 命题的换质位命题仍是 A 命题, O 命题的换质位命题仍是 O 命题,在这两种情况下,换质位命题与其前提是等值的。

因此,换质位法只对 A 命题和 O 命题是有效的,对 I 命题是无效的,对 E 命题进行限制换质位才是有效的。也可以用一个图表来完整刻画这种推理:

\begin{center}
\begin{tabular}{|l|l|}
\hline
\multicolumn{2}{|c|}{换质位表} \\
\hline
前提 & 完全换质位命题 \\
\hline
A:所有 $S$ 是 $P$ & A:所有非 $P$ 是非 $S$ \\
\hline
E :没有 $S$ 是 $P$ & O :有非 $P$ 不是非 $S$(限制) \\
\hline
I:有 $S$ 是 $P$ & (换质位无效) \\
\hline
O :有 $S$ 不是 $P$ & O :有非 $P$ 不是非 $S$ \\
\hline
\end{tabular}
\end{center}

还有一些其他类型的直接推论,也都有各自的分类与特定名称,但并不需要引入新原理,我们在此就不再讨论了。

若要解决关于命题之间关系的某些问题,最好的方法就是研究从其中一个可以推得另一个的各种直接推论。例如,假定命题"所有外科医生都是内科医生"为真,是否可以推知"没有非外科医生是非内科医生"的真假情况?在此可以给出一个有用的方法,就是尽可能从给定命题推出多个有效结论,来看要考察的命题——或其矛盾命题和反对命题——是否能从为真的原命题有效地推出。上面的例子中,已知"所有 $S$ 是 $P$",我们可以有效地推出其换质位命题"所有非 $P$ 是非 $S$",再限制换位得"有非 $S$是非 $P$"——按照传统逻辑,它是已知命题的有效结论,因此是真的。根据逻辑方阵,它与被考察的命题"没有非 $S$ 是非 $P$"为矛盾关系,因此被考察的命题就是假的。

如 1.9 节所指出的,一个有效推理,如果前提为真,其结论必然为真。但如果前提为假,结论却可能为真。我们立即会联想到限制换位、限制换质位以及差等关系推理,它们正是后面一种情况的例子。例如,从假前提"所有动物是猫",根据差等关系推理,可以推出"有动物是猫"这样一个真结论。而假前提"所有父母都是学者"限制换位后也可以得到一个真结论"有学者是父母"。因此,如果已知一个命题为假,那么另一个(与之多少有些关系的)命题的真假情况就成了问题。比较好的方法是,从已知命题的矛盾命题或被考察命题本身着手。因为一个假命题的矛盾命题必然为真,所有从后者开始的有效推理也必然是真命题。而如果从被考察命题能够推出已知为假的命题,那么它本身也必然是假的。

\begin{center}
\begin{tabular}{|l|l|}
\hline
\multicolumn{2}{|c|}{换位法、换质法、完全换质位法} \\
\hline
\multicolumn{2}{|c|}{换位法} \\
\hline
被换位命题 & 换位命题 \\
\hline
A:所有 $S$ 是 $P$ & I:有 $P$ 是 $S$ \\
\hline
E :没有 S 是 $P$ & E :没有 $P$ 是 $S$ \\
\hline
I:有 $S$ 是 $P$ & I:有 $P$ 是 $S$ \\
\hline
O :有 $S$ 不是 $P$ & (换位无效) \\
\hline
\multicolumn{2}{|c|}{换质法} \\
\hline
被换质命题 & 换质命题 \\
\hline
A:所有 $S$ 是 $P$ & E :没有 $S$ 是非 $P$ \\
\hline
E :没有 $S$ 是 $P$ & A:所有 $S$ 是非 $P$ \\
\hline
I:有 $S$ 是 $P$ & O:有 $S$ 不是非 $P$ \\
\hline
O :有 $S$ 不是 $P$ & I:有 $S$ 是非 $P$ \\
\hline
\multicolumn{2}{|c|}{换质位法} \\
\hline
前提 & 完全换质位命题 \\
\hline
A:所有 $S$ 是 $P$ & A:所有非 $P$ 是非 $S$ \\
\hline
E :没有 $S$ 是 $P$ & O :有非 $P$ 不是非 $S$(限制) \\
\hline
I:有 $S$ 是 $P$ & (换质位无效) \\
\hline
O :有 $S$ 不是 $P$ & O :有非 $P$ 不是非 $S$ \\
\hline
\end{tabular}
\end{center}

\footnotetext{(3)我们有时用类的"相对补"(relative complement)来进行推论,即它的补包含在另外一个类中。比如:"我的孩子"这个类有一个子类"我的女儿",后者的补是另一个子类"我的不是女儿的孩子",即"我的儿子"的类。但换质法以及其他直接推论通常建基于类的绝对补之上,正如上面所定义的那样。}

\begin{center}
\fbox{\parbox{0.95\textwidth}{
\textbf{本节要点}
\begin{itemize}
\item \textbf{换位法}:交换命题的主谓项位置
  \begin{itemize}
  \item 对E和I命题总是有效的
  \item 对A命题仅限制换位有效
  \item 对O命题无效
  \end{itemize}
\item \textbf{换质法}:改变命题的质并用谓项的补替换原谓项
  \begin{itemize}
  \item 对所有标准式直言命题都有效
  \item 所有命题与其换质命题在逻辑上等价
  \end{itemize}
\item \textbf{换质位法}:用谓项的补替换主项,用主项的补替换谓项
  \begin{itemize}
  \item 对A和O命题有效
  \item 对I命题无效
  \item 对E命题只有限制换质位有效
  \end{itemize}
\item 通过这些直接推论方法,可以解决许多关于命题间关系的问题
\end{itemize}
}}
\end{center} 