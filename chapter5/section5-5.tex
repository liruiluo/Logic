\section*{5.5 其他直接推论}
除了基于对当关系的直接推论,还有另外一些直接推论,本节我们来讨论其中的三种。

\section*{A.换位法(Conversion)}
第一种叫做换位法,它是一种仅仅通过交换命题中主、谓项的位置而进行的推论。对于 E 命题和 I 命题来说,换位法肯定是有效的。很显然,若断言"没有人是天使",也就可以断言"没有天使是人"。这两个命题可以通过换位法进行有效的互推。同样显然的是,"有作家是妇女"与"有妇女是作家"在逻辑上也是等价的,可以通过换位从其中一个有效地推出另一个。一个标准式直言命题叫做另一个的换位命题,如果它是通过交换另一个命题的主、谓项的位置而得到的。例如,"没有理想主义者是政治家"是"没有政治家是理想主义者"的换位命题,它们可以通过换位有效地互推。换位法直接推论中的前提叫做被换位命题(convertend),结论叫做换位命题(converse)。

请注意,从被换位 A 命题不能普通有效地推出换位 A 命题。比如,已知"所有狗是动物",当然不能有效地推出它的换位命题"所有动物是狗",因为被换位命题为真,而换位命题为假。传统逻辑自然也认识到这一点,但它认为对于 A 命题进行某种类似换位的推论可以是有效的。如 5.4 节表明,依据对当方阵,从 A 命题(所有 $S$ 是 $P$ )可以有效地推出其相应的下位 I命题(有 $S$ 是 $P$ )。A 命题说的是 $S$ 类中全部元素的情况,而 I 命题则限制为只述及 $S$ 类中的部分元素的情况。我们已经知道 I 命题是可以有效换位的。

这样,给定一个 A 命题(所有 $S$ 是 $P$ ),就可以根据差等关系,有效地得到相应的下位命题(有 $S$ 是 $P$ ),而下位命题(有 $S$ 是 $P$ )又可以进行有效换位,因此,通过差等关系和换位法的结合,从所有 $\boldsymbol{S}$ 都是 $\boldsymbol{P}$ 就可有效地推出有 $\boldsymbol{P}$ 是 $\boldsymbol{S}$ 。这种推论称为限制换位(或"偶然换位"[conver-\\
sion per accidens]),即交换主谓项的位置,同时将命题的量由全称改为特称。因此,按照传统逻辑的认识,"所有狗都是动物"可以有效地推出"有动物是狗",这个推论就是"限制换位"。下一节将进一步探讨这个问题。

请注意,作为限制换位结论的换位命题与原来的 A 命题并不等价,原因在于限制换位需要改变命题的量,把全称改为特称。因此,限制换位的结果不是一个 A 命题而是 I 命题,它不可能与被换位命题有同样的意义,从而不可能在逻辑上等价。但 E 命题的换位命题仍是一个 E 命题, I命题的换位命题仍是一个 I 命题,在这样的情况下,被换位命题与换位命题有同样的量,并且在逻辑上是等价的。

最后需要注意的是, O 命题的换位一般是无效的。 O 命题"有动物不是狗"很明显是真的,但它的换位命题"有狗不是动物"显然是假的。O命题与其换位命题并不等价。

一命题的换位命题总与原命题词项相同(只是位置互换),并且质也相同。下表是对传统换位推论的完整描述:

\section*{B.换质法(Obversion)}
接下来讨论的直接推论类型叫做换质法。在解说之前,我们先简要回顾一下"类"这个概念,并由此引人一些新的概念,以便更容易讨论换质法。一个类就是具有某种共同属性的所有对象的汇集。这种共同属性叫做 "类的定义特征"(class-defining characteristic)。举例来说,所有人的类就是所有具有"是人"这个特征的事物的集合,属性"是人"就是这个类的定义特征。类的定义特征不一定是"简单"的属性,任何一个属性都可以确定一个类。比如"左撒子、有红头发并且是学生"这个复杂属性就确定了一个类——所有是左撇子、有红头发的学生的类。

所有的类都有一个相应的补类(complementary class),或简称补 (complement),即不属于原来的类的所有东西的汇集。比如,所有人的类

的补就是所有不是人的东西组成的类。该类的定义特征是不是人这样一个 (否定的)属性。所有人的类的补包括除人之外的所有东西:鞋子、轮船、封蜡和大白菜等等——但不包括国王,因为国王是人。把所有人的类的补称为"非人的类"更简洁一些,词项 $S$ 所指称的类的补则由词项非 $S$ 指称,因而可以说词项非 $S$ 就是词项 $S$ 的补。 ${ }^{[3]}$

我们在两种意义上使用"补"这个词:一是类意义上的补,二是词项的补。尽管两者有所不同,但却是密切联系着的。一个词项是另一词项的词项补,仅当第一个词项指称第二个词项所指称的类的补。应当说明的是,正如一个类是其(类)补的补一样,一个词项也是其(词项)补的补。其中用到了"双重否定"法则,这样就可以省去许多用做前缀的 "非"字。例如,如果把词项"选举人"的补写做"非选举人",而"非选举人"的补就简记为"选举人",而不是"非非选举人"。必须注意不要把反对词项当做互补词项,比如将"懦夫"等同于"非英雄"。没有既是懦夫又是英雄的人,但并非每个人--当然更不是任何东西--都必须或者是懦夫或者是英雄,所以词项"懦夫"与"英雄"之间是反对关系。再比如 "胜者"的补不是"败者"而是"非胜者",因为并非所有东西——或者说所有人——必须或是胜者或是败者。但每个东西必定或是胜者或是非胜者。

了解了词项补的含义,换质法就比较容易描述了。在换质法中,主项保持不变,被换质命题的量也不需改变。对一个命题进行换质,就是改变其质,并用谓项的补替换原来的谓项。例如下面这个 A 命题:

所有居民都是选举人。

换质后成为一个 E 命题:

没有居民是非选举人。

泉然,这样两个命题在逻辑上是等价的,因此从一个可以有效地推出另一个。换质法应用到任何标准式直言命题,都是有效的直接推论。例如,下面的 E 命题:

换质后得到一个等值的 A 命题:

所有仲裁人都是不偏心的。

同样的,I 命题:

有金属是导体。

换质后得到一个 O 命题:

有金属不是非导体。

最后,()命题:

有国家不是好战的。

换质后得到一个 I 命题:

有国家是不好战的。

换质法直接推论中的前提叫做被换质命题(obvertend),结论叫做换质命题(obverse)。所有标准式直言命题与其换质命题在逻辑上都是等价的,所以,对任何一个标准式直言命题而言,换质法都是有效的。要得到一个命题的换质命题,不需改变原命题的量和主项,而是要改变它的质,并用谓项的补替换原来的谓项。下表对传统上的换质推理进行了全面的刻画:

\begin{center}
\begin{tabular}{|l|l|}
\hline
\multicolumn{2}{|c|}{换质表} \\
\hline
被换质命题 & 换质命题 \\
\hline
A:所有 $S$ 是 $P$ & E :没有 $S$ 是非 $P$ \\
\hline
E :没有 S 是 $P$ & A:所有 $S$ 是非 $P$ \\
\hline
I:有 $S$ 是 $P$ & O :有 $S$ 不是非 $P$ \\
\hline
O :有 $S$ 不是 $P$ & I :有 $S$ 是非 $P$ \\
\hline
\end{tabular}
\end{center}

\section*{C.换质位法(Contraposition)}
讨论第三种直接推论并不需要引人新的原理,从一定意义上讲,这种方法可以还原为前面两种推论。对给定的命题进行换质位,就是将主项换为原命题谓项的补,并将其谓项换为原命题主项的补。例如,A命题:

所有会员都是选举人。

换质位后是 A 命题:

所有非选举人都是非会员。

容易见得,以上两个命题在逻辑上是等价的。对 A 命题进行换质位是有效的直接推论形式。对 A 命题首先换质,再换位,然后再换质,于是就从最初的"所有 $S$ 是 $P$"转化为"所有非 $P$ 是非 $S$"。因此,对任何一个 A 命题进行换质位,都是将原命题先换质,再换位,然后再换质。

换质位法用于 A 命题是最有用的,用于 O 命题也是有效的直接推论形式。例如对于 $O$ 命题:

有学生不是理想主义者。

换质位后得到一个有点绕口的 O 命题:

有非理想主义者不是非学生。

它与第一个命题在逻辑上是等价的。如果每次只转化一步,即先换质,再换位,再换质,那么就可以显示出其逻辑等价性。可把其中的推论用公式表示为:从"有 $S$ 不是 $P$"换质得"有 $S$ 是非 $P$",再换位得"有非 $P$ 是 $S$",继续换质得"有非 $P$ 不是非 $S$"(换质位命题)。

一般说来,换质位法对于 I 命题无效。用下面这个真的 I 命题可以证明这一点:

有公民是非议员。

换质位后得到一个假命题:

有议员是非公民。

其无效的原因在于对 I 命题进行换质位,就要对 I 命题先换质,再换位,然后再换质。I命题"有 $S$ 是 $P$"换质后得 O 命题"有 $S$ 不是非 $P$",而后者一般不能有效换位。

E 命题"没有 $S$ 是 $P$"的换质位命题是"没有非 $P$ 是非 $S$",这也不是从原命题有效地得出的,下面的例子可以说明这一点, E 命题:

没有摔跤运动员是体弱的人。

其为真,但完全换质位命题却是假的:

没有非体弱的人是非挥跤运动员。

为了得到其换质位命题,我们对 E 命题先进行换质,再换位,然后再换质,就可以找到无效的原因。 E 命题"没有 $S$ 是 $P$"换质后得 A 命题 "所有 $S$ 是非 $P$"。一般说来,A命题不能有效地换位,除非进行限制换位。于是,通过限制换位得"有非 $P$ 是 $S$",再换质得"有非 $P$ 不是非 $S$",我们称之为"限制换质位"。下一节我们将进一步讨论这个问题。

请注意,通过限制性换质位法,我们可从一个 E 命题推得一个 O 命题——一即从"没有 $S$ 是 $P$"推出"有非 $P$ 不是非 $S$"——与限制换位有同样的特点。由于从全称命题只能推特称命题,结果得到的换质位命题与原命题意义不同,与作为原命题的 E 命题逻辑上不等价。而 A 命题的换质位命题仍是 A 命题, O 命题的换质位命题仍是 O 命题,在这两种情况下,换质位命题与其前提是等值的。

因此,换质位法只对 A 命题和 O 命题是有效的,对 I 命题是无效的,对 E 命题进行限制换质位才是有效的。也可以用一个图表来完整刻画这种

\begin{center}
\begin{tabular}{|l|l|}
\hline
\multicolumn{2}{|c|}{换质位表} \\
\hline
前提 & 完全换质位命题 \\
\hline
A:所有 $S$ 是 $P$ & A:所有非 $P$ 是非 $S$ \\
\hline
E :没有 $S$ 是 $P$ & O :有非 $P$ 不是非 $S$(限制) \\
\hline
I:有 $S$ 是 $P$ & (换质位无效) \\
\hline
O :有 $S$ 不是 $P$ & O :有非 $P$ 不是非 $S$ \\
\hline
\end{tabular}
\end{center}

还有一些其他类型的直接推论,也都有各自的分类与特定名称,但并不需要引人新原理,我们在此就不再讨论了。

若要解决关于命题之间关系的某些问题,最好的方法就是研究从其中一个可以推得另一个的各种直接推论。例如,假定命题"所有外科医生都是内科医生"为真,是否可以推知"没有非外科医生是非内科医生"的真假情况?在此可以给出一个有用的方法,就是尽可能从给定命题推出多个有效结论,来看要考察的命题——或其矛盾命题和反对命题——是否能从为真的原命题有效地推出。上面的例子中,已知"所有 $S$ 是 $P$",我们可以有效地推出其换质位命题"所有非 $P$ 是非 $S$",再限制换位得"有非 $S$是非 $P$"——按照传统逻辑,它是已知命题的有效结论,因此是真的。根据逻辑方阵,它与被考察的命题"没有非 $S$ 是非 $P$"为矛盾关系,因此被考察的命题就是假的。

如 1.9 节所指出的,一个有效推理,如果前提为真,其结论必然为真。但如果前提为假,结论却可能为真。我们立即会联想到限制换位、限制换质位以及差等关系推理,它们正是后面一种情况的例子。例如,从假前提"所有动物是猫",根据差等关系推理,可以推出"有动物是猫"这样一个真结论。而假前提"所有父母都是学者"限制换位后也可以得到一个真结论"有学者是父母"。因此,如果已知一个命题为假,那么另一个(与之多少有些关系的)命题的真假情况就成了问题。比较好的方法是,从已知命题的矛盾命题或被考察命题本身着手。因为一个假命题的矛盾命题必然为真,所有从后者开始的有效推理也必然是真命题。而如果从被考察命题能够推出已知为假的命题,那么它本身也必然是假的。

\begin{center}
\begin{tabular}{|l|l|}
\hline
\multicolumn{2}{|c|}{换位法、换质法、完全换质位法} \\
\hline
\multicolumn{2}{|c|}{换位法} \\
\hline
被换位命题 & 换位命题 \\
\hline
A:所有 $S$ 是 $P$ & I:有 $P$ 是 $S$ \\
\hline
E :没有 S 是 $P$ & E :没有 $P$ 是 $S$ \\
\hline
I:有 $S$ 是 $P$ & I:有 $P$ 是 $S$ \\
\hline
O :有 $S$ 不是 $P$ & (换位无效) \\
\hline
\multicolumn{2}{|c|}{换质法} \\
\hline
被换质命题 & 换质命题 \\
\hline
A:所有 $S$ 是 $P$ & E :没有 $S$ 是非 $P$ \\
\hline
E :没有 $S$ 是 $P$ & A:所有 $S$ 是非 $P$ \\
\hline
I:有 $S$ 是 $P$ & O:有 $S$ 不是非 $P$ \\
\hline
O :有 $S$ 不是 $P$ & I:有 $S$ 是非 $P$ \\
\hline
\multicolumn{2}{|c|}{换质位法} \\
\hline
前提 & 完全换质位命题 \\
\hline
A:所有 $S$ 是 $P$ & A:所有非 $P$ 是非 $S$ \\
\hline
E :没有 $S$ 是 $P$ & O :有非 $P$ 不是非 $S$(限制) \\
\hline
I:有 $S$ 是 $P$ & (换质位无效) \\
\hline
O :有 $S$ 不是 $P$ & O :有非 $P$ 不是非 $S$ \\
\hline
\end{tabular}
\end{center}
\footnotetext{(3)我们有时用类的"相对补"(relative complement)来进行推论,即它的补包含在另外一个类中。比如:"我的孩子"这个类有一个子类"我的女儿",后者的补是另一个子类"我的不是女儿的孩子",即"我的儿子"的类。但换质法以及其他直接推论通常建基于类的绝对补之上,正如上面所定义的那样。} 