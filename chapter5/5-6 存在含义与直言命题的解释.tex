\section{存在含义与直言命题的解释}

\begin{logicbox}[title=章节导读]
直言命题的解释关系到逻辑推理的正确性。本节讨论一个关键问题:直言命题是否含有存在预设?通过分析传统解释与现代布尔解释的区别,我们将了解不同解释如何影响对当方阵及直接推论的有效性,从而为后续对三段论的分析奠定基础。
\end{logicbox>

进一步分析和评估由直言命题构成的论证,需要对它们进行图示与符号化。但是,把 A、E、I、O 命题符号化,必定会遇到而且必须解决一个深层的逻辑问题——一个上千年长期争论的问题。本节我们就来说明这个问题,同时提供一种解决方案。以此为基础,也可以对三段论做出融贯的分析。

首先要说明的是,这并不是一个简单的问题。但只要我们弄清如下关于直言命题的解释(称为\textbf{布尔解释}[Boolean interpretation]),则后面关于三段论的分析并不需要对有关争议的深度把握。如果能掌握本节最后所总结的讨论结果,就可以顺利越过此前的复杂讨论。

要理解这个结果,必须弄清有些命题有\textbf{存在含义}(existential import),有些则没有。如果说出一个命题就肯定了某种对象的存在,那么就说这个命题有存在含义。

为什么初学逻辑就要关心这个看上去很深奥的问题呢?这是因为,特定论证中所用的命题中是否有存在含义,将直接影响到该论证中推理的正确性。对直言命题必须有一个清晰、融贯的解释,以便能确定什么东西可以从它们正确地推出,同时避免错误推论。

\subsection{特称命题的存在含义}

先看 I 命题和 O 命题,它们肯定有存在含义。例如 I 命题"有士兵是英雄"说的是至少存在一个是英雄的士兵。而 O 命题"有狗不是同伴"说的是至少存在一个不是同伴的狗。特称命题 I 和 O,一般说来,确实断定了主项(例句中的士兵和狗)指称的类不为空——士兵的类和狗的类(如果给出的例子为真的话)中至少有一个元素。\cite{russell1905}

\subsection{传统解释的困境}

如果确实如此,即如果 I 和 O 命题有存在含义(没人会否认),会有什么问题呢?问题在于这种状况的后果令人十分不安。先前我们已经说过,通过差等关系推论,I 命题可以从相应的 A 命题有效地推出,也就是说,从"所有蜘蛛都是八脚动物"可以有效地推出"有蜘蛛是八脚动物"。同样,我们说 O 命题可以有效地从 E 命题推得。但如果 I 和 O 命题有存在含义,而它们分别是从 A 和 E 命题得到的,那么 A 和 E 命题必定也要有存在含义。因为一个有存在含义的命题不可能有效地从另一个没有同样含义的命题得到。\cite{russell1905}

这种结果造成了一个严重的问题。我们知道在传统逻辑方阵中,A 和 O 命题是矛盾关系。"所有丹麦人都说英语"与"有丹麦人不说英语"是互为矛盾的。具有矛盾关系的命题不可同真,因为其中必有一假。两者也不可同假,因为其中必有一真。但如果像上文总结的那样,对应的 A 和 O命题确实有存在含义的话,那么,两个矛盾命题就可能同时为假!举例来说,A命题"所有火星人都是金发碧眼的"与其对应的 O 命题"有火星人不是金发碧眼的"互为矛盾,如果它们都有存在含义的话——即我们要把它们看做都断言存在火星人的话——那么,如果火星上没有居民则两个命题都是假的。我们当然知道火星上没有人,火星人的类是空类,据此上述例子中给出的两个命题都是假的。而如果两者都是假的,它们就不可能是矛盾关系!

由此看来,传统对当方阵是有不妥之处的。假如它所说 A 和 E 命题有效地蕴涵相应的 I 和 O 命题是正确的话,那么,它断言 A 和 O 命题之间有矛盾关系就不正确了,同样,认为 I 和 O 命题为下反对关系也是不正确的。

\subsection{全面存在预设的尝试}

那么我们该怎么办呢?传统逻辑方阵还能否加以挽救?挽救是可以的,但代价很高。我们可以引入\textbf{预设}(presupposition)概念来恢复逻辑方阵的地位。我们早已注意到(见 4.3 节),对于一些复杂问语,只有已经预设了先行问题的答案,才能适当地回答"是"或"否"。只有预设了你偷过钱是真的,才能用"是"或"否"来回答"你把偷来的钱花光了吗"这样的问题,否则是不合理的。

现在,为挽救传统逻辑方阵,我们可以主张所有直言命题,即四种标准式命题 A、E、I、O——都预设(在上述含义下)它们涉及的类均不为空,即都有元素。也就是说,要使命题的真假情况以及它们之间的逻辑关系都成立并可以得到合理的解答(在这种解释下),就必须预设它们绝不涉及空类。这样,就可以保留传统对当方阵中构建的各种关系:A 与 E 仍是反对关系,I 与 O 仍是下反对关系,A与 O、E 与 I 仍是矛盾关系。然而,为了保证这个结果,必须诉诸其\textbf{全面存在预设}(blanket presupposition),即预设全部词项指称的类(及其补类)都有元素,都不为空。\cite{wiebe1991}

那么,我们为什么不能就此罢休呢?存在预设对于挽救亚里士多德型逻辑既是必要的也是充分的。而且,预设在很多情况下与现代语言的日常用法是一致的。如果有人告诉你说"桶里的苹果都是甜的",而你向桶里一看,却什么都没有,那你会怎么说?你可能不会说刚才的话是假的或真的,而是指出这里没有苹果。你会解释说,说话人犯了一个错误,即当时的存在预设(桶里有苹果)是假的。事实上,这种纠正已经表明我们理解并基本接受了日常语言中的预设。

\subsection{为何全面存在预设不可接受}

然而不幸的是,用来挽救逻辑方阵的这种全面预设却要付出一个过重的代价,是我们不能接受的。我们有充分的理由不这样做,在此列举三条理由。

首先,引人预设确实能够保留 A、E、I 和 O 之间的对当关系,但却付出了不能刻画某些我们需要的断言的代价,即不能再刻画那些否定有元素存在的命题了。而这样的否定有时非常重要,是必须明确的。

其次,即使是日常语言的用法,也并不完全与全面存在预设一致,有时我们说的话并不假定所谈的类中有元素。例如,你说"所有非法侵入者都要被起诉",这句话根本不预设非法侵人者的类中已经有元素,相反,你这样说正是为了保证这个类维持空类。

再次,在科学界及其他理论界,我们通常希望进行没有任何存在预设的推理。例如牛顿第一运动定律断定的是不受任何外力作用的物体必然保持静止状态或匀速直线运动。这种定律可以是真的,而物理学家表述它并为它辩护的时候,并没有预设不受任何外力作用的物体存在。

这些问题的存在使得上述全面存在预设不能为现代逻辑学家所接受。我们应当放弃曾长期被认为是正确的亚里士多德型解释,而采用关于直言命题的现代解释。

\subsection{布尔解释}

直言命题的现代解释不再假定我们言说的类中必定有元素。拒绝这种假定的解释称为\textbf{布尔解释}。英国逻辑学家、数学家乔治•布尔(George Boole,1815-1864)是现代符号逻辑奠基人之一,这种新的解释就是以他的名字命名的。\cite{boole1854}

在本书以下部分,我们均采纳关于直言命题的布尔解释。现在我们就来阐明这种解释:

1.在某些方面,传统解释仍然成立。$\mathbf{I}$ 和 $\mathbf{O}$ 命题在布尔解释中仍然有存在含义。所以,如果 $S$ 类为空,那么,命题"有 $S$ 是 $P$"为假,命题 "有 $S$ 不是 $P$"也为假。

2.全称命题 $\mathbf{A}$ 和 $\mathbf{E}$ 与特称命题 $\mathbf{O}$ 和 $\mathbf{I}$ 之间的矛盾关系也保持为真。也就是说,命题"所有人是会死的"与"有人不是会死的"互为矛盾,而命题"没有神灵是会死的"与"有神灵是会死的"亦互为矛盾。

3.在布尔解释中上述关系是完全融贯的,这是因为,\textbf{全称命题被解释为没有存在含义}。因此,即使 $S$ 类为空,命题"所有 $S$ 是 $P$"仍可以为真,"没有 $S$ 是 $P$"也可以为真。例如,即使独角兽不存在,"所有独角兽是有角的"与"没有独角兽是有翅膀的"都可以为真。而如果不存在独角兽,I 命题"有独角兽是有角的"就是假的,O 命题"有独角兽不是有翅膀的"同样为假。

4.在日常话语中,有时我们说出一个全称命题,确实假定了某事物的存在。当然,布尔解释也允许有这种表述,但要求用两个命题来表述,一个是有存在含义的特称命题,加之一个没有存在含义的全称命题。

5.采纳布尔解释会带来一些重要变化。相应的 $\mathbf{A、E}$ 命题可以同真,因此它们之间不再是反对关系。这似乎有点怪异,在后面 10.2 和 10.3 部分将给出详细的说明。现在弄清如下这点就足够了:在布尔解释中,"所有独角兽是有翅膀的"断言的是"如果有独角兽,那么,它是有翅膀的"。而"没有独角兽是有翅膀的"断定的是"如果有独角兽,那么,它是没有翅膀的"。如果确实不存在独角兽,这两个"如果……那么……"型的命题都可以为真。

6.类似的,在布尔解释中,因为 I 和 O 命题确实有存在含义。所以,如果主项指称的类为空,相应的 I 和 O 命题都是假的,因此相应的 I 和 O命题之间也不再是下反对关系。

7.在布尔解释中,差等关系——从 A 命题推出相应的 I 命题,从 E命题推出相应的 O 命题——不是普遍有效的。从一个没有存在含义的命题当然不能得出一个有存在含义的命题。

8.布尔解释保留了一些直接推理:$\mathbf{E}$ 命题和 $\mathbf{I}$ 命题的换位推理,$\mathbf{A}$ 命题和 O 命题的换质位推理,所有命题的换质推理。但限制换位、限制换质位推理不再有效。

9.在布尔解释下,逻辑方阵转变为如下情形:方阵周边的关系不再成立,而对角线上的矛盾关系保持不变。

简言之,现代逻辑学家否定了全面存在预设。对于一个不能明确断定其中有元素的类,我们就不能假定它有元素,否则就是错的。任何依据这种错误假定的论证都会产生\textbf{存在预设谬误},简称为\textbf{存在谬误}。现在有了清晰的布尔解释,我们就可以构造一个有力的体系,将标准式直言命题推理符号化、图示化。



\begin{center}
\fbox{\parbox{0.95\textwidth}{
\textbf{本节要点}
\begin{itemize}
\item \textbf{存在含义}:指命题肯定了某种对象的存在
  \begin{itemize}
  \item 特称命题(I和O)肯定有存在含义
  \item 如果主项指称的类为空,特称命题为假
  \end{itemize}
\item \textbf{传统解释的困境}:
  \begin{itemize}
  \item 若A和E命题有存在含义,当主项为空类时,A和O命题可同假
  \item 导致对当方阵中的矛盾关系不成立
  \end{itemize}
\item \textbf{布尔解释}的主要特点:
  \begin{itemize}
  \item 全称命题(A和E)没有存在含义
  \item 如果主项指称的类为空,全称命题可为真
  \item 保留了对角矛盾关系,但取消了对当方阵周边关系
  \item 差等关系(从A到I,从E到O)不再普遍有效
  \item 保留了E和I命题的换位推理,A和O命题的换质位推理,以及所有命题的换质推理
  \end{itemize}
\item \textbf{存在谬误}:依据错误的存在预设进行推理的谬误
\end{itemize}
}}
\end{center}