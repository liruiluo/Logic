\section{直言命题及其类别}

\begin{quotation}
\textit{直言命题是演绎理论的基石,它们关注类与类之间的关系。本节将介绍直言命题的基本类型及其特征,为理解演绎论证奠定基础。}
\end{quotation}

\subsection{直言命题的基本概念}

亚里士多德对演绎的研究主要集中在由一种特殊命题组成的论证上,这种命题是关于范畴(categories)和类(classes)的,被称为\textbf{直言命题}(categorical proposition)。直言命题是演绎理论的基石。要了解这种关于类的演绎理论,必须首先对直言命题进行非常精细的分析。请考虑如下论证:

\begin{quote}
没有运动员是素食主义者,所有足球队员都是运动员,
\end{quote}

\begin{quote}
所以,没有足球队员是素食主义者。
\end{quote}

这个论证中的三个命题都是直言命题,包括两个前提、一个结论。这些命题肯定或否定某个类$\boldsymbol{S}$全部或部分地包含于另一个类$\boldsymbol{P}$之中。三个命题涉及的是运动员的类、素食者的类和足球队员的类。

\subsection{类之间的关系}

有关类的知识在第3章讨论定义时已经简要地说明,一个\textbf{类}就是共有某种特定属性的所有对象(objects)的汇集。两个类之间有着多种不同的关系。

1.如果一个类的所有元素(member)都是另一个类的元素,例如狗的类与哺乳动物的类,则称第一个类包含于(be included)或包括在(be contained)第二个类之中;

2.如果一个类中有元素是另一个类的元素,但并非其所有元素都是另一个类的元素,例如女人的类和运动员的类,则称第一个类部分地包含于第二个类之中;

3.如果两个类没有共同的元素,例如三角形的类和圆形的类,则称这两个类之间是相互排斥(exclude)的。

\subsection{直言命题的四种标准形式}

类与类之间的这些关系被直言命题所肯定或否定,其结果是恰好能形成直言命题的四种标准形式,可分别由如下标准命题例示:

1.所有政客是说谎者。\\
2.没有政客是说谎者。\\
3.有政客是说说者。\\
4.有政客不是说谎者。

下面我们就细致地考察直言命题这四种标准形式。

\paragraph{全称肯定命题}
第一个例子——所有政客是说谎者——是一个\textbf{全称肯定命题}。其中涉及两个类,即政客的类和说谎者的类,它说的是第一个类包含于或包括在第二个类中。全称肯定命题断言第一个类中所有元素都是第二个类的元素。在这个例子中,主项"政客"指称(designate)政客的类,谓项"说谎者"指称说谎者的类。所有全称肯定命题都可以写成如下形式:

所有$S$是$P$。

其中字母$S$和$P$分别代表主项和谓项。"全称肯定命题"这一名称是恰当的,因为这个命题肯定了两个类之间的包含于关系,并且是完全或者说全部包含于关系:断言$S$的所有元素同时都是$P$的元素。

\paragraph{全称否定命题}
第二个例子——没有政客是说谎者——是一个\textbf{全称否定命题}。它是对全部政客而言,否定他们是说谎者。就这样两个类来说,全称否定命题断言第一个类与第二个类是完全排斥的,也就是说第一个类中没有元素是第二个类的元素。所有全称否定命题都可以写成如下形式:

没有$S$是$P$。

其中$S$和$P$也分别代表主项和谓项。"全称否定命题"这一名称是恰当的,因为这个命题否定了这两个类之间的包含于关系——并且是全部否定:断言在$S$的所有元素中,没有一个是$P$的元素。${ }^{(1)}$

\paragraph{特称肯定命题}
第三个例子—有政客是说谎者——是一个\textbf{特称肯定命题}。显然,这个例子肯定的是政客类中有元素(也)是说谎者类的元素。但并没有对政客类作全部断言:它说的并不是所有政客,而是某个或某些政客是说谎者。此命题既没有肯定也没有否定所有政客是说谎者,对此并没有给出主张。从字面含义看,它并没有断言有政客不是说谎者,尽管在某些语境中它可能暗含这样的意思。这个命题的字面含义或者说最小的(minimal)解释,即政客的类和说谎者的类之间有某个或某些元素是共同的。为确定性起见,我们这里采取最小解释。

"有"(some)这个词的含义是不确定的。它指的是"至少有一个"、"至少有两个",还是"至少有一百个"呢?到底有多少个?尽管与某些场合中的通常用法不太一致,但为了保持确定性,我们一般把"有"看做"至少有一个"的意思。这样,特称肯定命题可以写成如下形式:

有$S$是$P$。

它断言的是,主项$\boldsymbol{S}$指称的类中至少有一个元素是谓项$\boldsymbol{P}$指称的类的元素。"特称肯定命题"这个名称是恰当的,因为这种命题肯定了类之间具有某种包含于关系,但不是全部而只是部分地(partially)肯定第一个类${ }^{(1)}$中的某个或某些元素包含于第二个类。

\paragraph{特称否定命题}
第四个例子——有政客不是说谎者———是一个\textbf{特称否定命题}。这个例子,正如上面的例子一样,谈论的并不是全部政客,而只是政客类中某个或某些元素,因而是特称的。不同于第三个例子的是,它并非肯定第一类中的某部分包含于第二个类中,相反,它是否定的。所有特称否定命题可以写成如下形式:

有$S$不是$P$。

它断言的是,主项$\boldsymbol{S}$指称的类中至少有一个元素被谓项$\boldsymbol{P}$指称的类的全体所排斥。

\subsection{标准式直言命题的多样性}

并非所有标准式直言命题都像以上四个例子那样简单明了。标准式命题的主项、谓项指称的都是类,但这些词项可能是复杂的表达式而非一个单词。举例来说,在命题"所有这个职位的候选人都是诚实而正直的人"中,主项是"这个职位的候选人",谓项是"诚实而正直的人"。

\subsection{直言命题的传统符号}

曾经有一种传统观点,认为所有演绎论证都可以用类或范畴以及它们之间的关系加以分析。这样,如上说明的直言命题的四种标准形式,就被认为是所有演绎论证的基石:

\begin{quote}
全称肯定命题(称为\textbf{A命题})\\
全称否定命题(称为\textbf{E命题})\\
特称肯定命题(称为\textbf{I命题})\\
特称否定命题(称为\textbf{O命题})
\end{quote}

(尽管这种传统观点是不正确的,但)${ }^{(1)}$ 的确有许多逻辑理论——正如我们将要看到的——就是以这四种命题为基础建立起来的。

\footnotetext{(1)我国逻辑教材中一般把全称否定命题的形式写为:"所有$S$不是$P$",其与"没有$S$是$P$"同义。但根据英语语法,"All $S$ are not $P$"并不与"No $S$ are $P$"同义,而等义于"Not all $S$ are $P$"。故英文著作一般将"No $S$ is(are)$P$"作为全称否定命题的形式。}

\footnotetext{(1)括号内的话为译者所加。} 

\begin{center}
\fbox{\parbox{0.95\textwidth}{
\textbf{本节要点}
\begin{itemize}
\item \textbf{直言命题}是关于范畴和类的特殊命题,是演绎理论的基石
\item 类之间的三种基本关系:
  \begin{itemize}
  \item 完全包含于关系(如狗类与哺乳动物类)
  \item 部分包含于关系(如女人类与运动员类)
  \item 互相排斥关系(如三角形类与圆形类)
  \end{itemize}
\item 直言命题的四种标准形式:
  \begin{itemize}
  \item \textbf{全称肯定命题}(A命题):所有S是P
  \item \textbf{全称否定命题}(E命题):没有S是P
  \item \textbf{特称肯定命题}(I命题):有S是P
  \item \textbf{特称否定命题}(O命题):有S不是P
  \end{itemize}
\item 标准式直言命题的主项、谓项可以是复杂表达式
\item "有"(some)在逻辑中解释为"至少有一个"
\item 直言命题是许多逻辑理论的基础
\end{itemize}
}}
\end{center} 