\section*{5.2 直言命题及其类别}
亚里士多德对演绎的研究主要集中在由一种特殊命题组成的论证上,这种命题是关于范畴(categories)和类(classes)的,被称为"直言 (categorical)命题"。直言命题是演绎理论的基石。要了解这种关于类的演绎理论,必须首先对直言命题进行非常精细的分析。请考虑如下论证:

\begin{displayquote}
没有运动员是素食主义者,所有足球队员都是运动员,
\end{displayquote}

\begin{displayquote}
所以,没有足球队员是素食主义者。
\end{displayquote}

这个论证中的三个命题都是直言命题,包括两个前提、一个结论。这些命题肯定或否定某个类 $\boldsymbol{S}$ 全部或部分地包含于另一个类 $\boldsymbol{P}$ 之中。三个命题涉及的是运动员的类、素食者的类和足球队员的类。

有关类的知识在第 3 章讨论定义时已经简要地说明,一个类就是共有某种特定属性的所有对象(objects)的汇集。两个类之间有着多种不同的关系。

1.如果一个类的所有元素(member)都是另一个类的元素,例如狗的类与哺乳动物的类,则称第一个类包含于(be included)或包括在(be contained)第二个类之中;

2.如果一个类中有元素是另一个类的元素,但并非其所有元素都是另一个类的元素,例如女人的类和运动员的类,则称第一个类部分地包含于第二个类之中;

3.如果两个类没有共同的元素,例如三角形的类和圆形的类,则称这两个类之间是相互排斥(exclude)的。

类与类之间的这些关系被直言命题所肯定或否定,其结果是恰好能形成直言命题的四种标准形式,可分别由如下标准命题例示:

1.所有政客是说谎者。\\
2.没有政客是说谎者。\\
3.有政客是说说者。\\
4.有政客不是说谎者。

下面我们就细致地考察直言命题这四种标准形式。\\
第一个例子——所有政客是说谎者——是一个全称肯定命题。其中涉及两个类,即政客的类和说谎者的类,它说的是第一个类包含于或包括在第二个类中。全称肯定命题断言第一个类中所有元素都是第二个类的元素。在这个例子中,主项"政客"指称(designate)政客的类,谓项"说谎者"指称说谎者的类。所有全称肯定命题都可以写成如下形式:

其中字母 $S$ 和 $P$ 分别代表主项和谓项。"全称肯定命题"这一名称是恰当的,因为这个命题肯定了两个类之间的包含于关系,并且是完全或者说全部包含于关系:断言 $S$ 的所有元素同时都是 $P$ 的元素。

第二个例子——没有政客是说谎者——是一个全称否定命题。它是对全部政客而言,否定他们是说谎者。就这样两个类来说,全称否定命题断言第一个类与第二个类是完全排斥的,也就是说第一个类中没有元素是第二个类的元素。所有全称否定命题都可以写成如下形式:

没有 $S$ 是 $P$ 。\\
其中 $S$ 和 $P$ 也分别代表主项和谓项。"全称否定命题"这一名称是恰当的,因为这个命题否定了这两个类之间的包含于关系——并且是全部否定:断言在 $S$ 的所有元素中,没有一个是 $P$ 的元素。 ${ }^{(1)}$

第三个例子—有政客是说谎者——是一个特称肯定命题。显然,这个例子肯定的是政客类中有元素(也)是说谎者类的元素。但并没有对政客类作全部断言:它说的并不是所有政客,而是某个或某些政客是说谎者。此命题既没有肯定也没有否定所有政客是说谎者,对此并没有给出主张。从字面含义看,它并没有断言有政客不是说谎者,尽管在某些语境中它可能暗含这样的意思。这个命题的字面含义或者说最小的(minimal)解释,即政客的类和说谎者的类之间有某个或某些元素是共同的。为确定性起见,我们这里采取最小解释。\\
"有"(some)这个词的含义是不确定的。它指的是"至少有一个"、 "至少有两个",还是"至少有一百个"呢?到底有多少个?尽管与某些场合中的通常用法不太一致,但为了保持确定性,我们一般把"有"看做 "至少有一个"的意思。这样,特称肯定命题可以写成如下形式:

有 $S$ 是 $P$ 。\\
它断言的是,主项 $\boldsymbol{S}$ 指称的类中至少有一个元素是谓项 $\boldsymbol{P}$ 指称的类的元素。"特称肯定命题"这个名称是恰当的,因为这种命题肯定了类之间具有某种包含于关系,但不是全部而只是部分地(partially)肯定第一个类

\footnotetext{(1)我国逻辑教材中一般把全称否定命题的形式写为:"所有 $S$ 不是 $P$",其与"没有 $S$ 是 $P$"同义。但根据英语语法,"All $S$ are not $P$"并不与"No $S$ are $P$"同义,而等义于"Not all $S$ are $P$"。故英文著作一般将"No $S$ is(are)$P$"作为全称否定命题的形式。
}中的某个或某些元素包含于第二个类。\\
第四个例子——有政客不是说谎者———是——个特称否定命题。这个例子,正如上面的例子一样,谈论的并不是全部政客,而只是政客类中某个或某些元素,因而是特称的。不同于第三个例子的是,它并非肯定第一类中的某部分包含于第二个类中,相反,它是否定的。所有特称否定命题可以写成如下形式:

\section*{有 $S$ 不是 $P$ 。}
\section*{它断言的是,主项 $\boldsymbol{S}$ 指称的类中至少有一个元素被谓项 $\boldsymbol{P}$ 指称的类的全体所排斥。}
并非所有标准式直言命题都像以上四个例子那样简单明了。标准式命题的主项、谓项指称的都是类,但这些词项可能是复杂的表达式而非一个单词。举例来说,在命题"所有这个职位的候选人都是诚实而正直的人"中,主项是"这个职位的候选人",谓项是"诚实而正直的人"。

曾经有一种传统观点,认为所有演绎论证都可以用类或范畴以及它们之间的关系加以分析。这样,如上说明的直言命题的四种标准形式,就被认为是所有演绎论证的基石:

\begin{displayquote}
全称肯定命题(称为 A 命题)\\
全称否定命题(称为 E 命题)\\
特称肯定命题(称为 I 命题)\\
特称否定命题(称为()命题)
\end{displayquote}

(尽管这种传统观点是不正确的,但)${ }^{(1)}$ 的确有许多逻辑理论一一正如我们将要看到的——就是以这四种命题为基础建立起来的。
\footnotetext{(1)括㚽内的话为译者所加。
} 