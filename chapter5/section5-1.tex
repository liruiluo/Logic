\section{演绎理论}

\begin{quotation}
\textit{演绎推理是逻辑学的核心内容,它探讨如何从已知前提得出必然的结论。本章将介绍演绎理论的基本概念和方法,帮助读者理解逻辑推理的本质和规则。}
\end{quotation}

\subsection{演绎论证的本质}

前面几章探讨的主要是语言及其对论证的影响,现在我们来讨论论证本身。首先来分析一种特殊的论证——\textbf{演绎}。\textbf{演绎论证}是这样一种论证,其前提被要求为结论的真提供决定性基础。如果前提之真确实能够决定其结论为真,那么,这个论证就是\textbf{有效的}。任何一个演绎论证都或者有效或者无效:如果不可能出现前提真而结论假的情况,那么论证就是有效的,否则就是无效的。

\textbf{演绎理论}旨在阐明有效论证的前提与结论之间的关系,为评估演绎论证提供方法。也就是说,演绎理论要给出区别有效演绎与无效演绎的方法。为此,历史上出现了两种杰出的理论。第一种被称为\textbf{古典逻辑}(classical)或\textbf{亚里士多德型逻辑},开创这种理论的是古希腊大哲学家亚里士多德。另一种称为\textbf{现代逻辑}或\textbf{现代符号逻辑}。本章与接下来的两章(即$5、6、7$三章)主要探讨古典逻辑问题,而$8、9、10$三章主要探讨现代逻辑问题。

\subsection{亚里士多德的逻辑贡献}

亚里士多德是古代伟大智者之一。在柏拉图学园钻研20年之后,他成为亚历山大大帝的家庭教师,后来建立了自己的学园:Lyceum(吕克昂),在那里他做出了许多杰出贡献,几乎涵盖了人类知识的所有领域。亚里士多德去世以后,他关于推理的论述被收集成册,称为《工具论》(Organon)。虽然一直到公元2世纪"逻辑"这个词才获得它的现代含义,但逻辑学的主题早已在《工具论》中确定了。 

\begin{center}
\fbox{\parbox{0.95\textwidth}{
\textbf{本节要点}
\begin{itemize}
\item \textbf{演绎论证}是前提为结论的真提供决定性基础的论证
\item 论证的\textbf{有效性}:不可能前提为真而结论为假
\item \textbf{演绎理论}:阐明有效论证中前提与结论之间的关系,提供评估方法
\item 历史上两种主要的演绎理论:
  \begin{itemize}
  \item \textbf{古典逻辑}(亚里士多德型逻辑)
  \item \textbf{现代逻辑}(现代符号逻辑)
  \end{itemize}
\item 亚里士多德在《工具论》中奠定了逻辑学的基础,涵盖了广泛的知识领域
\end{itemize}
}}
\end{center} 