\section*{舄 5 䓙}
\section*{5.1 演绎理论}
前面几章探讨的主要是语言及其对论证的影响,现在我们来讨论论证本身。首先来分析一种特殊的论证——演绎。演绎论证是这样一种论证,其前提被要求为结论的真提供决定性基础。如果前提之真确实能够决定其结论为真,那么,这个论证就是有效的。任何一个演绎论证都或者有效或者无效:如果不可能出现前提真而结论假的情况,那么论证就是有效的,否则就是无效的。

演绎理论旨在阐明有效论证的前提与结论之间的关系,为评估演绎论证提供方法。也就是说,演绎理论要给出区别有效演绎与无效演绎的方法。为此,历史上出现了两种杰出的理论。第一种被称为"古典的" (classical)或"亚里士多德型的"逻辑,开创这种理论的是古希腊大哲学家亚里士多德。另一种称为"现代"逻辑或"现代符号"逻辑。本章与接下来的两章(即 $5 、 6 、 7$ 三章)主要探讨古典逻辑问题,而 8、9、10 三章主要探讨现代逻辑问题。

亚里士多德是古代伟大智者之一。在柏拉图学园钻研 20 年之后,他成为亚历山大大帝的家庭教师,后来建立了自己的学园:Lyceum(吕克昂),在那里他做出了许多杰出贡献,几乎涵盖了人类知识的所有领域。亚里士多德去世以后,他关于推理的论述被收集成册,称为《工具论》 (Organon)。虽然一直到公元 2 世纪"逻辑"这个词才获得它的现代含义,但逻辑学的主题早已在《工具论》中确定了。 