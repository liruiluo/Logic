\section{演绎理论}

\begin{logicbox}[title=引言]
\textit{演绎推理是逻辑学的核心内容,它探讨如何从已知前提得出必然的结论。本章将介绍演绎理论的基本概念和方法,帮助读者理解逻辑推理的本质和规则。}
\end{logicbox}

\subsection{演绎论证的本质}

\begin{theorembox}[title=演绎论证的基本概念]
\logicemph{研究转向}:前面几章探讨的主要是语言及其对论证的影响,现在我们来讨论论证本身。

\logicemph{演绎论证的定义}:首先来分析一种特殊的论证——\logicterm{演绎}。\logicterm{演绎论证}是这样一种论证,其前提被要求为结论的真提供决定性基础。

\logicemph{有效性标准}:
\begin{itemize}
  \item 如果前提之真确实能够决定其结论为真,那么,这个论证就是\logicterm{有效的}
  \item 任何一个演绎论证都或者有效或者无效
  \item 如果不可能出现前提真而结论假的情况,那么论证就是有效的,否则就是无效的
\end{itemize}
\end{theorembox}

\begin{theorembox}[title=演绎理论的目标与发展]
\logicemph{理论目标}:\logicterm{演绎理论}旨在阐明有效论证的前提与结论之间的关系,为评估演绎论证提供方法。

\logicemph{核心任务}:演绎理论要给出区别有效演绎与无效演绎的方法。

\logicemph{历史发展}:为此,历史上出现了两种杰出的理论:
\begin{itemize}
  \item \logicterm{古典逻辑}(classical)或\logicterm{亚里士多德型逻辑}:开创这种理论的是古希腊大哲学家亚里士多德
  \item \logicterm{现代逻辑}或\logicterm{现代符号逻辑}:采用符号化方法的现代逻辑系统
\end{itemize}

\logicemph{本书安排}:本章与接下来的两章(即5、6、7三章)主要探讨古典逻辑问题,而8、9、10三章主要探讨现代逻辑问题。
\end{theorembox}

\subsection{亚里士多德的逻辑贡献}

\begin{theorembox}[title=亚里士多德的学术生涯]
\logicemph{伟大智者}:亚里士多德是古代伟大智者之一。

\logicemph{学术历程}:
\begin{itemize}
  \item 在柏拉图学园钻研20年,奠定了深厚的哲学基础
  \item 成为亚历山大大帝的家庭教师,影响了历史进程
  \item 建立了自己的学园:Lyceum(吕克昂)
  \item 在那里他做出了许多杰出贡献,几乎涵盖了人类知识的所有领域
\end{itemize}

\logicemph{逻辑学贡献}:
\begin{itemize}
  \item 亚里士多德去世以后,他关于推理的论述被收集成册,称为\logicterm{《工具论》}(Organon)
  \item 虽然一直到公元2世纪"逻辑"这个词才获得它的现代含义
  \item 但逻辑学的主题早已在《工具论》中确定了
\end{itemize}

\logicwarn{历史意义}:亚里士多德奠定了逻辑学的基础,其理论体系影响了两千多年的逻辑思维发展。
\end{theorembox}

\chaptersummary{
\logicterm{演绎论证}是前提为结论的真提供决定性基础的论证,其\logicemph{有效性}要求不可能前提为真而结论为假。演绎理论阐明有效论证中前提与结论之间的关系,历史上形成了\logicterm{古典逻辑}(亚里士多德型)和\logicterm{现代符号逻辑}两大理论体系。
}