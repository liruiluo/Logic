\section{论证的辨识}

\begin{quotation}
\textit{辨识论证中的前提和结论是逻辑分析的关键步骤,通过掌握辨识方法能够准确把握论证的真正含义。}
\end{quotation}

\subsection{结论和前提指示词}

如前所见,出现在论证性话语中的命题的次序不能作为辨识其结论或前提的依据。那么用什么来辨识呢?有一些被叫做\textbf{结论指示词}的词或短语有助于这样的辨识,因为它们典型地适合引导出一个论证的结论。下面所列的就是部分结论指示词${}^{(1)}$:

\begin{center}
\begin{tabular}{|l|l|}
\hline
therefore(所以) & for these reasons(基于这些理由) \\
\hline
hence(因此) & it follows that(可推得) \\
\hline
thus(因而) & we may infer(我们可推出) \\
\hline
so(故而) & I conclude that(我推断) \\
\hline
accordingly(由此可见) & which shows that(这表明) \\
\hline
in consequence(于是) & which means that(这意味着) \\
\hline
consequently(可得) & which entails that(据此可得) \\
\hline
proves that(据此证明) & which implies that(这蕴涵) \\
\hline
as a result(之所以) & which allows us to infer that(据此我们可以推出) \\
\hline
for this reason(为此缘故) & which points to the conclusion that(据此可得结论) \\
\hline
\end{tabular}
\end{center}

另一些词或短语典型地适合作为论证\textbf{前提}的标志,因而被叫做\textbf{前提指示词}。通常,但非总是,跟在任一前提指示词之后的命题就是某个论证的前提。下面所列的是部分前提指示词:

\begin{center}
\begin{tabular}{|l|l|}
\hline
since(因为) & may be inferred from(可从......推出) \\
\hline
because(由于) & may be derived from(可从......引申) \\
\hline
for(因) & may be deduced from(可从......得出) \\
\hline
as(根据) & in view of the fact that(有鉴于) \\
\hline
follows from(从......推出) & the reason is that(理由是) \\
\hline
as shown by(正如......所表明) & for the reason is that(理由在于) \\
\hline
inasmuch as(缘于) & as indicated by(正如......所示) \\
\hline
\end{tabular}
\end{center}

\footnotetext{(1)下列英汉指示词并非一一对应,在自然语言中识别论证,主要应诉诸语境分析。——译者注(以下凡脚注均为译者注,不一一标明)}

\subsection{语境中的论证}

上面所列的词和短语可以帮助我们认识话语中所含的论证,辨识其前提或结论,但它们在实际论证中并不一定出现。论证的出现可以由话语的背景或意义来表明。例如,一个女作家用如下陈述对吸烟提出严厉的批评:

\begin{quotation}
是否吸烟是在拥有关于烟草对健康的致命影响的充足信息的情况下做出的有意识的决定。无疑,那些对此做出不明智选择的人,应为其导致健康恶化的后果负责。${}^{[23]}$
\end{quotation}

这段话中既没有前提指示词,也没有结论指示词,但其中所含的论证是很清楚的。同样,下面一段话中所包含的论证可以从其所含命题本身的意义辨识出来:

\begin{displayquote}
近年来,有关死刑处罚的威慑作用的论证受到人们的反驳。谋杀率最高的二十个州中的十八个州有死刑处罚。谋杀率最高的十七个大城市拥有死刑处罚的司法权。过去十年中,得克萨斯州处死的罪犯比其他任何一个州都多,但得州仍有三个城市的谋杀率列于谋杀率最高的二十五个城市之中。近二十年来,有两个接壤的州的谋杀率基本相当,一个是没有死刑处罚的密歇根州,另一个是有死刑处罚的印第安纳州。${}^{[24]}$
\end{displayquote}

这些语段的论证性功能由它们的\textbf{语境}和它们的\textbf{意义}展现出来。这就好比当我说晚饭时带只龙虾回家,你不会怀疑我是打算吃掉而不是饲养它。

另一个没有结论或前提指示词的论证出现在最近一篇为比例代表制进行辩护的文章中:

\begin{quotation}
单一成员选区(the single-member-district)的选举制度看来有许多严重的弊端。这种制度通常不能代表为数众多的选民的意志,它产生的立法机关不能准确反映公众的看法,它歧视第三党,挫伤选民投票的积极性。${}^{[25]}$
\end{quotation}

虽然可将这段话看做是首先陈述一个广泛的实际情况,然后用单一成员选区的选举制度的各种后果去阐明它,但也可以将这段话同样很好地理解为一个首先陈述其结论,然后从支持这个结论的前提推出这个结论的论证。

下面一段最高法院关于公立学校反种族隔离问题的评判中,有一个既无结论指示词又无前提指示词的更复杂一些的论证:

\begin{displayquote}
在学生人数上存在种族不平衡的现象,这不等于表明乡村学校不履行其法律义务。种族平衡本身不是目的。若种族不平衡是由于违背宪法使然,必须予以追究。只要杜绝违法的种族不平衡,乡村学校并没有义务去纠正因人口因素而造成的种族不平衡。${}^{[26]}$
\end{displayquote}

这段话的第一个句子是其所含论证的结论,这个结论可以被解释为"种族不平衡的存在并不表明乡村学校违背了法律"。我们怎么知道它是结论呢?这里\textbf{语境}是决定性的:接在第一个句子后面的几个句子提供了之所以如此的理由。我们看到,在第一个句子中所指谓的"乡村学校"的行为处于争论之中;后面的几个句子表达了几个与乡村学校的行为有关的更一般的命题。话语中所选用的语词也给出了线索,虽然短语"不等于表明"不是结论指示词,但它传达了这样一个暗示,即第一个句子是这段话的逻辑终点。

包含论证的语段经常含有一些既不能作为前提也不能作为结论的附加材料。有时那些提供\textbf{背景信息}的材料能使读者(或听者)理解论证是关于什么内容的。在下面这段话中,论证出现在最后一句中,但如果不抓住它前面句子的内容,这个论证就不可理解:

\begin{quotation}
由于政府削减了对学生的财政援助,许多一流学院和大学都将较大比例的学费收入用做贫困学生的奖学金。正如慈善捐款可免征所得税一样,这部分学费也应享受税务免征。${}^{[27]}$
\end{quotation}

严格地讲,这段话中的第一个句子不是论证的一部分,但没有这句话我们就不能理解"这部分"学费是指用做奖学金的那部分学费。据此理解,我们就可以对该论证做如下解释:

\begin{enumerate}
  \item 对贫困者的慈善捐款是免征所得税的。
  \item 很大一部分学费收入被学校用来作为给贫困学生提供奖学金的慈善捐款。\\
  所以,作为给贫困学生提供奖学金的那部分学费应免征所得税。
\end{enumerate}

可见,上下文中命题之间相互参照,对于理解论证本身是必不可少的。哲学家阿瑟·叔本华在为自杀行为进行(无罪)辩护时所做的一个论证就例示了这种对相互参照的依赖:

\begin{displayquote}
如果刑法禁止自杀,那么在基督教中这并不是一个有根据的论证;而且这个禁令是荒唐的,因为有什么惩罚能让一个连死都不怕的人害怕呢?${}^{[28]}$
\end{displayquote}

这段话分号前面的句子既非前提也非结论,但是若没有它,我们就不知道在随后出现的论证结论("这个禁令是荒唐的")中的"禁令"乃指谓刑法的自杀禁令。

\subsection{非陈述形式的前提}

在上一例子中,论证前提以疑问句的形式出现:"有什么惩罚能让一个连死都不怕的人害怕呢?"而正如1.2节所述,问题无所断定,不表达命题,那么一个疑问句何以能起到前提的作用呢?这取决于该问句是\textbf{反诘问句}。就是说,当提问者相信问题的答案显然或确定无疑时,可用问句暗示或设定一个前提。在上例中,叔本华认为他的问题的答案明显是"没有",因而,尽管以问句的形式出现,其论证的前提乃是这样一个不言而喻的命题:"没有任何惩罚能够让一个连死都不怕的人害怕。"

前提之一是反诘问句,而问题的答案被设定为明显的,这样的论证是非常普遍的,它们也很有修辞效果。可是这样使用问句是有风险的。例如苏格拉底的如下论证:

\begin{quotation}
美诺啊,如果没有人欲望痛苦,那么就没有人欲望罪恶;因为除了欲望和拥有灾难,还有什么是痛苦的呢?${}^{[29]}$
\end{quotation}

严格地讲,其中的问句既不真也不假。如果设定为显然或确定无疑的答案事实上并非如此,那么这个论证就是有缺陷的,而其缺陷正可能被问句掩盖。苏格拉底所假定的痛苦就是欲望和拥有罪恶这个答案是正确的吗?回答并不是显而易见的。

依赖反诘问句的论证,其结论有时是可疑的。人们使用设定有明显答案的问句来做论证的前提,有时就是为了回避直截了当地肯定其前提的责任,而实际上其设定的答案是含糊的甚或是假的。

不过,把真正的反诘问句作为前提使用确是一种很机敏的方法。通过暗示被期望的答案并且引导读者自己引出那个答案,可以增强论证的说服力。考虑下面两个使用反诘问句的例子。《新约全书》中有如下一段话:

\begin{displayquote}
人若说:"我爱神",却恨他的兄弟,就是说谎者:因为不爱他所看见的兄弟,如何能爱他看不见的神呢?${}^{[30]}$
\end{displayquote}

在最近的一篇对安乐死主张进行评论的文章中,有下面一段论证:

\begin{displayquote}
如果安乐死的权利基于自己的决定,那么将其限制到垂死病人就是不合情理的。如果人们有死亡权,那么为什么必须要等到已濒临死亡的时候才能行使这个权利呢?${}^{[31]}$
\end{displayquote}

在上面的两例论证中,两个设问的答案(一个是"不能爱他的兄弟的人也不能爱神",另一个是"人们不必要等到已濒临死亡的时候才能行使死亡权")都被假定是非常明显的。这些答案就是支持预期结论的前提。两个预期的结论分别是:"爱神的人不能恨他的兄弟"和"如果人们有基于自己决定的安乐死的权利,那么就不能将死亡权限制到垂死病人"。

\subsection{命令式的结论}

有时论证的结论可以采用祈使句或命令句的形式。在给出劝说我们去采取一个特定的行动的理由后,我们被指导要如此这般地去行动。例如《箴言》\footnote{Proverbs,见《旧约全书》。}中有这样一句话:

\begin{displayquote}
智慧为首,所以要求得智慧。
\end{displayquote}

在《哈姆雷特》中,波洛涅斯对他的儿子雷欧提斯提出如下忠告:

\begin{displayquote}
不要向人告贷,也不要借钱给人;\\
因为债款放了出去,往往不但丢了本钱,而且还失去了朋友;\\
向人告贷的结果,容易养成因循懒惰的习惯。${}^{[32]}$
\end{displayquote}

因为命令句像一般疑问句一样不能表达命题,所以(严格来讲)命令句不能作为论证的结论。但是为简明计,我们可以把在这些语境中的命令句与命题同样对待,这是很有益处的。在这些语境中听者(或读者)被告知他们\textbf{应当}(should)或\textbf{应该}(ought to)以在命令句中已经说明的方式去行事。那么上面两个论证中的结论可以被解释为:"求得智慧是你应当去做的事情"和"你应该既不借钱给别人也不向别人告贷"。

几乎所有人都会同意,这种断言可以是或真或假的。如果在一个应当去做某事的命令和一个应当做某事的陈述之间有什么区别的话,这种区别恰恰是一个困难的问题,这个问题在这里不必探究。通过忽略这种区别(如果确有区别的话),我们可以对用命令形式和用陈述形式表达结论的论证做相同的处理。

我们的目的是要更彻底地理解论证。这需要借助于澄清论证的构成命题的作用,尽可能减少依赖语境的因素,从而使论证得以完整地重塑。我们要聚焦于命题本身,探索它们是真的还是假的,它们蕴涵着什么,它们是否被别的命题所蕴涵,在某个论证中它们是否被作为前提或结论。我们要抓住命题的实质,而无论它们的语法形式是什么。

\subsection{短语形式的论证}

有些论证的完整重塑仅限于语法方面。论证由命题组成,但是表示命题(因此也能表示前提)的话语有时可能采用短语的形式,而不是陈述句形式。下面讨论地外生命可能性的一段话,可以很好地例示这一点:

\begin{displayquote}
地外有生命吗?至今仍无定论。但是,有大量的行星;有能够无须近地恒星的能量而生存的生物;有丰富的能产生水的广袤无垠的氢和氧的宇宙资源;有行星产生内部热量的各种自然方式;有生命能在海底火山产生并且能足够耐寒地繁殖变体,从而把它们的后代传播到别的世界的可能性;有能够作为星际交流运载工具的坚固的陨石,凡此种种,生命在宇宙的其他地方演化的思想似乎不再像几年前那样让人感到异想天开。${}^{[33]}$
\end{displayquote}

这段话的结论——地球以外有生命的观念至少比以前更能让人接受——得到六个独立前提的支持,每个前提都让人注意到近来发现的事实或可能性,每个前提都表达支持存在地外生命的理由。当这些前提被重新解释为陈述句时,如:(1)宇宙间有大量的其他行星存在;(2)有许多生命能够不依靠近地恒星的能量而生存;等等,这段话中所表达的论证也就变得明显起来。

\subsection{未明确陈述的命题}

当论证中有一个或更多构成命题未明确陈述出来但又假设能为人理解时,论证的分析可能变得更复杂。在2000年4月美国最高法院对著名的米兰达规则进行辩论的会议上,就有这样的例子。(米兰达规则规定:除非被监禁的嫌疑人在审问开始前被告知有权保持沉默并有权请律师,否则法庭不得采信嫌疑人在接受警察审问时所做的认罪供述。)米兰达规则的辩护人论证如下:

\begin{quotation}
如果米兰达规则被推翻,将不再强制性地要求警察预先给予(有权保持沉默等的)告知;如果不强制性地要求警察预先给予告知,他们将不会预先告知。但是因为警察的审问是在公共视域以外进行,仅当总是给予米兰达告知,这些审问的完善性才能得到维护。${}^{[34]}$
\end{quotation}

此处辩护人论证的结论——必须始终给予预先告知,最高法院不应当推翻米兰达规则——不必陈述出来。

在另一个完全不同的语境中,著名小说家安奈斯·林这样描写她的一个小说人物:"梦想家拒绝平凡,杰伊向往平凡。"${}^{[35]}$我们可以推断出作者试图传达的内容——"杰伊不是梦想家"——即使没有陈述出来。

由于论证者假设的论证前提之一是人所共知的或他认为很容易就被人承认的,就可能不陈述出来。在莎士比亚的《裘利斯·恺撒》中,当马克·安东尼正在做关于恺撒的野心的著名演说时,一个市民听众评论恺撒说:

\begin{quotation}
如果他真的有野心,那就证明他的确够不上称王的位置。${}^{[36]}$
\end{quotation}

这是一个论证,但省略了一部分前提,它明显依赖一个合理的但未陈述出来的前提:"不愿接受王冠的人一定没有野心"。日常语言中的三段论论证经常依赖某个未陈述出来的命题。这样的论证叫做\textbf{省略三段论}。${}^{[37]}$

究竟如何揭示说话人所依赖的(省略)命题,有时并不是很明显的,尽管一旦将其表示出来就很容易被接受。在最近的一部有关美国奴隶制的历史争论以及在那个争论中道德论证所起的作用的著作中,作者写道:

\begin{displayquote}
如果不相信道德论证能产生任何影响,那就是不相信共和政体的政府。${}^{[38]}$
\end{displayquote}

在这个省略式中,未陈述出来的前提是这样一个断定:"相信共和政体的政府要求人们相信道德论证能产生一定的影响"——一个我们多数人都会认可的断言。

此外,省略三段论所依赖的未陈述出来的命题有可能并不显然,而是可质疑的;不把它明确陈述出来,可能正是为了使之避受责难。例如,使用胚胎干细胞进行医学研究广受质疑,一位美国参议员用下面的省略三段论抨击允许政府筹措资金进行这项研究的法案:

\begin{displayquote}
这项研究(包含对胚胎干细胞的使用)是非法的,这是因为:故意杀死人类胚胎是这项研究的基本组成部分。${}^{[39]}$
\end{displayquote}

该论证陈述出来的前提是真的:如果胚胎不被杀死,这项研究是不可能进行的。但是,这项研究是非法的这个结论却依赖其未表述出来的命题:杀死人类胚胎是非法的,而这正是一个处于激烈争论中的论断。

省略三段论极其依赖语境,也经常依赖于听话者关于某个表述出来的命题为假的知识。当论证的目的是强调某个命题的虚假性时,说话人常常构造这样一个假言命题:以该命题作前件("如果"部分),以一个普遍认为为假的命题作后件("那么"部分)。例如,18世纪著名的巴伐利亚风琴制造商之一约瑟夫·瑞普就他的管风琴说过一句广为人知的豪言:"如果在欧洲能发现更好的管风琴,那么我的名字就叫杰克。"因为所有人都会明白,在一个真的假言陈述中,如果后件是假的,前件就不能是真的。对这个假言命题的肯定,实际上就是一个省略式论证,即旨在嘲讽其前件命题:在欧洲能发现更好的管风琴。论证的结论(在欧洲不能发现更好的管风琴)和另一前提(我的名字不叫杰克)都没有表述出来。${}^{[40]}$

\begin{center}
\fbox{\parbox{0.9\textwidth}{
  \centering
  \textbf{论证辨识的关键要点}\\
  辨识论证时需要注意结论和前提指示词,分析语境意义,\\
  理解非陈述形式和未明确陈述的命题,从而准确把握论证的真正结构。
}}
\end{center} 