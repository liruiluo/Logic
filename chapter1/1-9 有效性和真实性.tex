\section{有效性和真实性}

\begin{quotation}
\textit{理解有效性与真实性的区别是逻辑分析的基础,这两个概念分别适用于论证和命题,正确把握它们的关系对评估推理至关重要。}
\end{quotation}

如前表明,一个成功的演绎论证是有效的。\textbf{有效性}指谓命题之间的一种关联——作为演绎论证前提的命题集和作为该论证的结论的一个命题之间的关联。如果后者是逻辑必然地从前者推出的,我们就说该论证是有效的。因为归纳论证永远达不到逻辑必然,有效性永远不适用它们。有效性也永远不能适用于任何独立的单一命题本身,因为在任何一个命题内部都不可能找到这种必需的关联。

另一方面,\textbf{真和假}都是单个命题的特征。在论证中作为前提的单个陈述可能是真的或假的,作为其结论的陈述可能是真的或假的。结论可以被有效地推论出来,但是说任何结论或任何单一的前提本身是有效的或无效的,都是无意义的。

\subsection{真实性与命题}

真,是其断言与实际情形相一致的命题的属性。当我断定苏必利尔湖是北美洲五大湖中最大的湖时,我的断言确与实际情形相一致,从而就是真的。如果我说北美洲五大湖中最大的是密歇根湖,我的断定就与实在世界不一致,因此就是假的。这个对比是重要的:\textbf{真和假是单一的命题或陈述的属性;有效性和无效性是论证的属性}。

正如有效性这个概念不适用于单一的命题,真这个概念也不能应用于论证。一个论证中的几个命题,其中的一些(或全部)可以是真的,并且其中的一些(或全部)可以是假的。但是论证作为一个整体,既不"真"也不"假"。关于世界的陈述的命题可以是真的或假的,由从一个命题集到其他命题的推论构成的演绎论证可以是有效的或无效的。

真(或假)命题与有效(或无效)论证之间的关系处于演绎逻辑的中心地位。本书第二部分主要致力于分析它们之间的这些复杂关系。不过,在这里对有效性和真实性之间的关系作一初步的讨论也是适宜的。

\subsection{有效性与真假前提}

即使一个论证的一个或几个前提不是真的,这个论证也可能是有效的,我们的讨论就从强调这一点入手。每个论证都对其前提和从这些前提推导出的结论之间的关联做出断言;即使这些前提被证明为假或其真实性受到质疑,这种关联也是成立的。1858年亚伯拉罕•林肯在与斯蒂芬•道格拉斯的一场争论中有力地运用了这一点。林肯对强迫逃跑到北方各州的奴隶返回他们在南方的主人家里的德雷德-司各特决议进行抨击说:

\begin{quotation}
考虑到人们的论辩能力,我把德雷德•司各特决议用三段论形式表述如下,可以看出这个论证中是否有错误:

任何一个州的任何法规和法律都不能破坏美国宪法中所清楚、明确地规定的权利。

美国宪法中清楚、明确地规定了对奴隶的财产权。

所以任何一个州的任何法规和法律都不能破坏对奴隶的财产权。
\end{quotation}

我相信这个论证挑不出什么毛病。假设其前提都是真的,从这些前提必然会推出上述结论,这个结论我完全有能力理解。但我认为其中确有一个毛病,但这毛病不在于推理,事实上这个毛病是有一个前提是错误的。我相信对奴隶的财产权并不是宪法中清楚明确地规定的,而道格拉斯法官认为是的。我相信最高法院和那个决议(德雷德•司各特决议)的拥护者要想在宪法中查找到对奴隶的财产权的清楚明确的规定将会是徒劳的。所以我说,我认为事实上上述推理前提之一不是真的。$^{[48]}$

在他加以概括并予以抨击的那个论证中,林肯发现第二个前提——美国宪法中规定了对奴隶的财产权——明显是假的。他指出,那个论证中的推理不是错误的,然而它的结论却没有得到证明。林肯的逻辑观点是正确的:甚至在一个论证的结论和其一个或几个前提都为假时,这个论证也可以是有效的。我们再一次强调,一个论证的有效性仅仅依赖于其前提与结论之间的关联。

\subsection{真实性和有效性的组合关系}

在有效的和无效的论证中,真的和假的前提与结论之间有许多可能的组合。考虑下列作为例示的论证,每一个论证之前都有一个对其前提与结论的真假组合的简短陈述。通过这些例证,我们可以完整地列出有关真实性和有效性之间的关系的重要原则。

I.有些有效的论证只包含真命题——真前提和真结论。

\begin{quotation}
所有哺乳动物都有肺,所有鲸鱼都是哺乳动物,所以所有鲸鱼都有肺。
\end{quotation}

II.有些有效的论证只包含假命题:

\begin{quotation}
所有四条腿的生物都有翅膀,所有蜘蛛都是四条腿的,所以所有蜘蛛都有翅膀。
\end{quotation}

这个论证是有效的,因为如果其前提是真的,其结论也一定是真的——即使我们知道这个论证的前提和结论实际上都是假的。

III.有些无效的论证只包含真命题——它们的所有前提都是真的,它们的结论也是真的。

\begin{quotation}
如果我拥有福特•诺克斯的所有财富,那么我将是富有的,我不拥有福特•诺克斯的所有财富,所以我不是富有的。
\end{quotation}

IV.一些无效的论证只包含真前提,但有一个假结论。这可以用一个与前面的论证(III)在形式上完全相同,仅仅换一个假结论的论证来说明:

\begin{quotation}
如果比尔•盖茨拥有福特•诺克斯的所有财富,那么比尔•盖茨将是富有的,

比尔•盖茨不拥有福特•诺克斯的所有财富,

所以比尔•盖茨不是富有的。
\end{quotation}

这个论证的前提是真的,但结论是假的。这样的论证不会是有效的,因为有效的论证不可能前提真而结论假。

V.有些有效的论证有假前提和真结论:

\begin{quotation}
所有鱼是哺乳动物,

所有鲸是鱼,

所以所有鲸是哺乳动物。
\end{quotation}

如我们所知,这个论证的结论是真的;而这个结论可以从两个都不符合实际的假前提有效地推论出来。

VI.一些无效的论证也有假前提和真结论:

\begin{quotation}
所有哺乳动物都有翅膀,

所有鲸都有翅膀,

所以所有鲸都是哺乳动物。
\end{quotation}

从例V和例VI可以看出,很显然我们不能从一个论证有假前提和真结论推断该论证是有效的还是无效的。

VII.当然,有些无效的论证包含的都是假命题——假前提和假结论:

\begin{quotation}
所有哺乳动物都有翅膀,

所有鲸都有翅膀,

所以所有哺乳动物都是鲸。
\end{quotation}

这七个例子清楚地表明,有结论为假的有效论证(例II),也有结论为真的无效论证(例III和例VI)。因此很显然,一个论证的结论的真或假自身并不决定那个论证的有效性或无效性。此外,一个论证有效不能保证其结论的真实性(例II)。

下面两个表格(涉及前面的七个例子)清楚地表明了论证前提与结论可能组合的种类。第一个表格表明,无效论证可以具有真的和假的前提与结论的每一种可能组合:

\begin{center}
\begin{tabular}{|c|c|c|}
\hline
\multicolumn{3}{|c|}{无效的论证} \\
\hline
 & 真结论 & 假结论 \\
\hline
真前提 & 例III & 例IV \\
\hline
假前提 & 例VI & 例VII \\
\hline
\end{tabular}
\end{center}

第二个表格表明,有效论证只能具有真的和假的前提与结论的可能组合的三种情况:

\begin{center}
\begin{tabular}{|c|c|c|}
\hline
\multicolumn{3}{|c|}{有效的论证} \\
\hline
 & 真结论 & 假结论 \\
\hline
真前提 & 例I & \multicolumn{1}{|c|}{} \\
\hline
假前提 & 例V & 例II \\
\hline
\end{tabular}
\end{center}

\subsection{有效论证与可靠性}

第二个表格的那个空格位置显示了非常重要的一点:如果一个论证是有效的并且其前提都是真的,我们就可以断定其结论也是真的。换句话说:\textbf{如果一个论证是有效的并且其结论是假的,那么其前提不会都是真的}——肯定有至少一个假前提。

有效加上真前提等于\textbf{可靠}。一个可靠的论证必定有真结论。只有可靠的论证才能确立其结论的真实性。如果一个演绎论证不是可靠的——也就是说,如果这个论证不是有效的,或者如果其前提并非都是真的——即使其结论事实上是真的,其结论的真实性在论证中也得不到确立。

检验前提的真实性或虚假性是一般科学的任务,因为前提完全可以涉及任意的题材。逻辑学家的主要兴趣不在于命题的真实性或虚假性,而在于命题之间的\textbf{逻辑关系}。所谓命题之间的"逻辑"关系,指的是决定其所出现于其中的论证的(形式)正确性或不正确性的命题之间的那些关系。决定论证的(形式)正确性或不正确性的任务正好落在逻辑学的领域内。逻辑学家甚至对前提可能为假的论证的(形式)正确性感兴趣。

为什么不把我们的研究限制在真前提论证的范围内,而忽略所有其他论证呢?这是因为,那些前提的真实性不为人们所知的论证的(形式)正确性,有可能是非常重要的。例如,在科学研究中,我们通过推断出可检验的结果来检验理论——但是我们不能预先知道哪个理论是真的。又如,在日常生活中,我们经常必须在可供选择的两个行动方向之间做出选择,推断每个行动方向的后果。为了避免选错,我们必须对可供选择的两个选项做出正确的推理,将每个选择作为一个前提。如果我们仅仅对前提为真的论证感兴趣,那么只有当我们知道了可供选择的两个前提哪个是真的,我们才能知道去找出它的结论集。但是如果我们知道可供选择的两个前提哪个是真的,我们就完全不必对它做出推理,因为我们推理的目的就是帮助我们对应该断定哪一个可供选择的前提为真做出决定。所以,如果把我们的注意力限制在前提已知为真的论证上,将使我们的目标不能实现。

确定演绎论证的有效性或无效性的一些富有成效的方法将在本书第二部分介绍并阐释。

\begin{center}
\fbox{\parbox{0.9\textwidth}{
  \centering
  \textbf{有效性与真实性的关系}\\
  有效性适用于论证,描述前提与结论之间的逻辑关系;\\
  真实性适用于命题,描述命题与世界事实的一致性;\\
  只有既有效又具有真前提的论证(可靠论证)才能确立结论的真实性。
}}
\end{center} 