\section{有效性和真实性}

\begin{logicbox}[title=引言]
\textit{理解有效性与真实性的区别是逻辑分析的基础,这两个概念分别适用于论证和命题,正确把握它们的关系对评估推理至关重要。}
\end{logicbox}

如前所述,成功的演绎论证具有\logicterm{有效性}。\logicemph{有效性}是一种特殊的逻辑关系——它描述的是演绎论证中前提集与结论之间的必然联系。当结论从前提中\logicwarn{逻辑必然地}推出时,我们说该论证是有效的。

\begin{theorembox}[title=有效性的适用范围]
\logicterm{有效性概念的限制}:
\begin{itemize}
  \item \logicwarn{不适用于归纳论证}:因为归纳论证永远达不到逻辑必然性
  \item \logicwarn{不适用于单个命题}:任何独立的命题内部都不存在前提与结论的关联
  \item \logicemph{仅适用于演绎论证}:描述前提与结论之间的逻辑关系
\end{itemize}
\end{theorembox}

与有效性形成对比,\logicterm{真和假}是单个命题的属性:
\begin{itemize}
  \item 论证中的每个前提都可能是真的或假的
  \item 论证的结论可能是真的或假的
  \item 结论可以被\logicemph{有效地推论出来}
  \item 但说任何单一命题"有效"或"无效"都是\logicwarn{无意义的}
\end{itemize}

\subsection{真实性与命题}

\logicterm{真实性}是命题与现实世界相符合的属性。当命题的断言与实际情况一致时,该命题就是\logicemph{真的};当命题的断言与实际情况不符时,该命题就是\logicwarn{假的}。

\begin{examplebox}[title=真实性的例子]
\begin{itemize}
  \item \logicemph{真命题}:"苏必利尔湖是北美洲五大湖中最大的湖"——与实际情况一致
  \item \logicwarn{假命题}:"密歇根湖是北美洲五大湖中最大的湖"——与实际情况不符
\end{itemize}
\end{examplebox}

\begin{theorembox}[title=真实性与有效性的根本区别]
\logicemph{适用对象不同}:
\begin{itemize}
  \item \logicterm{真实性和虚假性}:单一命题或陈述的属性
  \item \logicterm{有效性和无效性}:论证的属性
\end{itemize}

\logicemph{概念的不可互换性}:
\begin{itemize}
  \item 有效性概念\logicwarn{不适用于}单一命题
  \item 真实性概念\logicwarn{不适用于}论证整体
\end{itemize}
\end{theorembox}

论证中的各个命题可以分别是真的或假的,但\logicwarn{论证作为整体既不"真"也不"假"}。论证只能是有效的或无效的。

\logicemph{核心关系}:真(假)命题与有效(无效)论证之间的关系构成了演绎逻辑的核心内容。本书第二部分将详细分析这些复杂关系,这里我们先进行初步讨论。

\subsection{有效性与真假前提}

\logicwarn{重要原理}:即使论证的一个或多个前提为假,该论证仍可能是有效的。这是理解有效性概念的关键点。

\logicemph{有效性的独立性}:每个论证都声称其前提与结论之间存在特定的逻辑关联。这种逻辑关联的成立与否\logicwarn{独立于前提的实际真假值}。即使前提被证明为假或其真实性受到质疑,逻辑关联本身仍然可以是有效的。

历史上有一个著名的例子完美地说明了这一点。1858年,亚伯拉罕·林肯在与斯蒂芬·道格拉斯的辩论中,巧妙地运用了这一逻辑原理来批评德雷德-司各特决议:

\begin{quotation}
考虑到人们的论辩能力,我把德雷德•司各特决议用三段论形式表述如下,可以看出这个论证中是否有错误:

任何一个州的任何法规和法律都不能破坏美国宪法中所清楚、明确地规定的权利。

美国宪法中清楚、明确地规定了对奴隶的财产权。

所以任何一个州的任何法规和法律都不能破坏对奴隶的财产权。
\end{quotation}

我相信这个论证挑不出什么毛病。假设其前提都是真的,从这些前提必然会推出上述结论,这个结论我完全有能力理解。但我认为其中确有一个毛病,但这毛病不在于推理,事实上这个毛病是有一个前提是错误的。我相信对奴隶的财产权并不是宪法中清楚明确地规定的,而道格拉斯法官认为是的。我相信最高法院和那个决议(德雷德•司各特决议)的拥护者要想在宪法中查找到对奴隶的财产权的清楚明确的规定将会是徒劳的。所以我说,我认为事实上上述推理前提之一不是真的。$^{[48]}$

\begin{examplebox}[title=林肯的逻辑分析]
林肯对这个论证的分析展现了深刻的逻辑洞察:

\logicemph{林肯的发现}:
\begin{itemize}
  \item 第二个前提("美国宪法中清楚、明确地规定了对奴隶的财产权")是\logicwarn{假的}
  \item 论证的\logicterm{推理形式}本身是\logicemph{正确的}
  \item 但由于前提为假,结论\logicwarn{没有得到证明}
\end{itemize}

\logicemph{逻辑原理}:林肯正确地认识到,即使论证的前提和结论都可能为假,论证本身仍可能是有效的。
\end{examplebox}

这个历史例子完美地说明了一个\logicemph{核心逻辑原理}:\logicwarn{论证的有效性仅仅依赖于前提与结论之间的逻辑关联,而不依赖于前提或结论的实际真假值}。

\subsection{真实性和有效性的组合关系}

为了全面理解真实性与有效性的关系,我们需要系统地考察它们的各种可能组合。通过以下七个典型例子,我们可以完整地展现这两个概念之间的复杂关系。

\begin{theorembox}[title=真实性与有效性的组合类型]
在演绎论证中,前提和结论的真假值与论证的有效性可以形成多种组合。以下例子将系统地展示所有可能的情况。
\end{theorembox}

\begin{examplebox}[title=类型I:有效论证 + 真前提 + 真结论]
\begin{quotation}
所有哺乳动物都有肺,所有鲸鱼都是哺乳动物,所以所有鲸鱼都有肺。
\end{quotation}

\logicemph{分析}:这是理想的论证形式——有效的逻辑结构配合真实的前提,必然得出真实的结论。
\end{examplebox}

\begin{examplebox}[title=类型II:有效论证 + 假前提 + 假结论]
\begin{quotation}
所有四条腿的生物都有翅膀,所有蜘蛛都是四条腿的,所以所有蜘蛛都有翅膀。
\end{quotation}

\logicemph{分析}:这个论证是\logicterm{有效的},因为\logicwarn{如果}前提为真,结论也必然为真——尽管我们知道前提和结论实际上都是假的。这说明有效性与命题的实际真假值无关。
\end{examplebox}

\begin{examplebox}[title=类型III:无效论证 + 真前提 + 真结论]
\begin{quotation}
如果我拥有福特•诺克斯的所有财富,那么我将是富有的,我不拥有福特•诺克斯的所有财富,所以我不是富有的。
\end{quotation}

\logicemph{分析}:这个论证是\logicwarn{无效的},尽管前提和结论都是真的。这说明结论的真实性不能保证论证的有效性。
\end{examplebox}

\begin{examplebox}[title=类型IV:无效论证 + 真前提 + 假结论]
\begin{quotation}
如果比尔•盖茨拥有福特•诺克斯的所有财富,那么比尔•盖茨将是富有的,

比尔•盖茨不拥有福特•诺克斯的所有财富,

所以比尔•盖茨不是富有的。
\end{quotation}

\logicemph{分析}:这个论证的前提是真的,但结论是假的。\logicwarn{这样的论证必然是无效的},因为有效论证不可能出现前提真而结论假的情况。
\end{examplebox}

\begin{examplebox}[title=类型V:有效论证 + 假前提 + 真结论]
\begin{quotation}
所有鱼是哺乳动物,

所有鲸是鱼,

所以所有鲸是哺乳动物。
\end{quotation}

\logicemph{分析}:这个论证的结论是真的,而且可以从两个假前提中\logicterm{有效地}推出。这说明有效论证可以从假前提得出真结论。
\end{examplebox}

\begin{examplebox}[title=类型VI:无效论证 + 假前提 + 真结论]
\begin{quotation}
所有哺乳动物都有翅膀,

所有鲸都有翅膀,

所以所有鲸都是哺乳动物。
\end{quotation}

\logicemph{分析}:这个论证也有假前提和真结论,但它是\logicwarn{无效的}。
\end{examplebox}

\logicwarn{重要观察}:从类型V和类型VI的对比可以看出,我们\logicemph{不能}仅从论证有假前提和真结论就推断该论证是有效还是无效的。

\begin{examplebox}[title=类型VII:无效论证 + 假前提 + 假结论]
\begin{quotation}
所有哺乳动物都有翅膀,

所有鲸都有翅膀,

所以所有哺乳动物都是鲸。
\end{quotation}

\logicemph{分析}:这个论证包含的都是假命题,且论证形式无效。
\end{examplebox}

\begin{theorembox}[title=七个例子的重要启示]
通过以上七个例子,我们可以得出几个\logicemph{关键结论}:

\begin{itemize}
  \item 存在结论为假的\logicterm{有效论证}(类型II)
  \item 存在结论为真的\logicwarn{无效论证}(类型III和VI)
  \item \logicwarn{结论的真假不能决定论证的有效性}
  \item \logicwarn{论证有效不能保证结论的真实性}
\end{itemize}
\end{theorembox}

为了更清楚地展示这些关系,我们用两个表格来总结所有可能的组合:

\begin{center}
\textbf{无效论证的所有可能组合}
\begin{tabular}{|c|c|c|}
\hline
\multicolumn{3}{|c|}{\logicwarn{无效的论证}} \\
\hline
 & \logicemph{真结论} & \logicwarn{假结论} \\
\hline
\logicemph{真前提} & 类型III & 类型IV \\
\hline
\logicwarn{假前提} & 类型VI & 类型VII \\
\hline
\end{tabular}
\end{center}

\logicemph{观察}:无效论证可以具有前提与结论的\logicwarn{任何真假组合}。

\begin{center}
\textbf{有效论证的可能组合}
\begin{tabular}{|c|c|c|}
\hline
\multicolumn{3}{|c|}{\logicterm{有效的论证}} \\
\hline
 & \logicemph{真结论} & \logicwarn{假结论} \\
\hline
\logicemph{真前提} & 类型I & \logicwarn{不可能} \\
\hline
\logicwarn{假前提} & 类型V & 类型II \\
\hline
\end{tabular}
\end{center}

\logicemph{关键观察}:有效论证只能有\logicterm{三种}真假组合,\logicwarn{不可能}出现真前提配假结论的情况。

\subsection{有效论证与可靠性}

表格中的空白位置揭示了一个\logicemph{极其重要的逻辑原理}:

\begin{theorembox}[title=有效性的根本保证]
\logicemph{正向原理}:如果论证是有效的且前提都为真,则结论必然为真。

\logicemph{逆向原理}:如果论证是有效的且结论为假,则至少有一个前提必然为假。

这两个原理是\logicterm{逻辑等价的},共同构成了有效性概念的核心。
\end{theorembox}

\subsection{可靠性概念}

基于以上分析,我们可以引入一个重要概念:

\begin{theorembox}[title=可靠性的定义]
\logicterm{可靠论证} = \logicemph{有效性} + \logicemph{真前提}

\logicemph{可靠论证的特征}:
\begin{itemize}
  \item 必定有\logicterm{真结论}
  \item 是\logicwarn{唯一能够确立结论真实性}的论证类型
  \item 提供了从前提到结论的\logicterm{可靠推理路径}
\end{itemize}
\end{theorembox}

\logicwarn{重要区别}:如果演绎论证不是可靠的(即不是有效的,或前提不全为真),那么即使其结论事实上为真,该论证也\logicemph{无法确立}结论的真实性。

\subsection{逻辑学的研究范围}

\begin{theorembox}[title=逻辑学与其他学科的分工]
\logicemph{科学的任务}:检验前提的真实性或虚假性(涉及具体的经验内容)

\logicemph{逻辑学的任务}:分析命题之间的\logicterm{逻辑关系}(涉及推理的形式结构)

\logicterm{逻辑关系}:决定论证形式正确性或不正确性的命题间关系,独立于命题的具体内容。
\end{theorembox}

\subsection{为什么研究假前提论证?}

一个重要问题是:为什么不把研究限制在真前提论证的范围内?答案在于\logicemph{前提真实性未知的论证}具有重要价值:

\begin{examplebox}[title=研究假前提论证的重要性]
\logicemph{科学研究中}:
\begin{itemize}
  \item 我们通过推断可检验结果来检验理论
  \item 但我们\logicwarn{无法预先知道}哪个理论是真的
  \item 必须先分析推理的有效性,再检验前提的真实性
\end{itemize}

\logicemph{日常决策中}:
\begin{itemize}
  \item 我们需要在不同行动方案间做选择
  \item 必须推断每个选择的后果
  \item 如果只关注已知为真的前提,就失去了推理的意义
\end{itemize}
\end{examplebox}

\logicwarn{逻辑悖论}:如果我们只研究前提已知为真的论证,那么当我们知道前提为真时,就不需要推理了——因为推理的目的正是帮助我们确定哪些前提应该被接受为真。

\logicemph{展望}:确定演绎论证有效性的系统方法将在本书第二部分详细介绍。

\chaptersummary{
\logicterm{有效性}和\logicterm{真实性}是两个根本不同的概念:有效性适用于论证,描述前提与结论之间的逻辑关系;真实性适用于命题,描述命题与现实世界的符合程度。

通过七个典型例子的分析,我们发现:结论的真假不能决定论证的有效性,论证的有效性也不能保证结论的真实性。只有\logicemph{可靠论证}(既有效又具有真前提的论证)才能确立结论的真实性。

逻辑学的核心任务是分析推理的形式结构,而不是判断命题的具体真假。这种分工使得逻辑学能够为科学研究和日常推理提供普遍适用的分析工具。
}