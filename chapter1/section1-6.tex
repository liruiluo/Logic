\section{论证和说明}

\begin{quotation}
\textit{区分论证和说明是逻辑分析的关键步骤,了解二者的差异能够帮助我们准确判断语段的真正意图。}
\end{quotation}

许多语段,无论是书面语还是口语,看起来好像是论证,实际上不是论证而是\textbf{说明}。即使有某些前提或结论指示词出现,例如"因为"、"由于"、"因此"等,也不能解决问题,因为这些语词既可用在论证中也可用在说明中。我们必须知道在这些语段中作者的意图。$^{[41]}$

\subsection{论证与说明的区别}

请比较下面两段话:

\begin{displayquote}
\textbf{例1:}为你自己积攒财宝在天上,天上没有虫子咬,不能锈坏,也没有贼挖窟窿来偷,\textit{因为}你的财宝在哪里,你的心也在哪里。
\end{displayquote}

\begin{displayquote}
\textbf{例2:}所以它(那座塔)名叫巴别,\textit{因为}耶和华在那里变乱天下人的言语。\\
——《创世记》11:19
\end{displayquote}

第一段话是一个清楚的\textbf{论证},它的结论,即一个人必须积攒财宝在天上,由前提(这里用"因为"来标明)一个人的财宝积攒在哪里,他的心也在哪里来支持。但是第二段不是论证,尽管它非常恰当地使用了"所以"一词。它\textbf{说明}了这座塔(其建造过程在《创世记》中有详细的叙述)为什么叫巴别。它告诉我们,因为之前人类在那里使用的是同一种语言,现在被许多语言变乱了,所以给塔起了这个名字。$^{[42]}$

这段话假设读者知道那座塔有这个名字,意图是说明为什么给塔起了这个名字。短语"所以它名叫巴别"不是结论而是完成了对这个名字的说明。句子"因为耶和华在那里变乱天下人的言语"不是前提,它不能作为相信巴别是那座塔名字的原因,因为巴别是那座塔的名字的事实是这段话所要为之做说明的读者所知道的。在这个语境中"因为"指示的是接下来要说明将巴别这个名字给予那座塔的原因。

\subsection{判断标准}

上面两段话说明一个事实,表面上相似的语段可能具有完全不同的功能。任何一个特定的语段究竟是\textbf{论证还是说明},这取决于那个语段所服务的\textbf{目的}。如果我们的目的是要\textbf{确立某个命题}$Q$的真,为此我们提出某个证据$P$来支持$Q$,我们可以恰当地说"$Q$因为$P$"。也就是说我们为$Q$建立一个论证,$P$是我们的前提。

但是假设$Q$是已知为真的。在这种情况下我们不必提出任何理由来支持它的真——但是我们可以希望对它为什么是真的给出一个\textbf{说明}。这样我们也可以说"$Q$因为$P$"——但在这种情况下,我们不是为$Q$建立一个论证,而是给出一个对$Q$的说明。

\subsection{实例分析}

在回答关于类星体(在我们的星系以外很远地方的一类天体)的外观颜色的问题时,一位科学家写道:

\begin{displayquote}
最远的类星体看上去像强烈的红外辐射光点。这是因为太空散布着吸收蓝光的氢微粒(大约每立方米两个微粒),如果你从可见的白光里过滤掉蓝光,那么剩下的就是红光。在其到达地球的数十亿光年的旅程中,类星体光被大气中的氢微粒吸去了全部的蓝光,留下的只有红光。$^{[43]}$
\end{displayquote}

这段话不是论证,它不是试图要让读者确信类星体具有像它们所显示的外观颜色,而是\textbf{说明}它们具有这个外观颜色的原因。

同样,在讨论不列颠对非洲早期发展的影响时,一个历史学家写道:

\begin{displayquote}
塞拉利昂在1808年成为英国直辖殖民地不是因为它的繁荣,而是因为它的萧条。由于战争和商业不景气的负担,塞拉利昂的私营公司不能支付它们的费用,而刚刚废除了贩卖奴隶制度的英国政府感到有必要接管它。$^{[44]}$
\end{displayquote}

这里没有对塞拉利昂在1808年成为英国直辖殖民地这个结论进行论证。塞拉利昂在那时确实成了英国直辖殖民地。但这是为什么呢?乃是由于在本例和前例中,"因为"很明显是\textbf{说明的标志},而不是论证的标志。

\subsection{区分方法}

我们怎么才能断定一个语段的目的是打算\textbf{说明}还是打算\textbf{说服}人呢?通常我们可以根据"$Q$因为$P$"这个形式提问,对于作者来说$Q$的身份是什么,以此来做出区分:

\begin{itemize}
  \item 若$Q$是一个其真实性\textbf{需要建立}的命题,那么"因为$P$"可能给出了支持其为真的前提,这样"$Q$因为$P$"就是一个\textbf{论证}。
  \item 若$Q$是一个已知其为真,或至少在这个语境中其真是没有疑问的命题,那么"因为$P$"就可能是对为什么$Q$成为真命题的阐释,这样"$Q$因为$P$"就是一个\textbf{说明}。
\end{itemize}

在一个说明中,人们必须把\textbf{被说明的东西}与\textbf{用来说明的东西}区别开来。在上面《创世记》所做出的说明中,被说明的内容是为何那座塔具有名字巴别,说明的内容是在那里耶和华变乱天下人的言语。在上面给出的历史学例子中,被说明的内容是塞拉利昂成为不列颠直辖殖民地,说明的内容是塞拉利昂公司的无支付能力和不列颠政府就此做出的回应。

\subsection{模糊界限}

有时被称做说明的东西实际上可能是论证,反之亦然。不久前,《纽约时报》因对待男女性别的不平等做法而受到一个读者的批评,因为它对一个著名女演员的不断增长的体重加以评论,但对在同一篇报道中提到的一个杰出商人的不断增长的体重却没有评论。后有另一个读者对此做出回应:

\begin{displayquote}
E.R.福克斯的抱怨——你特别提到凯瑟琳·丹尼芙"也许不像她以前那么苗条",但你没有提及唐纳德·杜鲁普不断增加的腰围——很容易说明。杜鲁普先生从未裸体出现在电影中以使他的体形成为人们感兴趣的事情。$^{[45]}$
\end{displayquote}

这不是一个真正的说明,而是一个\textbf{论证}。它有两个前提:
\begin{enumerate}
  \item 裸体外表出现在电影中使一个人的外表成为人们感兴趣的事情
  \item 杜鲁普先生从未以裸体外表出现在电影中,而丹尼芙女士有过
\end{enumerate}

因此,报纸对如此出现在电影中的名人的体形加以评论,而忽略未如此出现在电影中的名人的体形,这种做法就是合乎情理的(这个读者的主张),据此抱怨对待男女性别不平等就是不应当的。

为了区别说明和论证,我们必须对\textbf{语境}有一定的敏感性。总会有一些语段,其目的难以确定。一个其目的难以确定的语段可能需要给予两种同样有道理的"解读"——用一种方法去解读,被当做论证;用另一种方法解读,就是说明。

\begin{center}
\fbox{\parbox{0.9\textwidth}{
  \centering
  \textbf{论证与说明的关键差异}\\
  论证旨在证明某个命题的真实性,而说明则是解释已知为真的命题为何如此。\\
  辨别二者需要考察语段的目的、命题的地位以及语境因素。
}}
\end{center} 