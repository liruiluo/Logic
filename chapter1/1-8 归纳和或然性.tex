\section{归纳和或然性}

\begin{logicbox}[title=引言]
\textit{归纳论证是科学研究和日常推理的基础,它与演绎论证有着根本的区别,理解其依赖的或然性原则是进行有效分析的关键。}
\end{logicbox}

\logicterm{归纳论证}与演绎论证的根本区别在于其对前提与结论关系的不同要求。归纳论证\logicwarn{不要求}前提必然地支持结论,而是提出一个相对较弱但极其重要的要求:前提\logicterm{或然性地}支持结论。

\begin{theorembox}[title=归纳论证的基本特征]
\logicemph{核心特征}:
\begin{itemize}
  \item 前提为结论提供\logicterm{或然性支持}而非必然性支持
  \item \logicterm{或然性}本质上是必然性的缺乏
  \item \logicwarn{不适用有效性概念}:归纳论证既不是有效的也不是无效的
\end{itemize}

\logicemph{评估标准}:
\begin{itemize}
  \item 可以评估为"较强"或"较弱"
  \item 可以评估为"较好"或"较差"
  \item 前提授予结论的或然性程度越高,论证价值越大
\end{itemize}

\logicwarn{重要限制}:即使所有前提都为真且提供很强支持,归纳论证的结论也\logicemph{不是必然得出的}。
\end{theorembox}

\logicemph{科学意义}:对归纳论证进行评估是科学研究的核心任务之一。归纳理论、推理技巧、评估方法以及概率量化等内容将在本书第三部分详细介绍。$^{[46]}$

\subsection{归纳与演绎的根本区别}

归纳论证和演绎论证之间存在\logicemph{根本性的区别},这种区别体现在它们对新信息的不同反应上。

\begin{theorembox}[title=归纳论证的开放性特征]
\logicemph{或然性的可变性}:
\begin{itemize}
  \item 归纳论证的前提对结论的支持具有\logicterm{程度性}
  \item \logicterm{附加信息}可能强化或弱化这种或然性支持
  \item 新发现的事实可能改变我们对论证强度的评估
\end{itemize}

\logicemph{证据的开放性}:
\begin{itemize}
  \item 在归纳推理中,\logicwarn{永远不会穷尽所有相关证据}
  \item 总是存在发现新证据的可能性
  \item 新证据可能与已有证据相冲突
\end{itemize}

\logicwarn{确定性的限制}:正是由于这种证据的开放性,我们\logicemph{不能断定任何归纳论证的结论具有绝对的确定性},即使该结论被认为具有很高的可能性。
\end{theorembox}

\begin{theorembox}[title=演绎论证的封闭性特征]
与归纳论证形成鲜明对比,\logicterm{演绎论证}具有\logicemph{封闭性}特征:

\logicemph{二值性质}:
\begin{itemize}
  \item 演绎论证\logicwarn{不能}越来越好或越来越差
  \item 在显示前提与结论关系上要么成功(有效)要么失败(无效)
  \item 不存在程度上的变化
\end{itemize}

\logicemph{有效性的稳定性}:
\begin{itemize}
  \item 如果论证有效,\logicwarn{没有附加前提}可以增强其有效性
  \item 有效性不受外部信息影响
  \item 逻辑关系一旦确立就不会改变
\end{itemize}
\end{theorembox}

\begin{examplebox}[title=演绎论证的稳定性示例]
考虑经典的三段论:
\begin{itemize}
  \item \logicterm{前提1}:凡人皆终有一死
  \item \logicterm{前提2}:苏格拉底是人
  \item \logicterm{结论}:苏格拉底终有一死
\end{itemize}

这个结论\logicemph{必然地}从前提推出,无论我们后来发现什么其他信息(如苏格拉底的外貌、天使的性质、奶牛的习性等),都\logicwarn{不能影响}原论证的有效性。
\end{examplebox}

\logicemph{单调性原理}:对于每个有效的演绎论证,无论添加什么性质的前提,原结论都能从扩大的前提集中必然地推出。\logicwarn{有效性具有绝对性}:没有任何东西能使有效论证"更有效",也没有任何东西能使有效推导"更严格"或"更合乎逻辑"。

\subsection{归纳论证示例及其特征}

归纳论证的情况截然不同。归纳论证所声称的前提与结论之间的关系\logicwarn{远非如此严格},与演绎论证有本质区别。

\begin{examplebox}[title=归纳论证的可变性示例]
考虑以下归纳论证:
\begin{displayquote}
大部分公司法律顾问是保守主义者,\\
安吉拉•帕尔默瑞是一个公司法律顾问,\\
所以安吉拉•帕尔默瑞很可能是保守主义者。
\end{displayquote}

\logicemph{初始评估}:这是一个相当强的归纳论证。如果两个前提都为真,结论很可能为真而非假。
\end{examplebox}

然而,与苏格拉底必死性的演绎论证形成鲜明对比,\logicemph{新信息可能显著改变这个归纳论证的强度}:

\begin{examplebox}[title=弱化论证的新信息]
假设我们发现:
\begin{itemize}
  \item \logicterm{新信息1}:安吉拉•帕尔默瑞是美国公民自由权协会(ACLU)的一名官员
  \item \logicterm{新前提}:美国公民自由权协会的大部分官员不是保守主义者
\end{itemize}

\logicemph{结果}:原结论(安吉拉是保守主义者)不再显得很可能,原论证被\logicwarn{大大弱化}。

\logicemph{极端情况}:如果新前提改为全称命题"没有美国公民自由权协会的官员是保守主义者",那么就能从扩大的前提集\logicterm{演绎地推出}与原结论相反的结论。
\end{examplebox}

\begin{examplebox}[title=强化论证的新信息]
相反,假设我们发现以下信息:
\begin{itemize}
  \item \logicterm{新信息1}:安吉拉•帕尔默瑞长期是国家步枪协会(NRA)的一名官员
  \item \logicterm{新信息2}:安吉拉•帕尔默瑞被任命为保守的《国家评论》报的特约撰稿人
\end{itemize}

\logicemph{结果}:通过这个扩大的前提集,原来的结论得到了\logicemph{比原来更强的支持}。这些新信息都指向同一个方向,增强了安吉拉是保守主义者的可能性。
\end{examplebox}

这个例子清楚地展示了归纳论证的\logicterm{非单调性}特征:新信息可能增强或削弱论证的强度,这与演绎论证的单调性形成鲜明对比。

\subsection{两类论证的本质特征}

\begin{theorembox}[title=两类论证的本质特征对比]
归纳和演绎的根本区别在于它们对前提与结论关系的不同断言:

\logicemph{演绎论证}:
\begin{itemize}
  \item 断言结论从前提\logicterm{绝对必然地}推出
  \item 必然性\logicwarn{不是程度问题}
  \item 不受任何其他情况影响
  \item 具有\logicterm{单调性}:新信息不改变有效性
\end{itemize}

\logicemph{归纳论证}:
\begin{itemize}
  \item 断言结论仅仅\logicterm{或然性地}从前提推出
  \item 或然性\logicwarn{是程度问题}
  \item 受其他情况影响
  \item 具有\logicterm{非单调性}:新信息可能改变强度
\end{itemize}
\end{theorembox}

\logicwarn{识别注意事项}:
\begin{itemize}
  \item 归纳论证\logicwarn{不总是}明确表明其结论仅在某种或然程度上推出
  \item 论证中出现"或然性"一词\logicwarn{不一定}表明该论证是归纳的
  \item 存在关于或然性本身的\logicterm{严格演绎论证}$^{[47]}$(将在第14章讨论)
\end{itemize}

\chaptersummary{
\logicterm{演绎论证}和\logicterm{归纳论证}代表了两种根本不同的推理模式。演绎论证的结论\logicemph{必然地}从前提推出,具有封闭性和单调性,新信息不能影响其有效性。归纳论证的结论仅\logicemph{或然地}从前提推出,具有开放性和非单调性,其强度可能随新证据的增加而增强或减弱。

这种根本区别决定了两类论证采用不同的评估标准:演绎论证评估有效性,归纳论证评估强度。它们在不同领域发挥着不同的作用:演绎论证在数学和逻辑中占主导地位,归纳论证在科学研究和日常推理中不可或缺。
}