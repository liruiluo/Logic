\section{命题与语句}

\begin{logicbox}[title=引言]
\textit{命题是逻辑推理的基本单位,理解命题的本质及其表达方式是进行逻辑分析的前提。}
\end{logicbox}

\logicterm{命题}是可以被判断为真或假的陈述。这一特征使命题区别于其他类型的语言表达:

\begin{theorembox}[title=命题的基本特征]
\begin{itemize}
  \item \logicemph{可判断性}:命题可以被肯定或否定
  \item \logicemph{真假性}:每个命题都有确定的真值(真或假)
  \item \logicemph{陈述性}:命题断定某种事态的存在或不存在
\end{itemize}
\end{theorembox}

与命题不同,\logicwarn{问题、命令和感叹}都不具备真假性:
\begin{itemize}
  \item 问题用于询问信息,如"今天几点了?"
  \item 命令用于指示行动,如"请关上门。"
  \item 感叹用于表达情感,如"多么美丽的景色!"
\end{itemize}

这些语言形式都不能被判断为真或假,因此不是命题。

\logicemph{真假二值原理}是命题逻辑的基础:任何命题都必须是\logicemph{或真或假}的,不存在第三种可能。即使我们暂时不知道某个命题的真假值,它本身仍然具有确定的真假性。例如,"宇宙中其他星球上存在生命"这个命题,虽然我们目前无法确定其真假,但它客观上要么为真,要么为假。

\subsection{命题与语句的区别}

在逻辑学中,我们必须明确区分\logicterm{命题}和\logicterm{语句}这两个概念:

\begin{theorembox}[title=命题与语句的区别]
\begin{itemize}
  \item \logicterm{语句}:表达命题的具体语言形式,包括词汇、语法结构等
  \item \logicterm{命题}:语句所表达的抽象意义内容,即所断定的事态
  \item \logicemph{关键区别}:不同的语句可以表达同一个命题
\end{itemize}
\end{theorembox}

\begin{examplebox}[title=同一命题的不同表达]
以下两个语句在形式上完全不同,但表达了同一个命题:
\begin{itemize}
  \item Leslie won the election.(莱斯利赢了这场选举。)
  \item The election was won by Leslie.(这场选举由莱斯利赢得。)
\end{itemize}

虽然这两个语句在词汇数量、语法结构和表达方式上都不相同,但它们断定的是同一个事实,因此表达了同一个命题。
\end{examplebox}

命题的另一个重要特征是其\logicemph{语言无关性}。语句总是属于特定的语言,而命题作为抽象的意义内容,可以用不同语言的语句来表达:

\begin{examplebox}[title=跨语言的命题表达]
以下四个语句分别来自不同的语言,但表达了同一个命题:
\begin{itemize}
  \item It is raining.(英语)
  \item Está lloviendo.(西班牙语)
  \item Il pleut.(法语)
  \item Es regnet.(德语)
\end{itemize}

尽管这些语句在语言形式上完全不同,但它们都断定了同一个事态——"正在下雨",因此表达了同一个命题。
\end{examplebox}

\subsection{语境与命题}

\logicterm{语境}对命题的确定具有重要影响。同一个语句在不同的语境中可能表达完全不同的命题:

\begin{examplebox}[title=语境对命题的影响]
考虑语句:"美国最大的州曾经是一个独立的共和国。"

\begin{itemize}
  \item \logicemph{20世纪上半叶}:这个语句表达了关于得克萨斯州的\logicterm{真命题}
  \item \logicemph{21世纪}:同一语句表达了关于阿拉斯加州的\logicwarn{假命题}
\end{itemize}

这说明语句的意义和所表达的命题会随着\logicterm{时间语境}的变化而改变。
\end{examplebox}

\logicwarn{术语说明}:在逻辑学文献中,"命题"和"陈述"这两个术语经常互换使用,虽然它们在严格意义上并不完全等同。本书将根据上下文的需要使用这两个术语,但读者应理解它们在大多数情况下指向同一概念。

\subsection{简单命题与复合命题}

根据内部结构的复杂程度,命题可以分为\logicterm{简单命题}和\logicterm{复合命题}两大类:

\begin{theorembox}[title=命题的分类]
\begin{itemize}
  \item \logicterm{简单命题}:不包含其他命题作为组成部分的命题
  \item \logicterm{复合命题}:由两个或多个简单命题通过逻辑联结词组合而成的命题
\end{itemize}
\end{theorembox}

前面提到的"莱斯利赢了这场选举"、"天在下雨"等都是简单命题的例子。而复合命题则包含其他命题作为其组成部分。让我们通过一个历史实例来理解复合命题:

\begin{quotation}
美军与俄军正迅速赶往易北河会师。英军已兵临汉堡和不来梅城下,把占领丹麦的德军置于被切断后路的险境。意大利的波伦亚已经失守,而亚历山大率领的盟军部队正向波河流域挺进。俄军已于4月13日攻克维也纳,正沿着多瑙河乘胜前进。\cite{shirer1960}
\end{quotation}

这段文字包含了多个复合命题。例如,"英军已兵临汉堡和不来梅城下"是一个\logicterm{联言命题},它由两个简单命题组成:
\begin{itemize}
  \item "英军已兵临汉堡城下"
  \item "英军已兵临不来梅城下"
\end{itemize}

\logicemph{联言命题的特点}是:当我们断定一个联言命题为真时,就等于同时断定其所有组成部分都为真。

然而,并非所有复合命题都像联言命题那样断定其所有组成部分为真。让我们看看其他类型的复合命题:

\begin{examplebox}[title=选言命题]
"巡回法庭或者是有用的,或者是无用的。"\cite{lincoln1861}

这是一个\logicterm{选言命题}(析取命题),它的特点是:
\begin{itemize}
  \item 不断定任何一个具体的分支命题为真
  \item 只断定至少有一个分支命题为真
  \item 即使某个分支命题为假,整个选言命题仍可能为真
\end{itemize}
\end{examplebox}

\begin{examplebox}[title=假言命题]
"如果上帝不存在,则有必要捏造一个上帝。"\cite{voltaire1770}

这是一个\logicterm{假言命题}(条件命题),它的特点是:
\begin{itemize}
  \item 不断定前件"上帝不存在"为真
  \item 不断定后件"有必要捏造一个上帝"为真
  \item 只断定前件与后件之间的条件关系
  \item 即使前件和后件都为假,整个假言命题仍可能为真
\end{itemize}
\end{examplebox}

\chaptersummary{
\logicterm{命题}是逻辑推理的基本单位,具有明确的真假性。命题与语句不同:语句是表达命题的语言形式,而命题是语句所表达的抽象内容。同一命题可以用不同语句表达,也可以用不同语言表达。命题分为\logicemph{简单命题}和\logicemph{复合命题},后者通过逻辑联结词将简单命题组合而成,包括联言命题、选言命题、假言命题等不同类型。
}

在后续章节中,我们将深入分析各种简单命题和复合命题的逻辑结构及其在论证中的作用。