\section{命题与语句}

\begin{logicbox}[title=引言]
\textit{命题是逻辑推理的基本单位,理解命题的本质及其表达方式是进行逻辑分析的前提。}
\end{logicbox}

\logicterm{命题}是一种可以被肯定或否定的东西。也就是说,命题不同于问题、命令和感叹。问题可以被提问,命令可以被下达,感叹可以被发出,但它们本身都不能被肯定或否定。唯有命题断定了事情是(或不是)如此这般,因而也唯有命题才会是\logicemph{真的}或者是\logicemph{假的}。真与假并不适用于问题、命令或感叹。

再者,任一命题必是\logicemph{或真或假}的,尽管我们可能并不知道某一特定命题究竟是真的还是假的。"宇宙中其他星球上有生命存在"这个命题,就是一个我们迄今还不知道其真假的命题。但对地球外生命之存在的这种断定本身或者是真的,或者不是真的。简言之,或真或假是命题的一个基本特征。

\subsection{命题与语句的区别}

\begin{theorembox}[title=重要区分]
依学界惯例,要把\logicterm{命题}与用来断定命题的\logicterm{语句}区别开来。两个由不同语词以不同方式组成的语句,可能在同一语境中具有同样的意义,被用来表达同一个命题。
\end{theorembox}

例如:

\begin{itemize}
  \item Leslie won the election.(莱斯利赢了这场选举。)
  \item The election was won by Leslie.(这场选举由莱斯利赢得。)
\end{itemize}

这显然是两个不同的语句,前一个由四个词组成,后一个是六个词,以及起首词不同等等。而这两个陈述句无疑具有相同的意义。\logicterm{命题}这个术语所指谓的就是人们通常使用陈述句所断定的东西。

再者,一个语句总是使用它的特定语言的语句,而命题并不属于任何特定的语言,一个特定的命题可以在许多语言中被断定。例如:

\begin{itemize}
  \item It is raining.(天在下雨。下同)
  \item Está lloviendo.
  \item Il pleut.
  \item Es regnet.
\end{itemize}

这当然是四个不同的语句,分属不同的语言:英语、西班牙语、法语和德语。但它们都具有同样的意义,从而都可以用来断定同一命题。

\subsection{语境与命题}

在不同的\logicterm{语境}中,同一个语句也可能被用来做非常不同的陈述。例如:

\begin{center}
"美国最大的州曾经是一个独立的共和国。"
\end{center}

这个语句在20世纪上半叶说出,就是做了关于得克萨斯州的一个真陈述;而在现在说出就做了关于阿拉斯加州的一个假陈述。显然,时间语境的变化,可以使完全相同的语句断定非常不同的命题或陈述。("命题"和"陈述"这两个术语并不完全同义,但在逻辑研究的文本中它们经常被用做同义词。有些逻辑学专家更喜欢使用"陈述"而不愿意使用"命题",但在逻辑学历史上后者更为常用。本书同时使用这两个术语。)

\subsection{简单命题与复合命题}

上面所举出的命题的例子都是\textbf{简单命题}:"莱斯利赢了这场选举","天在下雨"等等,然而命题也经常是\textbf{复合的}——在一个命题中包含着别的命题。考虑如下关于1945年希特勒第三帝国末日的一段话:

\begin{quotation}
美军与俄军正迅速赶往易北河会师。英军已兵临汉堡和不来梅城下,把占领丹麦的德军置于被切断后路的险境。意大利的波伦亚已经失守,而亚历山大率领的盟军部队正向波河流域挺进。俄军已于4月13日攻克维也纳,正沿着多瑙河乘胜前进。\cite{shirer1960}
\end{quotation}

这段话就含有几个复合命题。例如,"英军已兵临汉堡和不来梅城下",就是"英军已兵临汉堡城下"与"英军已兵临不来梅城下"这两个命题的\textbf{联言式}。而这个联言命题本身又作为分支属于一个更大的联言命题:"英军已兵临汉堡和不来梅城下,(英军)把占领丹麦的德军置于被切断后路的险境。"这段话中的每一个命题都是被肯定的,也就是说,都被断言为真。肯定两个命题的联言式,就等于同时肯定这两个分支命题。

但是,也有一些复合命题并不断定其所有分支命题为真。例如:

\begin{center}
"巡回法庭或者是有用的,或者是无用的。"\cite{lincoln1861}
\end{center}

这是一个\textbf{选言命题}(或称析取命题),它并没有肯定任何一个分支命题,而只是肯定了整个复合的"或者一或者"析取命题。析取命题为真时,其某个分支命题可以为假。再如:

\begin{center}
"如果上帝不存在,则有必要捏造一个上帝。"\cite{voltaire1770}
\end{center}

这个复合命题是一个\textbf{假言命题}(或称条件命题),其支命题也同样没有被肯定,既没有肯定"上帝不存在",也没有肯定"有必要捏造一个上帝",而只是通过这种假言或条件陈述肯定了整个"如果一则"命题。即使分支命题均为假,条件陈述亦可为真。

\chaptersummary{

  命题是断定事实的陈述,必定或真或假,可以通过不同语句表达,\\
  且可以组合形成各种复合结构。

}

本书将逐次分析多种简单命题和复合命题的内在结构。