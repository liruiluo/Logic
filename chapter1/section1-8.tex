\section*{1.8 归纳和或然性}
归纳论证不要求它们的前提必然地支持结论,纵然其前提是真的。它提出一个较弱的但仍然是很重要的要求:其前提或然性地支持结论。或然性总是必然性的缺乏,因而上述关于有效性和无效性的讨论并不适用于归纳论证:归纳论证既不是有效的也不是无效的。 ${ }^{[46]}$ 当然,我们仍然可以对它们进行评估。实际上,对归纳论证进行评估是任何领域的科学家最主要的任务之一。归纳论证的前提为它的结论提供某种支持,前提授予结论的或然性程度越高,论证的价值也就越大。一般情况下,我们可以说归纳论证"较好"或"较差","较弱"或"较强",等等。但是,甚至在所有前提都是真的并且对其结论提供了非常强的支持的情况下,归纳论证的结论也不是必然得出的。归纳理论,归纳推理的技巧,评估归纳论证的方法,以及量化和推测或然概率的方法等将在本书第三部分详加介绍。

归纳论证和演绎论证之间的区别是根本性的。因为归纳论证的前提对其结论的支持都具有某种程度的或然性,附加的信息就有可能强化或弱化这种或然性。新发现的事实可以使我们改变对或然性的估价,可能导致我们对归纳论证的判定比我们原想的更好(或更差)。在归纳论证的领域——即使当结论被认为具有很高可能性的情况下——永远不会穷尽所有的证据。正是这种发现与我们以前所相信的证据相冲突的新材料的可能性,使得我们不能断定任何归纳论证的结论具有绝对的确定性。

相反,演绎论证却不能越来越好或越来越差。它们在显示前提和结论之间的令人信服的关系上要么成功要么失败。这个对比揭示了演绎和归纳之间的根本差异。如果一个演绎论证是有效的,就没有附加的前提可以增强这个论证的有效性。例如,如果凡人皆终有一死,并且如果苏格拉底是人,我们就可以毫无保留地得出结论,苏格拉底终有一死——即苏格拉底终有一死的结论总能从那两个前提推论出来,而不管世界上别的什么可能是真的,也不管别的什么信息被发现或增加到该论证的前提当中。比如我们后来又知道苏格拉底难看,或天使永生,或奶牛产奶,但这些发现和别的发现都不能对原来的论证产生任何影响。

就每一个有效的演绎论证来说,不管附加前提的性质如何,从其前提必然推出的结论同样也能必然地从任何扩大的前提集推论出来。如果一个论证是有效的,世界上就没有什么东西能使它更有效;如果一个结论是从某个前提集有效推出的,就没有什么东西可以增加到这个前提集当中使得该结论的推出变得更严格、更合乎逻辑或更有效。

但归纳论证并不是这样,归纳论证所断言的前提和结论之间的关系远非如此严格,与演绎论证有本质上的不同。考虑下面这个归纳论证:

\begin{displayquote}
大部分公司法律顾问是保守主义者,\\
安吉拉•帕尔默瑞是一个公司法律顾问,所以安吉拉•帕尔默瑞很可能是保守主义者。
\end{displayquote}

45 这是一个非常好的归纳论证,它的第一个前提是真的,如果它的第二个前提也是真的,则其结论很可能就是真的而不是假的。但是在这种场合(与有关苏格拉底的必死性的论证形成鲜明对照),若增加某个新前提到原来的论证之中,就可能会弱化或强化(依据新前提的内容)原来的论证。假设我们还知道:

\section*{安吉拉•帕尔默瑞是美国公民自由权协会(ACLU)的一名官员。}
又假设在原论证中增加一个(真)前提:

美国公民自由权协会的大部分官员不是保守主义者。

那么那个结论(安吉拉•帕尔默瑞是一个保守主义者)不再看起来非常可能,原来的归纳论证由于这个关于安吉拉•帕尔默瑞的附加信息的出现而被大大弱化。而如果上述前提被改造成全称命题:

没有美国公民自由权协会的官员是保守主义者。

那么就会有效地从被断定的前提集演绎地推出与原来结论相反的结论。

另一方面,假设我们通过增加下面的附加前提来扩大原来的前提集:

安吉拉•帕尔默瑞长期是国家步枪协会(NRA)的一名宫员。

和\\
安吉拉•帕尔默瑞被任命为保守的《国家评论》报的特约撰稿人。

那么通过这个扩大了的前提集,原来的结论就得到了比原来的前提集更大的支持。

总之,归纳和演绎的区别依赖于两类论证对前提和结论之间的关系所作断言的性质。我们可以将两类论证的特征表示如下:

演绎论证是一种其结论被断言为从其前提绝对必然地推出的论证,这种必然性不是一个程度问题,不以任何其他事物情况为转移。反之,归纳论证是一种其结论被断言为仅仅或然性地从其前提推出的论证,这种或然性是一个程度问题,其程度受可能出现的其他事物情况的影响。

归纳论证并不总是明确表明其结论仅仅是在某种或然程度上推出来的。另一方面,在一个论证中出现"或然性"一词也并不一定表明该论证就是归纳的。这是因为有一些严格的演绎论证是关于或然性本身的。 ${ }^{[47]}$ 在这种论证中,事件之间的确定联系的或然性是从另外的事件的或然性演绎推出的,这个问题将在第 14 章讨论。