\section{演绎和有效性}

\begin{logicbox}[title=引言]
\textit{演绎论证是逻辑学的核心研究对象,了解演绎论证的本质及其有效性标准是进行逻辑分析的基础。}
\end{logicbox}

每一个论证都是断言其前提为结论的真提供理由。实际上这种断言正是论证的标志。但是论证有两大不同种类:\logicterm{演绎论证}和\logicterm{归纳论证}。这两类论证在其前提支持结论的方式上有着根本的不同。本节我们对演绎论证作一个简要的阐释。

任一演绎论证均断言其前提\logicemph{决定性地}(conclusively)支持结论。相反,归纳论证均没有这种断言。在对一个语段的解释中,如果我们判定它做出了这样的断言,我们就将其视为演绎论证;如果我们判定它没有做出这样的断言,我们就将其视为归纳论证。因为每个论证都会对决定性支持结论或者做出断言,或者不做出断定,所以每个论证或者是演绎的,或者是归纳的。

当一个论证断言它的前提(如果是真的)为它的结论的真提供了无可辩驳的理由时,这个断言或者是正确的或者是不正确的。如果是正确的,这个论证就是\logicemph{有效的}。如果不是正确的(也就是说,即使前提是真的,也不能无可辩驳地确立其结论的真),那么这个论证就是\logicwarn{无效的}。

\begin{theorembox}[title=有效性定义]
对于逻辑学家而言,有效性这个术语仅仅对演绎论证才是适当的。说一个演绎论证是有效的,就是说如果其前提是真的,其结论为假就是不可能的。这样我们可把\logicterm{有效性}定义如下:一个演绎论证是有效的,即如果其前提是真的,则其结论必定是真的。
\end{theorembox}

每个演绎论证都要求其前提为其结论的真提供担保,但并非所有演绎论证都能做到这个要求。不能做到这个要求的演绎论证就是无效的。

因为每个演绎论证就其目标的实现而言或者是成功的或者是不成功的,所以每个演绎论证或者是有效的或者是无效的。这一点非常重要:如果一个演绎论证不是有效的,它一定是无效的;如果它不是无效的,它一定是有效的。

演绎逻辑的中心任务(将在本书第二部分详细讨论)就是对有效论证和无效论证做出区分。为此,古往今来逻辑学家们发明了许多非常有效的方法。但用来判定论证有效性的传统方法不同于大多数现代逻辑学家使用的方法。前者被叫做\logicterm{古典逻辑},发端于亚里士多德的分析工作,本书的第5、6、7章将对此加以阐释。\logicterm{现代符号逻辑}的方法将在本书的第8、9、10章详加介绍。尽管两个流派的逻辑学家们在方法上和对某些论证的具体阐释上不尽一致,但他们都同意演绎逻辑的主要任务是开发一种能使我们区分有效论证与无效论证的工具。

\chaptersummary{
演绎论证断言其前提\logicemph{决定性地}支持结论,当且仅当前提为真时结论必然为真,这种论证才被视为\logicemph{有效};判定论证有效性是演绎逻辑的核心任务。\logicterm{古典逻辑}和\logicterm{现代符号逻辑}都致力于区分有效论证与无效论证。
}