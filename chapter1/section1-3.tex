\section{论证、前提与结论}

\begin{quotation}
\textit{论证是逻辑思维的核心,通过合理的前提推导出有效的结论是逻辑学的基本过程。}
\end{quotation}

命题是构成\textbf{论证}的部件。\textbf{推论}这个术语则指谓以一个或更多命题作为出发点,得出另一命题的过程。逻辑学家即通过检验这种过程的出发点与结果及它们之间的关系,以判定一个推论是否正确。这种命题系列即构成一个论证。因而对于任一可能的推论,都有一个相应的论证。

\subsection{论证的本质}

论证是逻辑学所关心的主要对象。逻辑学家所使用的\textbf{论证}一词,就是指谓任一这样的命题组:一个命题从其他命题推出,后者给前者之为真提供支持或根据。当然,"论证"一词也经常在其他含义上使用,但在逻辑学中严格地限于上述含义。

显然,在这种严格含义上,一个论证不只是一组命题的汇集,一段包含一些相互关联的命题的话语可能并不包含任何论证。若要给出一个论证,则命题系列必须含有一种结构,对这种结构的描述通常要使用\textbf{前提}与\textbf{结论}这两个术语。一个论证的结论,就是以论证中的其他命题为根据所得出的那个命题,而这些其他命题,即被肯定(或假定)为接受结论的根据或理由的命题,则是该论证的前提。

\subsection{最简论证的形式}

最简单的论证是由一个前提和一个从该前提推出或被它所蕴涵的结论构成的论证。这种论证的前提与结论可以分别用两个不同的语句表述,例如出现在阿拉巴马州地理课本封签中的如下论证:

\begin{quotation}
在地球上最先出现生命时没有人存在。因此,任何关于生命起源的陈述都应视为理论的而非事实的陈述。
\end{quotation}

最简论证的前提和结论也可能被表述在同一个句子中,如下述论证:

\begin{quotation}
因为最近的进化史研究已经证明所有人都是从同一小群非洲祖先演变而来,若仍相信种族间有极大差异,则如同仍相信地球是扁平的一样荒谬可笑。${}^{[4]}$
\end{quotation}

即使在最简论证中,结论陈述也有可能出现在那个唯一的前提之前。这时候,两个命题同样既可以两个语句出现,亦可在同一个句子中出现。前者例如:

\begin{quotation}
食品与药物管理局应立即禁止烟草买卖。要知道,抽烟是导致死亡的一种最可预防的原因。${}^{[5]}$
\end{quotation}

同一陈述中所表述的论证结论在前的一个例子是:

\begin{quotation}
凡法皆恶,乃因凡法皆为自由之违背。${}^{[6]}$
\end{quotation}

大多数论证都比这些论证复杂得多。我们将会看到,有些非常复杂的论证包含由多个支命题构成的复合命题。但是不管简单还是复杂,任何论证都是由一组命题构成,其中一个命题是\textbf{结论},其他命题是用以支持结论的\textbf{前提}。

\subsection{论证与假言命题的区别}

因为一个论证由一组命题构成,故而单一命题自身不可能是论证。但有些复合命题与论证非常近似,需细心辨识以免把它们混同于论证。考虑如下假言命题:

\begin{displayquote}
如果火星在其具有与地球相似的大气层和相似气候的早期曾有生命演化,那么目前科学家确信的在我们的星系中存在的无数颗其他星球上也会有生命演化。
\end{displayquote}

在这个\textbf{假言命题}中,无论第一个支命题"火星在其具有与地球相似的大气层和相似气候的早期曾有生命演化",还是第二个支命题"目前科学家确信的在我们的星系中存在的无数颗其他星球上也会有生命演化",都没有被肯定。整个命题肯定的只是前者蕴涵后者,而两者却可以都是假的。其中没有推论得以构成,没有结论被论证为真。这是一个假言命题,而不是一个论证。现再请考虑如下段落:

\begin{quotation}
看来,目前科学家确信的在我们的星系中存在的无数颗其他星球上会有生命演化,因为火星在其具有与地球相似的大气层和相似气候的早期非常可能曾有生命演化。${}^{[7]}$
\end{quotation}

此处我们的确得到一个\textbf{论证}。命题"火星非常可能曾有生命演化"被肯定为一个前提,而命题"无数颗其他星球上会有生命演化"被从该前提推出并被论证为真。这样,假言命题可能看上去很像一个论证,但其并不是一个论证,两者不应混淆。如何识别论证,是后面1.5节讨论的主题。

\subsection{有结构的命题系列与论证}

最后应强调指出,任一论证都是有结构的命题系列,但并非任一有结构的命题系列都是论证。请考虑从近期的非洲游记上摘录的一段话:

\begin{displayquote}
骆驼并不在驼峰中储水。它们每次喝水都非常猛,在十分钟的时段中能饮入28加仑水,把这些水均匀地分布到全身。而后其耗水却非常节俭。它们的尿液黏稠、粪便干燥,并以浅呼吸而紧闭其口。如非不得已,它们一般不出汗……在失水程度达到体重的三分之一时也能存活,然后再痛饮一次并且感觉良好。${}^{[8]}$
\end{displayquote}

这段有结构的命题系列中并没有任何论证。

\begin{center}
\fbox{\parbox{0.9\textwidth}{
  \centering
  \textbf{论证的基本结构}\\
  论证是由前提和结论组成的有结构命题系列,\\
  其中结论由前提通过推理过程得出。
}}
\end{center} 