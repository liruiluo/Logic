\section*{1.1 什么是逻辑学}
逻辑学是研究用于区分正确推理与不正确推理的方法和原理的学问。正确推理的界定有着许多客观标准,而如果不了解这些标准,也就无法运用它们。逻辑学研究的宗旨,就是发现并塑述这些标准,使之能够检验论证,把好的论证与坏的论证区别开来。

逻辑学家所关心的推理遍及所有领域:科学与医药,伦理与法律,政治与商务,运动与博弃,直至平凡的日常生活。其中所使用的多种多样的推理,都是逻辑学家感兴趣的。本书将要分析的大量论证,就涉及许多非常不同的领域。但我们所关心的不是这些论证的题材,而始终是它们的形式(form)与品质(quality),目的在于学会如何检验与评价论证。

逻辑学家并不关心推理的思想过程,而只关心这种过程的结果,即论证。论证是推理的产品,可以被完整地写出来,并予以检验与分析。对逻辑学家来说,就每一个论证都可提出如下问题:论证所得出的结论是从论证所使用的前提或假定推出的吗?论证的前提能够为接受其结论提供良好的理由吗?如果论证的前提的确能够为接受结论提供充分的根据,也就是说,如果断定前提为真就能够保证可断定结论为真,那么其所使用的推理就是正确的,否则就是不正确的。

不能说只有学了逻辑学才能进行良好的或正确的推理,正如不能说只有学了生理学的运动员才能跑得快一样。并不懂得发生在其身体上的实际 4 过程的运动员经常有出色的表现,而有些学习生理学的优等生,尽管有许多关于身体机能方面的知识,但在运动场上却难有作为。同样,学了逻辑学并不能确保能够进行正确的推理。

然而,一个学了逻辑学的人,比之一个从未思考过推理原理的人,其进行正确推理的可能性要大得多。这首先因为学习逻辑学可以习得许多检验推理的正确性的方法,能够更容易地识别推理错误,从而使这些错误不容易在推理中滞留。在这些被识别出的错误中,有些普通的推理谬误,或所谓"自然"错误,是只要把它们充分弄清就很容易避免的。

学习逻辑学能够提高人的推理素养的另一个原因是:它给了人们训练 (practice)分析论证以及建构自己的论证的一种机会。推理是一种我们不但要僅而且要做(do)的事情,因而其既属科学亦属艺术,需要把握技术

和开发技能。就此目标,本书提供了丰富的习题训练,以增强这种技术与技能。

在人类生活中,有些事情并不能完全用逻辑方法加以分析,有些问题并不能用论证(即使是良好的论证)来解决。有时求助于情感比逻辑论证更有效力,在某些语境中或许也更为适当。但是,在那些必须依靠下判断的地方,正确推理终究是其最坚实的基础。运用逻辑学的方法与技术,人们可以有效地区分正确的推理与不正确的推理,这种方法与技术就是本书的主题。 