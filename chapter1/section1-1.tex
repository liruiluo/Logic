\section{什么是逻辑学}

\begin{quotation}
\textit{逻辑学为我们提供了判断推理正确性的方法和标准,使我们能够有效地思考和表达自己的想法。}
\end{quotation}

\textbf{逻辑学}是研究用于区分\textbf{正确推理}与\textbf{不正确推理}的方法和原理的学问。正确推理的界定有着许多客观标准,而如果不了解这些标准,也就无法运用它们。逻辑学研究的宗旨,就是发现并塑述这些标准,使之能够检验论证,把好的论证与坏的论证区别开来。

\subsection{逻辑学的研究对象}

逻辑学家所关心的推理遍及所有领域:
\begin{itemize}
  \item 科学与医药
  \item 伦理与法律
  \item 政治与商务
  \item 运动与博弈
  \item 平凡的日常生活
\end{itemize}

其中所使用的多种多样的推理,都是逻辑学家感兴趣的。本书将要分析的大量论证,就涉及许多非常不同的领域。但我们所关心的不是这些论证的\textbf{题材},而始终是它们的\textbf{形式}(form)与\textbf{品质}(quality),目的在于学会如何检验与评价论证。

\subsection{论证与推理}

逻辑学家并不关心推理的思想\textbf{过程},而只关心这种过程的\textbf{结果},即\textbf{论证}。论证是推理的产品,可以被完整地写出来,并予以检验与分析。对逻辑学家来说,就每一个论证都可提出如下问题:
\begin{enumerate}
  \item 论证所得出的\textbf{结论}是从论证所使用的\textbf{前提}或假定推出的吗?
  \item 论证的前提能够为接受其结论提供良好的理由吗?
\end{enumerate}

如果论证的前提的确能够为接受结论提供充分的根据,也就是说,如果断定前提为真就能够保证可断定结论为真,那么其所使用的推理就是\textbf{正确的},否则就是\textbf{不正确的}。

\subsection{逻辑学习的意义}

不能说只有学了逻辑学才能进行良好的或正确的推理,正如不能说只有学了生理学的运动员才能跑得快一样。并不懂得发生在其身体上的实际过程的运动员经常有出色的表现,而有些学习生理学的优等生,尽管有许多关于身体机能方面的知识,但在运动场上却难有作为。同样,学了逻辑学并不能确保能够进行正确的推理。

然而,一个学了逻辑学的人,比之一个从未思考过推理原理的人,其进行正确推理的可能性要大得多。这主要有两个原因:

\begin{enumerate}
  \item 学习逻辑学可以习得许多检验推理的正确性的方法,能够更容易地识别\textbf{推理错误},从而使这些错误不容易在推理中滞留。在这些被识别出的错误中,有些普通的\textbf{推理谬误},或所谓"自然"错误,是只要把它们充分弄清就很容易避免的。
  
  \item 学习逻辑学能够提高人的推理素养,它给了人们训练(practice)\textbf{分析论证}以及\textbf{建构论证}的机会。推理是一种我们不但要学而且要做(do)的事情,因而其既属\textbf{科学}亦属\textbf{艺术},需要把握技术和开发技能。就此目标,本书提供了丰富的习题训练,以增强这种技术与技能。
\end{enumerate}

\subsection{逻辑学的局限与价值}

在人类生活中,有些事情并不能完全用逻辑方法加以分析,有些问题并不能用论证(即使是良好的论证)来解决。有时求助于\textbf{情感}比逻辑论证更有效力,在某些语境中或许也更为适当。但是,在那些必须依靠理性判断的地方,\textbf{正确推理}终究是其最坚实的基础。

\begin{center}
\fbox{\parbox{0.9\textwidth}{
  \centering
  \textbf{逻辑学的价值}\\
  运用逻辑学的方法与技术,人们可以有效地区分正确的推理与不正确的推理,\\
  这种方法与技术就是本书的主题。
}}
\end{center} 