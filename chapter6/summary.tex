\chaptersummary{
第6章考察\logicterm{标准式直言三段论}:组成成分、形式、\logicemph{有效性}和制约其正确使用的规则。

\subsection{三段论的基本元素}
6.1节给出了三段论\logicterm{大项}、\logicterm{小项}和\logicterm{中项}的定义:

\begin{itemize}
  \item \logicterm{大项}:结论的谓项
  \item \logicterm{小项}:结论的主项
  \item \logicterm{中项}:两个前提中都出现,但结论中不出现的第三个项
\end{itemize}

继而又分别定义了\logicterm{大前提}和\logicterm{小前提},包含大项的前提叫做大前提,包含小项的前提叫做小前提。如果几个命题出现的次序正好是:大前提在第一位、小前提在第二位、结论在最后,我们就把这样的三段论指定为\logicterm{标准式的}。

6.1节也说明了三段论的\logicterm{式}与\logicterm{格}是如何确定的。

三段论的\logicterm{式}由识别三个命题类型的字母来确定,即A、E、I、O中的三个。总共有64个不同式。

三段论的\logicterm{格}由\logicterm{中项}在前提中的不同位置来确定。对四个可能的格描述并定义如下:

\begin{itemize}
\item \logicterm{第一格}:中项在大前提中做主项、在小前提中做谓项。\\
模式为:$M-P, S-M$,所以$S-P$。

\item \logicterm{第二格}:中项在两个前提中都做谓项。\\
模式为:$P-M, S-M$,所以$S-P$。

\item \logicterm{第三格}:中项在两个前提中都做主项。\\
模式为:$M-P, M-S$,所以$S-P$。

\item \logicterm{第四格}:中项在大前提中做谓项、在小前提中做主项。\\
模式为:$P-M, M-S$,所以$S-P$。
\end{itemize}

6.2节说明了标准式三段论的\logicterm{式}与\logicterm{格}如何共同地确定其逻辑形式。由于64个式每一个都有四个格,所以共有256个标准式的直言三段论,但其中只有一小部分是\logicemph{有效式}。
}

\subsection{检验三段论有效性}
6.3节介绍检验三段论\logicemph{有效性}的\logicterm{文恩图}方法,即在几个交叉的圆中,作上恰当的标记或涂上阴影以表示前提的含义。

6.4节阐明标准式三段论的六条基本规则,同时定义了违反各条规则所造成的\logicwarn{谬误}。

\begin{itemize}
\item \logicterm{规则1}:一个\logicemph{有效的}标准式直言三段论必须仅仅包含三个项,在整个论证中,每一个项都须在相同的意义上使用。\\
违反本规则所犯的\logicwarn{错误}:\logicwarn{四项谬误}。

\item \logicterm{规则2}:在一个\logicemph{有效的}标准式直言三段论中,中项必须至少在一个前提中周延。\\
违反本规则所犯的\logicwarn{错误}:\logicwarn{中项不周延谬误}。

\item \logicterm{规则3}:在一个\logicemph{有效的}标准式直言三段论中,在结论中周延的项在前提中也必须周延。\\
违反本规则所犯的\logicwarn{错误}:\logicwarn{大项不当周延谬误},或者\logicwarn{小项不当周延谬误}。

\item \logicterm{规则4}:任何有两个否定前提的标准式三段论都不是\logicemph{有效的}。\\
违反本规则所犯的\logicwarn{错误}:\logicwarn{排斥前提谬误}。

\item \logicterm{规则5}:如果一个标准式三段论有一个前提是否定的,那么结论必须是否定的。\\
违反本规则所犯的\logicwarn{错误}:\logicwarn{从否定推肯定谬误}。

\item \logicterm{规则6}:一个\logicemph{有效的}标准式直言三段论,如果结论为特称命题,那么其前提不能都是全称的。\\
违反本规则所犯的\logicwarn{错误}:\logicwarn{存在谬误}。
\end{itemize}

\subsection{有效三段论形式}
6.5节给出了标准式直言三段论的15个\logicemph{有效形式}的说明,识别它们的格与式,并说明了它们传统的拉丁名称:

\begin{itemize}
\item \logicterm{第一格}:AAA-1(Barbara)、EAE-1(Celarent)、AII-1(Darii)、EIO-1(Ferio)
\item \logicterm{第二格}:AEE-2(Camestres)、EAE-2(Cesare)、AOO-2(Baroko)、EIO-2(Festino)
\item \logicterm{第三格}:AII-3(Datisi)、IAI-3(Disamis)、EIO-3(Ferison)、OAO-3(Bokardo)
\item \logicterm{第四格}:AEE-4(Camenes)、IAI-4(Dimaris)、EIO-4(Fresison)
\end{itemize}

6.6节展示了15个\logicemph{有效形式}的演绎推导,通过排除法程序,证明了只有15个形式是完全遵守三段论的六条基本规则的。

\begin{center}
\fbox{\parbox{0.95\textwidth}{
\textbf{第6章核心内容}
\begin{itemize}
\item \logicterm{三段论的组成元素}:大项、小项、中项、大前提、小前提
\item \logicterm{三段论形式}:由式(64种组合)和格(4种位置)确定,共256种可能形式
\item \logicterm{有效性检验方法}:
  \begin{itemize}
  \item \logicterm{文恩图解法}:直观图形表示法
  \item \logicterm{规则检验法}:应用六条基本规则
  \end{itemize}
\item \logicterm{六条基本规则}:避免四项、中项至少周延一次、结论中周延的项在前提中也周延等
\item \logicterm{15个有效形式}:从256种可能形式中筛选出符合六条规则的形式
\item \logicterm{演绎推导}:根据结论类型和规则限制,系统性地排除\logicwarn{无效}形式
\end{itemize}
}}
\end{center}