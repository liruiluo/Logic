\section*{第6章概要}
第6章考察标准式直言三段论:组成成分、形式、有效性和制约其正确使用的规则。

6. 1 节给出了三段论大项、小项和中项的定义:

\begin{itemize}
  \item 大项:结论的谓项
  \item 小项:结论的主项
  \item 中项:两个前提中都出现,但结论中不出现的第三个项
\end{itemize}

继而又分别定义了大前提和小前提,包含大项的前提叫做大前提,包含小项的前提叫做小前提。如果几个命题出现的次序正好是:大前提在第一位、小前提在第二位、结论在最后,我们就把这样的三段论指定为标准式的。\\
6.1 节也说明了三段论的式与格是如何确定的。

三段论的式由识别三个命题类型的字母来确定,即A、E、I、O中的三个。总共有 64 个不同式。

三段论的格由中项在前提中的不同位置来确定。对四个可能的格描述并定义如下:

第一格:中项在大前提中做主项、在小前提中做谓项。\\
模式为:$M-P, S-M$ ,所以 $S-P$ 。\\
第二格:中项在两个前提中都做谓项。\\
模式为:$P-M, S-M$ ,所以 $S-P$ 。\\
第三格:中项在两个前提中都做主项。

模式为:$M-P, M-S$ ,所以 $S-P$ 。\\
第四格:中项在大前提中做谓项、在小前提中做主项。\\
模式为:$P-M, M-S$ ,所以 $S-P$ 。\\
6.2 节说明了标准式三段论的式与格如何共同地确定其逻辑形式。由于 64 个式每一个都有四个格,所以共有 256 个标准式的直言三段论,但其中只有一小部分是有效式。

6. 3 节介绍检验三段论有效性的文恩图方法,即在几个交叉的圆中,作上恰当的标记或涂上阴影以表示前提的含义。\\
6.4 节阐明标准式三段论的六条基本规则,同时定义了违反各条规则所造成的谬误。\\
-规则 1 一个有效的标准式直言三段论必须仅仅包含三个项,在整个论证中,每一个项都须在相同的意义上使用。

违反本规则所犯的错误:四项谬误。\\
-规则 2 在一个有效的标准式直言三段论中,中项必须至少在一个前提中周延。

违反本规则所犯的错误:中项不周延谬误。\\
-规则 3 在一个有效的标准式直言三段论中,在结论中周延的项在前提中也必须周延。

违反本规则所犯的错误:大项不当周延谬误,或者小项不当周延谬误。\\
-规则 4 任何有两个否定前提的标准式三段论都不是有效的。\\
违反本规则所犯的错误:排斥前提谬误。\\
-规则 5 如果一个标准式三段论有一个前提是否定的,那么结论必须是否定的。

违反本规则所犯的错误:从否定推肯定谬误。\\
-规则 6 一个有效的标准式直言三段论,如果结论为特称命题,那么其前提不能都是全称的。

违反本规则所犯的错误:存在谬误。\\
6. 5 节给出了标准式直言三段论的 15 个有效形式的说明,识别它们的格与式,并说明了它们传统的拉丁名称:

AAA-1(Barbara)、EAE-1(Celarent)、AII-1(Darii)、EIO-1(Fe- rio)、AEE-2(Camestres)、EAE-2(Cesare)、AOO-2(Baroko)、EIO-2\\
(Festino)、AII-3(Datisi)、IAI-3(Disamis)、EIO-3(Ferison)、OAO-3 (Bokardo)、AEE-4(Camenes)、IAI-4(Dimaris)、EIO-4(Fresison)。

6. 6 节展示了 15 个有效形式的演绎推导,通过排除法程序,证明了只有 15 个形式是完全遵守三段论的六条基本规则的。 