\section{直言三段论15个有效形式的演绎推导}

\begin{quotation}
本节通过应用三段论的六条基本规则,系统地演绎推导出标准式直言三段论的15个有效形式。我们将根据结论的不同类型(A、E、I或O命题)分类讨论,分析哪些形式符合逻辑规则而保留下来,从而完成从256个可能形式到15个有效形式的演绎。
\end{quotation}

直言三段论的15个有效形式是从256个可能形式中排除无效式以后得以确立的。我们可以通过确定哪些形式违反了三段论的基本规则来实行这种排除——演绎推导出三段论的15个有效形式。

对逻辑初学者来说,不必一定弄清如何排除无效式的细节。但对于那些从三段论分析的复杂性中获取乐趣的人而言,这应是一种虽有难度但令人愉悦的挑战。如果只想认识和把握三段论的有效式,即6.5节讲到的那些内容,就可以绕过本节不看。

这种演绎推导并不那么容易理解。从事这项工作必须清晰地记住以下两点:(1)6.4节设定的六条三段论基本规则;(2)三段论四个格的模式,即图6-11。

根据结论的不同形式,我们首先把三段论的所有可能形式分为四组。每个结论都是A、E、I、O四种直言命题之一,没有其他可能,据此可以分四种情形考察一个有效的三段论需要具备什么特性,即可以这样提问:如果结论是A命题,通过某一条或几条规则能够排除什么形式;如果结论是E命题可以排除什么形式,以此类推。下面我们就逐个进行考察。

\subsection{情形1:如果三段论的结论是$\mathbf{A}$命题}
在这种情形下,前提不可能是E命题,也不可能是O命题,因为如果前提为否定命题的话,结论就应该是否定的(规则5)。所以,两个前提必定是A命题或I命题。小前提不能是I命题,因为小项(结论的主项,也就是一个A命题的主项)在结论中是周延的,如果小前提是I命题,那么在前提中不周延的项在结论中周延,违反了规则3。两个前提,即大前提和小前提,不能是I和A,因为如果是的话,有两种可能,或者是在结论中周延的项在前提中不周延,违反规则3,或者是中项两次不周延,违反规则2。所以两个前提(结论是A命题时)必须都是A命题,这意味着唯一有效的形式是AAA式。而第二格的AAA式会使中项两次不周延,第三格和第四格的AAA式都会造成前提中不周延的项在结论中周延的错误。所以,如果三段论的结论是A命题,唯一的有效形式就是第一格的AAA式,即AAA-1,传统上称这个有效形式为Barbara。

\textbf{情形1的总结}:如果三段论的结论是$\mathbf{A}$命题,只能有一种有效形式:AAA-1-Barbara。

\subsection{情形2:如果三段论的结论是$\mathbf{E}$命题}
E命题的主项和谓项都是周延的,因此,如果结论为E命题,三段论前提中的三个项也都必须至少周延一次$^{(1)}$,这只有当前提之一也是E命题时才有可能。但不能两个前提都是E命题,因为不能允许两个否定前提(规则4),同理可知另一个前提也不能是O命题。另一个前提也不能是I命题,否则在结论中周延的项在前提中不周延,违反规则3。这样,另一个前提必须是A命题,两个前提的组合可能是AE或EA。因此,在结论是E命题的情况下,可能的正确形式为AEE和EAE。

如果是AEE式,它不能是第一格,也不能是第三格。因为如果是这两个格的话,结论中周延的项在前提中不周延。所以,有效的AEE式只能是第二格的,即AEE-2(传统上称为Camestres),或者是第四格的,即AEE-4(传统上称为Camenes)。如果是EAE式,它不能是第三格,也不能是第四格,因为那也都导致结论中周延的项在前提中不周延。所以,有效的EAE式只能或者是第一格的,即EAE-1(传统上称为Celarent),或者是第二格的,即EAE-2(传统上称为Cesare)。

\textbf{情形2的总结}:如果三段论的结论是$\mathbf{E}$命题,只能有四种有效形式:AEE-2、AEE-4、EAE-1和EAE-2——分别是Camestres、Camenes、Celarent和Cesare。

\subsection{情形3:如果三段论的结论是I命题}
在这种情形下,前提不能是E或O命题,因为如果有一个否定前提的话,结论也应该是否定的。两个前提也不能都是A命题,因为结论为特称的三段论其前提不能都是全称的(规则6)。同样,两个前提也不能都是I命题,因为中项必须至少在一个前提中周延(规则2)。这样,前提的组合必须是AI或者IA,因而结论为I命题的三段论可能的有效形式为AII和IAI。

AII在第二格和第四格中不可能有效,因为中项至少要周延一次。因此保留下来的AII式就是AII-1(传统上称为Darii)和AII-3(传统上称为Datisi)。如果是IAI式,它不能是IAI-1和IAI-2,因为这两个形式都违反中项至少在一个前提中周延的规则。剩下的有效形式就是IAI-3(传统上称为Disamis)和IAI-4(传统上称为Dimaris)。

\textbf{情形3的总结}:如果三段论的结论是I命题,只能有四种有效形式:AII-1、AII-3、IAI-3和IAI-4——分别是Darii、Datisi、Disamis和Dimaris。

\subsection{情形4:如果结论是$\mathbf{O}$命题}
在这种情形下,大前提不能是I命题,因为结论中周延的项在前提中也必须周延。所以大前提可能是A命题、E命题或者O命题。

假设大前提是A命题。这样,小前提就不能是A命题和E命题,因为结论为特称(O命题)时,前提不能都是全称的。小前提也不能是I命题,否则,或者中项一次也不周延(违反规则2),或者结论中周延的项在前提中不周延。因此,如果大前提是A命题,小前提必须是O命题,结果就是AOO式。但在第四格,AOO式不可能有效,因为中项两次不周延。在第一格和第三格也不可能有效,因为结论中周延的项在前提中不周延。因此当大前提是A命题时,AOO式保留下来的有效形式只有第二格AOO-2(传统上称为Baroko)。

再假设(如果结论是O命题)大前提是E命题。在这种情况下,小前提将不能是E命题或O命题,因为不允许两个否定前提。小前提也不能是A命题,因为结论如果为特称的,前提就不能是两个全称命题(规则6)。因而只剩下了EIO式——它在四种格中都是有效的,传统上分别叫做Ferio(EIO-1)、Festino(EIO-2)、Ferison(EIO-3)和Fresison(EIO-4)。

最后,假设大前提是O命题。同样小前提也不能是E命题或O命题,因为不能允许两个否定前提。小前提也不能是I命题,因为那样的话,或者中项一次都不周延,或者结论中周延的项在前提中不周延。因此,如果大前提是O命题,小前提必须是A命题,即必为OAO式。但要排除OAO-1,因为中项两次都不周延。也要排除OAO-2和OAO-4,因为这两种情况都会使结论中周延的项在前提中不周延。于是就只剩下一个有效形式OAO-3(传统上称为Bokardo)。

\textbf{情形4的总结}:如果结论是O命题,则有六个有效形式:AOO-2、EIO-1、EIO-2、EIO-3、EIO-4和OAO-3,分别叫做Baroko、Ferio、Festino、Ferison、Fresison和Bokardo。

以上的分析通过排除法证明了直言三段论恰有15个有效形式:结论是A命题时有1个,结论是E命题时有4个,结论是I命题时有4个,而结论为O命题时有6个。这15个有效形式中,四个是第一格的,四个是第二格的,四个是第三格的,三个是第四格的。这样,就完成了标准式直言三段论的15个有效形式的演绎推导。

\footnotetext{(1)据规则$2、3$。}

\begin{center}
\fbox{\parbox{0.95\textwidth}{
\textbf{本节要点}
\begin{itemize}
\item 通过应用六条基本规则,可以演绎推导出15个有效的三段论形式
\item 根据结论的类型可分为四种情形:
  \begin{itemize}
  \item \textbf{结论是A命题}:只有1个有效形式(AAA-1 Barbara)
  \item \textbf{结论是E命题}:有4个有效形式(AEE-2, AEE-4, EAE-1, EAE-2)
  \item \textbf{结论是I命题}:有4个有效形式(AII-1, AII-3, IAI-3, IAI-4)
  \item \textbf{结论是O命题}:有6个有效形式(AOO-2, EIO-1, EIO-2, EIO-3, EIO-4, OAO-3)
  \end{itemize}
\item 这15个有效形式在四个格中的分布:
  \begin{itemize}
  \item 第一格:4个(AAA-1, EAE-1, AII-1, EIO-1)
  \item 第二格:4个(AEE-2, EAE-2, AOO-2, EIO-2)
  \item 第三格:4个(AII-3, IAI-3, EIO-3, OAO-3)
  \item 第四格:3个(AEE-4, IAI-4, EIO-4)
  \end{itemize}
\end{itemize}
}}
\end{center} 