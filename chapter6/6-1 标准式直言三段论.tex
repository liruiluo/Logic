\section{标准式直言三段论}

\begin{logicbox}[title=引言]
\logicterm{三段论}是演绎推理的经典形式,由三个直言命题组成,包含三个词项,每个词项恰好出现两次。本节将介绍标准式直言三段论的基本要素,包括词项、式和格的概念,为后续讨论三段论的\logicemph{有效性}奠定基础。
\end{logicbox}

\logicterm{三段论}是从两个前提推得一个结论的演绎论证。\logicterm{直言三段论}是由三个直言命题组成的演绎论证,其中包含且仅包含三个词项,每个词项在其构成命题中恰好出现两次。

如果一个直言三段论的前提、结论都是标准式的直言命题(A、E、I、O),并且以特定的标准顺序组合在一起,就称为\logicterm{标准式直言三段论}。要确定标准式直言三段论的顺序,必须首先说明直言三段论的词项和前提的特定名称,为简便起见,本章将直言三段论简称为"\logicterm{三段论}"。其他三段论将会在后面的章节进行讨论。

\subsection{大项、小项和中项}
\begin{theorembox}[title=三段论词项的定义]
标准式三段论的结论是一个标准式直言命题,三段论的三个词项有两个会在其中出现。因此,通过结论可以识别三段论的词项。

结论的谓项称为三段论的\logicterm{大项}。

结论的主项称为三段论的\logicterm{小项}。

在结论中不出现,而在前提中出现两次的项,即三段论的第三个项,称为\logicterm{中项}。
\end{theorembox}

\begin{examplebox}[title=三段论词项识别示例]
例如下面这个标准式三段论中:

没有英雄是胆小鬼,

有士兵是胆小鬼,

所以,有士兵不是英雄。

"士兵"是\logicterm{小项},"英雄"是\logicterm{大项},结论中没有出现的"胆小鬼"是\logicterm{中项}。
\end{examplebox}

标准式三段论的前提因其中出现的项而得名。大项和小项必定出现在不同的前提中,包含大项的前提称为\logicterm{大前提},包含小项的前提称为\logicterm{小前提}。在上述三段论中,大前提是"没有英雄是胆小鬼",小前提是"有士兵是胆小鬼"。

如前所述,如果前提以特定的标准顺序排列,就称这个三段论为\logicterm{标准式}。现在即可描述这个顺序:在标准式三段论中,\logicterm{大前提}处在第一位,\logicterm{小前提}处在第二位,结论在最后。应当强调的是,大前提不是根据其位置而确定的,而是因为其中包含\logicterm{大项},大项又是由结论的谓项定义的。同样的,小前提也不是根据其位置而确定的,而是因为其中包含\logicterm{小项},小项又是由结论的主项定义的。

\subsection{式}
\begin{theorembox}[title=三段论的式]
标准式三段论的\logicterm{式}由所含直言命题的类型而定(以字母 A、E、I、O为标志)。每个三段论的式都由三个按特定顺序排列的字母组成。第一个字母指的是大前提的类型,第二个字母指的是小前提的类型,第三个字母指的是结论的类型。
\end{theorembox}

例如,在上述作为例子的三段论中,大前提是一个 E命题,小前提是一个 I 命题,结论是一个 O 命题,所以,这个三段论的式就是 \logicterm{EIO 式}。

\subsection{格}
只有\logicterm{式},还不能完全刻画标准三段论的形式。考虑下面两个三段论:

\begin{examplebox}[title=相同式但不同格的三段论]
(A)所有大科学家都是大学毕业生,

有专业运动员是大学毕业生,

所以,有专业运动员是大科学家。

和

(B)所有艺术家都是自我主义者,

有艺术家是乞丐,

所以,有乞丐是自我主义者。

两个三段论的式都是 AII,但它们的形式并不相同。
\end{examplebox}

如果我们展示出它们的逻辑"骨架",就能十分清楚地揭示出其形式上的不同之处。把小项记为 $S$ ,大项记为 $P$ ,中项记为 $M$ ,并用"$\therefore$"表示"所以",这两个三段论的形式或"骨架"分别是:

(A)所有 $P$ 是 $M$ ,

有 $S$ 是 $M$ ,

$\therefore$ 有 $S$ 是 $P$ 。

(B)所有 $M$ 是 $P$ ,

有 $M$ 是 $S$ ,

$\therefore$ 有 $S$ 是 $P$ 。

在记为(A)的第一个三段论中,\logicterm{中项}在两个前提中都做谓项,而记为(B)的第二个三段论,\logicterm{中项}在两个前提中都做主项。这两个例子表明,尽管三段论的形式可以部分地由\logicterm{式}来描述,但相同式的三段论还是有区别的,这就要看中项的相对位置。\cite{lukasiewicz1957} 然而,我们可以通过陈述一个三段论的格和式来完整地描述其形式,它的\logicterm{格}表明了中项在前提中的位置。

\begin{theorembox}[title=三段论的四格]
显然,三段论有且只有四种不同的\logicterm{格}。\logicterm{中项}可能在大前提中做主项、在小前提中做谓项,或者在两个前提中都做谓项,或者在两个前提中都做主项,或者在大前提中做谓项、在小前提中做主项。中项的这些可能组合分别构成了三段论的第一、第二、第三和第四格。
\end{theorembox}

四个格的模式可依次排列如下,其中只显示了中项的相对位置,而隐藏了它们的式,也就是说既不出现量项也不出现联项:

$$
\begin{array}{llll}
M-P & P-M & M-P & P-M \\
\frac{S-M}{\therefore S-P} & \frac{S-M}{\therefore S-P} & \frac{M-S}{\therefore S-P} & \frac{M-S}{\therefore S-P} \\
\begin{array}{l}
\text { 第一格 }
\end{array} & \begin{array}{l}
\text { 第二格 } \\
\end{array} & \begin{array}{l}
\text { 第三格 }
\end{array} & \begin{array}{l}
\text { 第四格 }
\end{array}
\end{array}
$$

要完整地描述一个标准三段论的形式,只要指明其\logicterm{式}和\logicterm{格}即可。例如,任何一个第二格 AOO 式(简记为 AOO-2)的三段论都有如下形式:

所有 $P$ 是 $M$ ,

有 $S$ 不是 $M$ ,

所以,有 $S$ 不是 $P$ 。

从无限多样的不同题材中把形式抽象出来,我们会得到许多不同的标准三段论的形式。假如把它们排列一下,从AAA、AAE、AAI、AAO、AEA、AEE、AEI、AEO、AIA $\cdots \cdots$ 以此类椎,直到 OOO 式,共可列举出 64 个不同的式。由于每个式都可以与四个不同的格进行组合,于是,标准式的三段论就必然呈现出 \logicemph{256 个不同的形式}。但正如我们将要看到的,其中只有少数形式是\logicemph{有效的}。

