\section{三段论规则和三段论谬误}

\begin{quotation}
为了系统判断三段论的有效性,逻辑学发展出一系列规则。本节介绍六条重要的三段论规则,并详细说明违反这些规则会导致的各种形式谬误。掌握这些规则与谬误,不仅有助于检验三段论的有效性,还能帮助我们在实际推理中避免常见的逻辑错误。
\end{quotation}

在许多情况下,一个三段论并不能真正推得其结论。为帮助人们避免常见的错误,人们制定了一系列规则(本书列出六条)用来范导论证:对于任何给定的标准式三段论,通过考察其中是否有违反规则的情况,就能对它进行评判。

违反任何一条规则都会导致错误。这是一种特殊种类的论证错误,所以我们称之为\textbf{三段论谬误};又因为这种错误是论证形式方面的,所以称之为\textbf{形式谬误}(与第4章所讲非形式谬误相对照)。在三段论论证中,必须谨防违反规则,避免产生谬误。每一种形式谬误都有一个传统名称,以下详加介绍。

\subsection{规则1 避免四项}
一个有效的标准式直言三段论必须仅仅包含三个项,在整个论证中,每一个项都须在相同的意义上使用。

在直言三段论中,结论断定了两个项即主项(小项)与谓项(大项)之间的关系。因此,只有前提断定的是这两个项分别与同一个第三项(中项)的联系时,结论才能是合理的。如果前提不能做到这一点,就不能在结论的两个项之间建立联系,论证就不能进行。所以,每个有效的直言三段论必须只有三个项——不能多也不能少。如果包含了多于三个的项,三段论就是无效的。这种谬误叫做\textbf{四项谬误}。

这种谬误通常源于语词歧义,即用同一个词或短语表达两种不同的含义。最常见的是中项的含义发生转换,同一个词以某种用法与小项发生联系,而以另一种用法与大项发生联系。这样一来,与结论中的两个项发生联系的是两个不同的项(而不是同一个中项),所以结论断定的关系也就不能成立。$^{[4]}$

本章开始定义"直言三段论"时,就指出每一个三段论一定有且只有三个项。$^{[5]}$ 所以,可以把这个规则("避免四项")看做是一个论证成为一个真正的三段论的保证。

\subsection{规则2 中项至少在一个前提中周延}
如果(如5.3节所说明)命题述及一个项所指称的全部对象,该项在命题中就是"周延"的。如果中项在两个前提中都不周延,推出结论所需要的词项关联就不能建立。

历史学家芭芭拉•塔克曼(Barbara Tuchman)认为,许多无政府主义的早期批判家是以下面这样一个"无意识的三段论"为依据进行论证的。

\begin{quote}
所有俄国人是革命者,\\
所有无政府主义者是革命者,\\
所以,所有无政府主义者是俄国人。$^{[6]}$
\end{quote}

这个三段论显然是无效的。错误在于它根据无政府主义者、俄国人两个类分别与革命者的类之间的联系,断定了前两个类的关系——但革命者这个项在两个前提中都是不周延的。第一个前提没有述及全部革命者,第二个前提同样没有。"革命者"在论证中做中项,如果它在三段论两个前提中都不周延,那么三段论就不可能是有效的。这样的谬误叫做\textbf{中项不周延谬误}。

这个规则的依据是小项和大项之间的联系需要中项做中介。而要建立这种联系,结论的主项或者谓项就必须与中项所指称类的全部对象相关联。否则,结论中的两个项就有可能分别与中项的不同部分发生联系,因而不必然与另一个项相关联。

这恰好是上面给出的三段论所存在的问题。俄国人只包含在革命者类的一部分当中(据第一个前提),无政府主义者也只是包含在革命者类的一部分之中(据第二个前提)——这两部分却是与另一个类(三段论的中项)的不同部分发生联系的,所以,中项就不能成功地联结小项和大项。一个有效的三段论,其中项必定至少在一个前提中周延。

\subsection{规则3 在结论中周延的项在前提中也必须周延}
述及一个类的全部对象,比述及其中某些对象要断定更多。所以,如果三段论前提中不周延的项在结论中周延,也就是结论断定了比前提更多东西。但是,有效的论证要求其前提必须能逻辑地推出结论,结论绝不能比前提断定得更多。可以说,在结论中周延而在前提中不周延的项确实是个信号,说明结论超出了前提,跑得太远了。这种谬误叫做\textbf{不当周延}。

结论可能是小项(主项)超出了前提,或者大项(谓项)超出了前提。所以,不当周延有两种不同形式,我们分别给它们一个名字:

\textbf{大项不当周延}("非法大项"),\\
\textbf{小项不当周延}("非法小项")。

举个例子来说明第一种,看下面这个三段论:

\begin{quote}
所有的狗是动物,\\
没有猫是狗,\\
所以,没有猫是动物。
\end{quote}

很明显,这个论证是不对的,但错在哪里呢?就错在结论是对所有动物的断言,即结论断定的是所有动物都在猫的类之外,而前提并没有对所有动物做出断言——故结论不当地超出了前提的断定。由于"动物"在三段论中做大项,所以此处的谬误就是非法大项。

再举个例子来说明第二种,看下面这个三段论:

\begin{quote}
所有传统教徒都是原教旨主义者(fundamentalist),\\
所有传统教徒都是宽容堕胎行为的,\\
所以,所有宽容堕胎行为的都是原教旨主义者。
\end{quote}

我们立刻会感觉到这个论证也有问题,其错误就在于:结论断定了所有堕胎行为的宽容者,而在前提中并没有这样的断言,没有述及所有宽容堕胎行为者的情况。这样,结论就不能为前提所担保。这个例子中"宽容堕胎行为的"是小项,所以此处的谬误就是非法小项。

\subsection{规则4 避免出现两个否定前提}
任何否定命题($\mathbf{E}$ 或 $\mathbf{O}$)都否认类的包含关系,断定一个类的部分或者全部被排除在另一类的全体之外。但是,由两个断定这种排斥性的前提不能得出结论中的联系,因此,不可能是有效的论证。这种错误叫做\textbf{排斥前提谬误}。

理解这个谬误需要进一步思考。考虑三段论的小项 $S$、大项 $P$ 和中项 $M$,对于这三个项之间的联系,两个否定前提能告诉我们什么呢?它们说明 $S$(结论的主项)完全或部分地排斥 $M$(中项)的一部分或者全部,并且 $P$(结论的谓项)完全或部分地排斥 $M$ 的一部分或者全部。但是,不管 $S$ 和 $P$ 的关系如何,这些关系中的任何一个都可能成立。这样的否定前提不能告诉我们 $S$ 和 $P$ 之间究竟是包含还是排斥,究竟是全部地包含或排斥,还是部分地包含或排斥。因此,如果三段论的两个前提都是否定的,论证肯定是无效的。

\subsection{规则5 如果有一个前提是否定的,那么结论必须是否定的}
如果结论是肯定的,也就是说,如果它断言两个类中的一个($S$ 或 $P$)完全或部分地包含在另一个之中,那么,前提必须断定这样的第三个类存在才能推出结论,即第三个类必须包含第一个并且被第二个包含,而类之间的这种包含关系只能由肯定命题表示。所以,肯定的结论只能由两个肯定的前提得到。违反这条规则的错误叫做\textbf{从否定推肯定谬误}。

要想得出肯定结论必须要有两个肯定前提,如上所述,我们可以确定地说,只要两个前提中有一个是否定的,结论就必须也是否定的,否则论证无效。

与其他谬误不同,这个谬误并不常见,因为对于任何从否定前提得肯定结论的论证,很容易就可以看出是极不合理的。举一个例子就能说明:

\begin{quote}
没有诗人是会计,\\
有艺术家是诗人,\\
所以,有艺术家是会计。
\end{quote}

立即可以看到,由第一个前提对诗人和会计的排斥关系的断言,已使得该论证不可能为艺术家和会计之间的包含关系提供任何有效辩护。

\subsection{规则6 两个全称前提得不出特称结论}
在直言三段论的布尔解释中(见5.6节),全称命题(A和E)没有存在含义,但特称命题($\mathbf{I}$ 和 $\mathbf{O}$)却有存在含义。只要像本书这样设定了布尔解释,就要避免从没有存在含义的前提得出有存在含义的结论。

最后这个规则在传统逻辑或者亚里士多德逻辑对直言三段论的解释中并不需要,因为它们并不关心存在含义问题。但是,仔细考虑一下预设问题就会很清楚,如果一个论证的前提根本没有断定什么东西存在,但是从这些前提却推出了有些东西的存在,那么结论就是不合理的。这种错误叫做\textbf{存在谬误}。

下面这个例子就犯有这种谬误:

\begin{quote}
所有宠物都是家养动物,\\
没有独角兽是家养动物,\\
所以,有独角兽不是宠物。
\end{quote}

假如这个论证的结论是全称的"没有独角兽是宠物",它是完全有效的。在传统解释下,由于全称命题与特称命题一样都有存在含义,例子中的结论只是上述有效论证结论的"下位"。

但从布尔解释的角度说,上例的结论("有独角兽不是宠物")不仅仅是个"下位",因为特称命题与全称命题有很大不同。结论是特称的 $\mathbf{O}$ 命题,有存在含义,而 $\mathbf{E}$ 命题("所有独角兽不是宠物")是没有存在含义的。传统观点下接受的推论在布尔解释下不再被接受,因为在后者看来这样的论证犯了存在谬误——种在传统解释下不会出现的错误。$^{[7]}$

以上给出的六条规则只适用于标准式直言三段论。它们提供了足够的工具,用以检验这一领域内任何论证的有效性。对于任一标准式直言三段论,如果违反了任一规则就是无效的,如果遵循了所有的规则就一定是有效的。

\footnotetext{(4)关于词项歧义的一个经典例子是论证"有机物是化学物质,化学物质是无机物,所以,有机物是无机物"。从表面看,它的式是AAA-1,应该是个有效论证。但是,中项"化学物质"的含义在两个前提中有所不同,所以,这个三段论实际上包含了四个项,犯有四项谬误,是无效的。}
\footnotetext{(5)请注意,三段论可能包含多于三个词项。它只需包含三种不同的项即可,但每种项可能重复出现。实际上,一个标准式直言三段论包含四个词项,但其中有一个重复出现两次。}
\footnotetext{(6)见Barbara Tuchman,《历史的棱镜》(The Proud Tower)(New York:MacMillan,1966),130页。}
\footnotetext{(7)关于存在含义问题,可以参考5.6节。}

\begin{center}
\fbox{\parbox{0.95\textwidth}{
\textbf{本节要点}
\begin{itemize}
\item \textbf{三段论六大规则}:
  \begin{itemize}
  \item \textbf{规则1 避免四项}:有效三段论必须只有三个项,且每项意义不变
  \item \textbf{规则2 中项至少在一个前提中周延}:否则无法建立结论项之间的联系
  \item \textbf{规则3 结论中周延的项在前提中也必须周延}:结论不能断定比前提更多内容
  \item \textbf{规则4 避免出现两个否定前提}:仅依靠排斥关系无法建立结论项之间的联系
  \item \textbf{规则5 若有一个前提是否定的,结论必须是否定的}:肯定结论需要两个肯定前提
  \item \textbf{规则6 两个全称前提得不出特称结论}:无存在含义的前提不能推出有存在含义的结论
  \end{itemize}
\item \textbf{三段论谬误}:
  \begin{itemize}
  \item \textbf{四项谬误}:通常因词项歧义而包含超过三个项
  \item \textbf{中项不周延谬误}:中项在两个前提中都不周延
  \item \textbf{不当周延}:分为大项不当周延和小项不当周延
  \item \textbf{排斥前提谬误}:两个否定前提不能确立结论项之间的联系
  \item \textbf{从否定推肯定谬误}:从包含否定前提的论证得出肯定结论
  \item \textbf{存在谬误}:从不含存在含义的前提得出有存在含义的结论
  \end{itemize}
\end{itemize}
}}
\end{center} 