\section{三段论论证的形式性质}

\begin{logicbox}[title=引言]
三段论的\logicemph{有效性}完全取决于其逻辑形式,而非内容。本节将阐明三段论的形式性质,解释为什么相同形式的三段论具有相同的\logicemph{有效性}或\logicwarn{无效性},并介绍如何使用\logicterm{逻辑类推法}来检验和论证三段论的有效性。
\end{logicbox}

\begin{theorembox}[title=三段论的形式性质]
三段论的形式由\logicterm{格}与\logicterm{式}唯一确定——从逻辑的观点讲,这种形式是三段论最重要的方面。三段论的\logicemph{有效性}与\logicwarn{无效性}(其构成命题都是偶真的)仅仅依赖于形式,而完全独立于具体内容和题材。
\end{theorembox}

例如,任何形式为 AAA-1 的三段论:

\begin{examplebox}[title=AAA-1形式的三段论]
所有 $M$ 是 $P$ ,

所有 $S$ 是 $M$ ,

所以,所有 $S$ 是 $P$ 。

无论其题材是什么,它都是\logicemph{有效的}论证。这就是说,无论用什么词项代替这种形式或结构中的字母 $S$、$P$ 和 $M$,得到的论证总是\logicemph{有效的}。
\end{examplebox}

例如用这几个字母分别代表"雅典人"、"人"和"希腊人",代入后就得到这样一个\logicemph{有效}论证:

所有希腊人是人,

所有雅典人是希腊人,

所以,所有雅典人是人。

同样,下面这个论证也是\logicemph{有效的}:

所有钠盐是水溶性物质,

所有肥皂是钠盐,

所以,所有肥皂是水溶性物质。

说一个\logicemph{有效的}三段论是\logicemph{有效的}论证,是仅就其形式而言的。这说明如果某个三段论是\logicemph{有效的},那么,任何与它形式相同的三段论也是\logicemph{有效的}。如果一个三段论是\logicwarn{无效的},那么,任何与它形式相同的三段论也是\logicwarn{无效的}。$^{[2]}$ 这是人们在实际论辩中经常使用\logicterm{逻辑类推法}而获得的共识。

\begin{examplebox}[title=逻辑类推法的应用]
假如有人提出下面这个论证:

所有自由主义者都是国家健康保险的支持者,

有行政人员是国家健康保险的支持者,

所以,有行政人员是自由主义者。

我们会感觉到,无论其构成命题的真假,这个论证是\logicwarn{无效的}。揭示这种三段论荒谬性的最好方式,是构造一个形式相同但其\logicwarn{无效性}可直接显示出来的论证。比如,我们可以这样去问,你是否也可以说:

所有兔子都是跑得很快的,

有马是跑得很快的,

所以,有马是兔子。

我们可以补充说明:你不可能为后面这个论证作辩护,因为毫无疑问,其前提明显为\logicemph{真}但结论明显为\logicwarn{假}。你刚才的论证与这个马兔论证的形式完全相同。马兔论证是\logicwarn{无效的},所以你刚才的论证也是\logicwarn{无效的}。
\end{examplebox}

\logicterm{逻辑类推}是一种很好的论辩方法,是用于争辩的有力武器之一。

这种\logicterm{逻辑类推法}的根据是:直言三段论的\logicemph{有效性}或\logicwarn{无效性}是纯形式问题。要证明任何荒谬论证\logicwarn{无效},都可以找另一个论证,使之与一个明显\logicwarn{无效的}即其前提明显为\logicemph{真}而结论明显为\logicwarn{假}的论证有着相同形式。(不过应当牢记,\logicwarn{无效}论证也可能得到为\logicemph{真}的结论——说推理是\logicwarn{无效的},只是意味着结论与前提之间不构成逻辑蕴涵关系,或者说它们之间的关系不是必然联系。)

\logicwarn{但是,这种检验论证有效性的方法有很大的局限性。}有时很难一下子"想出"恰当的逻辑类推。并且,三段论论证有太多\logicwarn{无效的}形式(200多个)。此外,尽管我们只要想到一个前提为\logicemph{真}而结论为\logicwarn{假}的逻辑类推,就可以证明原论证的形式\logicwarn{无效},但是,若我们不能想到这样的逻辑类推,并不就能证明该形式\logicemph{有效},因为这可能只是由于我们的思维局限性所使然。很可能实际上存在着\logicwarn{无效性}类推,只是我们没有想到而已。这就需要一种更有效力的方法,来判定形式\logicemph{有效}或\logicwarn{无效}的三段论。本章以下各节就是要介绍检验三段论的一些最有力的方法。

\footnotetext{(2)严格来说,这一论断只适用于被解释为不含存在预设的标准式三段论。对于某些其他三段论形式来说,尽管逻辑类比未必总是有效的,但如果用所依靠的前提所含的具体概念来代替这个形式中出现的词项,那么所得到的论证仍然是有效的——即如果前提为真,结论也必为真。}

\chaptersummary{
三段论的\logicemph{有效性}或\logicwarn{无效性}仅取决于其逻辑形式(\logicterm{格}与\logicterm{式}的组合),相同形式的三段论具有相同的有效性特征。\logicterm{逻辑类推法}通过构造相同形式但前提明显\logicemph{真}而结论明显\logicwarn{假}的论证,证明某个形式的三段论\logicwarn{无效}。但逻辑类推法有局限性:难以找到恰当的类推例子,无法有效证明三段论形式的\logicemph{有效性},且三段论\logicwarn{无效}形式太多(200多个)。因此需要更系统的方法来判定三段论的有效性。
}