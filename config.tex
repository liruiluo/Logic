% 配置文件:包含所有宏包、字体、页面、标题、页眉页脚等设置
% 仅需在 main.tex 用 % 配置文件:包含所有宏包、字体、页面、标题、页眉页脚等设置
% 仅需在 main.tex 用 % 配置文件:包含所有宏包、字体、页面、标题、页眉页脚等设置
% 仅需在 main.tex 用 % 配置文件:包含所有宏包、字体、页面、标题、页眉页脚等设置
% 仅需在 main.tex 用 \input{config.tex} 引用

% 宏包加载

% This LaTeX document needs to be compiled with XeLaTeX.
\documentclass[a4paper,11pt,twoside]{book}
\usepackage[utf8]{inputenc}
\usepackage{amsmath}
\usepackage{amsfonts}
\usepackage{amssymb}
\usepackage[version=4]{mhchem}
\usepackage{extpfeil}
\usepackage{stmaryrd}
\usepackage{graphicx}
\usepackage[export]{adjustbox}
\graphicspath{ {./images/} }
\usepackage{multirow}
\usepackage{fvextra, csquotes}
\usepackage{xeCJK}
\usepackage{polyglossia}
\usepackage{fontspec}
\usepackage{unicode-math}
\usepackage{geometry}
\usepackage{fancyhdr}
\usepackage{titlesec}
\usepackage{indentfirst}
\usepackage{setspace}
\usepackage{xcolor}

% 添加 biblatex 支持
\usepackage[style=gb7714-2015,backend=biber,language=chinese]{biblatex}
\DeclareLanguageMapping{chinese}{chinese}
\addbibresource{references.bib}

% 添加hyperref包以启用PDF中的可跳转链接
\usepackage[hidelinks,bookmarks=true,bookmarksnumbered=true,colorlinks=true,linkcolor=blue,citecolor=blue]{hyperref}

% 设置页面边距
\geometry{top=2.5cm, bottom=2.5cm, left=2.5cm, right=2.5cm}

% 设置中文字体
\setCJKmainfont{Source Han Sans SC}[
  BoldFont = Source Han Sans SC Bold,
  ItalicFont = Source Han Sans SC,
  BoldItalicFont = Source Han Sans SC Bold
]
\setCJKmonofont{Source Han Sans SC}
\newCJKfontfamily\cnfont{Source Han Sans SC}

% 设置英文字体
\setmainfont{Times New Roman}

% 确保特定中文字符始终使用中文字体
\DeclareTextFontCommand{\textcn}{\cnfont}
\DeclareTextFontCommand{\texten}{\normalfont}

% 设置数学字体
\setmathfont{TeX Gyre Termes Math}

% 简化的标点和符号设置
\punctstyle{kaiming}
\xeCJKsetup{CJKecglue={\hskip 0.15em plus 0.05em minus 0.05em}}

% 确保中文文本和标点符号使用中文字体的命令
\newcommand{\zhtext}[1]{{\cnfont #1}}

% 添加中英文混排支持
\XeTeXlinebreaklocale "zh"
\XeTeXlinebreakskip = 0pt plus 1pt

% 确保特定中文字符总是使用中文字体
\newcommand{\cn}[1]{{\cnfont #1}}

% 设置页眉页脚样式
\pagestyle{fancy}
\fancyhf{}
\fancyhead[LE,RO]{\thepage}
\fancyhead[RE]{\leftmark}
\fancyhead[LO]{\rightmark}
\renewcommand{\headrulewidth}{0.4pt}
\renewcommand{\footrulewidth}{0pt}
\setlength{\headheight}{14.5pt}

% 设置章节标题格式
\titleformat{\chapter}[display]
{\normalfont\bfseries\Huge}
{\chaptertitlename\ \thechapter}{20pt}{\Huge}
\titlespacing*{\chapter}{0pt}{50pt}{40pt}

% 设置节标题格式
\titleformat{\section}
{\normalfont\Large\bfseries}
{\thesection}{1em}{}
\titlespacing*{\section}{0pt}{3.5ex plus 1ex minus .2ex}{2.3ex plus .2ex}

% 设置段落间距和首行缩进
\setlength{\parindent}{2em}
\setlength{\parskip}{1ex}
\onehalfspacing

%New command to display footnote whose markers will always be hidden
\let\svthefootnote\thefootnote
\newcommand\blfootnotetext[1]{%
  \let\thefootnote\relax\footnote{#1}%
  \addtocounter{footnote}{-1}%
  \let\thefootnote\svthefootnote%
}

%Overriding the \footnotetext command to hide the marker if its value is `0`
\let\svfootnotetext\footnotetext
\renewcommand\footnotetext[2][?]{%
  \if\relax#1\relax%
    \ifnum\value{footnote}=0\blfootnotetext{#2}\else\svfootnotetext{#2}\fi%
  \else%
    \if?#1\ifnum\value{footnote}=0\blfootnotetext{#2}\else\svfootnotetext{#2}\fi%
    \else\svfootnotetext[#1]{#2}\fi%
  \fi
}

% 设置目录格式
\renewcommand{\contentsname}{\centerline{\Large\bfseries 目录}}
% \renewcommand{\cftbeforetoctitleskip}{0pt}
% \renewcommand{\cftaftertoctitleskip}{2em}
% \renewcommand{\cftchapdotsep}{\cftdotsep}
% \renewcommand{\cftchapfont}{\bfseries}
% \renewcommand{\cftsecfont}{\normalfont}
% \renewcommand{\cftsubsecfont}{\normalfont\itshape}

% 其他自定义命令  引用

% 宏包加载

% This LaTeX document needs to be compiled with XeLaTeX.
\documentclass[a4paper,11pt,twoside]{book}
\usepackage[utf8]{inputenc}
\usepackage{amsmath}
\usepackage{amsfonts}
\usepackage{amssymb}
\usepackage[version=4]{mhchem}
\usepackage{extpfeil}
\usepackage{stmaryrd}
\usepackage{graphicx}
\usepackage[export]{adjustbox}
\graphicspath{ {./images/} }
\usepackage{multirow}
\usepackage{fvextra, csquotes}
\usepackage{xeCJK}
\usepackage{polyglossia}
\usepackage{fontspec}
\usepackage{unicode-math}
\usepackage{geometry}
\usepackage{fancyhdr}
\usepackage{titlesec}
\usepackage{indentfirst}
\usepackage{setspace}
\usepackage{xcolor}

% 添加 biblatex 支持
\usepackage[style=gb7714-2015,backend=biber,language=chinese]{biblatex}
\DeclareLanguageMapping{chinese}{chinese}
\addbibresource{references.bib}

% 添加hyperref包以启用PDF中的可跳转链接
\usepackage[hidelinks,bookmarks=true,bookmarksnumbered=true,colorlinks=true,linkcolor=blue,citecolor=blue]{hyperref}

% 设置页面边距
\geometry{top=2.5cm, bottom=2.5cm, left=2.5cm, right=2.5cm}

% 设置中文字体
\setCJKmainfont{Source Han Sans SC}[
  BoldFont = Source Han Sans SC Bold,
  ItalicFont = Source Han Sans SC,
  BoldItalicFont = Source Han Sans SC Bold
]
\setCJKmonofont{Source Han Sans SC}
\newCJKfontfamily\cnfont{Source Han Sans SC}

% 设置英文字体
\setmainfont{Times New Roman}

% 确保特定中文字符始终使用中文字体
\DeclareTextFontCommand{\textcn}{\cnfont}
\DeclareTextFontCommand{\texten}{\normalfont}

% 设置数学字体
\setmathfont{TeX Gyre Termes Math}

% 简化的标点和符号设置
\punctstyle{kaiming}
\xeCJKsetup{CJKecglue={\hskip 0.15em plus 0.05em minus 0.05em}}

% 确保中文文本和标点符号使用中文字体的命令
\newcommand{\zhtext}[1]{{\cnfont #1}}

% 添加中英文混排支持
\XeTeXlinebreaklocale "zh"
\XeTeXlinebreakskip = 0pt plus 1pt

% 确保特定中文字符总是使用中文字体
\newcommand{\cn}[1]{{\cnfont #1}}

% 设置页眉页脚样式
\pagestyle{fancy}
\fancyhf{}
\fancyhead[LE,RO]{\thepage}
\fancyhead[RE]{\leftmark}
\fancyhead[LO]{\rightmark}
\renewcommand{\headrulewidth}{0.4pt}
\renewcommand{\footrulewidth}{0pt}
\setlength{\headheight}{14.5pt}

% 设置章节标题格式
\titleformat{\chapter}[display]
{\normalfont\bfseries\Huge}
{\chaptertitlename\ \thechapter}{20pt}{\Huge}
\titlespacing*{\chapter}{0pt}{50pt}{40pt}

% 设置节标题格式
\titleformat{\section}
{\normalfont\Large\bfseries}
{\thesection}{1em}{}
\titlespacing*{\section}{0pt}{3.5ex plus 1ex minus .2ex}{2.3ex plus .2ex}

% 设置段落间距和首行缩进
\setlength{\parindent}{2em}
\setlength{\parskip}{1ex}
\onehalfspacing

%New command to display footnote whose markers will always be hidden
\let\svthefootnote\thefootnote
\newcommand\blfootnotetext[1]{%
  \let\thefootnote\relax\footnote{#1}%
  \addtocounter{footnote}{-1}%
  \let\thefootnote\svthefootnote%
}

%Overriding the \footnotetext command to hide the marker if its value is `0`
\let\svfootnotetext\footnotetext
\renewcommand\footnotetext[2][?]{%
  \if\relax#1\relax%
    \ifnum\value{footnote}=0\blfootnotetext{#2}\else\svfootnotetext{#2}\fi%
  \else%
    \if?#1\ifnum\value{footnote}=0\blfootnotetext{#2}\else\svfootnotetext{#2}\fi%
    \else\svfootnotetext[#1]{#2}\fi%
  \fi
}

% 设置目录格式
\renewcommand{\contentsname}{\centerline{\Large\bfseries 目录}}
% \renewcommand{\cftbeforetoctitleskip}{0pt}
% \renewcommand{\cftaftertoctitleskip}{2em}
% \renewcommand{\cftchapdotsep}{\cftdotsep}
% \renewcommand{\cftchapfont}{\bfseries}
% \renewcommand{\cftsecfont}{\normalfont}
% \renewcommand{\cftsubsecfont}{\normalfont\itshape}

% 其他自定义命令  引用

% 宏包加载

% This LaTeX document needs to be compiled with XeLaTeX.
\documentclass[a4paper,11pt,twoside]{book}
\usepackage[utf8]{inputenc}
\usepackage{amsmath}
\usepackage{amsfonts}
\usepackage{amssymb}
\usepackage[version=4]{mhchem}
\usepackage{extpfeil}
\usepackage{stmaryrd}
\usepackage{graphicx}
\usepackage[export]{adjustbox}
\graphicspath{ {./images/} }
\usepackage{multirow}
\usepackage{fvextra, csquotes}
\usepackage{xeCJK}
\usepackage{polyglossia}
\usepackage{fontspec}
\usepackage{unicode-math}
\usepackage{geometry}
\usepackage{fancyhdr}
\usepackage{titlesec}
\usepackage{indentfirst}
\usepackage{setspace}
\usepackage{xcolor}

% 添加 biblatex 支持
\usepackage[style=gb7714-2015,backend=biber,language=chinese]{biblatex}
\DeclareLanguageMapping{chinese}{chinese}
\addbibresource{references.bib}

% 添加hyperref包以启用PDF中的可跳转链接
\usepackage[hidelinks,bookmarks=true,bookmarksnumbered=true,colorlinks=true,linkcolor=blue,citecolor=blue]{hyperref}

% 设置页面边距
\geometry{top=2.5cm, bottom=2.5cm, left=2.5cm, right=2.5cm}

% 设置中文字体
\setCJKmainfont{Source Han Sans SC}[
  BoldFont = Source Han Sans SC Bold,
  ItalicFont = Source Han Sans SC,
  BoldItalicFont = Source Han Sans SC Bold
]
\setCJKmonofont{Source Han Sans SC}
\newCJKfontfamily\cnfont{Source Han Sans SC}

% 设置英文字体
\setmainfont{Times New Roman}

% 确保特定中文字符始终使用中文字体
\DeclareTextFontCommand{\textcn}{\cnfont}
\DeclareTextFontCommand{\texten}{\normalfont}

% 设置数学字体
\setmathfont{TeX Gyre Termes Math}

% 简化的标点和符号设置
\punctstyle{kaiming}
\xeCJKsetup{CJKecglue={\hskip 0.15em plus 0.05em minus 0.05em}}

% 确保中文文本和标点符号使用中文字体的命令
\newcommand{\zhtext}[1]{{\cnfont #1}}

% 添加中英文混排支持
\XeTeXlinebreaklocale "zh"
\XeTeXlinebreakskip = 0pt plus 1pt

% 确保特定中文字符总是使用中文字体
\newcommand{\cn}[1]{{\cnfont #1}}

% 设置页眉页脚样式
\pagestyle{fancy}
\fancyhf{}
\fancyhead[LE,RO]{\thepage}
\fancyhead[RE]{\leftmark}
\fancyhead[LO]{\rightmark}
\renewcommand{\headrulewidth}{0.4pt}
\renewcommand{\footrulewidth}{0pt}
\setlength{\headheight}{14.5pt}

% 设置章节标题格式
\titleformat{\chapter}[display]
{\normalfont\bfseries\Huge}
{\chaptertitlename\ \thechapter}{20pt}{\Huge}
\titlespacing*{\chapter}{0pt}{50pt}{40pt}

% 设置节标题格式
\titleformat{\section}
{\normalfont\Large\bfseries}
{\thesection}{1em}{}
\titlespacing*{\section}{0pt}{3.5ex plus 1ex minus .2ex}{2.3ex plus .2ex}

% 设置段落间距和首行缩进
\setlength{\parindent}{2em}
\setlength{\parskip}{1ex}
\onehalfspacing

%New command to display footnote whose markers will always be hidden
\let\svthefootnote\thefootnote
\newcommand\blfootnotetext[1]{%
  \let\thefootnote\relax\footnote{#1}%
  \addtocounter{footnote}{-1}%
  \let\thefootnote\svthefootnote%
}

%Overriding the \footnotetext command to hide the marker if its value is `0`
\let\svfootnotetext\footnotetext
\renewcommand\footnotetext[2][?]{%
  \if\relax#1\relax%
    \ifnum\value{footnote}=0\blfootnotetext{#2}\else\svfootnotetext{#2}\fi%
  \else%
    \if?#1\ifnum\value{footnote}=0\blfootnotetext{#2}\else\svfootnotetext{#2}\fi%
    \else\svfootnotetext[#1]{#2}\fi%
  \fi
}

% 设置目录格式
\renewcommand{\contentsname}{\centerline{\Large\bfseries 目录}}
% \renewcommand{\cftbeforetoctitleskip}{0pt}
% \renewcommand{\cftaftertoctitleskip}{2em}
% \renewcommand{\cftchapdotsep}{\cftdotsep}
% \renewcommand{\cftchapfont}{\bfseries}
% \renewcommand{\cftsecfont}{\normalfont}
% \renewcommand{\cftsubsecfont}{\normalfont\itshape}

% 其他自定义命令  引用

% 宏包加载

% This LaTeX document needs to be compiled with XeLaTeX.
\documentclass[a4paper,11pt,twoside]{book}
\usepackage[utf8]{inputenc}
\usepackage{amsmath}
\usepackage{amsfonts}
\usepackage{amssymb}
\usepackage[version=4]{mhchem}
\usepackage{extpfeil}
\usepackage{stmaryrd}
\usepackage{graphicx}
\usepackage[export]{adjustbox}
\graphicspath{ {./images/} }
\usepackage{multirow}
\usepackage{fvextra, csquotes}
\usepackage{xeCJK}
\usepackage{polyglossia}
\usepackage{fontspec}
\usepackage{unicode-math}
\usepackage{geometry}
\usepackage{fancyhdr}
\usepackage{titlesec}
\usepackage{indentfirst}
\usepackage{setspace}
\usepackage{xcolor}

% 添加 biblatex 支持
\usepackage[style=gb7714-2015,backend=biber,language=chinese]{biblatex}
\DeclareLanguageMapping{chinese}{chinese}
\addbibresource{references.bib}

% 添加hyperref包以启用PDF中的可跳转链接
\usepackage[hidelinks,bookmarks=true,bookmarksnumbered=true,colorlinks=true,linkcolor=blue,citecolor=blue]{hyperref}

% 设置页面边距
\geometry{top=2.5cm, bottom=2.5cm, left=2.5cm, right=2.5cm}

% 设置中文字体
\setCJKmainfont{Source Han Sans SC}[
  BoldFont = Source Han Sans SC Bold,
  ItalicFont = Source Han Sans SC,
  BoldItalicFont = Source Han Sans SC Bold
]
\setCJKmonofont{Source Han Sans SC}
\newCJKfontfamily\cnfont{Source Han Sans SC}

% 设置英文字体
\setmainfont{Times New Roman}

% 确保特定中文字符始终使用中文字体
\DeclareTextFontCommand{\textcn}{\cnfont}
\DeclareTextFontCommand{\texten}{\normalfont}

% 设置数学字体
\setmathfont{TeX Gyre Termes Math}

% 简化的标点和符号设置
\punctstyle{kaiming}
\xeCJKsetup{CJKecglue={\hskip 0.15em plus 0.05em minus 0.05em}}

% 确保中文文本和标点符号使用中文字体的命令
\newcommand{\zhtext}[1]{{\cnfont #1}}

% 添加中英文混排支持
\XeTeXlinebreaklocale "zh"
\XeTeXlinebreakskip = 0pt plus 1pt

% 确保特定中文字符总是使用中文字体
\newcommand{\cn}[1]{{\cnfont #1}}

% 设置页眉页脚样式
\pagestyle{fancy}
\fancyhf{}
\fancyhead[LE,RO]{\thepage}
\fancyhead[RE]{\leftmark}
\fancyhead[LO]{\rightmark}
\renewcommand{\headrulewidth}{0.4pt}
\renewcommand{\footrulewidth}{0pt}
\setlength{\headheight}{14.5pt}

% 设置章节标题格式
\titleformat{\chapter}[display]
{\normalfont\bfseries\Huge}
{\chaptertitlename\ \thechapter}{20pt}{\Huge}
\titlespacing*{\chapter}{0pt}{50pt}{40pt}

% 设置节标题格式
\titleformat{\section}
{\normalfont\Large\bfseries}
{\thesection}{1em}{}
\titlespacing*{\section}{0pt}{3.5ex plus 1ex minus .2ex}{2.3ex plus .2ex}

% 设置段落间距和首行缩进
\setlength{\parindent}{2em}
\setlength{\parskip}{1ex}
\onehalfspacing

%New command to display footnote whose markers will always be hidden
\let\svthefootnote\thefootnote
\newcommand\blfootnotetext[1]{%
  \let\thefootnote\relax\footnote{#1}%
  \addtocounter{footnote}{-1}%
  \let\thefootnote\svthefootnote%
}

%Overriding the \footnotetext command to hide the marker if its value is `0`
\let\svfootnotetext\footnotetext
\renewcommand\footnotetext[2][?]{%
  \if\relax#1\relax%
    \ifnum\value{footnote}=0\blfootnotetext{#2}\else\svfootnotetext{#2}\fi%
  \else%
    \if?#1\ifnum\value{footnote}=0\blfootnotetext{#2}\else\svfootnotetext{#2}\fi%
    \else\svfootnotetext[#1]{#2}\fi%
  \fi
}

% 设置目录格式
\renewcommand{\contentsname}{\centerline{\Large\bfseries 目录}}
% \renewcommand{\cftbeforetoctitleskip}{0pt}
% \renewcommand{\cftaftertoctitleskip}{2em}
% \renewcommand{\cftchapdotsep}{\cftdotsep}
% \renewcommand{\cftchapfont}{\bfseries}
% \renewcommand{\cftsecfont}{\normalfont}
% \renewcommand{\cftsubsecfont}{\normalfont\itshape}

% 其他自定义命令 