% ===============================================
% 逻辑学导论 - 专业排版配置文件
% 优化版本:增强视觉效果和专业性
% ===============================================

% 文档类设置 - 使用专业书籍格式
\documentclass[a4paper,12pt,twoside,openright]{book}

% ===============================================
% 核心宏包加载
% ===============================================
\usepackage[utf8]{inputenc}
\usepackage{amsmath,amsfonts,amssymb}
\usepackage[version=4]{mhchem}
\usepackage{extpfeil,stmaryrd}
\usepackage{graphicx}
\usepackage[export]{adjustbox}
\graphicspath{ {./images/} }
\usepackage{multirow,array,booktabs}
\usepackage{fvextra,csquotes}

% 中文支持
\usepackage{xeCJK}
\usepackage{polyglossia}
\usepackage{fontspec}
\usepackage{unicode-math}

% 页面布局和设计
\usepackage{geometry}
\usepackage{fancyhdr}
\usepackage{titlesec}
\usepackage{indentfirst}
\usepackage{setspace}

% 颜色和视觉效果
\usepackage{xcolor}
\usepackage{tikz}
\usepackage{tcolorbox}
\usepackage{mdframed}
\usepackage{framed}
\usepackage{enumitem}
\usepackage{caption}
\usepackage{subcaption}

% 参考文献支持
\usepackage[style=gb7714-2015,backend=biber,language=chinese]{biblatex}
\DeclareLanguageMapping{chinese}{chinese}
\addbibresource{references.bib}

% PDF链接和书签
\usepackage[hidelinks,bookmarks=true,bookmarksnumbered=true,colorlinks=true,linkcolor=LogicBlue,citecolor=LogicGreen,urlcolor=LogicRed]{hyperref}

% ===============================================
% 颜色定义 - 专业配色方案
% ===============================================
\definecolor{LogicBlue}{RGB}{25,118,210}      % 主色调:深蓝色
\definecolor{LogicGreen}{RGB}{46,125,50}      % 辅助色:深绿色
\definecolor{LogicRed}{RGB}{198,40,40}        % 强调色:深红色
\definecolor{LogicGray}{RGB}{97,97,97}        % 中性色:深灰色
\definecolor{LogicLightBlue}{RGB}{227,242,253} % 浅蓝色背景
\definecolor{LogicLightGreen}{RGB}{232,245,233} % 浅绿色背景
\definecolor{LogicLightGray}{RGB}{250,250,250} % 浅灰色背景
\definecolor{LogicDarkBlue}{RGB}{13,71,161}   % 深蓝色
\definecolor{LOGICDARKBLUE}{RGB}{13,71,161}   % 深蓝色(兼容性)
\definecolor{LogicAccent}{RGB}{255,193,7}     % 金黄色强调

% ===============================================
% 页面布局设置
% ===============================================
\geometry{
  top=3cm,
  bottom=3cm,
  left=3cm,
  right=2.5cm,
  headheight=25pt,
  headsep=20pt,
  footskip=30pt,
  marginparwidth=2cm,
  marginparsep=0.5cm
}

% ===============================================
% 字体设置 - 专业字体配置
% ===============================================

% 中文字体设置 - 使用思源黑体
\setCJKmainfont{Source Han Sans SC}[
  BoldFont = Source Han Sans SC Bold,
  ItalicFont = Source Han Sans SC Light,
  BoldItalicFont = Source Han Sans SC Bold,
  Scale = 1.0
]
\setCJKsansfont{Source Han Sans SC}[
  BoldFont = Source Han Sans SC Bold,
  Scale = 1.0
]
\setCJKmonofont{Source Han Sans SC}[Scale = 0.9]

% 英文字体设置 - 使用专业字体组合
\setmainfont{Times New Roman}[
  BoldFont = Times New Roman Bold,
  ItalicFont = Times New Roman Italic,
  BoldItalicFont = Times New Roman Bold Italic,
  Scale = 1.0
]
\setsansfont{Arial}[
  BoldFont = Arial Bold,
  ItalicFont = Arial Italic,
  BoldItalicFont = Arial Bold Italic,
  Scale = 0.95
]
\setmonofont{Courier New}[Scale = 0.85]

% 数学字体设置
\setmathfont{TeX Gyre Termes Math}

% 中文字体命令
\newCJKfontfamily\cnfont{Source Han Sans SC}
\newCJKfontfamily\cntitle{Source Han Sans SC Bold}
\DeclareTextFontCommand{\textcn}{\cnfont}
\DeclareTextFontCommand{\texten}{\normalfont}

% 中文排版设置
\punctstyle{kaiming}
\xeCJKsetup{CJKecglue={\hskip 0.15em plus 0.05em minus 0.05em}}
\XeTeXlinebreaklocale "zh"
\XeTeXlinebreakskip = 0pt plus 1pt

% 中文字体命令
\newcommand{\zhtext}[1]{{\cnfont #1}}
\newcommand{\cn}[1]{{\cnfont #1}}

% ===============================================
% 页眉页脚设计 - 专业样式
% ===============================================
\pagestyle{fancy}
\fancyhf{}

% 页眉设置
\fancyhead[LE]{\color{LogicGray}\small\thepage\quad\textbar\quad\leftmark}
\fancyhead[RO]{\color{LogicGray}\small\rightmark\quad\textbar\quad\thepage}

% 页眉装饰线
\renewcommand{\headrulewidth}{0.5pt}
\renewcommand{\headrule}{\hbox to\headwidth{%
  \color{LogicBlue}\leaders\hrule height \headrulewidth\hfill}}

% 页脚设置
\fancyfoot[C]{}
\renewcommand{\footrulewidth}{0pt}
\setlength{\headheight}{25pt}

% 章节首页样式
\fancypagestyle{plain}{%
  \fancyhf{}
  \fancyfoot[C]{\color{LogicGray}\small\thepage}
  \renewcommand{\headrulewidth}{0pt}
  \renewcommand{\footrulewidth}{0pt}
}

% ===============================================
% 章节标题设计 - 现代专业样式
% ===============================================

% 章标题格式
\titleformat{\chapter}[display]
{\normalfont\cntitle\color{LogicDarkBlue}}
{\tikz[remember picture,overlay]{
  \fill[LogicLightBlue] (current page.north west) rectangle ([yshift=-3cm]current page.north east);
  \node[anchor=south east,font=\fontsize{60}{60}\selectfont\color{LogicBlue!30}]
    at ([xshift=-2cm,yshift=-1cm]current page.north east) {\thechapter};
}
\vspace{-2cm}\chaptertitlename\ \thechapter}
{20pt}
{\Huge\color{LogicDarkBlue}}
\titlespacing*{\chapter}{0pt}{0pt}{40pt}

% 节标题格式
\titleformat{\section}
{\normalfont\Large\bfseries\color{LogicBlue}}
{\colorbox{LogicLightBlue}{\makebox[2em]{\color{LogicDarkBlue}\thesection}}\quad}
{0pt}
{}
\titlespacing*{\section}{0pt}{3.5ex plus 1ex minus .2ex}{2.3ex plus .2ex}

% 子节标题格式
\titleformat{\subsection}
{\normalfont\large\bfseries\color{LogicGreen}}
{\thesubsection\quad}
{0pt}
{}
\titlespacing*{\subsection}{0pt}{3ex plus 1ex minus .2ex}{2ex plus .2ex}

% 子子节标题格式
\titleformat{\subsubsection}
{\normalfont\normalsize\bfseries\color{LogicGray}}
{\thesubsubsection\quad}
{0pt}
{}
\titlespacing*{\subsubsection}{0pt}{2.5ex plus 1ex minus .2ex}{1.5ex plus .2ex}

% ===============================================
% 段落和间距设置
% ===============================================
\setlength{\parindent}{2em}
\setlength{\parskip}{0.5ex plus 0.2ex minus 0.1ex}
\onehalfspacing
\setlength{\baselineskip}{1.2\baselineskip}

%New command to display footnote whose markers will always be hidden
\let\svthefootnote\thefootnote
\newcommand\blfootnotetext[1]{%
  \let\thefootnote\relax\footnote{#1}%
  \addtocounter{footnote}{-1}%
  \let\thefootnote\svthefootnote%
}

%Overriding the \footnotetext command to hide the marker if its value is `0`
\let\svfootnotetext\footnotetext
\renewcommand\footnotetext[2][?]{%
  \if\relax#1\relax%
    \ifnum\value{footnote}=0\blfootnotetext{#2}\else\svfootnotetext{#2}\fi%
  \else%
    \if?#1\ifnum\value{footnote}=0\blfootnotetext{#2}\else\svfootnotetext{#2}\fi%
    \else\svfootnotetext[#1]{#2}\fi%
  \fi
}

% ===============================================
% 专业文本框和环境设置
% ===============================================

% 定义专业的文本框样式
\tcbuselibrary{skins,breakable,theorems}

% 定义重点提示框
\newtcolorbox{logicbox}[1][]{
  enhanced,
  colback=LogicLightBlue,
  colframe=LogicBlue,
  boxrule=1pt,
  arc=3pt,
  left=10pt,
  right=10pt,
  top=8pt,
  bottom=8pt,
  breakable,
  #1
}

% 定义定理框
\newtcolorbox{theorembox}[1][]{
  enhanced,
  colback=LogicLightGreen,
  colframe=LogicGreen,
  boxrule=1pt,
  arc=3pt,
  left=10pt,
  right=10pt,
  top=8pt,
  bottom=8pt,
  breakable,
  fonttitle=\bfseries,
  title=定理,
  #1
}

% 定义例题框
\newtcolorbox{examplebox}[1][]{
  enhanced,
  colback=LogicLightGray,
  colframe=LogicGray,
  boxrule=1pt,
  arc=3pt,
  left=10pt,
  right=10pt,
  top=8pt,
  bottom=8pt,
  breakable,
  fonttitle=\bfseries,
  title=例题,
  #1
}

% 定义注意事项框
\newtcolorbox{notebox}[1][]{
  enhanced,
  colback=yellow!10,
  colframe=LogicAccent,
  boxrule=1pt,
  arc=3pt,
  left=10pt,
  right=10pt,
  top=8pt,
  bottom=8pt,
  breakable,
  fonttitle=\bfseries,
  title=注意,
  #1
}

% ===============================================
% 列表和枚举优化
% ===============================================
\setlist[itemize]{
  leftmargin=2em,
  itemsep=0.3ex,
  parsep=0.2ex,
  topsep=0.5ex
}

\setlist[enumerate]{
  leftmargin=2em,
  itemsep=0.3ex,
  parsep=0.2ex,
  topsep=0.5ex
}

% ===============================================
% 目录格式优化
% ===============================================
\renewcommand{\contentsname}{\centerline{\Large\bfseries\color{LogicDarkBlue} 目录}}

% ===============================================
% 图表标题优化
% ===============================================
\captionsetup{
  font={small,bf},
  labelfont={color=LogicBlue},
  textfont={color=LogicGray},
  margin=20pt,
  skip=10pt
}

% ===============================================
% 自定义命令
% ===============================================

% 强调文本命令
\newcommand{\logicemph}[1]{\textcolor{LogicBlue}{\textbf{#1}}}
\newcommand{\logicterm}[1]{\textcolor{LogicGreen}{\textbf{#1}}}
\newcommand{\logicwarn}[1]{\textcolor{LogicRed}{\textbf{#1}}}

% 章节总结命令
\newcommand{\chaptersummary}[1]{
  \begin{logicbox}[title=本章要点]
    #1
  \end{logicbox}
}