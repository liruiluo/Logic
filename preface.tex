\section*{内容简介}

\begin{center}
\rule{0.5\textwidth}{0.4pt}
\end{center}

\begin{quotation}
\large\textit{『逻辑学为人类思维提供指引,就如同灯塔为迷航的船只指明方向。』}
\end{quotation}

\begin{center}
\rule{0.5\textwidth}{0.4pt}
\end{center}

\vspace{1em}

本书旨在系统地介绍\textbf{逻辑学}的各个方面,从基础概念到高级主题,力求用简明有逻辑的语言将逻辑学的理论和应用梳理清楚。全书结构清晰,内容由浅入深,既有理论探讨,又有实践指导。

作为一门研究\textbf{推理方法}与\textbf{原则}的学科,逻辑学在哲学、数学、计算机科学等诸多领域都有着重要的应用。本书从逻辑学的基本概念出发,逐步深入到以下核心内容:

\begin{itemize}
  \item \textbf{命题逻辑}:研究命题之间的逻辑关系
  \item \textbf{谓词逻辑}:分析命题内部结构
  \item \textbf{模态逻辑}:处理必然性与可能性
  \item \textbf{归纳逻辑}:探讨归纳推理的方法
  \item \textbf{逻辑悖论}:分析逻辑中的矛盾现象
  \item \textbf{非经典逻辑}:介绍多值逻辑等新发展
\end{itemize}

本书既可作为大学逻辑学课程的参考教材,也适合对逻辑学感兴趣的读者自学使用。

\begin{center}
\fbox{\parbox{0.9\textwidth}{
  \centering
  通过学习本书,读者将能够:\\
  \begin{minipage}{0.85\textwidth}
  \begin{itemize}
    \item 识别和评估日常生活中的论证
    \item 避免常见的逻辑谬误
    \item 构建清晰、有效的论证
    \item 培养批判性思维能力
  \end{itemize}
  \end{minipage}
}}
\end{center}

\vspace{1em}

希望本书能够帮助读者培养\textbf{逻辑思维},提高分析问题和解决问题的能力,在这个信息爆炸的时代,具备清晰、理性的思考方式。